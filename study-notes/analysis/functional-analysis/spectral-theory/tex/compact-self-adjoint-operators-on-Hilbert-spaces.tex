
          %%%%% ~~~~~~~~~~~~~~~~~~~~ %%%%%

\section{Compact self-adjoint operators on Hilbert spaces}
\setcounter{theorem}{0}
\setcounter{equation}{0}

%\cite{vanDerVaart1996}
%\cite{Kosorok2008}

%\renewcommand{\theenumi}{\alph{enumi}}
%\renewcommand{\labelenumi}{\textnormal{(\theenumi)}$\;\;$}
\renewcommand{\theenumi}{\roman{enumi}}
\renewcommand{\labelenumi}{\textnormal{(\theenumi)}$\;\;$}

          %%%%% ~~~~~~~~~~~~~~~~~~~~ %%%%%

\begin{theorem}\mbox{}
\vskip 0.1cm
\noindent
Let $H$ be a Hilbert space over $\C$.
Then, for each compact self-adjoint linear operator $T \in K(H)$ on $H$, the following statements are true:
\begin{enumerate}
\item
	$\left\{\,\overset{{\color{white}.}}{\pm}\Vert\,T\,\Vert\,\right\} \,\bigcap\, \sigma_{\textnormal{pt}}(\,T\,)$
	$\neq$ $\varemptyset$.
\item
	The set of non-zero eigenvalues of T is non-empty and is either finite or consists
	of a sequence which tends to zero. Each non-zero eigenvalue is real and has finite
	multiplicity. Eigenvectors corresponding to different eigenvalues are orthogonal.
\item
	The number of non-zero eigenvalues of T (repeated according to multiplicity)
	is equal to the rank \,$r(T)$\, of \,$T$.\,
	(Note that $r(T)$ may be $\infty$.)
	There exists a (finite or infinite) sequence
	\,$\left\{\,\overset{{\color{white}.}}{v_{n}}\,\right\}^{r(T)}_{n=1}$\,
	of mutually orthogonal unit eigenvectors of $T$
	that forms an orthonormal basis for the closure
	\,$\overline{\image(T)}$\,
	of the image of \,$T$,\, and the operator \,$T$\, has the representation
	\begin{equation*}
	T(x)
	\;\; = \;\;
		\overset{r(T)}{\underset{n=1}{\sum}}\;\,
		\lambda_{n}\,\langle\, x , v_{n} \,\rangle\,v_{n}\,,
	\quad
	\textnormal{for each \,$x \in H$},
	\end{equation*}
	where 
	\,$\left\{\,\overset{{\color{white}.}}{\lambda_{n}}\,\right\}^{r(T)}_{n=1}$\,
	%\,$\left\{\,\lambda_{n}\,\right\}^{r(T)}_{n=1}$\,
	is the set of non-zero eigenvalues of \,$T$.
\item
	$H \,=\, \ker\!\left(\,T\,\right) \,\oplus\, \overline{\image(T)}$\,
	gives an orthogonal decomposition of \,$H$\, into two closed subpaces.
\item
	If \,$\ker(\,T\,) = \{\,0\,\}$,\, then the eigenvectors of \,$T$ form an orthonormal basis for \,$H$
	(hence, \,$H$\, is separable).
\item
	If \,$\ker(\,T\,) = \{\,0\,\}$\, and \,$\dim H = \infty$,\,
	then \,$T$ has infinitely many distinct eigenvalues.
\item
	If \,$H$\, is separable, then there exists an orthonormal basis \,$\mathfrak{B}$\, for \,$H$\,
	consisting entirely of eigenvectors of \,$T$,\, where \,$\mathfrak{B}$\, is of the following form:
	\begin{equation*}
	\mathfrak{B}
	\;\; = \;\;
		\left\{\;\overset{{\color{white}.}}{z_{m}}\;\right\}_{m =1}^{\textnormal{null}(T)}
		\;\;\bigsqcup\;\;
		\left\{\;\overset{{\color{white}.}}{v_{n}}\;\right\}_{n =1}^{r(T)},
	\end{equation*}
	where
	\begin{itemize}
	\item
		$\textnormal{null}(T) \,:=\, \dim\ker(\,T\,) \,\in\, \{\,0,\infty\,\}\,\cup\,\N$,\,
	\item
		$\left\{\;\overset{{\color{white}.}}{z_{m}}\;\right\}_{m =1}^{\textnormal{null}(T)}$\,
		is an orthonormal basis for \,$\ker(\,T\,)$,\, and
	\item
		\,$\left\{\;\overset{{\color{white}.}}{v_{n}}\;\right\}_{n =1}^{r(T)}$\,
		is an orthonormal basis for
		\,$\overline{\image(\,T\,)}$.\,
	\end{itemize}
%	This basis has the form {en}r(T)
%n=1
%∪
%{zm}n(T)
%m=1, where {en}r(T)
%n=1 is an orthonormal basis of Im T and {zm}n(T)
%m=1 is an
%orthonormal basis of Ker T.
\end{enumerate}
\end{theorem}

          %%%%% ~~~~~~~~~~~~~~~~~~~~ %%%%%
