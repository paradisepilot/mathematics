
          %%%%% ~~~~~~~~~~~~~~~~~~~~ %%%%%

\section{Compact operators on Hilbert spaces}
\setcounter{theorem}{0}
\setcounter{equation}{0}

%\cite{vanDerVaart1996}
%\cite{Kosorok2008}

%\renewcommand{\theenumi}{\alph{enumi}}
%\renewcommand{\labelenumi}{\textnormal{(\theenumi)}$\;\;$}
\renewcommand{\theenumi}{\roman{enumi}}
\renewcommand{\labelenumi}{\textnormal{(\theenumi)}$\;\;$}

          %%%%% ~~~~~~~~~~~~~~~~~~~~ %%%%%

\begin{theorem}\mbox{}
\vskip 0.1cm
\noindent
Let $H$ be a Hilbert space over $\C$, and $B(H)$ the vector space of bounded linear operators on $H$.
Then, for each $T \in B(H)$,
\begin{equation*}
\textnormal{$T$\, is compact if and only if \,$T^{*}$\, is compact.}
\end{equation*}
\end{theorem}

          %%%%% ~~~~~~~~~~~~~~~~~~~~ %%%%%

\vskip 0.5cm
\begin{theorem}\mbox{}
\vskip 0.1cm
\noindent
Let $H$ be a Hilbert space over $\C$, and $K(H)$ the vector space of compact linear operators on $H$.
Then, for each $T \in K(H)$, the following statements are true:
\begin{enumerate}
\item
	$\image(T)$\, and \,$\overline{\image(T)}$\, are separable.
%\item
%	For each $\lambda \in \C \left\backslash\{\,0\,\}\right.$,
%	\begin{equation*}
%	\dim\ker\!\left(\,T \overset{{\color{white}.}}{-} \lambda\cdot\textnormal{id}_{H}\,\right) < \infty,
%	\quad\textnormal{and}\quad
%	\image\!\left(\,T \overset{{\color{white}.}}{-} \lambda\cdot\textnormal{id}_{H}\,\right)\;\;\textnormal{is closed}
%	\end{equation*}
\item
	For each $\lambda \in \C \left\backslash\{\,0\,\}\right.$, we have:
	\begin{itemize}
	\item
		$\image\!\left(\,T \overset{{\color{white}.}}{-} \lambda\cdot\textnormal{id}_{H}\,\right)$\, is closed,
	\item
		$\dim\ker\!\left(\,T \overset{{\color{white}.}}{-} \lambda\cdot\textnormal{id}_{H}\,\right)$
		$=$
		$\dim\ker\!\left(\,T^{*} \overset{{\color{white}.}}{-} \overline{\lambda}\cdot\textnormal{id}_{H}\,\right)$
		$<$ $\infty$,
	\item
		%\begin{equation*}
		either\;\;
		$(\,\lambda\,, \overline{\lambda}\,) \,\in\, \rho(\,T\,) \times \rho(T^{*})$
		\;\; or \;\;
		$(\,\lambda\,, \overline{\lambda}\,) \,\in\, \sigma_{\textnormal{pt}}(\,T\,) \times \sigma_{\textnormal{pt}}(T^{*})$
		%\end{equation*}
	\end{itemize}
\item
	If \,$\dim H = \infty$,\, then \,$0 \in \sigma(\,T\,)$.
\item
	If \,$H$\, is not separable, then \,$0 \in \sigma_{\textnormal{pt}}(\,T\,)$.
\item
	If \,$H$\, is separable, then either \,$0 \in \sigma_{\textnormal{pt}}(\,T\,)$,\, or
	\,$0 \in \sigma(\,T\,) \left\backslash \sigma_{\textnormal{pt}}(\,\overset{{\color{white}.}}{T}\,)\right.$.
\item
	For each real $r > 0$, the set of distinct eigenvalues $\lambda \in \C$ of $T$
	satisfying \,$\vert\,\lambda\,\vert \geq r$\, is finite.
\item
	The point spectrum \,$\sigma_{\textnormal{pt}}(\,T\,)$\, of $\,T$\, is at most countably infinite.
\item
	$0 \in \C$\, is the only possible accumulation point of \,$\sigma_{\textnormal{pt}}(\,T\,)$;\,
	in other words, every infinite sequence of distinct eigenvalues of \,$T$\, converges to \,$0 \in \C$.
\end{enumerate}
\end{theorem}

          %%%%% ~~~~~~~~~~~~~~~~~~~~ %%%%%
