
%%%%%%%%%%
\begin{frame}{\headingColor\bf\LARGE The Dirac equation}

\scriptsize
\vskip 0.25cm

\begin{itemize}
\item
	Dirac sought a new equation of motion by seeking a first-order linear differential operator
	\begin{equation*}
	D
	\;=\;
		\i\,\gamma^{\mu}\partial_{\mu}
	\;=\;
		\i\left(\,
			\gamma^{0}\dfrac{\partial}{\partial x_{0}} 
			\,+\, \gamma^{1}\dfrac{\partial}{\partial x_{1}} 
			\,+\, \gamma^{2}\dfrac{\partial}{\partial x_{2}} 
			\,+\, \gamma^{3}\dfrac{\partial}{\partial x_{3}} 
			\,\right)
	\end{equation*}
	whose square is {\color{red}minus} the Minkowskian d'Alembertian
	\;$\textnormal{\large$\Box$}
	%\;=\;
	%	\dfrac{\partial^{2}}{\partial t^{2}} \,-\, \Delta
	\;=\;
		\dfrac{\partial^{2}}{\partial t^{2}}
		\,-\, \dfrac{\partial^{2}}{\partial x_{1}^{2}}
		\,-\, \dfrac{\partial^{2}}{\partial x_{2}^{2}}
		\,-\, \dfrac{\partial^{2}}{\partial x_{3}^{2}}
	$.\,
\item
	Dirac arrived at the famous Dirac equation:
	\begin{equation*}
	\left(\,\i\cdot\gamma^{\mu}\partial_{\mu} - m\,\right)\psi \, = \, 0
	\end{equation*}
	\vskip -6.0px
	The requirement
	\,$
	-\,g^{\mu\nu}\partial_{\mu}\partial_{\nu}
	\,=\,
		{\color{red}-\,\textnormal{\large$\Box$}}
	\;{\color{red}=}\;
		{\color{red}\left(\,\i\cdot\gamma^{\mu}\partial_{\mu}\,\right)^{2}}
	%\,=\,
	%	\left(\,\i\cdot\gamma^{\mu}\partial_{\mu}\,\right)
	%	\left(\,\i\cdot\gamma^{\nu}\partial_{\nu}\,\right)
	\,=\,
		-\,\gamma^{\mu}\gamma^{\nu}\partial_{\mu}\partial_{\nu}
	\,=\,
		-\left(\dfrac{\gamma^{\mu}\gamma^{\nu}+\gamma^{\nu}\gamma^{\mu}}{2}\right)\partial_{\mu}\partial_{\nu}
		$\,
	implies that \,$\gamma^{\mu}$\, must satisfy:
	\begin{equation}
	\label{GammaMatricesSatisfyCliffordRelation}
	\left\{\,\gamma^{\mu}\,,\,\gamma^{\nu}\,\right\}
	\;=\;
		2\cdot g^{\mu\nu}\,,
	\end{equation}
	where
	\,$\left\{\,\gamma^{\mu}\,,\,\gamma^{\nu}\,\right\} \,:=\, \gamma^{\mu}\gamma^{\nu} + \gamma^{\mu}\gamma^{\nu}$\,
	and
	\,$g^{\mu\nu} \,=\, \diag(1,-1,-1,-1)$.
\item
	\begin{equation*}
	\begin{array}{c}\textnormal{Dirac} \\ \textnormal{equation} \end{array}
	\Longleftrightarrow\;\;\,
	\i\,\gamma^{\mu}\partial_{\mu}\psi = m\,\psi
	\;\;\Longrightarrow\;\;
	%\i\,\gamma^{\mu}\partial_{\mu}\left(\i\,\gamma^{\mu}\partial_{\mu}\overset{{\color{white}$.$}{\psi}\right) = m\,\i\,\gamma^{\mu}\partial_{\mu}\psi
	\underset{-\,\Box\psi}{\underbrace{{\color{white}.}{\color{red}\i\,\gamma^{\mu}\partial_{\mu}}\!\left(
		\i\,\gamma^{\nu}\partial_{\nu}\overset{{\color{white}.}}{\psi}
		\right)}}
		= m\cdot\underset{m\,\psi}{\underbrace{{\color{red}\i\,\gamma^{\mu}\partial_{\mu}}\overset{{\color{white}.}}{\psi}}}
	\;\;\,\Longrightarrow
	\begin{array}{c}\textnormal{Klein-Gordon} \\ \textnormal{equation} \end{array}
	\end{equation*}
\end{itemize}

\end{frame}
\normalsize

%%%%%%%%%%
\begin{frame}{\headingColor\bf\LARGE The Dirac equation (cont'd)}

\scriptsize
\vskip 0.25cm

\begin{itemize}
\item
	Dirac realized that the condition \eqref{GammaMatricesSatisfyCliffordRelation} cannot be satisfied
	if the \,$\gamma^{\mu}$'s\, were just complex numbers, but that condition is indeed satisfied by
	certain $4 \times 4$ complex matrices.
\item
	This observation led to the following realization (when expressed in modern geometric language):
	The ``wave function'' \,$\psi$\, must be smooth sections of a spinor bundle over spacetime.
\end{itemize}

\end{frame}
\normalsize

%%%%%%%%%%
