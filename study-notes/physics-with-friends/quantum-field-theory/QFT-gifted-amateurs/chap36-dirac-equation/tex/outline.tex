
%%%%%%%%%%
\begin{frame}{\headingColor\bf\LARGE Outline}

\scriptsize
\vskip 0.2cm

\begin{itemize}
\item
	Lagrangian encodes physics.
	\vskip -0.1px
	Principle of stationary action: Equation of motion is Euler-Lagrange (EL) equation.
\item
	Gauge invariance constrains the admissible form of the Lagrangian.
	\vskip 1.0px
	\begin{itemize}\itemindent=-10px
	{\scriptsize
	\item
		fermions: spinor field (section of spinor bundle on spacetime), Dirac equation
	\item
		bosons: complex scalar (i.e., non-spinor) multiplet field, Klein-Gordon
	\item
		(gauge) bosons: connection, Yang-Mills
	\item
		Higgs boson: real scalar (i.e., non-spinor) singlet field
	}
	\end{itemize}
\item
	Solutions of EL equations of Standard Model free-field Lagrangians admit plane-wave solutions.
	\vskip -0.1px
	Full solutions admit the form of Fourier transforms.
\item
	Canonical quantization:
	\vskip -0.1px
	Re-interpret coefficients in the integrand of free-field solutions (in the form of Fourier transforms)
	as creation and annihilation operators (on certain Fock space).
	\vskip -0.1px
	What happens with the conjugate momentum (of the field)?
\item

\item

\end{itemize}

\end{frame}
\normalsize

%%%%%%%%%%
