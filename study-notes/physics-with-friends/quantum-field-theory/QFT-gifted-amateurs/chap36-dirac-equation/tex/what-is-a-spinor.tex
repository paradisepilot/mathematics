
%%%%%%%%%%
\begin{frame}{\headingColor\bf\LARGE What is a \textit{spinor}?}

\scriptsize
\vskip 8px

\textbf{Clifford algebra, Lorentz group \& Spin group}
\vskip 1px
\begin{itemize}
\item
	Condition \eqref{GammaMatricesSatisfyCliffordRelation}
	\,({\color{red}$\gamma^{\mu}\gamma^{\nu} \,=\, 2\,g^{\mu\nu}$}, with $g^{\mu\nu} = \diag(1,-1-1-1)$)\,
	is the defining relations of what is known as the Clifford algebra \,$\Cl(1,3)$; see \S6.3, \cite{Hamilton2017}.
\item
	$\Cl(1,3)$\, contains the orthochronous spin group \,$\Spin^{\uparrow}(1,3)$;\,
	see Proposition 6.5.4, \cite{Hamilton2017}.
\item
	$\Spin^{\uparrow}(1,3)$\, is the universal covering of the {\color{red}proper orthochronous Lorentz group}
	\,$\SO^{\uparrow}(1,3)$;\, see Definition 6.1.16 and Corollary 6.5.16, \cite{Hamilton2017}.\,
	$\ker\!\left(\Spin^{\uparrow}(1,3) \overset{\pi}{\longrightarrow} \SO^{\uparrow}(1,3)\right) \,=\, \{\,\pm1\,\}$;\,
	see Theorem 6.5.13, \cite{Hamilton2017}.
\item
	Condition \eqref{GammaMatricesSatisfyCliffordRelation}
	\,$\Longrightarrow$\,
	$\Spin^{\uparrow}(1,3)$\, acts on the space of solutions of the Dirac equation.
\end{itemize}

\vskip 8px
\textbf{Relationship between \,$\SU(2)$\, and \,$\widetilde{\SO^{\uparrow}(1,3)} \,=\, \Spin^{\uparrow}(1,3) \;\,{\color{red}\cong}\;\, {\color{red}\SL(2,\C)}$}
\vskip 1px
\begin{itemize}
\item
	$\mathfrak{sl}(2,\C) \,\cong\, \mathfrak{su}(2) \oplus \i\,\mathfrak{su}(2)$\; (the copies of $\mathfrak{su}(2)$ commute)
	\vskip 1px
	{\tiny(Skew self-adjoint matrices with trace zero plus self-adjoint matrices with trace zero gives all matrices with trace zero.)}
	\vskip 5px
\end{itemize}

\vskip 8px
\textbf{Symmetry group of a quantum system \& action of $\SO(3)$ on the system}
\vskip 1px
\begin{itemize}
\item
	The state space of a quantum system is the projective space $\P(H)$ over a complex Hilbert space $H$.
\item
	The \textbf{symmetry group} of a quantum system with state space $\P(H)$ is the projective unitary group of $H$, i.e.
	\vskip -5.0px
	\begin{equation*}
	\P U(H) \; := \; \left.\overset{{\color{white}.}}{U(H)}\,\right/\{\,e^{\i\,\theta}\cdot\mathbf{1}_{H}\,\}_{\theta\in\Re}
	\end{equation*}
\item
	The action of \,$\SO(3)$\, on a quantum system with state space \,$\P(H)$\,
	is thus a (Lie group) homomorphism
	\,$\SO(3) \longrightarrow \P U(H)$,\,
	i.e., a projective unitary representation of \,$\SO(3)$.
\end{itemize}

\end{frame}
\normalsize

%%%%%%%%%%
%\begin{frame}{\headingColor\bf\LARGE What is a \textit{spinor}? (cont'd)}
%
%\scriptsize 
%\vskip 10px
%
%\textbf{Symmetry group of a quantum system \& action of $\SO^{\uparrow}(1,3)$ on the system}
%\begin{itemize}
%\item
%	The state space of a quantum system is the projective space $\P(H)$ over a complex Hilbert space $H$.
%\item
%	The \textbf{symmetry group} of a quantum system with state space $\P(H)$ is the projective unitary group of $H$, i.e.
%	\vskip -5.0px
%	\begin{equation*}
%	\P U(H) \; := \; \left.\overset{{\color{white}.}}{U(H)}\,\right/\{\,e^{\i\,\theta}\cdot\mathbf{1}_{H}\,\}_{\theta\in\Re}
%	\end{equation*}
%\item
%	The action of \,$\SO^{\uparrow}(1,3)$\, on a quantum system with state space \,$\P(H)$\,
%	is thus a (Lie group) homomorphism
%	\,$\SO^{\uparrow}(1,3) \longrightarrow \P U(H)$,\,
%	i.e., a projective unitary representation of \,$\SO^{\uparrow}(1,3)$.
%\end{itemize}
%
%\end{frame}
%\normalsize

%%%%%%%%%%
\begin{frame}{\headingColor\bf\LARGE What is a \textit{spinor}? (cont'd)}

\scriptsize
\vskip 5px

\textbf{Relations between \,$\SO(3) \longrightarrow \P U(H)$, \,$\SO(3) \longrightarrow U(H)$,\, and \,$\SU(2) \longrightarrow U(H)$}
\begin{itemize}
\item
	Every \,$\SO(3) \overset{\theta}{\longrightarrow} U(H)$\, gives rise to a
	projective unitary representation (symmetry group of a quantum system) simply via composition
	\,$\SO(3) \overset{\theta}{\longrightarrow} U(H) \overset{Q}{\longrightarrow} \P U(H) := U(H)\;/\,\{\,e^{\i\,\theta}\cdot\mathbf{1}_{H}\,\}$.
	\vskip 5px
	\textbf{\color{red}Converse is false.}
	\begin{center}
	\vskip -5px
	\begin{tikzcd}
	\& U(H) \arrow[d, "Q"]
	\\
	\SO(3)
		\arrow[ru, dashed, "{\color{red}\exists\;\theta\,?}"]
		\arrow[r, swap, "\rho"]
	\& \P U(H)
	\end{tikzcd}
	\quad\quad\quad\quad\quad\quad
	\begin{tikzcd}
	\SU(2) %\,=\,\widetilde{\SO^{\uparrow}(1,3)}
		\arrow[r, "{\color{red}\widetilde{\rho}}"]		
		\arrow[d, swap, "\pi"]
	\& U(H) \arrow[d, "Q"]
	\\
	\SO(3)
		\arrow[r, swap, "\rho"]
	\& \P U(H)
	\end{tikzcd}
	\end{center}
	\vskip 5px
\item
	However, a \textbf{partial converse} is true: If \,$\dim_{\C}(H) < \infty$,\, then, for each projective unitary representation
	\,$\SO(3) \overset{\rho}{\longrightarrow} \P U(H)$,\,
	there exists an (ordinary) unitary representation
	\,$\widetilde{\SO(3)} = \SU(2) \overset{\widetilde{\rho}}{\longrightarrow} U(H)$\,
	such that \,$\rho \circ \pi = Q \circ \widetilde{\rho}$\,;\, see Theorem 16.47, \cite{Hall2013}.
	\vskip 5px
\item
	So, there are {\color{red}two} types of finite-dimensional projective unitary representations
	\,$\SO(3) \overset{\rho}{\longrightarrow} \P U(H)$\,:\,
	\begin{itemize}\itemindent=-10px
	{\scriptsize
	\item
		either \,$\rho$\, is induced by an (ordinary) unitary representation
		\,$\SO(3) \overset{\theta}{\longrightarrow} U(H)$\,
	\item
		or \,\,$\rho$\, cannot be so induced (\,still have \,$\SU(2) \overset{\widetilde{\rho}}{\longrightarrow} U(H)$\,;\,
		{\color{red}elements of $H$ are then called \textit{spinors}}\,)
	}
	\end{itemize}
	\vskip 8px
\item
	{\color{blue}
	The notion of spinors ``extends'' to the scenario:
	\begin{equation*}
	\SO(3) \rightsquigarrow \SO^{\uparrow}(1,3)\,,
	\quad\;
	\SU(2) \rightsquigarrow \Spin^{\uparrow}(1,3) \cong \SL(2,\C)\,,
	\quad\;
	U(H) \rightsquigarrow {\color{red}\textnormal{Aut}(H)}
	\end{equation*}
	}
\end{itemize}

\end{frame}
\normalsize

%%%%%%%%%%
\begin{frame}{\headingColor\bf\LARGE What is a \textit{spinor}? (cont'd)}

\scriptsize
\vskip 5px

\begin{itemize}
\item
	{\color{blue}
	The notion of spinors ``extends'' to the scenario:
	\begin{equation*}
	\SO(3) \rightsquigarrow \SO^{\uparrow}(1,3)\,,
	\quad\;
	\SU(2) \rightsquigarrow \Spin^{\uparrow}(1,3) \cong \SL(2,\C)\,,
	\quad\;
	U(H) \rightsquigarrow {\color{red}\textnormal{Aut}(H)}
	\end{equation*}
	}
\item
	{\color{white}.}
	\begin{center}
	\vskip -10px
	\begin{tikzcd}
	\& \textnormal{Aut}(H) \arrow[d, "Q"]
	\\
	\SO^{\uparrow}(1,3)
		\arrow[ru, dashed, "{\color{red}\exists\;\theta\,?}"]
		\arrow[r, swap, "\rho"]
	\& \P\textnormal{Aut}(H)
	\end{tikzcd}
	\quad\quad\quad\quad\quad\quad
	\begin{tikzcd}
	\Spin^{\uparrow}(1,3) %\,=\,\widetilde{\SO^{\uparrow}(1,3)}
		\arrow[r, "{\color{red}\widetilde{\rho}}"]		
		\arrow[d, swap, "\pi"]
	\& \textnormal{Aut}(H) \arrow[d, "Q"]
	\\
	\SO^{\uparrow}(1,3)
		\arrow[r, swap, "\rho"]
	\& \P\textnormal{Aut}(H)
	\end{tikzcd}
	\end{center}
	\vskip 10px
\item
	So, there are {\color{red}two} types of finite-dimensional projective representations
	\,$\SO^{\uparrow}(1,3) \overset{\rho}{\longrightarrow} \P\textnormal{Aut}(H)$\,:\,
	\begin{itemize}\itemindent=-10px
	{\scriptsize
	\item
		either \,$\rho$\, is induced by an (ordinary) representation
		\,$\SO^{\uparrow}(1,3) \overset{\theta}{\longrightarrow} \textnormal{Aut}(H)$\,
	\item
		or \,\,$\rho$\, cannot be so induced (\,still have \,$\Spin^{\uparrow}(1,3) \overset{\widetilde{\rho}}{\longrightarrow} \textnormal{Aut}(H)$\,;\,
		{\color{red}elements of $H$ are also called \textit{spinors}}\,)
	}
	\end{itemize}
\end{itemize}

\end{frame}
\normalsize

%%%%%%%%%%
