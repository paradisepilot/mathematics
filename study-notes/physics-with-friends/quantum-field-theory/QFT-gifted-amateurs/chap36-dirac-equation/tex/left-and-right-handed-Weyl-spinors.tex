
%%%%%%%%%%
\begin{frame}{\headingColor\bf\Large Left- \& Right-handed Weyl Spinors + Dirac spinors}

\scriptsize
\vskip 5px

\textbf{Representations of $\Spin^{\uparrow}(1,3) \,\cong\, \widetilde{\SO^{\uparrow}(1,3)}$}
\begin{itemize}
\item
	$\widetilde{\SO^{\uparrow}(1,3)} \,=\, \Spin^{\uparrow}(1,3) \;\,{\color{red}\cong}\;\, {\color{red}\SL(2,\C)}$.
\item
	$\mathfrak{sl}(2,\C) \,\cong\, \mathfrak{su}(2) + \i\,\mathfrak{su}(2)$\; (the copies of $\mathfrak{su}(2)$ commute)
	\vskip 1px
	{\tiny(Skew self-adjoint matrices with trace zero plus self-adjoint matrices with trace zero gives all matrices with trace zero.)}
\item
	The finite-dimensional irreducible representations of \,$\SU(2)$\, have been classified.
	\vskip 5px
	For each \,$s = 0, \frac{1}{2}, 1, \frac{3}{2}, \ldots\,$,\, there exists a unique representation 
	\,$\rho_{s} : \SU(2) \longrightarrow \GL(\Re,2s+1)$
\item
	The irreducible representations of \,$\Spin^{\uparrow}(1,3)$\, are parametrized by
	the ordered pairs {\color{red}\,$(s_{+},s_{-})$\,} of non-negative multiples of \,$\frac{1}{2}$,\, where
	\,$s_{+}$\, refers to \,$\mathfrak{su}(2) \subset \mathfrak{sl}(2,\C)$\,
	and
	\,$s_{-}$\, refers to \,$\i\,\mathfrak{su}(2) \subset \mathfrak{sl}(2,\C)$.\,
	See Theorem on page 517, \cite{Woit2017}.
\end{itemize}

\vskip 5px
\textbf{Weyl spinors}
\begin{itemize}
\item
	Left-handed: $\left(\frac{1}{2},0\right)$ representation of $\Spin^{\uparrow}(1,3)$\,;\;\,
	right-handed: $\left(0,\frac{1}{2}\right)$
\item
	The \textbf{\color{red}parity transformation} transforms left-handed Weyl spinors to right-handed ones, and vice versa.
	See page 174, \cite{Robinson2011}.
\end{itemize}

\vskip 5px
\textbf{Dirac spinors:}\,
The $\left(\frac{1}{2},0\right) \oplus \left(0,\frac{1}{2}\right)$ representation of $\Spin^{\uparrow}(1,3)$
\begin{itemize}
\item
	{\color{red}$\left(\frac{1}{2},0\right) \oplus \left(0,\frac{1}{2}\right)$\,
	corresponds to action of Dirac $\gamma$-matrices on $\C^{4}$.}
	See \S4.3.7, \cite{Robinson2011} \,or\, \S41.2, \cite{Woit2017}.
\end{itemize}

\end{frame}
\normalsize

%%%%%%%%%%
