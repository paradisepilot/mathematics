
%%%%%%%%%%
\begin{frame}{\headingColor\bf\Large Representations of $\Spin^{\uparrow}(1,3) \,\cong\, \widetilde{\SO^{\uparrow}(1,3)}$}

\scriptsize
\vskip 5px

\begin{itemize}
\item
	$\widetilde{\SO^{\uparrow}(1,3)} \,=\, \Spin^{\uparrow}(1,3) \;\,{\color{red}\cong}\;\, {\color{red}\SL(2,\C)}$.
	\vskip 5px
\item
	$\mathfrak{sl}(2,\C) \,\cong\, \mathfrak{su}(2) \oplus \i\,\mathfrak{su}(2)$\; (the copies of $\mathfrak{su}(2)$ commute)
	\vskip 1px
	{\tiny(Skew self-adjoint matrices with trace zero plus self-adjoint matrices with trace zero gives all matrices with trace zero.)}
	\vskip 5px
\item
	Generators of \,$\mathfrak{sl}(2,\C) \,\cong\, \mathfrak{spin}(1,3)$:
	\vskip -5px
	\begin{equation*}
	\textnormal{\textbf{spatial rotations:}}\; J_{x}, J_{y}, J_{z}\,,
	\quad
	\textnormal{\textbf{boosts:}}\; K_{x}, K_{y}, K_{z}
	\end{equation*}
	\begin{equation*}
	[\,J_{i},J_{j}\,] = \i\,\epsilon_{ijk}\,J_{k}\,,
	\quad
	[\,J_{i},K_{j}\,] = \i\,\epsilon_{ijk}\,K_{k}\,,
	\quad
	[\,K_{i},K_{j}\,] = -\,\i\,\epsilon_{ijk}\,J_{k}
	\end{equation*}
	\begin{equation*}
	N^{\pm}_{i} \; := \; {\color{red}\dfrac{1}{2}}\!\left(\,J_{i}\,\pm\,\i\,K_{i}\,\right)
	\quad\Longrightarrow\quad
	[\,N^{+}_{i},N^{+}_{j}\,] = \i\,\epsilon_{ijk}\,N^{+}_{k}\,,
	\quad
	[\,N^{-}_{i},N^{-}_{j}\,] = \i\,\epsilon_{ijk}\,N^{-}_{k}\,,
	\quad
	[\,N^{-}_{i},N^{+}_{j}\,] = 0
	\end{equation*}
	Hence, $\langle\,N^{+}_{x},N^{+}_{y},N^{+}_{z}\,\rangle \,\cong\, \langle\,N^{-}_{x},N^{-}_{y},N^{-}_{z}\,\rangle \,\cong\, \mathfrak{su}(2)$.
	\vskip 5px
\item
	Aside:\, The $\frac{1}{2}$ above \,$\Longrightarrow$\,
	for spinors, a spatial (planar) rotation of {\color{red}$360^{\circ}$ induces multiplication by $-1$.}
	\vskip 5px
\item
	The finite-dimensional irreducible representations of \,$\SU(2)$\, have been classified.
	\vskip 5px
	For each \,$s = 0, \frac{1}{2}, 1, \frac{3}{2}, \ldots\,$,\, there exists a unique representation 
	\,$\rho_{s} : \SU(2) \longrightarrow \GL(\Re,2s+1)$
	\vskip 5px
\item
	The irreducible representations of \,$\Spin^{\uparrow}(1,3)$\, are
	{\color{red}parametrized by the ordered pairs \,$(s_{+},s_{-})$\,} of non-negative multiples of \,$\frac{1}{2}$,\, where
	\,$s_{+}$\, refers to \,$\mathfrak{su}(2) \subset \mathfrak{sl}(2,\C)$\,
	and
	\,$s_{-}$\, refers to \,$\i\,\mathfrak{su}(2) \subset \mathfrak{sl}(2,\C)$.
	\vskip 2.5px
	See Theorem on page 517, \cite{Woit2017}.
\end{itemize}

\end{frame}
\normalsize

%%%%%%%%%%
\begin{frame}{\headingColor\bf\Large Left- \& Right-handed Weyl Spinors + Dirac spinors}

\scriptsize
\vskip 7.5px

\textbf{Representations of $\Spin^{\uparrow}(1,3) \,\cong\, \widetilde{\SO^{\uparrow}(1,3)}$}
\begin{itemize}
\item
	The irreducible representations of \,$\Spin^{\uparrow}(1,3)$\, are parametrized by
	the ordered pairs {\color{red}\,$(s_{+},s_{-})$\,} of non-negative multiples of \,$\frac{1}{2}$,\, where
	\,$s_{+}$\, refers to \,$\mathfrak{su}(2) \subset \mathfrak{sl}(2,\C)$\,
	and
	\,$s_{-}$\, refers to \,$\i\,\mathfrak{su}(2) \subset \mathfrak{sl}(2,\C)$.\,
	See Theorem on page 517, \cite{Woit2017}.
\end{itemize}

\vskip 10px

\textbf{Weyl spinors}
\begin{itemize}
\item
	Left-handed: $\left(\frac{1}{2},0\right)$ representation of $\Spin^{\uparrow}(1,3)$\,;\;\,
	right-handed: $\left(0,\frac{1}{2}\right)$
\item
	Aside:\, The {\color{red}parity transformation} transforms left-handed Weyl spinors to right-handed ones, and vice versa.
	See page 174, \cite{Robinson2011}.
	Full Lorentz invariance implies both types of Weyl spinors must ``occur in nature.''
\end{itemize}

\vskip 10px

\textbf{Dirac spinors:}\,
The reducible \,$\left(\frac{1}{2},0\right) \oplus \left(0,\frac{1}{2}\right)$\, representation of \,$\Spin^{\uparrow}(1,3)$
\begin{itemize}
\item
	{\color{red}$\left(\frac{1}{2},0\right) \oplus \left(0,\frac{1}{2}\right)$\,
	corresponds to action of Dirac $\gamma$-matrices on $\C^{4}$.}
	See \S4.3.7, \cite{Robinson2011} \,or\, \S41.2, \cite{Woit2017}.
	%\vskip 2.5px
	%{\color{blue}Still not sure:\; Whether action of Dirac $\gamma$-matrices includes parity.}
\item
	Aside: \,$\lambda \in \left(\frac{1}{2},0\right)$
	\;$\Longrightarrow$\;\,
	$\i\,\sigma^{2}\!\left(\overline{\lambda}\right) \in \left(0,\frac{1}{2}\right)$.\,
	See \S4.3.12, \cite{Robinson2011}.
	\vskip 1.5px
	Hence, can write general element of
	\,$\left(\frac{1}{2},0\right) \oplus \left(0,\frac{1}{2}\right)$\,
	as
	\,$\left(\!\begin{array}{c} \lambda \\ \i\,\sigma^{2}\!\left(\overline{\rho}\right)\!\!\end{array}\right)$,\,
	where
	\;$\lambda,\, \rho \in \left(\frac{1}{2},0\right)$.\,
	Thus,
	\vskip 1px
	\begin{equation*}
	\i\,\gamma^{2}(\,\overline{\,\cdot\,}\,)
		\,:\,
		\left(\frac{1}{2},0\right) \oplus \left(0,\frac{1}{2}\right)
		\,\longrightarrow\,
		\left(\frac{1}{2},0\right) \oplus \left(0,\frac{1}{2}\right)
		\,:\,
		\left(\!\begin{array}{c} \lambda \\ \i\,\sigma^{2}\!\left(\overline{\rho}\right)\!\!\end{array}\right)
		\,\longmapsto\,
		\left(\!\begin{array}{c} \rho \\ \i\,\sigma^{2}\!\left(\overline{\lambda}\right)\!\!\end{array}\right)
	\end{equation*}
\end{itemize}

\end{frame}
\normalsize

%%%%%%%%%%
