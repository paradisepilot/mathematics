
%%%%%%%%%%
\begin{frame}{\headingColor\bf\LARGE The Lorentz group}

\scriptsize
\vskip 5.0px

Can be generated by the following three $1$-parameter families of rotations and boosts
(see {\color{red}\S1.4, \cite{Robinson2011}}):
\vskip -5.0px
{\color{white}.}

\begin{multicols}{2}

	\begin{minipage}{7cm}
	\begin{center}
	\textbf{Spatial rotations}
	\end{center}
	\begin{equation*}
	R_{x}(\theta_{x})
	\; := \;\!
		{\tiny\left(\begin{array}{cccc}
			1 & {\color{white}...}0 & {\color{white}...}0 & {\color{white}.}0
			\\
			0 & {\color{white}...}1 & {\color{white}...}0 & {\color{white}.}0
			\\
			0 & {\color{white}...}0 & {\color{white}-}\,\cos\theta_{x} & \sin\theta_{x}
			\\
			0 & {\color{white}...}0 & -\,\sin\theta_{x} & \cos\theta_{x}
			\end{array}\right)}
	\end{equation*}
	\vskip 12px
	\begin{equation*}
	R_{y}(\theta_{y})
	\; := \;\!
		{\tiny\left(\begin{array}{cccc}
			1 & {\color{white}...}0 & {\color{white}...}0 & {\color{white}...}0
			\\
			0 & {\color{white}-}\,\cos\theta_{y} & {\color{white}...}0 & -\,\sin\theta_{y}
			\\
			0 & {\color{white}...}0 & {\color{white}...}1 & {\color{white}...}0
			\\
			0 & {\color{white}-}\,\sin\theta_{y} & {\color{white}...}0 & {\color{white}-}\,\cos\theta_{y}
			\end{array}\right)}
	\end{equation*}
	\vskip 12px
	\begin{equation*}
	R_{z}(\theta_{z})
	\; := \;\!
		{\tiny\left(\begin{array}{cccc}
			1 & {\color{white}...}0 & {\color{white}..}0 & {\color{white}.}0
			\\
			0 & {\color{white}-}\,\cos\theta_{z} & {\color{white}.}\sin\theta_{z} & {\color{white}.}0
			\\
			0 & -\,\sin\theta_{z} & {\color{white}.}\cos\theta_{z} & {\color{white}.}0
			\\
			0 & {\color{white}...}0 & {\color{white}..}0 & {\color{white}.}1
			\end{array}\right)}
	\end{equation*}
	\end{minipage}

\columnbreak

	\begin{minipage}{7cm}
	\begin{center}
	\textbf{Boosts}
	\end{center}
	\begin{equation*}
	B_{x}(\phi_{x})
	\; := \;\!
		{\tiny\left(\begin{array}{cccc}
			{\color{white}-}\,\cosh\phi_{x} & -\,\sinh\phi_{x} & {\color{white}...}0 & {\color{white}...}0
			\\
			 -\,\sinh\phi_{x} & {\color{white}-}\,\cosh\phi_{x} & {\color{white}...}0 & {\color{white}...}0
			\\
			0 & {\color{white}...}0 & {\color{white}...}1 & {\color{white}...}0
			\\
			0 & {\color{white}...}0 & {\color{white}...}0 & {\color{white}...}1
			\end{array}\right)}
	\end{equation*}
	\vskip 0.1px
	\begin{equation*}
	B_{y}(\phi_{y})
	\; := \;\!
		{\tiny\left(\begin{array}{cccc}
			{\color{white}-}\,\cosh\phi_{x} & {\color{white}...}0 & -\,\sinh\phi_{x} & {\color{white}...}0
			\\
			0 & {\color{white}...}1 & {\color{white}...}0 & {\color{white}...}0
			\\
			-\,\sinh\phi_{x} & {\color{white}...}0 & {\color{white}-}\,\cosh\phi_{x} & {\color{white}...}0
			\\
			0 & {\color{white}...}0 & {\color{white}...}0 & {\color{white}...}1
			\end{array}\right)}
	\end{equation*}
	\vskip 0.1px
	\begin{equation*}
	B_{z}(\phi_{z})
	\; := \;\!
		{\tiny\left(\begin{array}{cccc}
			{\color{white}-}\,\cosh\phi_{x} & {\color{white}...}0 & {\color{white}...}0 & -\,\sinh\phi_{x}
			\\
			0 & {\color{white}...}1 & {\color{white}...}0 & {\color{white}...}0
			\\
			0 & {\color{white}...}0 & {\color{white}...}1 & {\color{white}...}0
			\\
			-\,\sinh\phi_{x} & {\color{white}...}0 & {\color{white}...}0 & {\color{white}-}\,\cosh\phi_{x}
			\end{array}\right)}
	\end{equation*}
	\end{minipage}

\end{multicols}

\end{frame}
\normalsize

%%%%%%%%%%
\begin{frame}{\headingColor\bf\Large Generators of the Lie algebra of the Lorentz group}

\footnotesize
\vskip 15px

We follow {\color{red}\S3.3, \cite{Robinson2011}}.
\vskip -15px
\begin{equation*}
J_{x}
\;\; := \;\;
	-\,\i\cdot\left.\dfrac{\d}{\d\,\theta_{x}}\right\vert_{\theta_{x}=0}\, R_{x}(\theta_{x})
\;\; = \;\;
	-\,\i\cdot\left.\dfrac{\d}{\d\,\theta_{x}}\right\vert_{\theta_{x}=0}\!
	{\tiny\left(\begin{array}{cccc}
		1 & 0 & {\color{white}...}0 & 0
		\\
		0 & 1 & {\color{white}...}0 & 0
		\\
		0 & 0 & {\color{white}-}\,\cos\theta_{x} & \sin\theta_{x}
		\\
		0 & 0 & -\,\sin\theta_{x} & \cos\theta_{x}
		\end{array}\right)}
\;\; = \;\;
	\overset{{\color{white}1}}{{\tiny\left(\begin{array}{cccc}
		0 & 0 & {\color{white}.}0 & {\color{white}...}0
		\\
		0 & 0 & {\color{white}.}0 & {\color{white}...}0
		\\
		0 & 0 & {\color{white}.}0 & -\,\i
		\\
		0 & 0 & {\color{white}.}\i & {\color{white}...}0
		\end{array}\right)}}
\end{equation*}

\vskip 10px
{\scriptsize Similarly,}
\begin{equation*}
J_{y}
\;\; = \;\;
	{\tiny\left(\begin{array}{cccc}
		0 & {\color{white}...}0 & 0 & {\color{white}.}0
		\\
		0 & {\color{white}...}0 & 0 & {\color{white}.}\i
		\\
		0 & {\color{white}...}0 &0 & {\color{white}.}0
		\\
		0 & -\,\i & 0 & {\color{white}.}0
		\end{array}\right)}\,,
\quad\quad
J_{z}
\;\; = \;\;
	{\tiny\left(\begin{array}{cccc}
		0 & {\color{white}.}0 & {\color{white}...}0 & 0
		\\
		0 & {\color{white}.}0 & -\,\i & 0
		\\
		0 & {\color{white}.}\i & {\color{white}...}0 & 0
		\\
		0 & {\color{white}.}0 & {\color{white}...}0 & 0
		\end{array}\right)}
\end{equation*}

\vskip 10px
\begin{equation*}
K_{x}
\;\; := \;\;
	-\,\i\cdot\left.\dfrac{\d}{\d\,\phi_{x}}\right\vert_{\phi_{x}=0}\, B_{x}(\phi_{x})
\;\; = \;\;
	\cdots
\;\; = \;\;
	{\tiny\left(\begin{array}{cccc}
		0 & \i & 0 & 0
		\\
		\i & 0 & 0 & 0
		\\
		0 & 0 & 0 & 0
		\\
		0 & 0 & 0 & 0
		\end{array}\right)}\,,
\end{equation*}
\vskip 5px
\begin{equation*}
K_{y}
\;\; = \;\;
	{\tiny\left(\begin{array}{cccc}
		0 & 0 & \i & 0
		\\
		0 & 0 & 0 & 0
		\\
		\i & 0 & 0 & 0
		\\
		0 & 0 & 0 & 0
		\end{array}\right)}\,,
\quad\quad
K_{z}
\;\; = \;\;
	{\tiny\left(\begin{array}{cccc}
		0 & 0 & 0 & \i
		\\
		0 & 0 & 0 & 0
		\\
		0 & 0 & 0 & 0
		\\
		\i & 0 & 0 & 0
		\end{array}\right)}
\end{equation*}

\end{frame}
\normalsize

%%%%%%%%%%
\begin{frame}{\headingColor\bf\Large Commutation relations among \;$J_{x}, J_{y}, J_{z},\; K_{x}, K_{y}, K_{z}$}

\large
\vskip -10px

\begin{eqnarray*}
\left[\,J_{i}\,,\,J_{j}\,\right] & = & {\color{white}-}\,\i\cdot\varepsilon_{ijk} \cdot J_{k}
\\
\left[\,J_{i}\,,\,K_{j}\,\right] & \overset{{\color{white}\textnormal{1}}}{=} & {\color{white}-}\,\i\cdot\varepsilon_{ijk} \cdot K_{k}
\\
\left[\,K_{i}\,,\,K_{j}\,\right] & \overset{{\color{white}\textnormal{1}}}{=} & -\,\i\cdot\varepsilon_{ijk} \cdot J_{k}
\end{eqnarray*}

\vskip 10px
{\small Define:}
\vskip -25px
\begin{equation*}
N^{\pm}_{i} \;\; := \;\; \dfrac{1}{2}\left(\,J_{i}\,\pm\,K_{i}\,\right)
\end{equation*}
{\small Then,}
\vskip -25px
{\color{white}.}
\begin{multicols}{2}
	\begin{minipage}{7cm}
	\begin{eqnarray*}
	\left[\,N^{+}_{i}\,,\,N^{+}_{j}\,\right] & = & \i\cdot\varepsilon_{ijk} \cdot N^{+}_{k}
	\\
	\left[\,N^{-}_{i}\,,\,N^{-}_{j}\,\right] & \overset{{\color{white}\textnormal{1}}}{=} & \i\cdot\varepsilon_{ijk} \cdot N^{-}_{k}
	\end{eqnarray*}
	\end{minipage}
\columnbreak
	\begin{minipage}{7cm}
	{\color{white}.}
	\vskip -2.5px
	\begin{eqnarray*}
	\left[\,N^{-}_{i}\,,\,N^{+}_{j}\,\right] & \overset{{\color{white}\textnormal{1}}}{=} & 0
	\end{eqnarray*}
	\end{minipage}
\end{multicols}
\vskip 2px
{\small
{\color{red}Two} commuting copies of \,{\color{red}$\mathfrak{su}(2)$}\, in \,$\mathfrak{so}^{\uparrow}(1,3)$:
\,$\langle\,N^{+}_{x},N^{+}_{y},N^{+}_{z}\,\rangle$\; and \,$\langle\,N^{-}_{x},N^{-}_{y},N^{-}_{z}\,\rangle$
}

\end{frame}
\normalsize

%%%%%%%%%%
