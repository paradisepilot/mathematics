
          %%%%% ~~~~~~~~~~~~~~~~~~~~ %%%%%

\chapter{The double covering: $\SU(2) \longrightarrow \SO(3)$}
\setcounter{theorem}{0}
\setcounter{equation}{0}

%\cite{vanDerVaart1996}
%\cite{Kosorok2008}

%\renewcommand{\theenumi}{\alph{enumi}}
%\renewcommand{\labelenumi}{\textnormal{(\theenumi)}$\;\;$}
\renewcommand{\theenumi}{\roman{enumi}}
\renewcommand{\labelenumi}{\textnormal{(\theenumi)}$\;\;$}

          %%%%% ~~~~~~~~~~~~~~~~~~~~ %%%%%

\section{The Lie groups  \,$\SL(n,\C)$,\, $\SO(n)$\, and \,$\SU(n)$}

%\vskip 0.5cm
%$\SU(2) \overset{\textnormal{\scriptsize 2\textnormal{$:$}1{\color{white}.}}}{\longrightarrow} \SOup(1,3)$

          %%%%% ~~~~~~~~~~~~~~~~~~~~ %%%%%

\vskip 0.3cm
\begin{definition}
\mbox{}
\vskip 0.1cm
\noindent
The \textbf{orthogonal group of degree $n$} is defined as follows:
\begin{equation*}
\textnormal{O}(n)
\; := \;
	\left\{\;\,
		g \overset{{\color{white}.}}{\in} \textnormal{GL}(n,\Re)
		\;\left\vert\;\,
			g^{T} \cdot g = I_{n}
			\right.
		\;\right\}
\end{equation*}
The \textbf{special orthogonal group of degree $n$} is defined as follows:
\begin{equation*}
\textnormal{SO}(n)
\; := \;
	\left\{\;\,
		g \overset{{\color{white}.}}{\in} \textnormal{GL}(n,\Re)
		\;\left\vert\;\,
			g^{T} \cdot g = I_{n}\,,
			\;
			\textnormal{det}(g) = 1
			\right.
		\;\right\}
\end{equation*}
\end{definition}

          %%%%% ~~~~~~~~~~~~~~~~~~~~ %%%%%

\vskip 0.5cm
\begin{definition}
\mbox{}
\vskip 0.1cm
\noindent
The \textbf{unitary group of degree $n$} is defined as follows:
\begin{equation*}
\textnormal{U}(n)
\; := \;
	\left\{\;\,
		g \overset{{\color{white}.}}{\in} \textnormal{GL}(n,\C)
		\;\;\left\vert\;\;
			g^{\dagger} \cdot g \overset{{\color{white}1}}{=} I_{n}
			\right.
		\;\right\}
\end{equation*}
where \,$g^{\dagger}$\, is the conjugate transpose of
\,$g \in \textnormal{GL}(n,\C)$\,
and
\,$I_{n} \in \textnormal{GL}(n,\C)$\,
is the identity matrix.
\vskip 0.1cm
\noindent
The \textbf{special unitary group of degree $n$} is defined as follows:
\begin{equation*}
\textnormal{SU}(n)
\; := \;
	\left\{\;\,
		g \overset{{\color{white}.}}{\in} \textnormal{GL}(n,\C)
		\;\,\left\vert\;\;
			g^{\dagger} \cdot g \overset{{\color{white}1}}{=} I_{n}\,,
			\;
			\textnormal{det}(g) = 1
			\right.
		\;\right\}
\end{equation*}
\end{definition}

          %%%%% ~~~~~~~~~~~~~~~~~~~~ %%%%%

\vskip 1.0cm
\section{The Lie groups \,$\SO(3)$\, and \,$\SU(2)$}

          %%%%% ~~~~~~~~~~~~~~~~~~~~ %%%%%

\vskip 0.3cm
\begin{proposition}[Parametrization of \,$\textnormal{SU}(2)$]
\mbox{}
\vskip 0.1cm
\noindent
$\textnormal{SU}(2)$ admits the following parametrization:
\begin{equation*}
\textnormal{SU}{(2)}
\; := \;
	\left\{\,
		\left.
		\left(\begin{array}{rr}
		a & -\overline{b}
		\\
		\overset{{\color{white}-}}{b} & \overline{a}
		\end{array}\right)
		\overset{{\color{white}.}}{\in}
		\C^{2 \times 2}
		\;\;\right\vert\;\,
			\vert\, a \,\vert^{2} \,+\, \vert\, b \,\vert^{2} \,=\, 1
		\;\right\}
\end{equation*}
Hence, the (real) Lie group {\color{red}$\textnormal{SU}(2)$ is diffeomorphic to $S^{3}$}, the $3$-dimensional unit sphere
(in $4$-dimensional Euclidean space).
In particular, $\textnormal{SU}(2)$ is a {\color{red}simply connected} $3$-dimensional real manifold.
\end{proposition}
\proof
Suppose:
\begin{equation*}
g
\; = \;
	\left(\begin{array}{cc}
		a & c
		\\
		\overset{{\color{white}-}}{b} & d
		\end{array}\right)
\; \in \;
\textnormal{SU}(2)
\end{equation*}
First note that the component form of the condition \,$g^{\dagger}\cdot g = I_{2}$\, is:
\begin{equation*}
\left(\begin{array}{cc}
	1 & 0
	\\
	\overset{{\color{white}-}}{0} & 1
	\end{array}\right)
\; = \;
	g^{\dagger} \cdot g
\; = \;
	\left(\begin{array}{cc}
		\overline{a} & \overline{b}
		\\
		\overset{{\color{white}-}}{\overline{c}} & \overline{d}
		\end{array}\right)
	\cdot
	\left(\begin{array}{cc}
		a & c
		\\
		\overset{{\color{white}-}}{b} & d
		\end{array}\right)
\; = \;
	\left(\begin{array}{cc}
		a\overline{a} + b\overline{b} & \overline{a}c + \overline{b}d
		\\
		\overset{{\color{white}-}}{a\overline{c} + b\overline{d}} & c\overline{c} + d\overline{d}
		\end{array}\right)
\end{equation*}
Thus, we see that
\begin{equation*}
g
\; = \;
	\left(\begin{array}{cc}
		a & c
		\\
		\overset{{\color{white}-}}{b} & d
		\end{array}\right)
\;\in\;
	\textnormal{SU}(2)
\quad\Longleftrightarrow\quad
\left\{
	\begin{array}{ccc}
		g^{\dagger} \cdot g &=& I_{2}
		\\
		\det(g) &=& \overset{{\color{white}1}}{1}
		\end{array}
		\right.
\quad\Longleftrightarrow\quad
\left\{
	\begin{array}{ccc}
	\vert\,a\,\vert^{2} + \vert\,b\,\vert^{2} &=& 1
	\\
	\vert\,c\,\vert^{2} + \vert\,d\,\vert^{2} &\overset{{\color{white}1}}{=}& 1
	\\
	a\overline{c} \;\, + \,\; b\overline{d} &\overset{{\color{white}1}}{=}& 0
	\\
	ad \;\, - \,\; bc &\overset{{\color{white}1}}{=}& 1
	\end{array}
	\right.
\end{equation*}
Next, note that
\begin{equation*}
a\overline{c} + b\overline{d} = 0
\quad\Longleftrightarrow\quad
	\left\langle
		\left(\begin{array}{c} a \\ b \end{array}\right)
		\,,\,
		\left(\begin{array}{c} c \\ d \end{array}\right)
		\right\rangle_{\C^{2}}
	\;=\;
	0
\end{equation*}
Since
\,$\dim_{\C}\left(\begin{array}{c} a \\ b \end{array}\right)^{\perp} =\, 1$,\,
the above equality (i.e., orthogonality of the two columns of \,$g$) implies:
\begin{equation*}
\left(\begin{array}{c} c \\ d \end{array}\right)
\; \in \;
	\left(\begin{array}{c} a \\ b \end{array}\right)^{\perp}
\; = \;
	\textnormal{span}_{\C}\left\{
		\left(\begin{array}{r} -\overline{b} \\ \overline{a} \end{array}\right)
		\right\}
\quad\Longleftrightarrow\quad
\left(\begin{array}{c} c \\ d \end{array}\right)
\; = \;
	\lambda \left(\begin{array}{r} -\overline{b} \\ \overline{a} \end{array}\right),
	\;\;
	\textnormal{for some $\lambda \in \C$}
\end{equation*}
So, we now know that $g$ has the form:
\begin{equation*}
g
\; = \;
	\left(\begin{array}{rr}
		a & -\lambda\,\overline{b}
		\\
		\overset{{\color{white}-}}{b} & \lambda\,\overline{a}
		\end{array}\right)
\end{equation*}
Next,
\begin{equation*}
1
\,=\, \det(g)
\,=\, a\cdot(\lambda\,\overline{a}) - b \cdot (-\lambda\,\overline{b})
\,=\, \lambda\cdot(\vert\,a\,\vert^{2} + \vert\,b\,\vert^{2})
\quad\Longrightarrow\quad
	\lambda = 1
\end{equation*}
We may now conclude that
\begin{equation*}
g
\; = \;
	\left(\begin{array}{rr}
		a & -\,\overline{b}
		\\
		\overset{{\color{white}-}}{b} & \overline{a}
		\end{array}\right),\,
\quad
\textnormal{where \,$\vert\,a\,\vert^{2} + \vert\,b\,\vert^{2} = 1$}
\end{equation*}
This completes the proof of the Proposition.
\qed

          %%%%% ~~~~~~~~~~~~~~~~~~~~ %%%%%

\vskip 0.5cm
\noindent
We next establish the fact that \,$\SU(2)$\, is the universal covering space of \,$\SO(3)$.\,
We begin by identifying a \,$3$-dimensional \,$\Re$-linear subspace of \,$\C^{2 \times 2}$\,
and define an inner product on it such that the resulting inner product space over $\Re$
is isomorphic (as a real inner product space) to the \,$3$-dimensional Euclidean space \,$\mathbb{E}^{3}$.

\vskip 0.3cm
\noindent
Recall that \,$\su(2) \subset \C^{2 \times 2}$\, consists of all \,$2 \times 2$\, complex matrices
which are {\color{red}skew-Hermitian} ($Z^{\dagger} = -\,Z$) and have {\color{red}trace zero}.
Recall that, for \,$Z \in \C^{2 \times 2}$,\,
\begin{equation*}
Z^{\dagger} \; = \; - \, Z
\quad\Longleftrightarrow\quad
\left(\,\begin{array}{cc}
	\overline{z_{11}} & \overline{z_{21}}
	\\
	\overline{z_{12}} & \overline{z_{22}}
	\end{array}\,\right)
\; = \;
-\,\left(\,\begin{array}{cc}
	z_{11} & z_{12}
	\\
	z_{21} & z_{22}
	\end{array}\,\right)
\quad\Longleftrightarrow\quad
\left\{\begin{array}{l}
	\overline{z_{11}} = - \, z_{11} \;\;\Longleftrightarrow\;\; z_{11} \in \i\cdot\Re
	\\
	\overline{z_{21}} \overset{{\color{white}1}}{=} - \, z_{12}
	\\
	\overline{z_{22}} \overset{{\color{white}1}}{=} - \, z_{22} \;\;\Longleftrightarrow\;\; z_{22} \in \i\cdot\Re
	\end{array}\right.
\end{equation*}
Thus, the fact that \,$\su(2)$\, is the set of skew-Hermitian and trace-zero $2 \times 2$ complex matrices
implies that \,$\su(2)$\, can be set-theoretically parametrized as follows:
\begin{equation*}
\su(2)
\;\; = \;\;
	\left\{\;\left.
		\left(\,\begin{array}{cc}
			\i\,x_{1} & x_{2} + \i\,x_{3}
			\\
			- \, x_{2} + \i\,x_{3} & -\,\i\,x_{1}
			\end{array}\,\right)
		\in \C^{2 \times 2}
		\;\;\,\right\vert\;\,
		x_{1}, x_{2}, x_{3} \in \Re
		\;\right\}
\end{equation*}
Clearly, \,$\su(2)$\, is an $\Re$-linear subspace of \,$\C^{2 \times 2}$,\,
and \,$\su(2)$\, is thus isomorphic to \,$\Re^{3}$ as a vector space over \,$\Re$.\,
We next define the following inner product on \,$V$\,:
\begin{eqnarray*}
\left\langle\,
	\overset{{\color{white}.}}{X}
	\, , \,
	Y
	\,\right\rangle_{\!\su(2)}
& := &
	-\,\dfrac{1}{2}\,
	\trace\!\left(\,X \overset{{\color{white}1}}{\cdot} Y\,\right)
\;\; = \;\;
	-\,\dfrac{1}{2}\cdot\trace\!\left(
		\left(\begin{array}{cc}
			\i\,x_{1} & x_{2} + \i\,x_{3}
			\\
			- \, x_{2} + \i\,x_{3} & -\,\i\,x_{1}
			\end{array}\right)
		\cdot
		\left(\begin{array}{cc}
			\i\,y_{1} & y_{2} + \i\,y_{3}
			\\
			- \, y_{2} + \i\,y_{3} & -\,\i\,y_{1}
			\end{array}\right)
		\right)
\\
& = &
	-\,\dfrac{1}{2}\cdot\trace\!\left(\!
		\begin{array}{cc}
			-\,(x_{1}y_{1} + x_{2}y_{2} + x_{3}y_{3}) + \i\,(x_{2}y_{3}-x_{3}y_{2}) & (-x_{1}y_{3}+x_{3}y_{1}) + \i\,(x_{1}y_{2} - x_{2}y_{1})
			\\
			(-x_{3}y_{1}+x_{1}y_{3}) + \i\,(-x_{2}y_{1}+x_{1}y_{2}) & -\,(x_{1}y_{1} + x_{2}y_{2} - x_{3}y_{3}) - \i\,(x_{2}y_{3}-x_{3}y_{2})
			\end{array}
		\!\right)
\\
& = &
	\overset{{\color{white}1}}{x_{1}\,y_{1} + x_{2}\,y_{2} + x_{3}\,y_{3}}
\end{eqnarray*}
Thus,
\,$\left(\;\overset{{\color{white}.}}{\su(2)}\,,\,\langle\,\cdot\,,\,\cdot\,\rangle_{\,\su(2)}\,\right)$\,
is indeed isomorphic (as a real inner product space) to the $3$-dimensional Euclidean space \,$\mathbb{E}^{3}$.\,
In what follows, we thus view \,$\su(2)$\, as a copy of \,$\mathbb{E}^{3}$.\,

          %%%%% ~~~~~~~~~~~~~~~~~~~~ %%%%%

\vskip 0.5cm
\noindent
\begin{proposition}
\mbox{}
\vskip 0.1cm
\noindent
Let
\,$\textnormal{Ad} : \SU(2) \longrightarrow \GL(\su(2))$\,
be the adjoint representation of \,$\SU(2)$.\,
%\,$\textnormal{Ad} : \SU(2) \longrightarrow \Isom(\su(2)) \cong \Isom(\mathbb{E}^{3}) = \textnormal{O}(3) : U \longmapsto \textnormal{Ad}_{U}$\, 
%by:
%\begin{equation*}
%\textnormal{Ad}_{U}(\,X\,)
%\;\; := \;\;
%	U \cdot X \cdot U^{-1}\,,
%\quad
%\textnormal{for each \,$X \in \su(2) \cong \mathbb{E}^{3}$}
%\end{equation*}
Then, the following statements are true:
\begin{enumerate}
\item
	The adjoint representation
	\,$\textnormal{Ad} : \SU(2) \longrightarrow \GL(\su(2))$\,
	is given by:
	%\,$\textnormal{Ad} : \SU(2) \longrightarrow \Isom(\su(2)) \cong \Isom(\mathbb{E}^{3}) = \textnormal{O}(3) : U \longmapsto \textnormal{Ad}_{U}$\,
	\begin{equation*}
	\textnormal{Ad}_{U}(\,X\,)
	\;\; := \;\;
		U \cdot X \cdot U^{-1}\,,
	\quad
	\textnormal{for each \,$U \in \SU(2)$,\,  $X \in \su(2) \cong \mathbb{E}^{3}$}
	\end{equation*}
\item
	Under the inner product space isomorphism
	\,$\su(2) \cong \mathbb{E}^{3}$,\,
	the image of the adjoint representation
	\,$\textnormal{Ad} : \SU(2) \longrightarrow \GL(\su(2))$\,
	is isomorphic to
	\,$\SO(3) \subset \textnormal{O}(3) = \Isom(\mathbb{E}^{3}) \cong  \Isom(\su(2)) \subset \GL(\su(2))$.\,
	Consequently, we may regard the adjoint representation of \,$\SU(2)$\, to be:
	\begin{equation*}
	\textnormal{Ad} : \SU(2) \,\longrightarrow\, \SO(\su(2)) \cong \SO(3)
	\end{equation*}
	given by:
	\begin{equation*}
	\textnormal{Ad}_{U}(\,X\,)
	\;\; := \;\;
		U \cdot X \cdot U^{-1}\,,
	\quad
	\textnormal{for each \,$U \in \SU(2)$,\,  $X \in \su(2) \cong \mathbb{E}^{3}$}
	\end{equation*}
\item
	The map \,$\textnormal{Ad} : \SU(2) \longrightarrow \SO(3)$\, is a homomorphism.
\item
	$\ker(\textnormal{Ad}) \, = \, \{\,\pm \, I_{2}\,\}$
\item
	The map \,$\textnormal{Ad} : \SU(2) \longrightarrow \SO(3)$\, is surjective.
\end{enumerate}
\end{proposition}
\proof
\begin{enumerate}
\item
	Recall that
	\,$\textnormal{Ad} : \SU(2) \longrightarrow \GL(\su(2))$\,
	is defined by:
	\,$\textnormal{Ad}_{U} \,=\, \d_{e}\iota_{U}$,\,
	where
	\,$\iota_{U} : \SU(2) \longrightarrow \SU(2)$\,
	is the inner automorphism defined by:
	\begin{equation*}
	\iota_{U}(A) \;\; := \;\; U \cdot A \cdot U^{-1}
	\end{equation*}
	So, for each
	\,$U \in \SU(2)$\, and \,$X \in \su(2)$,\,
	\begin{eqnarray*}
	\textnormal{Ad}_{U}(X)
	& = &
		\left(\,\d_{e}\,\iota_{U}\,\right)\!\left(\,\overset{{\color{white}.}}{X}\,\right)
	\;\; = \;\;
		\left(\,\d_{e}\,\iota_{U}\,\right)\!\left(\,
			\left.\dfrac{\d}{\d\,t}\right\vert_{t=0}\;e^{tX}
			\,\right)
	\;\; = \;\;
		\left.\dfrac{\d}{\d\,t}\right\vert_{t = 0}\;
		\iota_{U}\!\left(\,e^{tX}\,\right)
	\\
	& = &
		\left.\dfrac{\d}{\d\,t}\right\vert_{t = 0}\;
		U \cdot e^{tX} \cdot U^{-1}
	\;\; = \;\;
		U \cdot \left(\,\left.\dfrac{\d}{\d\,t}\right\vert_{t=0}\;e^{tX}\,\right) \cdot U^{-1}
	\\
	& = &
		\overset{{\color{white}\textnormal{\normalsize$1$}}}{U} \cdot X \cdot U^{-1},
	\end{eqnarray*}
	as required.
\item
	First, we note that, for each \,$U \in \SU(2)$\, and \,$X \in V$,\,
	we indeed have that \,$\textnormal{Ad}_{U}(X) \in V$\,:
	\begin{eqnarray*}
	\left(\,\textnormal{Ad}_{U}(\overset{{\color{white}.}}{X})\,\right)^{\dagger}
	& \!=\! &
		\left(\, U \cdot \overset{{\color{white}.}}{X} \cdot U^{-1} \,\right)^{\dagger}
	\;\, = \;\,
		\left(\, U \cdot \overset{{\color{white}.}}{X} \cdot U^{\dagger} \,\right)^{\dagger}
	\;\, = \;\,
		U^{\dagger\dagger} \cdot X^{\dagger} \cdot U^{\dagger}
	\;\, = \;\,
		U \cdot (-X) \cdot U^{-1}
	\;\, = \;\,
		- \, \textnormal{Ad}_{U}(\,X\,)\,,
	\\
	\trace\!\left(\,\textnormal{Ad}_{U}(\overset{{\color{white}.}}{X})\,\right)
	& \!=\! &
		\trace\!\left(\, U \cdot \overset{{\color{white}.}}{X} \cdot U^{-1}\right)
	\;\, = \;\,
		\trace\!\left(\, \overset{{\color{white}.}}{X} \cdot U^{-1} \cdot U \,\right)
	\;\, = \;\,
		\trace\!\left(\, \overset{{\color{white}.}}{X} \,\right)
	\;\, = \;\,
		0
	\end{eqnarray*}
	which shows that \,$\textnormal{Ad}_{U}(X)$\, is skew-Hermitian and have trace-zero;
	therefore, indeed, \,$\textnormal{Ad}_{U}(X) \in \su(2)$.\,
	Next, we show that, for each \,$U \in \SU(2)$, the map
	\,$\textnormal{Ad}_{U} : \su(2) \cong \mathbb{E}^{3}\longrightarrow V \cong \mathbb{E}^{3}$\,
	preserves the inner product on \,$\su(2)$.\,
	To this end, note:
	\begin{eqnarray*}
		\left\langle\;
		\textnormal{Ad}_{U}(\overset{{\color{white}.}}{X})
		\; , \;
		\textnormal{Ad}_{U}(Y)
		\;\right\rangle_{\!\su(2)}
	& = &
		-\,\dfrac{1}{2}\,
		\trace\!\left(\,
				U \overset{{\color{white}1}}{\cdot} X \cdot U^{-1} \cdot U \cdot Y \cdot U^{-1}
			\,\right)
	\;\; = \;\;
		-\,\dfrac{1}{2}\,
		\trace\!\left(\,
			U \overset{{\color{white}1}}{\cdot} X \cdot Y \cdot U^{-1}
			\,\right)
	\\
	& = &
		-\,\dfrac{1}{2}\,
		\trace\!\left(\,
			X \overset{{\color{white}1}}{\cdot} Y \cdot U^{-1} \cdot U
			\,\right)
	\;\; = \;\;
		-\,\dfrac{1}{2}\,
		\trace\!\left(\,
			X \overset{{\color{white}1}}{\cdot} Y
			\,\right)
	\\
	& =: &
		\left\langle\;
			\overset{{\color{white}.}}{X}
			\, , \,
			Y
			\;\right\rangle_{\!\su(2)}
	\end{eqnarray*}
	Thus, under the inner product space isomorphism
	\,$\su(2) \cong \mathbb{E}^{3}$,\,
	we have:
	\,$\textnormal{Ad}_{U} \in \textnormal{O}(3)$,\,
	which furthermore immediately implies
	\,$\det\textnormal{Ad}_{U} \in \{\,\pm 1\,\}$.\,
	On the other hand, connectedness of \,$\SU(2)$\, and continuity of
	\,$\det$\, and \,$\textnormal{Ad}$\, together imply that the image of the map
	\,$\det \,\circ\, \textnormal{Ad} : \SU(2) \longrightarrow \Re$\,
	is a singleton subset of \,$\Re$.\,
	We may now conclude that
	 \,$\det\textnormal{Ad}_{U} = + 1$,\,
	for each \,$U \in \SU(2)$.\,
	This proves that the image of
	\,$\textnormal{Ad} : \SU(2) \longrightarrow \Isom(\su(2)) \cong \Isom(\mathbb{E}^{3})$\,
	is indeed \,$\SO(3)$.
\item
	Simply note:
	\begin{equation*}
	\textnormal{Ad}_{U_{1}U_{2}}(X)
	\; = \;
		(U_{1}U_{2}) \cdot X (U_{1}U_{2})^{-1}
	\; = \;
		U_{1} \cdot U_{2} \cdot X \cdot U_{2}^{-1} \cdot U_{1}^{-1}
	\; = \;
		\textnormal{Ad}_{U_{1}}\!\left(\,\textnormal{Ad}_{U_{2}}(\overset{{\color{white}.}}{X})\,\right)
	\; = \;
		\textnormal{Ad}_{U_{1}} \,\circ\, \textnormal{Ad}_{U_{2}}\left(\,\overset{{\color{white}.}}{X}\,\right),
	\end{equation*}
	which shows
	\,$\textnormal{Ad}(\,U_{1} \cdot U_{2}\,) \, = \, \textnormal{Ad}(U_{1}) \,\circ\, \textnormal{Ad}(U_{2})$.\,
	Thus, \,$\textnormal{Ad} : \SU(2) \longrightarrow \SO(3)$\, is indeed a homomorphism.
\item
	Obviously, \,$\{\,\pm\,I_{2}\,\} \subset \ker(\textnormal{Ad})$.\,
	For the reserve inclusion, let
	\,$U \in \ker(\textnormal{Ad}) \subset \SU(2)$,\,
	i.e.,
	\,$UXU^{-1} = X$,\, for each \,$X \in \su(2) \cong \mathbb{E}^{3}$;\,
	equivalently, $UX = XU$, for each $X \in \su(2) \cong \mathbb{E}^{3}$.
	Now,
	\begin{eqnarray*}
	U X
	& \!=\!\! &
		\left(\begin{array}{rr}
			a & -\,\overline{b}
			\\
			b & \overline{a}
			\end{array}\right)
		\cdot
		\left(\begin{array}{cc}
			\i\,x_{1} & x_{2} + \i\,x_{3}
			\\
			-\,x_{2} + \i\,x_{3} & -\,\i\,x_{1}
			\end{array}\right)
	\;\; = \;\;
		\left(\begin{array}{cc}
			\i\,ax_{1} - \overline{b}\,(-\,x_{2}+\i\,x_{3}) & {\color{white}-}\,a\,(x_{2}+\i\,x_{3}) + \i\,\overline{b}x_{1}
			\\
			\i\,bx_{1} + \overline{a}\,(-\,x_{2}+\i\,x_{3}) & {\color{white}-}\,b\,(x_{2}+\i\,x_{3}) - \i\,\overline{a}x_{1}
			\end{array}\!\right)
	\\ \\
	X U
	& \!=\!\! &
		\left(\begin{array}{cc}
			\i\,x_{1} & x_{2} + \i\,x_{3}
			\\
			-\,x_{2} + \i\,x_{3} & -\,\i\,x_{1}
			\end{array}\right)
		\cdot
		\left(\begin{array}{rr}
			a & -\,\overline{b}
			\\
			b & \overline{a}
			\end{array}\right)
	\;\; = \;\;
		\left(\begin{array}{cc}
			\i\,x_{1}a + (x_{2}+\i\,x_{3})\,b &  -\,\i\,x_{1}\overline{b} + (x_{2}+\i\,x_{3})\,\overline{a}
			\\
			(-\,x_{2}+\i\,x_{3})\,a - \i\,x_{1}b & -\,(-\,x_{2}+\i\,x_{3})\,\overline{b} - \i\,x_{1}\overline{a}
			\end{array}\!\right)
	\end{eqnarray*}
	Thus,
	\begin{equation}\label{UXeqXU}
	UX = XU
	\quad\Longrightarrow\quad
	\left\{\begin{array}{ccc}
		-\,\overline{b}\,(-\,x_{2} + \i\,x_{3}) & = & b\,(x_{2} + \i\,x_{3})
		\\
		a\,(x_{2} + \i\,x_{3}) + \i\,\overline{b}x_{1} & \overset{{\color{white}1}}{=} & -\,\i\,x_{1}\overline{b} + (x_{2} + \i\,x_{3})\,\overline{a}
		\end{array}\right.
	\end{equation}

	\vskip 0.2cm
	\noindent
	\textbf{Claim 1:}\quad
	The validity of the second equality in \eqref{UXeqXU} for arbitrary 
	\,$x_{1},\, x_{2},\, x_{3} \,\in\, \Re$\,
	 implies that
	$a \in \Re$, $b = 0$, as well as the validity of the first equality in \eqref{UXeqXU}.
	\vskip 0.1cm
	\noindent
	Proof of Claim 1:\;\;
	Note that
	\,$a\,(x_{2} + \i\,x_{3}) + \i\,\overline{b}x_{1}  \, = \, -\,\i\,x_{1}\overline{b} + (x_{2} + \i\,x_{3})\,\overline{a}$\,
	\;\;$\Longrightarrow$\;\;
	\,$(a - \overline{a}) \, (x_{2} + \i\,x_{3}) \, = \, - \, \i \, 2 \, x_{1}\,\overline{b}$,\,
	which is equivalent to
	\begin{equation*}
	\i \cdot 2 \cdot (x_{2} + \i\,x_{3}) \cdot \textnormal{Im}(a) \;\; = \;\; - \, \i \, \cdot 2 \cdot x_{1} \cdot \overline{b} 
	\end{equation*}
	However, \,$x_{1},\, x_{2},\, x_{3} \,\in\, \Re$\, are arbitrary, i.e.,
	the above equality must hold for each \,$x_{1},\, x_{2},\, x_{3} \,\in\, \Re$.\,
	Choosing \,$x_{2} = 1$\, and \,$x_{1} = x_{3} = 0$\, yields:
	\,$\textnormal{Im}(a) = 0$,\, i.e., \,$a \in \Re$.\,
	On the other hand, choosing \,$x_{1} = 1$\, and \,$x_{2} = x_{3} = 0$\,
	yields
	\,$\overline{b} = 0$,\, i.e., \,$b = 0$.\,
	Lastly, \,$b = 0$\, also immediately implies the validity of the first equality in \eqref{UXeqXU}.
	This completes the proof of Claim 1.
	
	\vskip 0.3cm
	\noindent
	By Claim 1, we have:
	\begin{equation*}
	U
	\;\; = \;\;
		\left(\,\begin{array}{rr}
			a & -\,\overline{b}
			\\
			b & \overline{a}
			\end{array}\,\right)
	\;\; = \;\;
		a \cdot
		\left(\,\begin{array}{rr}
			1 & 0
			\\
			0 & 1
			\end{array}\,\right),
	\end{equation*}
	where \,$a \in \Re$.\,
	Now, recall also that \,$U \in \SU(2)$;\, in particular, \,$1 \,=\, \det(U) \,=\, a^{2}$,\,
	which implies \,$a = \pm\,1$.\,
	We have thus shown that \,$U \in \{\,\pm\,I_{2}\,\}$,\,
	for each \,$U \in \ker(\textnormal{Ad})$.\,
	Thus, \,$\ker(\textnormal{Ad}) \subset \{\,\pm\,I_{2}\,\}$,\,
	and we may now conclude that \,$\ker(\textnormal{Ad}) = \{\,\pm\,I_{2}\,\}$,\,
	as required.
\item
	\noindent
	\textbf{Claim 2:}\quad
	$\textnormal{ad} \, := \, \d(\textnormal{Ad}) : \su(2) \longrightarrow \so(3)$\,
	is a Lie algebra isomorphism.
	\vskip -0.01cm
	\noindent
	Proof of Claim 2:\;\;
	Since
	\,$\textnormal{Ad} : \SU(2) \longrightarrow \SO(3)$\,
	is a Lie group homomorphism, it follows that
	\,$\textnormal{ad} \, := \, \d(\textnormal{Ad}) : \su(2) \longrightarrow \so(3)$\,
	is a Lie algebra homomorphism.
	Direct computations (omitted) show that \,$\textnormal{ad}$\, maps
	any basis for \,$\su(2)$\, to a basis for \,$\so(3)$;\, thus
	\,$\textnormal{ad} : \su(2) \longrightarrow \so(3)$\,
	is also a vector space isomorphism.
	Therefore,
	\,$\textnormal{ad} : \su(2) \longrightarrow \so(3)$\,
	is a Lie algebra homomorphism.
	This proves Claim 2.
	
	\vskip 0.2cm
	\noindent
	For an alternative (computational) proof of Claim 2, see \S3.5, p.86-87, \cite{Baker2002}.

	\vskip 0.2cm
	\noindent
	Since
	\begin{itemize}
	\item
		both \,$\SU(2)$\, and \,$\SO(3)$\, are connected Lie groups (Lie subgroups of general linear groups),
	\item
		$\textnormal{ad} \, := \, \d(\textnormal{Ad}) : \su(2) \longrightarrow \so(3)$\,
		is a Lie algebra isomorphism (by Claim 2),
	\end{itemize}
	it follows -- by Theorem 4.15, p.92, \cite{Sepanski2010} -- that
	\,$\textnormal{Ad} : \SU(2) \longrightarrow \SO(3)$\,
	is a covering map; in particular, it is surjective, as required.
	\qed
\end{enumerate}

          %%%%% ~~~~~~~~~~~~~~~~~~~~ %%%%%

\vskip 0.5cm
\noindent
\begin{corollary}
\mbox{}
\vskip 0.05cm
\noindent
$\SO(3)$\, is homeomorphic to \,$\Re\mathbb{P}^{3}$.\,
\end{corollary}

          %%%%% ~~~~~~~~~~~~~~~~~~~~ %%%%%

\vskip 0.5cm
\begin{remark}
\mbox{}
\vskip 0.1cm
\noindent
Consider
\,$U \, = \, \left(\,\begin{array}{cc} e^{\i\,{\color{red}\theta/2}} & 0 \\ 0 & e^{-\,\i\,\theta/2} \end{array}\!\!\right) \,\in\, \SU(2)$.\,
Then, it its image
\,$\textnormal{Ad}_{U} \in \SO(3)$\,
is counter-clockwise rotation (about the $x_{1}$-axis) of the $x_{2}x_{3}$-plane by the angle \,{\color{red}$\theta$}.\,
\end{remark}
\proof
We compute:
\begin{eqnarray*}
\textnormal{Ad}_{U}(X)
& = &
	UXU^{-1}
\;\; = \;\;
	\left(\,\begin{array}{cc}
		e^{\i\,{\color{red}\theta/2}} & 0
		\\
		0 & e^{-\,\i\,\theta/2}
		\end{array}\!\!\right)
	\cdot
	\left(\,\begin{array}{cc}
		\i\,x_{1} & x_{2} + \i\,x_{3}
		\\
		- \, x_{2} + \i\,x_{3} & -\,\i\,x_{1}
		\end{array}\,\right)
	\cdot
	\left(\,\begin{array}{cc}
		e^{-\,\i\,\theta/2} & 0
		\\
		0 & e^{\i\,\theta/2}
		\end{array}\!\!\right)
\\
& = &
	\left(\,\begin{array}{cc}
		e^{\i\,\theta/2} \cdot \i x_{1} & \quad\,e^{\i\,\theta/2} \cdot (x_{2} + \i\,x_{3})
		\\
		e^{-\,\i\,\theta/2} \cdot (-\,x_{2} + \i\,x_{3}) & -\,e^{-\,\i\,\theta/2} \cdot \i \cdot x_{1}
		\end{array}\!\!\right)
	\cdot
	\left(\,\begin{array}{cc}
		e^{-\,\i\,\theta/2} & 0
		\\
		0 & e^{\i\,\theta/2}
		\end{array}\!\!\right)
\\
& = &
	\left(\,\begin{array}{cc}
		\i \, x_{1} & \quad{\color{black}e^{\i\,\theta} \cdot (x_{2} + \i\,x_{3})}
		\\
		-\,e^{-\,\i\,\theta} \cdot (x_{2} - \i\,x_{3}) & -\, \i \, x_{1}
		\end{array}\!\!\right)
\\
& = &
	\left(\,\begin{array}{cc}
		\i \, x_{1} & \quad{\color{red}e^{\i\,\theta} \cdot (x_{2} + \i\,x_{3})}
		\\
		\overset{{\color{white}1}}{-\;\overline{e^{\i\,\theta} \cdot (x_{2} + \i\,x_{3})}} & -\, \i \, x_{1}
		\end{array}\!\!\right)
\end{eqnarray*}
So, the action of
\,$U \, = \, \left(\,\begin{array}{cc} e^{\i\,{\color{black}\theta/2}} & 0 \\ 0 & e^{-\,\i\,\theta/2} \end{array}\!\!\right) \,\in\, \SU(2)$\,
via \,$\textnormal{Ad}_{U}$\, on
\,$X
\, = \,
\left(\,\begin{array}{cc}
	\i\,x_{1} & x_{2} + \i\,x_{3}
	\\
	- \, x_{2} + \i\,x_{3} & -\,\i\,x_{1}
	\end{array}\,\right)
\in \su(2) \cong \mathbb{E}^{3}$\,
is such that it performs the counter-clockwise rotation (about the $x_{1}$-axis)
on the $x_{2}x_{3}$-plane by the angle \,{\color{black}$\theta$},\, leaving the $x_{1}$-axis invariant.
Thus, we see that, for
\,$U \, = \, \left(\,\begin{array}{cc} e^{\i\,{\color{black}\theta/2}} & 0 \\ 0 & e^{-\,\i\,\theta/2} \end{array}\!\!\right) \,\in\, \SU(2)$,\,
its image
\,$\textnormal{Ad}_{U} \in \SO(3)$\,
under
\,$\textnormal{Ad} : \SU(2) \longrightarrow \SO(3)$\,
is indeed counter-clockwise rotation of the $x_{2}x_{3}$-plane by the angle \,{\color{black}$\theta$}.\,
Note also:
\begin{eqnarray*}
e^{\i\theta} \cdot ({\color{white}-}\,x_{2}+\i\,x_{3})
& \!\!=\!\! &
	(\cos\theta + \i\sin\theta) \cdot ({\color{white}-}\,x_{2}+\i\,x_{3})
\; = \;
	({\color{white}-}\,x_{2}\cos\theta - x_{3}\sin\theta) \,+\, \i\,(x_{2}\sin\theta + x_{3}\cos\theta)
\\
e^{-\,\i\theta}\,(-\,x_{2}+\i\,x_{3})
& \!\!=\!\! &
	(\cos\theta - \i\sin\theta) \cdot ({\color{black}-}\,x_{2}+\i\,x_{3})
\; = \;
	(-\,x_{2}\cos\theta + x_{3}\sin\theta) \,+\, \i\,(x_{2}\sin\theta + x_{3}\cos\theta)
\end{eqnarray*}
\qed

          %%%%% ~~~~~~~~~~~~~~~~~~~~ %%%%%

\vskip 1.0cm
\section{The Lie groups \,$\SOup(1,3)$\, and \,$\SL(2,\C)$}

          %%%%% ~~~~~~~~~~~~~~~~~~~~ %%%%%

\vskip 0.3cm
\begin{definition}
\mbox{}
\vskip 0.1cm
\noindent
The \textbf{special linear group of degree $n$ over $\C$} is defined as follows:
\begin{equation*}
\SL(n,\C)
\; := \;
	\left\{\;\,
		g \overset{{\color{white}.}}{\in} \textnormal{GL}(n,\C)
		\;\,\left\vert\;\;
			\det(\,g\,) \overset{{\color{white}1}}{=} 1
			\right.
		\;\right\}
\end{equation*}
\end{definition}

\vskip 0.5cm
\noindent
Let \,$Q_{(1,n)} \,:=\, \diag(-1,1,\cdots,1) \in \Re^{(n+1) \times (n+1)}$.

\vskip 0.5cm
\begin{definition}[$\textnormal{O}(1,n)$]
\mbox{}
\vskip 0.1cm
\noindent
The \textbf{Lorentz} group is defined to be:
\begin{equation*}
\textnormal{O}(1,n)
\; := \;
	\left\{\;\,
		A \overset{{\color{white}.}}{\in} \textnormal{GL}(n+1,\Re)
		\;\left\vert\;\,
			A^{T} \cdot Q_{(1,n)} \cdot A = Q_{(1,n)}
			\right.
		\;\right\}
\end{equation*}
\end{definition}

\vskip 0.5cm
\begin{proposition}
\mbox{}
\vskip 0.1cm
\noindent
For each \,$A \in \textnormal{O}(1,n)$, we have:
\begin{enumerate}
\item
	$\det(A) = \pm 1$,\, and
\item
	$\vert\,A^{0}_{0}\,\vert^{2} \;\geq\; 1$;\, equivalently, either \,$A^{0}_{0} \,\geq\, 1$\, or \,$A^{0}_{0} \,\leq\, -1$
\end{enumerate}
\end{proposition}
\proof
\begin{enumerate}
\item
\item
\end{enumerate}
\qed

\vskip 0.5cm
\begin{definition}[$\SO(1,n)$\, and \,$\SO^{\uparrow}(1,n)$]
\mbox{}
\vskip 0.1cm
\noindent
The \textbf{proper Lorentz} group is defined to be:
\begin{equation*}
\SO(1,n)
\; := \;
	\left\{\;\,
		A \overset{{\color{white}.}}{\in} \GL(n+1,\Re)
		\;\left\vert\;\,
			\begin{array}{ccc}
			A^{T} \cdot Q_{(1,n)} \cdot A \,=\, Q_{(1,n)}\,,
			\\
			\textnormal{det}(A) \,\overset{{\color{white}1}}{=}\, 1
			\end{array}
			\right.
		\;\right\}
\end{equation*}
The \textbf{proper orthochronous Lorentz} group is defined to be:
\begin{equation*}
\SO^{\uparrow}(1,n)
\; := \;
	\left\{\;\,
		A \overset{{\color{white}.}}{\in} \GL(n+1,\Re)
		\;\left\vert\;\,
			\begin{array}{ccc}
			A^{T} \cdot Q_{(1,n)} \cdot A \,=\, Q_{(1,n)}\,,
			\\
			\textnormal{det}(A) \,\overset{{\color{white}1}}{=}\, 1\,,
			\\
			A^{0}_{0} \,\overset{{\color{white}1}}{\geq}\, 0
			\end{array}
			\right.
		\;\right\}
\end{equation*}
\end{definition}

          %%%%% ~~~~~~~~~~~~~~~~~~~~ %%%%%

          %%%%% ~~~~~~~~~~~~~~~~~~~~ %%%%%
