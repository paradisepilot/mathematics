
          %%%%% ~~~~~~~~~~~~~~~~~~~~ %%%%%

\chapter{What is the Higgs boson?}
\setcounter{theorem}{0}
\setcounter{equation}{0}

%\cite{vanDerVaart1996}
%\cite{Kosorok2008}

%\renewcommand{\theenumi}{\alph{enumi}}
%\renewcommand{\labelenumi}{\textnormal{(\theenumi)}$\;\;$}
\renewcommand{\theenumi}{\roman{enumi}}
\renewcommand{\labelenumi}{\textnormal{(\theenumi)}$\;\;$}

          %%%%% ~~~~~~~~~~~~~~~~~~~~ %%%%%

\begin{itemize}
\item
	A particle is an ``excitation'' of a quantum field.
\item
	The Higgs field was introduced into the Standard Model of Particle Physics
	in order give (positive) mass to fermions and the mediating bosons of the
	weak nuclear force.
\item
	Quantum field theory was developed to resolve the incompatibility
	between Quantum Mechanics and Special Relativity.
\item
	Each species of elementary particles is characterized by an irreducible
	representation of the Poincaré group.
	Each such irreducible representation is determined by the eigenvalues
	of its two Casimir operators.
	The physical interpretation of these two eigenvalues are the mass and spin
	of the species of elementary particles characterized by the given
	irreducible representation of the Poincaré group.
\item
	how the Standard Model was actually experimentally tested:
	\begin{enumerate}
	\item
		the Standard Model enables one to compute the likelihood of
		the possible outcomes of particle collision experiments
		(more precisely, probabilities of the species and momenta of
		the outgoing particles, given the species and momenta
		of the incoming particles), and
	\item
		experimental physicists/engineers have constructed sufficiently
		powerful devices to perform sufficiently energetic particle collision
		experiments as well as measure their outcomes (though they
		needed several decades to design and build such colliders/detectors).
	\end{enumerate}
\end{itemize}

          %%%%% ~~~~~~~~~~~~~~~~~~~~ %%%%%

