
          %%%%% ~~~~~~~~~~~~~~~~~~~~ %%%%%

\chapter{The structure of a quantum field theory}
\setcounter{theorem}{0}
\setcounter{equation}{0}

%\cite{vanDerVaart1996}
%\cite{Kosorok2008}

%\renewcommand{\theenumi}{\alph{enumi}}
%\renewcommand{\labelenumi}{\textnormal{(\theenumi)}$\;\;$}
\renewcommand{\theenumi}{\roman{enumi}}
\renewcommand{\labelenumi}{\textnormal{(\theenumi)}$\;\;$}

          %%%%% ~~~~~~~~~~~~~~~~~~~~ %%%%%

A quantum field theory is a modification of quantum mechanics
so that the resulting theory becomes compatible with special relativity,
at least in the following sense:
\begin{itemize}
\item
	space and time must be treated on equal footing in a quantum field theory, and
\item
	a quantum field theory must be able to describe (highly energetic)
	interactions of (initial) particles where
	\begin{itemize}
	\item
		the final (i.e., post-interaction) number of particles is different
		from the initial (i.e., pre-interaction) number of particles, and
	\item
		the species of the post-interaction particles are different
		from those of the pre-interaction particles.
	\end{itemize}
	This is because, according to special relativity, energy and mass are exchangeable;
	hence, sufficiently (highly) energetic particle interactions are capable of converting
	the mass and energy of the pre-interaction particles into a collection of post-interaction
	particles that has a different number of particles and are of different species as the pre-interaction particles.
\end{itemize}

          %%%%% ~~~~~~~~~~~~~~~~~~~~ %%%%%

\noindent
In a quantum field theory:
\begin{itemize}
\item
	Recall that, in quantum mechanics, the states of a particle are represented by one-dimensional subspaces
	in a complex Hilbert space $\mathcal{H}$.
	\vskip 0.1cm
	Recall that the Poincaré group is the symmetry group of spacetime in special relativity.
	One thus expects that the Poincaré group to admit a unitary action on $\mathcal{H}$.
	\vskip 0.1cm
	We assume that the particle is ``elementary'' (i.e., is not the bound state of more elementary constituent particles)
	if and only if its corresponding Poincaré group (unitary) action on $\mathcal{H}$ is irreducible.
\item
	Since elementary particles exhibit ``internal symmetry'', we also expect that,
	for each species of elementary particles and
	for each internal symmetry exhibited by this given species of elementary particles,
	there corresponds an irreducible unitary representation of the internal symmetry group
	on some finite-dimensional complex Hilbert space.
	The finite-dimensionality assumption here is due to the belief (and experimental observations)
	that the number of possible states ``generated'' under each such internal symmetry group action is finite.
\item
	In order to avoid ``action at a distance'', one therefore expects that each (non-interacting) species
	of elementary particles to be mathematically describable -- in ``pre-quantized'' form -- by a section of a complex vector bundle
	over spacetime (position Minkowski space), equipped with a Poincaré group action
	as well as (fibrewise) actions of internal symmetry groups.
	The fibre of this complex vector bundle will be the tensor product of $L^{2}(\Re^{1,3},\d\mu)$
	with the state spaces (which are finite-dimensional complex Hilbert spaces) of the internal symmetry groups
	of the given elementary particle species.
	\vskip 0.1cm
	One such appropriate class of complex vector bundles is the vector bundles associated with principle fibre bundles.	
\item
	Since physical phenomena should be unaffected by the aforementioned
	vector bundles and principal fibre bundles are ``coordinatized'', one expects that
	any physical law to be Poincaré-invariant as well as gauge invariant (i.e., invariant under the internal symmetry actions). 
\item
	Poincaré-invariant and gauge-invariant Lagrangian density: $\mathcal{L}(\,\cdots\,)$.
	From classical mechanics, one expects the Lagrangian to involve the field $\phi$ as well as its first derivatives:
	\begin{equation*}
	\mathcal{L} \;\; = \;\; \mathcal{L}(\phi,\partial\phi),
	\end{equation*}
	where the occurrence of the first derivative $\partial\phi$ of the field $\phi$ is expected to be related to dynamics.
	\vskip 0.1cm
	However, on an associated vector bundle of a principal fibre bundle, there is in fact no canonically defined notion
	of derivatives (of sections of the vector bundle).
	We need the extra structure of a \textit{connection} $A$ on the principal fibre bundle
	in order to be able to define the (covariant) derivative $D_{A}(\,\phi\,)$ of $\phi$ with respect to $A$.
	Thus, we expect the Lagrangian to be of the following form instead:
	\begin{equation*}
	\mathcal{L} \;\; = \;\; \mathcal{L}\!\left(\,\overset{{\color{white}.}}{\phi}\,,D_{A}(\,\phi\,)\right).
	\end{equation*}
\item
	\begin{equation*}
	\mathcal{L}
	\;\; = \;\;
		\mathcal{L}_{\textnormal{\,free}}\!\left(\,\overset{{\color{white}.}}{\phi}\,,D_{A}(\,\phi\,)\right)
	\,+\,
		\mathcal{L}_{\textnormal{\,int}}\!\left(\,\overset{{\color{white}.}}{\phi}\,,D_{A}(\,\phi\,)\right)
	\end{equation*}
\end{itemize}

          %%%%% ~~~~~~~~~~~~~~~~~~~~ %%%%%

