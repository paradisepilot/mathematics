
          %%%%% ~~~~~~~~~~~~~~~~~~~~ %%%%%

\chapter{Quantum electrodynamics}
\setcounter{theorem}{0}
\setcounter{equation}{0}

%\cite{vanDerVaart1996}
%\cite{Kosorok2008}

%\renewcommand{\theenumi}{\alph{enumi}}
%\renewcommand{\labelenumi}{\textnormal{(\theenumi)}$\;\;$}
\renewcommand{\theenumi}{\roman{enumi}}
\renewcommand{\labelenumi}{\textnormal{(\theenumi)}$\;\;$}

          %%%%% ~~~~~~~~~~~~~~~~~~~~ %%%%%

\begin{itemize}
\item
	Quantum electrodynamics (QED) is a quantum field theory that can describe
	the interaction between an electrically charged particle and an electromagnetic field
	whose theoretical predictions agree with experimental observations with astounding accuracy
	(e.g., 1 part in a billion, or even better).
\item
	We first develop the interaction-free pre-quantized field theory for the electron.
	This amounts to the following:
	\begin{itemize}
	\item
		Seek a special-relativistic relation between energy and momentum satisfied by the electron.
	\item
		Convert this special-relativistic energy-momentum relation into a relativistic wave equation
		for $\C$-valued functions on (position) Minkowski spacetime.
		This ``conversion'' (relativistic energy-momentum relation to relativistic wave equation)
		is called \textbf{first quantization}.
		The resulting partial differential equation is the Dirac equation.
	\item
		Solutions to the Dirac equation are interpreted to be ``pre-quantized fields'' associated to electrons, and
		these solutions can be expressed using the Fourier transform (integrals over momentum Minkowski spacetime).
	\item
		Solutions to the Dirac equation are then ``re-interpreted'' (this is ``\textbf{second quantization}'')
		as operators on a suitable Fock space,
		which is a complex Hilbert space constructed via tensor products and direct sums of single-particle state spaces
		(themselves complex Hilbert spaces).
		Coefficients appearing in the integrand of the Fourier transform (integral over momentum Minkowski spacetime)
		of a solution of the Dirac equation then become (are interpreted)
		as momentum-specific creation and annihilation operators.
	\item
		The ``outcome'' of applying the creation operator corresponding to a specific momentum $4$-vector
		to the vacuum state (a special element of the aforementioned Fock space)
		is an element of the Fock space that is the quantum state corresponding to the occurrence of an electron
		with the given momentum $4$-vector.
		Such an application (of a creation operator to the vacuum state, resulting in the creation of an electron)
		is called an \textbf{excitation} of the vacuum. 
	\end{itemize}
\item
	The interaction-free pre-quantized field theory for the electromagnetic field is similar.
\item
	Interactions via perturbation theory
	\begin{itemize}
	\item
		The Dirac equation can be reverse-engineered to produce a non-interaction Lagrangian density for the electron.
		Similarly, the Maxwell equation can be reverse-engineered to produce a non-interaction Lagrangian density for the electromagnetic field.
	\item
		A Lagrangian density suitable for describing the interaction between the electron and the electromagnetic field
		is obtained by adding the two aforementioned non-interaction Lagrangian densities, aa well as
		a new ``interaction term'' (the product of the interaction-free fields times a coupling constant).
	\item
		The QED coupling constant is determined experimentally to be small.
	\item
		Using the smallness of the coupling constant, perturbation theory is applied to compute
		\textit{decay rates} and \textit{scattering cross sections}
		as power series in the coupling constant.
	\end{itemize}
\end{itemize}

          %%%%% ~~~~~~~~~~~~~~~~~~~~ %%%%%

