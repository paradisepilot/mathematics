
          %%%%% ~~~~~~~~~~~~~~~~~~~~ %%%%%

\chapter{The double cover $\SL(2,\C) \longrightarrow \SOup(1,3)$}
\setcounter{theorem}{0}
\setcounter{equation}{0}

%\cite{vanDerVaart1996}
%\cite{Kosorok2008}

%\renewcommand{\theenumi}{\alph{enumi}}
%\renewcommand{\labelenumi}{\textnormal{(\theenumi)}$\;\;$}
\renewcommand{\theenumi}{\roman{enumi}}
\renewcommand{\labelenumi}{\textnormal{(\theenumi)}$\;\;$}

          %%%%% ~~~~~~~~~~~~~~~~~~~~ %%%%%

\section{The Lie groups \,$\SOup(1,3)$\, and \,$\SL(2,\C)$}

          %%%%% ~~~~~~~~~~~~~~~~~~~~ %%%%%

%\vskip 0.3cm
%\begin{definition}
%\mbox{}
%\vskip 0.1cm
%\noindent
%The \textbf{special linear group of degree $n$ over $\C$} is defined as follows:
%\begin{equation*}
%\SL(n,\C)
%\; := \;
%	\left\{\;\,
%		g \overset{{\color{white}.}}{\in} \textnormal{GL}(n,\C)
%		\;\,\left\vert\;\;
%			\det(\,g\,) \overset{{\color{white}1}}{=} 1
%			\right.
%		\;\right\}
%\end{equation*}
%\end{definition}

Let \,$Q_{(1,n)} \,:=\, \diag(-1,1,\cdots,1) \in \Re^{(n+1) \times (n+1)}$.

\vskip 0.5cm
\begin{definition}[$\textnormal{O}(1,n)$]
\mbox{}
\vskip 0.1cm
\noindent
The \textbf{Lorentz group of \,$\textnormal{O}(1,n)$}\, is defined to be:
\begin{equation*}
\textnormal{O}(1,n)
\; := \;
	\left\{\;\,
		A \overset{{\color{white}.}}{\in} \textnormal{GL}(n+1,\Re)
		\;\left\vert\;\,
			A^{T} \cdot Q_{(1,n)} \cdot A = Q_{(1,n)}
			\right.
		\;\right\}
\end{equation*}
\end{definition}

\vskip 0.5cm
\begin{proposition}
\mbox{}
\vskip 0.1cm
\noindent
For each \,$A \in \textnormal{O}(1,n)$, we have:
\begin{enumerate}
\item
	$\det(A) = \pm 1$,\, and
\item
	$\vert\,A^{0}_{0}\,\vert^{2} \;\geq\; 1$;\, equivalently, either \,$A^{0}_{0} \,\geq\, 1$\, or \,$A^{0}_{0} \,\leq\, -1$
\end{enumerate}
\end{proposition}
\proof
\begin{enumerate}
\item
	\begin{eqnarray*}
	&&
		A^{T} \cdot Q_{(1,n)} \cdot A \,=\, Q_{(1,n)}
	\\
	& \Longrightarrow &
		\det(Q_{(1,n)})
		\, = \,
		\det\!\left(\,A^{T} \overset{{\color{white}1}}{\cdot} Q_{(1,n)} \cdot A\,\right)
		\,=\,
		\det(\,A^{T}\,) \cdot \det(Q_{(1,n)}) \cdot \det(A)
	\\
	& \Longrightarrow &
		-1
		\, = \,
		\det(Q_{(1,n)})
		\, = \,
		\det(\,A^{T}\,) \cdot \det(Q_{(1,n)}) \cdot \det(A)
		\, = \,
		\det(\,A^{T}\,) \cdot (\,-1\,)\cdot \det(A)
		\, = \,
		\det(A)^{2}
	\\
	& \Longrightarrow &
		\det(A)
		\, \overset{{\color{white}1}}{=} \,
		\pm 1
	\end{eqnarray*}
\item
	Note
	\begin{equation*}
	A^{T} \cdot Q_{(1,n)} \cdot A \,=\, Q_{(1,n)}
	\quad\Longleftrightarrow\quad
	\overset{3}{\underset{\rho\,=\,0}{\sum}}\;
		\overset{3}{\underset{\mu\,=\,0}{\sum}}\;
		A_{\mu\rho}\,\eta_{\rho\sigma}\,A_{\sigma\nu}
		\; = \;
		\eta_{\mu\nu}\,,
	\end{equation*}
	which -- upon setting $\mu = \nu = 0$ -- implies:
	\begin{eqnarray*}
	&&
		A_{{\color{red}0}0} \cdot \eta_{00} \cdot A_{0{\color{red}0}}
		\,+\,
		A_{{\color{red}0}1} \cdot \eta_{11} \cdot A_{1{\color{red}0}}
		\,+\,
		A_{{\color{red}0}2} \cdot \eta_{22} \cdot A_{2{\color{red}0}}
		\,+\,
		A_{{\color{red}0}3} \cdot \eta_{33} \cdot A_{3{\color{red}0}}
		\; = \;
		\eta_{{\color{red}00}}
	\\
	& \Longrightarrow &
		A_{{\color{red}0}0}\,(-1)\,A_{0{\color{red}0}}
		\,+\,
		A_{{\color{red}0}1}\,(+1)\,A_{1{\color{red}0}}
		\,+\,
		A_{{\color{red}0}2}\,(+1)\,A_{2{\color{red}0}}
		\,+\,
		A_{{\color{red}0}3}\,(+1)\,A_{3{\color{red}0}}
		\; = \;
		-\,1
	\\
	& \Longrightarrow &
		- \, A_{{\color{red}0}0}^{2}
		\; + \,
		\overset{3}{\underset{k\,=\,0}{\sum}}\;
		A_{{\color{red}0}\rho}^{2}
		\;\; = \;\,
			-\,1
	\;\;\; \Longrightarrow \;\;\;\,
		A_{{\color{red}0}0}^{2}
		\;\; = \;\,
			1
			\; + \,
			\overset{3}{\underset{k\,=\,0}{\sum}}\;
			A_{{\color{red}0}\rho}^{2}
		\;\; \geq \;\,
			1\,,
	\end{eqnarray*}
	as required.
	\qed
\end{enumerate}

          %%%%% ~~~~~~~~~~~~~~~~~~~~ %%%%%

\vskip 0.5cm
\begin{corollary}
\mbox{}
\vskip 0.1cm
\noindent
The \textbf{Lorentz group of \,$\textnormal{O}(1,n)$}\, is the disjoint union of the following four connected components:
\begin{eqnarray*}
&&
{\color{white}\bigsqcup}\;
\left\{\;
	A \overset{{\color{white}.}}{\in} \textnormal{O}(1,n)
	\;\left\vert\;
		\begin{array}{ccc}
		\textnormal{det}(A) \,\overset{{\color{white}1}}{=}\, +1
		\\
		A^{0}_{0} \,\overset{{\color{white}1}}{\geq}\, 1
		\end{array}
		\right.
	\!\right\}
\;\bigsqcup\;
\left\{\;
	A \overset{{\color{white}.}}{\in} \textnormal{O}(1,n)
	\;\left\vert\;
		\begin{array}{ccc}
		\textnormal{det}(A) \,\overset{{\color{white}1}}{=}\, +1
		\\
		A^{0}_{0} \,\overset{{\color{white}1}}{\leq}\, -1
		\end{array}
		\right.
	\!\right\}
\\
&&
\bigsqcup\;
\left\{\;
	A \overset{{\color{white}.}}{\in} \textnormal{O}(1,n)
	\;\left\vert\;
		\begin{array}{ccc}
		\textnormal{det}(A) \,\overset{{\color{white}1}}{=}\, -1
		\\
		A^{0}_{0} \,\overset{{\color{white}1}}{\geq}\, 1
		\end{array}
		\right.
	\!\right\}
\;\bigsqcup\;
\left\{\;
	A \overset{{\color{white}.}}{\in} \textnormal{O}(1,n)
	\;\left\vert\;
		\begin{array}{ccc}
		\textnormal{det}(A) \,\overset{{\color{white}1}}{=}\, -1
		\\
		A^{0}_{0} \,\overset{{\color{white}1}}{\leq}\, -1
		\end{array}
		\right.
	\!\right\}
\end{eqnarray*}
\end{corollary}

          %%%%% ~~~~~~~~~~~~~~~~~~~~ %%%%%

\vskip 0.5cm
\begin{definition}[$\SO(1,n)$\, and \,$\SO^{\uparrow}(1,n)$]
\mbox{}
\vskip 0.1cm
\noindent
The \textbf{special Lorentz} group is defined to be:
\begin{eqnarray*}
\SO(1,n)
& := &
	\left\{\;
		A \overset{{\color{white}.}}{\in} \GL(n+1,\Re)
		\;\left\vert\,
			\begin{array}{ccc}
			A^{T} \cdot Q_{(1,n)} \cdot A \,=\, Q_{(1,n)}
			\\
			\textnormal{det}(A) \,\overset{{\color{white}1}}{=}\, 1
			\end{array}
			\right.
		\!\right\}
\\
& = &
	\left\{\;
		A \overset{{\color{white}.}}{\in} \textnormal{O}(1,n)
		\;\left\vert\;
			\begin{array}{ccc}
			\textnormal{det}(A) \,\overset{{\color{white}1}}{=}\, +1
			\\
			A^{0}_{0} \,\overset{{\color{white}1}}{\geq}\, 1
			\end{array}
			\right.
		\!\right\}
	\;\bigsqcup\;
	\left\{\;
		A \overset{{\color{white}.}}{\in} \textnormal{O}(1,n)
		\;\left\vert\;
			\begin{array}{ccc}
			\textnormal{det}(A) \,\overset{{\color{white}1}}{=}\, +1
			\\
			A^{0}_{0} \,\overset{{\color{white}1}}{\leq}\, -1
			\end{array}
			\right.
		\!\right\}
\end{eqnarray*}
The \textbf{orthochronous special Lorentz} group is defined to be:
\begin{equation*}
\SO^{\uparrow}(1,n)
\; := \;
	\left\{\;
		A \overset{{\color{white}.}}{\in} \GL(n+1,\Re)
		\;\left\vert\,
			\begin{array}{ccc}
			A^{T} \cdot Q_{(1,n)} \cdot A \,=\, Q_{(1,n)}
			\\
			\textnormal{det}(A) \,\overset{{\color{white}1}}{=}\, 1
			\\
			A^{0}_{0} \,\overset{{\color{white}1}}{\geq}\, 1
			\end{array}
			\right.
		\!\right\}
\; = \;
	\left\{\;
		A \overset{{\color{white}.}}{\in} \textnormal{O}(1,n)
		\;\left\vert\;
			\begin{array}{ccc}
			\textnormal{det}(A) \,\overset{{\color{white}1}}{=}\, +1
			\\
			A^{0}_{0} \,\overset{{\color{white}1}}{\geq}\, 1
			\end{array}
			\right.
		\!\right\}
\end{equation*}
\end{definition}

          %%%%% ~~~~~~~~~~~~~~~~~~~~ %%%%%


\vskip 1.0cm
\noindent
\textbf{\large The \,$\SL(2,\C)$\, action on Minkowski space \,$\Re^{1,3}$}
\vskip 0.3cm
\noindent
We start by identifying an $\Re$-linear subspace \,$V$\, of \,$\C^{2 \times 2}$\, 
and equipping it with an inner product such that the resulting inner product space
is isomorphic to the Minkowski space \,$\Re^{1,3}$.\,

\vskip 0.2cm
\noindent
\begin{proposition}
\mbox{}
\vskip 0.1cm
\noindent
Let \,$\SkHermTwoC \subset \C^{2 \times 2}$\, be the set of all $2 \times 2$ skew-Hermitian ($X^{\dagger} = -X$) complex matrices:
\begin{equation*}
\SkHermTwoC
\;\; := \;\;
	\left\{\;
		\left.\left(\begin{array}{cc}
			\i\,(x_{0}+x_{1}) & x_{2}+\i\,x_{3}
			\\
			-\,x_{2}+\i\,x_{3} & \i\,(x_{0}-x_{1})
			\end{array}\right)
		\in
		\C^{2 \times 2}
		\;\;\right\vert\;\;
		x_{0}, x_{1}, x_{2}, x_{4} \in \Re
		\;\right\}
\end{equation*}
Define the bilinear map \,$\langle\,\cdot\,,\,\cdot\,\rangle_{\SkHermTwoC}$\, on \,$\SkHermTwoC$\, as follows:
\begin{eqnarray*}
\left\langle\,
	\overset{{\color{white}.}}{X}
	\, , \,
	Y
	\,\right\rangle_{\!\SkHermTwoC}
& := &
	\dfrac{1}{4}\cdot\left(\,
		\det(X + Y)
		\, \overset{{\color{white}1}}{-} \,
		\det(X - Y)
		\,\right)
\end{eqnarray*}
Then, the following statements are true:
\begin{enumerate}
\item
	$\SkHermTwoC$\, is an $\Re$-linear subspace of \,$\C^{2 \times 2}$\, with \,$\dim_{\Re}(\SkHermTwoC) = 4$.\,
\item
	$\left(\,\overset{{\color{white}.}}{\SkHermTwoC}\,,\,\langle\,\cdot\,,\,\cdot\,\rangle_{\SkHermTwoC}\,\right)$\,
	is isometric to the $4$-dimensional Minkowski spacetime \,$\Re^{1,4}$.\,
\end{enumerate}
\end{proposition}
\proof
\begin{enumerate}
\item
	%It is clear that \,$\SkHermTwoC$\, is an $\Re$-linear subspace of \,$\C^{2 \times 2}$\, with \,$\dim_{\Re}(\SkHermTwoC) = 4$.\,
	Trivial.
\item
	First, note that:
	\begin{eqnarray*}
	\det(X)
	& = &
		\det\!\left(\begin{array}{cc}
			\i\,(x_{0}+x_{1}) & x_{2} + \i\,x_{3}
			\\
			- \, x_{2} + \i\,x_{3} & \i\,(x_{0}-x_{1})
			\end{array}\right)
	\\
	& = &
		\i^{2}(x_{0}+x_{1})(x_{0}-x_{1}) \,-\, (x_{2}+\i\,x_{3})(-x_{2}+\i\,x_{3})
	\;\; = \;\;
		-\,(x_{0}^{2}-x_{1}^{2}) \,+\, (x_{2}+\i\,x_{3})(x_{2}-\i\,x_{3})
	\\
	& = &
		\overset{{\color{white}1}}{-\,x_{0}^{2} \,+\, x_{1}^{2} \,+\, x_{2}^{2} \,+\, x_{3}^{2}}
	\end{eqnarray*}
	Next, we compute:
	\begin{eqnarray*}
	\left\langle\,
		\overset{{\color{white}.}}{X}
		\, , \,
		Y
		\,\right\rangle_{\!\SkHermTwoC}
	& := &
		\dfrac{1}{4}\cdot\left(\,
			\det(X + Y)
			\, \overset{{\color{white}1}}{-} \,
			\det(X - Y)
			\,\right)
	\\
	& = &
		{\color{white}-} \, \dfrac{1}{4}\cdot
		\det\!\left(\begin{array}{cc}
			\i\,(x_{0}+y_{0}+x_{1}+y_{1}) & (x_{2}+y_{2}) + \i\,(x_{3}+y_{3})
			\\
			- \, (x_{2}+y_{2}) + \i\,(x_{3}+y_{3}) & \i\,(x_{0}+y_{0}-x_{1}-y_{1})
			\end{array}\right)
	\\
	&&
		{\color{black}-} \, \dfrac{1}{4}\cdot
		\det\!\left(\begin{array}{cc}
			\i\,(x_{0}-y_{0}+x_{1}-y_{1}) & (x_{2}-y_{2}) + \i\,(x_{3}-y_{3})
			\\
			- \, (x_{2}-y_{2}) + \i\,(x_{3}-y_{3}) & \i\,(x_{0}-y_{0}-x_{1}+y_{1})
			\end{array}\right)
	\\
	& = &
		{\color{white}-} \, \dfrac{1}{4}\cdot\left(\,
			\overset{{\color{white}1}}{-}\,
			(x_{0}+y_{0})^{2}
			\,+\, (x_{1}+y_{1})^{2}
			\,+\, (x_{2}+y_{2})^{2}
			\,+\, (x_{3}+y_{3})^{2}
			\,\right)
	\\
	& &
		{\color{black}-} \, \dfrac{1}{4}\cdot\left(\,
			\overset{{\color{white}1}}{-}\,
			(x_{0}-y_{0})^{2}
			\,+\, (x_{1}-y_{1})^{2}
			\,+\, (x_{2}-y_{2})^{2}
			\,+\, (x_{3}-y_{3})^{2}
			\,\right)
	\\
	& = &
		{\color{white}-} \, \dfrac{1}{4}\cdot\left(\,
			\overset{{\color{white}1}}{-}\,
			x_{0}^{2} - 2\,x_{0}y_{0} - y_{0}^{2}
			\;+\,
			\overset{3}{\underset{k\,=\,1}{\sum}}\,(x_{k}^{2} + 2\,x_{k}y_{k}+y_{k}^{2})
			\,\right)
	\\
	& &
		{\color{black}-} \, \dfrac{1}{4}\cdot\left(\,
			\overset{{\color{white}1}}{-}\,
			x_{0}^{2} + 2\,x_{0}y_{0} - y_{0}^{2}
			\;+\,
			\overset{3}{\underset{k\,=\,1}{\sum}}\,(x_{k}^{2} - 2\,x_{k}y_{k}+y_{k}^{2})
			\,\right)
	\\
	& = &
		\overset{{\color{white}1}}{-\,x_{0}y_{0} \,+\, x_{1}\,y_{1} \,+\, x_{2}\,y_{2} \,+\, x_{3}\,y_{3}}
	\end{eqnarray*}
	This proves that
	\,$\left(\,\overset{{\color{white}.}}{\SkHermTwoC}\,,\,\langle\,\cdot\,,\,\cdot\,\rangle_{\SkHermTwoC}\,\right)$\,
	is indeed isometric to the $4$-dimensional\\ Minkowski spacetime \,$\Re^{1,4}$.\,
	\qed
\end{enumerate}


          %%%%% ~~~~~~~~~~~~~~~~~~~~ %%%%%
