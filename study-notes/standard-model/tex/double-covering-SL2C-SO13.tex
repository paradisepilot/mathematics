
          %%%%% ~~~~~~~~~~~~~~~~~~~~ %%%%%

\chapter{The double cover $\SL(2,\C) \longrightarrow \SOup(1,3)$}
\setcounter{theorem}{0}
\setcounter{equation}{0}

%\cite{vanDerVaart1996}
%\cite{Kosorok2008}

%\renewcommand{\theenumi}{\alph{enumi}}
%\renewcommand{\labelenumi}{\textnormal{(\theenumi)}$\;\;$}
\renewcommand{\theenumi}{\roman{enumi}}
\renewcommand{\labelenumi}{\textnormal{(\theenumi)}$\;\;$}

          %%%%% ~~~~~~~~~~~~~~~~~~~~ %%%%%

\section{The Lie groups \,$\SOup(1,3)$\, and \,$\SL(2,\C)$}

          %%%%% ~~~~~~~~~~~~~~~~~~~~ %%%%%

%\vskip 0.3cm
%\begin{definition}
%\mbox{}
%\vskip 0.1cm
%\noindent
%The \textbf{special linear group of degree $n$ over $\C$} is defined as follows:
%\begin{equation*}
%\SL(n,\C)
%\; := \;
%	\left\{\;\,
%		g \overset{{\color{white}.}}{\in} \textnormal{GL}(n,\C)
%		\;\,\left\vert\;\;
%			\det(\,g\,) \overset{{\color{white}1}}{=} 1
%			\right.
%		\;\right\}
%\end{equation*}
%\end{definition}

Let \,$Q_{(1,n)} \,:=\, \diag(-1,1,\cdots,1) \in \Re^{(n+1) \times (n+1)}$.

\vskip 0.5cm
\begin{definition}[$\textnormal{O}(1,n)$]
\mbox{}
\vskip 0.1cm
\noindent
The \textbf{Lorentz group of \,$\textnormal{O}(1,n)$}\, is defined to be:
\begin{equation*}
\textnormal{O}(1,n)
\; := \;
	\left\{\;\,
		A \overset{{\color{white}.}}{\in} \textnormal{GL}(n+1,\Re)
		\;\left\vert\;\,
			A^{T} \cdot Q_{(1,n)} \cdot A = Q_{(1,n)}
			\right.
		\;\right\}
\end{equation*}
\end{definition}

\vskip 0.5cm
\begin{proposition}
\mbox{}
\vskip 0.1cm
\noindent
For each \,$A \in \textnormal{O}(1,n)$, we have:
\begin{enumerate}
\item
	$\det(A) = \pm 1$,\, and
\item
	$\vert\,A^{0}_{0}\,\vert^{2} \;\geq\; 1$;\, equivalently, either \,$A^{0}_{0} \,\geq\, 1$\, or \,$A^{0}_{0} \,\leq\, -1$
\end{enumerate}
\end{proposition}
\proof
\begin{enumerate}
\item
	\begin{eqnarray*}
	&&
		A^{T} \cdot Q_{(1,n)} \cdot A \,=\, Q_{(1,n)}
	\\
	& \Longrightarrow &
		\det(Q_{(1,n)})
		\, = \,
		\det\!\left(\,A^{T} \overset{{\color{white}1}}{\cdot} Q_{(1,n)} \cdot A\,\right)
		\,=\,
		\det(\,A^{T}\,) \cdot \det(Q_{(1,n)}) \cdot \det(A)
	\\
	& \Longrightarrow &
		-1
		\, = \,
		\det(Q_{(1,n)})
		\, = \,
		\det(\,A^{T}\,) \cdot \det(Q_{(1,n)}) \cdot \det(A)
		\, = \,
		\det(\,A^{T}\,) \cdot (\,-1\,)\cdot \det(A)
		\, = \,
		\det(A)^{2}
	\\
	& \Longrightarrow &
		\det(A)
		\, \overset{{\color{white}1}}{=} \,
		\pm 1
	\end{eqnarray*}
\item
	Note
	\begin{equation*}
	A^{T} \cdot Q_{(1,n)} \cdot A \,=\, Q_{(1,n)}
	\;\;\;\Longleftrightarrow\;\;\;
	\overset{3}{\underset{\rho\,=\,0}{\sum}}\;
		\overset{3}{\underset{\mu\,=\,0}{\sum}}\;
		A^{T}_{{\color{red}\mu\rho}}\,\eta_{\rho\sigma}\,A_{\sigma\nu}
		\, = \,
		\eta_{\mu\nu}
	\;\;\;\Longleftrightarrow\;\;\;
	\overset{3}{\underset{\rho\,=\,0}{\sum}}\;
		\overset{3}{\underset{\mu\,=\,0}{\sum}}\;
		A_{{\color{red}\rho\mu}}\,\eta_{\rho\sigma}\,A_{\sigma\nu}
		\, = \,
		\eta_{\mu\nu}\,,
	\end{equation*}
	which -- upon setting $\mu = \nu = 0$ -- implies:
	\begin{eqnarray*}
	&&
		A_{0{\color{red}0}} \cdot \eta_{00} \cdot A_{0{\color{red}0}}
		\,+\,
		A_{1{\color{red}0}} \cdot \eta_{11} \cdot A_{1{\color{red}0}}
		\,+\,
		A_{2{\color{red}0}} \cdot \eta_{22} \cdot A_{2{\color{red}0}}
		\,+\,
		A_{3{\color{red}0}} \cdot \eta_{33} \cdot A_{3{\color{red}0}}
		\; = \;
		\eta_{{\color{red}00}}
	\\
	& \Longrightarrow &
		A_{0{\color{red}0}}\,(-1)\,A_{0{\color{red}0}}
		\,+\,
		A_{1{\color{red}0}}\,(+1)\,A_{1{\color{red}0}}
		\,+\,
		A_{2{\color{red}0}}\,(+1)\,A_{2{\color{red}0}}
		\,+\,
		A_{3{\color{red}0}}\,(+1)\,A_{3{\color{red}0}}
		\; = \;
		-\,1
	\\
	& \Longrightarrow &
		- \, A_{0{\color{red}0}}^{2}
		\; + \,
		\overset{3}{\underset{k\,=\,0}{\sum}}\;
		A_{k{\color{red}0}}^{2}
		\;\; = \;\,
			-\,1
	\;\;\; \Longrightarrow \;\;\;\,
		A_{{\color{red}0}0}^{2}
		\;\; = \;\,
			1
			\; + \,
			\overset{3}{\underset{k\,=\,0}{\sum}}\;
			A_{k{\color{red}0}}^{2}
		\;\; \geq \;\,
			1\,,
	\end{eqnarray*}
	as required.
	\qed
\end{enumerate}

          %%%%% ~~~~~~~~~~~~~~~~~~~~ %%%%%

\vskip 0.5cm
\begin{corollary}
\mbox{}
\vskip 0.1cm
\noindent
The \textbf{Lorentz group of \,$\textnormal{O}(1,n)$}\, is the disjoint union of the following four connected components:
\begin{eqnarray*}
&&
{\color{white}\bigsqcup}\;
\left\{\;
	A \overset{{\color{white}.}}{\in} \textnormal{O}(1,n)
	\;\left\vert\;
		\begin{array}{ccc}
		\textnormal{det}(A) \,\overset{{\color{white}1}}{=}\, +1
		\\
		A^{0}_{0} \,\overset{{\color{white}1}}{\geq}\, 1
		\end{array}
		\right.
	\!\right\}
\;\bigsqcup\;
\left\{\;
	A \overset{{\color{white}.}}{\in} \textnormal{O}(1,n)
	\;\left\vert\;
		\begin{array}{ccc}
		\textnormal{det}(A) \,\overset{{\color{white}1}}{=}\, +1
		\\
		A^{0}_{0} \,\overset{{\color{white}1}}{\leq}\, -1
		\end{array}
		\right.
	\!\right\}
\\
&&
\bigsqcup\;
\left\{\;
	A \overset{{\color{white}.}}{\in} \textnormal{O}(1,n)
	\;\left\vert\;
		\begin{array}{ccc}
		\textnormal{det}(A) \,\overset{{\color{white}1}}{=}\, -1
		\\
		A^{0}_{0} \,\overset{{\color{white}1}}{\geq}\, 1
		\end{array}
		\right.
	\!\right\}
\;\bigsqcup\;
\left\{\;
	A \overset{{\color{white}.}}{\in} \textnormal{O}(1,n)
	\;\left\vert\;
		\begin{array}{ccc}
		\textnormal{det}(A) \,\overset{{\color{white}1}}{=}\, -1
		\\
		A^{0}_{0} \,\overset{{\color{white}1}}{\leq}\, -1
		\end{array}
		\right.
	\!\right\}
\end{eqnarray*}
\end{corollary}

          %%%%% ~~~~~~~~~~~~~~~~~~~~ %%%%%

\vskip 0.5cm
\begin{definition}[$\SO(1,n)$\, and \,$\SO^{\uparrow}(1,n)$]
\mbox{}
\vskip 0.1cm
\noindent
The \textbf{special Lorentz group} is defined to be:
\begin{eqnarray*}
\SO(1,n)
& := &
	\left\{\;
		A \overset{{\color{white}.}}{\in} \GL(n+1,\Re)
		\;\left\vert\,
			\begin{array}{ccc}
			A^{T} \cdot Q_{(1,n)} \cdot A \,=\, Q_{(1,n)}
			\\
			\textnormal{det}(A) \,\overset{{\color{white}1}}{=}\, 1
			\end{array}
			\right.
		\!\right\}
\\
& = &
	\left\{\;
		A \overset{{\color{white}.}}{\in} \textnormal{O}(1,n)
		\;\left\vert\;
			\begin{array}{ccc}
			\textnormal{det}(A) \,\overset{{\color{white}1}}{=}\, +1
			\\
			A^{0}_{0} \,\overset{{\color{white}1}}{\geq}\, 1
			\end{array}
			\right.
		\!\right\}
	\;\bigsqcup\;
	\left\{\;
		A \overset{{\color{white}.}}{\in} \textnormal{O}(1,n)
		\;\left\vert\;
			\begin{array}{ccc}
			\textnormal{det}(A) \,\overset{{\color{white}1}}{=}\, +1
			\\
			A^{0}_{0} \,\overset{{\color{white}1}}{\leq}\, -1
			\end{array}
			\right.
		\!\right\}
\end{eqnarray*}
The \textbf{orthochronous special Lorentz group} is defined to be:
\begin{equation*}
\SO^{\uparrow}(1,n)
\; := \;
	\left\{\;
		A \overset{{\color{white}.}}{\in} \GL(n+1,\Re)
		\;\left\vert\,
			\begin{array}{ccc}
			A^{T} \cdot Q_{(1,n)} \cdot A \,=\, Q_{(1,n)}
			\\
			\textnormal{det}(A) \,\overset{{\color{white}1}}{=}\, 1
			\\
			A^{0}_{0} \,\overset{{\color{white}1}}{\geq}\, 1
			\end{array}
			\right.
		\!\right\}
\; = \;
	\left\{\;
		A \overset{{\color{white}.}}{\in} \textnormal{O}(1,n)
		\;\left\vert\;
			\begin{array}{ccc}
			\textnormal{det}(A) \,\overset{{\color{white}1}}{=}\, +1
			\\
			A^{0}_{0} \,\overset{{\color{white}1}}{\geq}\, 1
			\end{array}
			\right.
		\!\right\}
\end{equation*}
\end{definition}

          %%%%% ~~~~~~~~~~~~~~~~~~~~ %%%%%


\vskip 1.0cm
\noindent
\textbf{\large The \,$\SL(2,\C)$\, action on Minkowski space \,$\Re^{1,3}$}
\vskip 0.3cm
\noindent
We start by identifying an $\Re$-linear subspace \,$V$\, of \,$\C^{2 \times 2}$\, 
and equipping it with an inner product such that the resulting inner product space
is isomorphic to the Minkowski space \,$\Re^{1,3}$.\,

\vskip 0.2cm
\noindent
\begin{proposition}
\mbox{}
\vskip 0.1cm
\noindent
Let \,$\SkHermTwoC \subset \C^{2 \times 2}$\, be the set of all $2 \times 2$ skew-Hermitian ($X^{\dagger} = -X$) complex matrices:
\begin{equation*}
\SkHermTwoC
\;\; := \;\;
	\left\{\;
		\left.\left(\begin{array}{cc}
			\i\,(x_{0}+x_{1}) & x_{2}+\i\,x_{3}
			\\
			-\,x_{2}+\i\,x_{3} & \i\,(x_{0}-x_{1})
			\end{array}\right)
		\in
		\C^{2 \times 2}
		\;\;\right\vert\;\;
		x_{0}, x_{1}, x_{2}, x_{4} \in \Re
		\;\right\}
\end{equation*}
Define the bilinear map \,$\langle\,\cdot\,,\,\cdot\,\rangle_{\SkHermTwoC}$\, on \,$\SkHermTwoC$\, as follows:
\begin{eqnarray*}
\left\langle\,
	\overset{{\color{white}.}}{X}
	\, , \,
	Y
	\,\right\rangle_{\!\SkHermTwoC}
& := &
	\dfrac{1}{4}\cdot\left(\,
		\det(X + Y)
		\, \overset{{\color{white}1}}{-} \,
		\det(X - Y)
		\,\right)
\end{eqnarray*}
Then, the following statements are true:
\begin{enumerate}
\item
	$\SkHermTwoC$\, is an $\Re$-linear subspace of \,$\C^{2 \times 2}$\, with \,$\dim_{\Re}(\SkHermTwoC) = 4$.\,
\item
	$\left(\,\overset{{\color{white}.}}{\SkHermTwoC}\,,\,\langle\,\cdot\,,\,\cdot\,\rangle_{\SkHermTwoC}\,\right)$\,
	is isometric to the $4$-dimensional Minkowski spacetime \,$\Re^{1,4}$.\,
\end{enumerate}
\end{proposition}
\proof
\begin{enumerate}
\item
	%It is clear that \,$\SkHermTwoC$\, is an $\Re$-linear subspace of \,$\C^{2 \times 2}$\, with \,$\dim_{\Re}(\SkHermTwoC) = 4$.\,
	Trivial.
\item
	First, note that:
	\begin{eqnarray*}
	\det(X)
	& = &
		\det\!\left(\begin{array}{cc}
			\i\,(x_{0}+x_{1}) & x_{2} + \i\,x_{3}
			\\
			- \, x_{2} + \i\,x_{3} & \i\,(x_{0}-x_{1})
			\end{array}\right)
	\\
	& = &
		\i^{2}(x_{0}+x_{1})(x_{0}-x_{1}) \,-\, (x_{2}+\i\,x_{3})(-x_{2}+\i\,x_{3})
	\;\; = \;\;
		-\,(x_{0}^{2}-x_{1}^{2}) \,+\, (x_{2}+\i\,x_{3})(x_{2}-\i\,x_{3})
	\\
	& = &
		\overset{{\color{white}1}}{-\,x_{0}^{2} \,+\, x_{1}^{2} \,+\, x_{2}^{2} \,+\, x_{3}^{2}}
	\end{eqnarray*}
	Next, we compute:
	\begin{eqnarray*}
	\left\langle\,
		\overset{{\color{white}.}}{X}
		\, , \,
		Y
		\,\right\rangle_{\!\SkHermTwoC}
	& := &
		\dfrac{1}{4}\cdot\left(\,
			\det(X + Y)
			\, \overset{{\color{white}1}}{-} \,
			\det(X - Y)
			\,\right)
	\\
	& = &
		{\color{white}-} \, \dfrac{1}{4}\cdot
		\det\!\left(\begin{array}{cc}
			\i\,(x_{0}+y_{0}+x_{1}+y_{1}) & (x_{2}+y_{2}) + \i\,(x_{3}+y_{3})
			\\
			- \, (x_{2}+y_{2}) + \i\,(x_{3}+y_{3}) & \i\,(x_{0}+y_{0}-x_{1}-y_{1})
			\end{array}\right)
	\\
	&&
		{\color{black}-} \, \dfrac{1}{4}\cdot
		\det\!\left(\begin{array}{cc}
			\i\,(x_{0}-y_{0}+x_{1}-y_{1}) & (x_{2}-y_{2}) + \i\,(x_{3}-y_{3})
			\\
			- \, (x_{2}-y_{2}) + \i\,(x_{3}-y_{3}) & \i\,(x_{0}-y_{0}-x_{1}+y_{1})
			\end{array}\right)
	\\
	& = &
		{\color{white}-} \, \dfrac{1}{4}\cdot\left(\,
			\overset{{\color{white}1}}{-}\,
			(x_{0}+y_{0})^{2}
			\,+\, (x_{1}+y_{1})^{2}
			\,+\, (x_{2}+y_{2})^{2}
			\,+\, (x_{3}+y_{3})^{2}
			\,\right)
	\\
	& &
		{\color{black}-} \, \dfrac{1}{4}\cdot\left(\,
			\overset{{\color{white}1}}{-}\,
			(x_{0}-y_{0})^{2}
			\,+\, (x_{1}-y_{1})^{2}
			\,+\, (x_{2}-y_{2})^{2}
			\,+\, (x_{3}-y_{3})^{2}
			\,\right)
	\\
	& = &
		{\color{white}-} \, \dfrac{1}{4}\cdot\left(\,
			\overset{{\color{white}1}}{-}\,
			x_{0}^{2} - 2\,x_{0}y_{0} - y_{0}^{2}
			\;+\,
			\overset{3}{\underset{k\,=\,1}{\sum}}\,(x_{k}^{2} + 2\,x_{k}y_{k}+y_{k}^{2})
			\,\right)
	\\
	& &
		{\color{black}-} \, \dfrac{1}{4}\cdot\left(\,
			\overset{{\color{white}1}}{-}\,
			x_{0}^{2} + 2\,x_{0}y_{0} - y_{0}^{2}
			\;+\,
			\overset{3}{\underset{k\,=\,1}{\sum}}\,(x_{k}^{2} - 2\,x_{k}y_{k}+y_{k}^{2})
			\,\right)
	\\
	& = &
		\overset{{\color{white}1}}{-\,x_{0}y_{0} \,+\, x_{1}\,y_{1} \,+\, x_{2}\,y_{2} \,+\, x_{3}\,y_{3}}
	\end{eqnarray*}
	This proves that
	\,$\left(\,\overset{{\color{white}.}}{\SkHermTwoC}\,,\,\langle\,\cdot\,,\,\cdot\,\rangle_{\SkHermTwoC}\,\right)$\,
	is indeed isometric to the $4$-dimensional\\ Minkowski spacetime \,$\Re^{1,4}$.\,
	\qed
\end{enumerate}

          %%%%% ~~~~~~~~~~~~~~~~~~~~ %%%%%

\vskip 0.5cm
\noindent
\begin{proposition}
\mbox{}
\vskip 0.1cm
\noindent
Let
\,$\GL_{\Re}(\C^{2 \times 2})$\,
denote the set of all invertible $\Re$-linear maps
from \,$\C^{2 \times 2}$\, onto itself.\\
Define the map
\,$\Phi : \SL(2,\C) \longrightarrow \GL_{\Re}(\C^{2 \times 2})$\,
by:
\begin{equation*}
\Phi_{A}(\,X\,)
\;\; := \;\;
	A \cdot X \cdot A^{\dagger}\,,
\quad
\textnormal{for each \,$A \in \SL(2,\C)$,\,  $X \in \textnormal{SkHerm}(2,\C) \cong \Re^{1,3}$}\,,
\end{equation*}
where \,$A^{\dagger}$\, denotes the conjugate transpose of
\,$A \in \SL(2,\C)$.\,
Then, the following statements are true:
\begin{enumerate}
\item
	Under the inner product space isomorphism
	\,$\textnormal{SkHerm}(2,\C) \cong \Re^{1,3}$,\,
	the image of the map
	\,$\Phi : \SL(2,\C) \longrightarrow \GL_{\Re}(\C^{2 \times 2})$\,
	can be regarded as a subset of
	\,$\SOup(1,3) \subset \textnormal{O}(1,3) = \Isom(\Re^{1,3}) \cong  \Isom(\textnormal{SkHerm}(2,\C)) \subset \GL_{\Re}(\C^{2 \times 2})$.\,
	Consequently, we may regard the codomain of \,$\Phi$\, to be \,$\SOup(1,3)$;\,
	in other words,
	\begin{equation*}
	\Phi : \SL(2,\C) \,\longrightarrow\, \SOup(\textnormal{SkHerm}(2,\C)) \cong \SOup(1,3)
	\end{equation*}
	given by:
	\begin{equation*}
	\Phi_{A}(\,X\,)
	\;\; := \;\;
		A \cdot X \cdot A^{\dagger}\,,
	\quad
	\textnormal{for each \,$A \in \SL(2,\C)$,\,  $X \in \textnormal{SkHerm}(2,\C) \cong \Re^{1,3}$}
	\end{equation*}
\item
	The map \,$\Phi : \SL(2,\C) \longrightarrow \SOup(1,3)$\, is a homomorphism.
\item
	$\ker(\Phi) \, = \, \{\,\pm \, I_{2}\,\}$
\item
	The map \,$\Phi : \SL(2,\C) \longrightarrow \SOup(1,3)$\, is surjective.
\end{enumerate}
\end{proposition}
\proof
\begin{enumerate}
\item
	First, we note that, for each \,$A \in \SL(2,\C)$,\, $X \in \textnormal{SkHerm}(2,\C)$,\,
	we indeed have that \,$\Phi_{A}(X) \in \textnormal{SkHerm}(2,\C)$\,:
	\begin{eqnarray*}
	\left(\,\Phi_{A}(\overset{{\color{white}.}}{X})\,\right)^{\dagger}
	\;\, = \;\;
		\left(\, A \cdot \overset{{\color{white}.}}{X} \cdot A^{\dagger} \,\right)^{\dagger}
	\;\, = \;\;
		A^{\dagger\dagger} \cdot X^{\dagger} \cdot A^{\dagger}
	\;\, = \;\;
		A \cdot (-X) \cdot A^{\dagger}
	\;\, = \;\;
		- \, \Phi_{A}(X)
	\end{eqnarray*}
	which shows that \,$\Phi_{A}(X)$\, is indeed skew-Hermitian, i.e.,
	\,$\Phi_{A}(X) \in \textnormal{SkHerm}(2,\C)$.\,
	Next, we show that, for each \,$A \in \SL(2,\C)$, the map
	\,$\Phi_{A} : \textnormal{SkHerm}(2,\C) \cong \Re^{1,3} \longrightarrow \textnormal{SkHerm}(2,\C) \cong \Re^{1,3}$\,
	preserves the inner product on \,$\textnormal{SkHerm}(2,\C)$.\,
	To this end, note:
	\begin{eqnarray*}
	\left\langle\;
		\Phi_{A}(\overset{{\color{white}.}}{X})
		\; , \;
		\Phi_{A}(Y)
		\;\right\rangle_{\!\textnormal{SkHerm}(2,\C)}
	& = &
		\dfrac{1}{4}\cdot\left(\,
			\det(AXA^{\dagger} + AYA^{\dagger})
			\, \overset{{\color{white}1}}{-} \,
			\det(AXA^{\dagger} - AYA^{\dagger})
			\,\right)
	\\
	& = &
		\dfrac{1}{4}\cdot\left(\,
			\det(A(X + Y)A^{\dagger})
			\, \overset{{\color{white}1}}{-} \,
			\det(A(X - Y)A^{\dagger})
			\,\right)
	\\
	& = &
		\dfrac{1}{4}\cdot\left(\,
			(\det A)\,\det(X + Y)\det(A^{\dagger})
			\, \overset{{\color{white}1}}{-} \,
			(\det A)\,\det(X - Y)\det(A^{\dagger})
			\,\right)
	\\
	& = &
		\dfrac{1}{4}\cdot\left(\,
			\det(X + Y)
			\, \overset{{\color{white}1}}{-} \,
			\det(X - Y)
			\,\right),
		\;\;
		\textnormal{since \,$\det A = \det A^{\dagger} = 1$}
	\\
	& =: &
		\left\langle\;
			\overset{{\color{white}.}}{X}
			\, , \,
			Y
			\;\right\rangle_{\!\textnormal{SkHerm}(2,\C)}
	\end{eqnarray*}
	Thus, under the inner product space isomorphism
	\,$\textnormal{SkHerm}(2,\C) \cong \Re^{1,3}$,\,
	we have:
	\,$\Phi_{A} \in \textnormal{O}(1,3)$,\,
	which furthermore immediately implies
	\,$\det\Phi_{A} \in \{\,\pm 1\,\}$.\,
	On the other hand, connectedness of \,$\SL(2,\C)$\, and continuity of
	\,$\det$\, and \,$\Phi$\, together imply that the image of the map
	\,$\det \,\circ\, \Phi : \SL(2,\C) \longrightarrow \Re$\,
	is a singleton subset of \,$\Re$.\,
	We may now conclude that
	 \,$\det\Phi_{A} = + 1$,\,
	for each \,$A \in \SL(2,\C)$.\,
	Thus,
	 \,$\Phi_{A} \in \SO(1,3)$,\,
	for each \,$A \in \SL(2,\C)$.\,
	It remains to show that \,$\Phi_{A}$\, preserves time orientation of vectors in \,$\textnormal{SkHerm}(2,\C) \cong \Re^{1,3}$.\,
	Now, note that
	\begin{equation*}
	X_{(1,0,0,0)}
	\;\; := \;\;
		\left(\begin{array}{cc} \i & 0 \\ 0 & \i \end{array}\right)
	\;\; = \;\;
		\left(\begin{array}{cc} \i\,(1 + 0) & 0+\i\,0 \\ -\,0+\i\,0 & \i\,(1-0) \end{array}\right)
	\;\; \in \;\;
		\textnormal{SkHerm}(2,\C)
	\end{equation*}
	corresponds to the timelike vector
	\,$(1,0,0,0)^{T} \in \Re^{1,3}$.\,
	So, it suffices to show that
	\,$\Phi_{A}(\,X_{(1,0,0,0)}\,)$\,
	remains timelike for each \,$A \in \SL(2,\C)$.\,
	We write
	\begin{equation*}
	A
	\; = \;
		\left(\begin{array}{cc} a & b \\ c & d \end{array}\right)
	\; \in \;
		\SL(2,\C)
	\quad\textnormal{and}\quad
	\Phi_{A}(\,X_{(1,0,0,0)}\,)
	\; = \;
		\left(\begin{array}{cc} \i(x_{0}+x_{1}) & x_{2}+\i\,x_{3} \\ -\,x_{2}+\i\,x_{3} & \i(x_{0}-x_{1}) \end{array}\right)
	\end{equation*}
	We thus need to show \,$x_{0} > 0$.\,
	Now,
	\begin{eqnarray*}
		\left(\begin{array}{cc} \i(x_{0}+x_{1}) & x_{2}+\i\,x_{3} \\ -\,x_{2}+\i\,x_{3} & \i(x_{0}-x_{1}) \end{array}\right)
	& = &
		\left(\begin{array}{cc} a & b \\ c & d \end{array}\right)
		\cdot
		\left(\begin{array}{cc} \i & 0 \\ 0 & \i \end{array}\right)
		\cdot
		\left(\begin{array}{cc} \overline{a} & \overline{c} \\ \overline{b} & \overline{d} \end{array}\right)
	\\
	& = &
		\i\,\cdot
		\left(\begin{array}{cc} a & b \\ c & d \end{array}\right)
		\cdot
		\left(\begin{array}{cc} \overline{a} & \overline{c} \\ \overline{b} & \overline{d} \end{array}\right)
	\\
	& = &
		\i\,\cdot
		\left(\begin{array}{cc}
			\vert a \vert^{2} + \vert b \vert^{2} & a\overline{c} + b\overline{d}
			\\
			\overline{a}c + \overline{b}d & \vert c \vert^{2} + \vert d \vert^{2}
			\end{array}\right),
	\end{eqnarray*}
	which implies
	\begin{equation*}
	\left\{\begin{array}{ccc}
		x_{0} + x_{1} & = & \vert a \vert^{2} + \vert b \vert^{2}
		\\
		x_{0} - x_{1} & \overset{{\color{white}1}}{=} & \vert c \vert^{2} + \vert d \vert^{2}
		\end{array}\right.
	\end{equation*}
	which in turn implies
	\begin{equation*}
	x_{0}
	\;\; = \;\;
		\dfrac{1}{2}\left(\,
			\vert a \vert^{2} \overset{{\color{white}1}}{+} \vert b \vert^{2} + \vert c \vert^{2} + \vert d \vert^{2}
			\,\right)
	\;\; > \;\;
		0\,,
	\end{equation*}
	as required. This proves that the image of
	\,$\Phi : \SL(2,\C) \longrightarrow \Isom(\textnormal{SkHerm}(2,\C)) \cong \Isom(\Re^{1,3}) = \textnormal{O}(1,3)$\,
	is indeed a subset of \,$\SOup(1,3)$.
\item
\item
\item
\end{enumerate}
\qed


          %%%%% ~~~~~~~~~~~~~~~~~~~~ %%%%%
