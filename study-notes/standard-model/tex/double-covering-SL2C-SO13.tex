
          %%%%% ~~~~~~~~~~~~~~~~~~~~ %%%%%

\chapter{The double cover $\SL(2,\C) \longrightarrow \SOup(1,3)$}
\setcounter{theorem}{0}
\setcounter{equation}{0}

%\cite{vanDerVaart1996}
%\cite{Kosorok2008}

%\renewcommand{\theenumi}{\alph{enumi}}
%\renewcommand{\labelenumi}{\textnormal{(\theenumi)}$\;\;$}
\renewcommand{\theenumi}{\roman{enumi}}
\renewcommand{\labelenumi}{\textnormal{(\theenumi)}$\;\;$}

          %%%%% ~~~~~~~~~~~~~~~~~~~~ %%%%%

\section{The Lie groups \,$\SOup(1,3)$\, and \,$\SL(2,\C)$}

          %%%%% ~~~~~~~~~~~~~~~~~~~~ %%%%%

%\vskip 0.3cm
%\begin{definition}
%\mbox{}
%\vskip 0.1cm
%\noindent
%The \textbf{special linear group of degree $n$ over $\C$} is defined as follows:
%\begin{equation*}
%\SL(n,\C)
%\; := \;
%	\left\{\;\,
%		g \overset{{\color{white}.}}{\in} \textnormal{GL}(n,\C)
%		\;\,\left\vert\;\;
%			\det(\,g\,) \overset{{\color{white}1}}{=} 1
%			\right.
%		\;\right\}
%\end{equation*}
%\end{definition}

Let \,$Q_{(1,n)} \,:=\, \diag(-1,1,\cdots,1) \in \Re^{(n+1) \times (n+1)}$.

\vskip 0.5cm
\begin{definition}[$\textnormal{O}(1,n)$]
\mbox{}
\vskip 0.1cm
\noindent
The \textbf{Lorentz group of \,$\textnormal{O}(1,n)$}\, is defined to be:
\begin{equation*}
\textnormal{O}(1,n)
\; := \;
	\left\{\;\,
		A \overset{{\color{white}.}}{\in} \textnormal{GL}(n+1,\Re)
		\;\left\vert\;\,
			A^{T} \cdot Q_{(1,n)} \cdot A = Q_{(1,n)}
			\right.
		\;\right\}
\end{equation*}
\end{definition}

\vskip 0.5cm
\begin{proposition}
\mbox{}
\vskip 0.1cm
\noindent
For each \,$A \in \textnormal{O}(1,n)$, we have:
\begin{enumerate}
\item
	$\det(A) = \pm 1$,\, and
\item
	$\vert\,A^{0}_{0}\,\vert^{2} \;\geq\; 1$;\, equivalently, either \,$A^{0}_{0} \,\geq\, 1$\, or \,$A^{0}_{0} \,\leq\, -1$
\end{enumerate}
\end{proposition}
\proof
\begin{enumerate}
\item
	\begin{eqnarray*}
	&&
		A^{T} \cdot Q_{(1,n)} \cdot A \,=\, Q_{(1,n)}
	\\
	& \Longrightarrow &
		\det(Q_{(1,n)})
		\, = \,
		\det\!\left(\,A^{T} \overset{{\color{white}1}}{\cdot} Q_{(1,n)} \cdot A\,\right)
		\,=\,
		\det(\,A^{T}\,) \cdot \det(Q_{(1,n)}) \cdot \det(A)
	\\
	& \Longrightarrow &
		-1
		\, = \,
		\det(Q_{(1,n)})
		\, = \,
		\det(\,A^{T}\,) \cdot \det(Q_{(1,n)}) \cdot \det(A)
		\, = \,
		\det(\,A^{T}\,) \cdot (\,-1\,)\cdot \det(A)
		\, = \,
		\det(A)^{2}
	\\
	& \Longrightarrow &
		\det(A)
		\, \overset{{\color{white}1}}{=} \,
		\pm 1
	\end{eqnarray*}
\item
	Note
	\begin{equation*}
	A^{T} \cdot Q_{(1,n)} \cdot A \,=\, Q_{(1,n)}
	\;\;\;\Longleftrightarrow\;\;\;
	\overset{3}{\underset{\rho\,=\,0}{\sum}}\;
		\overset{3}{\underset{\mu\,=\,0}{\sum}}\;
		A^{T}_{{\color{red}\mu\rho}}\,\eta_{\rho\sigma}\,A_{\sigma\nu}
		\, = \,
		\eta_{\mu\nu}
	\;\;\;\Longleftrightarrow\;\;\;
	\overset{3}{\underset{\rho\,=\,0}{\sum}}\;
		\overset{3}{\underset{\mu\,=\,0}{\sum}}\;
		A_{{\color{red}\rho\mu}}\,\eta_{\rho\sigma}\,A_{\sigma\nu}
		\, = \,
		\eta_{\mu\nu}\,,
	\end{equation*}
	which -- upon setting $\mu = \nu = 0$ -- implies:
	\begin{eqnarray*}
	&&
		A_{0{\color{red}0}} \cdot \eta_{00} \cdot A_{0{\color{red}0}}
		\,+\,
		A_{1{\color{red}0}} \cdot \eta_{11} \cdot A_{1{\color{red}0}}
		\,+\,
		A_{2{\color{red}0}} \cdot \eta_{22} \cdot A_{2{\color{red}0}}
		\,+\,
		A_{3{\color{red}0}} \cdot \eta_{33} \cdot A_{3{\color{red}0}}
		\; = \;
		\eta_{{\color{red}00}}
	\\
	& \Longrightarrow &
		A_{0{\color{red}0}}\,(-1)\,A_{0{\color{red}0}}
		\,+\,
		A_{1{\color{red}0}}\,(+1)\,A_{1{\color{red}0}}
		\,+\,
		A_{2{\color{red}0}}\,(+1)\,A_{2{\color{red}0}}
		\,+\,
		A_{3{\color{red}0}}\,(+1)\,A_{3{\color{red}0}}
		\; = \;
		-\,1
	\\
	& \Longrightarrow &
		- \, A_{0{\color{red}0}}^{2}
		\; + \,
		\overset{3}{\underset{k\,=\,0}{\sum}}\;
		A_{k{\color{red}0}}^{2}
		\;\; = \;\,
			-\,1
	\;\;\; \Longrightarrow \;\;\;\,
		A_{{\color{red}0}0}^{2}
		\;\; = \;\,
			1
			\; + \,
			\overset{3}{\underset{k\,=\,0}{\sum}}\;
			A_{k{\color{red}0}}^{2}
		\;\; \geq \;\,
			1\,,
	\end{eqnarray*}
	as required.
	\qed
\end{enumerate}

          %%%%% ~~~~~~~~~~~~~~~~~~~~ %%%%%

\vskip 0.5cm
\begin{corollary}
\mbox{}
\vskip 0.1cm
\noindent
The \textbf{Lorentz group of \,$\textnormal{O}(1,n)$}\, is the disjoint union of the following four connected components:
\begin{eqnarray*}
&&
{\color{white}\bigsqcup}\;
\left\{\;
	A \overset{{\color{white}.}}{\in} \textnormal{O}(1,n)
	\;\left\vert\;
		\begin{array}{ccc}
		\textnormal{det}(A) \,\overset{{\color{white}1}}{=}\, +1
		\\
		A^{0}_{0} \,\overset{{\color{white}1}}{\geq}\, 1
		\end{array}
		\right.
	\!\right\}
\;\bigsqcup\;
\left\{\;
	A \overset{{\color{white}.}}{\in} \textnormal{O}(1,n)
	\;\left\vert\;
		\begin{array}{ccc}
		\textnormal{det}(A) \,\overset{{\color{white}1}}{=}\, +1
		\\
		A^{0}_{0} \,\overset{{\color{white}1}}{\leq}\, -1
		\end{array}
		\right.
	\!\right\}
\\
&&
\bigsqcup\;
\left\{\;
	A \overset{{\color{white}.}}{\in} \textnormal{O}(1,n)
	\;\left\vert\;
		\begin{array}{ccc}
		\textnormal{det}(A) \,\overset{{\color{white}1}}{=}\, -1
		\\
		A^{0}_{0} \,\overset{{\color{white}1}}{\geq}\, 1
		\end{array}
		\right.
	\!\right\}
\;\bigsqcup\;
\left\{\;
	A \overset{{\color{white}.}}{\in} \textnormal{O}(1,n)
	\;\left\vert\;
		\begin{array}{ccc}
		\textnormal{det}(A) \,\overset{{\color{white}1}}{=}\, -1
		\\
		A^{0}_{0} \,\overset{{\color{white}1}}{\leq}\, -1
		\end{array}
		\right.
	\!\right\}
\end{eqnarray*}
\end{corollary}

          %%%%% ~~~~~~~~~~~~~~~~~~~~ %%%%%

\vskip 0.5cm
\begin{definition}[$\SO(1,n)$\, and \,$\SO^{\uparrow}(1,n)$]
\mbox{}
\vskip 0.1cm
\noindent
The \textbf{special Lorentz group} is defined to be:
\begin{eqnarray*}
\SO(1,n)
& := &
	\left\{\;
		A \overset{{\color{white}.}}{\in} \GL(n+1,\Re)
		\;\left\vert\,
			\begin{array}{ccc}
			A^{T} \cdot Q_{(1,n)} \cdot A \,=\, Q_{(1,n)}
			\\
			\textnormal{det}(A) \,\overset{{\color{white}1}}{=}\, 1
			\end{array}
			\right.
		\!\right\}
\\
& = &
	\left\{\;
		A \overset{{\color{white}.}}{\in} \textnormal{O}(1,n)
		\;\left\vert\;
			\begin{array}{ccc}
			\textnormal{det}(A) \,\overset{{\color{white}1}}{=}\, +1
			\\
			A^{0}_{0} \,\overset{{\color{white}1}}{\geq}\, 1
			\end{array}
			\right.
		\!\right\}
	\;\bigsqcup\;
	\left\{\;
		A \overset{{\color{white}.}}{\in} \textnormal{O}(1,n)
		\;\left\vert\;
			\begin{array}{ccc}
			\textnormal{det}(A) \,\overset{{\color{white}1}}{=}\, +1
			\\
			A^{0}_{0} \,\overset{{\color{white}1}}{\leq}\, -1
			\end{array}
			\right.
		\!\right\}
\end{eqnarray*}
The \textbf{orthochronous special Lorentz group} is defined to be:
\begin{equation*}
\SO^{\uparrow}(1,n)
\; := \;
	\left\{\;
		A \overset{{\color{white}.}}{\in} \GL(n+1,\Re)
		\;\left\vert\,
			\begin{array}{ccc}
			A^{T} \cdot Q_{(1,n)} \cdot A \,=\, Q_{(1,n)}
			\\
			\textnormal{det}(A) \,\overset{{\color{white}1}}{=}\, 1
			\\
			A^{0}_{0} \,\overset{{\color{white}1}}{\geq}\, 1
			\end{array}
			\right.
		\!\right\}
\; = \;
	\left\{\;
		A \overset{{\color{white}.}}{\in} \textnormal{O}(1,n)
		\;\left\vert\;
			\begin{array}{ccc}
			\textnormal{det}(A) \,\overset{{\color{white}1}}{=}\, +1
			\\
			A^{0}_{0} \,\overset{{\color{white}1}}{\geq}\, 1
			\end{array}
			\right.
		\!\right\}
\end{equation*}
\end{definition}

          %%%%% ~~~~~~~~~~~~~~~~~~~~ %%%%%


\vskip 1.0cm
\noindent
\textbf{\large The \,$\SL(2,\C)$\, action on Minkowski space \,$\Re^{1,3}$}
\vskip 0.3cm
\noindent
We start by identifying an $\Re$-linear subspace \,$V$\, of \,$\C^{2 \times 2}$\, 
and equipping it with an inner product such that the resulting inner product space
is isomorphic to the Minkowski space \,$\Re^{1,3}$.\,

\vskip 0.2cm
\noindent
\begin{proposition}
\mbox{}
\vskip 0.1cm
\noindent
Let \,$\SkHermTwoC \subset \C^{2 \times 2}$\, be the set of all $2 \times 2$ skew-Hermitian ($X^{\dagger} = -X$) complex matrices:
\begin{equation*}
\SkHermTwoC
\;\; := \;\;
	\left\{\;
		\left.\left(\begin{array}{cc}
			\i\,(x_{0}+x_{1}) & x_{2}+\i\,x_{3}
			\\
			-\,x_{2}+\i\,x_{3} & \i\,(x_{0}-x_{1})
			\end{array}\right)
		\in
		\C^{2 \times 2}
		\;\;\right\vert\;\;
		x_{0}, x_{1}, x_{2}, x_{4} \in \Re
		\;\right\}
\end{equation*}
Define the bilinear map \,$\langle\,\cdot\,,\,\cdot\,\rangle_{\SkHermTwoC}$\, on \,$\SkHermTwoC$\, as follows:
\begin{eqnarray*}
\left\langle\,
	\overset{{\color{white}.}}{X}
	\, , \,
	Y
	\,\right\rangle_{\!\SkHermTwoC}
& := &
	\dfrac{1}{4}\cdot\left(\,
		\det(X + Y)
		\, \overset{{\color{white}1}}{-} \,
		\det(X - Y)
		\,\right)
\end{eqnarray*}
Then, the following statements are true:
\begin{enumerate}
\item
	$\SkHermTwoC$\, is an $\Re$-linear subspace of \,$\C^{2 \times 2}$\, with \,$\dim_{\Re}(\SkHermTwoC) = 4$.\,
\item
	$\left(\,\overset{{\color{white}.}}{\SkHermTwoC}\,,\,\langle\,\cdot\,,\,\cdot\,\rangle_{\SkHermTwoC}\,\right)$\,
	is isometric to the $4$-dimensional Minkowski spacetime \,$\Re^{1,4}$.\,
\end{enumerate}
\end{proposition}
\proof
\begin{enumerate}
\item
	%It is clear that \,$\SkHermTwoC$\, is an $\Re$-linear subspace of \,$\C^{2 \times 2}$\, with \,$\dim_{\Re}(\SkHermTwoC) = 4$.\,
	Trivial.
\item
	First, note that:
	\begin{eqnarray*}
	\det(X)
	& = &
		\det\!\left(\begin{array}{cc}
			\i\,(x_{0}+x_{1}) & x_{2} + \i\,x_{3}
			\\
			- \, x_{2} + \i\,x_{3} & \i\,(x_{0}-x_{1})
			\end{array}\right)
	\\
	& = &
		\i^{2}(x_{0}+x_{1})(x_{0}-x_{1}) \,-\, (x_{2}+\i\,x_{3})(-x_{2}+\i\,x_{3})
	\;\; = \;\;
		-\,(x_{0}^{2}-x_{1}^{2}) \,+\, (x_{2}+\i\,x_{3})(x_{2}-\i\,x_{3})
	\\
	& = &
		\overset{{\color{white}1}}{-\,x_{0}^{2} \,+\, x_{1}^{2} \,+\, x_{2}^{2} \,+\, x_{3}^{2}}
	\end{eqnarray*}
	Next, we compute:
	\begin{eqnarray*}
	\left\langle\,
		\overset{{\color{white}.}}{X}
		\, , \,
		Y
		\,\right\rangle_{\!\SkHermTwoC}
	& := &
		\dfrac{1}{4}\cdot\left(\,
			\det(X + Y)
			\, \overset{{\color{white}1}}{-} \,
			\det(X - Y)
			\,\right)
	\\
	& = &
		{\color{white}-} \, \dfrac{1}{4}\cdot
		\det\!\left(\begin{array}{cc}
			\i\,(x_{0}+y_{0}+x_{1}+y_{1}) & (x_{2}+y_{2}) + \i\,(x_{3}+y_{3})
			\\
			- \, (x_{2}+y_{2}) + \i\,(x_{3}+y_{3}) & \i\,(x_{0}+y_{0}-x_{1}-y_{1})
			\end{array}\right)
	\\
	&&
		{\color{black}-} \, \dfrac{1}{4}\cdot
		\det\!\left(\begin{array}{cc}
			\i\,(x_{0}-y_{0}+x_{1}-y_{1}) & (x_{2}-y_{2}) + \i\,(x_{3}-y_{3})
			\\
			- \, (x_{2}-y_{2}) + \i\,(x_{3}-y_{3}) & \i\,(x_{0}-y_{0}-x_{1}+y_{1})
			\end{array}\right)
	\\
	& = &
		{\color{white}-} \, \dfrac{1}{4}\cdot\left(\,
			\overset{{\color{white}1}}{-}\,
			(x_{0}+y_{0})^{2}
			\,+\, (x_{1}+y_{1})^{2}
			\,+\, (x_{2}+y_{2})^{2}
			\,+\, (x_{3}+y_{3})^{2}
			\,\right)
	\\
	& &
		{\color{black}-} \, \dfrac{1}{4}\cdot\left(\,
			\overset{{\color{white}1}}{-}\,
			(x_{0}-y_{0})^{2}
			\,+\, (x_{1}-y_{1})^{2}
			\,+\, (x_{2}-y_{2})^{2}
			\,+\, (x_{3}-y_{3})^{2}
			\,\right)
	\\
	& = &
		{\color{white}-} \, \dfrac{1}{4}\cdot\left(\,
			\overset{{\color{white}1}}{-}\,
			x_{0}^{2} - 2\,x_{0}y_{0} - y_{0}^{2}
			\;+\,
			\overset{3}{\underset{k\,=\,1}{\sum}}\,(x_{k}^{2} + 2\,x_{k}y_{k}+y_{k}^{2})
			\,\right)
	\\
	& &
		{\color{black}-} \, \dfrac{1}{4}\cdot\left(\,
			\overset{{\color{white}1}}{-}\,
			x_{0}^{2} + 2\,x_{0}y_{0} - y_{0}^{2}
			\;+\,
			\overset{3}{\underset{k\,=\,1}{\sum}}\,(x_{k}^{2} - 2\,x_{k}y_{k}+y_{k}^{2})
			\,\right)
	\\
	& = &
		\overset{{\color{white}1}}{-\,x_{0}y_{0} \,+\, x_{1}\,y_{1} \,+\, x_{2}\,y_{2} \,+\, x_{3}\,y_{3}}
	\end{eqnarray*}
	This proves that
	\,$\left(\,\overset{{\color{white}.}}{\SkHermTwoC}\,,\,\langle\,\cdot\,,\,\cdot\,\rangle_{\SkHermTwoC}\,\right)$\,
	is indeed isometric to the $4$-dimensional\\ Minkowski spacetime \,$\Re^{1,4}$.\,
	\qed
\end{enumerate}

          %%%%% ~~~~~~~~~~~~~~~~~~~~ %%%%%

\vskip 0.5cm
\noindent
\begin{theorem}
\mbox{}
\vskip 0.1cm
\noindent
Let
\,$\GL_{\Re}(\C^{2 \times 2})$\,
denote the set of all invertible $\Re$-linear maps
from \,$\C^{2 \times 2}$\, onto itself.\\
Define the map
\,$\Phi : \SL(2,\C) \longrightarrow \GL_{\Re}(\C^{2 \times 2})$\,
by:
\begin{equation*}
\Phi_{A}(\,X\,)
\;\; := \;\;
	A \cdot X \cdot A^{\dagger}\,,
\quad
\textnormal{for each \,$A \in \SL(2,\C)$,\,  $X \in \textnormal{SkHerm}(2,\C) \cong \Re^{1,3}$}\,,
\end{equation*}
where \,$A^{\dagger}$\, denotes the conjugate transpose of
\,$A \in \SL(2,\C)$.\,
Then, the following statements are true:
\begin{enumerate}
\item
	Under the inner product space isomorphism
	\,$\textnormal{SkHerm}(2,\C) \cong \Re^{1,3}$,\,
	the image of the map
	\,$\Phi : \SL(2,\C) \longrightarrow \GL_{\Re}(\C^{2 \times 2})$\,
	can be regarded as a subset of
	\,$\SOup(1,3) \subset \textnormal{O}(1,3) = \Isom(\Re^{1,3}) \cong  \Isom(\textnormal{SkHerm}(2,\C)) \subset \GL_{\Re}(\C^{2 \times 2})$.\,
	Consequently, we may regard the codomain of \,$\Phi$\, to be \,$\SOup(1,3)$;\,
	in other words,
	\begin{equation*}
	\Phi : \SL(2,\C) \,\longrightarrow\, \SOup(\textnormal{SkHerm}(2,\C)) \cong \SOup(1,3)
	\end{equation*}
	given by:
	\begin{equation*}
	\Phi_{A}(\,X\,)
	\;\; := \;\;
		A \cdot X \cdot A^{\dagger}\,,
	\quad
	\textnormal{for each \,$A \in \SL(2,\C)$,\,  $X \in \textnormal{SkHerm}(2,\C) \cong \Re^{1,3}$}
	\end{equation*}
\item
	The map \,$\Phi : \SL(2,\C) \longrightarrow \SOup(1,3)$\, is a homomorphism.
\item
	$\ker(\Phi) \, = \, \{\,\pm \, I_{2}\,\}$
\item
	The map \,$\Phi : \SL(2,\C) \longrightarrow \SOup(1,3)$\, is surjective.
\end{enumerate}
\end{theorem}
\proof
\begin{enumerate}
\item
	First, we note that, for each \,$A \in \SL(2,\C)$,\, $X \in \textnormal{SkHerm}(2,\C)$,\,
	we indeed have that \,$\Phi_{A}(X) \in \textnormal{SkHerm}(2,\C)$\,:
	\begin{eqnarray*}
	\left(\,\Phi_{A}(\overset{{\color{white}.}}{X})\,\right)^{\dagger}
	\;\, = \;\;
		\left(\, A \cdot \overset{{\color{white}.}}{X} \cdot A^{\dagger} \,\right)^{\dagger}
	\;\, = \;\;
		A^{\dagger\dagger} \cdot X^{\dagger} \cdot A^{\dagger}
	\;\, = \;\;
		A \cdot (-X) \cdot A^{\dagger}
	\;\, = \;\;
		- \, \Phi_{A}(X)
	\end{eqnarray*}
	which shows that \,$\Phi_{A}(X)$\, is indeed skew-Hermitian, i.e.,
	\,$\Phi_{A}(X) \in \textnormal{SkHerm}(2,\C)$.\,
	Next, we show that, for each \,$A \in \SL(2,\C)$, the map
	\,$\Phi_{A} : \textnormal{SkHerm}(2,\C) \cong \Re^{1,3} \longrightarrow \textnormal{SkHerm}(2,\C) \cong \Re^{1,3}$\,
	preserves the inner product on \,$\textnormal{SkHerm}(2,\C)$.\,
	To this end, note:
	\begin{eqnarray*}
	\left\langle\;
		\Phi_{A}(\overset{{\color{white}.}}{X})
		\; , \;
		\Phi_{A}(Y)
		\;\right\rangle_{\!\textnormal{SkHerm}(2,\C)}
	& = &
		\dfrac{1}{4}\cdot\left(\,
			\det(AXA^{\dagger} + AYA^{\dagger})
			\, \overset{{\color{white}1}}{-} \,
			\det(AXA^{\dagger} - AYA^{\dagger})
			\,\right)
	\\
	& = &
		\dfrac{1}{4}\cdot\left(\,
			\det(A(X + Y)A^{\dagger})
			\, \overset{{\color{white}1}}{-} \,
			\det(A(X - Y)A^{\dagger})
			\,\right)
	\\
	& = &
		\dfrac{1}{4}\cdot\left(\,
			(\det A)\,\det(X + Y)\det(A^{\dagger})
			\, \overset{{\color{white}1}}{-} \,
			(\det A)\,\det(X - Y)\det(A^{\dagger})
			\,\right)
	\\
	& = &
		\dfrac{1}{4}\cdot\left(\,
			\det(X + Y)
			\, \overset{{\color{white}1}}{-} \,
			\det(X - Y)
			\,\right),
		\;\;
		\textnormal{since \,$\det A = \det A^{\dagger} = 1$}
	\\
	& =: &
		\left\langle\;
			\overset{{\color{white}.}}{X}
			\, , \,
			Y
			\;\right\rangle_{\!\textnormal{SkHerm}(2,\C)}
	\end{eqnarray*}
	Thus, under the inner product space isomorphism
	\,$\textnormal{SkHerm}(2,\C) \cong \Re^{1,3}$,\,
	we have:
	\,$\Phi_{A} \in \textnormal{O}(1,3)$,\,
	which furthermore immediately implies
	\,$\det\Phi_{A} \in \{\,\pm 1\,\}$.\,
	On the other hand, connectedness of \,$\SL(2,\C)$\, and continuity of
	\,$\det$\, and \,$\Phi$\, together imply that the image of the map
	\,$\det \,\circ\, \Phi : \SL(2,\C) \longrightarrow \Re$\,
	is a singleton subset of \,$\Re$.\,
	We may now conclude that
	 \,$\det\Phi_{A} = + 1$,\,
	for each \,$A \in \SL(2,\C)$.\,
	Thus,
	 \,$\Phi_{A} \in \SO(1,3)$,\,
	for each \,$A \in \SL(2,\C)$.\,
	It remains to show that \,$\Phi_{A}$\, preserves time orientation of vectors in \,$\textnormal{SkHerm}(2,\C) \cong \Re^{1,3}$.\,
	Now, note that
	\begin{equation*}
	X_{(1,0,0,0)}
	\;\; := \;\;
		\left(\begin{array}{cc} \i & 0 \\ 0 & \i \end{array}\right)
	\;\; = \;\;
		\left(\begin{array}{cc} \i\,(1 + 0) & 0+\i\,0 \\ -\,0+\i\,0 & \i\,(1-0) \end{array}\right)
	\;\; \in \;\;
		\textnormal{SkHerm}(2,\C)
	\end{equation*}
	corresponds to the timelike vector
	\,$(1,0,0,0)^{T} \in \Re^{1,3}$.\,
	So, it suffices to show that
	\,$\Phi_{A}(\,X_{(1,0,0,0)}\,)$\,
	remains timelike for each \,$A \in \SL(2,\C)$.\,
	We write
	\begin{equation*}
	A
	\; = \;
		\left(\begin{array}{cc} a & b \\ c & d \end{array}\right)
	\; \in \;
		\SL(2,\C)
	\quad\textnormal{and}\quad
	\Phi_{A}(\,X_{(1,0,0,0)}\,)
	\; = \;
		\left(\begin{array}{cc} \i(x_{0}+x_{1}) & x_{2}+\i\,x_{3} \\ -\,x_{2}+\i\,x_{3} & \i(x_{0}-x_{1}) \end{array}\right)
	\end{equation*}
	We thus need to show \,$x_{0} > 0$.\,
	Now,
	\begin{eqnarray*}
		\left(\begin{array}{cc} \i(x_{0}+x_{1}) & x_{2}+\i\,x_{3} \\ -\,x_{2}+\i\,x_{3} & \i(x_{0}-x_{1}) \end{array}\right)
	& = &
		\left(\begin{array}{cc} a & b \\ c & d \end{array}\right)
		\cdot
		\left(\begin{array}{cc} \i & 0 \\ 0 & \i \end{array}\right)
		\cdot
		\left(\begin{array}{cc} \overline{a} & \overline{c} \\ \overline{b} & \overline{d} \end{array}\right)
	\\
	& = &
		\i\,\cdot
		\left(\begin{array}{cc} a & b \\ c & d \end{array}\right)
		\cdot
		\left(\begin{array}{cc} \overline{a} & \overline{c} \\ \overline{b} & \overline{d} \end{array}\right)
	\\
	& = &
		\i\,\cdot
		\left(\begin{array}{cc}
			\vert a \vert^{2} + \vert b \vert^{2} & a\overline{c} + b\overline{d}
			\\
			\overline{a}c + \overline{b}d & \vert c \vert^{2} + \vert d \vert^{2}
			\end{array}\right),
	\end{eqnarray*}
	which implies
	\begin{equation*}
	\left\{\begin{array}{ccc}
		x_{0} + x_{1} & = & \vert a \vert^{2} + \vert b \vert^{2}
		\\
		x_{0} - x_{1} & \overset{{\color{white}1}}{=} & \vert c \vert^{2} + \vert d \vert^{2}
		\end{array}\right.
	\end{equation*}
	which in turn implies
	\begin{equation*}
	x_{0}
	\;\; = \;\;
		\dfrac{1}{2}\left(\,
			\vert a \vert^{2} \overset{{\color{white}1}}{+} \vert b \vert^{2} + \vert c \vert^{2} + \vert d \vert^{2}
			\,\right)
	\;\; > \;\;
		0\,,
	\end{equation*}
	as required. This proves that the image of
	\,$\Phi : \SL(2,\C) \longrightarrow \Isom(\textnormal{SkHerm}(2,\C)) \cong \Isom(\Re^{1,3}) = \textnormal{O}(1,3)$\,
	is indeed a subset of \,$\SOup(1,3)$.
\item
	Simply observe:
	\begin{eqnarray*}
	\Phi_{A_{1}A_{2}}(X)
	& = &
		(A_{1}A_{2}) \cdot X (A_{1}A_{2})^{\dagger}
	\;\; = \;\;
		A_{1} \cdot A_{2} \cdot X \cdot A_{2}^{\dagger} \cdot A_{1}^{\dagger}
	\;\; = \;\;
		\Phi_{A_{1}}\!\left(\,\Phi_{A_{2}}(\overset{{\color{white}.}}{X})\,\right)
	\\
	& = &
		\Phi_{A_{1}} \,\circ\, \Phi_{A_{2}}\left(\,\overset{{\color{white}.}}{X}\,\right),
	\end{eqnarray*}
	which shows
	\,$\Phi(\,A_{1} \cdot A_{2}\,) \, = \, \Phi(A_{1}) \,\circ\, \Phi(A_{2})$.\,
	Thus, \,$\Phi : \SL(2,\C) \longrightarrow \SOup(1,3)$\, is indeed a homomorphism.
\item
	Obviously, \,$\{\,\pm\,I_{2}\,\} \subset \ker(\Phi)$.\,
	For the reserve inclusion, let
	\,$A \in \ker(\Phi) \subset \SL(2,\C)$,\,
	i.e.,
	\,$AXA^{\dagger} = X$,\, for each \,$X \in \textnormal{SkHerm}(2,\C) \cong \Re^{1,3}$.\,
	For
	\begin{equation*}
	X_{(1,0,0,0)}
	\;\; := \;\;
		\left(\begin{array}{cc} \i & 0 \\ 0 & \i \end{array}\right)
	\;\; = \;\;
		\left(\begin{array}{cc} \i\,(1 + 0) & 0+\i\,0 \\ -\,0+\i\,0 & \i\,(1-0) \end{array}\right)
	\;\; \in \;\;
		\textnormal{SkHerm}(2,\C)
	\end{equation*}
	we have:
	\begin{eqnarray*}
	\i \cdot I_{2}
	& = &
		\left(\begin{array}{cc} \i & 0 \\ 0 & \i \end{array}\right)
	\;\; = \;\;
		X_{(1,0,0,0)}
	\;\; = \;\;
		A \cdot X_{(1,0,0,0)} \cdot A^{\dagger}
	\;\; = \;\;
		A
		\cdot
		\left(\begin{array}{cc}
			\i & 0
			\\
			0 & \i
			\end{array}\right)
		\cdot
		A^{\dagger}
	\;\; = \;\;
		\i \cdot A \cdot A^{\dagger}
	\end{eqnarray*}
	which implies
	\,$A \cdot A^{\dagger} \,=\, I_{2}$,\,
	i.e.,
	\,$A \in \SU(2)$.\,
	In particular, \,$A$\, is a normal matrix (i.e., $A$ satisfies $A^{\dagger}A = AA^{\dagger}$).
	
	\vskip 0.3cm
	\noindent
	\textbf{Claim 1:}\quad
	There exist
	\,$\theta \in \Re$\,
	and
	\,$U \in \SU(2)$\,
	such that
	\begin{equation*}
	U \cdot A \cdot U^{-1}
	\; = \;
		U \cdot A \cdot U^{\dagger}
	\; = \;
		\diag(e^{\i\theta/2},e^{-\i\theta/2})
	\; \in \;
		\SU(2)
	\end{equation*}
	\vskip 0.1cm
	\noindent
	Proof of Claim 1:\;\;
	By the {\color{red}Complex Spectral Theorem}\footnote{Complex Spectral Theorem:
	\,$M \in \C^{n \times n}$\,
	is normal (i.e., $MM^{\dagger} = M^{\dagger}M$)
	if and only if $M$ is unitarily diagonalizable.
	See, e.g., Theorem 7.31, p.246, \cite{Axler2024}.},
	there exist \,$W \in \textnormal{U}(2)$\, and diagonal
	\,$D = \diag(a,b) \in \C^{2 \times 2}$\,
	such that
	\,$W \cdot A \cdot W^{\dagger} = W \cdot A \cdot W^{-1} = D = \diag(a,b)$.\,
	First, observe that 
	\,$\det D = \det(WAW^{-1}) = (\det W)(\det A)(\det W)^{-1} = \det A = 1$;\,
	hence,
	\,$b = a^{-1}$\,
	and,
	\,$D = \diag(a,a^{-1})$.\,
	Secondly, note that
	\,$D$\, is unitary; indeed, the unitarity of \,$A$\, and \,$W$\, imply:
	\begin{equation*}
	D^{\dagger} \cdot D
	\;\; = \;\;
		(W A W^{\dagger})^{\dagger} \cdot (W A W^{\dagger})
	\;\; = \;\;
		(W^{\dagger\dagger} A^{\dagger} W^{\dagger}) \cdot (W A W^{\dagger})
	\;\; = \;\;
		W A^{\dagger} W^{\dagger} W A W^{\dagger}
	\;\; = \;\;
		I_{2}\,,
	\end{equation*}
	which implies
	\,$a\,\overline{a} = \vert\,a\,\vert^{2} = 1$.\,
	So, we may write
	\,$a = e^{\i\theta/2}$,\,
	for some
	\,$\theta \in \Re$,\,
	and
	\,$D = \diag(e^{\i\theta/2},e^{-\i\theta/2}) \in \SU(2)$.\,
	Lastly, we choose \,$U \in \SU(2)$,\, as follows:
	Since \,$W \in \textnormal{U}(2)$ (i.e., $W^{\dagger}W = I_{2}$),\,
	it follows that
	\begin{eqnarray*}
	1
	& = &
		\det(I_{2})
	\\
	& = &
		\det(W^{\dagger}W)
	\;\; = \;\;
		\det(W^{\dagger})\cdot\det(W)
	\;\; = \;\;
		\det(\,\overline{W}^{T}\,)\cdot\det(W)
	\;\; = \;\;
		\overline{\det(W)}\cdot\det(W)
	\\
	& = &
		= \vert\,\det(W)\,\vert^{2}
	\end{eqnarray*}
	Hence,
	\,$\vert\,\det(W)\,\vert = 1$\,
	and we may write
	\,$\det(W)  = e^{\i\phi}$.\,
	Let \,$U := e^{-\i\phi/2} \cdot W$.\,
	Note
	\begin{eqnarray*}
	\det(U)
	& = &
		\det(\,e^{-\i\phi/2} \cdot W\,)
	\;\; = \;\;
		(e^{-\i\phi/2})^{2} \cdot \det(W)
	\;\; = \;\;
		e^{-\i\phi} \cdot e^{+\i\phi}
	\;\; = \;\;
		1,
	\quad\textnormal{and}\quad
	\\
	U^{\dagger}U{\color{white}2}
	& = &
		(e^{-\i\phi/2} \cdot W)^{\dagger} \cdot (e^{-\i\phi/2} \cdot W)
	\; = \;
		e^{+\i\phi/2} \cdot e^{-\i\phi/2} \cdot W^{\dagger} \cdot W
	\; = \;
		I_{2}
	\end{eqnarray*}
	Thus, \,$U \in \SU(2)$.\,
	And,
	\begin{eqnarray*}
	U \cdot A \cdot U^{-1}
	& = &
		(e^{-\i\phi/2}\,W) \cdot A \cdot (e^{-\i\phi/2}\,W)^{-1}
	\;\; = \;\;
		e^{-\i\phi/2} \cdot W \cdot A \cdot W^{-1} \cdot e^{+\i\phi/2}
	\;\; = \;\;
		W \cdot A \cdot W^{-1}
	\\
	& = &
		D
	\;\; = \;\;
		\diag(e^{\i\theta/2},e^{-\i\theta/2}) 
	\;\; \in \;\;
		\SU(2)
	\end{eqnarray*}
	This completes the proof of Claim 1.

	\vskip 0.3cm
	\noindent
	\textbf{Claim 2:}\quad
	For diagonal
	\,$D = \diag(e^{\i\theta/2},e^{-\i\theta/2}) \in \SL(2,\C)$,\,
	we have:
	\begin{equation*}
	\Phi_{D}
	\; = \;
		R_{23}(\theta)
	\; \in \;
		\SOup(\textnormal{SkHerm}(2,\C))
	\; \cong \;
		\SOup(1,3)\,,
	\end{equation*}
	where
	\,$R_{23}(\theta)$\,
	is the counterclockwise rotation of the
	$x_{2}x_{3}$-plane in \,$\Re^{1,3}$.
	\vskip 0.1cm
	\noindent
	Proof of Claim 2:\;\;
	\begin{eqnarray*}
	\Phi_{D}(X)
	& = &
		DXD^{-1}
	\;\; = \;\;
		\left(\,\begin{array}{cc}
			e^{\i\,{\color{red}\theta/2}} & 0
			\\
			0 & e^{-\,\i\,\theta/2}
			\end{array}\!\!\right)
		\cdot
		\left(\,\begin{array}{cc}
			\i\,(x_{0}+x_{1}) & x_{2} + \i\,x_{3}
			\\
			- \, x_{2} + \i\,x_{3} & \i\,(x_{0}-x_{1})
			\end{array}\,\right)
		\cdot
		\left(\begin{array}{cc}
			e^{-\,\i\,\theta/2} & 0
			\\
			0 & e^{\i\,\theta/2}
			\end{array}\!\right)
	\\
	& = &
		\left(\,\begin{array}{cc}
			e^{\i\,\theta/2} \cdot \i\,(x_{0}+x_{1}) & e^{\i\,\theta/2} \cdot (x_{2} + \i\,x_{3})
			\\
			e^{-\,\i\,\theta/2} \cdot (-\,x_{2} + \i\,x_{3}) & \;\; e^{-\,\i\,\theta/2} \cdot \i\,(x_{0}-x_{1})
			\end{array}\!\right)
		\cdot
		\left(\begin{array}{cc}
			e^{-\,\i\,\theta/2} & 0
			\\
			0 & e^{\i\,\theta/2}
			\end{array}\!\right)
	\\
	& = &
		\left(\,\begin{array}{cc}
			\i\,(x_{0}+x_{1}) & \quad{\color{black}e^{\i\,\theta} \cdot (x_{2} + \i\,x_{3})}
			\\
			-\,e^{-\,\i\,\theta} \cdot (x_{2} - \i\,x_{3}) & \i\,(x_{0}-x_{1})
			\end{array}\!\!\right)
	\\
	& = &
		\left(\,\begin{array}{cc}
			\i\,(x_{0}+x_{1}) & \quad{\color{red}e^{\i\,\theta} \cdot (x_{2} + \i\,x_{3})}
			\\
			\overset{{\color{white}1}}{-\;\overline{e^{\i\,\theta} \cdot (x_{2} + \i\,x_{3})}} & \i\,(x_{0}-x_{1})
			\end{array}\!\!\right)
	\end{eqnarray*}
	This completes the proof of Claim 2.

	\vskip 0.3cm
	\noindent
	Now, by Claim 1, we have
	\,$A = U^{-1} \cdot D \cdot U$,\,
	where
	\,$U \in \SU(2)$\,
	and
	\,$D  = \diag(e^{\i\theta/2},e^{-\i\theta/2}) \in \SU(2)$.\,
	And, by Claim 2, we have
	\,$\Phi_{D} = R_{23}(\theta)$,\,
	the counterclockwise rotation of the
	$x_{2}x_{3}$-plane in \,$\Re^{1,3}$.\,
	Hence,
	\begin{eqnarray*}
	&&
		A \in \ker(\Phi)
	\\
	& \Longrightarrow &
		\mathbf{1}_{\textnormal{SkHerm}(2,\C)}
		\; = \;
			\Phi(A)
		\; = \;
			\Phi(U^{-1} \cdot D \cdot U)
		\; = \;
			\Phi(U)^{-1} \circ \Phi_{D} \circ \Phi(U)
	\\
	& \Longrightarrow &
		R_{23}(\theta)
		\; = \;
			\Phi_{D}
		\; = \;
			\Phi(U) \circ \mathbf{1}_{\textnormal{SkHerm}(2,\C)} \circ \Phi(U)^{-1}
		\; = \;
			\Phi(U) \circ \Phi(U)^{-1}
		\; = \;
			\mathbf{1}_{\textnormal{SkHerm}(2,\C)}
	\\
	& \Longrightarrow &
		\theta
		\; = \;
			2\,\pi\,k\,,
		\;\;\textnormal{for some \,$k \in \N$}
	\\
	& \Longrightarrow &
		D
		\; = \;
			\diag(e^{\i\,k\pi},e^{-\i\,k\pi})
		\; = \;
			\left\{\begin{array}{cl}
				+\,I_{2}, & \textnormal{for even \,$k \in \N$}
				\\
				-\,I_{2}, & \textnormal{for \,odd\, \,$k \in \N$}
				\end{array}\right.
	\\
	& \Longrightarrow &
		A
		\; = \;
			U^{-1} \cdot D \cdot U
		\; = \;
			U^{-1} \cdot (\pm\,I_{2}) \cdot U
		\; = \;
			\pm \, U^{-1} \cdot U
		\; = \;
			\pm\,I_{2}
	\end{eqnarray*}
	This establishes that the required reverse inclusion: \,$\ker(\Phi) \subset \{\,\pm\,I_{2}\,\}$.\,
	We may now conclude that \,$\ker(\Phi) = \{\,\pm\,I_{2}\,\}$,\, as required.
\item
	\noindent
	\textbf{Claim 3:}\quad
	$\d(\Phi) : \sl(2,\C) \longrightarrow \so(1,3)$\,
	is a Lie algebra isomorphism.
	\vskip -0.01cm
	\noindent
	Proof of Claim 3:\;\;
	See proof of Theorem 6.12(i), p.175, \cite{Baker2002}.\\
	This proves Claim 3.

	\vskip 0.2cm
	\noindent
	Since
	\begin{itemize}
	\item
		both \,$\SL(2,\C)$\, and \,$\SOup(1,3)$\, are connected Lie groups (Lie subgroups of general linear groups),
	\item
		$\d(\Phi) : \sl(2,\C) \longrightarrow \so(3)$\,
		is a Lie algebra isomorphism (by Claim 3),
	\end{itemize}
	it follows -- by Theorem 4.15, p.92, \cite{Sepanski2010} -- that
	\,$\Phi : \SL(2,\C) \longrightarrow \SOup(1,3)$\,
	is a covering map; in particular, it is surjective, as required.
	\qed
\end{enumerate}


          %%%%% ~~~~~~~~~~~~~~~~~~~~ %%%%%
