
          %%%%% ~~~~~~~~~~~~~~~~~~~~ %%%%%

\chapter{Irreducible representations of semidirect products \,$G \ltimes\! H$\, with \,$H$\, Abelian}
\setcounter{theorem}{0}
\setcounter{equation}{0}

%\cite{vanDerVaart1996}
%\cite{Kosorok2008}

%\renewcommand{\theenumi}{\alph{enumi}}
%\renewcommand{\labelenumi}{\textnormal{(\theenumi)}$\;\;$}
\renewcommand{\theenumi}{\roman{enumi}}
\renewcommand{\labelenumi}{\textnormal{(\theenumi)}$\;\;$}

          %%%%% ~~~~~~~~~~~~~~~~~~~~ %%%%%

\section{Representation-theoretic prerequisites}

\begin{proposition}
\mbox{}
\vskip 0.1cm
\noindent
Every irreducible finite-dimensional {\color{red}complex}-linear representation
of an {\color{red}Abelian} group is one-dimensional.
More precsely, suppose
\,$G$\, is an Abelian group,
\,$\rho : G \longrightarrow \GL(V)$\,
is a complex-linear representation, where
\,$V$\, is a complex vector space with \,$\dim_{\C}(V) < \infty$.\,
Then,
\,$\dim_{\C}(V) \,=\, 1$.\,
\end{proposition}
\proof

\qed

\vskip 0.5cm
\begin{definition}
\mbox{}
\vskip 0.1cm
\noindent
Suppose
\,$\rho : G \longrightarrow \GL(V)$\,
is a finite-dimensional representation of
the group \,$G$\,
on the (finite-dimensional) vector space \,$V$\,
over a field \,$\F$.\,
Then, the \textbf{character} of the representation
\,$\rho$\,
is the $\F$-valued function
\,$\chi_{\rho} : G \longrightarrow \F$\,
defined on \,$G$\, given by:
\begin{equation*}
\chi_{\rho}(\,g\,)
\; := \;
	\trace\!\left(\,\overset{{\color{white}.}}{\rho(g)}\,\right),
\quad
\textnormal{for each \,$g \in G$}
\end{equation*}
\end{definition}

\vskip 0.5cm
\begin{proposition}
\mbox{}
\vskip 0.1cm
\noindent
Suppose
\,$\rho : G \longrightarrow \GL(V)$\,
and
\,$\rho^{\prime} : G \longrightarrow \GL(V^{\prime})$\,
are finite-dimensional representations of
the group \,$G$\, on the (finite-dimensional) vector spaces
\,$V$\, and \,$V^{\prime}$,\,
respectively, where
\,$V$\, and \,$V^{\prime}$\,
share a common base scalar field \,$\F$.\,
Let
\,$\chi_{\rho} : G \longrightarrow \F$\,
and
\,$\chi_{\rho^{\prime}} : G \longrightarrow \F$\,
be the characters of
\,$\rho$\, and \,$\rho^{\prime}$,\,
respectively.
Then, the following statements are true:
\begin{enumerate}
\item
	$\chi_{\rho} : G \longrightarrow \F$\,
	is constant on each conjugacy class of \,$G$.\,
\item
	$\chi_{\rho\,\oplus\,\rho^{\prime}} \; = \; \chi_{\rho} \,+\, \chi_{\rho^{\prime}}$\,
\end{enumerate}
\end{proposition}
\proof
\begin{enumerate}
\item
	Simply observe that, for arbitrary
	\,$g, h \in G$,\,
	we have:
	\begin{eqnarray*}
	\chi_{\rho}\!\left(\,h \cdot \overset{{\color{white}1}}{g} \cdot h^{-1}\,\right)
	& = &
		\trace\!\left(\,\rho(h \cdot \overset{{\color{white}1}}{g} \cdot h^{-1})\,\right)
	\;\; = \;\;
		\trace\!\left(\,\rho(h) \cdot \rho(\overset{{\color{white}1}}{g}) \cdot \rho(h)^{-1}\,\right)
	\\
	& = &
		\trace\!\left(\, \rho(h)^{-1} \cdot \rho(h) \cdot \rho(\overset{{\color{white}1}}{g}) \,\right)
	\;\; = \;\;
		\trace\!\left(\, \rho(\overset{{\color{white}1}}{g}) \,\right)
	\\
	& = &
		\chi_{\rho}\!\left(\, \overset{{\color{white}.}}{g} \,\right)
	\end{eqnarray*}
\item
	Trivial.
\end{enumerate}
\qed

          %%%%% ~~~~~~~~~~~~~~~~~~~~ %%%%%

\vskip 1.0cm
\section{Semidirect products}

\begin{definition}[Semidirect product]
\mbox{}
\vskip 0.1cm
\noindent
Suppose:
\begin{itemize}
\item
	$G$\, and \,$H$\, are two groups, and
\item
	$\rho : G \longrightarrow \Aut(H)$\,
	is a left action of \,$G$\, on \,$H$.\,
\end{itemize}
Then, the \textbf{semidirect product} \,$G \ltimes_{\rho}\! H$\,
of \,$G$\, and \,$H$\, \textbf{\color{red}with respect to \,$\rho$}\,
is defined to be the group
obtained by defining on the Cartesian product
\,$G \times H$\,
the multiplication law:
\begin{equation*}
(\,g_{1},h_{1}\,) \cdot (\,g_{2},h_{2}\,)
\;\; = \;\;
	\left(\;
		\overset{{\color{white}1}}{g_{1}} \cdot g_{2}
		\;\overset{{\color{white}1}}{,}\;
		%h_{1} \cdot \rho(g_{1})\!\left[\,\overset{{\color{white}.}}{h_{2}}\,\right]
		h_{1} \cdot \rho(g_{1})\!\left[\,h_{2}\,\right]
		\,\right)
\end{equation*}
The identity element of \,$G \ltimes_{\rho}\! H$\, is then
\begin{equation*}
1_{G \ltimes H}
\;\; = \;\;
	\left(\,
		\overset{{\color{white}1}}{1_{G}}
		\,\overset{{\color{white}1}}{,}\,
		1_{H}
		\,\right)
\end{equation*}
and the inverse of \,$(\,g,h\,) \in G \ltimes_{\rho}\! H$\, is given by:
\begin{equation*}
(\, g \,,\, h \,)^{-1}
\;\; = \;\;
	\left(\;
		g^{-1}
		\;\overset{{\color{white}1}}{,}\;
		\overset{{\color{white}1}}{\rho}(g^{-1})\!\left[\, h^{-1} \,\right]
		\;\right)
\end{equation*}
\end{definition}

          %%%%% ~~~~~~~~~~~~~~~~~~~~ %%%%%

%\vskip 0.5cm
%\subsection{A representation of a semidirect product is determined by the restrictions to its factors}

\vskip 0.5cm
\begin{proposition}[Proposition 7.3, p.150, \cite{Berndt2007}]
\mbox{}
\vskip 0.1cm
\noindent
Suppose:
\begin{itemize}
\item
	$G$\, is a group, \,$H$\, an {\color{red}Abelian} group,
\item
	$\rho : G \longrightarrow \Aut(H)$\,
	is a left action of \,$G$\, on \,$H$,\,
\item
	$\pi : G \ltimes_{\rho}\! H \longrightarrow \GL(V)$\,
	is an arbitrary representation of the semidirect product
	\,$G \ltimes_{\rho}\! H$\, with respect to \,$\rho$,\, and
\item
	$\pi_{G} : G \longrightarrow \GL(V)$\, and \,$\pi_{H} : G \longrightarrow \GL(V)$\,
	are the restrictions of \,$\pi$\, to \,$G$\, and \,$H$,\, respectively; i.e.,
	\,$\pi_{G}$\, and \,$\pi_{H}$\,
	are given by:
	\begin{equation*}
	\pi_{G}(\,g\,) \; := \; \pi\!\left(\,(\,\overset{{\color{white}1}}{g}\,,0_{H}\,)\,\right),
	\quad\quad
	\textnormal{and}
	\quad\quad
	\pi_{H}(\,h\,) \; := \; \pi\!\left(\,(\,1_{G}\,,\overset{{\color{white}1}}{h}\,)\,\right).
	\end{equation*}
\end{itemize}
Then, the following statements are true:
\begin{enumerate}
\item
	$\pi$\, is completely determined by its restrictions \,$\pi_{G}$\, and \,$\pi_{H}$,\,
	in the sense that
	\begin{equation*}
	\pi\!\left(\;(\,\overset{{\color{white}1}}{g}\,,h\,)\;\right)
	\; = \;
		\pi_{H}(\,h\,) \cdot \pi_{G}(\,g\,),
	\quad
	\textnormal{for each \,$g \in G,\; h \in H$}
	\end{equation*}
\item
	\,$\pi_{G}$\, and \,$\pi_{H}$\, satisfy the following equality:
	\begin{equation*}
	\pi_{H}\!\left(\,\rho(g)\,[\,\overset{{\color{white}.}}{h}\,]\,\right)
	\;\; = \;\;
		\pi_{G}\!\left(\; \overset{{\color{white}.}}{g} \,\right)
		\cdot
		\pi_{H}\!\left(\overset{{\color{white}.}}{{\color{white}g}}\!\! h \,\right)
		\cdot
		\pi_{G}\!\left(\; g^{-1} \,\right)
	\end{equation*}
\end{enumerate}
\end{proposition}
\proof
\begin{enumerate}
\item
	Note that
	\,$(\,g\,,h\,)
	\;=\;
		\left(\;1_{G} \cdot g\,,\, h + \rho(g)\,[\;\overset{{\color{white}.}}{0_{H}}\,]\,\right)
	\;=\;
		(\,1_{G}\,,h\,)\cdot (\,g\,,0_{H}\,)
	$.\,
	Hence,
	\begin{equation*}
	\pi\!\left(\;(\,\overset{{\color{white}1}}{g}\,,h\,)\;\right)
	\;\; = \;\;
		\pi\!\left(\;
			(\,1_{G}\,,h\,)
			\cdot
			(\,\overset{{\color{white}1}}{g}\,,0_{H})
			\;\right)
	\;\; = \;\;
		\pi\!\left(\;
			(\,1_{G}\,,\overset{{\color{white}.}}{h}\,)
			\;\right)
		\cdot
		\pi\!\left(\;
			(\,\overset{{\color{white}.}}{g}\,,\overset{{\color{white}.}}{0_{H}})
			\;\right)
	\;\; = \;\;
		\pi_{H}(\,h\,) \cdot \pi_{G}(\,g\,),
	\end{equation*}
	which proves that \,$\pi$\, is indeed completely determined by
	\,$\pi_{G}$\, and \,$\pi_{H}$.\,
\item
	Recall the multiplication law of \,$G \ltimes_{\rho}\! H$:\,
	\begin{equation*}
	(\,g_{1},h_{1}\,) \cdot (\,g_{2},h_{2}\,)
	\;\; = \;\;
		\left(\;
			\overset{{\color{white}1}}{g_{1}} \cdot g_{2}
			\;\overset{{\color{white}1}}{,}\;
			h_{1} + \rho(g_{1})\,[\,h_{2}\,]
			\;\right)
	\end{equation*}
	Hence, repeated application of (i) yields:
	\begin{eqnarray*}
	\pi_{H}(\,h_{1}\,) \cdot \pi_{G}(\,g_{1}\,)
	\cdot
	\pi_{H}(\,h_{2}\,) \cdot \pi_{G}(\,g_{2}\,)
	& = &
		\pi\!\left(\;
			(\,\overset{{\color{white}1}}{g_{1}},h_{1}\,)
			\;\right)
		\cdot
		\pi\!\left(\;
			(\,\overset{{\color{white}1}}{g_{2}},h_{2}\,)
			\;\right)
	\\
	& = &
		\pi\!\left(\;
			(\,g_{1},h_{1}\,) \overset{{\color{white}1}}{\cdot} (\,g_{2},h_{2}\,)
			\;\right)
	\\
	& = &
		\pi\!\left(\;
			\left(\;
				\overset{{\color{white}1}}{g_{1}} \cdot g_{2}
				\;\overset{{\color{white}1}}{,}\;
				h_{1} + \rho(g_{1})\,[\,h_{2}\,]
				\;\right)
			\;\right)
	\\
	& = &
		\pi_{H}\!\left(\;h_{1} \overset{{\color{white}1}}{+} \rho(g_{1})\,[\,h_{2}\,]\,\right)
		\cdot\,
		\pi_{G}(\;g_{1}\cdot g_{2}\,)
	\\
	& = &
		\pi_{H}\!\left(\;\overset{{\color{white}.}}{h_{1}}\;\right)
		\cdot\,
		\pi_{H}\!\left(\;\overset{{\color{white}1}}{\rho(g_{1})}\,[\,h_{2}\,]\,\right)
		\cdot\,
		\pi_{G}\!\left(\;\overset{{\color{white}.}}{g_{1}}\;\right)
		\cdot
		\pi_{G}\!\left(\;\overset{{\color{white}.}}{g_{2}}\;\right)
	\end{eqnarray*}
	Hence,
	\begin{equation*}
		\pi_{H}\!\left(\;\overset{{\color{white}.}}{h_{1}}\;\right)
		\cdot
		\pi_{H}\!\left(\,\overset{{\color{white}1}}{\rho(g_{1})}\,[\,h_{2}\,]\,\right)
		\cdot
		\pi_{G}\!\left(\;\overset{{\color{white}.}}{g_{1}}\;\right)
		\cdot
		\pi_{G}\!\left(\;\overset{{\color{white}.}}{g_{2}}\;\right)
	\;\; = \;\;
		\pi_{H}(\,h_{1}\,) \cdot \pi_{G}(\,g_{1}\,)
		\cdot
		\pi_{H}(\,h_{2}\,) \cdot \pi_{G}(\,g_{2}\,)
	\end{equation*}
	Setting, in the above equality,
	\,$g_{1} = g$,\,  $h_{1} = 0_{H}$,\, $g_{2} = 1_{G}$\, and \,$h_{2} = h$\,
	yields
	\begin{equation*}
		\pi_{H}\!\left(\;\overset{{\color{white}.}}{0_{H}}\;\right)
		\cdot
		\pi_{H}\!\left(\;\rho(g) \overset{{\color{white}1}}{\cdot} h\;\right)
		\cdot
		\pi_{G}\!\left(\;\overset{{\color{white}.}}{g}\;\right)
		\cdot
		\pi_{G}\!\left(\;\overset{{\color{white}.}}{1_{G}}\;\right)
	\;\; = \;\;
		\pi_{H}(\,0_{H}\,) \cdot \pi_{G}(\,g\,)
		\cdot
		\pi_{H}(\,h\,) \cdot \pi_{G}(\,1_{G}\,)\,,
	\end{equation*}
	which simplifies to
	\begin{equation*}
		\pi_{H}\!\left(\;\overset{{\color{white}1}}{\rho(g)}\,[\,h\,]\,\right)
		\cdot
		\pi_{G}\!\left(\;\overset{{\color{white}.}}{g}\;\right)
	\;\; = \;\;
		\pi_{G}(\,g\,)
		\cdot
		\pi_{H}(\,h\,)
	\end{equation*}
	which can further be rewritten as
	\begin{equation*}
		\pi_{H}\!\left(\;\overset{{\color{white}-}}{\rho(g)}\,[\,h\,]\,\right)
	\;\; = \;\;
		\pi_{G}(\,g\,)
		\cdot
		\pi_{H}(\,h\,)
		\cdot
		\pi_{G}\!\left(\,\overset{{\color{white}.}}{{\color{white}g}}\!\!g^{-1}\;\right),
	\end{equation*}
	as required. \qed
\end{enumerate}

          %%%%% ~~~~~~~~~~~~~~~~~~~~ %%%%%

\vskip 1.0cm
\section{Irreducible representations of semidirect products as induced representations}

          %%%%% ~~~~~~~~~~~~~~~~~~~~ %%%%%

\begin{theorem}
\mbox{}
\vskip 0.1cm
\noindent
Suppose:
\begin{itemize}
\item
	$G \ltimes_{\rho}\! H$\, is the semidirect product of the groups \,$G$\, and \,$H$\,
	with respect to the left action
	\,$\rho : G \longrightarrow \Aut(H)$,\,
	where the group \,$H$\, is Abelian.
\end{itemize}
Let
\begin{equation*}
\pi : G \ltimes_{\rho}\! H \longrightarrow \GL(V)
\end{equation*}
be an irreducible finite-dimensional representation.
\end{theorem}

          %%%%% ~~~~~~~~~~~~~~~~~~~~ %%%%%

%\vskip 0.5cm
%\subsection{Irreducible unitary representations of regular semidirect products}

\vskip 0.5cm
\begin{definition}[Definition 7.1, p.150, \cite{Berndt2007}]
\mbox{}
\vskip 0.1cm
\noindent
A semidirect
\,$G \ltimes_{\rho}\! H$\, of \,$G$\, and \,$H$\,
is said to be \textbf{regular} if \,... ...
\end{definition}

          %%%%% ~~~~~~~~~~~~~~~~~~~~ %%%%%

\vskip 0.5cm
\begin{theorem}[Theorem 7.7, p.151, \cite{Berndt2007}]
\mbox{}
\vskip 0.1cm
\noindent
Suppose:
\begin{itemize}
\item
	$G$\, and \,$H$\, are separable and locally compact groups, and
	\,$H$\, is furthermore abelian.
\item
	$\rho : G \longrightarrow \textnormal{Aut}(H)$\,
	is a left action of \,$G$\, on \,$H$,\, and
\item
	the semidirect product
	\,$G \ltimes_{\rho}\! H$\, of \,$G$\, amd \,$H$\,
	with respect to \,$\rho$\,
	is a regular.
\end{itemize}
Then, every irreducible unitary representation of
\,$G \ltimes_{\rho}\! H$\,
is unitarily equivalent to an induced representation.
\end{theorem}

          %%%%% ~~~~~~~~~~~~~~~~~~~~ %%%%%

\vskip 1.0cm
\section{Irreducible unitary representations of \,$\SL(2,\C) \ltimes \Re^{1,3}$}

          %%%%% ~~~~~~~~~~~~~~~~~~~~ %%%%%

Let
\,$\mathcal{H}_{m}^{\uparrow},\, \mathcal{H}_{m}^{\downarrow} \,\subset\, \Re^{1,3}$\,
be the orbits of
\,$\SL(2,\C) \curvearrowright \Re^{1,3}$\,
defined by:
\begin{eqnarray*}
\mathcal{H}_{m}^{\uparrow}
& := &
	\left\{\;\,
		\left.
		p = (\,p_{0},p_{1},p_{2},p_{3}\,)
		\overset{{\color{white}1}}{\in}
		\Re^{1,3}
		\;\;\;\right\vert
		\begin{array}{c}
			p_{0} \;{\color{red}>}\; 0\,,
			\\
			-m^{2} \, \overset{{\color{white}1}}{=} \, -\,(p_{0})^{2} + (p_{1})^{2} + (p_{2})^{2} + (p_{3})^{2}
			\end{array}
		\right\},
\\ \\
\mathcal{H}_{m}^{\downarrow}
& := &
	\left\{\;\,
		\left.
		p = (\,p_{0},p_{1},p_{2},p_{3}\,)
		\overset{{\color{white}1}}{\in}
		\Re^{1,3}
		\;\;\;\right\vert
		\begin{array}{c}
			p_{0} \;{\color{red}<}\; 0\,,
			\\
			-m^{2} \, \overset{{\color{white}1}}{=} \, -\,(p_{0})^{2} + (p_{1})^{2} + (p_{2})^{2} + (p_{3})^{2}
			\end{array}
		\right\}
\end{eqnarray*}
where
\,$m \in [\,0,\infty) \sqcup \i\,(\,0,\infty)$\, (hence, \,$m^{2}$\, varies over all of \,$\Re$).\,

\vskip 0.5cm
\noindent
Consider \,$m > 0$.\,
\begin{itemize}
\item
	$\textnormal{Stab}_{\SL(2,\C)}\!\left(\,(m,0,0,0)^{T}\,\right) \;\cong\; \SU(2)$
\item
	Let
	\,$\pi_{s} : \SU(2) \longrightarrow \GL(V_{s})$\,
	be the irreducible finite-dimensional complex representation of \,$\SU(2)$\,
	determined by
	\,$s \;\in\; \{\,0\,\} \,\bigsqcup\, \dfrac{1}{2}\cdot\N$\,
	(i.e., $\dim_{\C}(V_{s}) = 2s + 1$.)
\item
	$E_{m,s} \longrightarrow \mathcal{H}_{m}^{\uparrow}$\,
	is the vector bundle determined by \,$\pi_{s}$\, (how exactly?)
\item
	$\SL(2,\C) \ltimes \Re^{1,3}$\,
	has an induced action on the space
	\,$\Gamma(\,E_{m,s}\,)$\,
	of smooth sections of
	\,$E_{m,s}$.\,
\end{itemize}



          %%%%% ~~~~~~~~~~~~~~~~~~~~ %%%%%
