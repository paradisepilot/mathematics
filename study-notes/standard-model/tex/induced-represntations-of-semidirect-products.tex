
          %%%%% ~~~~~~~~~~~~~~~~~~~~ %%%%%

\chapter{Induced representations of semidirect products \,$G \ltimes\! H$\, with \,$H$\, abelian}
\setcounter{theorem}{0}
\setcounter{equation}{0}

%\cite{vanDerVaart1996}
%\cite{Kosorok2008}

%\renewcommand{\theenumi}{\alph{enumi}}
%\renewcommand{\labelenumi}{\textnormal{(\theenumi)}$\;\;$}
\renewcommand{\theenumi}{\roman{enumi}}
\renewcommand{\labelenumi}{\textnormal{(\theenumi)}$\;\;$}

          %%%%% ~~~~~~~~~~~~~~~~~~~~ %%%%%

\section{Semidirect products}

\begin{definition}
\mbox{}
\vskip 0.1cm
\noindent
Suppose:
\begin{itemize}
\item
	$G$\, and \,$H$\, are two groups, and
\item
	$\rho : G \longrightarrow \textnormal{Aut}(H)$\,
	is a left action of \,$G$\, on \,$H$.\,
\end{itemize}
Then, the \textbf{semidirect product} \,$G \ltimes_{\rho}\! H$\,
of \,$G$\, and \,$H$\, \textbf{\color{red}with respect to \,$\rho$}\,
is defined to be the group
obtained by defining on the Cartesian product
\,$G \times H$\,
the multiplication law:
\begin{equation*}
(\,g_{1},h_{1}\,) \cdot (\,g_{2},h_{2}\,)
\;\; = \;\;
	\left(\,
		\overset{{\color{white}1}}{g_{1}} \cdot g_{2}
		\,\overset{{\color{white}1}}{,}\,
		h_{1} \cdot \rho(g_{1}) \cdot h_{2}
		\,\right)
\end{equation*}
The identity element of \,$G \ltimes_{\rho}\! H$\, is then
\begin{equation*}
1_{G \ltimes H}
\;\; = \;\;
	\left(\,
		\overset{{\color{white}1}}{1_{G}}
		\,\overset{{\color{white}1}}{,}\,
		1_{H}
		\,\right)
\end{equation*}
and the inverse of \,$(\,g,h\,) \in G \ltimes_{\rho}\! H$\, is given by:
\begin{equation*}
(\, g \,,\, h \,)^{-1}
\;\; = \;\;
	\left(\;
		g^{-1}
		\;\overset{{\color{white}1}}{,}\;
		\overset{{\color{white}1}}{\rho}(g^{-1}) \cdot h^{-1}
		\;\right)
\end{equation*}
\end{definition}

          %%%%% ~~~~~~~~~~~~~~~~~~~~ %%%%%

\vskip 0.5cm
\section{A representation of a semidirect product is determined by the restrictions to its factors}

\begin{proposition}[Proposition 7.3, p.150, \cite{Berndt2007}]
\mbox{}
\vskip 0.1cm
\noindent
Suppose:
\begin{itemize}
\item
	$G$\, is a group, \,$H$\, an {\color{red}abelian} group,
\item
	$\rho : G \longrightarrow \textnormal{Aut}(H)$\,
	is a left action of \,$G$\, on \,$H$,\,
\item
	$\pi : G \ltimes_{\rho}\! H \longrightarrow \GL(V)$\,
	is an arbitrary representation of the semidirect product
	\,$G \ltimes_{\rho}\! H$\, with respect to \,$\rho$,\, and
\item
	$\pi_{G} : G \longrightarrow \GL(V)$\, and \,$\pi_{H} : G \longrightarrow \GL(V)$\,
	are the restrictions of \,$\pi$\, to \,$G$\, and \,$H$,\, respectively; i.e.,
	\,$\pi_{G}$\, and \,$\pi_{H}$\,
	are given by:
	\begin{equation*}
	\pi_{G}(\,g\,) \; := \; \pi\!\left(\,(\,\overset{{\color{white}1}}{g}\,,0\,)\,\right),
	\quad\quad
	\textnormal{and}
	\quad\quad
	\pi_{H}(\,h\,) \; := \; \pi\!\left(\,(\,1_{G}\,,\overset{{\color{white}1}}{h}\,)\,\right).
	\end{equation*}
\end{itemize}
Then, the following statements are true:
\begin{enumerate}
\item
	$\pi$\, is completely determined by its restrictions \,$\pi_{G}$\, and \,$\pi_{H}$,\, and
\item
	\,$\pi_{G}$\, and \,$\pi_{H}$\, satisfy the following equality:
	\begin{equation*}
	\pi_{H}\!\left(\,\rho(g) \cdot \overset{{\color{white}.}}{h}\,\right)
	\;\; = \;\;
		\pi_{G}\!\left(\; \overset{{\color{white}.}}{g} \,\right)
		\cdot
		\pi_{H}\!\left(\overset{{\color{white}.}}{{\color{white}g}}\!\! h \,\right)
		\cdot
		\pi_{G}\!\left(\; g^{-1} \,\right)
	\end{equation*}
\end{enumerate}
\end{proposition}
\proof
\begin{enumerate}
\item
	Note that
	\,$(\,g\,,h\,) \;=\; \left(\;1_{G} \cdot g\,,\, h + \rho(g)\cdot \overset{{\color{white}.}}{0}\,\right) \;=\; (\,1_{G}\,,h\,)\cdot (\,g\,,0\,)$.\,
	Hence,
	\begin{equation*}
	\pi\!\left(\;(\,\overset{{\color{white}1}}{g}\,,h\,)\;\right)
	\;\; = \;\;
		\pi\!\left(\;
			(\,1_{G}\,,h\,)
			\cdot
			(\,\overset{{\color{white}1}}{g}\,,0\,)
			\;\right)
	\;\; = \;\;
		\pi\!\left(\;
			(\,1_{G}\,,\overset{{\color{white}.}}{h}\,)
			\;\right)
		\cdot
		\pi\!\left(\;
			(\,\overset{{\color{white}.}}{g}\,,\overset{{\color{white}.}}{0}\,)
			\;\right)
	\;\; = \;\;
		\pi_{H}(\,h\,) \cdot \pi_{G}(\,g\,),
	\end{equation*}
	which proves that \,$\pi$\, is indeed completely determined by
	\,$\pi_{G}$\, and \,$\pi_{H}$.\,
\item
	Recall the multiplication law of \,$G \ltimes_{\rho}\! H$:\,
	\begin{equation*}
	(\,g_{1},h_{1}\,) \cdot (\,g_{2},h_{2}\,)
	\;\; = \;\;
		\left(\;
			\overset{{\color{white}1}}{g_{1}} \cdot g_{2}
			\;\overset{{\color{white}1}}{,}\;
			h_{1} + \rho(g_{1}) \cdot h_{2}
			\;\right)
	\end{equation*}
	Hence,
	\begin{eqnarray*}
	\pi_{H}(\,h_{1}\,) \cdot \pi_{G}(\,g_{1}\,)
	\cdot
	\pi_{H}(\,h_{2}\,) \cdot \pi_{G}(\,g_{2}\,)
	& = &
		\pi\!\left(\;
			(\,\overset{{\color{white}1}}{g_{1}},h_{1}\,)
			\;\right)
		\cdot
		\pi\!\left(\;
			(\,\overset{{\color{white}1}}{g_{2}},h_{2}\,)
			\;\right)
	\\
	& = &
		\pi\!\left(\;
			(\,g_{1},h_{1}\,) \overset{{\color{white}1}}{\cdot} (\,g_{2},h_{2}\,)
			\;\right)
	\\
	& = &
		\pi\!\left(\;
			\left(\;
				\overset{{\color{white}1}}{g_{1}} \cdot g_{2}
				\;\overset{{\color{white}1}}{,}\;
				h_{1} + \rho(g_{1}) \cdot h_{2}
				\;\right)
			\;\right)
	\\
	& = &
		\pi_{H}\!\left(\;h_{1} + \rho(g_{1}) \overset{{\color{white}1}}{\cdot} h_{2}\;\right)
		\cdot
		\pi_{G}(\;g_{1}\cdot g_{2}\,)
	\\
	& = &
		\pi_{H}\!\left(\;\overset{{\color{white}.}}{h_{1}}\;\right)
		\cdot
		\pi_{H}\!\left(\;\rho(g_{1}) \overset{{\color{white}1}}{\cdot} h_{2}\;\right)
		\cdot
		\pi_{G}\!\left(\;\overset{{\color{white}.}}{g_{1}}\;\right)
		\cdot
		\pi_{G}\!\left(\;\overset{{\color{white}.}}{g_{2}}\;\right)
	\end{eqnarray*}
	Hence,
	\begin{equation*}
		\pi_{H}\!\left(\;\overset{{\color{white}.}}{h_{1}}\;\right)
		\cdot
		\pi_{H}\!\left(\;\rho(g_{1}) \overset{{\color{white}1}}{\cdot} h_{2}\;\right)
		\cdot
		\pi_{G}\!\left(\;\overset{{\color{white}.}}{g_{1}}\;\right)
		\cdot
		\pi_{G}\!\left(\;\overset{{\color{white}.}}{g_{2}}\;\right)
	\;\; = \;\;
		\pi_{H}(\,h_{1}\,) \cdot \pi_{G}(\,g_{1}\,)
		\cdot
		\pi_{H}(\,h_{2}\,) \cdot \pi_{G}(\,g_{2}\,)
	\end{equation*}
	Setting, in the above equality,
	\,$g_{1} = g$,\,  $h_{1} = 0$,\, $g_{2} = 1_{G}$\, and \,$h_{2} = h$\,
	yields
	\begin{equation*}
		\pi_{H}\!\left(\;\overset{{\color{white}.}}{0}\;\right)
		\cdot
		\pi_{H}\!\left(\;\rho(g) \overset{{\color{white}1}}{\cdot} h\;\right)
		\cdot
		\pi_{G}\!\left(\;\overset{{\color{white}.}}{g}\;\right)
		\cdot
		\pi_{G}\!\left(\;\overset{{\color{white}.}}{1_{G}}\;\right)
	\;\; = \;\;
		\pi_{H}(\,0\,) \cdot \pi_{G}(\,g\,)
		\cdot
		\pi_{H}(\,h\,) \cdot \pi_{G}(\,1_{G}\,)\,,
	\end{equation*}
	which simplifies to
	\begin{equation*}
		\pi_{H}\!\left(\;\rho(g) \overset{{\color{white}1}}{\cdot} h\;\right)
		\cdot
		\pi_{G}\!\left(\;\overset{{\color{white}.}}{g}\;\right)
	\;\; = \;\;
		\pi_{G}(\,g\,)
		\cdot
		\pi_{H}(\,h\,)
	\end{equation*}
	which can further be rewritten as
	\begin{equation*}
		\pi_{H}\!\left(\;\rho(g) \overset{{\color{white}-}}{\cdot} h\;\right)
	\;\; = \;\;
		\pi_{G}(\,g\,)
		\cdot
		\pi_{H}(\,h\,)
		\cdot
		\pi_{G}\!\left(\,\overset{{\color{white}.}}{{\color{white}g}}\!\!g^{-1}\;\right),
	\end{equation*}
	as required. \qed
\end{enumerate}

          %%%%% ~~~~~~~~~~~~~~~~~~~~ %%%%%

\vskip 0.5cm
\section{Irreducible unitary representations of regular semidirect products}

\vskip 0.5cm
\begin{definition}[Definition 7.1, p.150, \cite{Berndt2007}]
\mbox{}
\vskip 0.1cm
\noindent
A semidirect
\,$G \ltimes_{\rho}\! H$\, of \,$G$\, amd \,$H$\,
is said to be \textbf{regular} if

\end{definition}

\vskip 0.5cm
\begin{theorem}[Theorem 7.7, p.151, \cite{Berndt2007}]
\mbox{}
\vskip 0.1cm
\noindent
Suppose:
\begin{itemize}
\item
	$G$\, and \,$H$\, are separable and locally compact groups, and
	\,$H$\, is furthermore abelian.
\item
	$\rho : G \longrightarrow \textnormal{Aut}(H)$\,
	is a left action of \,$G$\, on \,$H$,\, and
\item
	the semidirect product
	\,$G \ltimes_{\rho}\! H$\, of \,$G$\, amd \,$H$\,
	with respect to \,$\rho$\,
	is a regular.
\end{itemize}
Then, every irreducible unitary representation of
\,$G \ltimes_{\rho}\! H$\,
is unitarily equivalent to an induced representation.
\end{theorem}

          %%%%% ~~~~~~~~~~~~~~~~~~~~ %%%%%
