
          %%%%% ~~~~~~~~~~~~~~~~~~~~ %%%%%

\chapter{Two double coverings: $\SU(2) \longrightarrow \SO(3)$ and $\SL(2,\C) \longrightarrow \SOup(1,3)$}
\setcounter{theorem}{0}
\setcounter{equation}{0}

%\cite{vanDerVaart1996}
%\cite{Kosorok2008}

%\renewcommand{\theenumi}{\alph{enumi}}
%\renewcommand{\labelenumi}{\textnormal{(\theenumi)}$\;\;$}
\renewcommand{\theenumi}{\roman{enumi}}
\renewcommand{\labelenumi}{\textnormal{(\theenumi)}$\;\;$}

          %%%%% ~~~~~~~~~~~~~~~~~~~~ %%%%%

\section{The Lie groups  \,$\SL(n,\C)$,\, $\SO(n)$\, and \,$\SU(n)$}

%\vskip 0.5cm
%$\SU(2) \overset{\textnormal{\scriptsize 2\textnormal{$:$}1{\color{white}.}}}{\longrightarrow} \SOup(1,3)$

          %%%%% ~~~~~~~~~~~~~~~~~~~~ %%%%%

\vskip 0.3cm
\begin{definition}
\mbox{}
\vskip 0.1cm
\noindent
The \textbf{special linear group of degree $n$ over $\C$} is defined as follows:
\begin{equation*}
\SL(n,\C)
\; := \;
	\left\{\;\,
		g \overset{{\color{white}.}}{\in} \textnormal{GL}(n,\C)
		\;\,\left\vert\;\;
			\det(\,g\,) \overset{{\color{white}1}}{=} 1
			\right.
		\;\right\}
\end{equation*}
\end{definition}

\vskip 0.5cm
\begin{proposition}[Set-theoretic characterizations of \,$\mathfrak{sl}(n,\C)$]
\label{SetTheoreticCharacterizationOfslTwoC}
\mbox{}
\vskip -0.1cm
\begin{equation*}
\mathfrak{sl}(n,\C)
\; = \;
	\left\{\;
		X \,\in\, \mathfrak{gl}(n,\C) \,=\, \C^{n \times n}
		\;\left\vert\;\,
			\textnormal{trace}(X) = \overset{{\color{white}1}}{0}
			\right.
		\,\right\}
\end{equation*}
\end{proposition}
\proof
We invoke the fact that \,$\det(e^{\,t\,\cdot\,X}) \,=\, e^{\,t\,\cdot\,\textnormal{trace}(X)}$,
for each \,$X \in \C^{n \times n}$.
Thus,
\begin{eqnarray*}
&&
	X \,\in\, \mathfrak{sl}(n,\C)
	\quad\Longrightarrow\quad
	e^{\,t\cdot\,X} \,\in\, \textnormal{SL}(n,\C)
	\quad\Longrightarrow\quad
	\det\!\left(\,e^{\,t\cdot\,X}\,\right) \,=\, 1
\\
& \Longrightarrow\quad &
	\textnormal{trace}(X)
	\; = \;
		\left.\dfrac{\d}{\d\,t}\right\vert_{t=0}\left(\,\overset{{\color{white}1}}{e^{\,t\,\cdot\,\textnormal{trace}(X)}}\,\right)
	\; = \;
		\left.\dfrac{\d}{\d\,t}\right\vert_{t=0}\left(\,\overset{{\color{white}1}}{\det(e^{\,t\,\cdot\,X})}\,\right)
	\; = \;
		\left.\dfrac{\d}{\d\,t}\right\vert_{t=0}\left(\,\overset{{\color{white}1}}{1}\,\right)
	\; = \;
		0
\end{eqnarray*}
Conversely, suppose \,$\textnormal{trace}(X) = 0$.\,
Then, \,$\det(e^{\,t\,\cdot\,X}) \,=\, e^{\,t\,\cdot\,\textnormal{trace}(X)} \,=\, e^{\,t\,\cdot\,0} \,=\, 1$,\,
which implies that \,$e^{\,t\,\cdot\,X} \,\in\, \textnormal{SL}(n,\C)$,\, hence \,$X \,\in\, \mathfrak{sl}(n,\C)$.
This completes the proof of the equality (of sets) in question.
\qed

          %%%%% ~~~~~~~~~~~~~~~~~~~~ %%%%%

\vskip 1.0cm
\begin{definition}
\mbox{}
\vskip 0.1cm
\noindent
The \textbf{orthogonal group of degree $n$} is defined as follows:
\begin{equation*}
\textnormal{O}(n)
\; := \;
	\left\{\;\,
		g \overset{{\color{white}.}}{\in} \textnormal{GL}(n,\Re)
		\;\left\vert\;\,
			g^{T} \cdot g = I_{n}
			\right.
		\;\right\}
\end{equation*}
The \textbf{special orthogonal group of degree $n$} is defined as follows:
\begin{equation*}
\textnormal{SO}(n)
\; := \;
	\left\{\;\,
		g \overset{{\color{white}.}}{\in} \textnormal{GL}(n,\Re)
		\;\left\vert\;\,
			g^{T} \cdot g = I_{n}\,,
			\;
			\textnormal{det}(g) = 1
			\right.
		\;\right\}
\end{equation*}
\end{definition}

\vskip 0.5cm
\begin{proposition}[Set-theoretic characterizations of the Lie algebras of $\textnormal{O}(n)$ and $\textnormal{SO}(n)$]
\begin{eqnarray*}
\mathfrak{o}(n)
& = &
	\left\{\;\,
		X \overset{{\color{white}.}}{\in} \gl(n,\Re)
		\;\left\vert\;\,
			X^{T} = -X
			\right.
		\;\right\}
\\
\so(n)
& = &
	\left\{\;\,
		X \overset{{\color{white}.}}{\in} \gl(n,\Re)
		\;\left\vert\;\,
			X^{T} = -X\,,
			\;
			\textnormal{trace}(X) = 0
			\right.
		\;\right\}
\end{eqnarray*}
\end{proposition}

          %%%%% ~~~~~~~~~~~~~~~~~~~~ %%%%%

\vskip 0.5cm
\begin{definition}
\mbox{}
\vskip 0.1cm
\noindent
The \textbf{unitary group of degree $n$} is defined as follows:
\begin{equation*}
\textnormal{U}(n)
\; := \;
	\left\{\;\,
		g \overset{{\color{white}.}}{\in} \textnormal{GL}(n,\C)
		\;\;\left\vert\;\;
			g^{\dagger} \cdot g \overset{{\color{white}1}}{=} I_{n}
			\right.
		\;\right\}
\end{equation*}
where \,$g^{\dagger}$\, is the conjugate transpose of
\,$g \in \textnormal{GL}(n,\C)$\,
and
\,$I_{n} \in \textnormal{GL}(n,\C)$\,
is the identity matrix.
\vskip 0.1cm
\noindent
The \textbf{special unitary group of degree $n$} is defined as follows:
\begin{equation*}
\textnormal{SU}(n)
\; := \;
	\left\{\;\,
		g \overset{{\color{white}.}}{\in} \textnormal{GL}(n,\C)
		\;\,\left\vert\;\;
			g^{\dagger} \cdot g \overset{{\color{white}1}}{=} I_{n}\,,
			\;
			\textnormal{det}(g) = 1
			\right.
		\;\right\}
\end{equation*}
\end{definition}

          %%%%% ~~~~~~~~~~~~~~~~~~~~ %%%%%

\vskip 0.5cm
\begin{proposition}[Set-theoretic characterizations of \,$\mathfrak{u}(n)$, and $\mathfrak{su}(n)$]
\mbox{}
\vskip 0.1cm
\begin{enumerate}
\item
	\begin{equation*}
	\mathfrak{u}(n)
	\; = \;
		\left\{\;
			X \,\in\, \mathfrak{gl}(n,\C) \,=\, \C^{n \times n}
			\;\left\vert\;\,
				X + X^{\dagger} = \overset{{\color{white}1}}{0}
				\right.
			\,\right\}
	\end{equation*}
\item
	\begin{equation*}
	\mathfrak{su}(n)
	\; = \;
		\left\{\;
			X \,\in\, \mathfrak{gl}(n,\C) \,=\, \C^{n \times n}
			\;\left\vert\;\,
				\begin{array}{c}
				X + X^{\dagger} = \overset{{\color{white}1}}{0}
				\\
				\textnormal{trace}(X) = \overset{{\color{white}1}}{0}
				\end{array}
				\right.
			\,\right\}
	\end{equation*}
\end{enumerate}
\end{proposition}
\proof
\begin{enumerate}
\item
	\begin{eqnarray*}
	&&
		X \,\in\, \mathfrak{u}(n)
		\quad\Longrightarrow\quad
		e^{\,t\,\cdot\,X} \,\in\, \textnormal{U}(n)
	\\
	& \Longrightarrow\quad &
		I_{n}
			\,=\, \left(\,e^{\,t\,\cdot\,X}\,\right)^{\!\dagger} \cdot \left(\,e^{\,t\,\cdot\,X}\,\right)
			\,=\, \left(\,e^{\,t\,\cdot\,X^{\dagger}}\,\right) \cdot \left(\,e^{\,t\,\cdot\,X}\,\right)
			\,=\, e^{\,t\,\cdot\,(X^{\dagger}+X)}
	\\
	& \Longrightarrow\quad &
		X \,+\, X^{\dagger}
		\; = \;
			\left.\dfrac{\d}{\d\,t}\right\vert_{t=0}\left(\,\overset{{\color{white}1}}{e^{\,t\,\cdot\,(X+X^{\dagger})}}\,\right)
		\; = \;
			\left.\dfrac{\d}{\d\,t}\right\vert_{t=0}\left(\,\overset{{\color{white}1}}{I_{n}}\,\right)
		\; = \;
			0
	\end{eqnarray*}
	Conversely, suppose \,$X + X^{\dagger} \,=\, 0$.\,
	Then, \,$I_{n}$
	\,$=$\, $e^{\,0_{n \times n}}$
	\,$=$\, $e^{\,t\,\cdot(X^{\dagger}+X)}$
	\,$=\, \cdots \,=$\, $\left(e^{\,t\,\cdot\,X}\right)^{\!\dagger}\cdot\left(e^{\,t\,\cdot\,X}\right)$,\,
	which implies that \,$e^{\,t\,\cdot\,X} \,\in\, \textnormal{U}(n)$,\, hence \,$X \,\in\, \mathfrak{u}(n)$.
	This completes the proof of the equality (of sets) in question.
\item
	Immediate by (i) and Proposition \ref{SetTheoreticCharacterizationOfslTwoC}.
	\qed
\end{enumerate}

          %%%%% ~~~~~~~~~~~~~~~~~~~~ %%%%%

\vskip 1.0cm
\section{The Lie groups \,$\SO(3)$\, and \,$\SU(2)$}

          %%%%% ~~~~~~~~~~~~~~~~~~~~ %%%%%

\vskip 0.3cm
\begin{proposition}[Parametrization of \,$\textnormal{SU}(2)$]
\mbox{}
\vskip 0.1cm
\noindent
$\textnormal{SU}(2)$ admits the following parametrization:
\begin{equation*}
\textnormal{SU}{(2)}
\; := \;
	\left\{\,
		\left.
		\left(\begin{array}{rr}
		a & -\overline{b}
		\\
		\overset{{\color{white}-}}{b} & \overline{a}
		\end{array}\right)
		\overset{{\color{white}.}}{\in}
		\C^{2 \times 2}
		\;\;\right\vert\;\,
			\vert\, a \,\vert^{2} \,+\, \vert\, b \,\vert^{2} \,=\, 1
		\;\right\}
\end{equation*}
Hence, the (real) Lie group {\color{red}$\textnormal{SU}(2)$ is diffeomorphic to $S^{3}$}, the $3$-dimensional unit sphere
(in $4$-dimensional Euclidean space).
In particular, $\textnormal{SU}(2)$ is a {\color{red}simply connected} $3$-dimensional real manifold.
\end{proposition}
\proof
Suppose:
\begin{equation*}
g
\; = \;
	\left(\begin{array}{cc}
		a & c
		\\
		\overset{{\color{white}-}}{b} & d
		\end{array}\right)
\; \in \;
\textnormal{SU}(2)
\end{equation*}
First note that the component form of the condition \,$g^{\dagger}\cdot g = I_{2}$\, is:
\begin{equation*}
\left(\begin{array}{cc}
	1 & 0
	\\
	\overset{{\color{white}-}}{0} & 1
	\end{array}\right)
\; = \;
	g^{\dagger} \cdot g
\; = \;
	\left(\begin{array}{cc}
		\overline{a} & \overline{b}
		\\
		\overset{{\color{white}-}}{\overline{c}} & \overline{d}
		\end{array}\right)
	\cdot
	\left(\begin{array}{cc}
		a & c
		\\
		\overset{{\color{white}-}}{b} & d
		\end{array}\right)
\; = \;
	\left(\begin{array}{cc}
		a\overline{a} + b\overline{b} & \overline{a}c + \overline{b}d
		\\
		\overset{{\color{white}-}}{a\overline{c} + b\overline{d}} & c\overline{c} + d\overline{d}
		\end{array}\right)
\end{equation*}
Thus, we see that
\begin{equation*}
g
\; = \;
	\left(\begin{array}{cc}
		a & c
		\\
		\overset{{\color{white}-}}{b} & d
		\end{array}\right)
\;\in\;
	\textnormal{SU}(2)
\quad\Longleftrightarrow\quad
\left\{
	\begin{array}{ccc}
		g^{\dagger} \cdot g &=& I_{2}
		\\
		\det(g) &=& \overset{{\color{white}1}}{1}
		\end{array}
		\right.
\quad\Longleftrightarrow\quad
\left\{
	\begin{array}{ccc}
	\vert\,a\,\vert^{2} + \vert\,b\,\vert^{2} &=& 1
	\\
	\vert\,c\,\vert^{2} + \vert\,d\,\vert^{2} &\overset{{\color{white}1}}{=}& 1
	\\
	a\overline{c} \;\, + \,\; b\overline{d} &\overset{{\color{white}1}}{=}& 0
	\\
	ad \;\, - \,\; bc &\overset{{\color{white}1}}{=}& 1
	\end{array}
	\right.
\end{equation*}
Next, note that
\begin{equation*}
a\overline{c} + b\overline{d} = 0
\quad\Longleftrightarrow\quad
	\left\langle
		\left(\begin{array}{c} a \\ b \end{array}\right)
		\,,\,
		\left(\begin{array}{c} c \\ d \end{array}\right)
		\right\rangle_{\C^{2}}
	\;=\;
	0
\end{equation*}
Since
\,$\dim_{\C}\left(\begin{array}{c} a \\ b \end{array}\right)^{\perp} =\, 1$,\,
the above equality (i.e., orthogonality of the two columns of \,$g$) implies:
\begin{equation*}
\left(\begin{array}{c} c \\ d \end{array}\right)
\; \in \;
	\left(\begin{array}{c} a \\ b \end{array}\right)^{\perp}
\; = \;
	\textnormal{span}_{\C}\left\{
		\left(\begin{array}{r} -\overline{b} \\ \overline{a} \end{array}\right)
		\right\}
\quad\Longleftrightarrow\quad
\left(\begin{array}{c} c \\ d \end{array}\right)
\; = \;
	\lambda \left(\begin{array}{r} -\overline{b} \\ \overline{a} \end{array}\right),
	\;\;
	\textnormal{for some $\lambda \in \C$}
\end{equation*}
So, we now know that $g$ has the form:
\begin{equation*}
g
\; = \;
	\left(\begin{array}{rr}
		a & -\lambda\,\overline{b}
		\\
		\overset{{\color{white}-}}{b} & \lambda\,\overline{a}
		\end{array}\right)
\end{equation*}
Next,
\begin{equation*}
1
\,=\, \det(g)
\,=\, a\cdot(\lambda\,\overline{a}) - b \cdot (-\lambda\,\overline{b})
\,=\, \lambda\cdot(\vert\,a\,\vert^{2} + \vert\,b\,\vert^{2})
\quad\Longrightarrow\quad
	\lambda = 1
\end{equation*}
We may now conclude that
\begin{equation*}
g
\; = \;
	\left(\begin{array}{rr}
		a & -\,\overline{b}
		\\
		\overset{{\color{white}-}}{b} & \overline{a}
		\end{array}\right),\,
\quad
\textnormal{where \,$\vert\,a\,\vert^{2} + \vert\,b\,\vert^{2} = 1$}
\end{equation*}
This completes the proof of the Proposition.
\qed

          %%%%% ~~~~~~~~~~~~~~~~~~~~ %%%%%

\vskip 1.0cm
\section{The Lie groups \,$\SOup(1,3)$\, and \,$\SL(2,\C)$}

          %%%%% ~~~~~~~~~~~~~~~~~~~~ %%%%%

\noindent
Let \,$Q_{(1,n)} \,:=\, \diag(-1,1,\cdots,1) \in \Re^{(n+1) \times (n+1)}$.

\vskip 0.5cm
\begin{definition}[$\textnormal{O}(1,n)$]
\mbox{}
\vskip 0.1cm
\noindent
The \textbf{Lorentz} group is defined to be:
\begin{equation*}
\textnormal{O}(1,n)
\; := \;
	\left\{\;\,
		A \overset{{\color{white}.}}{\in} \textnormal{GL}(n+1,\Re)
		\;\left\vert\;\,
			A^{T} \cdot Q_{(1,n)} \cdot A = Q_{(1,n)}
			\right.
		\;\right\}
\end{equation*}
\end{definition}

\vskip 0.5cm
\begin{proposition}
\mbox{}
\vskip 0.1cm
\noindent
For each \,$A \in \textnormal{O}(1,n)$, we have:
\begin{enumerate}
\item
	$\det(A) = \pm 1$,\, and
\item
	$\vert\,A^{0}_{0}\,\vert^{2} \;\geq\; 1$;\, equivalently, either \,$A^{0}_{0} \,\geq\, 1$\, or \,$A^{0}_{0} \,\leq\, -1$
\end{enumerate}
\end{proposition}
\proof
\begin{enumerate}
\item
\item
\end{enumerate}
\qed

\vskip 0.5cm
\begin{definition}[$\SO(1,n)$\, and \,$\SO^{\uparrow}(1,n)$]
\mbox{}
\vskip 0.1cm
\noindent
The \textbf{proper Lorentz} group is defined to be:
\begin{equation*}
\SO(1,n)
\; := \;
	\left\{\;\,
		A \overset{{\color{white}.}}{\in} \GL(n+1,\Re)
		\;\left\vert\;\,
			\begin{array}{ccc}
			A^{T} \cdot Q_{(1,n)} \cdot A \,=\, Q_{(1,n)}\,,
			\\
			\textnormal{det}(A) \,\overset{{\color{white}1}}{=}\, 1
			\end{array}
			\right.
		\;\right\}
\end{equation*}
The \textbf{proper orthochronous Lorentz} group is defined to be:
\begin{equation*}
\SO^{\uparrow}(1,n)
\; := \;
	\left\{\;\,
		A \overset{{\color{white}.}}{\in} \GL(n+1,\Re)
		\;\left\vert\;\,
			\begin{array}{ccc}
			A^{T} \cdot Q_{(1,n)} \cdot A \,=\, Q_{(1,n)}\,,
			\\
			\textnormal{det}(A) \,\overset{{\color{white}1}}{=}\, 1\,,
			\\
			A^{0}_{0} \,\overset{{\color{white}1}}{\geq}\, 0
			\end{array}
			\right.
		\;\right\}
\end{equation*}
\end{definition}

          %%%%% ~~~~~~~~~~~~~~~~~~~~ %%%%%

          %%%%% ~~~~~~~~~~~~~~~~~~~~ %%%%%

          %%%%% ~~~~~~~~~~~~~~~~~~~~ %%%%%
