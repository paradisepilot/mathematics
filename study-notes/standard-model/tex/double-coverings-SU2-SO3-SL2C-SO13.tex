
          %%%%% ~~~~~~~~~~~~~~~~~~~~ %%%%%

\chapter{Two double coverings: $\SU(2) \longrightarrow \SO(3)$ and $\SL(2,\C) \longrightarrow \SOup(1,3)$}
\setcounter{theorem}{0}
\setcounter{equation}{0}

%\cite{vanDerVaart1996}
%\cite{Kosorok2008}

%\renewcommand{\theenumi}{\alph{enumi}}
%\renewcommand{\labelenumi}{\textnormal{(\theenumi)}$\;\;$}
\renewcommand{\theenumi}{\roman{enumi}}
\renewcommand{\labelenumi}{\textnormal{(\theenumi)}$\;\;$}

          %%%%% ~~~~~~~~~~~~~~~~~~~~ %%%%%

\section{The Lie groups  \,$\SL(n,\C)$,\, $\SO(n)$\, and \,$\SU(n)$}

%\vskip 0.5cm
%$\SU(2) \overset{\textnormal{\scriptsize 2\textnormal{$:$}1{\color{white}.}}}{\longrightarrow} \SOup(1,3)$

          %%%%% ~~~~~~~~~~~~~~~~~~~~ %%%%%

\vskip 0.3cm
\begin{definition}
\mbox{}
\vskip 0.1cm
\noindent
The \textbf{special linear group of degree $n$ over $\C$} is defined as follows:
\begin{equation*}
\SL(n,\C)
\; := \;
	\left\{\;\,
		g \overset{{\color{white}.}}{\in} \textnormal{GL}(n,\C)
		\;\,\left\vert\;\;
			\det(\,g\,) \overset{{\color{white}1}}{=} 1
			\right.
		\;\right\}
\end{equation*}
\end{definition}

\vskip 0.5cm
\begin{proposition}[Set-theoretic characterizations of \,$\mathfrak{sl}(n,\C)$]
\label{SetTheoreticCharacterizationOfslTwoC}
\mbox{}
\vskip -0.1cm
\begin{equation*}
\mathfrak{sl}(n,\C)
\; = \;
	\left\{\;
		X \,\in\, \mathfrak{gl}(n,\C) \,=\, \C^{n \times n}
		\;\left\vert\;\,
			\textnormal{trace}(X) = \overset{{\color{white}1}}{0}
			\right.
		\,\right\}
\end{equation*}
\end{proposition}
\proof
We invoke the fact that \,$\det(e^{\,t\,\cdot\,X}) \,=\, e^{\,t\,\cdot\,\textnormal{trace}(X)}$,
for each \,$X \in \C^{n \times n}$.
Thus,
\begin{eqnarray*}
&&
	X \,\in\, \mathfrak{sl}(n,\C)
	\quad\Longrightarrow\quad
	e^{\,t\cdot\,X} \,\in\, \textnormal{SL}(n,\C)
	\quad\Longrightarrow\quad
	\det\!\left(\,e^{\,t\cdot\,X}\,\right) \,=\, 1
\\
& \Longrightarrow\quad &
	\textnormal{trace}(X)
	\; = \;
		\left.\dfrac{\d}{\d\,t}\right\vert_{t=0}\left(\,\overset{{\color{white}1}}{e^{\,t\,\cdot\,\textnormal{trace}(X)}}\,\right)
	\; = \;
		\left.\dfrac{\d}{\d\,t}\right\vert_{t=0}\left(\,\overset{{\color{white}1}}{\det(e^{\,t\,\cdot\,X})}\,\right)
	\; = \;
		\left.\dfrac{\d}{\d\,t}\right\vert_{t=0}\left(\,\overset{{\color{white}1}}{1}\,\right)
	\; = \;
		0
\end{eqnarray*}
Conversely, suppose \,$\textnormal{trace}(X) = 0$.\,
Then, \,$\det(e^{\,t\,\cdot\,X}) \,=\, e^{\,t\,\cdot\,\textnormal{trace}(X)} \,=\, e^{\,t\,\cdot\,0} \,=\, 1$,\,
which implies that \,$e^{\,t\,\cdot\,X} \,\in\, \textnormal{SL}(n,\C)$,\, hence \,$X \,\in\, \mathfrak{sl}(n,\C)$.
This completes the proof of the equality (of sets) in question.
\qed

          %%%%% ~~~~~~~~~~~~~~~~~~~~ %%%%%

\vskip 1.0cm
\begin{definition}
\mbox{}
\vskip 0.1cm
\noindent
The \textbf{orthogonal group of degree $n$} is defined as follows:
\begin{equation*}
\textnormal{O}(n)
\; := \;
	\left\{\;\,
		g \overset{{\color{white}.}}{\in} \textnormal{GL}(n,\Re)
		\;\left\vert\;\,
			g^{T} \cdot g = I_{n}
			\right.
		\;\right\}
\end{equation*}
The \textbf{special orthogonal group of degree $n$} is defined as follows:
\begin{equation*}
\textnormal{SO}(n)
\; := \;
	\left\{\;\,
		g \overset{{\color{white}.}}{\in} \textnormal{GL}(n,\Re)
		\;\left\vert\;\,
			g^{T} \cdot g = I_{n}\,,
			\;
			\textnormal{det}(g) = 1
			\right.
		\;\right\}
\end{equation*}
\end{definition}

\vskip 0.5cm
\begin{proposition}[Set-theoretic characterizations of the Lie algebras of $\textnormal{O}(n)$ and $\textnormal{SO}(n)$]
\begin{eqnarray*}
\mathfrak{o}(n)
& = &
	\left\{\;\,
		X \overset{{\color{white}.}}{\in} \gl(n,\Re)
		\;\left\vert\;\,
			X^{T} = -X
			\right.
		\;\right\}
\\
\so(n)
& = &
	\left\{\;\,
		X \overset{{\color{white}.}}{\in} \gl(n,\Re)
		\;\left\vert\;\,
			X^{T} = -X\,,
			\;
			\textnormal{trace}(X) = 0
			\right.
		\;\right\}
\end{eqnarray*}
\end{proposition}

          %%%%% ~~~~~~~~~~~~~~~~~~~~ %%%%%

\vskip 0.5cm
\begin{definition}
\mbox{}
\vskip 0.1cm
\noindent
The \textbf{unitary group of degree $n$} is defined as follows:
\begin{equation*}
\textnormal{U}(n)
\; := \;
	\left\{\;\,
		g \overset{{\color{white}.}}{\in} \textnormal{GL}(n,\C)
		\;\;\left\vert\;\;
			g^{\dagger} \cdot g \overset{{\color{white}1}}{=} I_{n}
			\right.
		\;\right\}
\end{equation*}
where \,$g^{\dagger}$\, is the conjugate transpose of
\,$g \in \textnormal{GL}(n,\C)$\,
and
\,$I_{n} \in \textnormal{GL}(n,\C)$\,
is the identity matrix.
\vskip 0.1cm
\noindent
The \textbf{special unitary group of degree $n$} is defined as follows:
\begin{equation*}
\textnormal{SU}(n)
\; := \;
	\left\{\;\,
		g \overset{{\color{white}.}}{\in} \textnormal{GL}(n,\C)
		\;\,\left\vert\;\;
			g^{\dagger} \cdot g \overset{{\color{white}1}}{=} I_{n}\,,
			\;
			\textnormal{det}(g) = 1
			\right.
		\;\right\}
\end{equation*}
\end{definition}

          %%%%% ~~~~~~~~~~~~~~~~~~~~ %%%%%

\vskip 0.5cm
\begin{proposition}[Set-theoretic characterizations of \,$\mathfrak{u}(n)$, and $\mathfrak{su}(n)$]
\mbox{}
\vskip 0.1cm
\begin{enumerate}
\item
	\begin{equation*}
	\mathfrak{u}(n)
	\; = \;
		\left\{\;
			X \,\in\, \mathfrak{gl}(n,\C) \,=\, \C^{n \times n}
			\;\left\vert\;\,
				X + X^{\dagger} = \overset{{\color{white}1}}{0}
				\right.
			\,\right\}
	\end{equation*}
\item
	\begin{equation*}
	\mathfrak{su}(n)
	\; = \;
		\left\{\;
			X \,\in\, \mathfrak{gl}(n,\C) \,=\, \C^{n \times n}
			\;\left\vert\;\,
				\begin{array}{c}
				X + X^{\dagger} = \overset{{\color{white}1}}{0}
				\\
				\textnormal{trace}(X) = \overset{{\color{white}1}}{0}
				\end{array}
				\right.
			\,\right\}
	\end{equation*}
\end{enumerate}
\end{proposition}
\proof
\begin{enumerate}
\item
	\begin{eqnarray*}
	&&
		X \,\in\, \mathfrak{u}(n)
		\quad\Longrightarrow\quad
		e^{\,t\,\cdot\,X} \,\in\, \textnormal{U}(n)
	\\
	& \Longrightarrow\quad &
		I_{n}
			\,=\, \left(\,e^{\,t\,\cdot\,X}\,\right)^{\!\dagger} \cdot \left(\,e^{\,t\,\cdot\,X}\,\right)
			\,=\, \left(\,e^{\,t\,\cdot\,X^{\dagger}}\,\right) \cdot \left(\,e^{\,t\,\cdot\,X}\,\right)
			\,=\, e^{\,t\,\cdot\,(X^{\dagger}+X)}
	\\
	& \Longrightarrow\quad &
		X \,+\, X^{\dagger}
		\; = \;
			\left.\dfrac{\d}{\d\,t}\right\vert_{t=0}\left(\,\overset{{\color{white}1}}{e^{\,t\,\cdot\,(X+X^{\dagger})}}\,\right)
		\; = \;
			\left.\dfrac{\d}{\d\,t}\right\vert_{t=0}\left(\,\overset{{\color{white}1}}{I_{n}}\,\right)
		\; = \;
			0
	\end{eqnarray*}
	Conversely, suppose \,$X + X^{\dagger} \,=\, 0$.\,
	Then, \,$I_{n}$
	\,$=$\, $e^{\,0_{n \times n}}$
	\,$=$\, $e^{\,t\,\cdot(X^{\dagger}+X)}$
	\,$=\, \cdots \,=$\, $\left(e^{\,t\,\cdot\,X}\right)^{\!\dagger}\cdot\left(e^{\,t\,\cdot\,X}\right)$,\,
	which implies that \,$e^{\,t\,\cdot\,X} \,\in\, \textnormal{U}(n)$,\, hence \,$X \,\in\, \mathfrak{u}(n)$.
	This completes the proof of the equality (of sets) in question.
\item
	Immediate by (i) and Proposition \ref{SetTheoreticCharacterizationOfslTwoC}.
	\qed
\end{enumerate}

          %%%%% ~~~~~~~~~~~~~~~~~~~~ %%%%%

\section{The Lie groups \,$\SO(3)$,\, $\SU(2)$,\, $\SOup(1,3)$,\, and $\SL(2,\C)$}

          %%%%% ~~~~~~~~~~~~~~~~~~~~ %%%%%

\vskip 0.5cm
\begin{proposition}[Parametrization of \,$\textnormal{SU}(2)$]
\mbox{}
\vskip 0.1cm
\noindent
$\textnormal{SU}(2)$ admits the following parametrization:
\begin{equation*}
\textnormal{SU}{(2)}
\; := \;
	\left\{\,
		\left.
		\left(\begin{array}{rr}
		a & -\overline{b}
		\\
		\overset{{\color{white}-}}{b} & \overline{a}
		\end{array}\right)
		\overset{{\color{white}.}}{\in}
		\C^{2 \times 2}
		\;\;\right\vert\;\,
			\vert\, a \,\vert^{2} \,+\, \vert\, b \,\vert^{2} \,=\, 1
		\;\right\}
\end{equation*}
Hence, the (real) Lie group {\color{red}$\textnormal{SU}(2)$ is diffeomorphic to $S^{3}$}, the $3$-dimensional unit sphere
(in $4$-dimensional Euclidean space).
In particular, $\textnormal{SU}(2)$ is a {\color{red}simply connected} $3$-dimensional real manifold.
\end{proposition}
\proof
Suppose:
\begin{equation*}
g
\; = \;
	\left(\begin{array}{cc}
		a & c
		\\
		\overset{{\color{white}-}}{b} & d
		\end{array}\right)
\; \in \;
\textnormal{SU}(2)
\end{equation*}
First note that the component form of the condition \,$g^{\dagger}\cdot g = I_{2}$\, is:
\begin{equation*}
\left(\begin{array}{cc}
	1 & 0
	\\
	\overset{{\color{white}-}}{0} & 1
	\end{array}\right)
\; = \;
	g^{\dagger} \cdot g
\; = \;
	\left(\begin{array}{cc}
		\overline{a} & \overline{b}
		\\
		\overset{{\color{white}-}}{\overline{c}} & \overline{d}
		\end{array}\right)
	\cdot
	\left(\begin{array}{cc}
		a & c
		\\
		\overset{{\color{white}-}}{b} & d
		\end{array}\right)
\; = \;
	\left(\begin{array}{cc}
		a\overline{a} + b\overline{b} & \overline{a}c + \overline{b}d
		\\
		\overset{{\color{white}-}}{a\overline{c} + b\overline{d}} & c\overline{c} + d\overline{d}
		\end{array}\right)
\end{equation*}
Thus, we see that
\begin{equation*}
g
\; = \;
	\left(\begin{array}{cc}
		a & c
		\\
		\overset{{\color{white}-}}{b} & d
		\end{array}\right)
\;\in\;
	\textnormal{SU}(2)
\quad\Longleftrightarrow\quad
\left\{
	\begin{array}{ccc}
		g^{\dagger} \cdot g &=& I_{2}
		\\
		\det(g) &=& \overset{{\color{white}1}}{1}
		\end{array}
		\right.
\quad\Longleftrightarrow\quad
\left\{
	\begin{array}{ccc}
	\vert\,a\,\vert^{2} + \vert\,b\,\vert^{2} &=& 1
	\\
	\vert\,c\,\vert^{2} + \vert\,d\,\vert^{2} &\overset{{\color{white}1}}{=}& 1
	\\
	a\overline{c} \;\, + \,\; b\overline{d} &\overset{{\color{white}1}}{=}& 0
	\\
	ad \;\, - \,\; bc &\overset{{\color{white}1}}{=}& 1
	\end{array}
	\right.
\end{equation*}
Next, note that
\begin{equation*}
a\overline{c} + b\overline{d} = 0
\quad\Longleftrightarrow\quad
	\left\langle
		\left(\begin{array}{c} a \\ b \end{array}\right)
		\,,\,
		\left(\begin{array}{c} c \\ d \end{array}\right)
		\right\rangle_{\C^{2}}
	\;=\;
	0
\end{equation*}
Since
\,$\dim_{\C}\left(\begin{array}{c} a \\ b \end{array}\right)^{\perp} =\, 1$,\,
the above equality (i.e., orthogonality of the two columns of \,$g$) implies:
\begin{equation*}
\left(\begin{array}{c} c \\ d \end{array}\right)
\; \in \;
	\left(\begin{array}{c} a \\ b \end{array}\right)^{\perp}
\; = \;
	\textnormal{span}_{\C}\left\{
		\left(\begin{array}{r} -\overline{b} \\ \overline{a} \end{array}\right)
		\right\}
\quad\Longleftrightarrow\quad
\left(\begin{array}{c} c \\ d \end{array}\right)
\; = \;
	\lambda \left(\begin{array}{r} -\overline{b} \\ \overline{a} \end{array}\right),
	\;\;
	\textnormal{for some $\lambda \in \C$}
\end{equation*}
So, we now know that $g$ has the form:
\begin{equation*}
g
\; = \;
	\left(\begin{array}{rr}
		a & -\lambda\,\overline{b}
		\\
		\overset{{\color{white}-}}{b} & \lambda\,\overline{a}
		\end{array}\right)
\end{equation*}
Next,
\begin{equation*}
1
\,=\, \det(g)
\,=\, a\cdot(\lambda\,\overline{a}) - b \cdot (-\lambda\,\overline{b})
\,=\, \lambda\cdot(\vert\,a\,\vert^{2} + \vert\,b\,\vert^{2})
\quad\Longrightarrow\quad
	\lambda = 1
\end{equation*}
We may now conclude that
\begin{equation*}
g
\; = \;
	\left(\begin{array}{rr}
		a & -\,\overline{b}
		\\
		\overset{{\color{white}-}}{b} & \overline{a}
		\end{array}\right),\,
\quad
\textnormal{where \,$\vert\,a\,\vert^{2} + \vert\,b\,\vert^{2} = 1$}
\end{equation*}
This completes the proof of the Proposition.
\qed

          %%%%% ~~~~~~~~~~~~~~~~~~~~ %%%%%

%
          %%%%% ~~~~~~~~~~~~~~~~~~~~ %%%%%

\section{Irreducible complex representations of a real Lie algebra versus those of its complexification}
\setcounter{theorem}{0}
\setcounter{equation}{0}

%\cite{vanDerVaart1996}
%\cite{Kosorok2008}

%\renewcommand{\theenumi}{\alph{enumi}}
%\renewcommand{\labelenumi}{\textnormal{(\theenumi)}$\;\;$}
\renewcommand{\theenumi}{\roman{enumi}}
\renewcommand{\labelenumi}{\textnormal{(\theenumi)}$\;\;$}

          %%%%% ~~~~~~~~~~~~~~~~~~~~ %%%%%

\vskip 0.3cm
\begin{remark}
\mbox{}
\vskip 0.05cm
\noindent
Intuitively speaking, one could say that
the collection of irreducible finite-dimensional {\color{red}complex representations} of a {\color{red}real Lie algebra}
is the ``same'' as
the collection of irreducible finite-dimensional complex representations of its (Lie-algebra) {\color{red}complexification}.
The precise statement is stated as
Corollary \ref{ComplexIrrrepsOfARealLieAlgebraAreTheSameAsTheRealOnes}
\end{remark}

          %%%%% ~~~~~~~~~~~~~~~~~~~~ %%%%%

\vskip 0.5cm
\begin{definition}
\mbox{}
\vskip 0.05cm
\noindent
Let \,$V$\, be a finite-dimensional vector space over \,$\Re$.\,
Then, the \textbf{complexification} \,$V_{\C}$\, is
the (finite-dimensional) vector space \,$\C$\, obtained as follows:
\begin{itemize}
\item
	The underlying set (of vectors) of \,$V_{\C}$\, is the set of all formal linear combinations
	of the form:
	\begin{equation*}
	v_{1} \, + \, \i\cdot v_{2}\,,
	\end{equation*}
	where \,$v_{1}, v_{2} \in V$.\,
\item
	Vector addition in \,$V_{\C}$\, is defined as follows:
	\begin{equation*}
	\left(\;\overset{{\color{white}.}}{v_{1}} \, + \, \i\cdot v_{2}\,\right)
	\; + \;
	\left(\;\overset{{\color{white}.}}{w_{1}} \, + \, \i\cdot w_{2}\,\right)
	\;\; := \;\;
		\left(\,\overset{{\color{white}.}}{v_{1}} + w_{1} \,\right)
		\; + \;
		\i\cdot\left(\,\overset{{\color{white}.}}{v_{2}} + w_{2} \,\right)
	\end{equation*}	
\item
	Complex scalar multiplication on \,$V_{\C}$\, is defined as follows:
	\begin{equation*}
	(\,a + \i \, b\,) \cdot \left(\;\overset{{\color{white}.}}{v_{1}} \, + \, \i\cdot v_{2}\,\right)
	\;\; := \;\;
		\left(\,a\cdot\overset{{\color{white}.}}{v_{1}} - b \cdot v_{2} \,\right)
		\; + \;
		\i\cdot\left(\,a\cdot\overset{{\color{white}.}}{v_{2}} + b \cdot v_{1} \,\right)
	\end{equation*}	
\end{itemize}
\end{definition}

          %%%%% ~~~~~~~~~~~~~~~~~~~~ %%%%%

\vskip 0.5cm
\begin{proposition}
\mbox{}
\vskip 0.05cm
\noindent
Let \,$\mathfrak{g}$\, be a finite-dimensional real Lie algebra and
\,$\mathfrak{g}_{\C}$\, its vector-space complexification.
Then, the Lie bracket of \,$\mathfrak{g}$\, has a unique extension to \,$\mathfrak{g}_{\C}$\,
that makes  \,$\mathfrak{g}_{\C}$\, into a complex Lie algebra.
The resulting complex Lie algebra -- still denoted as \,$\mathfrak{g}_{\C}$\, -- is called
the \textbf{complexification} of \,$\mathfrak{g}$.\,
\end{proposition}
\proof
The unique extension
\,$[\;\cdot\,,\,\cdot\,]_{\mathfrak{g}_{\C}}$\,
to \,$\mathfrak{g}_{\C}$\, of
\,$[\;\cdot\,,\,\cdot\,]_{\mathfrak{g}}$\,
is given by
\begin{equation*}
\left[\;
	X_{1} \overset{{\color{white}.}}{+} \i\cdot Y_{1}
	\;,\,
	X_{2} \overset{{\color{white}.}}{+} \i\cdot Y_{2}
	\,\right]_{\mathfrak{g}_{\C}}
\;\, := \;\;
	\left(\;
		[\,X_{1},X_{2}\,]_{\mathfrak{g}}
		\; \overset{{\color{white}1}}{-} \,
		[\,Y_{1},Y_{2}\,]_{\mathfrak{g}}
		\,\right)
	\; + \;
	\i \cdot\! \left(\;
		[\,X_{1},Y_{2}\,]_{\mathfrak{g}}
		\; \overset{{\color{white}1}}{+} \,
		[\,Y_{1},X_{2}\,]_{\mathfrak{g}}
		\,\right)
\end{equation*}
for each \,$X_{1}, Y_{1}, X_{2}, Y_{2} \in \mathfrak{g}$.
For the proof, see Proposition 3.37, p.65, \cite{Hall2015}.
\qed

          %%%%% ~~~~~~~~~~~~~~~~~~~~ %%%%%

\vskip 0.5cm
\begin{proposition}[Universal property of the complexification of a real Lie algebra]
\label{UniqueExtensionOfRealLieAlgebraHomomorphisms}
\mbox{}
\vskip 0.05cm
\noindent
Let \,$\mathfrak{g}$\, be a finite-dimensional real Lie algebra and
\,$\mathfrak{g}_{\C}$\, its Lie-algebra complexification.
Let \,$\mathfrak{h}$\, be an arbitrary complex Lie algebra.
Then, every real Lie algebra homomorphism \,$\pi$\,
from \,$\mathfrak{g}$\, into \,$\mathfrak{h}$\,
extends uniquely to a complex Lie algebra homomorphsim \,$\pi_{\C}$\,
from \,$\mathfrak{g}_{\C}$\, into \,$\mathfrak{h}$.\,
\end{proposition}
\proof
The unique extension \,$\pi_{\C}$\, is given by
\begin{equation*}
\pi_{\C}\!\left(\,X \overset{{\color{white}.}}{+} \i\cdot Y \right)
\;\, := \;\;
	\pi(X) \overset{{\color{white}.}}{+} \i\cdot \pi(Y)\,,
\quad
\textnormal{for each \,$X, Y \in \mathfrak{g}$}
\end{equation*}
For the proof, see Proposition 3.39, p.67, \cite{Hall2015}.
\qed

          %%%%% ~~~~~~~~~~~~~~~~~~~~ %%%%%

\begin{proposition}
\label{propnIrreducibleComplexRepresentationsOfRealLieAlgebras}
\mbox{}
\vskip 0.05cm
\noindent
Let \,$\mathfrak{g}$\, be a real Lie algebra and
\,$\mathfrak{g}_{\C} = \mathfrak{g} + \i\cdot\mathfrak{g}$\,
its Lie-algebra complexification.
Let \,$V$\, be an arbitrary finite-dimensional complex vector space,
and \,$\gl(V)$\, the complex Lie algebra of all complex-linear maps from \,$V$\, into itself.
Then, the following statements are true:
\begin{enumerate}
\item
	Every complex Lie algebra homomorphism
	\,$\rho : \mathfrak{g}_{\C} \longrightarrow \gl(V)$\,
	induces a real Lie algebra homomorphism
	\,$\rho_{\,\Re} : \mathfrak{g} \longrightarrow \gl(V)$\,
	simply by restricting the domain of
	\,$\rho$\, from \,$\mathfrak{g}_{\C}$\, to \,$\mathfrak{g}$,\,
	and restricting the scalar field from \,$\C$\, to \,$\Re$.\, 
\item
	Every real Lie algebra homomorphism
	\,$\pi : \mathfrak{g} \longrightarrow \gl(V)$\,
	has a unique extension to a complex Lie algebra homomorphism
	\,$\pi_{\C} : \mathfrak{g}_{\C} \longrightarrow \gl(V)$.\,
\item
	$\pi$\, is irreducible if and only if \,$\pi_{\C}$\, is irreducible.
\end{enumerate}
\end{proposition}
\proof
\begin{enumerate}
\item
	Trivial.
\item
	Immediate by Proposition \ref{UniqueExtensionOfRealLieAlgebraHomomorphisms}.
\item
	Recall that a representation of (real or complex) Lie algebra is irreducible if it has
	no non-trivial invariant subspaces.
	Thus, the equivalence of the irreducibility of \,$\pi$\, and that of \,$\pi_{\C}$\,
	will follow from the fact that the two representations share the same collection
	of invariant subspaces. In other words, it suffices to establish the following:
	\vskip 0.2cm
	\noindent
	\textbf{Claim 1:}\quad
	Suppose \,$\{\,0\,\} \subsetneq W \subsetneq V$\, is a proper (complex) subspace of \,$V$.\,
	Then, \,$W$\, is \,$\pi$-invariant if and only if it is \,$\pi_{\C}$-invariant. 
	\vskip 0.1cm
	\noindent
	Proof of Claim 1:\quad
	Let \,$w \in W$\, be an arbitrary element of \,$W$.\,
	Suppose first that \,$W$\, is \,$\pi$-invariant.
	Then, for each \,$X, Y \in \mathfrak{g}$,\, we have:
	\begin{eqnarray*}
	\pi_{\C}\!\left(\,X \overset{{\color{white}.}}{+} \i\cdot Y \right) \cdot [\,w\,]
	& = &
		\left(\,\pi(X) \overset{{\color{white}.}}{+} \i\cdot \pi(Y) \right) \cdot [\,w\,]
	\\
	& = &
		\pi(X) \cdot w \;\overset{{\color{white}.}}{+}\; \i\cdot \pi(Y) \cdot w
	\;\; \in \;\;
		W \; + \; \i \cdot W
	\;\; = \;\;
		W,
	\end{eqnarray*}
	which proves that \,$W$\, is also \,$\pi_{\C}$-invariant.
	Conversely, now suppose that \,$W$\, is \,$\pi_{\C}$-invariant.
	Then, for each \,$X \in \mathfrak{g}$,\, we have:
	\begin{equation*}
	\pi(X)\cdot w
	\;\; = \;\;
		\left(\,\pi(X) \overset{{\color{white}.}}{+} \i\cdot \pi(0_{\mathfrak{g}}) \right) \cdot [\,w\,]
	\;\; = \;\;
		\pi_{\C}\!\left(\,X \overset{{\color{white}.}}{+} \i\cdot 0_{\mathfrak{g}} \right) \cdot [\,w\,]
	\;\; \in \;\;
		W,
	\end{equation*}
	which proves that \,$W$\, is also \,$\pi$-invariant.
	This proves Claim 1, as well as completes the proof of the Proposition.
	\qed
\end{enumerate}

\vskip 0.5cm
\begin{corollary}
\label{ComplexIrrrepsOfARealLieAlgebraAreTheSameAsTheRealOnes}
\mbox{}
\vskip 0.05cm
\noindent
Let \,$\mathfrak{g}$\, be a real Lie algebra and
\,$\mathfrak{g}_{\C} = \mathfrak{g} + \i\cdot\mathfrak{g}$\,
its Lie-algebra complexification.
Then, there is a bijection between their respective collections of finite-dimensional complex representations:
\begin{equation*}
\left\{\begin{array}{c}
	\textnormal{{\color{red}real} Lie algebra homomorphisms}
	\\
	{\color{red}\mathfrak{g}} \longrightarrow \gl(V),
	\\
	\textnormal{where $V$ is a finite-dimensional}
	\\
	\textnormal{\textbf{complex} vector space}
	\end{array}\right\}
\quad
\begin{array}{c}
\overset{\underset{{\color{white}.}}{\textnormal{extension}}}{\textnormal{\Huge$\longrightarrow$}}
\\
\underset{{\textnormal{restriction}}}{\textnormal{\Huge$\longleftarrow$}}
\end{array}
\quad
\left\{\begin{array}{c}
	\textnormal{{\color{red}complex} Lie algebra homomorphisms}
	\\
	{\color{red}\mathfrak{g}_{\C}} \longrightarrow \gl(V),
	\\
	\textnormal{where $V$ is a finite-dimensional}
	\\
	\textnormal{\textbf{complex} vector space}
	\end{array}\right\}
\end{equation*}
Furthermore, the above bijection preserves irreducibility.
\end{corollary}


          %%%%% ~~~~~~~~~~~~~~~~~~~~ %%%%%

%\vskip 0.5cm
%
          %%%%% ~~~~~~~~~~~~~~~~~~~~ %%%%%

\section{Irreducible finite-dimensional complex representations of \,$\mathfrak{su}(2)$}
\setcounter{theorem}{0}
\setcounter{equation}{0}

%\cite{vanDerVaart1996}
%\cite{Kosorok2008}

%\renewcommand{\theenumi}{\alph{enumi}}
%\renewcommand{\labelenumi}{\textnormal{(\theenumi)}$\;\;$}
\renewcommand{\theenumi}{\roman{enumi}}
\renewcommand{\labelenumi}{\textnormal{(\theenumi)}$\;\;$}

          %%%%% ~~~~~~~~~~~~~~~~~~~~ %%%%%

\vskip 0.3cm
\begin{remark}
\mbox{}
\vskip 0.1cm
\noindent
The (real) Lie group $\SO(3)$ has physical significance --
it is the rotation group of $3$-dimensional Euclidean space, and
it also shows up in the study of the Lorentz group and the Poincaré group.
The (double) universal covering space of $\SO(3)$ is (the simply connected real Lie group) $\SU(2)$.
We are thus interested in classifying the irreducible finite-dimensional complex representations of $\su(2)$.
By Corollary \ref{ComplexIrrrepsOfARealLieAlgebraAreTheSameAsTheRealOnes},
it is equivalent to classify the irreducible finite-dimensional complex representations of $\su(2) \otimes_{\Re} \C$.
Passing to the complexification will enable us to work with more convenient bases of Lie algebras.
\end{remark}

          %%%%% ~~~~~~~~~~~~~~~~~~~~ %%%%%

\vskip 0.5cm
\begin{proposition}[Set-theoretic characterizations of \,$\mathfrak{sl}(n,\C)$]
\label{SetTheoreticCharacterizationOfslTwoC}
\mbox{}
\vskip -0.1cm
\begin{equation*}
\mathfrak{sl}(n,\C)
\; = \;
	\left\{\;
		X \,\in\, \mathfrak{gl}(n,\C) \,=\, \C^{n \times n}
		\;\left\vert\;\,
			\textnormal{trace}(X) = \overset{{\color{white}1}}{0}
			\right.
		\,\right\}
\end{equation*}
\end{proposition}
\proof
We invoke the fact that \,$\det(e^{\,t\,\cdot\,X}) \,=\, e^{\,t\,\cdot\,\textnormal{trace}(X)}$,
for each \,$X \in \C^{n \times n}$.
Thus,
\begin{eqnarray*}
&&
	X \,\in\, \mathfrak{sl}(n,\C)
	\quad\Longrightarrow\quad
	e^{\,t\cdot\,X} \,\in\, \textnormal{SL}(n,\C)
	\quad\Longrightarrow\quad
	\det\!\left(\,e^{\,t\cdot\,X}\,\right) \,=\, 1
\\
& \Longrightarrow\quad &
	\textnormal{trace}(X)
	\; = \;
		\left.\dfrac{\d}{\d\,t}\right\vert_{t=0}\left(\,\overset{{\color{white}1}}{e^{\,t\,\cdot\,\textnormal{trace}(X)}}\,\right)
	\; = \;
		\left.\dfrac{\d}{\d\,t}\right\vert_{t=0}\left(\,\overset{{\color{white}1}}{\det(e^{\,t\,\cdot\,X})}\,\right)
	\; = \;
		\left.\dfrac{\d}{\d\,t}\right\vert_{t=0}\left(\,\overset{{\color{white}1}}{1}\,\right)
	\; = \;
		0
\end{eqnarray*}
Conversely, suppose \,$\textnormal{trace}(X) = 0$.\,
Then, \,$\det(e^{\,t\,\cdot\,X}) \,=\, e^{\,t\,\cdot\,\textnormal{trace}(X)} \,=\, e^{\,t\,\cdot\,0} \,=\, 1$,\,
which implies that \,$e^{\,t\,\cdot\,X} \,\in\, \textnormal{SL}(n,\C)$,\, hence \,$X \,\in\, \mathfrak{sl}(n,\C)$.
This completes the proof of the equality (of sets) in question.
\qed

\vskip 1.0cm
\begin{proposition}[Set-theoretic characterizations of \,$\mathfrak{u}(n)$, and $\mathfrak{su}(n)$]
\mbox{}
\vskip 0.1cm
\begin{enumerate}
\item
	\begin{equation*}
	\mathfrak{u}(n)
	\; = \;
		\left\{\;
			X \,\in\, \mathfrak{gl}(n,\C) \,=\, \C^{n \times n}
			\;\left\vert\;\,
				X + X^{\dagger} = \overset{{\color{white}1}}{0}
				\right.
			\,\right\}
	\end{equation*}
\item
	\begin{equation*}
	\mathfrak{su}(n)
	\; = \;
		\left\{\;
			X \,\in\, \mathfrak{gl}(n,\C) \,=\, \C^{n \times n}
			\;\left\vert\;\,
				\begin{array}{c}
				X + X^{\dagger} = \overset{{\color{white}1}}{0}
				\\
				\textnormal{trace}(X) = \overset{{\color{white}1}}{0}
				\end{array}
				\right.
			\,\right\}
	\end{equation*}
\end{enumerate}
\end{proposition}
\proof
\begin{enumerate}
\item
	\begin{eqnarray*}
	&&
		X \,\in\, \mathfrak{u}(n)
		\quad\Longrightarrow\quad
		e^{\,t\,\cdot\,X} \,\in\, \textnormal{U}(n)
	\\
	& \Longrightarrow\quad &
		I_{n}
			\,=\, \left(\,e^{\,t\,\cdot\,X}\,\right)^{\!\dagger} \cdot \left(\,e^{\,t\,\cdot\,X}\,\right)
			\,=\, \left(\,e^{\,t\,\cdot\,X^{\dagger}}\,\right) \cdot \left(\,e^{\,t\,\cdot\,X}\,\right)
			\,=\, e^{\,t\,\cdot\,(X^{\dagger}+X)}
	\\
	& \Longrightarrow\quad &
		X \,+\, X^{\dagger}
		\; = \;
			\left.\dfrac{\d}{\d\,t}\right\vert_{t=0}\left(\,\overset{{\color{white}1}}{e^{\,t\,\cdot\,(X+X^{\dagger})}}\,\right)
		\; = \;
			\left.\dfrac{\d}{\d\,t}\right\vert_{t=0}\left(\,\overset{{\color{white}1}}{I_{n}}\,\right)
		\; = \;
			0
	\end{eqnarray*}
	Conversely, suppose \,$X + X^{\dagger} \,=\, 0$.\,
	Then, \,$I_{n}$
	\,$=$\, $e^{\,0_{n \times n}}$
	\,$=$\, $e^{\,t\,\cdot(X^{\dagger}+X)}$
	\,$=\, \cdots \,=$\, $\left(e^{\,t\,\cdot\,X}\right)^{\!\dagger}\cdot\left(e^{\,t\,\cdot\,X}\right)$,\,
	which implies that \,$e^{\,t\,\cdot\,X} \,\in\, \textnormal{U}(n)$,\, hence \,$X \,\in\, \mathfrak{u}(n)$.
	This completes the proof of the equality (of sets) in question.
\item
	Immediate by (i) and Proposition \ref{SetTheoreticCharacterizationOfslTwoC}.
	\qed
\end{enumerate}

          %%%%% ~~~~~~~~~~~~~~~~~~~~ %%%%%

\vskip 0.5cm
\begin{proposition}[Some {\color{red}convenient generators} of \,$\mathfrak{su}(2)$\, and \,$\su(2) \otimes_{\Re} \C$]
\label{propnUsefulGeneratorsLittleSU2}
\mbox{}
\vskip 0.1cm
\noindent
Let \,$\sigma_{1},\, \sigma_{2},\, \sigma_{3} \,\in\, \C^{2 \times 2}$\, be the \textbf{Pauli spin matrices}, i.e.,
\begin{equation*}
\sigma_{1} \,=\, \sigma_{x} \,:=\, \left(\begin{array}{cc} 0 & 1 \\ 1 & 0 \end{array}\right),
\quad
\sigma_{2} \,=\, \sigma_{y} \,:=\, \left(\begin{array}{rr} 0 & -\i \\ \i & 0 \end{array}\right),
\quad
\sigma_{3} \,=\, \sigma_{z} \,:=\, \left(\begin{array}{rr} 1 & 0 \\ 0 & -1 \end{array}\right).
\end{equation*}
Define \,$J_{1},\, J_{2},\, J_{3},\, S_{1},\, S_{2},\, S_{3},\, S_{+},\, S_{-} \,\in\, \C^{2 \times 2}$\, as follows:
\begin{equation*}
Y_{1} \,:=\, \dfrac{\i}{2}\cdot\sigma_{1} \,=\, \dfrac{\i}{2}\cdot\left(\begin{array}{cc} 0 & 1 \\ 1 & 0 \end{array}\right),
\quad
Y_{2} \,:=\, \mathbf{{\color{red}-}}\,\dfrac{\i}{2}\cdot\sigma_{2} \,=\, \dfrac{1}{2}\cdot\left(\begin{array}{rr} 0 & -1 \\ 1 & 0 \end{array}\right),
\quad
Y_{3} \,:=\, \dfrac{\i}{2}\cdot\sigma_{3} \,=\, \dfrac{\i}{2}\cdot\left(\begin{array}{rr} 1 & 0 \\ 0 & -1 \end{array}\right),
\end{equation*}
\begin{equation*}
S_{1} \,:=\, \i \cdot Y_{1} \,=\, \dfrac{-1}{2} \cdot \left(\begin{array}{rr} 0 & 1 \\ 1 & 0 \end{array}\right),
\quad
S_{2} \,:=\, \i \cdot Y_{2} \,=\, \dfrac{\i}{2} \cdot \left(\begin{array}{rr} 0 & -1 \\ 1 & 0 \end{array}\right),
\quad
S_{3} \,:=\, \i \cdot Y_{3} \,=\, \dfrac{1}{2}\cdot\left(\begin{array}{rr} -1 & 0 \\ 0 & 1 \end{array}\right).
\end{equation*}
\begin{equation*}
S_{+} \; := \; S_{1} + \i\,S_{2} \; = \; \left(\begin{array}{rr} 0 & 0 \\ -1 & 0 \end{array}\right),
\quad\quad
S_{-} \; := \; S_{1} - \i\,S_{2} \; = \; \left(\begin{array}{rr} 0 & -1 \\ 0 & 0 \end{array}\right).
\end{equation*}
Then, the following statements are true:
\begin{enumerate}
\item
	$Y_{1},\, Y_{2},\, Y_{3} \,\in\, \mathfrak{su}(2)$.\,
	$Y_{1},\, Y_{2},\, Y_{3}$\,
	form a set of generators for the (real) Lie algebra \,$\mathfrak{su}(2)$\, of the (real) Lie group \,$\textnormal{SU}(2)$.\,
	$Y_{1},\, Y_{2},\, Y_{3}$\, satisfy the following commutation relations:
	\begin{equation*}
	\left[\,Y_{a}\,,\,Y_{b}\,\right] \;\; = \;\; \overset{3}{\underset{c\,=\,1}{\sum}}\;\varepsilon_{abc}\,Y_{c}\,,
	\quad
	\textnormal{for \,$a, b = 1,2,3$}.
	\end{equation*}
\item
	$S_{1},\, S_{2},\, S_{3} \,\in\, \mathfrak{su}(2) \otimes_{\Re} \C$,\,
	where
	\,$\mathfrak{su}(2) \otimes_{\Re} \C$\,
	is the complexification of (the real Lie algebra)
	\,$\mathfrak{su}(2)$.\,
	\,$S_{1},\, S_{2},\, S_{3}$\,
	satisfy the following commutation relations:
	\begin{equation*}
	\left[\,S_{a}\,\overset{{\color{white}1}}{,}\,S_{b}\,\right]
	\;\; = \;\;
		\sqrt{-1}\,\cdot\overset{3}{\underset{c\,=\,1}{\sum}}\;\varepsilon_{abc}\cdot S_{c}\,,
	\quad
	\textnormal{for each \,$a, b \in \{\,1,2,3\,\}$}\,,
	\end{equation*}
	where \,$\varepsilon_{abc}$\, is the fully anti-symmetric tensor.
\item
	$S_{+},\, S_{-} \,\in\, \mathfrak{su}(2) \otimes_{\Re} \C$,\,
	and
	\,$S_{+},\; S_{-},\; S_{3}$\, satisfy the following commutation relations:
	\begin{equation*}
	\left[\,S_{3}\,,\,S_{\pm}\,\right] \, = \, \pm\,S_{\pm}\,,
	\quad
	\left[\,S_{+}\,,\,S_{-}\,\right] \, = \, 2\,S_{3}
	\end{equation*}
\item
	Suppose:
	\begin{itemize}
	\item
		$V$\, is a (not necessarily finite-dimensional) complex vector space,
	\item
		$\rho : \mathfrak{su}(2) \otimes_{\Re} \C \longrightarrow \gl(V)$\,
		is a complex Lie algebra representation, and
	\item	
		$v \in V$\, and \,$\lambda \in \C$\, together satisfy \,$\rho(S_{3})(v) = \lambda \cdot v$.
	\end{itemize}	
	Then, \,$\rho(S_{+})(v) \,\in\, V$\, satisfies:
	\begin{equation*}
	\rho(S_{3})\!\left(\,\rho(\overset{{\color{white}.}}{S}_{+})(v)\,\right)
	\; = \;
		(\lambda+1) \cdot \rho(S_{+})(v)\,
	\end{equation*}
	and
	\,$\rho(S_{-})(v) \,\in\, V$\, satisfies:
	\begin{equation*}
	\rho(S_{3})\!\left(\,\rho(\overset{{\color{white}.}}{S}_{-})(v)\,\right)
	\; = \;
		(\lambda-1) \cdot \rho(S_{-})(v)\,
	\end{equation*}
\end{enumerate}
\end{proposition}
\proof
The Proposition follows straightforwardly by direct computations.
\qed

          %%%%% ~~~~~~~~~~~~~~~~~~~~ %%%%%

\vskip 0.5cm
\begin{corollary}
\label{RaisingLoweringEigenvalues}
\mbox{}
\vskip 0.1cm
\noindent
Suppose:
\begin{itemize}
\item
	$V$\, is a (not necessarily finite-dimensional) complex vector space,
\item
	$\rho : \mathfrak{su}(2) \otimes_{\Re} \C \longrightarrow \gl(V)$\,
	is a complex Lie algebra representation, and
\item	
	$0 \neq v \in V$\, is an eigenvector of \,$\rho(S_{3})$\, with eigenvalue \,$\lambda \in \C$.\,
\end{itemize}
Then, the following statements are true:
\begin{enumerate}
\item
	Either
	\,$\rho(\overset{{\color{white}.}}{S}_{+})(v) = 0$\,
	or 
	\,$\rho(\overset{{\color{white}.}}{S}_{+})(v)$\,
	is an eigenvector of
	\,$\rho(S_{3})$\,
	with eigenvalue
	\,$\lambda + 1 \in \C$.\,
\item
	Either
	\,$\rho(\overset{{\color{white}.}}{S}_{-})(v) = 0$\,
	or 
	\,$\rho(\overset{{\color{white}.}}{S}_{-})(v)$\,
	is an eigenvector of
	\,$\rho(S_{3})$\,
	with eigenvalue
	\,$\lambda - 1 \in \C$.\,
\end{enumerate}
\end{corollary}
\proof 
\begin{enumerate}
\item
	By Proposition \ref{propnUsefulGeneratorsLittleSU2},
	we know that \,$\rho(S_{+})(v)$\, satisfies the following equality:
	\begin{equation*}
	\rho(S_{3})\!\left(\,\rho(\overset{{\color{white}.}}{S}_{+})(v)\,\right)
	\; = \;
		(\lambda+1) \cdot \rho(S_{+})(v),\,
	\end{equation*}
	from which the desired conclusion immediately follows.
\item
	Similarly, by Proposition \ref{propnUsefulGeneratorsLittleSU2},
	we know that \,$\rho(S_{-})(v)$\, satisfies the following equality:
	\begin{equation*}
	\rho(S_{3})\!\left(\,\rho(\overset{{\color{white}.}}{S}_{-})(v)\,\right)
	\; = \;
		(\lambda-1) \cdot \rho(S_{-})(v),\,
	\end{equation*}
	from which the desired conclusion immediately follows.
	\qed
\end{enumerate}

          %%%%% ~~~~~~~~~~~~~~~~~~~~ %%%%%

\vskip 0.5cm
\begin{corollary}
\label{EigenvectorOfS3KilledBySplus}
\mbox{}
\vskip 0.1cm
\noindent
Suppose:
\begin{itemize}
\item
	$V$\, is a {\color{red}finite-dimensional} complex vector space, and
\item
	$\rho : \mathfrak{su}(2) \otimes_{\Re} \C \longrightarrow \gl(V)$\,
	is a complex Lie algebra representation.
\end{itemize}
Then, there exists an eigenvector \,$v \in V$\, of \,$\rho(S_{3})$\, such that
\,$\rho(\overset{{\color{white}.}}{S}_{+})(v) = 0$.\,
\end{corollary}
\proof
{\color{blue}Since the field of scalars is \,$\C$},\, the $\C$-linear map
\,$\rho(S_{3}) : V \longrightarrow V$\,
admits at least one eigenvalue-eigenvector pair: \,$\lambda \in \C$\, and \,$0 \neq w \in V$,\,
with \,$\rho(S_{3})(\,w\,) = \lambda \cdot w$.\,
Consider the following sequence of vectors in $V$:
\begin{equation*}
w,\;\;\; \rho(S_{+})(\,w\,),\;\;\; \rho(S_{+})^{2}(\,w\,),\;\;\; \rho(S_{+})^{3}(\,w\,),\;\;\; \ldots
\end{equation*}
If each vector in the above sequence were nonzero, then,
by Corollary \ref{RaisingLoweringEigenvalues}, they would form an infinite sequence of eigenvectors of
\,$\rho(S_{3})$\, with respective eigenvalues \,$\lambda,\;\lambda+1,\;\lambda+2,\;\lambda+3,\;\ldots$\,.
Since these eigenvalues are pairwise distinct, the aforementioned sequence of eigenvectors are linearly independent.
This contradicts the {\color{red}finite-dimensionality} of $V$.
Hence, we see that there must exist $k \geq 0$ such that
\,$\rho(S_{+})^{k}(\,w\,) \neq 0$\, and \,$\rho(S_{+})^{k+1}(\,w\,) = 0$.\,
Now, set \,$v := \rho(S_{+})^{k}(\,w\,) \neq 0$.\,
Then,
\begin{equation*}
\rho(S_{3})\!\left(\,\overset{{\color{white}.}}{v}\,\right)
\;\; = \;\;
	\rho(S_{3})\!\left(\,\rho(S_{+})^{k}(\overset{{\color{white}.}}{w})\,\right)
\;\; = \;\;
	(\lambda + k) \cdot \rho(S_{+})^{k}(\overset{{\color{white}.}}{w})
\;\; = \;\;
	(\lambda + k) \cdot v\,,
\end{equation*}
and
\begin{equation*}
\rho(S_{+})\!\left(\,\overset{{\color{white}.}}{v}\,\right)
\;\; = \;\;
	\rho(S_{+})\!\left(\,\overset{{\color{white}1}}{\rho}(S_{+})^{k}(w)\,\right)
\;\; = \;\;
	\rho(S_{+})^{k+1}(\,w\,)
\;\; = \;\;
	0\,,
\end{equation*}
which shows that \,$v \neq 0$\, is indeed an eigenvector of \,$\rho(S_{3})$\, that is annihilated by
\,$\rho(S_{+})$,\, as required.
\qed

          %%%%% ~~~~~~~~~~~~~~~~~~~~ %%%%%

\vskip 0.5cm
\begin{proposition}[The irreducible representations \,\textnormal{$\pi_{d} : \su(2) \otimes_{\Re} \C \longrightarrow \gl\!\left(\,\overset{{\color{white}.}}{\C}[X,Y]_{d}\,\right)$}]
\label{IrrepsLittleSU2CXYd}
\mbox{}
\vskip 0.1cm
\noindent
Let
\,$\C[X,Y]_{d}$,\,
where
\,$d \,\in\, \{\,0, 1, 2, 3, \ldots\,\}$,\,
be the complex vector space of homogeneous polynomials of degree $d$
in the indeterminates $X$ and $Y$ with complex coefficients.
\vskip 0.1cm
\noindent
Define
\,$\pi_{d} : \su(2) \otimes_{\Re} \C \longrightarrow \gl\!\left(\,\overset{{\color{white}.}}{\C}[X,Y]_{d}\,\right)$\,
by complex-linearly extending:
\begin{equation*}
\pi_{d}\!\left(\,S_{+}\,\right)
\; := \;
	-\,Y\cdot\dfrac{\partial}{\partial\,X},
\quad\;\;\;
\pi_{d}\!\left(\,S_{-}\,\right)
\; := \;
	-\,X\cdot\dfrac{\partial}{\partial\,Y},
\quad\;\;\;
\pi_{d}\!\left(\,S_{3}\,\right)
\; := \;
	-\,\dfrac{1}{2}\cdot\left(\,X\cdot\dfrac{\partial}{\partial\,X} - Y\cdot\dfrac{\partial}{\partial\,Y}\right)
\end{equation*}
Then, the following statements are true:
\begin{enumerate}
\item
	$\pi_{d} : \su(2) \otimes_{\Re} \C \longrightarrow \gl\!\left(\,\overset{{\color{white}.}}{\C}[X,Y]_{d}\,\right)$\,
	is a finite-dimensional complex representation of the complex Lie algebra
	\,$\su(2) \otimes_{\Re} \C$.\,
\item
	The representation
	\,$\pi_{d} : \su(2) \otimes_{\Re} \C \longrightarrow \gl\!\left(\,\overset{{\color{white}.}}{\C}[X,Y]_{d}\,\right)$\,
	is irreducible.
\end{enumerate}
\end{proposition}
\proof
\begin{enumerate}
\item
	Linearity of \,$\pi_{d}$\, holds by the definition of \,$\pi_{d}$\,
	(which is obtained by complex-linearly extending its values on the vectors in the basis
	\,$\{\,S_{+}, S_{-}, S_{3}\,\}$\, of \,$\su(2) \otimes_{\Re}\C$).
	Thus, in order to establish that \,$\pi_{d}$\, is a representation (i.e., it is a complex Lie algebra homomorphism),
	it remains only to show that \,$\pi_{d}$\, preserves commutation relations.
	To this end, observe that:
	\begin{eqnarray*}
	\left[\;
		\overset{{\color{white}1}}{\pi_{d}}\!\left(\,S_{+}\,\right)
		\, , \,
		\pi_{d}\!\left(\,S_{-}\,\right)
		\,\right]
		(X^{m}Y^{n})
	& = &
		\left[\;
			-\,Y\cdot\dfrac{\partial}{\partial\,X}
			\;\; , \,
			-\,X\cdot\dfrac{\partial}{\partial\,Y}
			\,\right]
			(X^{m}Y^{n})
	\;\; = \;\;
		\left[\;
			Y\cdot\dfrac{\partial}{\partial\,X}
			\;\; , \,
			X\cdot\dfrac{\partial}{\partial\,Y}
			\,\right]
			(X^{m}Y^{n})
	\\
	& = &
		Y\cdot\dfrac{\partial}{\partial\,X}\!\left(\,
			\overset{{\color{white}.}}{X}\,X^{m} \cdot n \cdot Y^{n-1}
			\,\right)
		\, - \,
		X\cdot\dfrac{\partial}{\partial\,Y}\!\left(\,
			\overset{{\color{white}.}}{Y} \cdot m \cdot X^{m-1} \, Y^{n}
			\,\right)
	\\
	& = &
		n \cdot Y \cdot \dfrac{\partial}{\partial\,X}\!\left(\,
			\overset{{\color{white}.}}{X^{m+1}} - Y^{n-1}
			\,\right)
		\, - \,
		m \cdot X \cdot \dfrac{\partial}{\partial\,X}\!\left(\,
			\overset{{\color{white}.}}{X^{m-1}} - Y^{n+1}
			\,\right)
	\\
	& = &
		n(m+1) \cdot \overset{{\color{white}1}}{X^{m}}\,Y^{n} \, - \, m(n+1) \cdot X^{m}\,Y^{n}
	\\
	& = &
		(n-m) \cdot \overset{{\color{white}1}}{X^{m}\,Y^{n}}
	\end{eqnarray*}
	On the other hand,
	\begin{eqnarray*}
	\pi_{d}(S_{3})\!\left(\,\overset{{\color{white}.}}{X^{m}Y^{n}}\,\right)
	& = &
		-\,\dfrac{1}{2}\cdot\left(\,X\cdot\dfrac{\partial}{\partial\,X} - Y\cdot\dfrac{\partial}{\partial\,Y}\right)
		\!\left(\,\overset{{\color{white}.}}{X^{m}Y^{n}}\,\right)
	\\
	& = &
		-\,\dfrac{1}{2}\cdot\left(\,
			\overset{{\color{white}.}}{X} \cdot m \cdot X^{m-1}\,Y^{n}
			\, - \,
			Y \cdot X^{m} \cdot n \cdot Y^{n-1}
			\,\right)
	\\
	& = &
		- \, \dfrac{1}{2} \cdot (m-n) \cdot \overset{{\color{white}1}}{X^{m}\,Y^{n}}
	\;\; = \;\;
		\dfrac{1}{2} \cdot (n-m) \cdot \overset{{\color{white}1}}{X^{m}\,Y^{n}}
	\\
	& = &
		\dfrac{1}{2} \cdot
		\left[\;
			\overset{{\color{white}1}}{\pi_{d}}\!\left(\,S_{+}\,\right)
			\, , \,
			\pi_{d}\!\left(\,S_{-}\,\right)
			\,\right]
			(X^{m}Y^{n})
	\end{eqnarray*}
	This proves that
	\begin{equation*}
	\left[\;
		\overset{{\color{white}1}}{\pi_{d}}\!\left(\,S_{+}\,\right)
		\, , \,
		\pi_{d}\!\left(\,S_{-}\,\right)
		\,\right]
	\;\; = \;\;
		2 \cdot \pi_{d}(S_{3})
	\end{equation*}
	Next, note
	\begin{eqnarray*}
	\left[\;
		\overset{{\color{white}1}}{\pi_{d}}\!\left(\,S_{3}\,\right)
		\, , \,
		\pi_{d}\!\left(\,S_{+}\,\right)
		\,\right]
		(X^{m}Y^{n})
	& = &
		\left[\;\,
			-\,\dfrac{1}{2}\cdot\left(\,X\cdot\dfrac{\partial}{\partial\,X} - Y\cdot\dfrac{\partial}{\partial\,Y}\right)
			\; , \;
			-\,Y\cdot\dfrac{\partial}{\partial\,X}
			\,\right]
			(X^{m}Y^{n})
	\\
	& = &
		\left[\;\,
			\dfrac{1}{2}\cdot\left(\,X\cdot\dfrac{\partial}{\partial\,X} - Y\cdot\dfrac{\partial}{\partial\,Y}\right)
			\; , \;
			Y\cdot\dfrac{\partial}{\partial\,X}
			\,\right]
			(X^{m}Y^{n})
	\\
	& = &
		\dfrac{1}{2}\cdot\left(\,X\cdot\dfrac{\partial}{\partial\,X} - Y\cdot\dfrac{\partial}{\partial\,Y}\right)
		\left(\, Y \cdot m \cdot X^{m-1}\,Y^{n} \,\right)
	\\
	&&
		-\,Y\cdot\dfrac{\partial}{\partial\,X}\left(\,
			\dfrac{1}{2}\left(\,
				X \cdot m \cdot X^{m-1}\,Y^{n} \,-\, Y \cdot X^{m} \cdot n \cdot Y^{n-1}
				\,\right)
			\,\right)
	\\
	& = &
		\dfrac{1}{2}\cdot\left(\,X\cdot\dfrac{\partial}{\partial\,X} - Y\cdot\dfrac{\partial}{\partial\,Y}\right)
		\left(\, m \cdot X^{m-1}\,Y^{n+1} \,\right)
	\\
	&&
		-\,Y\cdot\dfrac{\partial}{\partial\,X}\left(\,
			\dfrac{1}{2}\,(m-n)\,X^{m}\,Y^{n}
			\,\right)
	\\
	& = &
		\dfrac{1}{2}\left(\,
			\overset{{\color{white}.}}{X} \cdot m \cdot (m-1) \cdot X^{m-2}\,Y^{n+1}
			\,-\,
			Y \cdot m \cdot X^{m-1} \cdot (n+1) \cdot Y^{n}
			\,\right)
	\\
	&&
		-\,Y\cdot\dfrac{1}{2}\,(m-n)\cdot m \cdot X^{m-1}\,Y^{n}
	\\
	& = &
		\dfrac{1}{2}\left(\,
			m\,(m-1)\cdot \overset{{\color{white}.}}{X^{m-1}}\,Y^{n+1}
			\,-\,
			m\,(n+1)\cdot X^{m-1}\,Y^{n+1}
			\,\right)
	\\
	&&
		-\,\dfrac{1}{2} \cdot m\,(m-n) \cdot X^{m-1}\,Y^{n+1}
	\\
	& = &
		\dfrac{m}{2}\left(\,
			m - \overset{{\color{white}.}}{1} - n - 1 - m + n
			\,\right)
		X^{m-1}\,Y^{n+1}
	\;\; = \;\;
		\dfrac{m}{2}\left(\,-\,\overset{{\color{white}.}}{2}\,\right)X^{m-1}\,Y^{n+1}
	\\
	& = &
		-\,m\,\overset{{\color{white}1}}{X^{m-1}}\,Y^{n+1}
	\end{eqnarray*}
	and
	\begin{eqnarray*}
	\pi_{d}(S_{+})\!\left(\,\overset{{\color{white}.}}{X^{m}Y^{n}}\,\right)
	& = &
		-\,Y\cdot\dfrac{\partial}{\partial\,X}\!\left(\,\overset{{\color{white}.}}{X^{m}Y^{n}}\,\right)
	\;\; = \;\;
		-\, Y \cdot m \cdot X^{m-1}\,Y^{n}
	\;\; = \;\;
		-\, m \, X^{m-1}\,Y^{n+1}
	\\
	& = &
		\left[\;
			\overset{{\color{white}1}}{\pi_{d}}\!\left(\,S_{3}\,\right)
			\, , \,
			\pi_{d}\!\left(\,S_{+}\,\right)
			\,\right]
			(X^{m}Y^{n})
	\end{eqnarray*}
	This proves:
	\begin{equation*}
	\left[\;
		\overset{{\color{white}1}}{\pi_{d}}\!\left(\,S_{3}\,\right)
		\, , \,
		\pi_{d}\!\left(\,S_{+}\,\right)
		\,\right]
	\;\; = \;\;
		\pi_{d}(S_{+})
	\end{equation*}
	Similar calculations will show:
	\begin{equation*}
	\left[\;
		\overset{{\color{white}1}}{\pi_{d}}\!\left(\,S_{3}\,\right)
		\, , \,
		\pi_{d}\!\left(\,S_{-}\,\right)
		\,\right]
	\;\; = \;\;
		-\,\pi_{d}(S_{-})
	\end{equation*}
	This completes the proof that
	\,$\pi_{d} : \su(2) \otimes_{\Re} \C \longrightarrow \gl\!\left(\,\overset{{\color{white}.}}{\C}[X,Y]_{d}\,\right)$\,
	is indeed a complex representation of the complex Lie algebra
	\,$\su(2) \otimes_{\Re} \C$.\,
	Lastly, \,$\pi_{d}$\, is a finite-dimensional representation since
	\,$\dim_{C}\!\left(\,\overset{{\color{white}.}}{\C}[X,Y]_{d}\,\right) \,=\, d+1$.\,
\item
	\textit{This proof is found in Proposition 4.11, p.84, \cite{Hall2015}.}
	\vskip 0.1cm
	\noindent
	Let \,$W \subset \C[X,Y]_{d}$\, be an nonzero invariant subspace of \,$\C[X,Y]_{d}$.\,
	We need to show that \,$W = \C[X,Y]_{d}$.\,
	\vskip 0.3cm
	\noindent
	\textbf{Claim 1:}\quad
	For \,$k \in \{\,0,1,2,\cdots,d\,\}$,\, we have:
	\begin{eqnarray*}
	\pi_{d}\!\left(\,S_{+}\right)\!\left(\,\overset{{\color{.}}}{X^{d-k}}\,Y^{k}\,\right)
	& = &
		-\,(\,d-k\,) \cdot X^{d-k-1} \, Y^{k+1}
	\\
	\pi_{d}\!\left(\,S_{-}\right)\!\left(\,\overset{{\color{.}}}{X^{d-k}}\,Y^{k}\,\right)
	& = &
		\quad\quad\;\;\;
		 -\,k \cdot X^{d-k+1} \, Y^{k-1}
	\\
	\pi_{d}\!\left(\,S_{3}\right)\!\left(\,\overset{{\color{.}}}{X^{d-k}}\,Y^{k}\,\right)
	& = &
		\left(\,k-\dfrac{d}{2}\,\right) \cdot \overset{{\color{white}1}}{X^{d-k}\,Y^{k}}
	\end{eqnarray*}
	\proofof Claim 1:\quad Straightforward calculations.
	
	\vskip 0.5cm
	\noindent
	\textbf{Claim 2:}\quad $Y^{d} \in W$.
	\vskip -0.05cm
	\noindent
	\proofof Claim 2:\quad 
	Let \,$w \in W$.\, Then, \,$w$\, can be written as:
	\begin{equation*}
	w
	\;\; = \;\;
		a_{0}X^{d} \,+\, a_{1}X^{d-1}Y \,+\, a_{2}X^{d-2}Y^{2} \,+\, \cdots \,+\, a_{d-1}XY^{d-1} \,+\, a_{d}Y^{d},
	\end{equation*}
	where at least one of the coefficients
	\,$a_{0}, a_{1}, a_{2}, \ldots, a_{d} \,\in\, \C$\,
	is nonzero.
	Let \,$k_{0}$\, be the smallest value of \,$k$\, such that \,$a_{k} \neq 0$.\,
	If \,$k_{0} = d$,\ then \,$w = a_{k_{0}}\,Y^{k_{0}} = a_{d}\,Y^{d}$\, is a nonzero multiple of \,$Y^{d}$.\,
	Thus, \,$Y^{d} \,=\, \dfrac{1}{a_{d}}\,w \,\in\, W$,\, i.e., Claim 2 is true.
	If \,$k < d_{0}$,\, we consider
	\begin{equation*}
	\pi_{d}\!\left(\,S_{+}\right)^{d - k_{0}}w
	\end{equation*}
	By Claim 1, \,$\pi_{d}(S_{+})$\, raises the exponent of \,$Y$\, by \,$1$;\, hence,
	\,$\pi_{d}\!\left(\,S_{+}\right)^{d - k_{0}}$\,
	annihilates all terms in \,$w$\, except
	\begin{equation*}
	a_{k_{0}}\,X^{d-k_{0}}\,Y^{k_{0}}
	\end{equation*}
	Hence,
	\begin{eqnarray*}
	\pi_{d}\!\left(\,S_{+}\right)^{d - k_{0}}\!\left(\,\overset{{\color{white}1}}{w}\,\right)
	& = &
		\pi_{d}\!\left(\,S_{+}\right)^{d - k_{0}}\!\left(\,
			\overset{{\color{white}1}}{a_{k_{0}}}\,X^{d-k_{0}}\,Y^{k_{0}}
			\,\right)
	\;\; = \;\;
		\cdots
	\\
	& = &
		(\,-1\,)^{d-k_{0}} \cdot a_{k_{0}} \cdot (d-k_{0}) \cdot (d-k_{0}-1) \cdot 2 \cdot 1 \cdot Y^{d}\,,
	\end{eqnarray*}	
	which shows that
	\,$\pi_{d}\!\left(\,S_{+}\right)^{d - k_{0}}\!\left(\,\overset{{\color{white}1}}{w}\,\right)$\,
	is a nonzero multiple of \,$Y^{d}$\, (since $k_{0} < d$).
	Now, invariance of \,$W$\, and \,$w \in W$\, imply
	\,$\pi_{d}\!\left(\,S_{+}\right)^{d - k_{0}}\!\left(\,\overset{{\color{white}1}}{w}\,\right) \,\in\, W$,\,
	which in turn implies
	\,$Y^{d} \,\in\, W$.\,
	This proves Claim 2.

	\vskip 0.5cm
	\noindent
	\textbf{Claim 3:}\quad $X^{k}\,Y^{d-k} \in W$,\, for each \,$k \in \{\,0,1,2,\cdots,d\,\}$.
	\vskip -0.05cm
	\noindent
	\proofof Claim 3:\quad By Claim 1, Claim 2, and the invariance of \,$W$,\, we have:
	\begin{equation*}
	 -\,d \cdot X \, Y^{d-1}
	\; = \;
		\pi_{d}\!\left(\,S_{-}\right)\!\left(\;\overset{{\color{.}}}{Y^{d}}\,\right)
		\; \in \; W
	\quad \Longrightarrow \quad
		X \, Y^{d-1} \; \in \; W
	\end{equation*}
	By Claim 1 and the invariance of \,$W$,\, we have:
	\begin{equation*}
	 -\,(d-1) \cdot X^{2} \, Y^{d-2}
	\; = \;
		\pi_{d}\!\left(\,S_{-}\right)\!\left(\;\overset{{\color{.}}}{X}\,Y^{d-1}\,\right)
		\; \in \; W
	\quad \Longrightarrow \quad
		X^{2} \, Y^{d-2} \; \in \; W
	\end{equation*}

	\begin{equation*}
	\pi_{d}\!\left(\,S_{-}\right)\!\left(\,\overset{{\color{.}}}{X^{d-k}}\,Y^{k}\,\right)
	\;\; = \;\;
		 -\,k \cdot X^{d-k+1} \, Y^{k-1}
	\end{equation*}
	Claim 3 now follows by repeating the above argument suitably many times.
	
	\vskip 0.5cm
	\noindent
	Now, recall that
	\,$\C[X,Y]_{d}$\,
	is spanned over \,$\C$\, by
	\,$X^{d},\, X^{d-1}Y,\, X^{d-2}Y^{2},\, \cdots,\, XY^{d-1},\, Y^{d}$.\,
	Claim 3 now implies that \,$\C[X,Y]_{d} \,\subset\, W$,\,
	which in turn proves that
	\,$\C[X,Y]_{d} \,=\, W$.\
	This completes the proof of the irreducibility of the representation
	\,$\pi_{d} : \su(2) \otimes_{\Re} \C \longrightarrow \gl\!\left(\,\overset{{\color{white}.}}{\C}[X,Y]_{d}\,\right)$.\,
	\qed
\end{enumerate}

          %%%%% ~~~~~~~~~~~~~~~~~~~~ %%%%%

\vskip 0.5cm
\begin{theorem}[Irreducible finite-dimensional complex representations of \,$\su(2) \otimes_{\Re} \C$]
\mbox{}
\vskip 0.1cm
\noindent
Every irreducible finite-dimensional complex representation
\,$\rho : \su(2) \otimes_{\Re} \C \,\longrightarrow\, \gl\!\left(V\right)$\,
of the complex Lie algebra
\,$\su(2) \otimes_{\Re} \C$\,
is isomorphic to one of the representations explicitly constructed in Proposition \ref{IrrepsLittleSU2CXYd},
i.e., \,$\rho$\, is isomorphic to
\begin{equation*}
\pi_{d} : \su(2) \otimes_{\Re} \C \,\longrightarrow\, \gl\!\left(\,\overset{{\color{white}.}}{\C}[X,Y]_{d}\,\right)
\end{equation*}
for some \,$d \in \{\,0, 1, 2, \ldots \,\}$.
\end{theorem}
\proof
By Corollary \ref{EigenvectorOfS3KilledBySplus}, there exists an eigenvector
\,$0 \neq w \in V$\, of \,$\rho(S_{3})$\, such that \,$\rho(S_{+})(\,w\,) = 0$.\,
Let \,$\lambda \in \C$\, denote the eigenvalue of \,$w \in V$;\,
thus, \,$\rho(S_{3})(\,w\,) = \lambda \cdot w$.\,
Consider the following sequence of vectors in \,$V$:\,
\begin{equation*}
w,\;\;\; \rho(S_{-})(\,w\,),\;\;\; \rho(S_{-})^{2}(\,w\,),\;\;\; \rho(S_{-})^{3}(\,w\,),\;\;\; \ldots
\end{equation*}
If each vector in the above sequence were nonzero, then,
by Corollary \ref{RaisingLoweringEigenvalues}, they would form an infinite sequence of eigenvectors of
\,$\rho(S_{3})$\, with respective eigenvalues \,$\lambda,\;\lambda-1,\;\lambda-2,\;\lambda-3,\;\ldots$\,.
Since these eigenvalues are pairwise distinct, the aforementioned sequence of eigenvectors are linearly independent.
This contradicts the finite-dimensionality of $V$.
Hence, we see that there must exist $d \geq 0$ such that
\,$\rho(S_{-})^{d}(\,w\,) \neq 0$\, and \,$\rho(S_{-})^{d+1}(\,w\,) = 0$.\,
\vskip 0.5cm
\noindent
\textbf{Claim 1:}\quad
$\left\{\;\overset{{\color{white}-}}{w},\; \rho(S_{-})(\,w\,),\; \rho(S_{-})^{2}(\,w\,),\; \ldots\,,\; \rho(S_{-})^{d}(\,w\,) \;\right\}$\,
forms a basis for \,$V$.
\vskip 0.1cm
\noindent
\proofof Claim 1:\quad
First, note that the vectors are linearly independent since they are eigenvectors of
\,$\rho(S_{3})$\,
with pairwise distinct eigenvalues.
Next, we will show that their linear span is a $\rho$-invariant subspace of $V$,
from which Claim 1 follows immediately by the irreducibility of $\rho$.
By construction,
\,$\span_{\C}\!\left\{\;\rho(S_{-})^{k}(\,w\,)\,\right\}_{k=0}^{d}$\,
is invariant under \,$\rho(S_{3})$\, and \,$\rho(S_{-})$.\,
Hence, it remains only to show that the span is also invariant under \,$\rho(S_{+})$.\,
To this end, we proceed by induction on \,$k = 0, 1, 2, \ldots\,$\, that
\begin{equation*}
\rho(S_{+})\!\left(\,\overset{{\color{white}1}}{\rho}(S_{-})^{{\color{red}k}}(\,w\,)\,\right)
\;\; \in \;\;
	\span_{\C}\!\left\{\;
		\left.
		\overset{{\color{white}1}}{\rho}(S_{-})^{j}(w)
		\;\right\vert\;
		j = 0, 1, \ldots, {\color{red}k - 1}
		\;\right\},
\end{equation*}
where we follow the standard convention that \,$\span_{\C}\{\,\varemptyset\,\} = \{\,0\,\}$.\,
So, for \,$k = 0$,\, we indeed have:
\begin{equation*}
\rho(S_{+})\!\left(\,\overset{{\color{white}1}}{\rho}(S_{-})^{{\color{red}0}}(\,w\,)\,\right)
\; = \;
	\rho(S_{+})\!\left(\,\overset{{\color{white}.}}{w}\,\right)
\; = \;
	0
\; \in \;
	\{\,0\,\}
\; = \;
	\span_{\C}\!\left\{\,\overset{{\color{white}.}}{\varemptyset}\,\right\}
\; = \;
	\span_{\C}\!\left\{\;
		\left.
		\overset{{\color{white}1}}{\rho}(S_{-})^{j}(w)
		\;\right\vert\;
		j = 0, 1, \ldots, {\color{red}-1}
		\;\right\}
\end{equation*}
For the inductive step, we assume validity up to \,$k - 1$ (induction hypothesis),
and we seek to establish validity for \,$k$.\,
The precise statement of the induction hypothesis is:
\begin{equation*}
\rho(S_{+})\!\left(\,\overset{{\color{white}1}}{\rho}(S_{-})^{{\color{red}k-1}}(\,w\,)\,\right)
\;\; \in \;\;
	\span_{\C}\!\left\{\;
		\left.
		\overset{{\color{white}1}}{\rho}(S_{-})^{j}(w)
		\;\right\vert\;
		j = 0, 1, \ldots, {\color{red}k - 2}
		\;\right\},
\end{equation*}
which immediately implies:
\begin{eqnarray*}
\rho(S_{-})\,
\rho(S_{+})\!\left(\,\overset{{\color{white}1}}{\rho}(S_{-})^{{\color{red}k-1}}(\,w\,)\,\right)
& \in &
	\span_{\C}\!\left\{\;
		\left.
		\overset{{\color{white}1}}{\rho}(S_{-})^{j{\color{red}+1}}(w)
		\;\right\vert\;
		j = 0, 1, \ldots, {\color{red}k - 2}
		\;\right\}
\\
& \subset &
	\span_{\C}\!\left\{\;\;\;
		\left.
		\overset{{\color{white}1}}{\rho}(S_{-})^{j}(w)
		\;\;\,\right\vert\;
		j = 0, 1, \ldots, {\color{red}k - 1}
		\;\right\}
\end{eqnarray*}
Consequently,
\begin{eqnarray*}
\rho(S_{+})\!\left(\,\overset{{\color{white}1}}{\rho}(S_{-})^{{\color{red}k}}(\,w\,)\,\right)
& = &
	\rho(S_{+})\,,\,\rho(S_{-})\!\left(\,\overset{{\color{white}1}}{\rho}(S_{-})^{{\color{red}k-1}}(\,w\,)\,\right)
\\
&&
	\, - \,
	\rho(S_{-})\,\rho(S_{+})\!\left(\,\overset{{\color{white}1}}{\rho}(S_{-})^{k-1}(\,w\,)\,\right)
	\, + \,
	\rho(S_{-})\,\rho(S_{+})\!\left(\,\overset{{\color{white}1}}{\rho}(S_{-})^{k-1}(\,w\,)\,\right)
\\
& = &
	\left[\,\overset{{\color{white}1}}{\rho}(S_{+})\,,\,\rho(S_{-})\,\right]
	\!\left(\,\overset{{\color{white}1}}{\rho}(S_{-})^{k-1}(\,w\,)\,\right)
	\, + \,
	\rho(S_{-})\,\rho(S_{+})\!\left(\,\overset{{\color{white}1}}{\rho}(S_{-})^{k-1}(\,w\,)\,\right)
\\
& = &
	2 \cdot \rho(S_{3})\!\left(\,\overset{{\color{white}1}}{\rho}(S_{-})^{k-1}(\,w\,)\,\right)
	\, + \,
	\rho(S_{-})\,\rho(S_{+})\!\left(\,\overset{{\color{white}1}}{\rho}(S_{-})^{k-1}(\,w\,)\,\right)
\\
& \in &
	\span_{\C}\!\left\{\;
		\left.
		\overset{{\color{white}1}}{\rho}(S_{-})^{j}(w)
		\;\right\vert\;
		j = 0, 1, \ldots, {\color{red}k - 1}
		\;\right\}
\end{eqnarray*}
This proves Claim 1.
\vskip 0.5cm
\noindent
\textbf{Claim 2:}\quad $\lambda \,=\, \dfrac{d}{2}$
\vskip 0.1cm
\noindent
\proofof Claim 2:\quad
The matrix representative of \,$\rho(S_{3})$\, with respect to the basis\\
\begin{equation*}
\left\{\;
	\overset{{\color{white}1}}{w},\; \rho(S_{-})(w),\; \rho(S_{-})^{2}(w),\; \ldots\,,\; \rho(S_{-})^{d}(w)
	\;\right\}
\end{equation*}
for \,$V$\, is
\,$\diag(\,\lambda,\,\lambda-1,\ldots,\lambda - d\,)$,\,
whose trace is
\begin{equation*}
\overset{d}{\underset{k\,=\,0}{\sum}}\;(\lambda - k)
\;\; = \;\;
	\lambda\cdot(d+1) \;-\, \overset{d}{\underset{k\,=\,1}{\sum}}\;k
\;\; = \;\;
	\lambda\cdot(d+1) \;-\, \dfrac{d(d+1)}{2}
\;\; = \;\;
	\left(\,\lambda - \dfrac{d}{2}\,\right)\cdot(d+1)
\end{equation*}
On the other hand,
\begin{eqnarray*}
\trace\!\left(\,\overset{{\color{white}1}}{\rho}(S_{3})\,\right)
& = &
	\trace\!\left(\;\dfrac{1}{2}\cdot\!\left[\,\overset{{\color{white}1}}{\rho}(S_{+})\,,\,\rho(S_{-})\,\right]\,\right)
\;\; = \;\;
	\dfrac{1}{2}\cdot\trace\!\left(\,\left[\,\overset{{\color{white}1}}{\rho}(S_{+})\,,\,\rho(S_{-})\,\right]\,\right)
\\
& = &
	\dfrac{1}{2}\cdot\trace\!\left(\,
		\overset{{\color{white}1}}{\rho}(S_{+})\cdot\rho(S_{-})
		\, - \,
		\overset{{\color{white}1}}{\rho}(S_{-})\cdot\rho(S_{+})
		\,\right)
\;\; = \;\;
	\cdots
\\
& \overset{{\color{white}\textnormal{\Large$1$}}}{=} &
	0
\end{eqnarray*}
Combining the last two equalities yields:
\begin{equation*}
\left(\,\lambda - \dfrac{d}{2}\,\right)\cdot(d+1)
\;\; = \;\;
	\trace\!\left(\,\overset{{\color{white}1}}{\rho}(S_{3})\,\right)
\;\; = \;\;
	0,
\end{equation*}
from which Claim 2 follows.
\vskip 0.5cm
\noindent
\textbf{Claim 3:}\quad
Define the complex linear map
\,$\Psi : V \longrightarrow \C[X,Y]_{d}$\,
by complex-linearly extending:
\begin{equation*}
\Psi(\,w\,) \;\; := \;\; Y^{d}
\end{equation*}
and
\begin{equation*}
\Psi\!\left(\,\rho(S_{-})^{k}(\,\overset{{\color{white}1}}{w}\,)\,\right)
\;\; := \;\;
	\pi_{d}(S_{-})^{k}\!\left(\,Y^{d}\,\right)
\;\; = \;\;
	\pi_{d}(S_{-})^{k}\!\left(\,\Psi(\,\overset{{\color{white}1}}{w}\,)\,\right)
\end{equation*}
Then, \,$\Psi$\, is an intertwining map of the representations
\,$\rho$\, and \,$\pi_{d}$,\, i.e.,
\begin{equation*}
\Psi \,\circ\, \rho(A)
\;\; = \;\;
	\pi_{d}(A) \,\circ\, \Psi\,,
\quad
\textnormal{for each \,$A \,\in\, \su(2) \otimes_{\Re} \C$}
\end{equation*}
In particular, the two representations are isomorphic to each other.
\vskip 0.1cm
\noindent
\proofof Claim 3:\quad
First, recall that
\begin{equation*}
w,\;\; \rho(S_{-})(\,w\,),\;\; \rho(S_{-})^{2}(\,w\,),\;\; \ldots\,,\;\; \rho(S_{-})^{d}(\,w\,)
\end{equation*}
form a basis for \,$V$\, consisting of $\rho(S_{3})$-eigenvectors with respective eigenvalues
\begin{equation*}
\lambda = \dfrac{d}{2}\,,
\quad
\lambda - 1 = \dfrac{d}{2} -1\,,
\quad
\lambda - 2 = \dfrac{d}{2} -2\,,
\quad
\ldots\,,
\quad
\lambda - d = -\,\dfrac{d}{2}
\end{equation*}
On the other hand,
\begin{equation*}
Y^{d},\;\; \pi_{d}(S_{-})(\,Y^{d}\,),\;\; \pi_{d}(S_{-})^{2}(\,Y^{d}\,),\;\; \ldots\,,\;\; \pi_{d}(S_{-})^{d}(\,Y^{d}\,)
\end{equation*}
form a basis for \,$\C[X,Y]_{d}$\, consisting of $\pi_{d}(S_{3})$-eigenvectors with respective eigenvalues
\begin{equation*}
\lambda = \dfrac{d}{2}\,,
\quad
\lambda - 1 = \dfrac{d}{2} -1\,,
\quad
\lambda - 2 = \dfrac{d}{2} -2\,,
\quad
\ldots\,,
\quad
\lambda - d = -\,\dfrac{d}{2}
\end{equation*}
Thus, \,$\Psi : V \longrightarrow \C[X,Y]_{d}$\, is a vector space isomorphism, and
\,$\Psi$\, furthermore maps $\rho(S_{3})$-eigenvectors of $V$ to $\pi_{d}(S_{3})$-eigenvectors of $\C[X,Y]_{d}$,
preserving eigenvalues, which implies
\,$\Psi \,\circ\, \rho(S_{3}) \,=\, \pi_{d}(S_{3}) \,\circ \Psi$.\,
By construction, we also have
\,$\Psi \,\circ\, \rho(S_{-}) \,=\, \pi_{d}(S_{-}) \,\circ \Psi$.\,
Thus, in order to show that \,$\Psi$\, is indeed an intertwining map between the two representations
\,$\rho$\, and \,$\pi_{d}$,\,
it remains only to establish that
\begin{equation*}
\Psi \,\circ\, \rho(S_{+})
\;\; = \;\;
	\pi_{d}(S_{+}) \,\circ \Psi
\end{equation*}
More precisely, it suffices to show:
\begin{equation*}
\Psi \,\circ\, \rho(S_{+})\!\left(\,\rho(S_{-})^{k}(\,\overset{{\color{white}1}}{w}\,)\,\right)
\;\; = \;\;
	\pi_{d}(S_{+}) \,\circ \Psi \left(\,\rho(S_{-})^{k}(\,\overset{{\color{white}1}}{w}\,)\,\right),
\quad
\textnormal{for \,$k = 0, 1, 2, \ldots, d$}
\end{equation*}
We proceed by induction on $k$.
Validity for \,$k = 0$\, is straightforward:
\begin{eqnarray*}
\pi_{d}(S_{+}) \,\circ \Psi \left(\,\rho(S_{-})^{{\color{red}0}}(\,\overset{{\color{white}1}}{w}\,)\,\right)
& = &
	\pi_{d}(S_{+}) \,\circ \Psi \left(\,\overset{{\color{white}1}}{w}\,\right)
\;\; = \;\;
	\pi_{d}(S_{+})\!\left(\!\overset{{\color{white}.}}{{\color{white}i}}Y^{d}\,\right)
\;\; = \;\;
	-\,(d - {\color{red}d}) \cdot X^{d - {\color{red}d}-1} \cdot Y^{{\color{red}d}+1}
\\
& = &
	0
\;\; = \;\;
	\Psi(\,0\,)
\;\; = \;\;
	\Psi \,\circ\, \rho(S_{+})\!\left(\,\overset{{\color{white}1}}{w}\,\right)
\\
& = &
\Psi \,\circ\, \rho(S_{+})\!\left(\,\rho(S_{-})^{{\color{red}0}}(\,\overset{{\color{white}1}}{w}\,)\,\right),
\end{eqnarray*}
where the third equality follows from Claim 1 in the proof of Proposition \ref{IrrepsLittleSU2CXYd}(ii).
We now assume validity up to \,$k-1$\, and show validity for \,$k$.\,
\begin{eqnarray*}
\Psi \,\circ\, \rho(S_{+})\!\left(\,\rho(S_{-})^{k}(\,\overset{{\color{white}1}}{w}\,)\,\right)
& = &
	\Psi \,\circ\, \rho(S_{+}) \circ \rho(S_{-}) \left(\,\rho(S_{-})^{k-1}(\,\overset{{\color{white}1}}{w}\,)\,\right)
\\
& = &
	\Psi\!\left[\;
		\rho(S_{-}) \circ \rho(S_{+}) \left(\,\rho(S_{-})^{k-1}(\,\overset{{\color{white}1}}{w}\,)\,\right)
		\, + \,
		2\,\rho(S_{3}) \left(\,\rho(S_{-})^{k-1}(\,\overset{{\color{white}1}}{w}\,)\,\right)
		\,\right]
\\
& = &
	\Psi \circ \rho(S_{-}) \left[\;
		\rho(S_{+}) \left(\,\rho(S_{-})^{k-1}(\,\overset{{\color{white}1}}{w}\,)\,\right)
		\;\right]
	\, + \,
	2 \cdot\! \Psi \circ \rho(S_{3})
		\left(\,\rho(S_{-})^{k-1}(\,\overset{{\color{white}1}}{w}\,)\,\right)
\\
& = &
	\pi_{d}(S_{-}) \circ {\color{red}\Psi} \left[\;
		{\color{red}\rho(S_{+})} \left(\,\rho(S_{-})^{k-1}(\,\overset{{\color{white}1}}{w}\,)\,\right)
		\;\right]
	\, + \,
	2 \cdot\! \pi_{d}(S_{3}) \circ \Psi\!
		\left(\,\rho(S_{-})^{k-1}(\,\overset{{\color{white}1}}{w}\,)\,\right)
\\
& = &
	\pi_{d}(S_{-}) \circ {\color{red}\pi_{d}(S_{+})} \left[\;
		{\color{red}\Psi}\!\left(\,\rho(S_{-})^{k-1}(\,\overset{{\color{white}1}}{w}\,)\,\right)
		\;\right]
	\, + \,
	2 \cdot\! \pi_{d}(S_{3}) \circ \Psi\!
		\left(\,\rho(S_{-})^{k-1}(\,\overset{{\color{white}1}}{w}\,)\,\right)
\\
& = &
	\left(\,
		\pi_{d}(S_{-}) \circ \pi_{d}(S_{+})
		\, \overset{{\color{white}1}}{+} \,
		2 \cdot\! \pi_{d}(S_{3})
		\,\right)
	\left[\;
		\Psi\!\left(\,\rho(S_{-})^{k-1}(\,\overset{{\color{white}1}}{w}\,)\,\right)
		\;\right]
\\
& = &
	\pi_{d}(S_{+}) \circ \pi_{d}(S_{-})
	\left[\;
		\Psi\!\left(\,\rho(S_{-})^{k-1}(\,\overset{{\color{white}1}}{w}\,)\,\right)
		\;\right]
\\
& = &
	\pi_{d}(S_{+}) \circ \Psi
	\left[\;
		\rho(S_{-})\!\left(\,\rho(S_{-})^{k-1}(\,\overset{{\color{white}1}}{w}\,)\,\right)
		\;\right]
\\
& = &
	\pi_{d}(S_{+}) \circ \Psi
	\left[\;
		\rho(S_{-})^{k}(\,\overset{{\color{white}1}}{w}\,)
		\;\right],
\end{eqnarray*}
as required.
Note that the 5th equality above follows from the induction hypothesis,
whereas the 4th and the second last equalities follow from the already established facts that
\,$\Psi \circ \rho(S_{-}) \,=\, \pi_{d}(S_{-}) \circ \Psi$\,
and
\,$\Psi \circ \rho(S_{3}) \,=\, \pi_{d}(S_{3}) \circ \Psi$.\, 
This completes the proof of Claim 3, and hence that of the Theorem.
\qed

          %%%%% ~~~~~~~~~~~~~~~~~~~~ %%%%%

%\vskip 0.5cm
%
          %%%%% ~~~~~~~~~~~~~~~~~~~~ %%%%%

\chapter{Irreducible representations of $\textnormal{SO}(3)$}
\setcounter{theorem}{0}
\setcounter{equation}{0}

%\cite{vanDerVaart1996}
%\cite{Kosorok2008}

%\renewcommand{\theenumi}{\alph{enumi}}
%\renewcommand{\labelenumi}{\textnormal{(\theenumi)}$\;\;$}
\renewcommand{\theenumi}{\roman{enumi}}
\renewcommand{\labelenumi}{\textnormal{(\theenumi)}$\;\;$}

          %%%%% ~~~~~~~~~~~~~~~~~~~~ %%%%%

\section{Definition of \,$\textnormal{SO}(3)$}

          %%%%% ~~~~~~~~~~~~~~~~~~~~ %%%%%

\vskip 0.1cm
\begin{definition}[$\textnormal{O}(n)$ and $\textnormal{SO}(n)$]
\mbox{}
\vskip 0.1cm
\noindent
The \textbf{orthogonal group} is defined as follows:
\begin{equation*}
\textnormal{O}(n)
\; := \;
	\left\{\;\,
		g \overset{{\color{white}.}}{\in} \textnormal{GL}(n,\Re)
		\;\left\vert\;\,
			g^{T} \cdot g = I_{n}
			\right.
		\;\right\}
\end{equation*}
The \textbf{special orthogonal group} is defined as follows:
\begin{equation*}
\textnormal{SO}(n)
\; := \;
	\left\{\;\,
		g \overset{{\color{white}.}}{\in} \textnormal{GL}(n,\Re)
		\;\left\vert\;\,
			g^{T} \cdot g = I_{n}\,,
			\;
			\textnormal{det}(g) = 1
			\right.
		\;\right\}
\end{equation*}
\end{definition}

\begin{proposition}[Lie algebras of $\textnormal{O}(n)$ and $\textnormal{SO}(n)$]
\begin{eqnarray*}
\mathfrak{o}(n)
& = &
	\left\{\;\,
		X \overset{{\color{white}.}}{\in} \mathfrak{gl}(n,\Re)
		\;\left\vert\;\,
			X^{T} = -X
			\right.
		\;\right\}
\\
\mathfrak{so}(n)
& = &
	\left\{\;\,
		X \overset{{\color{white}.}}{\in} \mathfrak{gl}(n,\Re)
		\;\left\vert\;\,
			X^{T} = -X\,,
			\;
			\textnormal{trace}(X) = 0
			\right.
		\;\right\}
\end{eqnarray*}
\end{proposition}

          %%%%% ~~~~~~~~~~~~~~~~~~~~ %%%%%

\section{Generators of \;$\textnormal{SO}(3)$\, and \,$\mathfrak{so}(3)$}

          %%%%% ~~~~~~~~~~~~~~~~~~~~ %%%%%

\vskip 0.1cm
\noindent
Define:
\begin{equation*}
R_{1}(\theta)
\; := \;
	\left(\,
		\begin{array}{ccc}
			{\color{white}-}\cos\theta & -\sin\theta & {\color{white}-}0 \\
			{\color{white}-}\sin\theta & {\color{white}-}\cos\theta & {\color{white}-}0 \\
			{\color{white}-}0 & {\color{white}-}0 & {\color{white}-}1 \\
			\end{array}
		\,\right)
\end{equation*}
\begin{equation*}
R_{2}(\psi)
\; := \;
	\left(\,
		\begin{array}{ccc}
			{\color{white}-}\cos\psi & {\color{white}-}0 & {\color{black}-}\sin\psi \\
			{\color{white}-}0 & {\color{white}-}1 & {\color{white}-}0 \\
			{\color{white}-}\sin\psi & {\color{white}-}0 & {\color{white}-}\cos\psi \\
			\end{array}
		\,\right)
\end{equation*}
\begin{equation*}
R_{3}(\phi)
\; := \;
	\left(\,
		\begin{array}{ccc}
			{\color{white}-}1 & {\color{white}-}0 & {\color{white}-}0 \\
			{\color{white}-}0 & {\color{white}-}\cos\phi & -\sin\phi \\
			{\color{white}-}0 & {\color{white}-}\sin\phi & {\color{white}-}\cos\phi \\
			\end{array}
		\,\right)
\end{equation*}

          %%%%% ~~~~~~~~~~~~~~~~~~~~ %%%%%

\vskip 0.5cm
\begin{equation*}
X_{1}
\;\; := \;\;
	\left.\dfrac{\d}{\d\,\theta}\right\vert_{\theta = 0} R_{3}(\theta)
\;\; = \;
	\left.\left(\!
		\begin{array}{ccc}
			{\color{black}-}\sin\theta & {\color{black}-}\cos\theta & {\color{white}-}0 \\
			{\color{white}-}\cos\theta & {\color{black}-}\sin\theta & {\color{white}-}0 \\
			{\color{white}-}0 & {\color{white}-}0 & {\color{white}-}0 \\
			\end{array}
		\,\right)\right\vert_{\psi = 0}
\;\; = \;
	\left(
		\begin{array}{ccc}
			{\color{white}-}0 & {\color{black}-}1 & {\color{white}-}0 \\
			{\color{white}-}1 & {\color{white}-}0 & {\color{white}-}0 \\
			{\color{white}-}0 & {\color{white}-}0 & {\color{white}-}0 \\
			\end{array}
		\,\right)
\end{equation*}
\begin{equation*}
X_{2}
\;\; := \;\;
	\left.\dfrac{\d}{\d\,\psi}\right\vert_{\psi = 0} R_{2}(\psi)
\;\; = \;
	\left.\left(\!\!
		\begin{array}{ccc}
			{\color{black}-}\sin\psi & {\color{white}-}0 & {\color{black}-}\cos\psi \\
			{\color{white}-}0 & {\color{white}-}0 & {\color{white}-}0 \\
			{\color{white}-}\cos\psi & {\color{white}-}0 & {\color{black}-}\sin\psi \\
			\end{array}
		\,\right)\right\vert_{\psi = 0}
\;\; = \;
	\left(\!\!
		\begin{array}{ccc}
			{\color{white}-}0 & {\color{white}-}0 & {\color{black}-}1 \\
			{\color{white}-}0 & {\color{white}-}0 & {\color{white}-}0 \\
			{\color{white}-}1 & {\color{white}-}0 & {\color{white}-}0 \\
			\end{array}
		\,\right)
\end{equation*}
\begin{equation*}
X_{3}
\;\; := \;\;
	\left.\dfrac{\d}{\d\,\phi}\right\vert_{\phi = 0} R_{1}(\phi)
\;\; = \;
	\left.\left(\,
		\begin{array}{ccc}
			{\color{white}-}1 & {\color{white}-}0 & {\color{white}-}0 \\
			{\color{white}-}0 & {\color{black}-}\sin\phi & {\color{black}-}\cos\phi \\
			{\color{white}-}0 & {\color{white}-}\cos\phi & {\color{black}-}\sin\phi \\
			\end{array}
		\,\right)\right\vert_{\phi = 0}
\;\; = \;
	\left(\,
		\begin{array}{ccc}
			{\color{white}-}0 & {\color{white}-}0 & {\color{white}-}0 \\
			{\color{white}-}0 & {\color{white}-}0 & {\color{black}-}1 \\
			{\color{white}-}0 & {\color{white}-}1 & {\color{white}-}0 \\
			\end{array}
		\,\right)
\end{equation*}

\vskip 0.5cm
\begin{equation*}
\left[\,X_{1}\,,\,X_{2}\,\right] \;=\; +\,X_{3}
\quad
\left[\,X_{3}\,,\,X_{1}\,\right] \;=\; +\,X_{2}
\quad
\left[\,X_{2}\,,\,X_{3}\,\right] \;=\; +\,X_{1}
\end{equation*}

\vskip 0.5cm
\begin{equation*}
\left[\,X_{a}\,,\,X_{b}\,\right] \;=\; \epsilon_{abc}\,X_{c}
\end{equation*}

          %%%%% ~~~~~~~~~~~~~~~~~~~~ %%%%%

\vskip 0.5cm
\section{Properties of the generators \,$J_{1}, J_{2}, J_{3} \,\in\, \mathfrak{so}(3) \otimes_{\Re} \C$}

          %%%%% ~~~~~~~~~~~~~~~~~~~~ %%%%%

\noindent
\textbf{The generators \,$J_{n} \in \C^{3 \times 3}$\, of the Euler matrices}
\begin{equation*}
R_{n}(\theta)
\; = \;
	\exp\!\left(\;\sqrt{-1}\cdot\theta \overset{{\color{white}1}}{\cdot} J_{n}\,\right)
\; = \;
	\exp\!\left(\;\i\cdot\theta \overset{{\color{white}1}}{\cdot} J_{n}\,\right)
\end{equation*}
Alternatively, note:
\begin{equation*}
\i \cdot J_{1}
\;\; = \;\;
	\left.\dfrac{\d}{\d\,\phi}\right\vert_{\phi = 0} R_{1}(\phi)
\;\; = \;
	\left.\left(\,
		\begin{array}{ccc}
			{\color{white}-}1 & {\color{white}-}0 & {\color{white}-}0 \\
			{\color{white}-}0 & {\color{black}-}\sin\phi & {\color{black}-}\cos\phi \\
			{\color{white}-}0 & {\color{white}-}\cos\phi & {\color{black}-}\sin\phi \\
			\end{array}
		\,\right)\right\vert_{\phi = 0}
\;\; = \;
	\left(\,
		\begin{array}{ccc}
			{\color{white}-}0 & {\color{white}-}0 & {\color{white}-}0 \\
			{\color{white}-}0 & {\color{white}-}0 & {\color{black}-}1 \\
			{\color{white}-}0 & {\color{white}-}1 & {\color{white}-}0 \\
			\end{array}
		\,\right)
\end{equation*}
Multiplying both sides by \,$-\,\i = -\,\sqrt{-1}$\, gives:
\begin{equation*}
J_{1}
\;\; = \;
	\left(\!\!
		\begin{array}{ccc}
			{\color{white}-}0 & {\color{white}-}0 & {\color{white}-}0 \\
			{\color{white}-}0 & {\color{white}-}0 & {\color{black}-}\i \\
			{\color{white}-}0 & {\color{white}-}\i & {\color{white}-}0 \\
			\end{array}
		\,\right)
\end{equation*}
Similarly,
\begin{equation*}
\i \cdot J_{2}
\;\; = \;\;
	\left.\dfrac{\d}{\d\,\psi}\right\vert_{\psi = 0} R_{2}(\psi)
\;\; = \;
	\left.\left(\!\!
		\begin{array}{ccc}
			{\color{black}-}\sin\psi & {\color{white}-}0 & {\color{black}-}\cos\psi \\
			{\color{white}-}0 & {\color{white}-}0 & {\color{white}-}0 \\
			{\color{white}-}\cos\psi & {\color{white}-}0 & {\color{black}-}\sin\psi \\
			\end{array}
		\,\right)\right\vert_{\psi = 0}
\;\; = \;
	\left(\!\!
		\begin{array}{ccc}
			{\color{white}-}0 & {\color{white}-}0 & {\color{black}-}1 \\
			{\color{white}-}0 & {\color{white}-}0 & {\color{white}-}0 \\
			{\color{white}-}1 & {\color{white}-}0 & {\color{white}-}0 \\
			\end{array}
		\,\right)
\end{equation*}
Multiplying both sides by \,$-\,\i = -\,\sqrt{-1}$\, gives:
\begin{equation*}
J_{2}
\;\; = \;
	\left(\!\!
		\begin{array}{ccc}
			{\color{white}-}0 & {\color{white}-}0 & {\color{white}-}\i \\
			{\color{white}-}0 & {\color{white}-}0 & {\color{white}-}0 \\
			{\color{black}-}\i & {\color{white}-}0 & {\color{white}-}0 \\
			\end{array}
		\,\right)
\end{equation*}
Lastly,
\begin{equation*}
\i \cdot J_{3}
\;\; = \;\;
	\left.\dfrac{\d}{\d\,\theta}\right\vert_{\theta = 0} R_{3}(\theta)
\;\; = \;
	\left.\left(\!
		\begin{array}{ccc}
			{\color{black}-}\sin\theta & {\color{black}-}\cos\theta & {\color{white}-}0 \\
			{\color{white}-}\cos\theta & {\color{black}-}\sin\theta & {\color{white}-}0 \\
			{\color{white}-}0 & {\color{white}-}0 & {\color{white}-}0 \\
			\end{array}
		\,\right)\right\vert_{\psi = 0}
\;\; = \;
	\left(
		\begin{array}{ccc}
			{\color{white}-}0 & {\color{black}-}1 & {\color{white}-}0 \\
			{\color{white}-}1 & {\color{white}-}0 & {\color{white}-}0 \\
			{\color{white}-}0 & {\color{white}-}0 & {\color{white}-}0 \\
			\end{array}
		\,\right)
\end{equation*}
Multiplying both sides by \,$-\,\i = -\,\sqrt{-1}$\, gives:
\begin{equation*}
J_{3}
\;\; = \;
	\left(\!
		\begin{array}{ccc}
			{\color{white}-}0 & {\color{black}-}\i & {\color{white}-}0 \\
			{\color{white}-}\i & {\color{white}-}0 & {\color{white}-}0 \\
			{\color{white}-}0 & {\color{white}-}0 & {\color{white}-}0 \\
			\end{array}
		\,\right)
\end{equation*}

          %%%%% ~~~~~~~~~~~~~~~~~~~~ %%%%%

          %%%%% ~~~~~~~~~~~~~~~~~~~~ %%%%%

\begin{proposition}
{\color{white}.}\vskip -0.5cm{\color{white}.}
\begin{enumerate}
\item
	\textbf{Commutation relations:}\;\;
	\begin{equation*}
	\left[\,J_{k}\,\overset{{\color{white}1}}{,}\,J_{l}\,\right]
	\;\; = \;\;
		\sqrt{-1}\;\overset{3}{\underset{m=1}{\sum}}\,\varepsilon_{klm}\cdot J_{m}\,,
	\quad
	\textnormal{for each \,$k, l \in \{\,1,2,3\,\}$}\,,
	\end{equation*}
	where \,$\varepsilon_{klm}$\, is the fully anti-symmetric tensor.
\item
	\textbf{Raising and lowering operators:}\;\;
	Define \,$J_{+}\,,\, J_{-} \in \mathfrak{so}(3) \otimes_{\Re} \C \subset \mathcal{U}\!\left(\mathfrak{so}(3) \overset{{\color{white}.}}{\otimes_{\Re}} \C\right)$\, as follows:
	\begin{equation*}
	J_{\pm} \;\; := \;\; J_{1} \, \pm \sqrt{-1}\,J_{2}.
	\end{equation*}
	Then, the following equalities (of elements of $\mathcal{U}\!\left(\mathfrak{so}(3) \overset{{\color{white}.}}{\otimes_{\Re}} \C\right)$) hold:
	\begin{enumerate}
	\item
		$\left[\,J_{3}\,,\,J_{+}\,\right] \;=\; J_{+}$\,,
		\quad
		$\left[\,J_{3}\,,\,J_{-}\,\right] \;=\; -\,J_{-}$\,,
		\quad
		$\left[\,J_{+}\,,\,J_{-}\,\right] \;=\; 2\,J_{3}$
	\item
		$J^{2}$
		\; $=$ \; $(J_{3})^{2} \,-\, J_{3} \,+\, J_{+}J_{-}$
		\; $=$ \; $(J_{3})^{2} \,+\, J_{3} \,+\, J_{-}J_{+}$
	\item
		$(J_{\pm})^{\dagger} \; = \; J_{\mp}$
	\end{enumerate}
\item
	Suppose
	\,$\rho : \mathfrak{so}(3) \otimes_{\Re} \C \longrightarrow \mathfrak{gl}(V)$\,
	is an irreducible finite-dimensional complex representation, and
	\,$v \in V \backslash\{0\}$\, is an eigenvector of \,$\rho(J_{3})$\,
	corresponding to the eigenvalue \,$\lambda \in \Re$;\, thus, \,$\rho(J_{3})(v) \,=\, \lambda\,v$.
	Then, we have:
	\begin{equation*}
	\rho(J_{3})\!\left(\,\rho(J_{+})(\overset{{\color{white}-}}{v})\,\right) \, = \; (\lambda+1)\cdot\rho(J_{+})(v)\,
	\quad\textnormal{and}\quad\;
	\rho(J_{3})\!\left(\,\rho(J_{-})(\overset{{\color{white}-}}{v})\,\right) \, = \; (\lambda-1)\cdot\rho(J_{-})(v)
	\end{equation*}
\item
	\textbf{Casimir operator:}\;\;
	Define
	\,$J^{2}$
	\,$:=$\,
	$(J_{1})^{2} + (J_{2})^{2} + (J_{3})^{2}$
	\,$\in$\
	 $\mathcal{U}\!\left(\mathfrak{so}(3) \overset{{\color{white}.}}{\otimes_{\Re}} \C\right)$.
	Then,
	\begin{equation*}
	\left[\,J^{2}\,\overset{{\color{white}1}}{,}\,J_{k}\,\right]
	\;\; = \;\;
		0\,,
	\quad
	\textnormal{for each \,$k \in \{\,1,2,3\,\}$}\,.
	\end{equation*}
	Consequently (by Schur's Lemma, Corollary 4.30, \cite{Hall2015}), 
	\,$J^{2} \in \mathcal{U}\!\left(\mathfrak{so}(3) \overset{{\color{white}.}}{\otimes_{\Re}} \C\right)$\,
	acts as a scalar multiple of the identity in every irreducible
	representation\footnote{Furthermore, this scalar $\lambda \in \C$ uniquely determines
	the irreducible representation.
	Look up the classification theory of irreducible finite-dimensional complex representations
	of complex semisimple Lie algebras.
	Key words: Casimir operator, universal enveloping algebra. See Chapters 9 and 10, \cite{Hall2015}.}
	of \,$\mathcal{U}\!\left(\mathfrak{so}(3) \overset{{\color{white}.}}{\otimes_{\Re}} \C\right)$;\,
	more precisely, for each irreducible finite-dimensional complex representation
	\,$\rho : \mathcal{U}\!\left(\mathfrak{so}(3) \overset{{\color{white}.}}{\otimes_{\Re}} \C\right) \longrightarrow \mathfrak{gl}(V)$,\,
	we have \,$\rho(J^{2}) = \lambda \cdot \textnormal{\textbf{1}}_{V}$,\,
	for some \,$\lambda \in \C$.
\end{enumerate}
\end{proposition}

          %%%%% ~~~~~~~~~~~~~~~~~~~~ %%%%%

\begin{theorem}
{\color{white}.}\vskip -0.1cm
\noindent
\begin{enumerate}
\item
	The finite-dimensional irreducible representations of $\mathfrak{so}(3) \otimes_{\Re} \C$ is parametrized by the set
	\begin{equation*}
	\dfrac{1}{2} \cdot \Z
	\;\; := \;\;
		\left\{\;0 \,,\, \dfrac{1}{2} \,,\, 1 \,,\, \frac{3}{2} \,,\, 2 \,,\, \frac{5}{2} \,,\, \ldots \;\right\},
	\end{equation*}
	of non-negative integer multiples of \,$\dfrac{1}{2}$, in that, for each
	$s \in \dfrac{1}{2} \cdot \Z = \left\{\; 0 \,,\, \frac{1}{2}\,,\, 1\,,\, \frac{3}{2}\,,\, 2\,,\, \frac{5}{2}\,,\, \ldots \;\right\}$,
	there exists a unique (up to equivalence) complex representation
	$\rho_{s} : \mathcal{U}(\mathfrak{so}(3)\otimes_{\Re}\C) \longrightarrow \textnormal{End}(V_{s})$
	such that
	\begin{equation*}
	\rho_{s}(J^{2}) \; = \; s(s+1)\cdot\textnormal{\textbf{1}}_{V_{s}}.
	\end{equation*}
\item
	$\dim_{\C}(V_{s}) \, = \, 2s + 1$,\, for each
	\,$s \in \dfrac{1}{2} \cdot \Z = \left\{\; 0 \,,\, \frac{1}{2}\,,\, 1\,,\, \frac{3}{2}\,,\, 2\,,\, \frac{5}{2}\,,\, \ldots \;\right\}$.
\item
	For each
	\,$s \in \dfrac{1}{2} \cdot \Z = \left\{\; 0 \,,\, \frac{1}{2}\,,\, 1\,,\, \frac{3}{2}\,,\, 2\,,\, \frac{5}{2}\,,\, \ldots \;\right\}$,\,
	the spectrum
	$\sigma\!\left(\,\overset{{\color{white}-}}{\rho}_{s}(J_{3})\,\right)$
	of the operator $\rho_{s}(J_{3}) \in \textnormal{End}(V_{s})$
	consists of only eigenvalues and is given by:
	\begin{equation*}
	\sigma\!\left(\,\overset{{\color{white}-}}{\rho}_{s}(J_{3})\,\right)
	\;\; = \;\;
		\left\{\;
			-\overset{{\color{white}-}}{s} \,,\, -(s-1), -(s-2)
			\,,\;\, \ldots \,\;,\,
			(s-2) \,,\, (s-1) \,,\, s
			\;\right\},
	\end{equation*}
	and each eigenvalue in 
	$\sigma\!\left(\,\overset{{\color{white}-}}{\rho}_{s}(J_{3})\,\right)$
	has multiplicity one.
\item
	For each
	\,$s \in \dfrac{1}{2} \cdot \Z = \left\{\; 0 \,,\, \frac{1}{2}\,,\, 1\,,\, \frac{3}{2}\,,\, 2\,,\, \frac{5}{2}\,,\, \ldots \;\right\}$,\,
	let \,$v^{(s)}_{k} \in V_{s}\backslash\{0\}$\, be any normalized eigenvector
	of $\rho_{s}(J_{3})$ corresponding to the eigenvalue
	\,$k$ $\in$ $\sigma\!\left(\,\overset{{\color{white}-}}{\rho}_{s}(J_{3})\,\right)$
	$=$ $\left\{\;-\overset{{\color{white}-}}{s} \,,\, -(s-1) \,,\, \;\ldots\;,\, (s-1) \,,\, s\;\right\}$.\,
	Then, 
	\begin{enumerate}
	\item
		the eigenvectors
		\,$v^{(s)}_{-s} \,,\, v^{(s)}_{-(s-1)} \,,\; \ldots \;,\, v^{(s)}_{s-1} \,,\, v^{(s)}_{s}$\,
		form an orthonormal basis for $V_{s}$, and
	\item
		for each \,$k$ $\in$ $\sigma\!\left(\,\overset{{\color{white}-}}{\rho}_{s}(J_{3})\,\right)$
		$=$ $\left\{\;-\overset{{\color{white}-}}{s} \,,\, -(s-1) \,,\, \;\ldots\;,\, (s-1) \,,\, s\;\right\}$,\,
		we have:
		\begin{equation*}
		J_{\pm}\!\left(\,v^{(s)}_{k}\,\right)
		\; = \;
			\sqrt{{\color{white}.}
			s(s+1) - k(k \pm 1)
			{\color{white}.}}
			\,\cdot\,
			v^{(s)}_{k \pm 1}
		\end{equation*}
		In particular, \,$J_{\pm}\!\left(\,v^{(s)}_{\pm s}\,\right) \; = \; 0$.
	\end{enumerate}
\end{enumerate}
\end{theorem}

          %%%%% ~~~~~~~~~~~~~~~~~~~~ %%%%%


%\vskip 0.5cm
%
          %%%%% ~~~~~~~~~~~~~~~~~~~~ %%%%%

\section{Irreducible representations of the Lorentz algebra $\so(1,3)$}
\setcounter{theorem}{0}
\setcounter{equation}{0}

%\cite{vanDerVaart1996}
%\cite{Kosorok2008}

%\renewcommand{\theenumi}{\alph{enumi}}
%\renewcommand{\labelenumi}{\textnormal{(\theenumi)}$\;\;$}
\renewcommand{\theenumi}{\roman{enumi}}
\renewcommand{\labelenumi}{\textnormal{(\theenumi)}$\;\;$}

          %%%%% ~~~~~~~~~~~~~~~~~~~~ %%%%%

% \subsection{Definition \,$\textnormal{O}(1,n)$}

          %%%%% ~~~~~~~~~~~~~~~~~~~~ %%%%%

\vskip 0.5cm
\begin{proposition}[Parametrization of \,$\so(1,3)$]
\mbox{}
\vskip 0.1cm
\noindent
The Lie algebra \,$\so(1,3)$\, of the real Lie group \,$\SO^{\uparrow}(1,3)$\,
admits the following parametrization:
\begin{equation*}
\so{(1,3)}
\; = \;
	\left\{\;
		A
		\overset{{\color{white}.}}{\in}
		\Re^{4 \times 4}
		\;\;\left\vert\;\;
			A \;=\;
			\left(\begin{array}{rrrr}
			        0 &   a_{01} &  a_{02} & a_{03} \\
			a_{01} &           0 &  a_{12} & a_{13} \\
			a_{02} & -a_{12} &           0 & a_{23} \\
			a_{03} & -a_{13} & -a_{23} &          0 \\
			\end{array}\right)
			\right.
		\;\right\}
\end{equation*}
In particular,
\,$\dim_{\Re}\!\left(\overset{{\color{white}.}}{\SO^{\uparrow}(1,3)}\right)$
\,$=$\,
\,$\dim_{\Re}\!\left(\overset{{\color{white}.}}{\so(1,3)}\right)$
\,$=$\, $6$.\,
\end{proposition}
\proof
First, recall that every element of \,$\so(1,3)$\, is of the form \,$\alpha^{\prime}(0) \in \Re^{4 \times 4}$,\,
where
\,$\alpha : (-\varepsilon,\varepsilon) \longrightarrow \SO^{\uparrow}(1,3)$\,
is a smooth map from an open subinterval \,$(-\varepsilon,\varepsilon) \subset \Re$\,
containing \,$0 \in \Re$\, into
\,$\SO^{\uparrow}(1,3)$\,
such that
\,$\alpha(0) = I_{4 \times 4}$.\,
Thus, \,$\alpha(\,\cdot\,)$\, satisfies:
\begin{equation*}
\alpha(t)^{T} \cdot \Qot \cdot \alpha(t) \;\; = \;\; \Qot,
\quad
\textnormal{for each \,$t \in (-\varepsilon,\varepsilon)$}
\end{equation*}
Differentiation with respect to \,$t$\, yields:
\begin{equation*}
\alpha^{\prime}(t)^{T} \cdot \Qot \cdot \alpha(t) \;+\; \alpha(t)^{T} \cdot \Qot \cdot \alpha^{\prime}(t) \;\; = \;\; 0_{4 \times 4}
\end{equation*}
Evaluating at \,$t = 0$\, and recalling \,$\alpha(0) = I_{4 \times 4}$\, yields:
\begin{equation*}
\alpha^{\prime}(0)^{T} \cdot \Qot \;+\; \Qot \cdot \alpha^{\prime}(0) \;\; = \;\; 0_{4 \times 4}
\end{equation*}
Now, write:
\begin{equation*}
\alpha^{\prime}(0)
\;\; = \;\;
	\left(\begin{array}{cccc}
	a_{00} & a_{01} & a_{02} & a_{03}
	\\
	a_{10} & a_{11} & a_{12} & a_{13}
	\\
	a_{20} & a_{21} & a_{22} & a_{23}
	\\
	a_{30} & a_{31} & a_{32} & a_{33}
	\end{array}\right)
\;\; \in \;\;
	\Re^{4 \times 4}
\end{equation*}
Then,
\begin{equation*}
\alpha^{\prime}(0)^{T} \cdot \Qot
\;\; = \;\;
	\left(\begin{array}{cccc}
	a_{00} & a_{10} & a_{20} & a_{30}
	\\
	a_{01} & a_{11} & a_{21} & a_{31}
	\\
	a_{02} & a_{12} & a_{22} & a_{32}
	\\
	a_{03} & a_{13} & a_{23} & a_{33}
	\end{array}\right)
	\cdot
	\left(\begin{array}{rrrr}
	-1 & 0 & 0 & 0
	\\
	0 & 1 & 0 & 0
	\\
	0 & 0 & 1 & 0
	\\
	0 & 0 & 0 & 0
	\end{array}\right)
\;\; = \;\;
	\left(\begin{array}{cccc}
	-\,a_{00} & a_{10} & a_{20} & a_{30}
	\\
	-\,a_{01} & a_{11} & a_{21} & a_{31}
	\\
	-\,a_{02} & a_{12} & a_{22} & a_{32}
	\\
	-\,a_{03} & a_{13} & a_{23} & a_{33}
	\end{array}\right)
\end{equation*}
and
\begin{equation*}
\Qot \cdot \alpha^{\prime}(0)
\;\; = \;\;
	\left(\begin{array}{rrrr}
	-1 & 0 & 0 & 0
	\\
	0 & 1 & 0 & 0
	\\
	0 & 0 & 1 & 0
	\\
	0 & 0 & 0 & 0
	\end{array}\right)
	\cdot
	\left(\begin{array}{cccc}
	a_{00} & a_{01} & a_{02} & a_{03}
	\\
	a_{10} & a_{11} & a_{12} & a_{13}
	\\
	a_{20} & a_{21} & a_{22} & a_{23}
	\\
	a_{30} & a_{31} & a_{32} & a_{33}
	\end{array}\right)
\;\; = \;\;
	\left(\begin{array}{rrrr}
	-\,a_{00} & -\,a_{01} & -\,a_{02} & -\,a_{03}
	\\
	a_{10} & a_{11} & a_{12} & a_{13}
	\\
	a_{20} & a_{21} & a_{22} & a_{23}
	\\
	a_{30} & a_{31} & a_{32} & a_{33}
	\end{array}\right)
\end{equation*}
Thus,
\begin{equation*}
\alpha^{\prime}(0)^{T} \cdot \Qot \;+\; \Qot \cdot \alpha^{\prime}(0)
\;\; = \;\;
	\left(\begin{array}{cccc}
	-\,2\,a_{00} & a_{10}\,-\,a_{01} & a_{20}\,-\,a_{02} & a_{30}\,-\,a_{03}
	\\
	-\,a_{01} \,+\, a_{10} & 2\,a_{11} & a_{21}\,+\,a_{12} & a_{31}\,+\,a_{13}
	\\
	-\,a_{02} \,+\, a_{20} & a_{12}\,+\,a_{21} & 2\,a_{22} & a_{32}\,+\,a_{23}
	\\
	-\,a_{03} \,+\, a_{30} & a_{13}\,+\,a_{31} & a_{23}\,+\,a_{32} & 2\,a_{33}
	\end{array}\right)
\end{equation*}
Consequently,
\begin{equation*}
\alpha^{\prime}(0)^{T} \cdot \Qot \;+\; \Qot \cdot \alpha^{\prime}(0) \;\; = \;\; 0_{4 \times 4}
\quad\Longleftrightarrow\quad
\left\{\begin{array}{c}
	a_{00} \,=\, a_{11} \,=\, a_{22} \,=\, a_{33} \,=\, 0\,,
	\\
	a_{10} \,=\, a_{01}\,, a_{20} \,=\, a_{02}\,, a_{30} \,=\, a_{03}\,,
	\\
	a_{21} \,=\, -\,a_{12}\,, a_{31} \,=\, -\,a_{13}\,, a_{32} \,=\, -\,a_{23}\,.
	\end{array}\right.
\end{equation*}
This proves:
\begin{equation*}
\so{(1,3)}
\;\; \subset \;\;
	\left\{\;
		A
		\overset{{\color{white}.}}{\in}
		\Re^{4 \times 4}
		\;\;\left\vert\;\;
			A \;=\;
			\left(\begin{array}{rrrr}
			        0 &   a_{01} &  a_{02} & a_{03} \\
			a_{01} &           0 &  a_{12} & a_{13} \\
			a_{02} & -a_{12} &           0 & a_{23} \\
			a_{03} & -a_{13} & -a_{23} &          0 \\
			\end{array}\right)
			\right.
		\;\right\}
\end{equation*}
For the reverse inclusion, consider:
\begin{equation*}
A
\;\; = \;\;
	\left(\begin{array}{rrrr}
	        0 &   a_{01} &  a_{02} & a_{03} \\
		a_{01} &           0 &  a_{12} & a_{13} \\
		a_{02} & -a_{12} &           0 & a_{23} \\
		a_{03} & -a_{13} & -a_{23} &          0 \\
		\end{array}\right)
\;\; \in \;\;
	\Re^{4 \times 4}
\end{equation*}
The required reverse inclusion amounts to the statement that \,$A \in \so(1,3)$.\,
Now, define:
\begin{equation*}
\alpha : \Re \longrightarrow \Re^{4 \times 4} : t \longmapsto \exp\!\left(\,t \cdot \overset{{\color{white}.}}{A}\,\right)
\end{equation*}
Then, \,$\alpha^{\prime}(0) \,=\, A$.\,
Thus, to show that \,$A \in \so(1,3)$,\, it remains only to establish that
\,$\alpha(t) \in \SO^{\uparrow}(1,3)$.\,

\vskip 0.2cm
\noindent
To this end, first note that \,$A \in \Re^{4 \times 4}$\, satisfies:
\,$A^{T} \cdot \Qot + \Qot \cdot A \,=\, 0_{4 \times 4}$,\,
which implies:
\begin{equation*}
A^{T} \cdot \Qot \,=\, -\,\Qot \cdot A\,,
\end{equation*}
which in turn implies:
\begin{equation*}
\left(\,A^{T}\,\right)^{k} \cdot \Qot
\,=\,
	(-1) \cdot \left(\,A^{T}\,\right)^{k-1} \cdot \Qot \cdot A
\,=\,
	(-1)^{2} \cdot \left(\,A^{T}\,\right)^{k-2} \cdot \Qot \cdot A^{2}
\,=\,
	\cdots
\,=\,
	(-1)^{k} \cdot \Qot \cdot A^{k}
\end{equation*}
Therefore,
\begin{equation*}
\exp\!\left(\,t\overset{{\color{white}.}}{A}\,\right)^{T} \cdot \Qot \cdot \exp\!\left(\,t\overset{{\color{white}.}}{A}\,\right)
\,=\,
	\exp\!\left(\,\overset{{\color{white}.}}{t}A^{T}\,\right) \cdot \Qot \cdot \exp\!\left(\,t\overset{{\color{white}.}}{A}\,\right)
\,=\,
	\Qot \cdot \exp\!\left(\,-\,\overset{{\color{white}.}}{t}A\,\right) \cdot \exp\!\left(\,t\overset{{\color{white}.}}{A}\,\right)
\,=\,
	\Qot\,,
\end{equation*}
where the second last equality follows from:
\begin{eqnarray*}
\Qot \cdot \exp\!\left(\,-\,\overset{{\color{white}.}}{t}A\,\right)
& = &
	\Qot \cdot \left(\;\overset{\infty}{\underset{k=0}{\sum}}\,\dfrac{(-1)^{k}t^{k}A^{k}}{k!}\,\right)
\;\; = \;\;
	\left(\;\overset{\infty}{\underset{k=0}{\sum}}\,\dfrac{t^{k}\cdot(-1)^{k}\,\Qot\,A^{k}}{k!}\,\right)
\\
& = &
	\left(\;\overset{\infty}{\underset{k=0}{\sum}}\,\dfrac{t^{k}\cdot(A^{T})^{k}\,\Qot}{k!}\,\right)
\;\; = \;\;
	\left(\;\overset{\infty}{\underset{k=0}{\sum}}\,\dfrac{t^{k}\,(A^{T})^{k}}{k!}\,\right)
	\cdot\Qot
\\
& = &
	\exp\!\left(\,\overset{{\color{white}.}}{t}A^{T}\,\right) \cdot \Qot
\end{eqnarray*}
Thus, we now see that
\,$\alpha(t) \in \textnormal{O}(1,3)$,\, for each \,$t \in \Re$.\,
Next, recall that
\begin{equation*}
\det\!\left(\,\exp(t\overset{{\color{white}.}}{X})\,\right)
\,=\,
	e^{\trace(tX)}\,,
\quad
\textnormal{for each \,$t \in \Re$\, and \,$X \in \Re^{4 \times 4}$}
\end{equation*}
Hence,
\begin{equation*}
\det\!\left(\, \alpha(\overset{{\color{white}.}}{t}) \,\right)
\,=\,
	\det\!\left(\,\exp(t\overset{{\color{white}.}}{A})\,\right)
\,=\,
	e^{\trace(tA)}
\,=\,
	e^{0}
\,=\,
	1
\end{equation*}
Thus, we see furthermore that
\begin{equation*}
\alpha(t) \,\in\, \SO(1,3).
\end{equation*}
However, we also have \,$\alpha(0) \,=\, \exp(0\cdot\!A) \,=\, I_{4 \times 4}$.\,
Continuity of \,$\alpha(\,\cdot\,)$\, now implies that \,$\alpha(\,\cdot\,)$\,
must map all of \,$\Re$\, into the identity component of \,$\SO(1,3)$,\,
i.e., \,$\alpha(t) \in \SO^{\uparrow}(1,3)$,\,
for each \,$t \in \Re$.\,
This proves the reverse inclusion:
\begin{equation*}
\so{(1,3)}
\;\; \supset \;\;
	\left\{\;
		A
		\overset{{\color{white}.}}{\in}
		\Re^{4 \times 4}
		\;\;\left\vert\;\;
			A \;=\;
			\left(\begin{array}{rrrr}
			        0 &   a_{01} &  a_{02} & a_{03} \\
			a_{01} &           0 &  a_{12} & a_{13} \\
			a_{02} & -a_{12} &           0 & a_{23} \\
			a_{03} & -a_{13} & -a_{23} &          0 \\
			\end{array}\right)
			\right.
		\;\right\}
\end{equation*}
and thus completes the proof of the Proposition.
\qed

          %%%%% ~~~~~~~~~~~~~~~~~~~~ %%%%%

\vskip 0.5cm
\begin{corollary}[Generators of \,$\so(1,3)$\, \& their commutation relations]
\mbox{}
\vskip 0.1cm
\noindent
Define the following six matrices with real entries:
\begin{equation*}
R_{23}
\; := \,
	\left(\,\begin{array}{rrrr}
	0 & {\color{white}-}0 & {\color{white}-}0 & {\color{white}-}0 \\
	0 & 0 & 0 & 0 \\
	0 & 0 & 0 & -1 \\
	0 & 0 & 1 & 0 \\
	\end{array}\right),
\;\;
R_{31}
\; := \,
	\left(\,\begin{array}{rrrr}
	0 & {\color{white}-}0 & {\color{white}-}0 & {\color{white}-}0 \\
	0 & 0 & 0 & 1 \\
	0 & 0 & 0 & 0 \\
	0 & -1 & 0 & 0 \\
	\end{array}\right),
\;\;
R_{12}
\; := \,
	\left(\,\begin{array}{rrrr}
	0 & {\color{white}-}0 & {\color{white}-}0 & {\color{white}-}0 \\
	0 & 0 & -1 & 0 \\
	0 & 1 & 0 & 0 \\
	0 & 0 & 0 & 0 \\
	\end{array}\right)
\end{equation*}
\begin{equation*}
B_{01}
\; := \,
	\left(\,\begin{array}{rrrr}
	0 & {\color{white}-}1 & {\color{white}-}0 & {\color{white}-}0 \\
	1 & 0 & 0 & 0 \\
	0 & 0 & 0 & 0 \\
	0 & 0 & 0 & 0 \\
	\end{array}\right),
\;\;
B_{02}
\; := \,
	\left(\,\begin{array}{rrrr}
	0 & {\color{white}-}0 & {\color{white}-}1 & {\color{white}-}0 \\
	0 & 0 & 0 & 0 \\
	1 & 0 & 0 & 0 \\
	0 & 0 & 0 & 0 \\
	\end{array}\right),
\;\;
B_{03}
\; := \,
	\left(\,\begin{array}{rrrr}
	0 & {\color{white}-}0 & {\color{white}-}0 & {\color{white}-}1 \\
	0 & 0 & 0 & 0 \\
	0 & 0 & 0 & 0 \\
	1 & 0 & 0 & 0 \\
	\end{array}\right)
\end{equation*}
\vskip 0.3cm
\noindent
Define also the following six matrices with complex entries:
\vskip -0.9cm
\mbox{}
\begin{multicols}{2}
	\begin{minipage}{6.0cm}
	\begin{eqnarray*}
	J_{1}
	& := &
		\i \cdot R_{23}
	\;\; = \;\;
		\left(\,\begin{array}{rrrr}
		0 & {\color{white}-}0 & {\color{white}-}0 & {\color{white}-}0 \\
		0 & 0 & 0 & 0 \\
		0 & 0 & 0 & -\i \\
		0 & 0 & \i & 0 \\
		\end{array}\right),
	\\
	J_{2}
	& := &
		\i \cdot R_{31}
	\;\; = \;\;
		\left(\,\begin{array}{rrrr}
		0 & {\color{white}-}0 & {\color{white}-}0 & {\color{white}-}0 \\
		0 & 0 & 0 &  \i \\
		0 & 0 & 0 & 0 \\
		0 & -\i & 0 & 0 \\
		\end{array}\right),
	\\
	J_{3}
	& := &
		\i \cdot R_{12}
	\;\; = \;\;
		\left(\,\begin{array}{rrrr}
		0 & {\color{white}-}0 & {\color{white}-}0 & {\color{white}-}0 \\
		0 & 0 & -\i & 0 \\
		0 & \i & 0 & 0 \\
		0 & 0 & 0 & 0 \\
		\end{array}\right),
	\end{eqnarray*}
	\end{minipage}
\columnbreak
	\begin{minipage}{11.5cm}
	\begin{eqnarray*}
	K_{1}
	& := &
		\i \cdot B_{01}
	\;\; = \;\;
		\left(\,\begin{array}{rrrr}
		0 & {\color{white}-}\i & {\color{white}-}0 & {\color{white}-}0 \\
		\i & 0 & 0 & 0 \\
		0 & 0 & 0 & 0 \\
		0 & 0 & 0 & 0 \\
		\end{array}\right),
	\\
	K_{2}
	& := &
		\i \cdot B_{02}
	\;\; = \;\;
		\left(\,\begin{array}{rrrr}
		0 & {\color{white}-}0 & {\color{white}-}\i & {\color{white}-}0 \\
		0 & 0 & 0 & 0 \\
		\i & 0 & 0 & 0 \\
		0 & 0 & 0 & 0 \\
		\end{array}\right),
	\\
	K_{3}
	& := &
		\i \cdot B_{03}
	\;\; = \;\;
		\left(\,\begin{array}{rrrr}
		0 & {\color{white}-}0 & {\color{white}-}0 & {\color{white}-}\i \\
		0 & 0 & 0 & 0 \\
		0 & 0 & 0 & 0 \\
		\i & 0 & 0 & 0 \\
		\end{array}\right).
	\end{eqnarray*}
	\end{minipage}
\end{multicols}
\begin{multicols}{2}
	\begin{minipage}{8cm}
	\begin{eqnarray*}
	N^{+}_{1}
	\; := \;
		\dfrac{1}{2}\left(\,J_{1} + \i \, K_{1}\,\right)
	\; = \;
		\dfrac{1}{2}\,\cdot
		\left(\!\begin{array}{rrrr}
		 0 & {\color{black}-}1 & {\color{white}-}0 & {\color{white}-}0 \\
		-1 & 0 & 0 & 0 \\
		 0 & 0 & 0 & -\i \\
		 0 & 0 & \i & 0 \\
		\end{array}\right),
	\quad\quad{\color{white}.}
	\\
	N^{+}_{2}
	\; := \;
		\dfrac{1}{2}\left(\,J_{2} + \i \, K_{2}\,\right)
	\; = \;
		\dfrac{1}{2}\,\cdot
		\left(\!\begin{array}{rrrr}
		 0 & {\color{white}-}0 & {\color{black}-}1 & {\color{white}-}0 \\
		 0 & 0 & 0 & \i \\
		-1 & 0 & 0 & 0 \\
		 0 & -\i & 0 & 0 \\
		\end{array}\right),
	\quad\quad{\color{white}.}
	\\
	N^{+}_{3}
	\; := \;
		\dfrac{1}{2}\left(\,J_{3} + \i \, K_{3}\,\right)
	\; = \;
		\dfrac{1}{2}\,\cdot
		\left(\!\begin{array}{rrrr}
		 0 & {\color{white}-}0 & {\color{white}-}0 & {\color{black}-}1 \\
		 0 & 0 & -\i & 0 \\
		 0 &  \i & 0 & 0 \\
		-1 & 0 & 0 & 0 \\
		\end{array}\right),
	\quad\quad{\color{white}.}
	\end{eqnarray*}
	\end{minipage}
\columnbreak
	\begin{minipage}{12.0cm}
	\begin{eqnarray*}
	N^{-}_{1}
	\; := \;
		\dfrac{1}{2}\left(\,J_{1} - \i \, K_{1}\,\right)
	\; = \;
		\dfrac{1}{2}\,\cdot
		\left(\!\!\!\begin{array}{rrrr}
		{\color{white}-}0 & {\color{white}-}1 & {\color{white}-}0 & {\color{white}-}0 \\
		1 & 0 & 0 & 0 \\
		0 & 0 & 0 & -\i \\
		0 & 0 & \i & 0 \\
		\end{array}\right),
	\\
	N^{-}_{2}
	\; := \;
		\dfrac{1}{2}\left(\,J_{2} - \i \, K_{2}\,\right)
	\; = \;
		\dfrac{1}{2}\,\cdot
		\left(\!\!\!\begin{array}{rrrr}
		{\color{white}-}0 & {\color{white}-}0 & {\color{white}-}1 & {\color{white}-}0 \\
		0 & 0 & 0 & \i \\
		1 & 0 & 0 & 0 \\
		0 & -\i & 0 & 0 \\
		\end{array}\right),
	\\
	N^{-}_{3}
	\; := \;
		\dfrac{1}{2}\left(\,J_{3} - \i \, K_{3}\,\right)
	\; = \;
		\dfrac{1}{2}\,\cdot
		\left(\!\!\!\begin{array}{rrrr}
		{\color{white}-}0 & {\color{white}-}0 & {\color{white}-}0 & {\color{white}-}1 \\
		0 & 0 & -\i & 0 \\
		0 &  \i & 0 & 0 \\
		1 & 0 & 0 & 0 \\
		\end{array}\right).
	\end{eqnarray*}
	\end{minipage}
\end{multicols}
\noindent
Then, the following statements are true:
\begin{enumerate}
\item
	The matrices
	\,$R_{23},\; R_{31},\; R_{12},\; B_{01},\; B_{02},\; B_{03}$\,
	are elements of
	\,$\so(1,3)$,\, and they form a basis for \,$\so(1,3)$.\,
	The
	\,$15 \,= \left(\begin{array}{c}6 \\ 2\end{array}\right)$\,
	commutation relations satisfied by
	\,$R_{23},\, R_{31},\, R_{12},\, B_{01},\, B_{02},\, B_{03}$\,
	are:
	\begin{equation*}
	\begin{array}{lll}
	\left[\,R_{23}\,,\,R_{31}\,\right] \,=\, +\,R_{12}, &
	\left[\,R_{12}\,,\,R_{23}\,\right] \,=\, +\,R_{31}, &
	\left[\,R_{31}\,,\,R_{12}\,\right] \,=\, +\,R_{23},
	\\ \\
	\left[\,B_{01}\,,\,B_{02}\,\right] \,=\, -\,R_{12}, &
	\left[\,B_{03}\,,\,B_{01}\,\right] \,=\, -\,R_{31}, &
	\left[\,B_{02}\,,\,B_{03}\,\right] \,=\, -\,R_{23},
	\\ \\
	\left[\,R_{23}\,,\,B_{01}\,\right] \,=\, {\color{white}-}\,0,\;\;\;\; &
	\left[\,R_{23}\,,\,B_{02}\,\right] \,=\, +\,B_{03}, &
	\left[\,R_{23}\,,\,B_{03}\,\right] \,=\, -\,B_{02}, 
	\\
	\left[\,R_{31}\,,\,B_{01}\,\right] \,=\, -\,B_{03}, &
	\left[\,R_{31}\,,\,B_{02}\,\right] \,=\, {\color{white}-}\,0,\;\;\;\; &
	\left[\,R_{31}\,,\,B_{03}\,\right] \,=\, +\,B_{01},
	\\
	\left[\,R_{12}\,,\,B_{01}\,\right] \,=\, +\,B_{02}, &
	\left[\,R_{12}\,,\,B_{02}\,\right] \,=\, -\,B_{01}, &
	\left[\,R_{12}\,,\,B_{03}\,\right] \,=\, {\color{white}-}\,0,\;\;\;\;
	\end{array}
	\end{equation*}
\item
	The matrices
	\,$J_{1},\, J_{2},\, J_{3},\, K_{1},\, K_{2},\, K_{3}$\,
	are elements of
	\,$\so(1,3) \otimes_{\Re} \C$,\, and they form a basis for \,$\so(1,3) \otimes_{\Re} \C$.\,
	The
	\,$15 \,= \left(\begin{array}{c}6 \\ 2\end{array}\right)$\,
	commutation relations satisfied by
	\,$J_{1},\, J_{2},\, J_{3},\, K_{1},\, K_{2},\, K_{3}$\,
	are:
	\begin{equation*}
	\begin{array}{lll}
	\left[\,\;J_{1}\,,\,\;J_{2}\,\right] \,=\, + \, \i \, J_{3}, &
	\left[\,\;J_{3}\,,\,\;J_{1}\,\right] \,=\, + \, \i \, J_{2}, &
	\left[\,\;J_{2}\,,\,\;J_{3}\,\right] \,=\, + \, \i \, J_{1},
	\\ \\
	\left[\,K_{1}\,,\,K_{2}\,\right] \,=\, - \, \i \, J_{3}, &
	\left[\,K_{3}\,,\,K_{1}\,\right] \,=\, - \, \i \, J_{2}, &
	\left[\,K_{2}\,,\,K_{3}\,\right] \,=\, - \, \i \, J_{1},
	\\ \\
	\left[\,\;J_{1}\,,\,K_{1}\,\right] \,=\, {\color{white}-}\,0,\;\;\;\; &
	\left[\,\;J_{1}\,,\,K_{2}\,\right] \,=\, + \, \i \, K_{3}, &
	\left[\,\;J_{1}\,,\,K_{3}\,\right] \,=\, - \, \i \, K_{2},
	\\
	\left[\,\;J_{2}\,,\,K_{1}\,\right] \,=\, - \, \i \, K_{3}, &
	\left[\,\;J_{2}\,,\,K_{2}\,\right] \,=\, {\color{white}-}\,0,\;\;\;\; &
	\left[\,\;J_{2}\,,\,K_{3}\,\right] \,=\, + \, \i \, K_{1},
	\\
	\left[\,\;J_{3}\,,\,K_{1}\,\right] \,=\, + \, \i \, K_{2}, &
	\left[\,\;J_{3}\,,\,K_{2}\,\right] \,=\, - \, \i \, K_{1}, &
	\left[\,\;J_{3}\,,\,K_{3}\,\right] \,=\, {\color{white}-}\,0,\;\;\;\;
	\end{array}
	\end{equation*}
\item
	The matrices
	\,$N^{\pm}_{a}$,\, for \,$a \,\in\, \{\,1,2,3\,\}$,
	are elements of
	\,$\so(1,3) \otimes_{\Re} \C$,\, and they form a basis for \,$\so(1,3) \otimes_{\Re} \C$.\,
	The elements
	\,$N^{\pm}_{a}$,\, for \,$a \,\in\, \{\,1,2,3\,\}$,
	satisfy the following commutation relations:
	\begin{equation*}
	\left[\,N^{\pm}_{a}\,,\,N^{\pm}_{b}\,\right]
	\;\;=\;\;
		\i \cdot \overset{3}{\underset{c\,=\,1}{\sum}} \;\varepsilon_{abc} \, N^{\pm}_{c}\,,
	\quad
	\textnormal{for each \,$a, b, c \,\in\, \{\,1, 2, 3\,\}$}
	\end{equation*}
	\begin{equation*}
	\left[\,N^{+}_{a}\,,\,N^{-}_{b}\,\right] \;\;=\;\; 0\,,
	\quad
	\textnormal{for each \,$a, b \,\in\, \{\,1, 2, 3\,\}$}
	\end{equation*}
	Thus,
	\,$\so(1,3) \otimes_{\Re} \C$\,
	contains two commuting copies of
	\,$\su(2) \otimes_{\Re} \C$.\,
\end{enumerate}
\end{corollary}
\proof

\qed

          %%%%% ~~~~~~~~~~~~~~~~~~~~ %%%%%

%\subsection{Generators of \,$\mathfrak{su}(2)$}
%
%          %%%%% ~~~~~~~~~~~~~~~~~~~~ %%%%%
%
%\begin{proposition}[Characterizations of \,$\mathfrak{sl}(n)$, \,$\mathfrak{u}(n)$, and $\mathfrak{su}(n)$]
%\mbox{}
%\vskip 0.1cm
%\begin{enumerate}
%\item
%	\begin{equation*}
%	\mathfrak{sl}(n,\C)
%	\; = \;
%		\left\{\;
%			X \,\in\, \mathfrak{gl}(n,\C) \,=\, \C^{n \times n}
%			\;\left\vert\;\,
%				\textnormal{trace}(X) = \overset{{\color{white}1}}{0}
%				\right.
%			\,\right\}
%	\end{equation*}
%\item
%	\begin{equation*}
%	\mathfrak{u}(n)
%	\; = \;
%		\left\{\;
%			X \,\in\, \mathfrak{gl}(n,\C) \,=\, \C^{n \times n}
%			\;\left\vert\;\,
%				X + X^{\dagger} = \overset{{\color{white}1}}{0}
%				\right.
%			\,\right\}
%	\end{equation*}
%\item
%	\begin{equation*}
%	\mathfrak{su}(n)
%	\; = \;
%		\left\{\;
%			X \,\in\, \mathfrak{gl}(n,\C) \,=\, \C^{n \times n}
%			\;\left\vert\;\,
%				\begin{array}{c}
%				X + X^{\dagger} = \overset{{\color{white}1}}{0}
%				\\
%				\textnormal{trace}(X) = \overset{{\color{white}1}}{0}
%				\end{array}
%				\right.
%			\,\right\}
%	\end{equation*}
%\end{enumerate}
%\end{proposition}
%\proof
%\begin{enumerate}
%\item
%	We invoke the fact that \,$\det(e^{\,t\,\cdot\,X}) \,=\, e^{\,t\,\cdot\,\textnormal{trace}(X)}$,
%	for each \,$X \in \C^{n \times n}$.
%	Thus,
%	\begin{eqnarray*}
%	&&
%		X \,\in\, \mathfrak{sl}(n,\C)
%		\quad\Longrightarrow\quad
%		e^{\,t\cdot\,X} \,\in\, \textnormal{SL}(n,\C)
%		\quad\Longrightarrow\quad
%		\det\!\left(\,e^{\,t\cdot\,X}\,\right) \,=\, 1
%	\\
%	& \Longrightarrow\quad &
%		\textnormal{trace}(X)
%		\; = \;
%			\left.\dfrac{\d}{\d\,t}\right\vert_{t=0}\left(\,\overset{{\color{white}1}}{e^{\,t\,\cdot\,\textnormal{trace}(X)}}\,\right)
%		\; = \;
%			\left.\dfrac{\d}{\d\,t}\right\vert_{t=0}\left(\,\overset{{\color{white}1}}{\det(e^{\,t\,\cdot\,X})}\,\right)
%		\; = \;
%			\left.\dfrac{\d}{\d\,t}\right\vert_{t=0}\left(\,\overset{{\color{white}1}}{1}\,\right)
%		\; = \;
%			0
%	\end{eqnarray*}
%	Conversely, suppose \,$\textnormal{trace}(X) = 0$.\,
%	Then, \,$\det(e^{\,t\,\cdot\,X}) \,=\, e^{\,t\,\cdot\,\textnormal{trace}(X)} \,=\, e^{\,t\,\cdot\,0} \,=\, 1$,\,
%	which implies that \,$e^{\,t\,\cdot\,X} \,\in\, \textnormal{SL}(n,\C)$,\, hence \,$X \,\in\, \mathfrak{sl}(n,\C)$.
%	This completes the proof of the equality (of sets) in question.
%\item
%	\begin{eqnarray*}
%	&&
%		X \,\in\, \mathfrak{u}(n)
%		\quad\Longrightarrow\quad
%		e^{\,t\,\cdot\,X} \,\in\, \textnormal{U}(n)
%	\\
%	& \Longrightarrow\quad &
%		I_{n}
%			\,=\, \left(\,e^{\,t\,\cdot\,X}\,\right)^{\!\dagger} \cdot \left(\,e^{\,t\,\cdot\,X}\,\right)
%			\,=\, \left(\,e^{\,t\,\cdot\,X^{\dagger}}\,\right) \cdot \left(\,e^{\,t\,\cdot\,X}\,\right)
%			\,=\, e^{\,t\,\cdot\,(X^{\dagger}+X)}
%	\\
%	& \Longrightarrow\quad &
%		X \,+\, X^{\dagger}
%		\; = \;
%			\left.\dfrac{\d}{\d\,t}\right\vert_{t=0}\left(\,\overset{{\color{white}1}}{e^{\,t\,\cdot\,(X+X^{\dagger})}}\,\right)
%		\; = \;
%			\left.\dfrac{\d}{\d\,t}\right\vert_{t=0}\left(\,\overset{{\color{white}1}}{I_{n}}\,\right)
%		\; = \;
%			0
%	\end{eqnarray*}
%	Conversely, suppose \,$X + X^{\dagger} \,=\, 0$.\,
%	Then, \,$I_{n}$
%	\,$=$\, $e^{\,0_{n \times n}}$
%	\,$=$\, $e^{\,t\,\cdot(X^{\dagger}+X)}$
%	\,$=\, \cdots \,=$\, $\left(e^{\,t\,\cdot\,X}\right)^{\!\dagger}\cdot\left(e^{\,t\,\cdot\,X}\right)$,\,
%	which implies that \,$e^{\,t\,\cdot\,X} \,\in\, \textnormal{U}(n)$,\, hence \,$X \,\in\, \mathfrak{u}(n)$.
%	This completes the proof of the equality (of sets) in question.
%\item
%	Immediate by the preceding two statements.
%\end{enumerate}
%\qed
%
%\vskip 0.5cm
%\begin{proposition}[Generators of \,$\mathfrak{su}(2)$]
%\mbox{}
%\vskip 0.1cm
%\noindent
%Let \,$\sigma_{1},\, \sigma_{2},\, \sigma_{3} \,\in\, \C^{2 \times 2}$\, be the \textbf{Pauli spin matrices}, i.e.,
%\begin{equation*}
%\sigma_{1} \,=\, \sigma_{x} \,:=\, \left(\begin{array}{cc} 0 & 1 \\ 1 & 0 \end{array}\right),
%\quad
%\sigma_{2} \,=\, \sigma_{y} \,:=\, \left(\begin{array}{rr} 0 & -\i \\ \i & 0 \end{array}\right),
%\quad
%\sigma_{3} \,=\, \sigma_{z} \,:=\, \left(\begin{array}{rr} 1 & 0 \\ 0 & -1 \end{array}\right).
%\end{equation*}
%Define \,$J_{1},\, J_{2},\, J_{3},\, S_{+},\, S_{-},\, S_{3} \,\in\, \C^{2 \times 2}$\, as follows:
%\begin{equation*}
%J_{1} \,:=\, \dfrac{\i}{2}\cdot\sigma_{1} \,=\, \dfrac{\i}{2}\cdot\left(\begin{array}{cc} 0 & 1 \\ 1 & 0 \end{array}\right),
%\quad
%J_{2} \,:=\, \mathbf{{\color{red}-}}\,\dfrac{\i}{2}\cdot\sigma_{2} \,=\, \dfrac{1}{2}\cdot\left(\begin{array}{rr} 0 & -1 \\ 1 & 0 \end{array}\right),
%\quad
%J_{3} \,:=\, \dfrac{\i}{2}\cdot\sigma_{3} \,=\, \dfrac{\i}{2}\cdot\left(\begin{array}{rr} 1 & 0 \\ 0 & -1 \end{array}\right),
%\end{equation*}
%\begin{equation*}
%S_{+} \,:=\, \dfrac{1}{\i}\left(\,J_{1} + \i\,J_{2}\,\right) \,=\, \left(\begin{array}{cc} 0 & 0 \\ 1 & 0 \end{array}\right),
%\quad
%S_{-} \,:=\, \dfrac{1}{\i}\left(\,J_{1} - \i\,J_{2}\,\right) \,=\, \left(\begin{array}{rr} 0 & 1 \\ 0 & 0 \end{array}\right),
%\quad
%S_{3} \,:=\, \i\cdot J_{3} \,=\, \dfrac{1}{2}\cdot\left(\begin{array}{rr} -1 & 0 \\ 0 & 1 \end{array}\right).
%\end{equation*}
%Then, the following statements are true:
%\begin{enumerate}
%\item
%	$J_{1},\, J_{2},\, J_{3} \,\in\, \mathfrak{su}(2)$\,
%\item
%	$J_{1},\, J_{2},\, J_{3}$\,
%	form a set of generators for the (real) Lie algebra \,$\mathfrak{su}(2)$\, of the (real) Lie group \,$\textnormal{SU}(2)$.
%\item
%	$J_{1},\, J_{2},\, J_{3}$\, satisfy the following commutation relations:
%	\begin{equation*}
%	\left[\,J_{a}\,,\,J_{b}\,\right] \;\; = \;\; \overset{3}{\underset{c\,=\,1}{\sum}}\;\epsilon_{abc}\,J_{c}\,,
%	\quad
%	\textnormal{for \,$a, b = 1,2,3$}.
%	\end{equation*}
%\item
%	$S_{+},\, S_{-},\, S_{3} \,\in\, \mathfrak{su}(2) \otimes_{\Re} \C$,\,
%	where
%	\,$\mathfrak{su}(2) \otimes_{\Re} \C$\,
%	is the complexification of (the real Lie algebra)
%	\,$\mathfrak{su}(2)$.
%\item
%	$S_{+},\; S_{-},\; S_{3}$\, satisfy the following commutation relations:
%	\begin{equation*}
%	\left[\,S_{+}\,,\,S_{-}\,\right] \, = \, 2\,S_{3}\,,
%	\quad
%	\left[\,S_{3}\,,\,S_{\pm}\,\right] \, = \, \pm\,S_{\pm}
%	\end{equation*}
%\item
%	Suppose
%	\begin{itemize}
%	\item
%		$V$\, is a complex vector space,
%	\item
%		$\rho : \mathfrak{su}(2) \otimes_{\Re} \C \longrightarrow \textnormal{End}(V)$\,
%		is a Lie algebra representation, and
%	\item	
%		$v \in V$\, and \,$\lambda \in \C$\, together satisfy \,$\rho(S_{3})(v) = \lambda \cdot v$.
%	\end{itemize}	
%	Then, \,$\rho(S_{+})(v) \,\in\, V$\, satisfies:
%	\begin{equation*}
%	\rho(S_{3})\!\left(\,\rho(\overset{{\color{white}.}}{S}_{+})(v)\,\right)
%	\; = \;
%		(\lambda+1) \cdot \rho(S_{+})(v)\,
%	\end{equation*}
%	and
%	\,$\rho(S_{-})(v) \,\in\, V$\, satisfies:
%	\begin{equation*}
%	\rho(S_{3})\!\left(\,\rho(\overset{{\color{white}.}}{S}_{-})(v)\,\right)
%	\; = \;
%		(\lambda-1) \cdot \rho(S_{-})(v)\,
%	\end{equation*}
%\end{enumerate}
%\end{proposition}
%\proof
%The Corollary follows straightforwardly by direct computations.
%\qed
%
%          %%%%% ~~~~~~~~~~~~~~~~~~~~ %%%%%
%
%\vskip 0.5cm
%\begin{definition}[$\textnormal{O}(n)$ and $\textnormal{SO}(n)$]
%\mbox{}
%\vskip 0.1cm
%\noindent
%The \textbf{orthogonal group} is defined as follows:
%\begin{equation*}
%\textnormal{O}(n)
%\; := \;
%	\left\{\;\,
%		g \overset{{\color{white}.}}{\in} \textnormal{GL}(n,\Re)
%		\;\left\vert\;\,
%			g^{T} \cdot g = I_{n}
%			\right.
%		\;\right\}
%\end{equation*}
%The \textbf{special orthogonal group} is defined as follows:
%\begin{equation*}
%\textnormal{SO}(n)
%\; := \;
%	\left\{\;\,
%		g \overset{{\color{white}.}}{\in} \textnormal{GL}(n,\Re)
%		\;\left\vert\;\,
%			g^{T} \cdot g = I_{n}\,,
%			\;
%			\textnormal{det}(g) = 1
%			\right.
%		\;\right\}
%\end{equation*}
%\end{definition}
%
%          %%%%% ~~~~~~~~~~~~~~~~~~~~ %%%%%
%
%\begin{proposition}[Lie algebras of $\textnormal{O}(n)$ and $\textnormal{SO}(n)$]
%\begin{eqnarray*}
%\mathfrak{o}(n)
%& = &
%	\left\{\;\,
%		X \overset{{\color{white}.}}{\in} \mathfrak{gl}(n,\Re)
%		\;\left\vert\;\,
%			X^{T} = -X
%			\right.
%		\;\right\}
%\\
%\mathfrak{so}(n)
%& = &
%	\left\{\;\,
%		X \overset{{\color{white}.}}{\in} \mathfrak{gl}(n,\Re)
%		\;\left\vert\;\,
%			X^{T} = -X\,,
%			\;
%			\textnormal{trace}(X) = 0
%			\right.
%		\;\right\}
%\end{eqnarray*}
%\end{proposition}
%
%          %%%%% ~~~~~~~~~~~~~~~~~~~~ %%%%%
%
%\subsection{Generators of \;$\textnormal{SO}(3)$\, and \,$\mathfrak{so}(3)$}
%
%          %%%%% ~~~~~~~~~~~~~~~~~~~~ %%%%%
%
%\vskip 0.1cm
%\noindent
%\textbf{Euler matrices}
%\begin{equation*}
%R_{1}(\phi)
%\; := \;
%	\left(\,
%		\begin{array}{ccc}
%			{\color{white}-}1 & {\color{white}-}0 & {\color{white}-}0 \\
%			{\color{white}-}0 & {\color{white}-}\cos\phi & -\sin\phi \\
%			{\color{white}-}0 & {\color{white}-}\sin\phi & {\color{white}-}\cos\phi \\
%			\end{array}
%		\,\right)
%\end{equation*}
%\begin{equation*}
%R_{2}(\psi)
%\; := \;
%	\left(\,
%		\begin{array}{ccc}
%			{\color{white}-}\cos\psi & {\color{white}-}0 & {\color{white}-}\sin\psi \\
%			{\color{white}-}0 & {\color{white}-}1 & {\color{white}-}0 \\
%			-\sin\psi & {\color{white}-}0 & {\color{white}-}\cos\psi \\
%			\end{array}
%		\,\right)
%\end{equation*}
%\begin{equation*}
%R_{3}(\theta)
%\; := \;
%	\left(\,
%		\begin{array}{ccc}
%			{\color{white}-}\cos\theta & -\sin\theta & {\color{white}-}0 \\
%			{\color{white}-}\sin\theta & {\color{white}-}\cos\theta & {\color{white}-}0 \\
%			{\color{white}-}0 & {\color{white}-}0 & {\color{white}-}1 \\
%			\end{array}
%		\,\right)
%\end{equation*}
%
%          %%%%% ~~~~~~~~~~~~~~~~~~~~ %%%%%
%
%\vskip 0.5cm
%\noindent
%\textbf{The generators \,$J_{n} \in \C^{3 \times 3}$\, of the Euler matrices}
%\begin{equation*}
%R_{n}(\theta)
%\; = \;
%	\exp\!\left(\;\sqrt{-1}\cdot\theta \overset{{\color{white}1}}{\cdot} J_{n}\,\right)
%\; = \;
%	\exp\!\left(\;\i\cdot\theta \overset{{\color{white}1}}{\cdot} J_{n}\,\right)
%\end{equation*}
%Alternatively, note:
%\begin{equation*}
%\i \cdot J_{1}
%\;\; = \;\;
%	\left.\dfrac{\d}{\d\,\phi}\right\vert_{\phi = 0} R_{1}(\phi)
%\;\; = \;
%	\left.\left(\,
%		\begin{array}{ccc}
%			{\color{white}-}1 & {\color{white}-}0 & {\color{white}-}0 \\
%			{\color{white}-}0 & {\color{white}-}\sin\phi & {\color{white}-}\cos\phi \\
%			{\color{white}-}0 & {\color{black}-}\cos\phi & {\color{white}-}\sin\phi \\
%			\end{array}
%		\,\right)\right\vert_{\phi = 0}
%\;\; = \;
%	\left(\,
%		\begin{array}{ccc}
%			{\color{white}-}0 & {\color{white}-}0 & {\color{white}-}0 \\
%			{\color{white}-}0 & {\color{white}-}0 & {\color{white}-}1 \\
%			{\color{white}-}0 & {\color{black}-}1 & {\color{white}-}0 \\
%			\end{array}
%		\,\right)
%\end{equation*}
%Multiplying both sides by \,$-\,\i = -\,\sqrt{-1}$\, gives:
%\begin{equation*}
%J_{1}
%\;\; = \;
%	\left(\!\!
%		\begin{array}{ccc}
%			{\color{white}-}0 & {\color{white}-}0 & {\color{white}-}0 \\
%			{\color{white}-}0 & {\color{white}-}0 & {\color{black}-}\i \\
%			{\color{white}-}0 & {\color{white}-}\i & {\color{white}-}0 \\
%			\end{array}
%		\,\right)
%\end{equation*}
%Similarly,
%\begin{equation*}
%\i \cdot J_{2}
%\;\; = \;\;
%	\left.\dfrac{\d}{\d\,\psi}\right\vert_{\psi = 0} R_{2}(\psi)
%\;\; = \;
%	\left.\left(\!\!
%		\begin{array}{ccc}
%			{\color{white}-}\sin\psi & {\color{white}-}0 & {\color{black}-}\cos\psi \\
%			{\color{white}-}0 & {\color{white}-}0 & {\color{white}-}0 \\
%			{\color{white}-}\cos\psi & {\color{white}-}0 & {\color{white}-}\sin\psi \\
%			\end{array}
%		\,\right)\right\vert_{\psi = 0}
%\;\; = \;
%	\left(\!\!
%		\begin{array}{ccc}
%			{\color{white}-}0 & {\color{white}-}0 & {\color{black}-}1 \\
%			{\color{white}-}0 & {\color{white}-}0 & {\color{white}-}0 \\
%			{\color{white}-}1 & {\color{white}-}0 & {\color{white}-}0 \\
%			\end{array}
%		\,\right)
%\end{equation*}
%Multiplying both sides by \,$-\,\i = -\,\sqrt{-1}$\, gives:
%\begin{equation*}
%J_{2}
%\;\; = \;
%	\left(\!\!
%		\begin{array}{ccc}
%			{\color{white}-}0 & {\color{white}-}0 & {\color{white}-}\i \\
%			{\color{white}-}0 & {\color{white}-}0 & {\color{white}-}0 \\
%			{\color{black}-}\i & {\color{white}-}0 & {\color{white}-}0 \\
%			\end{array}
%		\,\right)
%\end{equation*}
%Lastly,
%\begin{equation*}
%\i \cdot J_{3}
%\;\; = \;\;
%	\left.\dfrac{\d}{\d\,\theta}\right\vert_{\theta = 0} R_{3}(\theta)
%\;\; = \;
%	\left.\left(\!
%		\begin{array}{ccc}
%			{\color{white}-}\sin\theta & {\color{white}-}\cos\theta & {\color{white}-}0 \\
%			{\color{black}-}\cos\theta & {\color{white}-}\sin\theta & {\color{white}-}0 \\
%			{\color{white}-}0 & {\color{white}-}0 & {\color{white}-}0 \\
%			\end{array}
%		\,\right)\right\vert_{\psi = 0}
%\;\; = \;
%	\left(
%		\begin{array}{ccc}
%			{\color{white}-}0 & {\color{white}-}1 & {\color{white}-}0 \\
%			{\color{black}-}1 & {\color{white}-}0 & {\color{white}-}0 \\
%			{\color{white}-}0 & {\color{white}-}0 & {\color{white}-}0 \\
%			\end{array}
%		\,\right)
%\end{equation*}
%Multiplying both sides by \,$-\,\i = -\,\sqrt{-1}$\, gives:
%\begin{equation*}
%J_{3}
%\;\; = \;
%	\left(\!
%		\begin{array}{ccc}
%			{\color{white}-}0 & {\color{black}-}\i & {\color{white}-}0 \\
%			{\color{white}-}\i & {\color{white}-}0 & {\color{white}-}0 \\
%			{\color{white}-}0 & {\color{white}-}0 & {\color{white}-}0 \\
%			\end{array}
%		\,\right)
%\end{equation*}
%
%          %%%%% ~~~~~~~~~~~~~~~~~~~~ %%%%%
%
%\subsection{Properties of the generators \,$J_{1}, J_{2}, J_{3} \,\in\, \mathfrak{so}(3) \otimes_{\Re} \C$}
%
%          %%%%% ~~~~~~~~~~~~~~~~~~~~ %%%%%
%
%\begin{proposition}
%{\color{white}.}\vskip -0.5cm{\color{white}.}
%\begin{enumerate}
%\item
%	\textbf{Commutation relations:}\;\;
%	\begin{equation*}
%	\left[\,J_{k}\,\overset{{\color{white}1}}{,}\,J_{l}\,\right]
%	\;\; = \;\;
%		\sqrt{-1}\;\overset{3}{\underset{m=1}{\sum}}\,\varepsilon_{klm}\cdot J_{m}\,,
%	\quad
%	\textnormal{for each \,$k, l \in \{\,1,2,3\,\}$}\,,
%	\end{equation*}
%	where \,$\varepsilon_{klm}$\, is the fully anti-symmetric tensor.
%\item
%	\textbf{Raising and lowering operators:}\;\;
%	Define \,$J_{+}\,,\, J_{-} \in \mathfrak{so}(3) \otimes_{\Re} \C \subset \mathcal{U}\!\left(\mathfrak{so}(3) \overset{{\color{white}.}}{\otimes_{\Re}} \C\right)$\, as follows:
%	\begin{equation*}
%	J_{\pm} \;\; := \;\; J_{1} \, \pm \sqrt{-1}\,J_{2}.
%	\end{equation*}
%	Then, the following equalities (of elements of $\mathcal{U}\!\left(\mathfrak{so}(3) \overset{{\color{white}.}}{\otimes_{\Re}} \C\right)$) hold:
%	\begin{enumerate}
%	\item
%		$\left[\,J_{3}\,,\,J_{+}\,\right] \;=\; J_{+}$\,,
%		\quad
%		$\left[\,J_{3}\,,\,J_{-}\,\right] \;=\; -\,J_{-}$\,,
%		\quad
%		$\left[\,J_{+}\,,\,J_{-}\,\right] \;=\; 2\,J_{3}$
%	\item
%		$J^{2}$
%		\; $=$ \; $(J_{3})^{2} \,-\, J_{3} \,+\, J_{+}J_{-}$
%		\; $=$ \; $(J_{3})^{2} \,+\, J_{3} \,+\, J_{-}J_{+}$
%	\item
%		$(J_{\pm})^{\dagger} \; = \; J_{\mp}$
%	\end{enumerate}
%\item
%	Suppose
%	\,$\rho : \mathfrak{so}(3) \otimes_{\Re} \C \longrightarrow \mathfrak{gl}(V)$\,
%	is an irreducible finite-dimensional complex representation, and
%	\,$v \in V \backslash\{0\}$\, is an eigenvector of \,$\rho(J_{3})$\,
%	corresponding to the eigenvalue \,$\lambda \in \Re$;\, thus, \,$\rho(J_{3})(v) \,=\, \lambda\,v$.
%	Then, we have:
%	\begin{equation*}
%	\rho(J_{3})\!\left(\,\rho(J_{+})(\overset{{\color{white}-}}{v})\,\right) \, = \; (\lambda+1)\cdot\rho(J_{+})(v)\,
%	\quad\textnormal{and}\quad\;
%	\rho(J_{3})\!\left(\,\rho(J_{-})(\overset{{\color{white}-}}{v})\,\right) \, = \; (\lambda-1)\cdot\rho(J_{-})(v)
%	\end{equation*}
%\item
%	\textbf{Casimir operator:}\;\;
%	Define
%	\,$J^{2}$
%	\,$:=$\,
%	$(J_{1})^{2} + (J_{2})^{2} + (J_{3})^{2}$
%	\,$\in$\
%	 $\mathcal{U}\!\left(\mathfrak{so}(3) \overset{{\color{white}.}}{\otimes_{\Re}} \C\right)$.
%	Then,
%	\begin{equation*}
%	\left[\,J^{2}\,\overset{{\color{white}1}}{,}\,J_{k}\,\right]
%	\;\; = \;\;
%		0\,,
%	\quad
%	\textnormal{for each \,$k \in \{\,1,2,3\,\}$}\,.
%	\end{equation*}
%	Consequently (by Schur's Lemma, Corollary 4.30, \cite{Hall2015}), 
%	\,$J^{2} \in \mathcal{U}\!\left(\mathfrak{so}(3) \overset{{\color{white}.}}{\otimes_{\Re}} \C\right)$\,
%	acts as a scalar multiple of the identity in every irreducible
%	representation\footnote{Furthermore, this scalar $\lambda \in \C$ uniquely determines
%	the irreducible representation.
%	Look up the classification theory of irreducible finite-dimensional complex representations
%	of complex semisimple Lie algebras.
%	Key words: Casimir operator, universal enveloping algebra. See Chapters 9 and 10, \cite{Hall2015}.}
%	of \,$\mathcal{U}\!\left(\mathfrak{so}(3) \overset{{\color{white}.}}{\otimes_{\Re}} \C\right)$;\,
%	more precisely, for each irreducible finite-dimensional complex representation
%	\,$\rho : \mathcal{U}\!\left(\mathfrak{so}(3) \overset{{\color{white}.}}{\otimes_{\Re}} \C\right) \longrightarrow \mathfrak{gl}(V)$,\,
%	we have \,$\rho(J^{2}) = \lambda \cdot \textnormal{\textbf{1}}_{V}$,\,
%	for some \,$\lambda \in \C$.
%\end{enumerate}
%\end{proposition}
%
%          %%%%% ~~~~~~~~~~~~~~~~~~~~ %%%%%
%
%\begin{theorem}
%{\color{white}.}\vskip -0.1cm
%\noindent
%\begin{enumerate}
%\item
%	The finite-dimensional irreducible representations of $\mathfrak{so}(3) \otimes_{\Re} \C$ is parametrized by the set
%	\begin{equation*}
%	\dfrac{1}{2} \cdot \Z
%	\;\; := \;\;
%		\left\{\;0 \,,\, \dfrac{1}{2} \,,\, 1 \,,\, \frac{3}{2} \,,\, 2 \,,\, \frac{5}{2} \,,\, \ldots \;\right\},
%	\end{equation*}
%	of non-negative integer multiples of \,$\dfrac{1}{2}$, in that, for each
%	$s \in \dfrac{1}{2} \cdot \Z = \left\{\; 0 \,,\, \frac{1}{2}\,,\, 1\,,\, \frac{3}{2}\,,\, 2\,,\, \frac{5}{2}\,,\, \ldots \;\right\}$,
%	there exists a unique (up to equivalence) complex representation
%	$\rho_{s} : \mathcal{U}(\mathfrak{so}(3)\otimes_{\Re}\C) \longrightarrow \textnormal{End}(V_{s})$
%	such that
%	\begin{equation*}
%	\rho_{s}(J^{2}) \; = \; s(s+1)\cdot\textnormal{\textbf{1}}_{V_{s}}.
%	\end{equation*}
%\item
%	$\dim_{\C}(V_{s}) \, = \, 2s + 1$,\, for each
%	\,$s \in \dfrac{1}{2} \cdot \Z = \left\{\; 0 \,,\, \frac{1}{2}\,,\, 1\,,\, \frac{3}{2}\,,\, 2\,,\, \frac{5}{2}\,,\, \ldots \;\right\}$.
%\item
%	For each
%	\,$s \in \dfrac{1}{2} \cdot \Z = \left\{\; 0 \,,\, \frac{1}{2}\,,\, 1\,,\, \frac{3}{2}\,,\, 2\,,\, \frac{5}{2}\,,\, \ldots \;\right\}$,\,
%	the spectrum
%	$\sigma\!\left(\,\overset{{\color{white}-}}{\rho}_{s}(J_{3})\,\right)$
%	of the operator $\rho_{s}(J_{3}) \in \textnormal{End}(V_{s})$
%	consists of only eigenvalues and is given by:
%	\begin{equation*}
%	\sigma\!\left(\,\overset{{\color{white}-}}{\rho}_{s}(J_{3})\,\right)
%	\;\; = \;\;
%		\left\{\;
%			-\overset{{\color{white}-}}{s} \,,\, -(s-1), -(s-2)
%			\,,\;\, \ldots \,\;,\,
%			(s-2) \,,\, (s-1) \,,\, s
%			\;\right\},
%	\end{equation*}
%	and each eigenvalue in 
%	$\sigma\!\left(\,\overset{{\color{white}-}}{\rho}_{s}(J_{3})\,\right)$
%	has multiplicity one.
%\item
%	For each
%	\,$s \in \dfrac{1}{2} \cdot \Z = \left\{\; 0 \,,\, \frac{1}{2}\,,\, 1\,,\, \frac{3}{2}\,,\, 2\,,\, \frac{5}{2}\,,\, \ldots \;\right\}$,\,
%	let \,$v^{(s)}_{k} \in V_{s}\backslash\{0\}$\, be any normalized eigenvector
%	of $\rho_{s}(J_{3})$ corresponding to the eigenvalue
%	\,$k$ $\in$ $\sigma\!\left(\,\overset{{\color{white}-}}{\rho}_{s}(J_{3})\,\right)$
%	$=$ $\left\{\;-\overset{{\color{white}-}}{s} \,,\, -(s-1) \,,\, \;\ldots\;,\, (s-1) \,,\, s\;\right\}$.\,
%	Then, 
%	\begin{enumerate}
%	\item
%		the eigenvectors
%		\,$v^{(s)}_{-s} \,,\, v^{(s)}_{-(s-1)} \,,\; \ldots \;,\, v^{(s)}_{s-1} \,,\, v^{(s)}_{s}$\,
%		form an orthonormal basis for $V_{s}$, and
%	\item
%		for each \,$k$ $\in$ $\sigma\!\left(\,\overset{{\color{white}-}}{\rho}_{s}(J_{3})\,\right)$
%		$=$ $\left\{\;-\overset{{\color{white}-}}{s} \,,\, -(s-1) \,,\, \;\ldots\;,\, (s-1) \,,\, s\;\right\}$,\,
%		we have:
%		\begin{equation*}
%		J_{\pm}\!\left(\,v^{(s)}_{k}\,\right)
%		\; = \;
%			\sqrt{{\color{white}.}
%			s(s+1) - k(k \pm 1)
%			{\color{white}.}}
%			\,\cdot\,
%			v^{(s)}_{k \pm 1}
%		\end{equation*}
%		In particular, \,$J_{\pm}\!\left(\,v^{(s)}_{\pm s}\,\right) \; = \; 0$.
%	\end{enumerate}
%\end{enumerate}
%\end{theorem}
%
%          %%%%% ~~~~~~~~~~~~~~~~~~~~ %%%%%
%

%\vskip 0.5cm
%
          %%%%% ~~~~~~~~~~~~~~~~~~~~ %%%%%

\chapter{Irreducible representations of semisimple complex Lie algebras via Casimir operators in universal enveloping algebras}
\setcounter{theorem}{0}
\setcounter{equation}{0}

%\cite{vanDerVaart1996}
%\cite{Kosorok2008}

%\renewcommand{\theenumi}{\alph{enumi}}
%\renewcommand{\labelenumi}{\textnormal{(\theenumi)}$\;\;$}
\renewcommand{\theenumi}{\roman{enumi}}
\renewcommand{\labelenumi}{\textnormal{(\theenumi)}$\;\;$}

          %%%%% ~~~~~~~~~~~~~~~~~~~~ %%%%%

\begin{enumerate}
\item
	An \textit{algebra} $\mathcal{A}$ over a field $\F$ is a vector space over $\F$
	equipped with a bilinear map $* : \mathcal{A} \times \mathcal{A} \longrightarrow \mathcal{A}$, i.e.,
	\begin{equation*}
	\begin{array}{c}
		(a x + y) * z = a (x * z) + y * z\,,
		\\
		\underset{{\color{white}-}}{\overset{{\color{white}-}}{\textnormal{and}}}
		\\
		x * (a y + z) = a(x * y) + x * z\,,
	\end{array}
	\quad
	\textnormal{for each \,$a, b \in \F$,\, and \,$x, y, z \in \mathcal{A}$}.
	\end{equation*}
	The algebra $\mathcal{A}$ is said to be \textit{associative} if
	\begin{equation*}
	x * (y * z) \; = \; (x * y) * z\,,
	\quad
	\textnormal{for each \,$x, y, z \in \mathcal{A}$}
	\end{equation*}
	The algebra $\mathcal{A}$ is said to be \textit{unital} if
	there exists a multiplicative unit $1 \in \mathcal{A}$, i.e.,
	\begin{equation*}
	1 * x \; = \; x \; = \; x * 1\,,
	\quad
	\textnormal{for each \,$x \in \mathcal{A}$}
	\end{equation*}
\item
	Every associative algebra $\mathcal{A} \cong (\,V,*\,)$
	canonically induces a Lie algebra structure on its underlying vector space $V$ via:
	\begin{equation*}
	\left[\,x\,,\,y\,\right] \; := \; x * y - y * x\,,
	\quad
	\textnormal{for \,$x, y \in V$}
	\end{equation*}
	We denote this canonically induced Lie algebra as $\mathcal{A}^{[,]}$.
	\vskip 0.2cm
	This fact begs the question:
	Does the reverse assoication exist? More precisely, given a Lie algebra $\mathfrak{g}$,
	does there exist an associative algebra $\mathcal{A}$ such that $\mathcal{A}^{[,]} \cong \mathfrak{g}$?
	And, if so, is such an $\mathcal{A}$ unique?
	\vskip 0.2cm
	The answers are Yes and Yes.
	This uniquely determined associative algebra is called the
	\textbf{universal enveloping algebra} of $\mathfrak{g}$,
	and is denoted by $\mathcal{U}(\mathfrak{g})$.
	\vskip 0.1cm
	The universal enveloping algebra can be characterized by the following universal property:
	\begin{center}
	\begin{minipage}{5.25in}
	A \textit{universal enveloping algebra}
	of a Lie algebra $\mathfrak{g}$ over a field $\F$
	is a unital associative algebra $\mathcal{U}$ over $\F$ together with a Lie algebra homomorphism
	$\iota : \mathfrak{g} \longrightarrow \mathcal{U}^{[,]}$ such that,
	for each Lie algebra homomorphism $\phi : \mathfrak{g} \longrightarrow \mathcal{A}^{[,]}$
	(where $\mathcal{A}$ is a unital associative algebra),
	there exists a unique associative algebra homomorphism
	$\phi^{\sharp} : \mathcal{U} \longrightarrow \mathcal{A}$
	whose induced Lie algebra homomorphism
	$(\phi^{\sharp})^{\flat} : \mathcal{U}^{[,]} \longrightarrow \mathcal{A}^{[,]}$
	satisfies:
	$\phi \,=\, (\phi^{\sharp})^{\flat} \,\circ\, \iota$.
	\end{minipage}
	\end{center}
	The universal enveloping algebra $\mathcal{U}(\mathfrak{g})$ can be explicitly constructed as follows:
	\begin{equation*}
	\mathcal{U}(\mathfrak{g}) \;\; \cong \; \left. \overset{{\color{white}.}}{\otimes(\mathfrak{g})} \right\slash \mathcal{I}\,,
	\end{equation*}
	where $\otimes(\mathfrak{g})$ is the tensor algebra of $\mathfrak{g}$, and
	$\mathcal{I} \subset \otimes(\mathfrak{g})$ is the two-sided ideal of $\otimes(\mathfrak{g})$
	generated by elements of the form:
	\begin{equation*}
	x \otimes y \,-\, y \otimes x \,-\, \left[\,x,y\,\right]\,,
	\quad
	\textnormal{for \,$x, y \in \mathfrak{g}$}.
	\end{equation*}
\item
	The representations of a Lie algebra $\mathfrak{g}$ over $\F$ and
	those of its universal enveloping algebra $\mathcal{U}(\mathfrak{g})$
	are in one-to-one correspondence.
	\vskip 0.1cm
	\proof Let $\rho : \mathfrak{g} \longrightarrow \mathfrak{gl}(V)$ be a representation,
	i.e., $\rho$ is a Lie algebra homomorphism from $\mathfrak{g}$ into $\mathfrak{gl}(V) := \textnormal{End}(V)^{[,]}$,
	for some vector space $V$ over $\F$.
	By the universal property of
	$\iota : \mathfrak{g} \longrightarrow \mathcal{U}(\mathfrak{g})^{[,]}$,
	there exists a unique associative algebra homomorphism
	$\rho^{\sharp} : \mathcal{U}(\mathfrak{g}) \longrightarrow \textnormal{End}(V)$
	such that its induced Lie algebra homomorphism
	$(\rho^{\sharp})^{\flat} : \mathcal{U}(\mathfrak{g})^{[,]} \longrightarrow \textnormal{End}(V)^{[,]} =: \mathfrak{gl}(V)$
	satisfies:
	$\rho \,=\, (\rho^{\sharp})^{\flat} \,\circ\, \iota$.
	The association $\rho \longmapsto \rho^{\sharp}$ defines a map
	$\Theta$ from the collection of the representations of $\mathfrak{g}$
	to that of the representations of $\mathcal{U}(\mathfrak{g})$.
	Injectivity of $\Theta$ follows from:
	$\Theta(\rho_{1}) = \Theta(\rho_{2})$
	$\Longleftrightarrow$
	$\rho_{1}^{\sharp} = \rho_{2}^{\sharp}$
	$\Longrightarrow$
	$\rho_{1} \,=\, (\rho_{1}^{\sharp})^{\flat} \circ \iota \,=\, (\rho_{2}^{\sharp})^{\flat} \circ \iota \,=\, \rho_{2}$.
	It remains to establish the surjectivity of $\Theta$.
	To this end, let $\Psi : \mathcal{U}(\mathfrak{g}) \longrightarrow \textnormal{End}(V)$ be an arbitrary representation.
	Define $\psi \, := \, \Psi^{\flat} \,\circ\, \iota : \mathfrak{g} \longrightarrow \textnormal{End}(V)^{[,]} =: \mathfrak{gl}(V)$.
	Then, $\psi$ is a representation of $\mathfrak{g}$.
	By the universal property of
	$\iota : \mathfrak{g} \longrightarrow \mathcal{U}(\mathfrak{g})^{[,]}$,
	there exists a unique associative algebra homomorphism
	$\psi^{\sharp} : \mathcal{U}(\mathfrak{g}) \longrightarrow \textnormal{End}(V)$
	such that its induced Lie algebra homomorphism
	$(\psi^{\sharp})^{\flat} : \mathcal{U}(\mathfrak{g})^{[,]} \longrightarrow \textnormal{End}(V)^{[,]} =: \mathfrak{gl}(V)$
	satisfies:
	$\psi \,=\, (\psi^{\sharp})^{\flat} \,\circ\, \iota$.
	The uniqueness of the associative algebra homomorphism $\psi^{\sharp}$ therefore implies that
	$\Psi = \psi^{\sharp} =: \Theta(\psi)$.
	This proves the surjectivity of $\Theta$.
	\qed
\item
	The \textbf{Casimir operators} of a finite-dimensional semisimple complex Lie algebra $\mathfrak{g}$
	form a distinguished basis for the centre $\mathcal{Z}(\mathcal{U}(\mathfrak{g}))$ of the universal enveloping algebra
	$\mathcal{U}(\mathfrak{g})$.
	\vskip 0.1cm
	By Schur's Lemma, each Casimir operator (being an element of $\mathcal{Z}(\mathcal{U}(\mathfrak{g})))$ acts
	as a scalar multiple of the identity map on the representation space of any finite-dimensional irreducible representation.
	\vskip 0.1cm
	The collection of the aforementioned scalar multiples corresponding to the Casimir operators uniquely determines
	the irreducible representation, in the following sense:
	\begin{theorem}
	{\color{white}.}\vskip -0.1cm
	\noindent Two finite-dimensional irreducible representations
	$\rho_{1} : \mathfrak{g} \longrightarrow \mathfrak{gl}(V_{1})$
	and
	$\rho_{2} : \mathfrak{g} \longrightarrow \mathfrak{gl}(V_{2})$	
	of a finite-dimensional semisimple complex Lie algebra $\mathfrak{g}$
	are equivalent if and only if the eigenvalues of $\rho_{1}(C)$ and $\rho_{2}(C)$ are equal,
	for each Casimir operator $C \in \mathcal{Z}(\mathcal{U}(\mathfrak{g}))$.
	\end{theorem}
	\proof ??? \qed
\end{enumerate}

          %%%%% ~~~~~~~~~~~~~~~~~~~~ %%%%%


%\vskip 0.5cm
%
          %%%%% ~~~~~~~~~~~~~~~~~~~~ %%%%%

\chapter{Induced representations of semidirect products \,$G \ltimes\! H$\, with \,$H$\, abelian}
\setcounter{theorem}{0}
\setcounter{equation}{0}

%\cite{vanDerVaart1996}
%\cite{Kosorok2008}

%\renewcommand{\theenumi}{\alph{enumi}}
%\renewcommand{\labelenumi}{\textnormal{(\theenumi)}$\;\;$}
\renewcommand{\theenumi}{\roman{enumi}}
\renewcommand{\labelenumi}{\textnormal{(\theenumi)}$\;\;$}

          %%%%% ~~~~~~~~~~~~~~~~~~~~ %%%%%

\section{Semidirect products}

\begin{definition}
\mbox{}
\vskip 0.1cm
\noindent
Suppose:
\begin{itemize}
\item
	$G$\, and \,$H$\, are two groups, and
\item
	$\rho : G \longrightarrow \textnormal{Aut}(H)$\,
	is a left action of \,$G$\, on \,$H$.\,
\end{itemize}
Then, the \textbf{semidirect product} \,$G \ltimes_{\rho}\! H$\,
of \,$G$\, and \,$H$\, \textbf{\color{red}with respect to \,$\rho$}\,
is defined to be the group
obtained by defining on the Cartesian product
\,$G \times H$\,
the multiplication law:
\begin{equation*}
(\,g_{1},h_{1}\,) \cdot (\,g_{2},h_{2}\,)
\;\; = \;\;
	\left(\,
		\overset{{\color{white}1}}{g_{1}} \cdot g_{2}
		\,\overset{{\color{white}1}}{,}\,
		h_{1} \cdot \rho(g_{1}) \cdot h_{2}
		\,\right)
\end{equation*}
The identity element of \,$G \ltimes_{\rho}\! H$\, is then
\begin{equation*}
1_{G \ltimes H}
\;\; = \;\;
	\left(\,
		\overset{{\color{white}1}}{1_{G}}
		\,\overset{{\color{white}1}}{,}\,
		1_{H}
		\,\right)
\end{equation*}
and the inverse of \,$(\,g,h\,) \in G \ltimes_{\rho}\! H$\, is given by:
\begin{equation*}
(\, g \,,\, h \,)^{-1}
\;\; = \;\;
	\left(\;
		g^{-1}
		\;\overset{{\color{white}1}}{,}\;
		\overset{{\color{white}1}}{\rho}(g^{-1}) \cdot h^{-1}
		\;\right)
\end{equation*}
\end{definition}

          %%%%% ~~~~~~~~~~~~~~~~~~~~ %%%%%

\vskip 0.5cm
\section{A representation of a semidirect product is determined by the restrictions to its factors}

\begin{proposition}[Proposition 7.3, p.150, \cite{Berndt2007}]
\mbox{}
\vskip 0.1cm
\noindent
Suppose:
\begin{itemize}
\item
	$G$\, is a group, \,$H$\, an {\color{red}abelian} group,
\item
	$\rho : G \longrightarrow \textnormal{Aut}(H)$\,
	is a left action of \,$G$\, on \,$H$,\,
\item
	$\pi : G \ltimes_{\rho}\! H \longrightarrow \GL(V)$\,
	is an arbitrary representation of the semidirect product
	\,$G \ltimes_{\rho}\! H$\, with respect to \,$\rho$,\, and
\item
	$\pi_{G} : G \longrightarrow \GL(V)$\, and \,$\pi_{H} : G \longrightarrow \GL(V)$\,
	are the restrictions of \,$\pi$\, to \,$G$\, and \,$H$,\, respectively; i.e.,
	\,$\pi_{G}$\, and \,$\pi_{H}$\,
	are given by:
	\begin{equation*}
	\pi_{G}(\,g\,) \; := \; \pi\!\left(\,(\,\overset{{\color{white}1}}{g}\,,0\,)\,\right),
	\quad\quad
	\textnormal{and}
	\quad\quad
	\pi_{H}(\,h\,) \; := \; \pi\!\left(\,(\,1_{G}\,,\overset{{\color{white}1}}{h}\,)\,\right).
	\end{equation*}
\end{itemize}
Then, the following statements are true:
\begin{enumerate}
\item
	$\pi$\, is completely determined by its restrictions \,$\pi_{G}$\, and \,$\pi_{H}$,\, and
\item
	\,$\pi_{G}$\, and \,$\pi_{H}$\, satisfy the following equality:
	\begin{equation*}
	\pi_{H}\!\left(\,\rho(g) \cdot \overset{{\color{white}.}}{h}\,\right)
	\;\; = \;\;
		\pi_{G}\!\left(\; \overset{{\color{white}.}}{g} \,\right)
		\cdot
		\pi_{H}\!\left(\overset{{\color{white}.}}{{\color{white}g}}\!\! h \,\right)
		\cdot
		\pi_{G}\!\left(\; g^{-1} \,\right)
	\end{equation*}
\end{enumerate}
\end{proposition}
\proof
\begin{enumerate}
\item
	Note that
	\,$(\,g\,,h\,) \;=\; \left(\;1_{G} \cdot g\,,\, h + \rho(g)\cdot \overset{{\color{white}.}}{0}\,\right) \;=\; (\,1_{G}\,,h\,)\cdot (\,g\,,0\,)$.\,
	Hence,
	\begin{equation*}
	\pi\!\left(\;(\,\overset{{\color{white}1}}{g}\,,h\,)\;\right)
	\;\; = \;\;
		\pi\!\left(\;
			(\,1_{G}\,,h\,)
			\cdot
			(\,\overset{{\color{white}1}}{g}\,,0\,)
			\;\right)
	\;\; = \;\;
		\pi\!\left(\;
			(\,1_{G}\,,\overset{{\color{white}.}}{h}\,)
			\;\right)
		\cdot
		\pi\!\left(\;
			(\,\overset{{\color{white}.}}{g}\,,\overset{{\color{white}.}}{0}\,)
			\;\right)
	\;\; = \;\;
		\pi_{H}(\,h\,) \cdot \pi_{G}(\,g\,),
	\end{equation*}
	which proves that \,$\pi$\, is indeed completely determined by
	\,$\pi_{G}$\, and \,$\pi_{H}$.\,
\item
	Recall the multiplication law of \,$G \ltimes_{\rho}\! H$:\,
	\begin{equation*}
	(\,g_{1},h_{1}\,) \cdot (\,g_{2},h_{2}\,)
	\;\; = \;\;
		\left(\;
			\overset{{\color{white}1}}{g_{1}} \cdot g_{2}
			\;\overset{{\color{white}1}}{,}\;
			h_{1} + \rho(g_{1}) \cdot h_{2}
			\;\right)
	\end{equation*}
	Hence,
	\begin{eqnarray*}
	\pi_{H}(\,h_{1}\,) \cdot \pi_{G}(\,g_{1}\,)
	\cdot
	\pi_{H}(\,h_{2}\,) \cdot \pi_{G}(\,g_{2}\,)
	& = &
		\pi\!\left(\;
			(\,\overset{{\color{white}1}}{g_{1}},h_{1}\,)
			\;\right)
		\cdot
		\pi\!\left(\;
			(\,\overset{{\color{white}1}}{g_{2}},h_{2}\,)
			\;\right)
	\\
	& = &
		\pi\!\left(\;
			(\,g_{1},h_{1}\,) \overset{{\color{white}1}}{\cdot} (\,g_{2},h_{2}\,)
			\;\right)
	\\
	& = &
		\pi\!\left(\;
			\left(\;
				\overset{{\color{white}1}}{g_{1}} \cdot g_{2}
				\;\overset{{\color{white}1}}{,}\;
				h_{1} + \rho(g_{1}) \cdot h_{2}
				\;\right)
			\;\right)
	\\
	& = &
		\pi_{H}\!\left(\;h_{1} + \rho(g_{1}) \overset{{\color{white}1}}{\cdot} h_{2}\;\right)
		\cdot
		\pi_{G}(\;g_{1}\cdot g_{2}\,)
	\\
	& = &
		\pi_{H}\!\left(\;\overset{{\color{white}.}}{h_{1}}\;\right)
		\cdot
		\pi_{H}\!\left(\;\rho(g_{1}) \overset{{\color{white}1}}{\cdot} h_{2}\;\right)
		\cdot
		\pi_{G}\!\left(\;\overset{{\color{white}.}}{g_{1}}\;\right)
		\cdot
		\pi_{G}\!\left(\;\overset{{\color{white}.}}{g_{2}}\;\right)
	\end{eqnarray*}
	Hence,
	\begin{equation*}
		\pi_{H}\!\left(\;\overset{{\color{white}.}}{h_{1}}\;\right)
		\cdot
		\pi_{H}\!\left(\;\rho(g_{1}) \overset{{\color{white}1}}{\cdot} h_{2}\;\right)
		\cdot
		\pi_{G}\!\left(\;\overset{{\color{white}.}}{g_{1}}\;\right)
		\cdot
		\pi_{G}\!\left(\;\overset{{\color{white}.}}{g_{2}}\;\right)
	\;\; = \;\;
		\pi_{H}(\,h_{1}\,) \cdot \pi_{G}(\,g_{1}\,)
		\cdot
		\pi_{H}(\,h_{2}\,) \cdot \pi_{G}(\,g_{2}\,)
	\end{equation*}
	Setting, in the above equality,
	\,$g_{1} = g$,\,  $h_{1} = 0$,\, $g_{2} = 1_{G}$\, and \,$h_{2} = h$\,
	yields
	\begin{equation*}
		\pi_{H}\!\left(\;\overset{{\color{white}.}}{0}\;\right)
		\cdot
		\pi_{H}\!\left(\;\rho(g) \overset{{\color{white}1}}{\cdot} h\;\right)
		\cdot
		\pi_{G}\!\left(\;\overset{{\color{white}.}}{g}\;\right)
		\cdot
		\pi_{G}\!\left(\;\overset{{\color{white}.}}{1_{G}}\;\right)
	\;\; = \;\;
		\pi_{H}(\,0\,) \cdot \pi_{G}(\,g\,)
		\cdot
		\pi_{H}(\,h\,) \cdot \pi_{G}(\,1_{G}\,)\,,
	\end{equation*}
	which simplifies to
	\begin{equation*}
		\pi_{H}\!\left(\;\rho(g) \overset{{\color{white}1}}{\cdot} h\;\right)
		\cdot
		\pi_{G}\!\left(\;\overset{{\color{white}.}}{g}\;\right)
	\;\; = \;\;
		\pi_{G}(\,g\,)
		\cdot
		\pi_{H}(\,h\,)
	\end{equation*}
	which can further be rewritten as
	\begin{equation*}
		\pi_{H}\!\left(\;\rho(g) \overset{{\color{white}-}}{\cdot} h\;\right)
	\;\; = \;\;
		\pi_{G}(\,g\,)
		\cdot
		\pi_{H}(\,h\,)
		\cdot
		\pi_{G}\!\left(\,\overset{{\color{white}.}}{{\color{white}g}}\!\!g^{-1}\;\right),
	\end{equation*}
	as required. \qed
\end{enumerate}

          %%%%% ~~~~~~~~~~~~~~~~~~~~ %%%%%

\vskip 0.5cm
\section{Irreducible unitary representations of regular semidirect products}

\vskip 0.5cm
\begin{definition}[Definition 7.1, p.150, \cite{Berndt2007}]
\mbox{}
\vskip 0.1cm
\noindent
A semidirect
\,$G \ltimes_{\rho}\! H$\, of \,$G$\, amd \,$H$\,
is said to be \textbf{regular} if

\end{definition}

\vskip 0.5cm
\begin{theorem}[Theorem 7.7, p.151, \cite{Berndt2007}]
\mbox{}
\vskip 0.1cm
\noindent
Suppose:
\begin{itemize}
\item
	$G$\, and \,$H$\, are separable and locally compact groups, and
	\,$H$\, is furthermore abelian.
\item
	$\rho : G \longrightarrow \textnormal{Aut}(H)$\,
	is a left action of \,$G$\, on \,$H$,\, and
\item
	the semidirect product
	\,$G \ltimes_{\rho}\! H$\, of \,$G$\, amd \,$H$\,
	with respect to \,$\rho$\,
	is a regular.
\end{itemize}
Then, every irreducible unitary representation of
\,$G \ltimes_{\rho}\! H$\,
is unitarily equivalent to an induced representation.
\end{theorem}

          %%%%% ~~~~~~~~~~~~~~~~~~~~ %%%%%


          %%%%% ~~~~~~~~~~~~~~~~~~~~ %%%%%

%\section{Unitary irreducible representations of the Poincaré group}

          %%%%% ~~~~~~~~~~~~~~~~~~~~ %%%%%

          %%%%% ~~~~~~~~~~~~~~~~~~~~ %%%%%
