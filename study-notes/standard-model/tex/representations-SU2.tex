
          %%%%% ~~~~~~~~~~~~~~~~~~~~ %%%%%

\section{Irreducible representations of \,$\mathfrak{su}(2)$}
\setcounter{theorem}{0}
\setcounter{equation}{0}

%\cite{vanDerVaart1996}
%\cite{Kosorok2008}

%\renewcommand{\theenumi}{\alph{enumi}}
%\renewcommand{\labelenumi}{\textnormal{(\theenumi)}$\;\;$}
\renewcommand{\theenumi}{\roman{enumi}}
\renewcommand{\labelenumi}{\textnormal{(\theenumi)}$\;\;$}

          %%%%% ~~~~~~~~~~~~~~~~~~~~ %%%%%

\begin{definition}[$\textnormal{U}(n)$ and $\mathfrak{u}(n)$]
\mbox{}
\vskip 0.1cm
\noindent
The \textbf{unitary group} is defined as follows:
\begin{equation*}
\textnormal{U}(n)
\; := \;
	\left\{\;\,
		g \overset{{\color{white}.}}{\in} \textnormal{GL}(n,\C)
		\;\left\vert\;\,
			g^{\dagger} \cdot g = I_{n}
			\right.
		\;\right\}
\end{equation*}
where \,$g^{\dagger}$\, is the conjugate transpose of
\,$g \in \textnormal{GL}(n,\C)$\,
and
\,$I_{n} \in \textnormal{GL}(n,\C)$\,
is the identity matrix.
\vskip 0.1cm
\noindent
The \textbf{special unitary group} is defined as follows:
\begin{equation*}
\textnormal{SU}(n)
\; := \;
	\left\{\;\,
		g \overset{{\color{white}.}}{\in} \textnormal{GL}(n,\C)
		\;\left\vert\;\,
			g^{\dagger} \cdot g = I_{n}\,,
			\;
			\textnormal{det}(g) = 1
			\right.
		\;\right\}
\end{equation*}
\end{definition}

\vskip 0.5cm
\begin{proposition}[Parametrization of \,$\textnormal{SU}(2)$]
\mbox{}
\vskip 0.1cm
\noindent
$\textnormal{SU}(2)$ admits the following parametrization:
\begin{equation*}
\textnormal{SU}{(2)}
\; := \;
	\left\{\,
		\left.
		\left(\begin{array}{rr}
		a & -\overline{b}
		\\
		\overset{{\color{white}-}}{b} & \overline{a}
		\end{array}\right)
		\overset{{\color{white}.}}{\in}
		\C^{2 \times 2}
		\;\;\right\vert\;\,
			\vert\, a \,\vert^{2} \,+\, \vert\, b \,\vert^{2} \,=\, 1
		\;\right\}
\end{equation*}
Hence, the (real) Lie group {\color{red}$\textnormal{SU}(2)$ is diffeomorphic to $S^{3}$}, the $3$-dimensional unit sphere
(in $4$-dimensional Euclidean space).
In particular, $\textnormal{SU}(2)$ is a {\color{red}simply connected} $3$-dimensional real manifold.
\end{proposition}
\proof
Suppose:
\begin{equation*}
g
\; = \;
	\left(\begin{array}{cc}
		a & c
		\\
		\overset{{\color{white}-}}{b} & d
		\end{array}\right)
\; \in \;
\textnormal{SU}(2)
\end{equation*}
First note that the component form of the condition \,$g^{\dagger}\cdot g = I_{2}$\, is:
\begin{equation*}
\left(\begin{array}{cc}
	1 & 0
	\\
	\overset{{\color{white}-}}{0} & 1
	\end{array}\right)
\; = \;
	g^{\dagger} \cdot g
\; = \;
	\left(\begin{array}{cc}
		\overline{a} & \overline{b}
		\\
		\overset{{\color{white}-}}{\overline{c}} & \overline{d}
		\end{array}\right)
	\cdot
	\left(\begin{array}{cc}
		a & c
		\\
		\overset{{\color{white}-}}{b} & d
		\end{array}\right)
\; = \;
	\left(\begin{array}{cc}
		a\overline{a} + b\overline{b} & \overline{a}c + \overline{b}d
		\\
		\overset{{\color{white}-}}{a\overline{c} + b\overline{d}} & c\overline{c} + d\overline{d}
		\end{array}\right)
\end{equation*}
Thus, we see that
\begin{equation*}
g
\; = \;
	\left(\begin{array}{cc}
		a & c
		\\
		\overset{{\color{white}-}}{b} & d
		\end{array}\right)
\;\in\;
	\textnormal{SU}(2)
\quad\Longleftrightarrow\quad
\left\{
	\begin{array}{ccc}
		g^{\dagger} \cdot g &=& I_{2}
		\\
		\det(g) &=& \overset{{\color{white}1}}{1}
		\end{array}
		\right.
\quad\Longleftrightarrow\quad
\left\{
	\begin{array}{ccc}
	\vert\,a\,\vert^{2} + \vert\,b\,\vert^{2} &=& 1
	\\
	\vert\,c\,\vert^{2} + \vert\,d\,\vert^{2} &\overset{{\color{white}1}}{=}& 1
	\\
	a\overline{c} \;\, + \,\; b\overline{d} &\overset{{\color{white}1}}{=}& 0
	\\
	ad \;\, - \,\; bc &\overset{{\color{white}1}}{=}& 1
	\end{array}
	\right.
\end{equation*}
Next, note that
\begin{equation*}
a\overline{c} + b\overline{d} = 0
\quad\Longleftrightarrow\quad
	\left\langle
		\left(\begin{array}{c} a \\ b \end{array}\right)
		\,,\,
		\left(\begin{array}{c} c \\ d \end{array}\right)
		\right\rangle_{\C^{2}}
	\;=\;
	0
\end{equation*}
Since
\,$\dim_{\C}\left(\begin{array}{c} a \\ b \end{array}\right)^{\perp} =\, 1$,\,
the above equality (i.e., orthogonality of the two columns of \,$g$) implies:
\begin{equation*}
\left(\begin{array}{c} c \\ d \end{array}\right)
\; \in \;
	\left(\begin{array}{c} a \\ b \end{array}\right)^{\perp}
\; = \;
	\textnormal{span}_{\C}\left\{
		\left(\begin{array}{r} -\overline{b} \\ \overline{a} \end{array}\right)
		\right\}
\quad\Longleftrightarrow\quad
\left(\begin{array}{c} c \\ d \end{array}\right)
\; = \;
	\lambda \left(\begin{array}{r} -\overline{b} \\ \overline{a} \end{array}\right),
	\;\;
	\textnormal{for some $\lambda \in \C$}
\end{equation*}
So, we now know that $g$ has the form:
\begin{equation*}
g
\; = \;
	\left(\begin{array}{rr}
		a & -\lambda\,\overline{b}
		\\
		\overset{{\color{white}-}}{b} & \lambda\,\overline{a}
		\end{array}\right)
\end{equation*}
Next,
\begin{equation*}
1
\,=\, \det(g)
\,=\, a\cdot(\lambda\,\overline{a}) - b \cdot (-\lambda\,\overline{b})
\,=\, \lambda\cdot(\vert\,a\,\vert^{2} + \vert\,b\,\vert^{2})
\quad\Longrightarrow\quad
	\lambda = 1
\end{equation*}
We may now conclude that
\begin{equation*}
g
\; = \;
	\left(\begin{array}{rr}
		a & -\,\overline{b}
		\\
		\overset{{\color{white}-}}{b} & \overline{a}
		\end{array}\right),\,
\quad
\textnormal{where \,$\vert\,a\,\vert^{2} + \vert\,b\,\vert^{2} = 1$}
\end{equation*}
This completes the proof of the Proposition.
\qed

          %%%%% ~~~~~~~~~~~~~~~~~~~~ %%%%%

\vskip 0.5cm
\begin{proposition}[Characterizations of \,$\mathfrak{sl}(n)$, \,$\mathfrak{u}(n)$, and $\mathfrak{su}(n)$]
\mbox{}
\vskip 0.1cm
\begin{enumerate}
\item
	\begin{equation*}
	\mathfrak{sl}(n,\C)
	\; = \;
		\left\{\;
			X \,\in\, \mathfrak{gl}(n,\C) \,=\, \C^{n \times n}
			\;\left\vert\;\,
				\textnormal{trace}(X) = \overset{{\color{white}1}}{0}
				\right.
			\,\right\}
	\end{equation*}
\item
	\begin{equation*}
	\mathfrak{u}(n)
	\; = \;
		\left\{\;
			X \,\in\, \mathfrak{gl}(n,\C) \,=\, \C^{n \times n}
			\;\left\vert\;\,
				X + X^{\dagger} = \overset{{\color{white}1}}{0}
				\right.
			\,\right\}
	\end{equation*}
\item
	\begin{equation*}
	\mathfrak{su}(n)
	\; = \;
		\left\{\;
			X \,\in\, \mathfrak{gl}(n,\C) \,=\, \C^{n \times n}
			\;\left\vert\;\,
				\begin{array}{c}
				X + X^{\dagger} = \overset{{\color{white}1}}{0}
				\\
				\textnormal{trace}(X) = \overset{{\color{white}1}}{0}
				\end{array}
				\right.
			\,\right\}
	\end{equation*}
\end{enumerate}
\end{proposition}
\proof
\begin{enumerate}
\item
	We invoke the fact that \,$\det(e^{\,t\,\cdot\,X}) \,=\, e^{\,t\,\cdot\,\textnormal{trace}(X)}$,
	for each \,$X \in \C^{n \times n}$.
	Thus,
	\begin{eqnarray*}
	&&
		X \,\in\, \mathfrak{sl}(n,\C)
		\quad\Longrightarrow\quad
		e^{\,t\cdot\,X} \,\in\, \textnormal{SL}(n,\C)
		\quad\Longrightarrow\quad
		\det\!\left(\,e^{\,t\cdot\,X}\,\right) \,=\, 1
	\\
	& \Longrightarrow\quad &
		\textnormal{trace}(X)
		\; = \;
			\left.\dfrac{\d}{\d\,t}\right\vert_{t=0}\left(\,\overset{{\color{white}1}}{e^{\,t\,\cdot\,\textnormal{trace}(X)}}\,\right)
		\; = \;
			\left.\dfrac{\d}{\d\,t}\right\vert_{t=0}\left(\,\overset{{\color{white}1}}{\det(e^{\,t\,\cdot\,X})}\,\right)
		\; = \;
			\left.\dfrac{\d}{\d\,t}\right\vert_{t=0}\left(\,\overset{{\color{white}1}}{1}\,\right)
		\; = \;
			0
	\end{eqnarray*}
	Conversely, suppose \,$\textnormal{trace}(X) = 0$.\,
	Then, \,$\det(e^{\,t\,\cdot\,X}) \,=\, e^{\,t\,\cdot\,\textnormal{trace}(X)} \,=\, e^{\,t\,\cdot\,0} \,=\, 1$,\,
	which implies that \,$e^{\,t\,\cdot\,X} \,\in\, \textnormal{SL}(n,\C)$,\, hence \,$X \,\in\, \mathfrak{sl}(n,\C)$.
	This completes the proof of the equality (of sets) in question.
\item
	\begin{eqnarray*}
	&&
		X \,\in\, \mathfrak{u}(n)
		\quad\Longrightarrow\quad
		e^{\,t\,\cdot\,X} \,\in\, \textnormal{U}(n)
	\\
	& \Longrightarrow\quad &
		I_{n}
			\,=\, \left(\,e^{\,t\,\cdot\,X}\,\right)^{\!\dagger} \cdot \left(\,e^{\,t\,\cdot\,X}\,\right)
			\,=\, \left(\,e^{\,t\,\cdot\,X^{\dagger}}\,\right) \cdot \left(\,e^{\,t\,\cdot\,X}\,\right)
			\,=\, e^{\,t\,\cdot\,(X^{\dagger}+X)}
	\\
	& \Longrightarrow\quad &
		X \,+\, X^{\dagger}
		\; = \;
			\left.\dfrac{\d}{\d\,t}\right\vert_{t=0}\left(\,\overset{{\color{white}1}}{e^{\,t\,\cdot\,(X+X^{\dagger})}}\,\right)
		\; = \;
			\left.\dfrac{\d}{\d\,t}\right\vert_{t=0}\left(\,\overset{{\color{white}1}}{I_{n}}\,\right)
		\; = \;
			0
	\end{eqnarray*}
	Conversely, suppose \,$X + X^{\dagger} \,=\, 0$.\,
	Then, \,$I_{n}$
	\,$=$\, $e^{\,0_{n \times n}}$
	\,$=$\, $e^{\,t\,\cdot(X^{\dagger}+X)}$
	\,$=\, \cdots \,=$\, $\left(e^{\,t\,\cdot\,X}\right)^{\!\dagger}\cdot\left(e^{\,t\,\cdot\,X}\right)$,\,
	which implies that \,$e^{\,t\,\cdot\,X} \,\in\, \textnormal{U}(n)$,\, hence \,$X \,\in\, \mathfrak{u}(n)$.
	This completes the proof of the equality (of sets) in question.
\item
	Immediate by the preceding two statements.
	\qed
\end{enumerate}

\vskip 0.5cm
\begin{proposition}[Generators of \,$\mathfrak{su}(2)$]
\mbox{}
\vskip 0.1cm
\noindent
Let \,$\sigma_{1},\, \sigma_{2},\, \sigma_{3} \,\in\, \C^{2 \times 2}$\, be the \textbf{Pauli spin matrices}, i.e.,
\begin{equation*}
\sigma_{1} \,=\, \sigma_{x} \,:=\, \left(\begin{array}{cc} 0 & 1 \\ 1 & 0 \end{array}\right),
\quad
\sigma_{2} \,=\, \sigma_{y} \,:=\, \left(\begin{array}{rr} 0 & -\i \\ \i & 0 \end{array}\right),
\quad
\sigma_{3} \,=\, \sigma_{z} \,:=\, \left(\begin{array}{rr} 1 & 0 \\ 0 & -1 \end{array}\right).
\end{equation*}
Define \,$J_{1},\, J_{2},\, J_{3},\, S_{1},\, S_{2},\, S_{3},\, S_{+},\, S_{-} \,\in\, \C^{2 \times 2}$\, as follows:
\begin{equation*}
Y_{1} \,:=\, \dfrac{\i}{2}\cdot\sigma_{1} \,=\, \dfrac{\i}{2}\cdot\left(\begin{array}{cc} 0 & 1 \\ 1 & 0 \end{array}\right),
\quad
Y_{2} \,:=\, \mathbf{{\color{red}-}}\,\dfrac{\i}{2}\cdot\sigma_{2} \,=\, \dfrac{1}{2}\cdot\left(\begin{array}{rr} 0 & -1 \\ 1 & 0 \end{array}\right),
\quad
Y_{3} \,:=\, \dfrac{\i}{2}\cdot\sigma_{3} \,=\, \dfrac{\i}{2}\cdot\left(\begin{array}{rr} 1 & 0 \\ 0 & -1 \end{array}\right),
\end{equation*}
\begin{equation*}
S_{1} \,:=\, \i \cdot Y_{1} \,=\, \dfrac{-1}{2} \cdot \left(\begin{array}{rr} 0 & 1 \\ 1 & 0 \end{array}\right),
\quad
S_{2} \,:=\, \i \cdot Y_{2} \,=\, \dfrac{\i}{2} \cdot \left(\begin{array}{rr} 0 & -1 \\ 1 & 0 \end{array}\right),
\quad
S_{3} \,:=\, \i \cdot Y_{3} \,=\, \dfrac{1}{2}\cdot\left(\begin{array}{rr} -1 & 0 \\ 0 & 1 \end{array}\right).
\end{equation*}
\begin{equation*}
S_{+} \; := \; S_{1} + \i\,S_{2} \; = \; \left(\begin{array}{rr} 0 & 0 \\ -1 & 0 \end{array}\right),
\quad\quad
S_{-} \; := \; S_{1} - \i\,S_{2} \; = \; \left(\begin{array}{rr} 0 & -1 \\ 0 & 0 \end{array}\right).
\end{equation*}
Then, the following statements are true:
\begin{enumerate}
\item
	$Y_{1},\, Y_{2},\, Y_{3} \,\in\, \mathfrak{su}(2)$.\,
	$Y_{1},\, Y_{2},\, Y_{3}$\,
	form a set of generators for the (real) Lie algebra \,$\mathfrak{su}(2)$\, of the (real) Lie group \,$\textnormal{SU}(2)$.\,
	$Y_{1},\, Y_{2},\, Y_{3}$\, satisfy the following commutation relations:
	\begin{equation*}
	\left[\,Y_{a}\,,\,Y_{b}\,\right] \;\; = \;\; \overset{3}{\underset{c\,=\,1}{\sum}}\;\varepsilon_{abc}\,Y_{c}\,,
	\quad
	\textnormal{for \,$a, b = 1,2,3$}.
	\end{equation*}
\item
	$S_{1},\, S_{2},\, S_{3} \,\in\, \mathfrak{su}(2) \otimes_{\Re} \C$,\,
	where
	\,$\mathfrak{su}(2) \otimes_{\Re} \C$\,
	is the complexification of (the real Lie algebra)
	\,$\mathfrak{su}(2)$.\,
	\,$S_{1},\, S_{2},\, S_{3}$\,
	satisfy the following commutation relations:
	\begin{equation*}
	\left[\,S_{a}\,\overset{{\color{white}1}}{,}\,S_{b}\,\right]
	\;\; = \;\;
		\sqrt{-1}\,\cdot\overset{3}{\underset{c\,=\,1}{\sum}}\;\varepsilon_{abc}\cdot S_{c}\,,
	\quad
	\textnormal{for each \,$a, b \in \{\,1,2,3\,\}$}\,,
	\end{equation*}
	where \,$\varepsilon_{abc}$\, is the fully anti-symmetric tensor.
\item
	$S_{+},\, S_{-} \,\in\, \mathfrak{su}(2) \otimes_{\Re} \C$,\,
	and
	\,$S_{+},\; S_{-},\; S_{3}$\, satisfy the following commutation relations:
	\begin{equation*}
	\left[\,S_{3}\,,\,S_{\pm}\,\right] \, = \, \pm\,S_{\pm}\,,
	\quad
	\left[\,S_{+}\,,\,S_{-}\,\right] \, = \, 2\,S_{3}
	\end{equation*}
\item
	Suppose
	\begin{itemize}
	\item
		$V$\, is a complex vector space,
	\item
		$\rho : \mathfrak{su}(2) \otimes_{\Re} \C \longrightarrow \textnormal{End}(V)$\,
		is a Lie algebra representation, and
	\item	
		$v \in V$\, and \,$\lambda \in \C$\, together satisfy \,$\rho(S_{3})(v) = \lambda \cdot v$.
	\end{itemize}	
	Then, \,$\rho(S_{+})(v) \,\in\, V$\, satisfies:
	\begin{equation*}
	\rho(S_{3})\!\left(\,\rho(\overset{{\color{white}.}}{S}_{+})(v)\,\right)
	\; = \;
		(\lambda+1) \cdot \rho(S_{+})(v)\,
	\end{equation*}
	and
	\,$\rho(S_{-})(v) \,\in\, V$\, satisfies:
	\begin{equation*}
	\rho(S_{3})\!\left(\,\rho(\overset{{\color{white}.}}{S}_{-})(v)\,\right)
	\; = \;
		(\lambda-1) \cdot \rho(S_{-})(v)\,
	\end{equation*}
\end{enumerate}
\end{proposition}
\proof
The Proposition follows straightforwardly by direct computations.
\qed

          %%%%% ~~~~~~~~~~~~~~~~~~~~ %%%%%

\vskip 0.5cm
\begin{proposition}[The representations \,\textnormal{$\pi_{d} : \su(2) \otimes_{\Re} \C \longrightarrow \gl\!\left(\,\overset{{\color{white}.}}{\C}[X,Y]_{d}\,\right)$}]
\mbox{}
\vskip 0.1cm
\noindent
Suppose:
\begin{itemize}
\item
	$d \,\in\, \{\,0, 1, 2, 3, \ldots\,\}$,
\item
	$\C[X,Y]_{d}$\,
	is the complex vector space of homogeneous polynomials of degree $d$
	in the indeterminates $X$ and $Y$ with complex coefficients.
\item
	Define
	\,$\pi_{d} : \su(2) \otimes_{\Re} \C \longrightarrow \gl\!\left(\,\overset{{\color{white}.}}{\C}[X,Y]_{d}\,\right)$\,
	by complex-linearly extending:
	\begin{equation*}
	\pi_{d}\!\left(\,S_{+}\,\right)
	\; := \;
		-\,Y\cdot\dfrac{\partial}{\partial\,X},
	\quad\;\;\;
	\pi_{d}\!\left(\,S_{-}\,\right)
	\; := \;
		-\,X\cdot\dfrac{\partial}{\partial\,Y},
	\quad\;\;\;
	\pi_{d}\!\left(\,S_{3}\,\right)
	\; := \;
		-\,\dfrac{1}{2}\cdot\left(\,X\cdot\dfrac{\partial}{\partial\,X} - Y\cdot\dfrac{\partial}{\partial\,Y}\right)
	\end{equation*}
\end{itemize}
Then, 
\,$\pi_{d} : \su(2) \otimes_{\Re} \C \longrightarrow \gl\!\left(\,\overset{{\color{white}.}}{\C}[X,Y]_{d}\,\right)$\,
is an irreducible finite-dimensional complex representation of the complex Lie algebra
\,$\su(2) \otimes_{\Re} \C$.\,
\end{proposition}
\proof
We first establish that \,$\pi_{d}$\, is indeed a representation (i.e., it is a Lie algebra homomorphism).
To this end, it suffices to establish that \,$\pi_{d}$\, preserves commutation relations.
Now, observe that:
\begin{eqnarray*}
\left[\;
	\overset{{\color{white}1}}{\pi_{d}}\!\left(\,S_{+}\,\right)
	\, , \,
	\pi_{d}\!\left(\,S_{-}\,\right)
	\,\right]
	(X^{m}Y^{n})
& = &
	\left[\;
		-\,Y\cdot\dfrac{\partial}{\partial\,X}
		\;\; , \,
		-\,X\cdot\dfrac{\partial}{\partial\,Y}
		\,\right]
		(X^{m}Y^{n})
\\
& = &
	\left[\;
		Y\cdot\dfrac{\partial}{\partial\,X}
		\;\; , \,
		X\cdot\dfrac{\partial}{\partial\,Y}
		\,\right]
		(X^{m}Y^{n})
\\
& = &
	Y\cdot\dfrac{\partial}{\partial\,X}\!\left(\,
		\overset{{\color{white}.}}{X}\,X^{m} \cdot n \cdot Y^{n-1}
		\,\right)
	\, - \,
	X\cdot\dfrac{\partial}{\partial\,Y}\!\left(\,
		\overset{{\color{white}.}}{Y} \cdot m \cdot X^{m-1} \, Y^{n}
		\,\right)
\\
& = &
	n \cdot Y \cdot \dfrac{\partial}{\partial\,X}\!\left(\,
		\overset{{\color{white}.}}{X^{m+1}} - Y^{n-1}
		\,\right)
	\, - \,
	m \cdot X \cdot \dfrac{\partial}{\partial\,X}\!\left(\,
		\overset{{\color{white}.}}{X^{m-1}} - Y^{n+1}
		\,\right)
\\
& = &
	n(m+1) \cdot \overset{{\color{white}1}}{X^{m}}\,Y^{n} \, - \, m(n+1) \cdot X^{m}\,Y^{n}
\\
& = &
	(n-m) \cdot \overset{{\color{white}1}}{X^{m}\,Y^{n}}
\end{eqnarray*}
On the other hand,
\begin{eqnarray*}
\pi_{d}(S_{3})\!\left(\,\overset{{\color{white}.}}{X^{m}Y^{n}}\,\right)
& = &
	-\,\dfrac{1}{2}\cdot\left(\,X\cdot\dfrac{\partial}{\partial\,X} - Y\cdot\dfrac{\partial}{\partial\,Y}\right)
	\!\left(\,\overset{{\color{white}.}}{X^{m}Y^{n}}\,\right)
\\
& = &
	-\,\dfrac{1}{2}\cdot\left(\,
		\overset{{\color{white}.}}{X} \cdot m \cdot X^{m-1}\,Y^{n}
		\, - \,
		Y \cdot X^{m} \cdot n \cdot Y^{n-1}
		\,\right)
\\
& = &
	- \, \dfrac{1}{2} \cdot (m-n) \cdot \overset{{\color{white}1}}{X^{m}\,Y^{n}}
\;\; = \;\;
	\dfrac{1}{2} \cdot (n-m) \cdot \overset{{\color{white}1}}{X^{m}\,Y^{n}}
\\
& = &
	\dfrac{1}{2} \cdot
	\left[\;
		\overset{{\color{white}1}}{\pi_{d}}\!\left(\,S_{+}\,\right)
		\, , \,
		\pi_{d}\!\left(\,S_{-}\,\right)
		\,\right]
		(X^{m}Y^{n})
\end{eqnarray*}
This proves that
\begin{equation*}
\left[\;
	\overset{{\color{white}1}}{\pi_{d}}\!\left(\,S_{+}\,\right)
	\, , \,
	\pi_{d}\!\left(\,S_{-}\,\right)
	\,\right]
\;\; = \;\;
	2 \cdot \pi_{d}(S_{3})
\end{equation*}
Next, note
\begin{eqnarray*}
\left[\;
	\overset{{\color{white}1}}{\pi_{d}}\!\left(\,S_{3}\,\right)
	\, , \,
	\pi_{d}\!\left(\,S_{+}\,\right)
	\,\right]
	(X^{m}Y^{n})
& = &
	\left[\;\,
		-\,\dfrac{1}{2}\cdot\left(\,X\cdot\dfrac{\partial}{\partial\,X} - Y\cdot\dfrac{\partial}{\partial\,Y}\right)
		\; , \;
		-\,Y\cdot\dfrac{\partial}{\partial\,X}
		\,\right]
		(X^{m}Y^{n})
\\
& = &
	\left[\;\,
		\dfrac{1}{2}\cdot\left(\,X\cdot\dfrac{\partial}{\partial\,X} - Y\cdot\dfrac{\partial}{\partial\,Y}\right)
		\; , \;
		Y\cdot\dfrac{\partial}{\partial\,X}
		\,\right]
		(X^{m}Y^{n})
\\
& = &
	\dfrac{1}{2}\cdot\left(\,X\cdot\dfrac{\partial}{\partial\,X} - Y\cdot\dfrac{\partial}{\partial\,Y}\right)
	\left(\, Y \cdot m \cdot X^{m-1}\,Y^{n} \,\right)
\\
&&
	-\,Y\cdot\dfrac{\partial}{\partial\,X}\left(\,
		\dfrac{1}{2}\left(\,
			X \cdot m \cdot X^{m-1}\,Y^{n} \,-\, Y \cdot X^{m} \cdot n \cdot Y^{n-1}
			\,\right)
		\,\right)
\\
& = &
	\dfrac{1}{2}\cdot\left(\,X\cdot\dfrac{\partial}{\partial\,X} - Y\cdot\dfrac{\partial}{\partial\,Y}\right)
	\left(\, m \cdot X^{m-1}\,Y^{n+1} \,\right)
\\
&&
	-\,Y\cdot\dfrac{\partial}{\partial\,X}\left(\,
		\dfrac{1}{2}\,(m-n)\,X^{m}\,Y^{n}
		\,\right)
\\
& = &
	\dfrac{1}{2}\left(\,
		\overset{{\color{white}.}}{X} \cdot m \cdot (m-1) \cdot X^{m-2}\,Y^{n+1}
		\,-\,
		Y \cdot m \cdot X^{m-1} \cdot (n+1) \cdot Y^{n}
		\,\right)
\\
&&
	-\,Y\cdot\dfrac{1}{2}\,(m-n)\cdot m \cdot X^{m-1}\,Y^{n}
\\
& = &
	\dfrac{1}{2}\left(\,
		m\,(m-1)\cdot \overset{{\color{white}.}}{X^{m-1}}\,Y^{n+1}
		\,-\,
		m\,(n+1)\cdot X^{m-1}\,Y^{n+1}
		\,\right)
\\
&&
	-\,\dfrac{1}{2} \cdot m\,(m-n) \cdot X^{m-1}\,Y^{n+1}
\\
& = &
	\dfrac{m}{2}\left(\,
		m - \overset{{\color{white}.}}{1} - n - 1 - m + n
		\,\right)
	X^{m-1}\,Y^{n+1}
\;\; = \;\;
	\dfrac{m}{2}\left(\,-\,\overset{{\color{white}.}}{2}\,\right)X^{m-1}\,Y^{n+1}
\\
& = &
	-\,m\,\overset{{\color{white}1}}{X^{m-1}}\,Y^{n+1}
\end{eqnarray*}
and
\begin{eqnarray*}
\pi_{d}(S_{+})\!\left(\,\overset{{\color{white}.}}{X^{m}Y^{n}}\,\right)
& = &
	-\,Y\cdot\dfrac{\partial}{\partial\,X}\!\left(\,\overset{{\color{white}.}}{X^{m}Y^{n}}\,\right)
\;\; = \;\;
	-\, Y \cdot m \cdot X^{m-1}\,Y^{n}
\;\; = \;\;
	-\, m \, X^{m-1}\,Y^{n+1}
\\
& = &
	\left[\;
		\overset{{\color{white}1}}{\pi_{d}}\!\left(\,S_{3}\,\right)
		\, , \,
		\pi_{d}\!\left(\,S_{+}\,\right)
		\,\right]
		(X^{m}Y^{n})
\end{eqnarray*}
This proves:
\begin{equation*}
\left[\;
	\overset{{\color{white}1}}{\pi_{d}}\!\left(\,S_{3}\,\right)
	\, , \,
	\pi_{d}\!\left(\,S_{+}\,\right)
	\,\right]
\;\; = \;\;
	\pi_{d}(S_{+})
\end{equation*}
Similar calculations will show:
\begin{equation*}
\left[\;
	\overset{{\color{white}1}}{\pi_{d}}\!\left(\,S_{3}\,\right)
	\, , \,
	\pi_{d}\!\left(\,S_{-}\,\right)
	\,\right]
\;\; = \;\;
	-\,\pi_{d}(S_{-})
\end{equation*}
This completes the proof that
\,$\pi_{d} : \su(2) \otimes_{\Re} \C \longrightarrow \gl\!\left(\,\overset{{\color{white}.}}{\C}[X,Y]_{d}\,\right)$\,
is indeed a complex representation of the complex Lie algebra
\,$\su(2) \otimes_{\Re} \C$.\,
It remains to show that \,$\pi_{d}$\, is irreducible.
\qed


          %%%%% ~~~~~~~~~~~~~~~~~~~~ %%%%%
