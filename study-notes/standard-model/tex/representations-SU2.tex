
          %%%%% ~~~~~~~~~~~~~~~~~~~~ %%%%%

\chapter{Irreducible representations of $\mathfrak{su}(2)$}
\setcounter{theorem}{0}
\setcounter{equation}{0}

%\cite{vanDerVaart1996}
%\cite{Kosorok2008}

%\renewcommand{\theenumi}{\alph{enumi}}
%\renewcommand{\labelenumi}{\textnormal{(\theenumi)}$\;\;$}
\renewcommand{\theenumi}{\roman{enumi}}
\renewcommand{\labelenumi}{\textnormal{(\theenumi)}$\;\;$}

          %%%%% ~~~~~~~~~~~~~~~~~~~~ %%%%%

\section{Definition of \,$\textnormal{SU}(2)$\, and \,$\mathfrak{su}(2)$}

          %%%%% ~~~~~~~~~~~~~~~~~~~~ %%%%%

\vskip 0.1cm
\begin{definition}[$\textnormal{U}(n)$ and $\mathfrak{u}(n)$]
\mbox{}
\vskip 0.1cm
\noindent
The \textbf{unitary group} is defined as follows:
\begin{equation*}
\textnormal{U}(n)
\; := \;
	\left\{\;\,
		g \overset{{\color{white}.}}{\in} \textnormal{GL}(n,\C)
		\;\left\vert\;\,
			g^{\dagger} \cdot g = I_{n}
			\right.
		\;\right\}
\end{equation*}
where \,$g^{\dagger}$\, is the conjugate transpose of
\,$g \in \textnormal{GL}(n,\C)$\,
and
\,$I_{n} \in \textnormal{GL}(n,\C)$\,
is the identity matrix.
\vskip 0.1cm
\noindent
The \textbf{special unitary group} is defined as follows:
\begin{equation*}
\textnormal{SU}(n)
\; := \;
	\left\{\;\,
		g \overset{{\color{white}.}}{\in} \textnormal{GL}(n,\C)
		\;\left\vert\;\,
			g^{\dagger} \cdot g = I_{n}\,,
			\;
			\textnormal{det}(g) = 1
			\right.
		\;\right\}
\end{equation*}
\end{definition}

\vskip 0.5cm
\noindent
\begin{proposition}[Parametrization of $\textnormal{SU}(2)$]
\mbox{}
\vskip 0.1cm
\noindent
$\textnormal{SU}(2)$ admits the following parametrization:
\begin{equation*}
\textnormal{SU}{(2)}
\; := \;
	\left\{\,
		\left.
		\left(\begin{array}{rr}
		a & -\overline{b}
		\\
		\overset{{\color{white}-}}{b} & \overline{a}
		\end{array}\right)
		\overset{{\color{white}.}}{\in}
		\C^{2 \times 2}
		\;\;\right\vert\;\,
			\vert\, a \,\vert^{2} \,+\, \vert\, b \,\vert^{2} \,=\, 1
		\;\right\}
\end{equation*}
\end{proposition}
\proof
Suppose:
\begin{equation*}
X
\; = \;
	\left(\begin{array}{cc}
		a & c
		\\
		\overset{{\color{white}-}}{b} & d
		\end{array}\right)
\; \in \;
\textnormal{SU}(2)
\end{equation*}
Then, \,$X$\, satisfies \,$\det(X) = 1$\, and
\begin{equation*}
\left(\begin{array}{cc}
	1 & 0
	\\
	\overset{{\color{white}-}}{0} & 1
	\end{array}\right)
\; = \;
	X^{\dagger} \cdot X
\; = \;
	\left(\begin{array}{cc}
		\overline{a} & \overline{b}
		\\
		\overset{{\color{white}-}}{\overline{c}} & \overline{d}
		\end{array}\right)
	\cdot
	\left(\begin{array}{cc}
		a & c
		\\
		\overset{{\color{white}-}}{b} & d
		\end{array}\right)
\; = \;
	\left(\begin{array}{cc}
		a\overline{a} + b\overline{b} & \overline{a}c + \overline{b}d
		\\
		\overset{{\color{white}-}}{a\overline{c} + b\overline{d}} & c\overline{c} + d\overline{d}
		\end{array}\right)
\; \in \;
\textnormal{SU}(2)
\end{equation*}
The above matrix equation together with \,$\det(X)=1$\, are equivalent to the following system of equations:
\begin{equation*}
\left\{
	\begin{array}{ccc}
	\vert\,a\,\vert^{2} + \vert\,b\,\vert^{2} &=& 1
	\\
	\vert\,c\,\vert^{2} + \vert\,d\,\vert^{2} &\overset{{\color{white}1}}{=}& 1
	\\
	a\overline{c} \;\, + \,\; b\overline{d} &\overset{{\color{white}1}}{=}& 0
	\\
	ad \;\, - \,\; bc &\overset{{\color{white}1}}{=}& 1
	\end{array}
	\right.
\end{equation*}
First, note that
\begin{equation*}
a\overline{c} + b\overline{d} = 0
\quad\Longleftrightarrow\quad
	\left\langle
		\left(\begin{array}{c} a \\ b \end{array}\right)
		\,,\,
		\left(\begin{array}{c} c \\ d \end{array}\right)
		\right\rangle_{\C^{2}}
	\;=\;
	0
\end{equation*}
Since
\,$\dim_{\C}\left(\begin{array}{c} a \\ b \end{array}\right)^{\perp} =\, 1$,\,
the above equality (i.e., orthogonality of the two columns of $X$) implies:
\begin{equation*}
\left(\begin{array}{c} c \\ d \end{array}\right)
\; \in \;
	\left(\begin{array}{c} a \\ b \end{array}\right)^{\perp}
\; = \;
	\textnormal{span}_{\C}\left\{
		\left(\begin{array}{r} -\overline{b} \\ \overline{a} \end{array}\right)
		\right\}
\quad\Longleftrightarrow\quad
\left(\begin{array}{c} c \\ d \end{array}\right)
\; = \;
	\lambda \left(\begin{array}{r} -\overline{b} \\ \overline{a} \end{array}\right),
	\;\;
	\textnormal{for some $\lambda \in \C$}
\end{equation*}
So, we now know that $X$ has the form:
\begin{equation*}
X
\; = \;
	\left(\begin{array}{rr}
		a & -\lambda\,\overline{b}
		\\
		\overset{{\color{white}-}}{b} & \lambda\,\overline{a}
		\end{array}\right)
\end{equation*}
Next,
\begin{equation*}
1
\,=\, \det(X)
\,=\, a\cdot(\lambda\,\overline{a}) - b \cdot (-\lambda\,\overline{b})
\,=\, \lambda\cdot(\vert\,a\,\vert^{2} + \vert\,b\,\vert^{2})
\quad\Longrightarrow\quad
	\lambda = 1
\end{equation*}
We may now conclude that
\begin{equation*}
X
\; = \;
	\left(\begin{array}{rr}
		a & -\,\overline{b}
		\\
		\overset{{\color{white}-}}{b} & \overline{a}
		\end{array}\right),\,
\quad
\textnormal{where \,$\vert\,a\,\vert^{2} + \vert\,b\,\vert^{2} = 1$}
\end{equation*}
This completes the proof of the Proposition.
\qed

\vskip 0.5cm
\begin{definition}[$\textnormal{O}(n)$ and $\textnormal{SO}(n)$]
\mbox{}
\vskip 0.1cm
\noindent
The \textbf{orthogonal group} is defined as follows:
\begin{equation*}
\textnormal{O}(n)
\; := \;
	\left\{\;\,
		g \overset{{\color{white}.}}{\in} \textnormal{GL}(n,\Re)
		\;\left\vert\;\,
			g^{T} \cdot g = I_{n}
			\right.
		\;\right\}
\end{equation*}
The \textbf{special orthogonal group} is defined as follows:
\begin{equation*}
\textnormal{SO}(n)
\; := \;
	\left\{\;\,
		g \overset{{\color{white}.}}{\in} \textnormal{GL}(n,\Re)
		\;\left\vert\;\,
			g^{T} \cdot g = I_{n}\,,
			\;
			\textnormal{det}(g) = 1
			\right.
		\;\right\}
\end{equation*}
\end{definition}

\begin{proposition}[Lie algebras of $\textnormal{O}(n)$ and $\textnormal{SO}(n)$]
\begin{eqnarray*}
\mathfrak{o}(n)
& = &
	\left\{\;\,
		X \overset{{\color{white}.}}{\in} \mathfrak{gl}(n,\Re)
		\;\left\vert\;\,
			X^{T} = -X
			\right.
		\;\right\}
\\
\mathfrak{so}(n)
& = &
	\left\{\;\,
		X \overset{{\color{white}.}}{\in} \mathfrak{gl}(n,\Re)
		\;\left\vert\;\,
			X^{T} = -X\,,
			\;
			\textnormal{trace}(X) = 0
			\right.
		\;\right\}
\end{eqnarray*}
\end{proposition}

          %%%%% ~~~~~~~~~~~~~~~~~~~~ %%%%%

\section{Generators of \;$\textnormal{SO}(3)$\, and \,$\mathfrak{so}(3)$}

          %%%%% ~~~~~~~~~~~~~~~~~~~~ %%%%%

\vskip 0.1cm
\noindent
\textbf{Euler matrices}
\begin{equation*}
R_{1}(\phi)
\; := \;
	\left(\,
		\begin{array}{ccc}
			{\color{white}-}1 & {\color{white}-}0 & {\color{white}-}0 \\
			{\color{white}-}0 & {\color{white}-}\cos\phi & -\sin\phi \\
			{\color{white}-}0 & {\color{white}-}\sin\phi & {\color{white}-}\cos\phi \\
			\end{array}
		\,\right)
\end{equation*}
\begin{equation*}
R_{2}(\psi)
\; := \;
	\left(\,
		\begin{array}{ccc}
			{\color{white}-}\cos\psi & {\color{white}-}0 & {\color{white}-}\sin\psi \\
			{\color{white}-}0 & {\color{white}-}1 & {\color{white}-}0 \\
			-\sin\psi & {\color{white}-}0 & {\color{white}-}\cos\psi \\
			\end{array}
		\,\right)
\end{equation*}
\begin{equation*}
R_{3}(\theta)
\; := \;
	\left(\,
		\begin{array}{ccc}
			{\color{white}-}\cos\theta & -\sin\theta & {\color{white}-}0 \\
			{\color{white}-}\sin\theta & {\color{white}-}\cos\theta & {\color{white}-}0 \\
			{\color{white}-}0 & {\color{white}-}0 & {\color{white}-}1 \\
			\end{array}
		\,\right)
\end{equation*}

          %%%%% ~~~~~~~~~~~~~~~~~~~~ %%%%%

\vskip 0.5cm
\noindent
\textbf{The generators \,$J_{n} \in \C^{3 \times 3}$\, of the Euler matrices}
\begin{equation*}
R_{n}(\theta)
\; = \;
	\exp\!\left(\;\sqrt{-1}\cdot\theta \overset{{\color{white}1}}{\cdot} J_{n}\,\right)
\; = \;
	\exp\!\left(\;\i\cdot\theta \overset{{\color{white}1}}{\cdot} J_{n}\,\right)
\end{equation*}
Alternatively, note:
\begin{equation*}
\i \cdot J_{1}
\;\; = \;\;
	\left.\dfrac{\d}{\d\,\phi}\right\vert_{\phi = 0} R_{1}(\phi)
\;\; = \;
	\left.\left(\,
		\begin{array}{ccc}
			{\color{white}-}1 & {\color{white}-}0 & {\color{white}-}0 \\
			{\color{white}-}0 & {\color{white}-}\sin\phi & {\color{white}-}\cos\phi \\
			{\color{white}-}0 & {\color{black}-}\cos\phi & {\color{white}-}\sin\phi \\
			\end{array}
		\,\right)\right\vert_{\phi = 0}
\;\; = \;
	\left(\,
		\begin{array}{ccc}
			{\color{white}-}0 & {\color{white}-}0 & {\color{white}-}0 \\
			{\color{white}-}0 & {\color{white}-}0 & {\color{white}-}1 \\
			{\color{white}-}0 & {\color{black}-}1 & {\color{white}-}0 \\
			\end{array}
		\,\right)
\end{equation*}
Multiplying both sides by \,$-\,\i = -\,\sqrt{-1}$\, gives:
\begin{equation*}
J_{1}
\;\; = \;
	\left(\!\!
		\begin{array}{ccc}
			{\color{white}-}0 & {\color{white}-}0 & {\color{white}-}0 \\
			{\color{white}-}0 & {\color{white}-}0 & {\color{black}-}\i \\
			{\color{white}-}0 & {\color{white}-}\i & {\color{white}-}0 \\
			\end{array}
		\,\right)
\end{equation*}
Similarly,
\begin{equation*}
\i \cdot J_{2}
\;\; = \;\;
	\left.\dfrac{\d}{\d\,\psi}\right\vert_{\psi = 0} R_{2}(\psi)
\;\; = \;
	\left.\left(\!\!
		\begin{array}{ccc}
			{\color{white}-}\sin\psi & {\color{white}-}0 & {\color{black}-}\cos\psi \\
			{\color{white}-}0 & {\color{white}-}0 & {\color{white}-}0 \\
			{\color{white}-}\cos\psi & {\color{white}-}0 & {\color{white}-}\sin\psi \\
			\end{array}
		\,\right)\right\vert_{\psi = 0}
\;\; = \;
	\left(\!\!
		\begin{array}{ccc}
			{\color{white}-}0 & {\color{white}-}0 & {\color{black}-}1 \\
			{\color{white}-}0 & {\color{white}-}0 & {\color{white}-}0 \\
			{\color{white}-}1 & {\color{white}-}0 & {\color{white}-}0 \\
			\end{array}
		\,\right)
\end{equation*}
Multiplying both sides by \,$-\,\i = -\,\sqrt{-1}$\, gives:
\begin{equation*}
J_{2}
\;\; = \;
	\left(\!\!
		\begin{array}{ccc}
			{\color{white}-}0 & {\color{white}-}0 & {\color{white}-}\i \\
			{\color{white}-}0 & {\color{white}-}0 & {\color{white}-}0 \\
			{\color{black}-}\i & {\color{white}-}0 & {\color{white}-}0 \\
			\end{array}
		\,\right)
\end{equation*}
Lastly,
\begin{equation*}
\i \cdot J_{3}
\;\; = \;\;
	\left.\dfrac{\d}{\d\,\theta}\right\vert_{\theta = 0} R_{3}(\theta)
\;\; = \;
	\left.\left(\!
		\begin{array}{ccc}
			{\color{white}-}\sin\theta & {\color{white}-}\cos\theta & {\color{white}-}0 \\
			{\color{black}-}\cos\theta & {\color{white}-}\sin\theta & {\color{white}-}0 \\
			{\color{white}-}0 & {\color{white}-}0 & {\color{white}-}0 \\
			\end{array}
		\,\right)\right\vert_{\psi = 0}
\;\; = \;
	\left(
		\begin{array}{ccc}
			{\color{white}-}0 & {\color{white}-}1 & {\color{white}-}0 \\
			{\color{black}-}1 & {\color{white}-}0 & {\color{white}-}0 \\
			{\color{white}-}0 & {\color{white}-}0 & {\color{white}-}0 \\
			\end{array}
		\,\right)
\end{equation*}
Multiplying both sides by \,$-\,\i = -\,\sqrt{-1}$\, gives:
\begin{equation*}
J_{3}
\;\; = \;
	\left(\!
		\begin{array}{ccc}
			{\color{white}-}0 & {\color{black}-}\i & {\color{white}-}0 \\
			{\color{white}-}\i & {\color{white}-}0 & {\color{white}-}0 \\
			{\color{white}-}0 & {\color{white}-}0 & {\color{white}-}0 \\
			\end{array}
		\,\right)
\end{equation*}

          %%%%% ~~~~~~~~~~~~~~~~~~~~ %%%%%

\section{Properties of the generators \,$J_{1}, J_{2}, J_{3} \,\in\, \mathfrak{so}(3) \otimes_{\Re} \C$}

          %%%%% ~~~~~~~~~~~~~~~~~~~~ %%%%%

\begin{proposition}
{\color{white}.}\vskip -0.5cm{\color{white}.}
\begin{enumerate}
\item
	\textbf{Commutation relations:}\;\;
	\begin{equation*}
	\left[\,J_{k}\,\overset{{\color{white}1}}{,}\,J_{l}\,\right]
	\;\; = \;\;
		\sqrt{-1}\;\overset{3}{\underset{m=1}{\sum}}\,\varepsilon_{klm}\cdot J_{m}\,,
	\quad
	\textnormal{for each \,$k, l \in \{\,1,2,3\,\}$}\,,
	\end{equation*}
	where \,$\varepsilon_{klm}$\, is the fully anti-symmetric tensor.
\item
	\textbf{Raising and lowering operators:}\;\;
	Define \,$J_{+}\,,\, J_{-} \in \mathfrak{so}(3) \otimes_{\Re} \C \subset \mathcal{U}\!\left(\mathfrak{so}(3) \overset{{\color{white}.}}{\otimes_{\Re}} \C\right)$\, as follows:
	\begin{equation*}
	J_{\pm} \;\; := \;\; J_{1} \, \pm \sqrt{-1}\,J_{2}.
	\end{equation*}
	Then, the following equalities (of elements of $\mathcal{U}\!\left(\mathfrak{so}(3) \overset{{\color{white}.}}{\otimes_{\Re}} \C\right)$) hold:
	\begin{enumerate}
	\item
		$\left[\,J_{3}\,,\,J_{+}\,\right] \;=\; J_{+}$\,,
		\quad
		$\left[\,J_{3}\,,\,J_{-}\,\right] \;=\; -\,J_{-}$\,,
		\quad
		$\left[\,J_{+}\,,\,J_{-}\,\right] \;=\; 2\,J_{3}$
	\item
		$J^{2}$
		\; $=$ \; $(J_{3})^{2} \,-\, J_{3} \,+\, J_{+}J_{-}$
		\; $=$ \; $(J_{3})^{2} \,+\, J_{3} \,+\, J_{-}J_{+}$
	\item
		$(J_{\pm})^{\dagger} \; = \; J_{\mp}$
	\end{enumerate}
\item
	Suppose
	\,$\rho : \mathfrak{so}(3) \otimes_{\Re} \C \longrightarrow \mathfrak{gl}(V)$\,
	is an irreducible finite-dimensional complex representation, and
	\,$v \in V \backslash\{0\}$\, is an eigenvector of \,$\rho(J_{3})$\,
	corresponding to the eigenvalue \,$\lambda \in \Re$;\, thus, \,$\rho(J_{3})(v) \,=\, \lambda\,v$.
	Then, we have:
	\begin{equation*}
	\rho(J_{3})\!\left(\,\rho(J_{+})(\overset{{\color{white}-}}{v})\,\right) \, = \; (\lambda+1)\cdot\rho(J_{+})(v)\,
	\quad\textnormal{and}\quad\;
	\rho(J_{3})\!\left(\,\rho(J_{-})(\overset{{\color{white}-}}{v})\,\right) \, = \; (\lambda-1)\cdot\rho(J_{-})(v)
	\end{equation*}
\item
	\textbf{Casimir operator:}\;\;
	Define
	\,$J^{2}$
	\,$:=$\,
	$(J_{1})^{2} + (J_{2})^{2} + (J_{3})^{2}$
	\,$\in$\
	 $\mathcal{U}\!\left(\mathfrak{so}(3) \overset{{\color{white}.}}{\otimes_{\Re}} \C\right)$.
	Then,
	\begin{equation*}
	\left[\,J^{2}\,\overset{{\color{white}1}}{,}\,J_{k}\,\right]
	\;\; = \;\;
		0\,,
	\quad
	\textnormal{for each \,$k \in \{\,1,2,3\,\}$}\,.
	\end{equation*}
	Consequently (by Schur's Lemma, Corollary 4.30, \cite{Hall2015}), 
	\,$J^{2} \in \mathcal{U}\!\left(\mathfrak{so}(3) \overset{{\color{white}.}}{\otimes_{\Re}} \C\right)$\,
	acts as a scalar multiple of the identity in every irreducible
	representation\footnote{Furthermore, this scalar $\lambda \in \C$ uniquely determines
	the irreducible representation.
	Look up the classification theory of irreducible finite-dimensional complex representations
	of complex semisimple Lie algebras.
	Key words: Casimir operator, universal enveloping algebra. See Chapters 9 and 10, \cite{Hall2015}.}
	of \,$\mathcal{U}\!\left(\mathfrak{so}(3) \overset{{\color{white}.}}{\otimes_{\Re}} \C\right)$;\,
	more precisely, for each irreducible finite-dimensional complex representation
	\,$\rho : \mathcal{U}\!\left(\mathfrak{so}(3) \overset{{\color{white}.}}{\otimes_{\Re}} \C\right) \longrightarrow \mathfrak{gl}(V)$,\,
	we have \,$\rho(J^{2}) = \lambda \cdot \textnormal{\textbf{1}}_{V}$,\,
	for some \,$\lambda \in \C$.
\end{enumerate}
\end{proposition}

          %%%%% ~~~~~~~~~~~~~~~~~~~~ %%%%%

\begin{theorem}
{\color{white}.}\vskip -0.1cm
\noindent
\begin{enumerate}
\item
	The finite-dimensional irreducible representations of $\mathfrak{so}(3) \otimes_{\Re} \C$ is parametrized by the set
	\begin{equation*}
	\dfrac{1}{2} \cdot \Z
	\;\; := \;\;
		\left\{\;0 \,,\, \dfrac{1}{2} \,,\, 1 \,,\, \frac{3}{2} \,,\, 2 \,,\, \frac{5}{2} \,,\, \ldots \;\right\},
	\end{equation*}
	of non-negative integer multiples of \,$\dfrac{1}{2}$, in that, for each
	$s \in \dfrac{1}{2} \cdot \Z = \left\{\; 0 \,,\, \frac{1}{2}\,,\, 1\,,\, \frac{3}{2}\,,\, 2\,,\, \frac{5}{2}\,,\, \ldots \;\right\}$,
	there exists a unique (up to equivalence) complex representation
	$\rho_{s} : \mathcal{U}(\mathfrak{so}(3)\otimes_{\Re}\C) \longrightarrow \textnormal{End}(V_{s})$
	such that
	\begin{equation*}
	\rho_{s}(J^{2}) \; = \; s(s+1)\cdot\textnormal{\textbf{1}}_{V_{s}}.
	\end{equation*}
\item
	$\dim_{\C}(V_{s}) \, = \, 2s + 1$,\, for each
	\,$s \in \dfrac{1}{2} \cdot \Z = \left\{\; 0 \,,\, \frac{1}{2}\,,\, 1\,,\, \frac{3}{2}\,,\, 2\,,\, \frac{5}{2}\,,\, \ldots \;\right\}$.
\item
	For each
	\,$s \in \dfrac{1}{2} \cdot \Z = \left\{\; 0 \,,\, \frac{1}{2}\,,\, 1\,,\, \frac{3}{2}\,,\, 2\,,\, \frac{5}{2}\,,\, \ldots \;\right\}$,\,
	the spectrum
	$\sigma\!\left(\,\overset{{\color{white}-}}{\rho}_{s}(J_{3})\,\right)$
	of the operator $\rho_{s}(J_{3}) \in \textnormal{End}(V_{s})$
	consists of only eigenvalues and is given by:
	\begin{equation*}
	\sigma\!\left(\,\overset{{\color{white}-}}{\rho}_{s}(J_{3})\,\right)
	\;\; = \;\;
		\left\{\;
			-\overset{{\color{white}-}}{s} \,,\, -(s-1), -(s-2)
			\,,\;\, \ldots \,\;,\,
			(s-2) \,,\, (s-1) \,,\, s
			\;\right\},
	\end{equation*}
	and each eigenvalue in 
	$\sigma\!\left(\,\overset{{\color{white}-}}{\rho}_{s}(J_{3})\,\right)$
	has multiplicity one.
\item
	For each
	\,$s \in \dfrac{1}{2} \cdot \Z = \left\{\; 0 \,,\, \frac{1}{2}\,,\, 1\,,\, \frac{3}{2}\,,\, 2\,,\, \frac{5}{2}\,,\, \ldots \;\right\}$,\,
	let \,$v^{(s)}_{k} \in V_{s}\backslash\{0\}$\, be any normalized eigenvector
	of $\rho_{s}(J_{3})$ corresponding to the eigenvalue
	\,$k$ $\in$ $\sigma\!\left(\,\overset{{\color{white}-}}{\rho}_{s}(J_{3})\,\right)$
	$=$ $\left\{\;-\overset{{\color{white}-}}{s} \,,\, -(s-1) \,,\, \;\ldots\;,\, (s-1) \,,\, s\;\right\}$.\,
	Then, 
	\begin{enumerate}
	\item
		the eigenvectors
		\,$v^{(s)}_{-s} \,,\, v^{(s)}_{-(s-1)} \,,\; \ldots \;,\, v^{(s)}_{s-1} \,,\, v^{(s)}_{s}$\,
		form an orthonormal basis for $V_{s}$, and
	\item
		for each \,$k$ $\in$ $\sigma\!\left(\,\overset{{\color{white}-}}{\rho}_{s}(J_{3})\,\right)$
		$=$ $\left\{\;-\overset{{\color{white}-}}{s} \,,\, -(s-1) \,,\, \;\ldots\;,\, (s-1) \,,\, s\;\right\}$,\,
		we have:
		\begin{equation*}
		J_{\pm}\!\left(\,v^{(s)}_{k}\,\right)
		\; = \;
			\sqrt{{\color{white}.}
			s(s+1) - k(k \pm 1)
			{\color{white}.}}
			\,\cdot\,
			v^{(s)}_{k \pm 1}
		\end{equation*}
		In particular, \,$J_{\pm}\!\left(\,v^{(s)}_{\pm s}\,\right) \; = \; 0$.
	\end{enumerate}
\end{enumerate}
\end{theorem}

          %%%%% ~~~~~~~~~~~~~~~~~~~~ %%%%%

