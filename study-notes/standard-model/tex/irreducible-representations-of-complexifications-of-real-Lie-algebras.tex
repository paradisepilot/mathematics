
          %%%%% ~~~~~~~~~~~~~~~~~~~~ %%%%%

\section{Irreducible complex representations of a real Lie algebra versus those of its complexification}
\setcounter{theorem}{0}
\setcounter{equation}{0}

%\cite{vanDerVaart1996}
%\cite{Kosorok2008}

%\renewcommand{\theenumi}{\alph{enumi}}
%\renewcommand{\labelenumi}{\textnormal{(\theenumi)}$\;\;$}
\renewcommand{\theenumi}{\roman{enumi}}
\renewcommand{\labelenumi}{\textnormal{(\theenumi)}$\;\;$}

          %%%%% ~~~~~~~~~~~~~~~~~~~~ %%%%%

\begin{definition}
\mbox{}
\vskip 0.05cm
\noindent
Let \,$V$\, be a finite-dimensional vector space over \,$\Re$.\,
Then, the \textbf{complexification} \,$V_{\C}$\, is
the (finite-dimensional) vector space \,$\C$\, obtained as follows:
\begin{itemize}
\item
	The underlying set (of vectors) of \,$V_{\C}$\, is the set of all formal linear combinations
	of the form:
	\begin{equation*}
	v_{1} \, + \, \i\cdot v_{2}\,,
	\end{equation*}
	where \,$v_{1}, v_{2} \in V$.\,
\item
	Vector addition in \,$V_{\C}$\, is defined as follows:
	\begin{equation*}
	\left(\;\overset{{\color{white}.}}{v_{1}} \, + \, \i\cdot v_{2}\,\right)
	\; + \;
	\left(\;\overset{{\color{white}.}}{w_{1}} \, + \, \i\cdot w_{2}\,\right)
	\;\; := \;\;
		\left(\,\overset{{\color{white}.}}{v_{1}} + w_{1} \,\right)
		\; + \;
		\i\cdot\left(\,\overset{{\color{white}.}}{v_{2}} + w_{2} \,\right)
	\end{equation*}	
\item
	Complex scalar multiplication on \,$V_{\C}$\, is defined as follows:
	\begin{equation*}
	(\,a + \i \, b\,) \cdot \left(\;\overset{{\color{white}.}}{v_{1}} \, + \, \i\cdot v_{2}\,\right)
	\;\; := \;\;
		\left(\,a\cdot\overset{{\color{white}.}}{v_{1}} - b \cdot v_{2} \,\right)
		\; + \;
		\i\cdot\left(\,a\cdot\overset{{\color{white}.}}{v_{2}} + b \cdot v_{1} \,\right)
	\end{equation*}	
\end{itemize}
\end{definition}

          %%%%% ~~~~~~~~~~~~~~~~~~~~ %%%%%

\vskip 0.5cm
\begin{proposition}
\mbox{}
\vskip 0.05cm
\noindent
Let \,$\mathfrak{g}$\, be a finite-dimensional real Lie algebra and
\,$\mathfrak{g}_{\C}$\, its vector-space complexification.
Then, the Lie bracket of \,$\mathfrak{g}$\, has a unique extension to \,$\mathfrak{g}_{\C}$\,
that makes  \,$\mathfrak{g}_{\C}$\, into a complex Lie algebra.
The resulting complex Lie algebra -- still denoted as \,$\mathfrak{g}_{\C}$\, -- is called
the \textbf{complexification} of \,$\mathfrak{g}$.\,
\end{proposition}
\proof
The unique extension
\,$[\;\cdot\,,\,\cdot\,]_{\mathfrak{g}_{\C}}$\,
to \,$\mathfrak{g}_{\C}$\, of
\,$[\;\cdot\,,\,\cdot\,]_{\mathfrak{g}}$\,
is given by
\begin{equation*}
\left[\;
	X_{1} \overset{{\color{white}.}}{+} \i\cdot Y_{1}
	\;,\,
	X_{2} \overset{{\color{white}.}}{+} \i\cdot Y_{2}
	\,\right]_{\mathfrak{g}_{\C}}
\;\, := \;\;
	\left(\;
		[\,X_{1},X_{2}\,]_{\mathfrak{g}}
		\; \overset{{\color{white}1}}{-} \,
		[\,Y_{1},Y_{2}\,]_{\mathfrak{g}}
		\,\right)
	\; + \;
	\i \cdot\! \left(\;
		[\,X_{1},Y_{2}\,]_{\mathfrak{g}}
		\; \overset{{\color{white}1}}{+} \,
		[\,Y_{1},X_{2}\,]_{\mathfrak{g}}
		\,\right)
\end{equation*}
for each \,$X_{1}, Y_{1}, X_{2}, Y_{2} \in \mathfrak{g}$.
For the proof, see Proposition 3.37, p.65, \cite{Hall2015}.
\qed

          %%%%% ~~~~~~~~~~~~~~~~~~~~ %%%%%

\vskip 0.5cm
\begin{proposition}[Universal property of the complexification of a real Lie algebra]
\label{UniqueExtensionOfRealLieAlgebraHomomorphisms}
\mbox{}
\vskip 0.05cm
\noindent
Let \,$\mathfrak{g}$\, be a finite-dimensional real Lie algebra and
\,$\mathfrak{g}_{\C}$\, its Lie-algebra complexification.
Let \,$\mathfrak{h}$\, be an arbitrary complex Lie algebra.
Then, every real Lie algebra homomorphism \,$\pi$\,
from \,$\mathfrak{g}$\, into \,$\mathfrak{h}$\,
extends uniquely to a complex Lie algebra homomorphsim \,$\pi_{\C}$\,
from \,$\mathfrak{g}_{\C}$\, into \,$\mathfrak{h}$.\,
\end{proposition}
\proof
The unique extension \,$\pi_{\C}$\, is given by
\begin{equation*}
\pi_{\C}\!\left(\,X \overset{{\color{white}.}}{+} \i\cdot Y \right)
\;\, := \;\;
	\pi(X) \overset{{\color{white}.}}{+} \i\cdot \pi(Y)\,,
\quad
\textnormal{for each \,$X, Y \in \mathfrak{g}$}
\end{equation*}
For the proof, see Proposition 3.39, p.67, \cite{Hall2015}.
\qed

          %%%%% ~~~~~~~~~~~~~~~~~~~~ %%%%%

\begin{proposition}
\mbox{}
\vskip 0.05cm
\noindent
Let \,$\mathfrak{g}$\, be a real Lie algebra and
\,$\mathfrak{g}_{\C}$\, its Lie-algebra complexification.
Let \,$V$\, be an arbitrary finite-dimensional complex vector space.
Then, the following statements are true:
\begin{enumerate}
\item
	Every real Lie algebra homomorphism
	\,$\pi : \mathfrak{g} \longrightarrow \End_{\Re}(V)$\,
	has a unique extension to a complex Lie algebra homomorphism
	\,$\pi_{\C} : \mathfrak{g}_{\C} \longrightarrow \End_{\C}(V)$.\,
\item
	$\pi$\, is irreducible if and only if \,$\pi_{\C}$\, is irreducible.
\end{enumerate}
\end{proposition}
\proof
\begin{enumerate}
\item
	Immediate by Proposition \ref{UniqueExtensionOfRealLieAlgebraHomomorphisms}.
\item
	Recall that a representation of (real or complex) Lie algebra is irreducible if it has
	no non-trivial invariant subspaces.
	Thus, the equivalence of the irreducibility of \,$\pi$\, and that of \,$\pi_{\C}$\,
	will follow from the fact that the two representations share the same collection
	of invariant subspaces. In other words, it suffices to establish the following:
	\vskip 0.2cm
	\noindent
	\textbf{Claim 1:}\quad
	Suppose \,$\{\,0\,\} \subsetneq W \subsetneq V$\, is a proper (complex) subspace of \,$V$.\,
	Then, \,$W$\, is \,$\pi$-invariant if and only if it is \,$\pi_{\C}$-invariant. 
	\vskip 0.1cm
	\noindent
	Proof of Claim 1:\quad
	Let \,$w \in W$\, be an arbitrary element of \,$W$.\,
	Suppose first that \,$W$\, is \,$\pi$-invariant.
	Then, for each \,$X, Y \in \mathfrak{g}$,\, we have:
	\begin{eqnarray*}
	\pi_{\C}\!\left(\,X \overset{{\color{white}.}}{+} \i\cdot Y \right) \cdot [\,w\,]
	& = &
		\left(\,\pi(X) \overset{{\color{white}.}}{+} \i\cdot \pi(Y) \right) \cdot [\,w\,]
	\\
	& = &
		\pi(X) \cdot w \;\overset{{\color{white}.}}{+}\; \i\cdot \pi(Y) \cdot w
	\;\; \in \;\;
		W \; + \; \i \cdot W
	\;\; = \;\;
		W,
	\end{eqnarray*}
	which proves that \,$W$\, is also \,$\pi_{\C}$-invariant.
	Conversely, now suppose that \,$W$\, is \,$\pi_{\C}$-invariant.
	Then, for each \,$X \in \mathfrak{g}$,\, we have:
	\begin{equation*}
	\pi(X)\cdot w
	\;\; = \;\;
		\left(\,\pi(X) \overset{{\color{white}.}}{+} \i\cdot \pi(0_{\mathfrak{g}}) \right) \cdot [\,w\,]
	\;\; = \;\;
		\pi_{\C}\!\left(\,X \overset{{\color{white}.}}{+} \i\cdot 0_{\mathfrak{g}} \right) \cdot [\,w\,]
	\;\; \in \;\;
		W,
	\end{equation*}
	which proves that \,$W$\, is also \,$\pi$-invariant.
	This proves Claim 1, as well as completes the proof of the Proposition.
\end{enumerate}
\qed

\vskip 0.5cm
{\color{red}
\begin{remark}
\mbox{}
\vskip 0.05cm
\noindent
Intuitively speaking, the preceding Proposition says that
the set of irreducible finite-dimensional complex representations of a real Lie algebra
is a ``subset'' of 
the set of irreducible finite-dimensional complex representations of its (Lie-algebra) complexification.
\end{remark}
}

          %%%%% ~~~~~~~~~~~~~~~~~~~~ %%%%%
