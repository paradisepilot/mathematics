
          %%%%% ~~~~~~~~~~~~~~~~~~~~ %%%%%

\section{Irreducible representations of \,$\so(3)$}
\setcounter{theorem}{0}
\setcounter{equation}{0}

%\cite{vanDerVaart1996}
%\cite{Kosorok2008}

%\renewcommand{\theenumi}{\alph{enumi}}
%\renewcommand{\labelenumi}{\textnormal{(\theenumi)}$\;\;$}
\renewcommand{\theenumi}{\roman{enumi}}
\renewcommand{\labelenumi}{\textnormal{(\theenumi)}$\;\;$}

          %%%%% ~~~~~~~~~~~~~~~~~~~~ %%%%%

\vskip 0.3cm
\begin{definition}[$\textnormal{O}(n)$ and $\textnormal{SO}(n)$]
\mbox{}
\vskip 0.1cm
\noindent
The \textbf{orthogonal group} is defined as follows:
\begin{equation*}
\textnormal{O}(n)
\; := \;
	\left\{\;\,
		g \overset{{\color{white}.}}{\in} \textnormal{GL}(n,\Re)
		\;\left\vert\;\,
			g^{T} \cdot g = I_{n}
			\right.
		\;\right\}
\end{equation*}
The \textbf{special orthogonal group} is defined as follows:
\begin{equation*}
\textnormal{SO}(n)
\; := \;
	\left\{\;\,
		g \overset{{\color{white}.}}{\in} \textnormal{GL}(n,\Re)
		\;\left\vert\;\,
			g^{T} \cdot g = I_{n}\,,
			\;
			\textnormal{det}(g) = 1
			\right.
		\;\right\}
\end{equation*}
\end{definition}

\vskip 0.5cm
\begin{proposition}[Set-theoretic characterizations of the Lie algebras of $\textnormal{O}(n)$ and $\textnormal{SO}(n)$]
\begin{eqnarray*}
\mathfrak{o}(n)
& = &
	\left\{\;\,
		X \overset{{\color{white}.}}{\in} \gl(n,\Re)
		\;\left\vert\;\,
			X^{T} = -X
			\right.
		\;\right\}
\\
\so(n)
& = &
	\left\{\;\,
		X \overset{{\color{white}.}}{\in} \gl(n,\Re)
		\;\left\vert\;\,
			X^{T} = -X\,,
			\;
			\textnormal{trace}(X) = 0
			\right.
		\;\right\}
\end{eqnarray*}
\end{proposition}

          %%%%% ~~~~~~~~~~~~~~~~~~~~ %%%%%

\clearpage
\vskip 0.5cm
\begin{proposition}[Some {\color{red}convenient generators} of \,$\so(3)$\, and \,$\so(3) \otimes_{\Re} \C$]
\mbox{}
\vskip 0.1cm
\noindent
Let
\,$R_{1}(\phi),\, R_{2}(\psi),\, R_{3}(\theta)$\,
be the one-parameter families of
\textbf{rotation matrices} in $3$-dimensional Euclidean space
around the $x$-, $y$, and $z$-axes, respectively, i.e.,
\begin{eqnarray*}
R_{1}(\phi)
& := &
	\left(\,
		\begin{array}{ccc}
			{\color{white}--}1 & {\color{white}-.}0 & {\color{white}-}0 \\
			{\color{white}--}0 & {\color{white}-.}\cos\phi & -\sin\phi \\
			{\color{white}--}0 & {\color{white}-.}\sin\phi & {\color{white}-}\cos\phi \\
			\end{array}
		\,\right)
\\
R_{2}(\psi)
& := &
	\left(\,
		\begin{array}{ccc}
			{\color{white}-}\cos\psi & {\color{white}-.}0 & {\color{white}--}\sin\psi \\
			{\color{white}-}0 & {\color{white}-.}1 & {\color{white}--}0 \\
			{\color{red}-}\sin\psi & {\color{white}-.}0 & {\color{white}--}\cos\psi \\
			\end{array}
		\right)
\\
R_{3}(\theta)
& := &
	\left(\,
		\begin{array}{ccc}
			{\color{white}-}\cos\theta & -\sin\theta & {\color{white}--}0 \\
			{\color{white}-}\sin\theta & {\color{white}-}\cos\theta & {\color{white}--}0 \\
			{\color{white}-}0 & {\color{white}-}0 & {\color{white}--}1 \\
			\end{array}
		\;\,\right)
\end{eqnarray*}
Let
\,$X_{1},\, X_{2},\, X_{3} \,\in\, \C^{3 \times 3}$\,
be the \textbf{infinitesimal generators} of
\,$R_{1}(\phi),\, R_{2}(\psi),\, R_{3}(\theta)$,\,
respectively, i.e.,
\begin{eqnarray*}
X_{1}
& := &
	\left.\dfrac{\d}{\d\,\phi}\right\vert_{\phi = 0} R_{1}(\phi)
\;\;\; = \;
	\left.\left(\,
		\begin{array}{ccc}
			{\color{white}-}1 & {\color{white}-}0 & {\color{white}-}0 \\
			{\color{white}-}0 & {\color{black}-}\sin\phi & {\color{black}-}\cos\phi \\
			{\color{white}-}0 & {\color{white}-}\cos\phi & {\color{black}-}\sin\phi \\
			\end{array}
		\,\right)\right\vert_{\phi = 0}
\,\, = \;
	\left(
		\begin{array}{ccc}
			{\color{white}-}0 & {\color{white}-}0 & {\color{white}-}0 \\
			{\color{white}-}0 & {\color{white}-}0 & {\color{black}-}1 \\
			{\color{white}-}0 & {\color{white}-}1 & {\color{white}-}0 \\
			\end{array}
		\,\right)
\\
X_{2}
& := &
	\left.\dfrac{\d}{\d\,\psi}\right\vert_{\psi = 0} R_{2}(\psi)
\;\; = \;
	\left.\left(\!\!
		\begin{array}{ccc}
			{\color{black}-}\sin\psi & {\color{white}-}0 & {\color{black}-}\cos\psi \\
			{\color{white}-}0 & {\color{white}-}0 & {\color{white}-}0 \\
			{\color{white}-}\cos\psi & {\color{white}-}0 & {\color{black}-}\sin\psi \\
			\end{array}
		\,\right)\right\vert_{\psi = 0}
\;\; = \;
	\left(
		\begin{array}{ccc}
			{\color{white}-}0 & {\color{white}-}0 & {\color{white}-}1 \\
			{\color{white}-}0 & {\color{white}-}0 & {\color{white}-}0 \\
			{\color{black}-}1 & {\color{white}-}0 & {\color{white}-}0 \\
			\end{array}
		\,\right)
\\
X_{3}
& := &
	\,\left.\dfrac{\d}{\d\,\theta}\right\vert_{\theta = 0} R_{3}(\theta)
\;\;\;\, = \;
	\left.\left(\!
		\begin{array}{ccc}
			{\color{black}-}\sin\theta & {\color{black}-}\cos\theta & {\color{white}-}0 \\
			{\color{white}-}\cos\theta & {\color{black}-}\sin\theta & {\color{white}-}0 \\
			{\color{white}-}0 & {\color{white}-}0 & {\color{white}-}0 \\
			\end{array}
		\,\right)\right\vert_{\psi = 0}
\;\;\, = \;
	\left(
		\begin{array}{ccc}
			{\color{white}-}0 & {\color{black}-}1 & {\color{white}-}0 \\
			{\color{white}-}1 & {\color{white}-}0 & {\color{white}-}0 \\
			{\color{white}-}0 & {\color{white}-}0 & {\color{white}-}0 \\
			\end{array}
		\,\right)
\end{eqnarray*}
Let \,$J_{1}, J_{2}, J_{3} \,\in\, \C^{3 \times 3}$\, be the \textbf{Euler matrices}, i.e.,
\begin{eqnarray*}
J_{1}
& := &
	\i \cdot X_{1}
\;\; = \;\;
	\left(\!\!
		\begin{array}{ccc}
			{\color{white}-}0 & {\color{white}-}0 & {\color{white}-}0 \\
			{\color{white}-}0 & {\color{white}-}0 & {\color{black}-}\i \\
			{\color{white}-}0 & {\color{white}-}\i & {\color{white}-}0 \\
			\end{array}
		\,\right)
\\
J_{2}
& := &
	\i \cdot X_{2}
\;\; = \;\;
	\left(\!\!
		\begin{array}{ccc}
			{\color{white}-}0 & {\color{white}-}0 & {\color{white}-}\i \\
			{\color{white}-}0 & {\color{white}-}0 & {\color{white}-}0 \\
			{\color{black}-}\i & {\color{white}-}0 & {\color{white}-}0 \\
			\end{array}
		\,\right)
\\
J_{3}
& := &
	\i \cdot X_{3}
\;\; = \;\,
	\left(\!
		\begin{array}{ccc}
			{\color{white}-}0 & {\color{black}-}\i & {\color{white}-}0 \\
			{\color{white}-}\i & {\color{white}-}0 & {\color{white}-}0 \\
			{\color{white}-}0 & {\color{white}-}0 & {\color{white}-}0 \\
			\end{array}
		\,\right)
\end{eqnarray*}
Define the \textbf{raising operator}
\,$J_{+} \in \C^{3 \times 3}$\,
and
\textbf{lowering operator}
\,$J_{-} \in \C^{3 \times 3}$\,
as follows:
\begin{equation*}
J_{\pm} \;\; := \;\; J_{1} \, \pm \sqrt{-1}\,J_{2}
\end{equation*}
Thus, explicitly,
\begin{equation*}
J_{+}
\;\; := \;\;
\left(\!
	\begin{array}{ccc}
		{\color{white}-}0 & {\color{white}-}0 & {\color{black}-}1 \\
		{\color{white}-}0 & {\color{white}-}0 & {\color{black}-}\i \\
		{\color{white}-}1 & {\color{white}-}\i & {\color{white}-}0 \\
		\end{array}
	\,\right),
\quad\quad
J_{-}
\;\; := \;\;
\left(\!
	\begin{array}{ccc}
		{\color{white}-}0 & {\color{white}-}0 & {\color{white}-}1 \\
		{\color{white}-}0 & {\color{white}-}0 & {\color{black}-}\i \\
		{\color{black}-}1 & {\color{white}-}\i & {\color{white}-}0 \\
		\end{array}
	\,\right)
\end{equation*}
Then, the following statements are true:
\begin{enumerate}
\item
	$X_{1},\, X_{2},\, X_{3}$ are elements of \,$\so(3)$,\,
	they are furthermore generators of \,$\so(3)$,\, and
	they satisfy the following commutation relations:
	\begin{equation*}
	\left[\,X_{1}\,,\,X_{2}\,\right] \;=\; +\,X_{3}
	\quad
	\left[\,X_{3}\,,\,X_{1}\,\right] \;=\; +\,X_{2}
	\quad
	\left[\,X_{2}\,,\,X_{3}\,\right] \;=\; +\,X_{1}
	\end{equation*}
	More succinctly,
	\begin{equation*}
	\left[\,X_{a}\,,\,X_{b}\,\right] \;\;=\;\; \overset{3}{\underset{c\,=\,1}{\sum}}\;\varepsilon_{abc}\,X_{c}\,,
	\quad
	\textnormal{for each \,$a, b \in \{\,1,2,3\,\}$}\,,
	\end{equation*}
	where \,$\varepsilon_{abc}$\, is the fully anti-symmetric tensor.
\item
	$J_{1},\, J_{2},\, J_{3}$\, are elements of \,$\so(3) \otimes_{\Re} \C$,\,
	they are furthermore generators of \,$\so(3) \otimes_{\Re} \C$,\, and
	they satisfy the following commutation relations:
	\begin{equation*}
	\left[\,J_{a}\,\overset{{\color{white}1}}{,}\,J_{b}\,\right]
	\;\; = \;\;
		\sqrt{-1}\,\cdot\overset{3}{\underset{c\,=\,1}{\sum}}\;\varepsilon_{abc}\cdot J_{c}\,,
	\quad
	\textnormal{for each \,$a, b \in \{\,1,2,3\,\}$}\,,
	\end{equation*}
	where \,$\varepsilon_{abc}$\, is the fully anti-symmetric tensor.
\item
	$J_{+},\, J_{-}$\, are elements of \,$\so(3) \otimes_{\Re} \C$.\,
	$J_{+},\, J_{-}$\, satisfy the following equality:
	\begin{equation*}
	(J_{\pm})^{\dagger} \;\; = \;\; J_{\mp}
	\end{equation*}
	$J_{+},\, J_{-},\, J_{3}$\, are generators of \,$\so(3) \otimes_{\Re} \C$,\, and
	they satisfy the following commutation relations:
	\begin{equation*}
	\left[\,J_{3}\,,\,J_{\pm}\,\right] \;=\; \pm\,J_{\pm}\,,
	\quad
	\left[\,J_{+}\,,\,J_{-}\,\right] \;=\; 2\,J_{3}
	\end{equation*}
\item
	Suppose
	\,$\rho : \so(3) \otimes_{\Re} \C \longrightarrow \gl(V)$\,
	is an irreducible finite-dimensional complex representation, and
	\,$v \in V \backslash\{0\}$\, is an eigenvector of \,$\rho(J_{3})$\,
	corresponding to the eigenvalue \,$\lambda \in \Re$;\, thus, \,$\rho(J_{3})(v) \,=\, \lambda\,v$.
	Then, we have:
	\begin{equation*}
	\rho(J_{3})\!\left(\,\rho(J_{+})(\overset{{\color{white}-}}{v})\,\right) \, = \; (\lambda+1)\cdot\rho(J_{+})(v)\,
	\quad\textnormal{and}\quad\;
	\rho(J_{3})\!\left(\,\rho(J_{-})(\overset{{\color{white}-}}{v})\,\right) \, = \; (\lambda-1)\cdot\rho(J_{-})(v)
	\end{equation*}
\item
	\textbf{Casimir operator:}\;\;
	Define
	\,$J^{2}$
	\,$:=$\,
	$(J_{1})^{2} + (J_{2})^{2} + (J_{3})^{2}$
	\,$\in$\
	 $\mathcal{U}\!\left(\so(3) \overset{{\color{white}.}}{\otimes_{\Re}} \C\right)$.\,
	Then, the following equalities of elements of
	\;$\mathcal{U}\!\left(\so(3) \overset{{\color{white}.}}{\otimes_{\Re}} \C\right)$
	hold:
	\begin{equation*}
	J^{2}
	\;\; =\;\;
		(J_{3})^{2} \,-\, J_{3} \,+\, J_{+}J_{-}
	\;\; =\;\;
		(J_{3})^{2} \,+\, J_{3} \,+\, J_{-}J_{+}\,,
	\end{equation*}
	and
	\begin{equation*}
	\left[\,J^{2}\,\overset{{\color{white}1}}{,}\,J_{a}\,\right]
	\;\; = \;\;
		0\,,
	\quad
	\textnormal{for each \,$a \in \{\,1,2,3\,\}$}\,.
	\end{equation*}
	Consequently (by Schur's Lemma, Corollary 4.30, \cite{Hall2015}), 
	\,$J^{2} \in \mathcal{U}\!\left(\so(3) \overset{{\color{white}.}}{\otimes_{\Re}} \C\right)$\,
	acts as a scalar multiple of the identity in every irreducible
	representation\footnote{Furthermore, this scalar $\lambda \in \C$ uniquely determines
	the irreducible representation.
	Look up the classification theory of irreducible finite-dimensional complex representations
	of complex semisimple Lie algebras.
	Key words: Casimir operator, universal enveloping algebra. See Chapters 9 and 10, \cite{Hall2015}.}
	of \,$\mathcal{U}\!\left(\so(3) \overset{{\color{white}.}}{\otimes_{\Re}} \C\right)$;\,
	more precisely, for each irreducible finite-dimensional complex representation
	\,$\rho : \mathcal{U}\!\left(\so(3) \overset{{\color{white}.}}{\otimes_{\Re}} \C\right) \longrightarrow \gl(V)$,\,
	we have \,$\rho(J^{2}) = \lambda \cdot \textnormal{\textbf{1}}_{V}$,\,
	for some \,$\lambda \in \C$.
\end{enumerate}
\end{proposition}

          %%%%% ~~~~~~~~~~~~~~~~~~~~ %%%%%

\vskip 0.5cm
\begin{theorem}
{\color{white}.}\vskip -0.1cm
\noindent
\begin{enumerate}
\item
	The finite-dimensional irreducible representations of $\so(3) \otimes_{\Re} \C$ is parametrized by the set
	\begin{equation*}
	\left\{\,\overset{{\color{white}.}}{0}\,\right\} \,\bigcup\, \dfrac{1}{2} \cdot \N
	\;\; := \;\;
		\left\{\;0 \,,\, \dfrac{1}{2} \,,\, 1 \,,\, \frac{3}{2} \,,\, 2 \,,\, \frac{5}{2} \,,\, \ldots \;\right\},
	\end{equation*}
	of non-negative integer multiples of \,$\dfrac{1}{2}$, in that, for each
	$s$
	$\in$ $\left\{\,\overset{{\color{white}.}}{0}\,\right\} \,\bigcup\, \dfrac{1}{2} \cdot \N$
	$=$ $\left\{\; 0 \,,\, \frac{1}{2}\,,\, 1\,,\, \frac{3}{2}\,,\, 2\,,\, \frac{5}{2}\,,\, \ldots \;\right\}$,
	there exists a unique (up to equivalence) complex representation
	$\rho_{s} : \mathcal{U}(\so(3)\otimes_{\Re}\C) \longrightarrow \textnormal{End}(V_{s})$
	such that
	\begin{equation*}
	\rho_{s}(J^{2}) \; = \; s(s+1)\cdot\textnormal{\textbf{1}}_{V_{s}}.
	\end{equation*}
\item
	$\dim_{\C}(V_{s}) \, = \, 2s + 1$,\, for each
	$s$
	$\in$ $\left\{\,\overset{{\color{white}.}}{0}\,\right\} \,\bigcup\, \dfrac{1}{2} \cdot \N$
	$=$ $\left\{\; 0 \,,\, \frac{1}{2}\,,\, 1\,,\, \frac{3}{2}\,,\, 2\,,\, \frac{5}{2}\,,\, \ldots \;\right\}$.
\item
	For each
	$s$
	$\in$ $\left\{\,\overset{{\color{white}.}}{0}\,\right\} \,\bigcup\, \dfrac{1}{2} \cdot \N$
	$=$ $\left\{\; 0 \,,\, \frac{1}{2}\,,\, 1\,,\, \frac{3}{2}\,,\, 2\,,\, \frac{5}{2}\,,\, \ldots \;\right\}$,\,
	the spectrum
	$\sigma\!\left(\,\overset{{\color{white}-}}{\rho}_{s}(J_{3})\,\right)$
	of the operator $\rho_{s}(J_{3}) \in \textnormal{End}(V_{s})$
	consists of only eigenvalues and is given by:
	\begin{equation*}
	\sigma\!\left(\,\overset{{\color{white}-}}{\rho}_{s}(J_{3})\,\right)
	\;\; = \;\;
		\left\{\;
			-\overset{{\color{white}-}}{s} \,,\, -(s-1), -(s-2)
			\,,\;\, \ldots \,\;,\,
			(s-2) \,,\, (s-1) \,,\, s
			\;\right\},
	\end{equation*}
	and each eigenvalue in 
	$\sigma\!\left(\,\overset{{\color{white}-}}{\rho}_{s}(J_{3})\,\right)$
	has multiplicity one.
\item
	For each
	$s$
	$\in$ $\left\{\,\overset{{\color{white}.}}{0}\,\right\} \,\bigcup\, \dfrac{1}{2} \cdot \N$
	$=$ $\left\{\; 0 \,,\, \frac{1}{2}\,,\, 1\,,\, \frac{3}{2}\,,\, 2\,,\, \frac{5}{2}\,,\, \ldots \;\right\}$,\,
	let \,$v^{(s)}_{k} \in V_{s}\backslash\{0\}$\, be any normalized eigenvector
	of $\rho_{s}(J_{3})$ corresponding to the eigenvalue
	\,$k$ $\in$ $\sigma\!\left(\,\overset{{\color{white}-}}{\rho}_{s}(J_{3})\,\right)$
	$=$ $\left\{\;-\overset{{\color{white}-}}{s} \,,\, -(s-1) \,,\, \;\ldots\;,\, (s-1) \,,\, s\;\right\}$.\,
	Then, 
	\begin{enumerate}
	\item
		the eigenvectors
		\,$v^{(s)}_{-s} \,,\, v^{(s)}_{-(s-1)} \,,\; \ldots \;,\, v^{(s)}_{s-1} \,,\, v^{(s)}_{s}$\,
		form an orthonormal basis for $V_{s}$, and
	\item
		for each \,$k$ $\in$ $\sigma\!\left(\,\overset{{\color{white}-}}{\rho}_{s}(J_{3})\,\right)$
		$=$ $\left\{\;-\overset{{\color{white}-}}{s} \,,\, -(s-1) \,,\, \;\ldots\;,\, (s-1) \,,\, s\;\right\}$,\,
		we have:
		\begin{equation*}
		J_{\pm}\!\left(\,v^{(s)}_{k}\,\right)
		\; = \;
			\sqrt{{\color{white}.}
			s(s+1) - k(k \pm 1)
			{\color{white}.}}
			\,\cdot\,
			v^{(s)}_{k \pm 1}
		\end{equation*}
		In particular, \,$J_{\pm}\!\left(\,v^{(s)}_{\pm s}\,\right) \; = \; 0$.
	\end{enumerate}
\end{enumerate}
\end{theorem}

          %%%%% ~~~~~~~~~~~~~~~~~~~~ %%%%%

