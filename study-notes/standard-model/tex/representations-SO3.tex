
          %%%%% ~~~~~~~~~~~~~~~~~~~~ %%%%%

\chapter{Irreducible representations of $\textnormal{SO}(3)$}
\setcounter{theorem}{0}
\setcounter{equation}{0}

%\cite{vanDerVaart1996}
%\cite{Kosorok2008}

%\renewcommand{\theenumi}{\alph{enumi}}
%\renewcommand{\labelenumi}{\textnormal{(\theenumi)}$\;\;$}
\renewcommand{\theenumi}{\roman{enumi}}
\renewcommand{\labelenumi}{\textnormal{(\theenumi)}$\;\;$}

          %%%%% ~~~~~~~~~~~~~~~~~~~~ %%%%%

\section{Definition of \,$\textnormal{SO}(3)$}

          %%%%% ~~~~~~~~~~~~~~~~~~~~ %%%%%

\vskip 0.1cm
\begin{definition}[$\textnormal{O}(n)$ and $\textnormal{SO}(n)$]
\mbox{}
\vskip 0.1cm
\noindent
The \textbf{orthogonal group} is defined as follows:
\begin{equation*}
\textnormal{O}(n)
\; := \;
	\left\{\;\,
		g \overset{{\color{white}.}}{\in} \textnormal{GL}(n,\Re)
		\;\left\vert\;\,
			g^{T} \cdot g = I_{n}
			\right.
		\;\right\}
\end{equation*}
The \textbf{special orthogonal group} is defined as follows:
\begin{equation*}
\textnormal{SO}(n)
\; := \;
	\left\{\;\,
		g \overset{{\color{white}.}}{\in} \textnormal{GL}(n,\Re)
		\;\left\vert\;\,
			g^{T} \cdot g = I_{n}\,,
			\;
			\textnormal{det}(g) = 1
			\right.
		\;\right\}
\end{equation*}
\end{definition}

          %%%%% ~~~~~~~~~~~~~~~~~~~~ %%%%%

\section{Generators of \,$\textnormal{SO}(3)$\, and \,$\mathfrak{so}(3)$}

          %%%%% ~~~~~~~~~~~~~~~~~~~~ %%%%%

\vskip 0.1cm
\noindent
\textbf{Euler matrices}
\begin{equation*}
R_{1}(\phi)
\; := \;
	\left(\,
		\begin{array}{ccc}
			{\color{white}-}1 & {\color{white}-}0 & {\color{white}-}0 \\
			{\color{white}-}0 & {\color{white}-}\cos\phi & -\sin\phi \\
			{\color{white}-}0 & {\color{white}-}\sin\phi & {\color{white}-}\cos\phi \\
			\end{array}
		\,\right)
\end{equation*}
\begin{equation*}
R_{2}(\psi)
\; := \;
	\left(\,
		\begin{array}{ccc}
			{\color{white}-}\cos\psi & {\color{white}-}0 & {\color{white}-}\sin\psi \\
			{\color{white}-}0 & {\color{white}-}1 & {\color{white}-}0 \\
			-\sin\psi & {\color{white}-}0 & {\color{white}-}\cos\psi \\
			\end{array}
		\,\right)
\end{equation*}
\begin{equation*}
R_{3}(\theta)
\; := \;
	\left(\,
		\begin{array}{ccc}
			{\color{white}-}\cos\theta & -\sin\theta & {\color{white}-}0 \\
			{\color{white}-}\sin\theta & {\color{white}-}\cos\theta & {\color{white}-}0 \\
			{\color{white}-}0 & {\color{white}-}0 & {\color{white}-}1 \\
			\end{array}
		\,\right)
\end{equation*}

          %%%%% ~~~~~~~~~~~~~~~~~~~~ %%%%%

\vskip 0.5cm
\noindent
\textbf{The generators \,$J_{n} \in \C^{3 \times 3}$\, of the Euler matrices}
\begin{equation*}
R_{n}(\theta)
\; = \;
	\exp\!\left(\;\sqrt{-1}\cdot\theta \overset{{\color{white}1}}{\cdot} J_{n}\,\right)
\; = \;
	\exp\!\left(\;\i\cdot\theta \overset{{\color{white}1}}{\cdot} J_{n}\,\right)
\end{equation*}
Alternatively, note:
\begin{equation*}
\i \cdot J_{1}
\;\; = \;\;
	\left.\dfrac{\d}{\d\,\phi}\right\vert_{\phi = 0} R_{1}(\phi)
\;\; = \;
	\left.\left(\,
		\begin{array}{ccc}
			{\color{white}-}1 & {\color{white}-}0 & {\color{white}-}0 \\
			{\color{white}-}0 & {\color{white}-}\sin\phi & {\color{white}-}\cos\phi \\
			{\color{white}-}0 & {\color{black}-}\cos\phi & {\color{white}-}\sin\phi \\
			\end{array}
		\,\right)\right\vert_{\phi = 0}
\;\; = \;
	\left(\,
		\begin{array}{ccc}
			{\color{white}-}0 & {\color{white}-}0 & {\color{white}-}0 \\
			{\color{white}-}0 & {\color{white}-}0 & {\color{white}-}1 \\
			{\color{white}-}0 & {\color{black}-}1 & {\color{white}-}0 \\
			\end{array}
		\,\right)
\end{equation*}
Multiplying both sides by \,$-\,\i = -\,\sqrt{-1}$\, gives:
\begin{equation*}
J_{1}
\;\; = \;
	\left(\!\!
		\begin{array}{ccc}
			{\color{white}-}0 & {\color{white}-}0 & {\color{white}-}0 \\
			{\color{white}-}0 & {\color{white}-}0 & {\color{black}-}\i \\
			{\color{white}-}0 & {\color{white}-}\i & {\color{white}-}0 \\
			\end{array}
		\,\right)
\end{equation*}
Similarly,
\begin{equation*}
\i \cdot J_{2}
\;\; = \;\;
	\left.\dfrac{\d}{\d\,\psi}\right\vert_{\psi = 0} R_{2}(\psi)
\;\; = \;
	\left.\left(\!\!
		\begin{array}{ccc}
			{\color{white}-}\sin\psi & {\color{white}-}0 & {\color{black}-}\cos\psi \\
			{\color{white}-}0 & {\color{white}-}0 & {\color{white}-}0 \\
			{\color{white}-}\cos\psi & {\color{white}-}0 & {\color{white}-}\sin\psi \\
			\end{array}
		\,\right)\right\vert_{\psi = 0}
\;\; = \;
	\left(\!\!
		\begin{array}{ccc}
			{\color{white}-}0 & {\color{white}-}0 & {\color{black}-}1 \\
			{\color{white}-}0 & {\color{white}-}0 & {\color{white}-}0 \\
			{\color{white}-}1 & {\color{white}-}0 & {\color{white}-}0 \\
			\end{array}
		\,\right)
\end{equation*}
Multiplying both sides by \,$-\,\i = -\,\sqrt{-1}$\, gives:
\begin{equation*}
J_{2}
\;\; = \;
	\left(\!\!
		\begin{array}{ccc}
			{\color{white}-}0 & {\color{white}-}0 & {\color{white}-}\i \\
			{\color{white}-}0 & {\color{white}-}0 & {\color{white}-}0 \\
			{\color{black}-}\i & {\color{white}-}0 & {\color{white}-}0 \\
			\end{array}
		\,\right)
\end{equation*}
Lastly,
\begin{equation*}
\i \cdot J_{3}
\;\; = \;\;
	\left.\dfrac{\d}{\d\,\theta}\right\vert_{\theta = 0} R_{3}(\theta)
\;\; = \;
	\left.\left(\!
		\begin{array}{ccc}
			{\color{white}-}\sin\theta & {\color{white}-}\cos\theta & {\color{white}-}0 \\
			{\color{black}-}\cos\theta & {\color{white}-}\sin\theta & {\color{white}-}0 \\
			{\color{white}-}0 & {\color{white}-}0 & {\color{white}-}0 \\
			\end{array}
		\,\right)\right\vert_{\psi = 0}
\;\; = \;
	\left(
		\begin{array}{ccc}
			{\color{white}-}0 & {\color{white}-}1 & {\color{white}-}0 \\
			{\color{black}-}1 & {\color{white}-}0 & {\color{white}-}0 \\
			{\color{white}-}0 & {\color{white}-}0 & {\color{white}-}0 \\
			\end{array}
		\,\right)
\end{equation*}
Multiplying both sides by \,$-\,\i = -\,\sqrt{-1}$\, gives:
\begin{equation*}
J_{3}
\;\; = \;
	\left(\!
		\begin{array}{ccc}
			{\color{white}-}0 & {\color{black}-}\i & {\color{white}-}0 \\
			{\color{white}-}\i & {\color{white}-}0 & {\color{white}-}0 \\
			{\color{white}-}0 & {\color{white}-}0 & {\color{white}-}0 \\
			\end{array}
		\,\right)
\end{equation*}

          %%%%% ~~~~~~~~~~~~~~~~~~~~ %%%%%

\section{Commutators in \,$\mathfrak{so}(3)$}

          %%%%% ~~~~~~~~~~~~~~~~~~~~ %%%%%

\begin{theorem}[Commutation relations in $J_{1}$, $J_{2}$ and $J_{3}$]
\begin{equation*}
\left[\,J_{k}\,\overset{{\color{white}1}}{,}\,J_{l}\,\right]
\;\; = \;\;
	\sqrt{-1}\;\overset{3}{\underset{m=1}{\sum}}\,\varepsilon_{klm}\cdot J_{m}
\end{equation*}
\end{theorem}

          %%%%% ~~~~~~~~~~~~~~~~~~~~ %%%%%

