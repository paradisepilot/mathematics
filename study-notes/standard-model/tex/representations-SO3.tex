
          %%%%% ~~~~~~~~~~~~~~~~~~~~ %%%%%

\chapter{Irreducible representations of $\textnormal{SO}(3)$}
\setcounter{theorem}{0}
\setcounter{equation}{0}

%\cite{vanDerVaart1996}
%\cite{Kosorok2008}

%\renewcommand{\theenumi}{\alph{enumi}}
%\renewcommand{\labelenumi}{\textnormal{(\theenumi)}$\;\;$}
\renewcommand{\theenumi}{\roman{enumi}}
\renewcommand{\labelenumi}{\textnormal{(\theenumi)}$\;\;$}

          %%%%% ~~~~~~~~~~~~~~~~~~~~ %%%%%

\section{Definition of \,$\textnormal{SO}(3)$}

          %%%%% ~~~~~~~~~~~~~~~~~~~~ %%%%%

\vskip 0.1cm
\begin{definition}[$\textnormal{O}(n)$ and $\textnormal{SO}(n)$]
\mbox{}
\vskip 0.1cm
\noindent
The \textbf{orthogonal group} is defined as follows:
\begin{equation*}
\textnormal{O}(n)
\; := \;
	\left\{\;\,
		g \overset{{\color{white}.}}{\in} \textnormal{GL}(n,\Re)
		\;\left\vert\;\,
			g^{T} \cdot g = I_{n}
			\right.
		\;\right\}
\end{equation*}
The \textbf{special orthogonal group} is defined as follows:
\begin{equation*}
\textnormal{SO}(n)
\; := \;
	\left\{\;\,
		g \overset{{\color{white}.}}{\in} \textnormal{GL}(n,\Re)
		\;\left\vert\;\,
			g^{T} \cdot g = I_{n}\,,
			\;
			\textnormal{det}(g) = 1
			\right.
		\;\right\}
\end{equation*}
\end{definition}

\begin{proposition}[Lie algebras of $\textnormal{O}(n)$ and $\textnormal{SO}(n)$]
\begin{eqnarray*}
\mathfrak{o}(n)
& = &
	\left\{\;\,
		X \overset{{\color{white}.}}{\in} \mathfrak{gl}(n,\Re)
		\;\left\vert\;\,
			X^{T} = -X
			\right.
		\;\right\}
\\
\mathfrak{so}(n)
& = &
	\left\{\;\,
		X \overset{{\color{white}.}}{\in} \mathfrak{gl}(n,\Re)
		\;\left\vert\;\,
			X^{T} = -X\,,
			\;
			\textnormal{trace}(X) = 0
			\right.
		\;\right\}
\end{eqnarray*}
\end{proposition}

          %%%%% ~~~~~~~~~~~~~~~~~~~~ %%%%%

\section{Generators of \;$\textnormal{SO}(3)$\, and \,$\mathfrak{so}(3)$}

          %%%%% ~~~~~~~~~~~~~~~~~~~~ %%%%%

\vskip 0.1cm
\noindent
\textbf{Euler matrices}
\begin{equation*}
R_{1}(\phi)
\; := \;
	\left(\,
		\begin{array}{ccc}
			{\color{white}-}1 & {\color{white}-}0 & {\color{white}-}0 \\
			{\color{white}-}0 & {\color{white}-}\cos\phi & -\sin\phi \\
			{\color{white}-}0 & {\color{white}-}\sin\phi & {\color{white}-}\cos\phi \\
			\end{array}
		\,\right)
\end{equation*}
\begin{equation*}
R_{2}(\psi)
\; := \;
	\left(\,
		\begin{array}{ccc}
			{\color{white}-}\cos\psi & {\color{white}-}0 & {\color{white}-}\sin\psi \\
			{\color{white}-}0 & {\color{white}-}1 & {\color{white}-}0 \\
			-\sin\psi & {\color{white}-}0 & {\color{white}-}\cos\psi \\
			\end{array}
		\,\right)
\end{equation*}
\begin{equation*}
R_{3}(\theta)
\; := \;
	\left(\,
		\begin{array}{ccc}
			{\color{white}-}\cos\theta & -\sin\theta & {\color{white}-}0 \\
			{\color{white}-}\sin\theta & {\color{white}-}\cos\theta & {\color{white}-}0 \\
			{\color{white}-}0 & {\color{white}-}0 & {\color{white}-}1 \\
			\end{array}
		\,\right)
\end{equation*}

          %%%%% ~~~~~~~~~~~~~~~~~~~~ %%%%%

\vskip 0.5cm
\noindent
\textbf{The generators \,$J_{n} \in \C^{3 \times 3}$\, of the Euler matrices}
\begin{equation*}
R_{n}(\theta)
\; = \;
	\exp\!\left(\;\sqrt{-1}\cdot\theta \overset{{\color{white}1}}{\cdot} J_{n}\,\right)
\; = \;
	\exp\!\left(\;\i\cdot\theta \overset{{\color{white}1}}{\cdot} J_{n}\,\right)
\end{equation*}
Alternatively, note:
\begin{equation*}
\i \cdot J_{1}
\;\; = \;\;
	\left.\dfrac{\d}{\d\,\phi}\right\vert_{\phi = 0} R_{1}(\phi)
\;\; = \;
	\left.\left(\,
		\begin{array}{ccc}
			{\color{white}-}1 & {\color{white}-}0 & {\color{white}-}0 \\
			{\color{white}-}0 & {\color{white}-}\sin\phi & {\color{white}-}\cos\phi \\
			{\color{white}-}0 & {\color{black}-}\cos\phi & {\color{white}-}\sin\phi \\
			\end{array}
		\,\right)\right\vert_{\phi = 0}
\;\; = \;
	\left(\,
		\begin{array}{ccc}
			{\color{white}-}0 & {\color{white}-}0 & {\color{white}-}0 \\
			{\color{white}-}0 & {\color{white}-}0 & {\color{white}-}1 \\
			{\color{white}-}0 & {\color{black}-}1 & {\color{white}-}0 \\
			\end{array}
		\,\right)
\end{equation*}
Multiplying both sides by \,$-\,\i = -\,\sqrt{-1}$\, gives:
\begin{equation*}
J_{1}
\;\; = \;
	\left(\!\!
		\begin{array}{ccc}
			{\color{white}-}0 & {\color{white}-}0 & {\color{white}-}0 \\
			{\color{white}-}0 & {\color{white}-}0 & {\color{black}-}\i \\
			{\color{white}-}0 & {\color{white}-}\i & {\color{white}-}0 \\
			\end{array}
		\,\right)
\end{equation*}
Similarly,
\begin{equation*}
\i \cdot J_{2}
\;\; = \;\;
	\left.\dfrac{\d}{\d\,\psi}\right\vert_{\psi = 0} R_{2}(\psi)
\;\; = \;
	\left.\left(\!\!
		\begin{array}{ccc}
			{\color{white}-}\sin\psi & {\color{white}-}0 & {\color{black}-}\cos\psi \\
			{\color{white}-}0 & {\color{white}-}0 & {\color{white}-}0 \\
			{\color{white}-}\cos\psi & {\color{white}-}0 & {\color{white}-}\sin\psi \\
			\end{array}
		\,\right)\right\vert_{\psi = 0}
\;\; = \;
	\left(\!\!
		\begin{array}{ccc}
			{\color{white}-}0 & {\color{white}-}0 & {\color{black}-}1 \\
			{\color{white}-}0 & {\color{white}-}0 & {\color{white}-}0 \\
			{\color{white}-}1 & {\color{white}-}0 & {\color{white}-}0 \\
			\end{array}
		\,\right)
\end{equation*}
Multiplying both sides by \,$-\,\i = -\,\sqrt{-1}$\, gives:
\begin{equation*}
J_{2}
\;\; = \;
	\left(\!\!
		\begin{array}{ccc}
			{\color{white}-}0 & {\color{white}-}0 & {\color{white}-}\i \\
			{\color{white}-}0 & {\color{white}-}0 & {\color{white}-}0 \\
			{\color{black}-}\i & {\color{white}-}0 & {\color{white}-}0 \\
			\end{array}
		\,\right)
\end{equation*}
Lastly,
\begin{equation*}
\i \cdot J_{3}
\;\; = \;\;
	\left.\dfrac{\d}{\d\,\theta}\right\vert_{\theta = 0} R_{3}(\theta)
\;\; = \;
	\left.\left(\!
		\begin{array}{ccc}
			{\color{white}-}\sin\theta & {\color{white}-}\cos\theta & {\color{white}-}0 \\
			{\color{black}-}\cos\theta & {\color{white}-}\sin\theta & {\color{white}-}0 \\
			{\color{white}-}0 & {\color{white}-}0 & {\color{white}-}0 \\
			\end{array}
		\,\right)\right\vert_{\psi = 0}
\;\; = \;
	\left(
		\begin{array}{ccc}
			{\color{white}-}0 & {\color{white}-}1 & {\color{white}-}0 \\
			{\color{black}-}1 & {\color{white}-}0 & {\color{white}-}0 \\
			{\color{white}-}0 & {\color{white}-}0 & {\color{white}-}0 \\
			\end{array}
		\,\right)
\end{equation*}
Multiplying both sides by \,$-\,\i = -\,\sqrt{-1}$\, gives:
\begin{equation*}
J_{3}
\;\; = \;
	\left(\!
		\begin{array}{ccc}
			{\color{white}-}0 & {\color{black}-}\i & {\color{white}-}0 \\
			{\color{white}-}\i & {\color{white}-}0 & {\color{white}-}0 \\
			{\color{white}-}0 & {\color{white}-}0 & {\color{white}-}0 \\
			\end{array}
		\,\right)
\end{equation*}

          %%%%% ~~~~~~~~~~~~~~~~~~~~ %%%%%

\section{Properties of the generators \,$J_{1}, J_{2}, J_{3} \,\in\, \mathfrak{so}(3) \otimes_{\Re} \C$}

          %%%%% ~~~~~~~~~~~~~~~~~~~~ %%%%%

\begin{proposition}
{\color{white}.}\vskip -0.5cm{\color{white}.}
\begin{enumerate}
\item
	\textbf{Commutation relations:}\;\;
	\begin{equation*}
	\left[\,J_{k}\,\overset{{\color{white}1}}{,}\,J_{l}\,\right]
	\;\; = \;\;
		\sqrt{-1}\;\overset{3}{\underset{m=1}{\sum}}\,\varepsilon_{klm}\cdot J_{m}\,,
	\quad
	\textnormal{for each \,$k, l \in \{\,1,2,3\,\}$}\,,
	\end{equation*}
	where \,$\varepsilon_{klm}$\, is the fully anti-symmetric tensor.
\item
	\textbf{Casimir operator:}\;\;
	Define
	%\,$J^{2} \, := \, (J_{1})^{2} + (J_{2})^{2} + (J_{3})^{2} \,\in\, \mathfrak{so}(3) \,\subset\, \mathfrak{so}(3) \otimes_{\Re} \C$.
	\,$J^{2} \, := \, (J_{1})^{2} + (J_{2})^{2} + (J_{3})^{2} \,\in\, \mathcal{U}\!\left(\mathfrak{so}(3) \overset{{\color{white}.}}{\otimes_{\Re}} \C\right)$.
	Then,
	\begin{equation*}
	\left[\,J^{2}\,\overset{{\color{white}1}}{,}\,J_{k}\,\right]
	\;\; = \;\;
		0\,,
	\quad
	\textnormal{for each \,$k \in \{\,1,2,3\,\}$}\,.
	\end{equation*}
	Consequently (by Schur's Lemma, Corollary 4.30, \cite{Hall2015}), 
	\,$J^{2} \in \mathcal{U}\!\left(\mathfrak{so}(3) \overset{{\color{white}.}}{\otimes_{\Re}} \C\right)$\, acts as a scalar multiple of the identity in every irreducible
	representation\footnote{Furthermore, this scalar $\lambda \in \C$ uniquely determines
	the irreducible representation.
	Look up the classification theory of irreducible finite-dimensional complex representations
	of complex semisimple Lie algebras.
	Key words: Casimir operator, universal enveloping algebra. See Chapters 9 and 10, \cite{Hall2015}.}
	of \,$\mathcal{U}\!\left(\mathfrak{so}(3) \overset{{\color{white}.}}{\otimes_{\Re}} \C\right)$;\, more precisely, for each irreducible finite-dimensional
	complex representation
	\,$\rho : \mathcal{U}\!\left(\mathfrak{so}(3) \overset{{\color{white}.}}{\otimes_{\Re}} \C\right) \longrightarrow \mathfrak{gl}(V)$,\,
	we have \,$\rho(J^{2}) = \lambda \cdot \textnormal{\textbf{1}}_{V}$,\,
	for some \,$\lambda \in \C$.
\item
	\textbf{Raising and lowering operators:}\;\;
	Define \,$J_{+}\,,\, J_{-} \in \mathfrak{so}(3) \otimes_{\Re} \C \subset \mathcal{U}\!\left(\mathfrak{so}(3) \overset{{\color{white}.}}{\otimes_{\Re}} \C\right)$\, as follows:
	\begin{equation*}
	J_{\pm} \;\; := \;\; J_{1} \, \pm \sqrt{-1}\,J_{2}.
	\end{equation*}
	Then, the following equalities (of elements of $\mathcal{U}\!\left(\mathfrak{so}(3) \overset{{\color{white}.}}{\otimes_{\Re}} \C\right)$) hold:
	\begin{enumerate}
	\item
		$\left[\,J_{3}\,,\,J_{+}\,\right] \;=\; J_{+}$\,,
		\quad
		$\left[\,J_{3}\,,\,J_{-}\,\right] \;=\; -\,J_{-}$\,,
		\quad
		$\left[\,J_{+}\,,\,J_{-}\,\right] \;=\; 2\,J_{3}$
	\item
		$J^{2}$
		\; $=$ \; $(J_{3})^{2} \,-\, J_{3} \,+\, J_{+}J_{-}$
		\; $=$ \; $(J_{3})^{2} \,+\, J_{3} \,+\, J_{-}J_{+}$
	\item
		$(J_{\pm})^{\dagger} \; = \; J_{\mp}$
	\end{enumerate}
\item
	Suppose
	\,$\rho : \mathfrak{so}(3) \otimes_{\Re} \C \longrightarrow \mathfrak{gl}(V)$\,
	is an irreducible finite-dimensional complex representation, and
	\,$v \in V \backslash\{0\}$\, is an eigenvector of \,$\rho(J_{3})$\,
	corresponding to the eigenvalue \,$\lambda \in \Re$;\, thus, \,$\rho(J_{3})(v) \,=\, \lambda\,v$.
	Then, we have:
	\begin{equation*}
	\rho(J_{3})\!\left(\,\rho(J_{+})(\overset{{\color{white}-}}{v})\,\right) \, = \; (\lambda+1)\cdot\rho(J_{+})(v)\,
	\quad\textnormal{and}\quad\;
	\rho(J_{3})\!\left(\,\rho(J_{-})(\overset{{\color{white}-}}{v})\,\right) \, = \; (\lambda-1)\cdot\rho(J_{-})(v)
	\end{equation*}
\end{enumerate}
\end{proposition}

          %%%%% ~~~~~~~~~~~~~~~~~~~~ %%%%%

