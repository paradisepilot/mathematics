
          %%%%% ~~~~~~~~~~~~~~~~~~~~ %%%%%

\chapter{Irreducible representations of semisimple complex Lie algebras via Casimir operators in universal enveloping algebras}
\setcounter{theorem}{0}
\setcounter{equation}{0}

%\cite{vanDerVaart1996}
%\cite{Kosorok2008}

%\renewcommand{\theenumi}{\alph{enumi}}
%\renewcommand{\labelenumi}{\textnormal{(\theenumi)}$\;\;$}
\renewcommand{\theenumi}{\roman{enumi}}
\renewcommand{\labelenumi}{\textnormal{(\theenumi)}$\;\;$}

          %%%%% ~~~~~~~~~~~~~~~~~~~~ %%%%%

\begin{enumerate}
\item
	An \textit{algebra} $\mathcal{A}$ over a field $\F$ is a vector space over $\F$
	equipped with a bilinear map $* : \mathcal{A} \times \mathcal{A} \longrightarrow \mathcal{A}$, i.e.,
	\begin{equation*}
	\begin{array}{c}
		(a x + y) * z = a (x * z) + y * z\,,
		\\
		\underset{{\color{white}-}}{\overset{{\color{white}-}}{\textnormal{and}}}
		\\
		x * (a y + z) = a(x * y) + x * z\,,
	\end{array}
	\quad
	\textnormal{for each \,$a, b \in \F$,\, and \,$x, y, z \in \mathcal{A}$}.
	\end{equation*}
	The algebra $\mathcal{A}$ is said to be \textit{associative} if
	\begin{equation*}
	x * (y * z) \; = \; (x * y) * z\,,
	\quad
	\textnormal{for each \,$x, y, z \in \mathcal{A}$}
	\end{equation*}
	The algebra $\mathcal{A}$ is said to be \textit{unital} if
	there exists a multiplicative unit $1 \in \mathcal{A}$, i.e.,
	\begin{equation*}
	1 * x \; = \; x \; = \; x * 1\,,
	\quad
	\textnormal{for each \,$x \in \mathcal{A}$}
	\end{equation*}
\item
	Every associative algebra $\mathcal{A} \cong (\,V,*\,)$
	canonically induces a Lie algebra structure on its underlying vector space $V$ via:
	\begin{equation*}
	\left[\,x\,,\,y\,\right] \; := \; x * y - y * x\,,
	\quad
	\textnormal{for \,$x, y \in V$}
	\end{equation*}
	We denote this canonically induced Lie algebra as $\mathcal{A}^{[,]}$.
	\vskip 0.2cm
	This fact begs the question:
	Does the reverse assoication exist? More precisely, given a Lie algebra $\mathfrak{g}$,
	does there exist an associative algebra $\mathcal{A}$ such that $\mathcal{A}^{[,]} \cong \mathfrak{g}$?
	And, if so, is such an $\mathcal{A}$ unique?
	\vskip 0.2cm
	The answers are Yes and Yes.
	This uniquely determined associative algebra is called the
	\textbf{universal enveloping algebra} of $\mathfrak{g}$,
	and is denoted by $\mathcal{U}(\mathfrak{g})$.
	\vskip 0.1cm
	The universal enveloping algebra can be characterized by the following universal property:
	\begin{center}
	\begin{minipage}{5.25in}
	A \textit{universal enveloping algebra}
	of a Lie algebra $\mathfrak{g}$ over a field $\F$
	is a unital associative algebra $\mathcal{U}$ over $\F$ together with a Lie algebra homomorphism
	$\iota : \mathfrak{g} \longrightarrow \mathcal{U}^{[,]}$ such that,
	for each Lie algebra homomorphism $\phi : \mathfrak{g} \longrightarrow \mathcal{A}^{[,]}$
	(where $\mathcal{A}$ is a unital associative algebra),
	there exists a unique associative algebra homomorphism
	$\phi^{\sharp} : \mathcal{U} \longrightarrow \mathcal{A}$
	whose induced Lie algebra homomorphism
	$(\phi^{\sharp})^{\flat} : \mathcal{U}^{[,]} \longrightarrow \mathcal{A}^{[,]}$
	satisfies:
	$\phi \,=\, (\phi^{\sharp})^{\flat} \,\circ\, \iota$.
	\end{minipage}
	\end{center}
	The universal enveloping algebra $\mathcal{U}(\mathfrak{g})$ can be explicitly constructed as follows:
	\begin{equation*}
	\mathcal{U}(\mathfrak{g}) \;\; \cong \; \left. \overset{{\color{white}.}}{\otimes(\mathfrak{g})} \right\slash \mathcal{I}\,,
	\end{equation*}
	where $\otimes(\mathfrak{g})$ is the tensor algebra of $\mathfrak{g}$, and
	$\mathcal{I} \subset \otimes(\mathfrak{g})$ is the two-sided ideal of $\otimes(\mathfrak{g})$
	generated by elements of the form:
	\begin{equation*}
	x \otimes y \,-\, y \otimes x \,-\, \left[\,x,y\,\right]\,,
	\quad
	\textnormal{for \,$x, y \in \mathfrak{g}$}.
	\end{equation*}
\item
	The representations of a Lie algebra $\mathfrak{g}$ over $\F$ and
	those of its universal enveloping algebra $\mathcal{U}(\mathfrak{g})$
	are in one-to-one correspondence.
	\vskip 0.1cm
	\proof Let $\rho : \mathfrak{g} \longrightarrow \mathfrak{gl}(V)$ be a representation,
	i.e., $\rho$ is a Lie algebra homomorphism from $\mathfrak{g}$ into $\mathfrak{gl}(V) := \textnormal{End}(V)^{[,]}$,
	for some vector space $V$ over $\F$.
	By the universal property of
	$\iota : \mathfrak{g} \longrightarrow \mathcal{U}(\mathfrak{g})^{[,]}$,
	there exists a unique associative algebra homomorphism
	$\rho^{\sharp} : \mathcal{U}(\mathfrak{g}) \longrightarrow \textnormal{End}(V)$
	such that its induced Lie algebra homomorphism
	$(\rho^{\sharp})^{\flat} : \mathcal{U}(\mathfrak{g})^{[,]} \longrightarrow \textnormal{End}(V)^{[,]} =: \mathfrak{gl}(V)$
	satisfies:
	$\rho \,=\, (\rho^{\sharp})^{\flat} \,\circ\, \iota$.
	The association $\rho \longmapsto \rho^{\sharp}$ defines a map
	$\Theta$ from the collection of the representations of $\mathfrak{g}$
	to that of the representations of $\mathcal{U}(\mathfrak{g})$.
	Injectivity of $\Theta$ follows from:
	$\Theta(\rho_{1}) = \Theta(\rho_{2})$
	$\Longleftrightarrow$
	$\rho_{1}^{\sharp} = \rho_{2}^{\sharp}$
	$\Longrightarrow$
	$\rho_{1} \,=\, (\rho_{1}^{\sharp})^{\flat} \circ \iota \,=\, (\rho_{2}^{\sharp})^{\flat} \circ \iota \,=\, \rho_{2}$.
	It remains to establish the surjectivity of $\Theta$.
	To this end, let $\Psi : \mathcal{U}(\mathfrak{g}) \longrightarrow \textnormal{End}(V)$ be an arbitrary representation.
	Define $\psi \, := \, \Psi^{\flat} \,\circ\, \iota : \mathfrak{g} \longrightarrow \textnormal{End}(V)^{[,]} =: \mathfrak{gl}(V)$.
	Then, $\psi$ is a representation of $\mathfrak{g}$.
	By the universal property of
	$\iota : \mathfrak{g} \longrightarrow \mathcal{U}(\mathfrak{g})^{[,]}$,
	there exists a unique associative algebra homomorphism
	$\psi^{\sharp} : \mathcal{U}(\mathfrak{g}) \longrightarrow \textnormal{End}(V)$
	such that its induced Lie algebra homomorphism
	$(\psi^{\sharp})^{\flat} : \mathcal{U}(\mathfrak{g})^{[,]} \longrightarrow \textnormal{End}(V)^{[,]} =: \mathfrak{gl}(V)$
	satisfies:
	$\psi \,=\, (\psi^{\sharp})^{\flat} \,\circ\, \iota$.
	The uniqueness of the associative algebra homomorphism $\psi^{\sharp}$ therefore implies that
	$\Psi = \psi^{\sharp} =: \Theta(\psi)$.
	This proves the surjectivity of $\Theta$.
	\qed
\item
	The \textbf{Casimir operators} of a finite-dimensional semisimple complex Lie algebra $\mathfrak{g}$
	form a distinguished basis for the centre $\mathcal{Z}(\mathcal{U}(\mathfrak{g}))$ of the universal enveloping algebra
	$\mathcal{U}(\mathfrak{g})$.
	\vskip 0.1cm
	By Schur's Lemma, each Casimir operator (being an element of $\mathcal{Z}(\mathcal{U}(\mathfrak{g})))$ acts
	as a scalar multiple of the identity map on the representation space of any finite-dimensional irreducible representation.
	\vskip 0.1cm
	The collection of the aforementioned scalar multiples corresponding to the Casimir operators uniquely determines
	the irreducible representation, in the following sense:
	\begin{theorem}
	{\color{white}.}\vskip -0.1cm
	\noindent Two finite-dimensional irreducible representations
	$\rho_{1} : \mathfrak{g} \longrightarrow \mathfrak{gl}(V_{1})$
	and
	$\rho_{2} : \mathfrak{g} \longrightarrow \mathfrak{gl}(V_{2})$	
	of a finite-dimensional semisimple complex Lie algebra $\mathfrak{g}$
	are equivalent if and only if the eigenvalues of $\rho_{1}(C)$ and $\rho_{2}(C)$ are equal,
	for each Casimir operator $C \in \mathcal{Z}(\mathcal{U}(\mathfrak{g}))$.
	\end{theorem}
	\proof ??? \qed
\end{enumerate}

          %%%%% ~~~~~~~~~~~~~~~~~~~~ %%%%%

