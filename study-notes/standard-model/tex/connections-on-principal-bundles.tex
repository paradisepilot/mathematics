
          %%%%% ~~~~~~~~~~~~~~~~~~~~ %%%%%

\chapter{Connections on principal bundles}
\setcounter{theorem}{0}
\setcounter{equation}{0}

%\cite{vanDerVaart1996}
%\cite{Kosorok2008}

%\renewcommand{\theenumi}{\alph{enumi}}
%\renewcommand{\labelenumi}{\textnormal{(\theenumi)}$\;\;$}
\renewcommand{\theenumi}{\roman{enumi}}
\renewcommand{\labelenumi}{\textnormal{(\theenumi)}$\;\;$}

          %%%%% ~~~~~~~~~~~~~~~~~~~~ %%%%%

A very short summary of how fibre bundles enter quantum field theory:
\begin{itemize}
\item
	The states of a quantum system (e.g., a single particle, or an ensemble of particles, etc.)
	are unit vectors (alternative, one-dimensional subspaces) in a suitable (possibly infinite-dimensional) complex Hilbert space.
\item
	If the quantum system in question possesses a certain symmetry,
	then the corresponding group (corresponding to the symmetry) will act on its state space (a complex Hilbert space).
	Certain particles possess internal symmetries.
	(Arguably, elementary particles are even determined/defined by the internal symmetries they possess or exhibit.)
	These internal symmetries are often ``continuous symmetries''. 
	This is where the theory of representations of Lie groups enters quantum field theory.
\item
	When we expand our scope of consideration from the local (or infinitesimal) to the global,
	it becomes clear that the class of mathematical objects suitable for modelling or representing quantum systems
	(e.g., elementary \textbf{\color{red}particles})
	at the pre-quantization stage are smooth \textbf{\color{red}sections of complex vector bundles over spacetime}
	(i.e., spacetime is the base space of such vector bundles),
	where the complex vector bundles admit fibre-preserving actions by Lie groups.
	Such sections of complex vector bundles are examples of what are called ``fields'' in the physics literature.
\item
	The Lagrangian describing the \textbf{dynamics} of a quantum system is generally expected to involve the ``derivatives''
	of the field (section of a complex vector bundle) that represents the quantum system.
	This is where the notion of \textbf{covariant derivatives} of sections of complex vector bundles
	-- hence the notion of \textbf{connections} on vector bundles -- enters quantum field theory.
\item
	The frame bundle of a vector bundle is a principal bundle.
	The connection on a vector bundle induces a connection on its frame bundle.
	The theory of connections on principal bundles is ``cleaner'' than that of connections on vector bundles;
	more precisely, the former can be given in global terms, and does not require the use of local coordinate charts.
	This is where the notion of \textbf{\color{red}connections on principal bundles} enters quantum field theory;
	connections on principal bundles represent \textbf{\color{red}force fields}.
\item
	We end by returning to a remark on Lagrangian densities:
	In  the Standard Model of Particle Physics (which is a quantum field theory),
	the ``interactions'' between particles (sections of complex vector bundles)
	and force fields (connections on principal bundles)
	appear in summands (of Lagrangian densities)
	through the covariant derivatives of the sections of complex vector bundles
	(``particles'') with respect to the connections (``force fields'') on principal bundles.
\end{itemize}

          %%%%% ~~~~~~~~~~~~~~~~~~~~ %%%%%

\section{Principal bundles}

\begin{definition}[Principal bundles]
\mbox{}
\vskip 0.15cm
\noindent
A \textbf{principal fibre bundle} (or \textbf{principal bundle}) is a fibre bundle
\,$\pi : P \longrightarrow M$\,
which satisfies each of the following conditions:
\begin{itemize}
\item
	The general fibre of the fibre bundle
	\,$\pi : P \longrightarrow M$\,
	is a Lie group \,$G$.\,
	The Lie group \,$G$\, is called the \textbf{structure group} of the principal bundle.
	% \begin{center}
	% \begin{tikzcd}
	% G
	% 	\arrow[rr, hook, thick]
	% &&
	% P
	% 	\arrow[dd, thick, two heads, "\pi"]
	% \\ \\
	% &&
	% M
	% \end{tikzcd}
	% \end{center}
	\begin{center}
	\begin{tikzcd}
	G
		\arrow[r, hook]
	&
	P
		\arrow[dd, two heads, "\pi"]
	\\ \\
	&
	M
	\end{tikzcd}
	\end{center}
\item
	There is a smooth action
	\,$P \times G \longrightarrow P$\,
	on the right.
\item
	The aforemention right action on \,$P$\, by \,$G$\,
	preserves each fibre of \,$\pi$\,
	and is simply transitive on each fibre of \,$\pi$,\,
	i.e., the action restricts to each fibre:
	\begin{center}
	\begin{tikzcd}
	P_{x} \times G
		\arrow[r]
	&
	P_{x}
	\end{tikzcd}
	\end{center}
	and, for each \,$x \in M$\, and each \,$p \in P_{x} = \pi^{-1}(x)$\,,
	the orbit map
	\begin{equation*}
	\begin{array}{ccc}
	G & \longrightarrow & P_{x}
	\\
	g & \longmapsto & p \cdot g
	\end{array}
	\end{equation*}
	is a bijection.
\item
	There exists a bundle atlas of \textbf{$G$-equivariant bundle charts}
	\,$\phi_{i} : \pi^{-1}(U_{i}) \longrightarrow U_{i} \times G$,\, i.e.,
	\renewcommand{\labelitemii}{$\circ$}
	\begin{itemize}
	\item
		$M \; = \; \underset{i}{\bigcup}\;U_{i}$\,,
	\item
		for each \,$p \in \pi^{-1}(U_{i})$\, and each \,$g \in G$,\, we have:
		\begin{equation*}
		\phi_{i}(\,p \cdot g\,) \;=\; \phi_{i}(p) \,\cdot\, g
		\end{equation*}
	\item
		for each \,$i$,\, the right action induced on \,$U_{i} \times G$\, by \,$\phi_{i}$\, is given by:
		\begin{equation*}
		\left(\, x \,\overset{{\color{white}1}}{,}\, h \,\right) \cdot\, g
		\; = \;
			\left(\, x \,\overset{{\color{white}1}}{,}\, h \cdot g \,\right),
		\end{equation*}
		for each \,$x \in M$\, and for each \,$g, h \in G$.\,
	\end{itemize}
\end{itemize}
\end{definition}

          %%%%% ~~~~~~~~~~~~~~~~~~~~ %%%%%

