
          %%%%% ~~~~~~~~~~~~~~~~~~~~ %%%%%

\chapter{Connections on principal bundles}
\setcounter{theorem}{0}
\setcounter{equation}{0}

%\cite{vanDerVaart1996}
%\cite{Kosorok2008}

%\renewcommand{\theenumi}{\alph{enumi}}
%\renewcommand{\labelenumi}{\textnormal{(\theenumi)}$\;\;$}
\renewcommand{\theenumi}{\roman{enumi}}
\renewcommand{\labelenumi}{\textnormal{(\theenumi)}$\;\;$}

          %%%%% ~~~~~~~~~~~~~~~~~~~~ %%%%%

A very short summary of how fibre bundles enter quantum field theory:
\begin{itemize}
\item
	The states of a quantum system (e.g., a single particle, or an ensemble of particles, etc.)
	are unit vectors (alternative, one-dimensional subspaces) in a suitable (possibly infinite-dimensional) complex Hilbert space.
\item
	If the quantum system in question possesses a certain symmetry,
	then the corresponding group (corresponding to the symmetry) will act on its state space (a complex Hilbert space).
	Certain particles possess internal symmetries.
	(Arguably, elementary particles are even determined/defined by the internal symmetries they possess or exhibit.)
	These internal symmetries are often ``continuous symmetries''. 
	This is where the theory of representations of Lie groups enters quantum field theory.
\item
	When we expand our scope of consideration from the local (or infinitesimal) to the global,
	it becomes clear that the class of mathematical objects suitable for modelling or representing quantum systems
	(e.g., elementary \textbf{\color{red}particles})
	at the pre-quantization stage are smooth \textbf{\color{red}sections of complex vector bundles over spacetime}
	(i.e., spacetime is the base space of such vector bundles),
	where the complex vector bundles admit fibre-preserving actions by Lie groups.
	Such sections of complex vector bundles are examples of what are called ``fields'' in the physics literature.
\item
	The Lagrangian describing the \textbf{dynamics} of a quantum system is generally expected to involve the ``derivatives''
	of the field (section of a complex vector bundle) that represents the quantum system.
	This is where the notion of \textbf{covariant derivatives} of sections of complex vector bundles
	-- hence the notion of \textbf{connections} on vector bundles -- enters quantum field theory.
\item
	The frame bundle of a vector bundle is a principal bundle.
	The connection on a vector bundle induces a connection on its frame bundle.
	The theory of connections on principal bundles is ``cleaner'' than that of connections on vector bundles;
	more precisely, the former can be given in global terms, and does not require the use of local coordinate charts.
	This is where the notion of \textbf{\color{red}connections on principal bundles} enters quantum field theory;
	connections on principal bundles represent \textbf{\color{red}force fields}.
\item
	We end by returning to a remark on Lagrangian densities:
	In  the Standard Model of Particle Physics (which is a quantum field theory),
	the ``interactions'' between particles (sections of complex vector bundles)
	and force fields (connections on principal bundles)
	appear in summands (of Lagrangian densities)
	through the covariant derivatives of the sections of complex vector bundles
	(``particles'') with respect to the connections (``force fields'') on principal bundles.
\end{itemize}

          %%%%% ~~~~~~~~~~~~~~~~~~~~ %%%%%

\clearpage
\section{Gauge invariance as invariance with respect to fibrewise recoordinatization by left translation}

          %%%%% ~~~~~~~~~~~~~~~~~~~~ %%%%%

\begin{remark}
\mbox{}
\vskip 0.1cm
\noindent
One of the axioms of quantum field theory is that the physics of a (relativistic) quantum
system is encoded in the \textbf{Lagrangian density} of the system.
The Lagrangian density is a $\Re$-valued function defined on the spacetime manifold, and it depends on
``fields'' and their first ``derivatives'' with respect to spacetime coordinates.
Fields that correspond to matter particles are sections of vector bundles associated with principal bundles.
Fields that correspond to forces are connections on principal bundles.
The structure groups of the aforementioned principal bundles correspond to the ``internal symmetries''
(i.e., non-spacetime symmetries) exhibited by the species of particles in question.
\vskip 0.1cm
The principle of \textbf{gauge invariance} is a fundamental principle of quantum field theory
which could be intuitively stated as follows:
physical phenomena (involving relativistic quantum systems) should not change under
an arbitrary smoothly varying local ``re-coordinatizations'' of the space of the internal symmetry.
More precisely, the principle of gauge invariance states that
the \textbf{Lagrangian density of a quantum system must be invariant with respect to gauge transformations}.
\end{remark}


\vskip 0.5cm
\begin{definition}[Global gauge transformation]
\mbox{}
\vskip -0.05cm
\noindent
Suppose
\,$G \xhookrightarrow{{\color{white}222}} P \xrightarrow{{\color{white}2}\pi{\color{white}2}} M$\,
is a principal bundle.
A \textbf{(global) gauge transformation} of \,$P$\, is a bundle automorphism
\,$f : P \xrightarrow{{\color{white}222}} P$\, that preserves fibres and is $G$-equivariant, i.e.,
\,$f$\, satisfies:
\begin{itemize}
\item
	$\pi \,\circ\, f \,=\, \pi$\,, and
\item
	$f(\,p \cdot g\,) \,=\, f(p) \cdot g$\,,
	for each \,$p \in P$\, and each \,$g \in G$.\,
\end{itemize}
\end{definition}


\vskip 0.5cm
\begin{proposition}[Proposition 5.3.3, p.266, \cite{Hamilton2017}]
\mbox{}
\vskip -0.05cm
\noindent
Suppose
\,$G \xhookrightarrow{{\color{white}222}} P \xrightarrow{{\color{white}2}\pi{\color{white}2}} M$\,
is a principal bundle.
Let \,$\mathscr{G}(P)$\, denote the group of gauge transformations of \,$P$.\,
Let \,$\mathscr{C}^{\infty}(P,G)^{G}$\, denote the group of $G$-valued functions
defined on $P$ defined as follows:
\begin{equation*}
\mathscr{C}^{\infty}(P,G)^{G}
\;\; := \;\;
	\left\{\;
		\left.
		\overset{{\color{white}1}}{\sigma} \in \mathscr{C}^{\infty}(P,G)
		\;\;\right\vert\;\,
		\sigma(\,p \cdot g\,) \,=\, g^{-1} \cdot \sigma(p) \cdot g
		\;\right\}
\end{equation*}
Define the map
\begin{equation*}
\begin{array}{ccl}
\mathscr{G}(P) & \xrightarrow{{\color{white}2222}} & \mathscr{C}^{\infty}(P,G)^{G}
\\
f & \xmapsto{{\color{white}2222}} & \sigma_{f}\,,
\end{array}
\end{equation*}
where \,$\sigma_{f} \in \mathscr{C}^{\infty}(P,G)^{G}$\, is defined (implicitly) by:
\begin{equation*}
f(p)
\;\; = \;\;
	p \cdot \sigma_{f}(p)\,,
\quad
\textnormal{for each \,$p \in P$}.
\end{equation*}
Then, the map
\,$\mathscr{G}(P) \xrightarrow{{\color{white}222}} \mathscr{C}^{\infty}(P,G)^{G}$\,
defined above is well-defined and is an isomorphism of groups.
\end{proposition}


\vskip 0.5cm
\begin{remark}
\mbox{}
\vskip 0.1cm
\noindent
Suppose
\,$G \xhookrightarrow{{\color{white}222}} P \xrightarrow{{\color{white}2}\pi{\color{white}2}} M$\,
is a principal bundle.
For each \,$p \in P$,\, the following map
\begin{equation*}
\begin{array}{ccl}
G
&
\xhookrightarrow{{\color{white}22}\iota_{p}{\color{white}22}}
&
P_{\pi(p)} \,:=\, \pi^{-1}\!\left(\,\overset{{\color{white}.}}{\pi(p)}\,\right) \,\subset\, P
\\
g
&
\xmapsto{{\color{white}222222}}
&
{\color{white}.}p \cdot g
\end{array}
\end{equation*}
maps the structure group \,$G$\, diffeomorphically onto the fibre
\,$P_{\pi(p)} = \pi^{-1}(\,\pi(p)\,)$\,
of
\,$P \xrightarrow{{\color{white}2}\pi{\color{white}2}} M$\,
that contains \,$p \in P$,\,
thereby transferring the Lie group structure of \,$G$\, to \,$P_{\pi(p)}$.\,
Now, let \,$f : P \xrightarrow{{\color{white}222}} P$\, be a gauge transformation,
i.e., \,$f \in \mathscr{G}(P)$.\,
We denote by \,$G_{p}$\, the Lie group (ismormphic to $G$) obtained by transferring the Lie group structure
of \,$G$\, to \,$P_{\pi(p)}$\, via
\,$G \xhookrightarrow{{\color{white}22}\iota_{p}{\color{white}22}}P_{\pi(p)}$.\,
Similarly, we denote by \,$G_{f(p)}$\, the Lie group (ismormphic to $G$)
obtained by transferring the Lie group structure of \,$G$\, to \,$P_{\pi(p)}$\, via
\,$G \xhookrightarrow{{\color{white}22}\iota_{f(p)}{\color{white}22}}P_{\pi(p)}$.\,
Note that we then have (by restricting $f$ to $P_{\pi(p)}$):
\begin{equation*}
\begin{array}{ccl}
G_{p}
&
\xrightarrow{{\color{white}22.}f{\color{white}.22}}
&
G_{f(p)}
\\
p \cdot g
&
\xmapsto{{\color{white}222.222}}
&
{\color{white}.} f(\,p \cdot g\,) \,=\, f(p) \cdot g \,=\, p \cdot \sigma_{f}(p) \cdot g\,,
\end{array}
\end{equation*}
for each \,$g \in G$.\,
We therefore see that, for each fixed \,$p \in P$,\,
the restriction of \,$f$\, to \,$P_{\pi(p)} = \pi^{-1}(\,\pi(p)\,)$\,
induces the following map on \,$G$:\,
\begin{equation*}
\begin{array}{ccl}
G
&
\xrightarrow{{\color{white}22.}\iota_{p}^{-1}\,\circ\,f\,\circ\;\iota_{p}{\color{white}.22}}
&
G
\\
g
&
\xmapsto{{\color{white}2222888448882222}}
&
{\color{white}.} \sigma_{f}(p) \cdot g\,,
\end{array}
\end{equation*}
for each \,$g \in G$.\,
Note also:
\begin{center}
\begin{tikzcd}
g
	\arrow[dd, mapsto]
	\arrow[rrrrrr, mapsto, bend left = 30]
&
G
	\arrow[rrrr, thick, "\iota_{p \cdot h}^{-1}\;\circ\;f\;\circ\;\iota_{p \cdot h}"]
	\arrow[dd, thick, "\Phi_{h}\;", swap]
&&&&
G
&
\!\!\!\!\!\!\!\! h^{-1} \cdot \sigma_{f}(p) \cdot h \cdot g
\\ \\
h \cdot g
	\arrow[rrrrrr, mapsto, bend right = 30]
&
G
	\arrow[rrrr, thick,"\iota_{p}^{-1}\;\circ\;f\;\circ\;\iota_{p}", swap]
&&&&
G
	\arrow[uu, thick,, "\;\Phi_{h}^{-1}", swap]
&
\!\!\!\!\!\! \sigma_{f}(p) \cdot h \cdot g
	\arrow[uu, mapsto]
\end{tikzcd}
\end{center}
\end{remark}


\vskip 0.5cm
\begin{remark}
\mbox{}
\vskip 0.05cm
\noindent
The following Lemma shows that, given any group $G$ and an arbitrary element $h \in G$,
we can obtain a new group $G_{(h)}$ that is isomorphic to $G$ via {\color{red}left} multiplication by $h$.
$G_{(h)}$ can be regarded as a sort of {\color{red}``re-coordinatization''} of $G$,
by ``shifting'' the identity element from $\mathds{1}_{G}$ to $\mathds{1}_{G_{(g)}} = h$.
This observation will allow us to regard a \textbf{global gauge transformation}
on a principal bundle
\,$G \xhookrightarrow{{\color{white}222}} P \xrightarrow{{\color{white}2}\pi{\color{white}2}} M$\,
as a smoothly varying family of fibre-wise re-coordinatizations.
The principle of \textbf{\color{red}gauge invariance} of physical phenomena can thus be regarded as
invariance of physical phenomena with respect to smoothly varying re-coordinatizations
of the internal symmetry group $G$ (over each point $x \in M$) as the base point \,$p \in M$\, varies.
\end{remark}


\vskip 0.5cm
\begin{lemma}[Each left translate of a group is itself a group]
\mbox{}
\vskip 0.05cm
\noindent
Suppose \,$G$\, is a group and \,$\mathds{1}_{G}$\, is the identity element of \,$G$.\,
We denote the multiplication in \,$G$\, with a dot (i.e., $g_{1} \cdot g_{2}$, for $g_{1}, g_{2} \in G$). 
For each \,$h \in G$,\, define:
\begin{equation*}
\begin{array}{cccl}
\star_{h} \;\; : & G \times G & \xrightarrow{{\color{white}22222}} & G
\\
& (g_{1},g_{2}) & \xmapsto{{\color{white}22222}} & h \cdot (h^{-1} \cdot g_{1}) \cdot (h^{-1} \cdot g_{2}) \,=\, g_{1} \cdot h^{-1} \cdot g_{2}
\end{array}
\end{equation*}
Then,
\begin{enumerate}
\item
	\vskip -0.1cm
	$G_{(h)} \; := \; \left(\,\overset{{\color{white}.}}{G}\,,\,\star_{h}\,\right)$\,
	forms a group, whose identity element is:
	\,$\mathds{1}_{G_{(h)}} = \, h$,\, and
\item
	the map
	\,$\Phi_{h} : G \xrightarrow{{\color{white}222}} G_{(h)} \,:\, g \xmapsto{{\color{white}222}} h \cdot g $\,
	is an isomorphism of groups.
\end{enumerate}
\end{lemma}
\proof
\begin{enumerate}
\item
	First, we show that \,$\star_{h}$\, is associative. To this end, observe:
	\begin{eqnarray*}
	g_{1} \,\star_{h}\, (\,g_{2} \,\star_{h}\, g_{3}\,)
	& = &
		g_{1} \,\star_{h}\, (\,g_{2} \cdot h^{-1} \cdot g_{3}\,)
	\;\; = \;\;
		g_{1} \cdot h^{-1} \cdot (\,g_{2} \cdot h^{-1} \cdot g_{3}\,)
	\\
	& = &
		(\,g_{1} \cdot h^{-1} \cdot g_{2}\,) \cdot h^{-1} \cdot g_{3}
	\;\; = \;\;
		(\,g_{1} \,\star_{h}\, g_{2}\,) \,\star_{h}\, g_{3}\,,
	\end{eqnarray*}
	as required.
	\vskip 0.2cm
	\noindent
	Next, we show that the identity element of \,$G_{(h)}$\, is indeed
	\,$\mathds{1}_{G_{(h)}} \,=\, h$\,:
	\begin{eqnarray*}
	g \,\star_{h}\, h
	& = &
		g \cdot h^{-1} \cdot h
	\;\; = \;\;
		g\,,
		\quad
		\textnormal{and}
	\\
	h \,\star_{h}\, g
	& = &
		h \cdot h^{-1} \cdot g
	\;\; = \;\;
		g\,,
	\end{eqnarray*}
	for each \,$g \in G_{(h)}$.\,
	\vskip 0.2cm
	\noindent
	Lastly, we show that, for each \,$g \in G_{(h)}$,\, its inverse in \,$G_{(h)}$\, is
	\begin{equation*}
	g_{1}
	\;\; := \;\;
		\Phi_{h}\!\left[\,\left(\,\overset{{\color{white}.}}{\Phi_{h}^{-1}(g)}\,\right)^{-1}\,\right]
	\;\; = \;\;
		\Phi_{h}\!\left[\,\left(\,\overset{{\color{white}.}}{h^{-1} \cdot g}\,\right)^{-1}\,\right]
	\;\; = \;\;
		\Phi_{h}\!\left[\,g^{-1} \cdot h\,\right]
	\;\; = \;\;
		h \cdot g^{-1} \cdot h
	\end{equation*}
	Indeed,
	\begin{eqnarray*}
	g \,\star_{h}\, g_{1}
	& = &
		g \cdot h^{-1} \cdot (\,h \cdot g^{-1} \cdot h\,)
	\;\; = \;\;
		h
	\;\; = \;\;
		\mathds{1}_{G_{(h)}}\,,
	\\
	g_{1} \,\star_{h}\, g
	& = &
		(\,h \cdot g^{-1} \cdot h\,) \cdot h^{-1} \cdot g
	\;\; = \;\;
		h
	\;\; = \;\;
		\mathds{1}_{G_{(h)}}
	\end{eqnarray*}
	This completes the proof of (i).
\item
	It is immediate that \,$\Phi_{h}$\, is a bijection.
	Next, note that \,$\Phi_{h}$\, is indeed a group homomorphism:
	\begin{eqnarray*}
	\Phi_{h}\!\left(\,
		g_{1} \overset{{\color{white}1}}{\cdot} g_{2}
		\,\right)
	& = &
		h \cdot g_{1} \cdot g_{2}
	\;\; = \;\;
		\left(\,h \cdot g_{1}\,\right)
		\cdot h^{-1} \cdot
		\left(\,h \cdot g_{2}\,\right)
	\;\; = \;\;
		\Phi_{h}\!\left(\,g_{1}\,\right)
		\,\star_{h}\,
		\Phi_{h}\!\left(\,g_{2}\,\right)
	\end{eqnarray*}
	This completes the proof that \,$\Phi_{h}$\, is a group isomorphism.
	\qed
\end{enumerate}

\noindent
\begin{remark}
\mbox{}
\vskip 0.05cm
\noindent
Note that \,$\star_{h}$\, can be given as a series of two compositions as indicated
in the following commutative diagram:
\begin{center}
\begin{tikzcd}
G_{(h)} \times G_{(h)}
	\arrow[rr, thick, "\star_{h}"]
	\arrow[dd, thick, "\Phi_{h}^{-1} \,\times\, \Phi_{h}^{-1}\;", swap]
&&
G_{(h)}
\\ \\
G \times G
	\arrow[rr, thick]
&&
G
	\arrow[uu, thick,, "\;\Phi_{h}", swap]
\end{tikzcd}
\end{center}
\end{remark}

          %%%%% ~~~~~~~~~~~~~~~~~~~~ %%%%%

\vskip 0.5cm
\section{Ehresmann connections on fibre bundles and parallel transport via horizontal lifts}

          %%%%% ~~~~~~~~~~~~~~~~~~~~ %%%%%

\begin{lemma}[Vertical tangent bundle of a fibre bundle]
\mbox{}
\vskip -0.05cm
\noindent
Suppose that
\,$F \xhookrightarrow{{\color{white}222}} P \xrightarrow{{\color{white}2}\pi{\color{white}2}} M$\,
is a fibre bundle, and \,$\mathcal{V}(P)$\, is the following disjoint union:
\begin{equation*}
\mathcal{V}(P)
\;\; := \;
	\underset{p\,\in\,P}{\bigsqcup}\,\ker(\;\d\pi_{p}\,)\,,
\end{equation*}
where, for each \,$p \in P$,\, the map
\,$\d\pi_{p} : \mathcal{T}_{p}P \xrightarrow{{\color{white}222}} \mathcal{T}_{\pi(p)}M$\,
is the differential of
\,$P \xrightarrow{{\color{white}2}\pi{\color{white}2}} M$\,
at \,$p \in P$.\,
Then, \,$\mathcal{V}(P)$\, is a subbundle of the tangent bundle \,$\mathcal{T}(P)$\, of \,$P$.\,
$\mathcal{V}(P)$\, is called the \textbf{vertical tangent bundle} of
the fibre bundle
\,$F \xhookrightarrow{{\color{white}222}} P \xrightarrow{{\color{white}2}\pi{\color{white}2}} M$.\,
\end{lemma}

\vskip 0.5cm
\begin{definition}[Ehresmann connection of a fibre bundle]
\mbox{}
\vskip -0.05cm
\noindent
An \,\textbf{Ehresmann connection}\, on a fibre bundle
\,$F \xhookrightarrow{{\color{white}222}} P \xrightarrow{{\color{white}2}\pi{\color{white}2}} M$\,
is a subbundle \,$\mathcal{H}$\, of the tangent bundle \,$\mathcal{T}(P)$\, of the total space \,$P$\,
such that
\,$\mathcal{T}(P) \,=\, \mathcal{H} \oplus \mathcal{V}(P)$.\,
\end{definition}

\vskip 0.5cm
\begin{lemma}[Unique maximal horizontal lift induced by Ehresmann connection]
\mbox{}
\vskip -0.05cm
\noindent
Suppose that
\,$F \xhookrightarrow{{\color{white}222}} P \xrightarrow{{\color{white}2}\pi{\color{white}2}} M$\,
is a fibre bundle, and \,$\mathcal{H} \subset \mathcal{V}(P)$\, an Ehresmann connection on \,$P$.\,
Then, for each smooth curve
\,$\gamma : [\,0,1] \xrightarrow{{\color{white}222}} M$\,
in the base manifold \,$M$
and each point
\,$p \,\in\, P_{\gamma(0)} \,:=\, \pi^{-1}(\gamma(0))$\,
in the fibre over
\,$\gamma(0) \in M$,\,
there exists a \textbf{\color{black}unique maximally defined initial horizontal lift}
\;\,$\widetilde{\gamma} : [\,0\,,\varepsilon\,) \xrightarrow{{\color{white}222}} P$\,
of \,$\gamma$ through \,$p$,\,
i.e., \,$\widetilde{\gamma}$\, is a curve in \,$P$\, that satisfies:
\begin{itemize}
\item
	$\widetilde{\gamma}(0) \,=\, p$\,,
\item
	$\pi \,\circ\, \widetilde{\gamma} \; = \; \gamma$\,,
\item
	$\widetilde{\gamma}^{\,\prime}(t) \,\in\, \mathcal{H}_{\,\widetilde{\gamma}(t)}$\,,\;
	for each \,$t \in [\,0\,,\varepsilon\,)$\,, and
% \item
% 	$\varepsilon > 0$\, is maximal\,, and
\item
	if
	\,$\xi : [\,0\,,\delta\,) \xrightarrow{{\color{white}222}} P$\,
	is any initial horizontal lift of \,$\gamma$,\, i.e.,
	\begin{equation*}
	\xi(0) =\, p\,,
	\quad
	\pi \,\circ\, \xi(t) \,=\, \gamma(t)\,,
	\quad\textnormal{and}\quad\;\;
	\xi(t) \in \mathcal{H}_{\,\xi(t)}\,,
	\quad
	\textnormal{for each \,$t \in [\,0\,,\delta\,)$}\,,
	\end{equation*}
	then we have: \,$\delta < \varepsilon$,\, and
	\;$\xi(t) = \widetilde{\gamma}(t)$\, for each \,$t \in [\,0\,,\delta\,)$.\,
\end{itemize}
\end{lemma}

          %%%%% ~~~~~~~~~~~~~~~~~~~~ %%%%%

\vskip 0.5cm
\section{Principal bundles and principal (Ehresmann) connections}

          %%%%% ~~~~~~~~~~~~~~~~~~~~ %%%%%

\begin{definition}[Principal bundles]
\mbox{}
\vskip 0.1cm
\noindent
A \textbf{principal fibre bundle} (or \textbf{principal bundle}) is a fibre bundle
\,$\pi : P \longrightarrow M$\,
which satisfies each of the following conditions:
\begin{itemize}
\item
	The general fibre of the fibre bundle
	\,$\pi : P \longrightarrow M$\,
	is a Lie group \,$G$.\,
	The Lie group \,$G$\, is called the \textbf{structure group} of the principal bundle.
	% \begin{center}
	% \begin{tikzcd}
	% G
	% 	\arrow[rr, hook, thick]
	% &&
	% P
	% 	\arrow[dd, thick, two heads, "\pi"]
	% \\ \\
	% &&
	% M
	% \end{tikzcd}
	% \end{center}
	\begin{center}
	\begin{tikzcd}
	G
		\arrow[r, hook]
	&
	P
		\arrow[dd, two heads, "\pi"]
	\\ \\
	&
	M
	\end{tikzcd}
	\end{center}
\item
	There is a smooth action
	\,$P \times G \longrightarrow P$\,
	on the right.
\item
	The aforemention right action on \,$P$\, by \,$G$\,
	preserves each fibre of \,$\pi$\,
	and is simply transitive on each fibre of \,$\pi$,\,
	i.e., the action restricts to each fibre:
	\begin{center}
	\begin{tikzcd}
	P_{x} \times G
		\arrow[r]
	&
	P_{x}
	\end{tikzcd}
	\end{center}
	and, for each \,$x \in M$\, and each \,$p \in P_{x} = \pi^{-1}(x)$\,,
	the orbit map
	\begin{equation*}
	\begin{array}{ccc}
	G & \longrightarrow & P_{x}
	\\
	g & \longmapsto & p \cdot g
	\end{array}
	\end{equation*}
	is a bijection.
\item
	There exists a bundle atlas of \textbf{$G$-equivariant bundle charts}
	\,$\phi_{i} : \pi^{-1}(U_{i}) \longrightarrow U_{i} \times G$,\, i.e.,
	\renewcommand{\labelitemii}{$\circ$}
	\begin{itemize}
	\item
		$M \; = \; \underset{i}{\bigcup}\;U_{i}$\,,
	\item
		for each \,$p \in \pi^{-1}(U_{i})$\, and each \,$g \in G$,\, we have:
		\begin{equation*}
		\phi_{i}(\,p \cdot g\,) \;=\; \phi_{i}(p) \,\cdot\, g
		\end{equation*}
	\item
		for each \,$i$,\, the right action induced on \,$U_{i} \times G$\, by \,$\phi_{i}$\, is given by:
		\begin{equation*}
		\left(\, x \,\overset{{\color{white}1}}{,}\, h \,\right) \cdot\, g
		\; = \;
			\left(\, x \,\overset{{\color{white}1}}{,}\, h \cdot g \,\right),
		\end{equation*}
		for each \,$x \in M$\, and for each \,$g, h \in G$.\,
	\end{itemize}
\end{itemize}
\end{definition}

          %%%%% ~~~~~~~~~~~~~~~~~~~~ %%%%%

% \vskip 0.5cm
% \begin{lemma}[Vertical subbundle of principal bundles]
% \mbox{}
% \vskip -0.05cm
% \noindent
% Suppose that
% \,$G \xhookrightarrow{{\color{white}222}} P \xrightarrow{{\color{white}2}\pi{\color{white}2}} M$\,
% is a principal bundle, and \,$\mathcal{V}(P)$\, is the following disjoint union:
% \begin{equation*}
% \mathcal{V}(P)
% \;\; := \;
% 	\underset{p\,\in\,P}{\bigsqcup}\,\ker(\,\d\pi_{p}\,)\,,
% \end{equation*}
% where, for each \,$p \in P$\, the map
% \,$\d\pi_{p} : T_{p}P \xrightarrow{{\color{white}222}} T_{\pi(p)}M$\,
% is the differential of
% \,$P \xrightarrow{{\color{white}2}\pi{\color{white}2}} M$\,
% at \,$p \in P$.\,
% Then, \,$\mathcal{V}(P)$\, is a subbundle of the tangent bundle \,$TP$\, of \,$P$.\,
% $\mathcal{V}(P)$\, is called the \textbf{vertical subbundle} of
% the principal bundle
% \,$G \xhookrightarrow{{\color{white}222}} P \xrightarrow{{\color{white}2}\pi{\color{white}2}} M$.\,
% \end{lemma}

          %%%%% ~~~~~~~~~~~~~~~~~~~~ %%%%%

\vskip 0.5cm
\begin{definition}[Principal Ehresmann connection on a principal bundle]
\mbox{}
\vskip -0.05cm
\noindent
An Ehresmann connection \,$\mathcal{H} \subset \mathcal{T}(P)$\, on a principal bundle
\,$G \xhookrightarrow{{\color{white}222}} P \xrightarrow{{\color{white}2}\pi{\color{white}2}} M$\,
is said to be \textbf{principal} if \;$\mathcal{H} \subset \mathcal{T}(P)$\, is right-invariant, i.e.,
\begin{equation*}
\mathcal{H}_{\,p \cdot g}
\;\; = \;\;
	\d(R_{g})_{p}\!\left(\,\overset{{\color{white}.}}{\mathcal{H}_{\,p}}\,\right),
\end{equation*}
for each \,$p \in P$\, and for each \,$g \in G$.\,
Here, \,$\d(R_{g})_{p}$\, denotes the differential at \,$p \in P$\,
of the right translation on \,$P$\, by \,$g$.\, 
\end{definition}

          %%%%% ~~~~~~~~~~~~~~~~~~~~ %%%%%

\vskip 0.5cm
\begin{definition}[Fundamential vector fields on principal bundles]
\mbox{}
\vskip -0.05cm
\noindent
Suppose
\,$G \xhookrightarrow{{\color{white}222}} P \xrightarrow{{\color{white}2}\pi{\color{white}2}} M$\,
is a principal bundle and
\,$\mathfrak{g}$\, is the Lie algebra of the structure group \,$G$.\,
For each \,$A \in \mathfrak{g}$,\, define \,$A^{\sharp} \in \Gamma(\mathcal{V}(P))$\, as follows:
\begin{equation*}
A^{\sharp}(\,p\,)
\;\, := \;\,
	\d(\iota_{p})_{e}\!\left(\,\overset{{\color{white}.}}{A}\,\right)
\; \in \;
	\mathcal{V}_{p}(P)
\; = \;
	\ker(\,\d\pi_{p}\,)\,,
\quad
\textnormal{for each \,$p \in P$}\,,
\end{equation*}
where \,$e \in G$\, is the identity element of \,$G$,\, and
\,$\iota_{p} : G \xrightarrow{{\color{white}222}} P : g \xmapsto{{\color{white}222}} p \cdot g$.\,
The vertical vector field \,$A^{\sharp} \in \Gamma(\mathcal{V}(P))$\, is  called the
\textbf{fundamental vector field} on \,$P$\, associated to \,$A \in \mathfrak{g}$.\, 
\end{definition}

          %%%%% ~~~~~~~~~~~~~~~~~~~~ %%%%%

\vskip 0.5cm
\begin{definition}[Connection $1$-form on a principal bundle]
\mbox{}
\vskip -0.05cm
\noindent
Suppose
\,$G \xhookrightarrow{{\color{white}222}} P \xrightarrow{{\color{white}2}\pi{\color{white}2}} M$\,
is a principal bundle and
\,$\mathfrak{g}$\, is the Lie algebra of the structure group \,$G$.\,
A \textbf{connection $1$-form} on the principal bundle
\,$G \xhookrightarrow{{\color{white}222}} P \xrightarrow{{\color{white}2}\pi{\color{white}2}} M$\,
is a $1$-form \,$\omega \in \Omega^{1}(P,\mathfrak{g})$\, on the total space \,$P$\, that satisfies
the following propertiees:
\begin{itemize}
\item
	$\left(\,R_{g}^{*}\,\omega\,\right)_{p}
	\,=\,
		\Ad_{g^{-1}} \,\circ\, \omega_{p} \,:
		\mathcal{T}_{p}\,P \xrightarrow{{\color{white}222}} \mathfrak{g}$\,,\;
	for each \,$g \in G$\, and each \,$p \in P$,\, and
\item
	$\omega_{p}\!\left(\,\overset{{\color{white}.}}{A^{\sharp}(p)}\,\right) \,=\, A$\,,\;
	for each \,$p \in P$\, and each \,$A \in \mathfrak{g}$,\,
	where \,$A^{\sharp} \in \Gamma(\mathcal{V}(P))$\, is the fundamental vector field
	on \,$P$\, associated to \,$A \in \mathfrak{g}$.\,
\end{itemize}
\end{definition}

          %%%%% ~~~~~~~~~~~~~~~~~~~~ %%%%%

\vskip 0.5cm
\begin{theorem}\textnormal{\bf(Bijection between principal connections and connection $1$-forms on a principal bundle, Theorem 5.2.2, p.262, \cite{Hamilton2017})}
\mbox{}
\vskip 0.05cm
\noindent
Suppose
\,$G \xhookrightarrow{{\color{white}222}} P \xrightarrow{{\color{white}2}\pi{\color{white}2}} M$\,
is a principal bundle and
\,$\mathfrak{g}$\, is the Lie algebra of the structure group \,$G$.\,
Then, there is a bijection between the set of principal Ehresmann connections on \,$P$\,
and the set of connection $1$-forms on \,$P$.\, More precisely,
\begin{enumerate}
\item
	Let \,$\omega \in \Omega^{1}(P,\mathfrak{g})$\, be a connection $1$-form.
	Define a subbundle \,$\mathcal{H}$\, of \,$\mathcal{T}(P)$\, as follows:
	\begin{equation*}
	\mathcal{H}_{p} \;\; := \;\; \ker\!\left(\,\overset{{\color{white}.}}{\omega_{p}}\,\right)
	\end{equation*}
	Then, \,$\mathcal{H}_{p}$\, is a principal Ehresmann connection on \,$P$.\,
\item
	Let \,$\mathcal{H} \subset \mathcal{T}(P)$\, be a principal Ehresmann connection on \,$P$.\,
	Define \,$\omega \in \Omega^{1}(P,\mathfrak{g})$\, as follows:
	\begin{equation*}
	\omega_{p}\!\left(\,
		X^{\sharp}_{p} \,+ Y_{p}
		\,\right)
	\;\; := \;\;
		X\,,
	\end{equation*}
	for each \,$p \in P$,\, each \,$X \in \mathfrak{g}$,\, and each \,$Y_{p} \in \mathcal{H}_{p}$.\,
	Then, \,$\omega \in \Omega^{1}(P,\mathfrak{g})$\, is a connection $1$-form.
\end{enumerate}
\end{theorem}

          %%%%% ~~~~~~~~~~~~~~~~~~~~ %%%%%

\vskip 0.5cm
\section{Transformation laws under gauge transformations (bundle automorphisms)}

          %%%%% ~~~~~~~~~~~~~~~~~~~~ %%%%%

\vskip 0.5cm
\begin{theorem}\textnormal{\bf(Transformation of connections under gauge transformations, Theorem 5.4.4, p.273, \cite{Hamilton2017})}
\mbox{}
\vskip 0.1cm
\noindent
Suppose:
\begin{itemize}
\item
	\vskip -0.1cm
	$G \xhookrightarrow{{\color{white}222}} P \xrightarrow{{\color{white}2}\pi{\color{white}2}} M$\,
	is a principal bundle and \,$\mathfrak{g}$\, is the Lie algebra of the structure group \,$G$,
\item
	$\omega \in \Omega^{1}(P,\mathfrak{g})$\, is a connection $1$-form,\,
	$F^{\,\omega} \in \Omega^{2}(P,\mathfrak{g})$\, is the curvature $2$-form of \,$\omega$,\,
	and
\item
	$f \in \mathscr{G}(P)$\, be a gauge transformation.
\end{itemize}
Let \;$\mu_{G} \in \Omega^{1}(G,\mathfrak{g})$\, be the Maurer-Cartan form of \,$G$,\, and
\,$\sigma_{f} \in \mathscr{C}^{\infty}(P,G)^{G}$\,
be defined implicitly by:
\begin{equation*}
f(p)
\;\; = \;\;
	p \cdot \sigma_{f}(p)\,,
\quad
\textnormal{for each \,$p \in P$}
\end{equation*}
Then, the following statements are true:
\begin{enumerate}
\item
	For each \,$p \in P$\, and each \,$X \in \mathcal{T}_{p}\,P$,\, we have:
	\begin{equation*}
	(f_{*})_{p}(\,X\,)
	\;\; = \;\;
		\left(\,D_{p}\,R_{\sigma_{f}(p)}\,\right)\!\left[\;\overset{{\color{white}.}}{X}\;\right]
		\; + \,
		\left(\,
			(\,\sigma_{f}^{*}\,\mu_{G}\,)_{p}
			\left[\;\overset{{\color{white}.}}{X}\;\right]
			\,\right)^{\sharp}_{\!f(p)}
	\end{equation*}
\item
	$f^{*}(\omega)$\, is a connection $1$-form on \,$P$.
\item
	$f^{*}(\omega)$\, can be given by:
	\begin{equation*}
	\left(\,f^{*}\omega\,\right)_{p}
	\;\; = \;\;
		\Ad_{\sigma_{f}(p)^{-1}}\,\circ\,\omega_{p}
		\; + \;
		\left(\,\sigma_{f}^{*}\,\mu_{G}\,\right)_{p}\,,
	\quad
	\textnormal{for each \,$p \in P$}
	\end{equation*}
\item
	The curvature $2$-form \,$F^{\,f^{*}(\omega)}$\, is given by:
	\begin{equation*}
	\left(\,F^{\,f^{*}\omega}\,\right)_{\!p}
	\;\; = \;\;
		\Ad_{\sigma_{f}(p)^{-1}} \,\circ\, F^{\,\omega}_{p}\,,
	\quad
	\textnormal{for each \,$p \in P$}
	\end{equation*}
\end{enumerate}
\end{theorem}
\proof
\begin{enumerate}
\item
	Let \,$p \in P$\, and \,$X \in \mathcal{T}_{p}\,P$.\,
	\vskip -0.05cm
	\noindent
	Let
	\,$\gamma : (-\varepsilon,\varepsilon) \xrightarrow{{\color{white}222}} P$\,
	be any smooth curve into \,$P$\, such that \,$\gamma(0) = p$\, and \,$\gamma^{\prime}(0) = X$.
	\vskip -0.05cm
	\noindent
	Let \,$\Phi : P \times G \xrightarrow{{\color{white}222}} P$\,
	be the right action of \,$G$\, on \,$P$,\, i.e., \,$\Phi(p,g) = p \cdot g$.\,
	\vskip 0.01cm
	\noindent
	Then, we have:
	\begin{eqnarray*}
	(f_{*})_{p}(\,X\,)
	& = &
		\left.\dfrac{\d}{\d\,t}\right\vert_{t = 0}\,
		f\!\left(\,\overset{{\color{white}.}}{\gamma(t)}\,\right)
	\;\; = \;\;
		\left.\dfrac{\d}{\d\,t}\right\vert_{t = 0}\,
		\gamma(t) \cdot \sigma_{f}(\gamma(t))
	\;\; = \;\;
		\left.\dfrac{\d}{\d\,t}\right\vert_{t = 0}\,
		\Phi\!\left(\,
			\overset{{\color{white}.}}{\gamma(t)}
			\;,\,
			\sigma_{f}(\gamma(t))
			\,\right)
	\\
	& = &
		\left(\,D_{(\,\gamma(0)\,,\,\sigma_{f}(\gamma(0))\,)}\Phi\,\right)\!
		\left(\,
			\gamma^{\prime}(0)
			\,,\,
			\left.\dfrac{\d}{\d\,t}\right\vert_{t=0}
			\sigma_{f}(\gamma(t))
			\,\right)
	\\
	& = &
		\left(\,D_{(\,p\,,\,\sigma_{f}(p)\,)}\Phi\,\right)\!
		\left(\,
			X
			\,,\,
			\left.\dfrac{\d}{\d\,t}\right\vert_{t=0}\,\sigma_{f}(\gamma(t))
			\,\right)
	\\
	& = &
		\left(\,D_{p}\,R_{\sigma_{f}(p)}\,\right)\!\left[\;\overset{{\color{white}.}}{X}\;\right]
		\; + \,
		\left(\,
			(\mu_{G})_{\sigma_{f}(p)}\!\left(\,
				\left.\dfrac{\d}{\d\,t}\right\vert_{t=0}\,\sigma_{f}(\gamma(t))
				\,\right)
			\,\right)^{\sharp}_{p\cdot\sigma_{f}(p)},
		\;\;
		\textnormal{by Proposition 3.5.4, p.149, \cite{Hamilton2017}}
	\\
	& = &
		\left(\,D_{p}\,R_{\sigma_{f}(p)}\,\right)\!\left[\;\overset{{\color{white}.}}{X}\;\right]
		\; + \,
		\left(\,
			(\mu_{G})_{\sigma_{f}(p)}\!\left(\,
				(D_{p}\,\sigma_{f})\!
				\left[\,\left.\dfrac{\d}{\d\,t}\right\vert_{t=0}\,\gamma(t)\,\right]
				\,\right)
			\,\right)^{\sharp}_{\!f(p)}
	\\
	& = &
		\left(\,D_{p}\,R_{\sigma_{f}(p)}\,\right)\!\left[\;\overset{{\color{white}.}}{X}\;\right]
		\; + \,
		\left(\,
			(\mu_{G})_{\sigma_{f}(p)}\!\left(\,
				(D_{p}\,\sigma_{f})\!
				\left[\;\overset{{\color{white}.}}{X}\;\right]
				\,\right)
			\,\right)^{\sharp}_{\!f(p)}
	\\
	& = &
		\left(\,D_{p}\,R_{\sigma_{f}(p)}\,\right)\!\left[\;\overset{{\color{white}.}}{X}\;\right]
		\; + \,
		\left(\,
			(\,\sigma_{f}^{*}\,\mu_{G}\,)_{p}
			\left[\;\overset{{\color{white}.}}{X}\;\right]
			% (\mu_{G})_{\sigma_{f}(p)}\!\left(\,
			% 	(D_{p}\,\sigma_{f})\!
			%	\left[\;\overset{{\color{white}.}}{X}\;\right]
			%	\,\right)
			\,\right)^{\sharp}_{\!f(p)}\,,
	\end{eqnarray*}
	where the last equality follows from the following commutative diagram:
	\begin{center}
	\begin{tikzcd}
	\mathcal{T}_{p}\,P
		\arrow[rrr, "D_{p}\,\sigma_{f}", swap]
		\arrow[rrrrrr, bend left = 35, "\left(\,\sigma_{f}^{*}\,\mu_{G}\,\right)_{p}"]
	&&&
	\mathcal{T}_{\sigma_{f}(p)}\,G
		\arrow[rrr, "(\,\mu_{G}\,)_{\sigma_{f}(p)}\;\;", swap]
	&&&
	\mathfrak{g}
	\end{tikzcd}
	\end{center}
\item
\item
	Let \,$p \in P$\, and \,$X \in \mathcal{T}_{p}\,P$.\,
	Then, we have:
	\begin{eqnarray*}
	(\,f^{*}\omega\,)_{p}\left[\;\overset{{\color{white}.}}{X}\;\right]
	& = &
		\omega_{f(p)}\!\left[\;(\overset{{\color{white}.}}{f_{*})_{p}(X)}\;\right]
	\\
	& = &
		\omega_{f(p)}\!\left[\;
			\left(\,D_{p}\,R_{\sigma_{f}(p)}\,\right)\!\left[\;\overset{{\color{white}.}}{X}\;\right]
			\; + \,
			\left(\,
				(\,\sigma_{f}^{*}\,\mu_{G}\,)_{p}
				\left[\;\overset{{\color{white}.}}{X}\;\right]
				\,\right)^{\sharp}_{\!f(p)}
			\;\right],
		\quad
		\textnormal{by (i)}
	\\
	& = &
		\omega_{\,p \cdot \sigma_{f}(p)}\!\left[\;
			\left(\,D_{p}\,R_{\sigma_{f}(p)}\,\right)\!\left[\;\overset{{\color{white}.}}{X}\;\right]
			\;\right]
		\; + \;
		\omega_{f(p)}\!\left[\;
			\left(\,
				(\,\sigma_{f}^{*}\,\mu_{G}\,)_{p}
				\left[\;\overset{{\color{white}.}}{X}\;\right]
				\,\right)^{\sharp}_{\!f(p)}
			\;\right]
	\\
	& = &
		\left(\,R_{\sigma_{f}(p)}^{*}\,\omega\,\right)_{\!p}
		\left[\;\overset{{\color{white}.}}{X}\;\right]
		\; + \;
		(\,\sigma_{f}^{*}\,\mu_{G}\,)_{p}
		\left[\;\overset{{\color{white}.}}{X}\;\right]
	\\
	& = &
		\left(\,\Ad_{\sigma_{f}(p)^{-1}}\,\circ\,\omega_{p}\,\right)\!
		\left[\;\overset{{\color{white}.}}{X}\;\right]
		\; + \;
		(\,\sigma_{f}^{*}\,\mu_{G}\,)_{p}
		\left[\;\overset{{\color{white}.}}{X}\;\right],
	\end{eqnarray*}
	where the last two equalities follow from the hypothesis that \,$\omega$\,
	is a connection $1$-form, as well as from the following commutative diagram:
	\begin{center}
	\begin{tikzcd}
	\mathcal{T}_{p}\,P
		\arrow[rrr, "D_{p}\,R_{\sigma_{f}(p)}", swap]
		\arrow[rrrrrr, bend left = 35, "\left(\,R_{\sigma_{f}(p)}^{*}\,\omega\,\right)_{p}"]
	&&&
	\mathcal{T}_{f(p)}\,P
		\arrow[rrr, "\omega_{f(p)}", swap]
	&&&
	\mathfrak{g}
	\end{tikzcd}
	\end{center}
	We may now conclude that
	\begin{equation*}
	f^{*}\omega
	\;\; = \;\;
		\Ad_{\sigma_{f}^{-1}} \,\circ\, \omega
		\; + \;
		\sigma_{f}^{*}(\,\mu_{G}\,)\,,
	\end{equation*}
	which completes the proof of (iii) as well as that of the present Proposition.
\item
	First, note that
	\,$F^{\,f^{*}\omega} \,=\, f^{*}\!\left(\,F^{\,\omega}\,\right)$,\, indeed:
	\begin{eqnarray*}
	F^{\,f^{*}\omega}
	& = &
		\d(\,\overset{{\color{white}.}}{f^{*}\omega}\,)
		\; + \;
		\dfrac{1}{2}\left[\,f^{*}\omega\,\overset{{\color{white}1}}{,}\,f^{*}\omega\,\right],
		\quad
		\textnormal{by structure equation}
	\\
	& = &
		f^{*}(\,\d\,\omega\,)
		\; + \;
		f^{*}\!\left(\,\dfrac{1}{2}\left[\,\omega\,\overset{{\color{white}1}}{,}\,\omega\,\right]\,\right),
		\quad
		\textnormal{by commutativity properties of pullbacks}
	\\
	& = &
		f^{*}\!\left(\;
			\d\,\omega
			\;+\;
			\dfrac{1}{2}\left[\,\omega\,\overset{{\color{white}1}}{,}\,\omega\,\right]
			\,\right)
	\;\; = \;\;
		f^{*}(\,F^{\,\omega}\,)
	\end{eqnarray*}
	Now, note that, for each \,$p \in P$,\, we have:\;
	$f(p) \,=\, p \cdot \sigma_{f}(p) \,=\, R_{\sigma_{f}(p)}\!\left(\,\overset{{\color{white}.}}{p}\,\right)$.\,
	Hence,
	\begin{equation*}
	\left(\,F^{\,f^{*}\omega}\,\right)_{\!p}
	\;\;\, = \;\;\,
		\left(\,f^{*}F^{\,\omega}\,\right)_{p}
	\;\;\, = \;\;\,
		\left(\,R^{*}_{\sigma_{f}(p)}(\,F^{\,\omega}\,)\,\right)_{\!p}
	\;\;\, = \;\;\,
		\Ad_{\sigma_{f}(p)^{-1}} \,\circ\, F^{\,\omega}_{p}
	\end{equation*}
	as required.
	\qed
\end{enumerate}

          %%%%% ~~~~~~~~~~~~~~~~~~~~ %%%%%

