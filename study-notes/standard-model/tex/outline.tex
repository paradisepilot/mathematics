
          %%%%% ~~~~~~~~~~~~~~~~~~~~ %%%%%

\chapter{Outline}
\setcounter{theorem}{0}
\setcounter{equation}{0}

%\cite{vanDerVaart1996}
%\cite{Kosorok2008}

%\renewcommand{\theenumi}{\alph{enumi}}
%\renewcommand{\labelenumi}{\textnormal{(\theenumi)}$\;\;$}
\renewcommand{\theenumi}{\roman{enumi}}
\renewcommand{\labelenumi}{\textnormal{(\theenumi)}$\;\;$}

          %%%%% ~~~~~~~~~~~~~~~~~~~~ %%%%%

\section{Key physical insight behind Standard Model: symmetries}

\begin{itemize}
\item
	\underline{Forbiddance of \textit{action at a distance} entails relevance of ``fields''}
	\begin{equation*}
	\left.\begin{array}{c}
	\textnormal{Inadmissibility of \textit{action at a distance}}
	\\
	+
	\\
	\textnormal{States of a quantum system are to be}
	\\
	\textnormal{described by $1$-dimensional subspaces}
	\\
	\textnormal{in a complex Hilbert space}
	\end{array}\right\}
	\quad\Longrightarrow\quad
	\left\{\begin{array}{c}
	\textit{pre-quantized}
	\\
	\textnormal{descriptions of particles}
	\\
	\textnormal{will be sections of appropriate}
	\\
	\textnormal{complex vector bundles $E$}
	\end{array}\right.
	\end{equation*}
\item
	\underline{Principle of gauge invariance}
	\begin{equation*}
	\left.\begin{array}{c}
	\textnormal{Change of coordinatizaion}
	\\
	\textnormal{of internal spaces}
	\\
	\textnormal{must NOT change physics}
	\end{array}\right\}
	\quad\Longrightarrow\quad
	\left\{\begin{array}{c}
	\textnormal{invariance of Lagrangian}
	\\
	\textnormal{under gauge transformations}
	\end{array}\right.
	\end{equation*}	
\item
	\underline{Symmetries exhibited by particles entail relevance of vector bundles and principal bundles}
	\begin{equation*}
	\left.\begin{array}{c}
	\textnormal{Particles exhibiting}
	\\
	\textnormal{internal symmetries}
	\end{array}\right\}
	\quad\Longrightarrow\quad
	\left\{\begin{array}{c}
	\textnormal{the complex vector bundle $E$}
	\\
	\textnormal{with compact Lie group $G$}
	\\
	\textnormal{as structure group, where $G$ is}
	\\
	\textnormal{unrelated to spacetime symmetries}
	\end{array}\right.
	\end{equation*}
	Principal bundles now enter scene because every vector bundle $E$, with vector space fibre $V$, is isomorphic
	to the associated vector bundle $E \cong P \times_{\rho} V$, where $P = \textnormal{Frame}(E)$ is the frame
	bundle of $E$, and $\rho : \textnormal{GL}(V) \longrightarrow \textnormal{GL}(V)$ is the identity map (homomorphism).
	Note also that if the structure group of $E$ is a proper subgroup $G$ of $\textnormal{GL}(V)$, then
	$P$ above can be replaced with the appropriate reduction of $\textnormal{Frame}(E)$ to $G$, and
	$\rho$ can be replaced with the identity map/homomorphism $G \longrightarrow G$.
\item
	\underline{Symmetries exhibited by particles entail relevance of \textit{connections} (a.k.a. \textit{gauge fields})}
	\begin{equation*}
	\left.\begin{array}{c}
	\textnormal{The Lagrangian density will involve}
	\\
	\textnormal{the fields themselves, as well as}
	\\
	\textnormal{the ``dynamics'' of the fields, i.e.,}
	\\
	\textnormal{first derivatives of the fields w.r.t. spacetime}
	\end{array}\right\}
	\quad\Longrightarrow\quad
	\left\{\begin{array}{c}
	\textnormal{Lagrangian will involve}
	\\
	\textnormal{\textit{connections} (``gauge fields'')}
	\\
	\textnormal{on the complex vector bundle $E$}
	\end{array}\right.
	\end{equation*}
	\textit{This is because complex vector bundles do NOT come canonically equipped with
	sufficient structure (i.e., connections) required for defining derivatives of its sections w.r.t. spacetime.}
\item
	\begin{equation*}
	%\left.
	\begin{array}{c}
	\textnormal{$G$-action on $E$}
	\end{array}
	%\right\}
	\quad\Longrightarrow\quad
	%\left\{
	\begin{array}{c}
	\textnormal{$G$-action on the connections (gauge fields) on $E$}
	\end{array}
	%\right.
	\end{equation*}
\item
	Suppose $E = P \times_{\rho}V$ is a complex vector bundle over a spacetime $\mathcal{M}$
	which is an associated vector bundle, where $P$ is a princpal $G$-bundle, and
	$\rho : G \longrightarrow \textnormal{GL}(V)$ is a representation.
	Then, is it true that every connection on $E$ is induced by a connection on $P$?
\end{itemize}

          %%%%% ~~~~~~~~~~~~~~~~~~~~ %%%%%

\section{Key physical insight behind Standard Model: Dirac's equation}
Dirac's equation seems to be the heart -- or key physical insight -- of the Standard Model (of particle physics),
in the sense that the rest of the Standard Model arguably follows from mathematics, or relatively straightforward
mathematical formulations of arguably self-evident physical constraints.
\begin{itemize}
\item
	Quantum mechanics was known to be insufficient -- there had been experimental observations
	that could not be explained with quantum mechanics.
	Quantum mechanics is incompatible with Special Relativity (e.g., there is an asymmetry between space and time
	in the Schr\"odinger equation).
	In particular, quantum mechanics fails to give accurate precisions for particles moving at speeds close to
	that of light.
\item
	\textbf{The Schr\"odinger equation:}
	\vskip 0.01cm
	According to Newtonian mechanics, the energy \,$E$\, and the momentum (vector) \,$\mathbf{p}$\,
	of a free particle (i.e., a particle on which no force is acting) of mass \,$m$\,
	satisfy the following identity:
	\begin{equation}
	\label{NewtonionEnergyMomemtumRelation}
	E \;=\; \dfrac{\Vert\,\mathbf{p}\,\Vert^{2}}{2m}
	\end{equation}
	The validity of this identity can be seen as follows:
	\begin{equation*}
	E
	\,:=\,
		\left(\!\begin{array}{c} \textnormal{total energy} \\ \textnormal{of the} \\ \textnormal{particle} \end{array}\!\right)
	\,=\,
		\left(\!\begin{array}{c} \textnormal{kinetic energy} \\ \textnormal{of the} \\ \textnormal{particle} \end{array}\!\right)
	\,=\,
		\dfrac{1}{2}\,m\,\Vert\,\mathbf{v}\,\Vert^{2}
	\,=\,
		\dfrac{(\,\Vert\,m\cdot\mathbf{v}\,\Vert\,)^{2}}{2m}
	\,=\,
		\dfrac{\Vert\,\mathbf{p}\,\Vert^{2}}{2m}\,,
	\end{equation*}
	where \,$\mathbf{v}$\, is the velocity vector of the particle.
	Writing
	\,$\mathbf{p} = (\,p_{1},p_{2},p_{3}\,)$,\,
	we may re-express
	\eqref{NewtonionEnergyMomemtumRelation}
	as follows:
	\begin{equation*}
	E \,-\, \dfrac{1}{2m}\left(\,p_{1}^{2} \overset{{\color{white}.}}{+} p_{2}^{2} + p_{3}^{2}\,\right) \;=\; 0
	\end{equation*}
	If we
	\begin{enumerate}
	\item
		replace -- according to canonical quantization -- the quantities
		\,$E$, $p_{1}$, $p_{2}$, $p_{3}$\,
		respectively with the following differential operations
		\begin{equation}
		\label{CanonicalQuantization}
		E \;\longmapsto\; \i\hbar\,\dfrac{\partial}{\partial t}\,,
		\quad\quad
		p_{j} \;\longmapsto\; -\,\i\hbar\,\dfrac{\partial}{\partial x_{j}}\,,
		\quad\quad
		\textnormal{and}
		\end{equation}
	\item
		let the resulting differential operator act on the wave function
		\,$\psi(t,x_{1},x_{2},x_{3})$\,,\,
	\end{enumerate}
	we arrive at the Schr\"odinger equation satisfied by the wave function
	\,$\psi(t,x_{1},x_{2},x_{3})$\,
	representing a free particle:
	\begin{equation*}
	\left(\,\i\hbar\,\dfrac{\partial}{\partial t} + \dfrac{\hbar^{2}}{2m}
		\left(
			\dfrac{\partial^{2}}{\partial x_{1}^{2}}
			+ \dfrac{\partial^{2}}{\partial x_{2}^{2}}
			+ \dfrac{\partial^{2}}{\partial x_{3}^{2}}
			\right)
		\,\right)
	\psi
	\,=\,
		0
	\end{equation*}
\item
	\textbf{The Klein-Gordon equation:}
	\vskip 0.01cm
	The asymmetry between time and space in the Schr\"odinger equation is obvious:
	the appearance of the first-order partial derivative with respect to time
	but second-order partial derivatives with respect to the spatial coordinates.
	One can also show that the Schr\"odinger equation is not Lorentz-invariant.
	It is thus clear that the Schr\"odinger equation is incompatible with special relativity.
	
	In order to remedy this, we derive a new quantum equation of motion by quantizing instead
	the relativistic counterpart of \eqref{NewtonionEnergyMomemtumRelation}:
	\begin{equation}
	\label{RelativisticEnergyMomemtumRelation}
	E^{2} \;=\; p^{2} \,+\, m^{2}
	\end{equation}
	Via the same quantization scheme
	\eqref{CanonicalQuantization}
	as before, we obtain the Klein-Gordon equation:
	\begin{equation*}
	\left(\,\dfrac{\partial^{2}}{\partial t^{2}}
	- \left(
		\dfrac{\partial^{2}}{\partial x_{1}^{2}}
		+ \dfrac{\partial^{2}}{\partial x_{2}^{2}}
		+ \dfrac{\partial^{2}}{\partial x_{3}^{2}}
		\right)
	\,+\,
		\dfrac{m^{2}}{\hbar^{2}}
		\,\right)
	\psi
	\,=\,
		0
	\end{equation*}
	The obvious asymmetry between space and time in the Schr\"odinger equation is no longer
	present in the Klein-Gordon equation.
	It is also straightforward to show that the Klein-Gordon equation is Lorentz-invariant.
	
	However, the Klein-Gordon equation is still theoretically unsatisfactory due to at least two facts:
	\begin{enumerate}
	\item[$\circ$]
		it is second-order in time, rendering it difficult to give it a dynamical interpretation, and
	\item[$\circ$]
		it allows negative-energy eigenstates.
	\end{enumerate}

\item
	\textbf{The Dirac equation:}
	\vskip 0.01cm
	Dirac sought a new equation of motion by seeking a first-order linear differential operator
	\begin{equation*}
	D
	\;=\;
		\gamma^{\mu}\partial_{\mu}
	\;=\;
		\gamma^{0}\dfrac{\partial}{\partial x_{0}} 
		\,+\, \gamma^{1}\dfrac{\partial}{\partial x_{1}} 
		\,+\, \gamma^{2}\dfrac{\partial}{\partial x_{2}} 
		\,+\, \gamma^{3}\dfrac{\partial}{\partial x_{3}} 
	\end{equation*}
	whose square is the d'Alembert operator
	\begin{equation*}
	\Box
	\;=\;
		\dfrac{\partial^{2}}{\partial t^{2}} \,-\, \Delta
	\;=\;
		\dfrac{\partial^{2}}{\partial t^{2}}
		\,-\, \dfrac{\partial^{2}}{\partial x_{1}^{2}}
		\,-\, \dfrac{\partial^{2}}{\partial x_{2}^{2}}
		\,-\, \dfrac{\partial^{2}}{\partial x_{3}^{2}}
	\end{equation*}
	in Minkowski spacetime. 
	Dirac arrived at the famous Dirac equation:
	\begin{equation*}
	\left(\,\i\cdot\gamma^{\mu}\partial_{\mu} - m\,\right)\psi \, = \, 0
	\end{equation*}
	A straightforward calculation shows that
	\,$\left(\,\i\cdot\gamma^{\mu}\partial_{\mu} - m\,\right)^{2} \,=\, \Box + m^{2}$\,
	if the ``coefficients'' \,$\gamma^{\mu}$\, satisfy:
	\begin{equation}
	\label{GammaMatricesSatisfyCliffordRelation}
	\left\{\,\gamma^{\mu}\,,\,\gamma^{\nu}\,\right\}
	\;=\;
		2\cdot g^{\mu\nu}\,,
	\end{equation}
	where
	\,$\left\{\,\gamma^{\mu}\,,\,\gamma^{\nu}\,\right\} \,:=\, \gamma^{\mu}\gamma^{\nu} + \gamma^{\mu}\gamma^{\nu}$\,
	and
	\,$g^{\mu\nu} \,=\, \diag(1,-1,-1,-1)$.\,

	Dirac realized that the condition \eqref{GammaMatricesSatisfyCliffordRelation} cannot be satisfied
	if the \,$\gamma^{\mu}$'s\, were just complex numbers, but that condition is indeed satisfied by
	certain $4 \times 4$ complex matrices.

	This observation led to the following realization (when expressed in modern geometric language):
	The ``wave function'' \,$\psi$\, must be smooth sections of a spinor bundle over spacetime.

\item
	\textbf{Relevance of spin structure: Each spinor bundle admits a unique Dirac operator}
	\vskip 0.01cm
	Recall that a spin structure on spacetime \,$M$\, 
	(pseudo-Riemannian 4-manifold with a Minkowski metric tensor)
	is a principal bundle 
	\,$\textnormal{Spin}(\Re^{1,3})  \hookrightarrow \widetilde{P} \longrightarrow M$\,
	over \,$M$\, which is a double cover of the orthonormal frame bundle of \,$M$:
	\begin{center}
	\begin{tikzcd}
	\textnormal{Spin}(\Re^{1,3})
		\arrow[rr]
		\arrow[dd, two heads, swap, "2:1\;"]
	&&
	\widetilde{P}
		\arrow[dd, two heads, "\;2:1"]
	\\ \\
	\mathcal{L}
		\arrow[rr]
	&&
	P
		\arrow[dd]
	\\ \\
	&&
	M
	\end{tikzcd}
	\end{center}
	where
	\begin{itemize}
	\item[$\circ$]
		$P \longrightarrow M$\, is the orthonormal frame bundle over \,$M$,\,
		which is a principal bundle whose structure group is the Lorentz group \,$\mathcal{L}$.\,
	\item[$\circ$]
		$\widetilde{P} \longrightarrow P$\, is the double covering,
	\item[$\circ$]
		$\textnormal{Spin}(\Re^{1,3}) \longrightarrow \mathcal{L}$\,
		is the universal (double) covering of the Lorentz group \,$\mathcal{L}$.
	\end{itemize}
	A spinor bundle is a complex vector bundle with fibre \,$\C^{2}$\, over spacetime \,$M$\, 
	associated to the principal bundle
	\,$\textnormal{Spin}(\Re^{1,3})  \hookrightarrow \widetilde{P} \longrightarrow M$\,
	via a representation:
	\begin{equation*}
	\rho \; : \; \textnormal{Spin}(\Re^{1,3}) \; \longrightarrow \textnormal{GL}(\C^{2})
	\end{equation*}
	The relevance of spin structures and spinor bundles is the following:
	\textbf{Each spinor bundle admits a unique Clifford connection, which in turn gives rise to a Dirac operator
	defined on that spinor bundle.
	In other words, smooth sections of spinor bundles are the entities for which the Dirac operator makes sense.}

\item
	The following considerations:
	\begin{itemize}
	\item[$\circ$]
		the local nature (no action at a distance) of a physical theory,
	\item[$\circ$]
		the fact that particles admit internal symmetries
	\end{itemize}
	together entails that the suitable mathematical objects to use to model
	internal symmetry varying over spacetime are:
	\begin{itemize}
	\item[$\circ$]
		principal fibre bundles
	\end{itemize}
	The (quantum mechanical) axiom that the states of a quantum system should be
	unit vectors in a given complex Hilbert space entails that the suitable mathematical
	objects to use to model particles are: smooth sections of complex vector bundles
	over spacetime associated to principal bundles whose structure groups induce
	(via group representations) the internal symmetries exhibited by the given quantum system.
\end{itemize}

          %%%%% ~~~~~~~~~~~~~~~~~~~~ %%%%%

\section{Outline -- older version}
\begin{itemize}
\item
	In the Standard Model of particle physics, ``particles'' are defined
	phenomenologically, in the sense that a ``particle'' is defined/identified/classified
	via its observed values of a number of characteristics,
	e.g. mass, charge, spin, colour, etc.
\item
	Each particle characteristic allow only a finite number of admissible values.
	These characteristics (hence the ``particles'' they phenomenologically define)
	are described mathematically as sections of certain vector bundles.
	
	Each admissible value of the characteristic corresponds
	to a $1$-dimensional subspace in the fibre.
	
	The laws of physics must be invariant under relabelling of the admissible values,
	which entails that the aforementioned vector bundles come with an action by a certain
	symmetry group $G$ that permutes the admissible values.
	In other words, these vector bundles are associated vector bundles of a
	principal fibre bundle with structure group $G$, induced by a certain representation
	of $G$ on the fibre space.
	
	These symmetries are called \textbf{gauge symmetries}.
\item
	The evolution in time of the ``particles'' is dictated by the curvature on the ambient vector bundle.
	In return, the presence of the particles affects the curvature on the ambient vector bundles.
	The dynamical interplay between the particle and curvature is determined by the principle
	of least action, i.e. the equation of motion is the Euler-Lagrange equations of a suitable
	action functional.
	
	The Lagrangian density (integrand of the action functional) must be
	Lorentz-invariant and gauge-invariant.
\item
	Creation and annihilation of particles are modelled by quantizing
	the Euler-Lagrange equations. 
\item
	Masses of fermions and the weak interaction bosons ($W^{\pm}$ and $Z^{0}$)
	are introduced by the Higgs mechanism.
	
	The non-zero ground state of the Higgs field ``breaks'' the fully gauge symmetry;
	after symmetry breaking, physics now ``happens'' on a sub-bundle of the original full bundle,
	with structure group $H \subset G$ being a subgroup of the original full unbroken gauge group $G$.
	Such an $H$ is the stabilizer subgroup of a certain vacuum element (a minimum of the Higgs potential)
	in the original fibre.
	
	``Mass'' terms for fermions and certain gauge bosons may appear
	when the original Lagrangian density is restricted to the reduced bundle,
	and re-expressed in terms intrinsic to the reduced bundle.
\end{itemize}

          %%%%% ~~~~~~~~~~~~~~~~~~~~ %%%%%

