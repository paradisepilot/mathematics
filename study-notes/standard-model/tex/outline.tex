
          %%%%% ~~~~~~~~~~~~~~~~~~~~ %%%%%

\chapter{Outline}
\setcounter{theorem}{0}
\setcounter{equation}{0}

%\cite{vanDerVaart1996}
%\cite{Kosorok2008}

%\renewcommand{\theenumi}{\alph{enumi}}
%\renewcommand{\labelenumi}{\textnormal{(\theenumi)}$\;\;$}
\renewcommand{\theenumi}{\roman{enumi}}
\renewcommand{\labelenumi}{\textnormal{(\theenumi)}$\;\;$}

          %%%%% ~~~~~~~~~~~~~~~~~~~~ %%%%%

\begin{itemize}
\item
	In the Standard Model of particle physics, ``particles'' are defined
	phenomenologically, in the sense that a ``particle'' is defined/identified/classified
	via its observed values of a number of characteristics,
	e.g. mass, charge, spin, colour, etc.
\item
	Each particle characteristic allow only a finite number of admissible values.
	These characteristics (hence the ``particles'' they phenomenologically define)
	are described mathematically as sections of certain vector bundles.
	
	Each admissible value of the characteristic corresponds
	to a $1$-dimensional subspace in the fibre.
	
	The laws of physics must be invariant under relabelling of the admissible values,
	which entails that the aforementioned vector bundles come with an action by a certain
	symmetry group $G$ that permutes the admissible values.
	In other words, these vector bundles are associated vector bundles of a
	principal fibre bundle with structure group $G$, induced by a certain representation
	of $G$ on the fibre space.
	
	These symmetries are called \textbf{gauge symmetries}.
\item
	The evolution in time of the ``particles'' is dictated by the curvature on the ambient vector bundle.
	In return, the presence of the particles affects the curvature on the ambient vector bundles.
	The dynamical interplay between the particle and curvature is determined by the principle
	of least action, i.e. the equation of motion is the Euler-Lagrange equations of a suitable
	action functional.
	
	The Lagrangian density (integrand of the action functional) must be
	Lorentz-invariant and gauge-invariant.
\item
	Creation and annihilation of particles are modelled by quantizing
	the Euler-Lagrange equations. 
\item
	Masses of fermions and the weak interaction bosons ($W^{\pm}$ and $Z^{0}$)
	are introduced by the Higgs mechanism.
	
	The non-zero ground state of the Higgs field ``breaks'' the fully gauge symmetry;
	after symmetry breaking, physics now ``happens'' on a sub-bundle of the original full bundle,
	with structure group $H \subset G$ being a subgroup of the original full unbroken gauge group $G$.
	Such an $H$ is the stabilizer subgroup of a certain vacuum element (a minimum of the Higgs potential)
	in the original fibre.
	
	``Mass'' terms for fermions and certain gauge bosons may appear
	when the original Lagrangian density is restricted to the reduced bundle,
	and re-expressed in terms intrinsic to the reduced bundle.
\end{itemize}

          %%%%% ~~~~~~~~~~~~~~~~~~~~ %%%%%

