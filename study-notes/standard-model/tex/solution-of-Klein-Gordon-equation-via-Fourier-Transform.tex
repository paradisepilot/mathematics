
          %%%%% ~~~~~~~~~~~~~~~~~~~~ %%%%%

\chapter{General solution of the Klein-Gordon equation via the Fourier Transform}
\setcounter{theorem}{0}
\setcounter{equation}{0}

%\cite{vanDerVaart1996}
%\cite{Kosorok2008}

%\renewcommand{\theenumi}{\alph{enumi}}
%\renewcommand{\labelenumi}{\textnormal{(\theenumi)}$\;\;$}
\renewcommand{\theenumi}{\roman{enumi}}
\renewcommand{\labelenumi}{\textnormal{(\theenumi)}$\;\;$}

          %%%%% ~~~~~~~~~~~~~~~~~~~~ %%%%%

\section{Solving linear PDE with constant coefficients via the Fourier transform}

We follow Section 4.3.1, p.177--186, \cite{Evans2010}.

\begin{definition}[Fourier and inverse Fourier transforms of $u \in L^{{\color{red}1}}(\Re^{n},\C)$]
\mbox{}
\vskip 0.1cm
\noindent
The \textbf{Fourier transform} $\widehat{u}$ of $u \in L^{1}(\Re^{n},\C)$ is, by definition,
\begin{equation*}
\widehat{u}(y)
\;\; := \;\;
	\dfrac{1}{(2\pi)^{n/2}}\,
	\int_{\Re^{n}}
		\exp\!\left({\color{red}-}\,\sqrt{-1}\, y \overset{{\color{white}.}}{\bullet} x\right) \cdot u(x)
		\;\d x\,,
\quad
\textnormal{for each \,$y \in \Re^{n}$}
\end{equation*}
Similarly, the \textbf{inverse Fourier transform} $\widecheck{u}$ of $u$ is, by definition,
\begin{equation*}
\widecheck{u}(x)
\;\; := \;\;
	\dfrac{1}{(2\pi)^{n/2}}\,
	\int_{\Re^{n}}
		\exp\!\left({\color{red}+}\,\sqrt{-1}\, x \overset{{\color{white}.}}{\bullet} y\right) \cdot u(y)
		\;\d y\,,
\quad
\textnormal{for each \,$x \in \Re^{n}$}
\end{equation*}
\end{definition}
Note that $u \in L^{1}(\Re^{n},\C)$ \,$\Longrightarrow$\, $\widehat{u}$, $\widecheck{u}$ $\in$ $L^{1}(\Re^{n},\C)$,
since $\vert\,\exp(\pm\sqrt{-1})\,x \bullet y\,\vert \equiv 1$.

\vskip 0.5cm
\begin{theorem}[Plancherel's Theorem]
\mbox{}
\vskip 0.1cm
\noindent
Suppose \,$u \,\in\, L^{1}(\Re^{n},\C) \,\bigcap\, L^{2}(\Re^{2},\C)$.\,
Then,
\begin{enumerate}
\item
	$\widehat{u}$, $\widecheck{u}$ $\in$ $L^{2}(\Re^{n}),\C$, and
\item
	\begin{equation*}
	\left\Vert\;\overset{{\color{white}-}}{u}\;\right\Vert_{L^{2}}
	\;\; = \;\;
		\left\Vert\;\overset{{\color{white}.}}{\widehat{u}}\;\right\Vert_{L^{2}}
	\;\; = \;\;
		\left\Vert\;\overset{{\color{white}.}}{\widecheck{u}}\;\right\Vert_{L^{2}}
	\end{equation*}
\end{enumerate}
\end{theorem}

\vskip 0.5cm
\begin{definition}[Fourier and inverse Fourier transforms of $u \in L^{{\color{red}2}}(\Re^{n},\C)$]
\mbox{}
\vskip 0.1cm
\noindent
The \textbf{Fourier transform} $\widehat{u} \in L^{2}(\Re^{n},\C)$ of $u \in L^{2}(\Re^{n},\C)$ is, by definition,
the unique element $\widehat{u} \in L^{2}(\Re^{n},\C)$ such that
\begin{equation*}
\underset{k \rightarrow\infty}{\lim} \left\Vert\; \widehat{u} \,-\, \widehat{u}_{k} \;\right\Vert_{L^{2}}  \;\; := \;\;0\,,
\end{equation*}
where $\{\,u_{k}\,\}_{k=1}^{\infty} \subset L^{1}(\Re^{n},\C) \,\bigcap\, L^{2}(\Re^{n},\C)$
is an arbitrary sequence such that
\begin{equation*}
\underset{k \rightarrow\infty}{\lim} \left\Vert\; u \,-\, u_{k} \;\right\Vert_{L^{2}}  \;\; := \;\;0\,,
\end{equation*}
The \textbf{inverse Fourier transform} $\widecheck{u} \in L^{2}(\Re^{n},\C)$ of $u \in L^{2}(\Re^{n},\C)$ can be defined analogously.
\end{definition}

\vskip 0.5cm
\begin{theorem}[Properties of the Fourier transform]
\label{ThmPropertiesFourierTransform}
\mbox{}
\vskip 0.1cm
\noindent
Suppose \,$u \,\in\, L^{2}(\Re^{2},\C)$.\,
Then, the following statements are true:
\begin{enumerate}
\item
	The Fourier transform preserves the $L^{2}$ inner product, i.e.,
	\begin{equation*}
	\int_{\Re}\, u(x) \cdot \overline{v(x)} \;\d x
	\;\; = \;\;
		\int_{\Re}\, \widehat{u}(y) \cdot \overline{\widehat{v}(x)} \;\d y
	\end{equation*}
\item
	For each multiindex $\alpha$ such that $D^{\alpha}u \in L^{2}(\Re^{n},\C)$, we have
	\begin{equation*}
	\left(\,D^{\alpha}u\,\right)^{\wedge}
	\;\; = \;\;
		\left(\,\sqrt{-1}\,y\,\right)^{\alpha}\cdot \widehat{u}
	\end{equation*}
\item
	For $u$, $v$ $\in$ $L^{1}(\Re^{n},\C) \bigcap L^{2}(\Re^{n},\C)$, we have
	\begin{equation*}
	\left(\,u * v\,\right)^{\wedge}
	\;\; = \;\;
		(2\pi)^{n/2} \cdot \widehat{u} \cdot \widehat{v}
	\end{equation*}
\item
	The Fourier transform and inverse Fourier transform are inverses of each other, i.e.,
	\begin{equation*}
	u \;\; = \;\; \left(\;\overset{{\color{white}.}}{\widehat{u}}\;\right)^{\vee}
	\end{equation*}
	More explicitly,
	\begin{equation*}
	u(x)
	\;\; = \;\;
		(\widehat{u})^{\vee}(x)
	\;\; = \;\;
		\dfrac{1}{(2\pi)^{n/2}}\,
		\int_{\Re^{n}}
			\exp\!\left({\color{red}+}\,\sqrt{-1}\, x \overset{{\color{white}.}}{\bullet} y\right) \cdot \widehat{u}(y)
			\;\d y\,,
	\quad
	\textnormal{for each \,$x \in \Re^{n}$}
	\end{equation*}
\end{enumerate}
\end{theorem}

          %%%%% ~~~~~~~~~~~~~~~~~~~~ %%%%%

\vskip0.5cm
\noindent
\textbf{Implications of Theorem \ref{ThmPropertiesFourierTransform} for solving linear constant-coefficient PDEs via the Fourier transform:}
\begin{itemize}
\item
	By Theorem \ref{ThmPropertiesFourierTransform}(ii), applying the Fourier transform
	to both sides of a linear constant-coefficient PDE yields an algebraic equation
	involving the Fourier transform $\widehat{\phi}$ of the unknown function $\phi$
	(desired solution to the given PDE).
\item
	If this resulting algebraic equation can be solved for $\widehat{\phi}$,
	then solving the original PDF for $\phi$ is reduced to computing
	inverse Fourier transform of $\widehat{\phi}$.
\end{itemize}

          %%%%% ~~~~~~~~~~~~~~~~~~~~ %%%%%

\vskip 1.0cm
\section{Lorentz-invariant measure on the positive-energy mass shell}

In the section, we follow Section 4.4, p.111--115, \cite{Talagrand2022}.
For $m \geq 0$, the positive-energy mass shell $X_{m}$ is the subset
of the energy-momentum Minkowski space defined as follows:
\begin{equation*}
X_{m}
\;\; := \;\;
	\left\{\;\,
		\overset{{\color{white}-}}{p} = (p_{0},p_{1},p_{2},p_{3}) \in \Re^{1,3}
		\,\;\left\vert\;
		\begin{array}{c}
			{\color{red}p_{0} \,\geq\, 0}
			\\
			\overset{{\color{white}-}}{p_{0}} \,=\, p_{1}^{2} + p_{2}^{2} + p_{3}^{2} + m^{2}
			\end{array}
		\right.
		\right\}
\end{equation*}
First, we make a few remarks:
\begin{itemize}
\item
	For $m > 0$, $X_{m}$ is the $\textnormal{SO}^{\uparrow}(1,3)$ orbit of $(m,0,0,0)$.
\item
	For $m = 0$, $X_{0} \backslash \{\,0\,\}$ is the $\textnormal{SO}^{\uparrow}(1,3)$ orbit of $(1,0,0,1)$.
\end{itemize}
We will show that the following is a Lorentz-invariant measure on $X_{m}$:
\begin{equation*}
\int_{X_{m}}\; f(E_{m,\mathbf{p}},\mathbf{p})\;\d\lambda_{m}
\;\; := \;\;
	\dfrac{1}{(2\pi)^{3}}\,\int f(E_{m,\mathbf{p}},\mathbf{p}) \; \dfrac{\d^{3}\mathbf{p}}{2\,\omega_{p}}\,,
\end{equation*}
where \,$E_{m,\mathbf{p}} \,:=\, +\,\sqrt{m^{2}+\mathbf{p}^{2}}$.

\begin{lemma}
\mbox{}
\vskip 0.1cm
\noindent
For $m \geq 0$ and $\varepsilon > 0$, we define:
\begin{equation*}
X_{m,\varepsilon}
\;\; := \;\;
	\left\{\;\,
		\overset{{\color{white}-}}{p} = (p_{0},p_{1},p_{2},p_{3}) \in \Re^{1,3}
		\,\;\left\vert\;
		\begin{array}{c}
			{\color{red}p_{0} \,\geq\, 0}
			\\
			\overset{{\color{white}-}}{m^{2}} \,<\, p_{0}^{2} - p_{1}^{2} - p_{2}^{2} - p_{3}^{2} \,<\, m^{2} + \varepsilon
			\end{array}
		\right.
		\right\}
\end{equation*}
Then, for continuous $f : \Re^{1,3} \longrightarrow \C$, the following limit:
\begin{equation*}
\underset{{\color{white}..}\varepsilon\rightarrow 0^{+}}{\lim}\;
\dfrac{1}{\varepsilon}
\cdot
\int_{X_{m,\varepsilon}} f(p_{0},p_{1},p_{2},p_{3}) \;\d p_{0}\,\d p_{1}\,\d p_{2}\,\d p_{3} 
\end{equation*}
exists and equals
\begin{equation*}
\underset{{\color{white}..}\varepsilon\rightarrow 0^{+}}{\lim}\,
\dfrac{1}{\varepsilon}
\cdot
\int_{X_{m,\varepsilon}} f(p_{0},p_{1},p_{2},p_{3}) \;\d p_{0}\,\d p_{1}\,\d p_{2}\,\d p_{3}
\;\; = \;\;
	\int_{\Re^{3}}\;
		f(E_{m,\mathbf{p}},p_{1},p_{2},p_{3})
	\;\dfrac{
		\d p_{1}\,\d p_{2}\,\d p_{3} 
		}{
		2 E_{m,\mathbf{p}}
		}
\end{equation*}
\end{lemma}
\proof
Let $h_{\mathbf{p}}(x) := \sqrt{x + \mathbf{p}^{2}}$.
Then, $h^{\prime}_{\mathbf{p}}(x) = \dfrac{1}{2}\left(x + \mathbf{p}^{2}\right)^{-1/2}$.
So, $h^{\prime}_{\mathbf{p}}(m^{2})$ $=$ $\dfrac{1}{2\sqrt{m^{2}+\mathbf{p}^{2}}}$
$=$ $\dfrac{1}{2 E_{m,\mathbf{p}}}$.
Hence,
\begin{equation*}
\sqrt{m^{2}+\mathbf{p}^{2}+\varepsilon}
\;\; = \;\;
	h_{\mathbf{p}}(m^{2}+\varepsilon)
\;\; \approx \;\;
	h_{\mathbf{p}}(m^{2})
	\;+\;
	\varepsilon\,h^{\prime}_{\mathbf{p}}(m^{2})
\;\; \approx \;\;
	E_{m,\mathbf{p}}
	\;+\;
	\dfrac{\varepsilon}{2 E_{m,\mathbf{p}}}
\end{equation*}
Thus, to first order in $\varepsilon$, we have the following approximation:
\begin{eqnarray*}
\int_{X_{m,\varepsilon}} f(p_{0},p_{1},p_{2},p_{3}) \;\d p_{0}\,\d p_{1}\,\d p_{2}\,\d p_{3} 
& = &
	\int_{\Re^{3}}
	\left(\,
		\int_{\sqrt{m^{2}+\mathbf{p}^{2}}}^{\sqrt{m^{2}+\mathbf{p}^{2}+\varepsilon}}
		f(p_{0},p_{1},p_{2},p_{3})
		\;\d p_{0}
		\right)
	\d p_{1}\,\d p_{2}\,\d p_{3} 
\\
& \approx &
	\int_{\Re^{3}}
	\left(\,
		\int_{E_{m,\mathbf{p}}}^{\,E_{m,\mathbf{p}}+\varepsilon/2E_{m,\mathbf{p}}}
		f({\color{red}E_{m,\mathbf{p}}},p_{1},p_{2},p_{3})
		\;\d p_{0}
		\right)
	\d p_{1}\,\d p_{2}\,\d p_{3} 
\\
& = &
	\int_{\Re^{3}}
	\left(\,
		\dfrac{\varepsilon}{2 E_{m,\mathbf{p}}}
		\cdot
		f(E_{m,\mathbf{p}},p_{1},p_{2},p_{3})
		\right)
	\d p_{1}\,\d p_{2}\,\d p_{3} 
\end{eqnarray*}
The above arguments show that the following limit:
\begin{equation*}
\underset{{\color{white}..}\varepsilon\rightarrow 0^{+}}{\lim}\;
\dfrac{1}{\varepsilon}
\cdot
\int_{X_{m,\varepsilon}} f(p_{0},p_{1},p_{2},p_{3}) \;\d p_{0}\,\d p_{1}\,\d p_{2}\,\d p_{3} 
\end{equation*}
exists and equals
\begin{equation*}
\underset{{\color{white}..}\varepsilon\rightarrow 0^{+}}{\lim}\;
\dfrac{1}{\varepsilon}
\cdot
\int_{X_{m,\varepsilon}} f(p_{0},p_{1},p_{2},p_{3}) \;\d p_{0}\,\d p_{1}\,\d p_{2}\,\d p_{3}
\;\; = \;\;
	\int_{\Re^{3}}\;
		f(E_{m,\mathbf{p}},p_{1},p_{2},p_{3})
	\;\dfrac{
		\d p_{1}\,\d p_{2}\,\d p_{3} 
		}{
		2 E_{m,\mathbf{p}}
		}
\end{equation*}
\qed

          %%%%% ~~~~~~~~~~~~~~~~~~~~ %%%%%

\vskip 1.0cm
\section{The Klein-Gordon equation}

\noindent
The Klein-Gordon equation is the following equation
for \,$\phi : \Re^{(+,-,-,-)} \longrightarrow \C$:
\begin{equation*}
\dfrac{\partial^{2}\phi}{\partial x_{0}^{2}}
\,-\,
\dfrac{\partial^{2}\phi}{\partial x_{1}^{2}}
\,-\,
\dfrac{\partial^{2}\phi}{\partial x_{2}^{2}}
\,-\,
\dfrac{\partial^{2}\phi}{\partial x_{3}^{2}}
\,+\,
m^{2}\phi
\; = \;
0
\end{equation*}
We write the Fourier transform \,$\widehat{\phi}$\, of \,$\phi$\, as follows:
\begin{equation*}
\widehat{\phi}(p)
\;\; := \;\;
	\dfrac{1}{(2\pi)^{n/2}}\,
	\int_{\Re^{1,3}}
		\exp\!\left({\color{red}-}\,\sqrt{-1}\, p \overset{{\color{white}.}}{\bullet} x\right) \cdot \phi(x)
		\;\d x\,,
\quad
\textnormal{for each \,$p = (p_{0},p_{1},p_{2},p_{3}) \in \Re^{1,3}$}
\end{equation*}
Applying the Fourier transform to the Klein-Gordon equation yields
the following algebraic equation for $\widehat{\phi}$:
\begin{equation*}
\left(\;
	-\,
	p_{0}^{2}
	\,+\,
	p_{1}^{2}
	\,+\,
	p_{2}^{2}
	\,+\,
	p_{3}^{2}
	\,+\,
	m^{2}
	\;\right)
	\cdot\,\widehat{\phi}
\; = \;
	0\,,
\end{equation*}
which immediately implies that \,$\widehat{\phi} \in L^{2}(\Re^{n},\C)$\, is given by:
\begin{equation*}
\widehat{\phi}(p_{0},p_{1},p_{2},p_{3})
\;\; = \;\;
	\left\{\begin{array}{cl}
		0, & \textnormal{for \,$p_{0}^{2} \,\neq\, p_{1}^{2} + p_{2}^{2} + p_{3}^{2} + m^{2}$}\,,
		\\
		\overset{{\color{white}1}}{\textnormal{arbitrary value}}, & \textnormal{otherwise (i.e., $p_{0}^{2} \,=\, p_{1}^{2} + p_{2}^{2} + p_{3}^{2} + m^{2}$)}
		\end{array}\right.
\end{equation*}
Thus, the most general form of the solution to the Klein-Gordon can be given as follows:
\begin{equation*}
\phi(x)
\;\; = \;\;
	\dfrac{1}{(2\pi)^{2}}\,
	\int_{\Re^{1,3}}
		\exp\!\left({\color{red}+}\,\sqrt{-1}\, x \overset{{\color{white}.}}{\bullet} p\right)
		\cdot
		a(p)
		\cdot
		\delta(-p_{0}^{2} + p_{1}^{2} + p_{2}^{2} + p_{3}^{2} + m^{2})
		\;\d p\,,
\quad
\textnormal{for each \,$x \in \Re^{n}$}\,,
\end{equation*}
where
\,$\delta(\,\cdot\,)$\, is the Dirac delta functional on \,$\Re^{1,3}$,\, and
\,$a(p) = a(p_{0},p_{1},p_{2},p_{3})$\, is an arbitrary measurable $\C$-valued function.
We note that the above solution has support on both the positive-energy and negative energy mass shells.
Using the Lorentz-invariance measure on the positive-energy mass shell $X_{m}$ described in the preceding section,
we can express the general solution of the Klein-Gordon equation as follows:
\begin{equation*}
\phi(x)
\; = \;
	\dfrac{1}{(2\pi)^{3}}\,
	\int_{\Re^{3}} \left(
		e^{-\,\overset{{\color{white}.}}{\sqrt{-1}}\,
			(x_{0} \cdot E_{m,\mathbf{p}} - \mathbf{x} \bullet \mathbf{p})
			}
		\cdot
		a(\mathbf{p})
		\; + \;
		e^{+\,\overset{{\color{white}.}}{\sqrt{-1}}\,
			(x_{0} \cdot E_{m,\mathbf{p}} - \mathbf{x} \bullet \mathbf{p})
			}
		\cdot
		b^{*}(\mathbf{p})
		\right)
		\dfrac{\d^{3}\mathbf{p}}{2 E_{m,\mathbf{p}}}\,,
\quad
\end{equation*}
for each \,$x \in \Re^{1,3}$,
where
\,$a(\mathbf{p}) = a(p_{1},p_{2},p_{3})$\,
and
\,$b(\mathbf{p}) = b(p_{1},p_{2},p_{3})$\,
are arbitrary measurable $\C$-valued functions.

          %%%%% ~~~~~~~~~~~~~~~~~~~~ %%%%%
