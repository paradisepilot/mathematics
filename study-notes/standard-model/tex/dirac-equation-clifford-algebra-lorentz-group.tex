
          %%%%% ~~~~~~~~~~~~~~~~~~~~ %%%%%

\chapter{From the Dirac equation, to the Clifford algebra, to the Lorentz group}
\setcounter{theorem}{0}
\setcounter{equation}{0}

%\cite{vanDerVaart1996}
%\cite{Kosorok2008}

%\renewcommand{\theenumi}{\alph{enumi}}
%\renewcommand{\labelenumi}{\textnormal{(\theenumi)}$\;\;$}
\renewcommand{\theenumi}{\roman{enumi}}
\renewcommand{\labelenumi}{\textnormal{(\theenumi)}$\;\;$}

          %%%%% ~~~~~~~~~~~~~~~~~~~~ %%%%%

\section{The Schr\"odinger equation}

\begin{itemize}
\item
	Newtonian Energy-Momentum Relation for a {\color{red}free} particle (no force acting on it; zero potential):
	\begin{equation}
	\label{NewtonionEnergyMomentumRelation}
	E \;=\; \dfrac{\Vert\,\mathbf{p}\,\Vert^{2}}{2m}
	\quad\;\;
	\Longleftrightarrow
	\quad\;\;
	E \,-\, \dfrac{1}{2m}\left(\,p_{1}^{2} \overset{{\color{white}.}}{+} p_{2}^{2} + p_{3}^{2}\,\right) \;=\; 0
	\end{equation}
	{\tiny
	\begin{equation*}
	\textnormal{Proof:}
	\quad\quad\quad\quad\quad\quad
	E
	\,:=\,
		\left(\!\begin{array}{c} \textnormal{total energy} \\ \textnormal{of the} \\ \textnormal{free particle} \end{array}\!\right)
	\,=\,
		\left(\!\begin{array}{c} \textnormal{kinetic energy} \\ \textnormal{of the} \\ \textnormal{particle} \end{array}\!\right)
	\,=\,
		\dfrac{1}{2}\,m\,\Vert\,\mathbf{v}\,\Vert^{2}
	\,=\,
		\dfrac{(\,\Vert\,m\cdot\mathbf{v}\,\Vert\,)^{2}}{2m}
	\,=\,
		\dfrac{\Vert\,\mathbf{p}\,\Vert^{2}}{2m}
	\quad\quad\quad\quad\quad\quad\quad\quad\quad\quad
	{\color{white}.}
	\end{equation*}
	}
	\vskip 1.0px
\item
	\textbf{First quantization:}\;\; Replacing
	\,$E$, $p_{1}$, $p_{2}$, $p_{3}$\,
	respectively with differential operators:
	\begin{equation}
	\label{FirstQuantization}
	E \;\longmapsto\; \i\,\dfrac{\partial}{\partial t}\,,
	\quad\quad
	p_{j} \;\longmapsto\; -\,\i\,\dfrac{\partial}{\partial x_{j}}\,,
	\quad\quad
	\textnormal{and}
	\end{equation}
	letting the resulting differential operator act on the wave function
	\,$\psi(t,x_{1},x_{2},x_{3})$\,
	yields the Schr\"odinger equation for a free particle:
	\begin{equation}
	\label{equationSchrodinger}
	\left(\;\i\,\dfrac{\partial}{\partial t} + \dfrac{1}{2m}\!
		\left(
			\dfrac{\partial^{2}}{\partial x_{1}^{2}}
			+ \dfrac{\partial^{2}}{\partial x_{2}^{2}}
			+ \dfrac{\partial^{2}}{\partial x_{3}^{2}}
			\right)
		\right)
	\psi
	\,=\,
		0
	\end{equation}
\item
	Schr\"odinger equation is NOT Lorentz-invariant (e.g., asymmetry between time and space) -- incompatible with Special Relativity.
\end{itemize}

          %%%%% ~~~~~~~~~~~~~~~~~~~~ %%%%%

\section{The Klein-Gordon equation}

\begin{itemize}
\item
	Special-relativistic Energy-Momentum Relation for a {\color{red}free} particle (see \S3.4, \cite{DeFariaDeMelo2010}):
	\begin{equation}
	\label{RelativisticEnergyMomentumRelation}
	E^{2} \;=\; p^{2} \,+\, m^{2}
	\end{equation}
	\vskip 1.0px
\item
	First quantization on \eqref{RelativisticEnergyMomentumRelation}:
	\begin{equation*}
	E \;\longmapsto\; \i\,\dfrac{\partial}{\partial t}\,,
	\quad\quad
	p_{j} \;\longmapsto\; -\,\i\,\dfrac{\partial}{\partial x_{j}}
	\end{equation*}
	now yields the Klein-Gordon equation for a free particle:
	\begin{equation}
	\label{equationKleinGordon}
	\left(\; \textnormal{\large$\Box$} \,+\, m^{2} \,\right)\psi
	\;=\,
	\left(\; \partial_{\mu}\partial^{\mu} \,+\, m^{2} \,\right)\psi
	\;=\,
	\left(\,\dfrac{\partial^{2}}{\partial t^{2}}
	- \left(
		\dfrac{\partial^{2}}{\partial x_{1}^{2}}
		+ \dfrac{\partial^{2}}{\partial x_{2}^{2}}
		+ \dfrac{\partial^{2}}{\partial x_{3}^{2}}
		\right)
	\,+\,
		m^{2}
		\,\right)
	\psi
	\;=\;
		0
	\end{equation}
\item
	The Klein-Gordon equation is indeed Lorentz-invariant but:
	\vskip 1.5px
	\begin{itemize}\itemindent=-10px
	{\scriptsize
	\item
		it permits negative-energy eigenstates; see Example 6.1, \cite{LancasterBlundell2014}
	\item
		 probabilistic interpretation of (squared modulus of)
		 Schr\"odinger's wave function is lost; see \S6.2, \cite{LancasterBlundell2014}
	}
	\end{itemize}
	\vskip 5.0px
\item
	The Dirac equation addresses the second problem (i.e., restores dynamical/probabilistic interpretation).
\end{itemize}

          %%%%% ~~~~~~~~~~~~~~~~~~~~ %%%%%

\section{The Dirac equation}

\begin{itemize}
\item
	Dirac sought a new equation of motion by seeking a first-order linear differential operator
	\begin{equation*}
	D
	\;=\;
		\i\,\gamma^{\mu}\partial_{\mu}
	\;=\;
		\i\left(\,
			\gamma^{0}\dfrac{\partial}{\partial x_{0}} 
			\,+\, \gamma^{1}\dfrac{\partial}{\partial x_{1}} 
			\,+\, \gamma^{2}\dfrac{\partial}{\partial x_{2}} 
			\,+\, \gamma^{3}\dfrac{\partial}{\partial x_{3}} 
			\,\right)
	\end{equation*}
	whose square is {\color{red}minus} the Minkowskian d'Alembertian
	\;$\textnormal{\large$\Box$}
	%\;=\;
	%	\dfrac{\partial^{2}}{\partial t^{2}} \,-\, \Delta
	\;=\;
		\dfrac{\partial^{2}}{\partial t^{2}}
		\,-\, \dfrac{\partial^{2}}{\partial x_{1}^{2}}
		\,-\, \dfrac{\partial^{2}}{\partial x_{2}^{2}}
		\,-\, \dfrac{\partial^{2}}{\partial x_{3}^{2}}
	$.\,
	\vskip 5px
\item
	Dirac arrived at the famous Dirac equation:
	\begin{equation}
	\label{TheDiracEquation}
	\left(\,\i\cdot\gamma^{\mu}\partial_{\mu} - m\,\right)\psi \, = \, 0
	\end{equation}
	\vskip -6.0px
	The requirement
	\,$
	-\,g^{\mu\nu}\partial_{\mu}\partial_{\nu}
	\,=\,
		{\color{red}-\,\textnormal{\large$\Box$}}
	\;{\color{red}=}\;
		{\color{red}\left(\,\i\cdot\gamma^{\mu}\partial_{\mu}\,\right)^{2}}
	%\,=\,
	%	\left(\,\i\cdot\gamma^{\mu}\partial_{\mu}\,\right)
	%	\left(\,\i\cdot\gamma^{\nu}\partial_{\nu}\,\right)
	\,=\,
		-\,\gamma^{\mu}\gamma^{\nu}\partial_{\mu}\partial_{\nu}
	\,=\,
		-\left(\dfrac{\gamma^{\mu}\gamma^{\nu}+\gamma^{\nu}\gamma^{\mu}}{2}\right)\partial_{\mu}\partial_{\nu}
		$\,
	implies that \,$\gamma^{\mu}$\, must satisfy:
	\begin{equation}
	\label{GammaMatricesSatisfyCliffordRelation}
	\left\{\,\gamma^{\mu}\,,\,\gamma^{\nu}\,\right\}
	\;=\;
		2\cdot g^{\mu\nu}\,,
	\end{equation}
	where
	\,$\left\{\,\gamma^{\mu}\,,\,\gamma^{\nu}\,\right\} \,:=\, \gamma^{\mu}\gamma^{\nu} + \gamma^{\nu}\gamma^{\mu}$\,
	and
	\,$g^{\mu\nu} \,=\, \diag(1,-1,-1,-1)$.
	\vskip 12px
\item
	{\color{white}.}
	\vskip -22px
	\begin{equation*}
	\begin{array}{c}\textnormal{Dirac} \\ \textnormal{equation} \end{array}
	\Longleftrightarrow\;\;\,
	\i\,\gamma^{\mu}\partial_{\mu}\psi = m\,\psi
	\;\;\Longrightarrow\;\;
	%\i\,\gamma^{\mu}\partial_{\mu}\left(\i\,\gamma^{\mu}\partial_{\mu}\overset{{\color{white}$.$}{\psi}\right) = m\,\i\,\gamma^{\mu}\partial_{\mu}\psi
	\underset{-\,\Box\psi}{\underbrace{{\color{white}.}{\color{red}\i\,\gamma^{\mu}\partial_{\mu}}\!\left(
		\i\,\gamma^{\nu}\partial_{\nu}\overset{{\color{white}.}}{\psi}
		\right)}}
		= m\cdot\underset{m\,\psi}{\underbrace{{\color{red}\i\,\gamma^{\mu}\partial_{\mu}}\overset{{\color{white}.}}{\psi}}}
	\;\;\,\Longrightarrow
	\begin{array}{c}\textnormal{Klein-Gordon} \\ \textnormal{equation} \end{array}
	\end{equation*}
\end{itemize}

\vskip 0.5cm
\begin{itemize}
\item
	Condition \eqref{GammaMatricesSatisfyCliffordRelation} cannot be satisfied by complex numbers:
	\begin{equation*}
	\left.\begin{array}{c}
		{\color{red}\left\{\,\gamma^{\mu}\,,\,\gamma^{\nu}\,\right\} \;=\;2\cdot g^{\mu\nu}}
		\\
		\overset{{\color{white}1}}{\gamma_{\mu}}\;\in\;\C,\;\;\forall\;\mu = 0,1,2,3
	\end{array}\right\}
	\quad\Longrightarrow\quad
	\textnormal{contradiction}
	\end{equation*}
	But, Dirac observed that Condition \eqref{GammaMatricesSatisfyCliffordRelation}
	could indeed be satisfied by a set of {\color{red}$4 \times 4$} complex matrices:
	\vskip -10px
	
\begin{equation*}
\gamma^{0}
\,=\,
	\textnormal{\tiny$\left(\!\!\begin{array}{rrrr}
		0 & 0 & {\color{white}-}1 & 0 \\
		0 & 0 & 0 & {\color{white}-}1 \\
		{\color{white}-}1 & 0 & 0 & 0 \\
		0 & {\color{white}-}1 & 0 & 0
		\end{array}\!\right)$}
\,=\,
	\textnormal{\tiny$\left(\!\begin{array}{cc}
		0 & {\color{white}.}I_{2} \\
		{\color{white}-}I_{2} & \overset{{\color{white}.}}{0} \\
		\end{array}\!\!\right)$},
\quad\quad
\gamma^{1}
\,=\,
	\textnormal{\tiny$\left(\!\begin{array}{rrrr}
		0 & 0 & 0 & {\color{white}-}1 \\
		0 & 0 & {\color{white}-}1 & 0 \\
		0 & -1 & 0 & 0 \\
		-1 & 0 & 0 & 0
		\end{array}\right)$}
\,=\,
	\textnormal{\tiny$\left(\!\begin{array}{cc}
		0 & \sigma_{1} \\
		-\sigma_{1} & \overset{{\color{white}.}}{0} \\
		\end{array}\!\!\right)$}
\end{equation*}
\begin{equation*}
\gamma^{2}
\,=\,
	\textnormal{\tiny$\left(\!\begin{array}{rrrr}
		0 & 0 & 0 & -\i \\
		0 & 0 & {\color{white}-}\i & 0 \\
		0 & {\color{white}-}\i & 0 & 0 \\
		-\i & 0 & 0 & 0
		\end{array}\right)$}
\,=\,
	\textnormal{\tiny$\left(\!\begin{array}{cc}
		0 & \sigma_{2} \\
		-\sigma_{2} & \overset{{\color{white}.}}{0} \\
		\end{array}\!\!\right)$},
\quad\quad
\gamma^{3}
\,=\,
	\textnormal{\tiny$\left(\!\begin{array}{rrrr}
		0 & 0 & {\color{white}-}1 & 0\\
		0 & 0 & 0 & -1 \\
		{\color{white}-}1 & 0 & 0 & 0 \\
		0 & -1 & 0 & 0
		\end{array}\right)$}
\,=\,
	\textnormal{\tiny$\left(\!\begin{array}{cc}
		0 & \sigma_{3} \\
		-\sigma_{3} & \overset{{\color{white}.}}{0} \\
		\end{array}\!\!\right)$}
\end{equation*}

	\vskip 5px
\item
	So, Dirac sought solutions \,$\psi \in C^{\infty}(\mathcal{M},{\color{red}\C^{4}})$\,
	for the Dirac equation: \,$\left(\,\i\,{\color{red}\gamma^{\mu}}\partial_{\mu} \overset{{\color{white}.}}{-} m\right)\psi = 0$.\,
%\item
%	This observation led to the following realization (when expressed in modern geometric language):
%	The ``wave function'' \,$\psi$\, must be smooth sections of a spinor bundle over spacetime.
\end{itemize}

          %%%%% ~~~~~~~~~~~~~~~~~~~~ %%%%%


          %%%%% ~~~~~~~~~~~~~~~~~~~~ %%%%%

