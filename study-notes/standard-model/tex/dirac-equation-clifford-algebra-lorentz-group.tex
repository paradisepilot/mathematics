
          %%%%% ~~~~~~~~~~~~~~~~~~~~ %%%%%

\chapter{From the Dirac equation, to the Clifford algebra, to the Lorentz group}
\setcounter{theorem}{0}
\setcounter{equation}{0}

%\cite{vanDerVaart1996}
%\cite{Kosorok2008}

%\renewcommand{\theenumi}{\alph{enumi}}
%\renewcommand{\labelenumi}{\textnormal{(\theenumi)}$\;\;$}
\renewcommand{\theenumi}{\roman{enumi}}
\renewcommand{\labelenumi}{\textnormal{(\theenumi)}$\;\;$}

          %%%%% ~~~~~~~~~~~~~~~~~~~~ %%%%%

\section{The {\color{red}appearance of \,$\sqrt{-1}$}\, in the Schr\"odinger equation -- the Stone theorem}

\textbf{Summary}
\vskip 0.1cm
\noindent
If we assume:
\begin{itemize}
\item
	the state \,$\psi(t) \in \mathbb{H}$\, at time \,$t \in \Re$\, of a quantum system
	is a {\color{red}unit vector} in a complex Hilbert space \,$\mathbb{H}$,\,
\item
	the evolution of \,$\psi(t)$\, with respect to time \,$t \in \Re$\,
	is {\color{red}induced by a one-parameter continuous group}
	of operators \,$U(t)$\, on \,$\mathbb{H}$
    (note that $U(t)$ is assumed to be independent of $\psi(t)$);\,
	more precisely, for each \,$t \in \Re$,\, there exists a (unitary) operator
	\,$U(t)$\, on \,$\mathbb{H}$\, such that,
	for each each admissible time evolution \,$\psi(t)$ of the given quantum system,
	we have:
	\begin{equation*}
	\psi(t) \; = \; U(t) \cdot \psi(0)\,,
	\quad
	\textnormal{for each \,$t \in \Re$}
	\end{equation*}
	We note that the operator \,$U(t)$\, is necessarily unitary since each \,$\psi(t)$\,
	is (by hypothesis) a unit vector in \,$\mathbb{H}$\, and
	\,$U(t)$\, needs to preserve the norms of the \,$\psi(t)$:\,
	\begin{equation*}
	\left\langle\,U(t)\cdot\psi(0)
		\,\overset{{\color{white}\textnormal{\large1}}}{,}\,
		U(t)\cdot\psi(0)\,\right\rangle
	\; = \;
		\left\langle\,\psi(t)
			\,\overset{{\color{white}\textnormal{\large1}}}{,}\,
			\psi(t)\,\right\rangle
	\; = \;
		1\,,
	\quad
	\textnormal{for each \,$t \in \Re$\, and each \,$\psi(0) \in \mathbb{H}$}
	\end{equation*}
\item
	the {\color{red}strong continuity} of
	\,$\left\{\,\overset{{\color{white}.}}{U(t)}\,\right\}_{t \in \Re}$\,
\end{itemize}
then, it follows from the Stone theorem that \,$\psi(t)$\, satisfies the Schr\"odinger equation:
\begin{equation*}
\dfrac{\d\,\psi(t)}{\d\,t}
\; = \;
\i\,A\cdot \psi(t)\,,
\quad
\textnormal{for each \,$t\in\Re$	}
\end{equation*}
where \,$A$\, is the infinitesimal generator of \,$\left\{\,U(t)\,\right\}_{t\in\Re}$.\,
In particular, note that
\textbf{\color{red}the appearance of \,$\i = \sqrt{-1}$\, is part of the conclusions of the Stone theorem}.

\vskip 0.5cm
\begin{definition}
\mbox{}
\vskip 0.05cm
\noindent
Let \,$\mathbb{H}$\, be a complex Hilbert space.
A \,\textbf{one-parameter unitary group on $\mathbb{H}$}\,
is a family \,$\{\,U(t)\,\}_{t\in\Re}$\, of unitary operators on \,$\mathbb{H}$\,
such that:
\begin{itemize}
\item
	$U(0) \,=\, \mathbf{1}_{\mathbb{H}}$,\, and
\item
	$U(t_{1}+t_{2}) \;=\; U(t_{1}) \,\circ\, U(t_{2})$,\,
	for each \,$t_{1}, t_{2} \in \Re$.\,
\end{itemize}
A one-parameter unitary group \,$\{\,U(t)\,\}_{t\in\Re}$\, in \,$\mathbb{H}$\,
is said to be \textbf{strongly continuous} if
\begin{equation*}
\underset{s \rightarrow t}{\lim}\;
\left\Vert\;
	U(s) \cdot x
	\,\overset{{\color{white}.}}{-}\,
	U(t) \cdot x
	\;\right\Vert_{\mathbb{H}}
\;\;=\;\;
	0\,,
\quad
\textnormal{for each \,$x \in \mathbb{H}$\, and each \,$t \in \Re$.\,}
\end{equation*}
\end{definition}

\vskip 0.3cm
\begin{definition}
\mbox{}
\vskip 0.05cm
\noindent
Suppose
\,$\{\,U(t)\,\}_{t\in\Re}$\,
is a strongly continuous one-parameter unitary group on a complex Hilbert space \,$\mathbb{H}$.\,
The \textbf{infinitesimal generator} of
\,$\{\,U(t)\,\}_{t\in\Re}$\,
is an operator
\,$A$\, on \,$\mathbb{H}$\, defined as follows:
\begin{equation*}
\textnormal{Domain}(A)
\;\; := \;\;
	\left\{\;
		\psi \in \mathbb{H}
		\;\;\left\vert\;\;
		\begin{array}{c}
		\underset{{\color{white}1}}{\textnormal{there exists \,$\phi \in \mathbb{H}$\,}}
		\\
		\underset{t\,\rightarrow\,0}{\lim}\,
		\left\Vert\;
			\phi
			\,-\,
			\dfrac{1}{\i}\cdot\dfrac{
			U(t) \cdot \psi \,-\, \psi
			}{t}
			\;\right\Vert_{\mathbb{H}}
		\,=\,0
		\end{array}
		\right.
		\;\right\},
\end{equation*}
and \,$A : \textnormal{Domain}(A) \xrightarrow{{\color{white}222}} \mathbb{H}$\,
is defined by:
\begin{equation*}
A(\psi)
\;\; := \;\;
	\underset{t\,\rightarrow\,0}{\lim}\;\dfrac{1}{\i}\cdot\dfrac{
		U(t) \cdot \psi \,-\, \psi
		}{t}\,,
\quad
\textnormal{for each \,$\psi \in \textnormal{Domain}(A)$.\,}
\end{equation*}
\end{definition}

\vskip 0.3cm
\begin{theorem}[The Stone Theorem]
\mbox{}
\vskip 0.05cm
\noindent
Suppose \,$\left\{\,U(t)\,\right\}_{t\in\Re}$\, is a strongly continuous one-parameter
unitary group on a complex Hilbert space \,$\mathbb{H}$.\,
Then, the following statements are true:
\begin{enumerate}
\item
	the infinitesimal generator \,$A$\, of \,$\left\{\,U(t)\,\right\}_{t\in\Re}$\,
	is densely defined and {\color{red}self-adjoint}, and
\item
	$U(t) \, = \, e^{\i t A}$,\,
	for each \,$t \in \Re$,\,
	where \,$e^{\i t A}$\, is defined via the functional calculus of \,$A$.\,
\end{enumerate}
\end{theorem}

\vskip 0.3cm
\begin{proposition}
\mbox{}
\vskip 0.05cm
\noindent
Suppose \,$A$\, is a {\color{red}self-adjoint} operator on a complex Hilbert space \,$\mathbb{H}$\,
and for each \,$t \in \Re$,\, define \,$U(t) \,:=\, e^{\i t A}$,\,
where \,$e^{\i t A}$\, is defined via the functional calculus of \,$A$.\,
Then, the following statements are true:
\begin{enumerate}
\item
	$\left\{\,U(t)\,\right\}_{t\in\Re}$\, is
	a strongly continuous one-parameter unitary group on \,$\mathbb{H}$,\,
\item
	\,$\textnormal{Domain}(A)$\, is given by:
	\begin{equation*}
	\textnormal{Domain}(A)
	\;\; = \;\;
		\left\{\;
			\psi \in \mathbb{H}
			\;\;\left\vert\;\;
			\begin{array}{c}
			\underset{{\color{white}1}}{\textnormal{there exists \,$\phi \in \mathbb{H}$\,}}
			\\
			\underset{t\,\rightarrow\,0}{\lim}\,
			\left\Vert\;
				\phi
				\,-\,
				\dfrac{1}{\i}\cdot\dfrac{
				U(t) \cdot \psi \,-\, \psi
				}{t}
				\;\right\Vert_{\mathbb{H}}
			\,=\,0
			\end{array}
			\right.
			\;\right\}\,,
	\quad
	\textnormal{and}
	\end{equation*}
\item
	for each \,$\psi \in \textnormal{Domain}(A)$,\, we have:
	\begin{equation*}
	\underset{t\,\rightarrow\,0}{\lim}\;
	\left\Vert\;
		A(\psi)
		\,-\,
		\dfrac{1}{\i}\cdot\dfrac{
			U(t) \cdot \psi \,-\, \psi
			}{t}
		\;\right\Vert_{\mathbb{H}}
	\;\; = \;\;
		0
	\end{equation*}
\end{enumerate}
\end{proposition}

\vskip 0.3cm
\begin{remark}[The Schr\"odinger equation]
\mbox{}
\vskip 0.05cm
\noindent
Suppose \,$\left\{\,U(t)\,\right\}_{t\in\Re}$\, is a strongly continuous one-parameter
unitary group on a complex Hilbert space \,$\mathbb{H}$.\,
Let \,$\psi(0) \in \mathbb{H}$\, be an arbitrary element in \,$\mathbb{H}$.\,
For each \,$t \in \Re$,\, define
\,$\psi(t) \,:=\, U(t)\cdot\psi(0) \,=\, e^{\i t A}\cdot\psi(0)$.\,
Then, \,$\psi(\cdot)$\, satisfies the {\color{red}Schr\"odinger equation}, i.e.,
\,$\psi(\cdot)$\, satisfies the following differential equation:
\begin{equation*}
\dfrac{\d}{\d t}\,\psi(t)
\;\; = \;\;
	\i \, A \cdot \psi(t)\,,
\quad
\textnormal{for each \,$t \in \Re$}\,,
\end{equation*}
where \,$A$\, is the infinitesimal generator of
\,$\left\{\,U(t)\,\right\}_{t\in\Re}$.\,
\end{remark}

          %%%%% ~~~~~~~~~~~~~~~~~~~~ %%%%%

\vskip 0.5cm
\section{The quantum momentum operator as quantization of the Hamiltonian vector field of the classical momentum observable}

In this section, we give an ``inspirational'' justification
of why the quantum momentum operator \,$\widehat{P}$\,
in position state space is given by:
\begin{equation*}
(\widehat{P}\psi)(x)
\;\; = \;\;
	-\,\i\,\dfrac{\d\psi}{\d x}
\end{equation*}

\vskip 0.5cm
\begin{definition}[Symplectic manifold]
\mbox{}
\vskip 0.05cm
\noindent
A \textbf{symplectic manifold} is an ordered pair
\,$\left(\,M\,,\,\omega\,\right)$,\,
where \,$M$\, is a smooth manifold, and
\,$\omega$\, is a smooth non-degenerate closed $2$-form defined on all of \,$M$.\,
\end{definition}

\vskip 0.5cm
\begin{definition}[Classifical observables and their Hamiltonian vector fields]
\mbox{}
\vskip 0.05cm
\noindent
Let
\,$\left(\,M\,,\,\omega\,\right)$,\,
be a symplectic manifold.
\begin{enumerate}
\item
	The (commutative) algebra of \textbf{classical observables} on \,$M$\, is, by definition,
	the algebra \,$\mathscr{C}^{\infty}(M,\omega)$\, of all smooth $\Re$-valued functions
	defined on \,$(M,\omega)$\,,
	where addition and multiplication are defined via pointwise operations.
\item
	For each \,$f \in \mathscr{C}^{\infty}(M,\omega)$,\,
	the \textbf{Hamiltonian vector field} \,$X_{f} \in \Gamma(\mathcal{T}M)$\,
	\textbf{generated by \,$f$\,} is defined (implicitly)
	by the following equality:
	\begin{equation*}
	X_{f} \,\lrcorner\, \omega
	\,+\,
	\d f
	\;\; = \;\;
		0\,,
	\end{equation*}
	or equivalently,
	\begin{equation*}
	\omega(X_{f},Y)
	\,+\,
	Y(f)
	\;\; = \;\;
		0\,,
	\quad
	\textnormal{for each \,$Y \in \Gamma(\mathcal{T}M)$}
	\end{equation*}
\item
	For each \,$f \in \mathscr{C}^{\infty}(M,\omega)$,\,
	the \textbf{canonical flow} \,$\rho_{f}(t)$\, \textbf{generated by \,$f$\,}
	is the flow of the Hamiltonian vector field \,$X_{f}$\, on \,$M$.\,
\end{enumerate}
\end{definition}

\vskip 0.5cm
\begin{definition}[Poisson bracket on classical observables on a symplectic manifold]
\mbox{}
\vskip 0.05cm
\noindent
The \textbf{Poison bracket}
\,$
\left\{\,\cdot\,,\,\cdot\,\right\} :
\mathscr{C}^{\infty}(M,\omega) \times \mathscr{C}^{\infty}(M,\omega)
\xrightarrow{{\color{white}222}}
\mathscr{C}^{\infty}(M,\omega)
$\,
of a symplectic manifold
\,$\left(M,\omega\right)$\,
is defined by:
\begin{equation*}
\left\{f,g\right\}
\;\; := \;\;
	X_{f}(g)\,,
\quad
\textnormal{for each \,$f, g \in \mathscr{C}^{\infty}(M,\omega)$}
\end{equation*}
\end{definition}

\vskip 0.5cm
\begin{remark}
\mbox{}
\vskip 0.05cm
\noindent
In local coordinates, we have:
\begin{equation*}
\omega
\; = \;
	\d p_{a} \,\wedge\, \d q^{a}
\quad\quad\quad
\textnormal{and}
\quad\quad\quad
\d f
\; = \;
	\dfrac{\partial f}{\partial q^{a}}\,\d q^{a}
	\, + \,
	\dfrac{\partial f}{\partial p_{a}}\,\d p_{a}
\end{equation*}
Hence,
\begin{equation*}
X_{f}
\;\; = \;\;
	\dfrac{\partial f}{\partial p_{a}}\,\dfrac{\partial}{\partial q^{a}}
	\, - \,
	\dfrac{\partial f}{\partial q^{a}}\,\dfrac{\partial}{\partial p_{a}}
\end{equation*}
and \,$\rho_{f}(t)$\, is obtained by solving Hamilton's equations for \,$f$:\,
\begin{equation*}
\dot{q}^{a} \; = \; \dfrac{\partial f}{\partial p_{a}}
\quad\quad\quad
\textnormal{and}
\quad\quad\quad
\dot{p}_{a} \; = \; -\,\dfrac{\partial f}{\partial q_{a}}
\end{equation*}
The Poisson bracket of \,$f, g \in \mathscr{C}^{\infty}(M,\omega)$\, is expressed in local coordinates by:
\begin{equation*}
\left\{f,g\right\}
\;\; = \;\;
	\dfrac{\partial f}{\partial p_{a}}
	\dfrac{\partial g}{\partial q^{a}}
	\, - \,
	\dfrac{\partial g}{\partial p_{a}}
	\dfrac{\partial f}{\partial q^{a}}
\end{equation*}
\end{remark}

\vskip 0.5cm
\begin{remark}\label{MomentumHamiltonianVectorFieldIsSpatialDerivative}
\textnormal{\bf(The Hamiltonian vector field \,$X_{p_{a}}$\,
generated by the momentum coordinate \,$p_{a}$\, is
the operator \,$\partial\slash\partial x^{a}$ on phase space)}
\mbox{}
\vskip 0.15cm
\noindent
Now, let \,$Q \in \Re^{3}$\, and \,$M \,=\, \mathcal{T}^{*}(Q) \,\cong\, \Re^{6}$.\,
Here, \,$Q$\, is the position space of a classical particle, and
we coordinatize \,$Q$\, with \,$\mathbf{x} \in \Re^{3}$.\,
The cotangent bundle
\,$M \,=\, \mathcal{T}^{*}(Q) \,\cong\, \Re^{6}$\,
is the \textbf{\color{red}phase space} of the classical particle;
we coordinatize \,$M$\, with
\,$\left(\mathbf{x},\mathbf{p}\right) \in \Re^{6}$,\,
where
\,$\mathbf{x} \in \Re^{3}$\, parametrizes the position of the classical particle, and
\,$\mathbf{p} \in \Re^{3}$\, parametrizes its momentum.
\vskip 0.3cm
\noindent
$M = \mathcal{T}^{*}Q$\, becomes a symplectic manifold when endowed with the canonical $2$-form:
\begin{equation*}
\omega
\;\; = \;\;
	\d p_{a} \,\wedge\, \d x^{a}
\end{equation*}
\vskip 0.3cm
\noindent
For the momentum coordinate function \,$p_{a} \in \mathscr{C}^{\infty}(M,\omega)$,\,
its Hamiltonian vector field is given in local coordinates by:
\begin{equation*}
{\color{red}X_{p_{a}}}
\;\; = \;\;
	\dfrac{\partial {\color{red}p_{a}}}{\partial p_{b}} \cdot \dfrac{\partial}{\partial x^{b}}
	\, - \,
	\dfrac{\partial {\color{red}p_{a}}}{\partial x^{b}} \cdot \dfrac{\partial}{\partial p_{b}}
\;\; = \;\;
	\delta^{b}_{a} \cdot \dfrac{\partial}{\partial x^{b}}
	\, - \,
	0 \cdot \dfrac{\partial}{\partial p_{b}}
\;\; = \;\;
	{\color{red}\dfrac{\partial}{\partial x^{a}}}
\end{equation*}
\vskip 0.3cm
\noindent
And, the canonical flow \,$\rho_{p_{a}}(t)$\, generated by \,$p_{a} \in \mathscr{C}^{\infty}(M,\omega)$\,
is spatial translation in the $x^{a}$-direction.
Indeed, the Hamilton's equations in this case are:
\begin{equation*}
\left\{\quad
\begin{array}{ccccc}
\dot{x}^{b}
& = &
	{\color{white}-}\,\dfrac{\partial {\color{red}p_{a}}}{\partial p_{b}}
& = &
	\delta^{b}_{a}\,,
\\
\dot{p}_{b}
& = &
	\overset{{\color{white}1}}{-\,\dfrac{\partial {\color{red}p_{a}}}{\partial q_{b}}}
& = &
	0
\end{array}
\right.
\end{equation*}
whose solutions are:
\begin{equation*}
\left\{\quad
\begin{array}{ccl}
x^{b}(t)
& = &
	x^{b}(0)
	\,+\,
	t\cdot\delta^{b}_{a}
\\
\overset{{\color{white}1}}{p_{b}(t)}
& = &
	p_{b}(0)
\end{array}
\right.
\end{equation*}
% \vskip 0.3cm
% \noindent
% Define $H \in \mathscr{C}^{\infty}(M,\omega)$ as:
% \begin{equation*}
% H(\mathbf{x},\mathbf{p})
% \;\; = \;\;
% 	\dfrac{1}{2m}\,
% 	\overset{n}{\underset{j\,=\,1}{\sum}}\;
% 	p_{j}^{2}
% 	\;+\;
% 	V(\mathbf{x})
% \end{equation*}
\end{remark}

\vskip 0.5cm
\begin{remark}[Quantum momentum operator via Fourier transform]
\mbox{}
\vskip 0.05cm
\noindent
For simplicity, we restrict attention to one spatial dimension here.
According to \textbf{First Quantization},
\begin{itemize}
\item
	the commutative algebra
	\,$\mathscr{C}^{\infty}(M,\omega)$\,
	is to be replaced with a suitable complex Hilbert space \,$\mathbb{H}$\,
	(of $\C$-valued functions defined on spacetime \,$\Re^{1,1}$),
\item
	the Hamiltonian vector field
	\,$X_{p} \,=\, \dfrac{\partial}{\partial x}$\,
	(see Remark \ref{MomentumHamiltonianVectorFieldIsSpatialDerivative})
	-- which is defined on $\mathscr{C}^{\infty}(M,\omega)$ --
	will be replaced with a suitable operator on \,$\mathbb{H}$.\,
\end{itemize}
Define the \textbf{momentum operator} by:
\begin{equation*}
\widehat{P}
\;\; := \;\;
	-\,\i\,\dfrac{\partial}{\partial x}
\end{equation*}
Then, for each sufficiently nice unit vector \,$\psi \in L^{2}(\Re)$,\,
we have:
\begin{equation*}
\left\langle\;
	\overset{{\color{white}.}}{\psi}
	\;,\,
	\widehat{P}^{m}(\psi)
	\;\right\rangle
\;\; = \;\;
\left\langle\;
	\psi
	\,,\,
	\left(-\,\i\,\dfrac{\partial}{\partial x}\right)^{m}\!(\psi)
	\;\right\rangle
\;\; = \;\;
	\int_{\Re}\;\,
		k^{m}\cdot\left\vert\,\widehat{\psi}(k)\,\right\vert^{2}
		\d\,k\,,
\end{equation*}
where \,$\widehat{\psi}$\, is the Fourier transform of \,$\psi$,\, i.e.,\,
\begin{equation*}
\widehat{\psi}(k)
\;\; := \;\;
	\dfrac{1}{\sqrt{2\pi}}
	\int_{\Re}\;
		e^{-\,\i kx} \cdot \psi(x)
		\;\d\,x
\end{equation*}
Note that, by the above observation/calculations, we see that
\begin{itemize}
\item
	the Fourier transform maps from the \textbf{position state space}
	to the \textbf{momentum state space}, and
\item
	on the momentum state space, the momentum operator acts simply by
	\textbf{multiplication by the momentum coordinate function \,$k$}.
\end{itemize}
\end{remark}

\vskip 0.3cm
\begin{remark}
\mbox{}
\vskip 0.05cm
\noindent
Suppose \,$\mathbb{H}$\, is the state space of a quantum system \,$Q$\,
(hence, \,$\mathbb{H}$\, is a complex Hilbert space), and
\,$\psi \in \mathbb{H}$\, is an arbitrary ``sufficiently nice'' unit vector.
Then, the quantity
\,$
\left\langle\;
	\overset{{\color{white}.}}{\psi}
	\;,\,
	\widehat{P}^{m}(\psi)
	\;\right\rangle
$\,
has the following physical interpretation:
\begin{equation*}
\left\langle\;
	\overset{{\color{white}.}}{\psi}
	\;,\,
	\widehat{P}^{m}(\psi)
	\;\right\rangle
\;\; = \;\;
	\left\{\;
		\begin{array}{c}
		\textnormal{expected value of}
		\\
		\textnormal{the $m^{\textnormal{th}}$ power of}
		\\
		\textnormal{the momentum of }
		\\
		\textnormal{the quantum system $Q$}
		\\
		\textnormal{given that $Q$ is in state $\psi$}
		\end{array}
		\;\right\}
\end{equation*}
\end{remark}

          %%%%% ~~~~~~~~~~~~~~~~~~~~ %%%%%

\vskip 0.5cm
\section{The Schr\"odinger equation}

\begin{itemize}
\item
	Newtonian Energy-Momentum Relation for a {\color{red}free} particle (no force acting on it; zero potential):
	\begin{equation}
	\label{NewtonionEnergyMomentumRelation}
	E \;=\; \dfrac{\Vert\,\mathbf{p}\,\Vert^{2}}{2m}
	\quad\;\;
	\Longleftrightarrow
	\quad\;\;
	E \,-\, \dfrac{1}{2m}\left(\,p_{1}^{2} \overset{{\color{white}.}}{+} p_{2}^{2} + p_{3}^{2}\,\right) \;=\; 0
	\end{equation}
	\noindent
	\proof
	\begin{equation*}
	E
	\; := \;
		\left(\;\begin{array}{c}
			\textnormal{\small total energy} \\
			\textnormal{\small of the} \\
			\textnormal{\small free particle}
			\end{array}\;\right)
	\; = \;
		\left(\;\begin{array}{c}
			\textnormal{\small kinetic energy} \\
			\textnormal{\small of the} \\
			\textnormal{\small particle}
			\end{array}\;\right)
	\; = \;
		\dfrac{1}{2}\,m\,\Vert\,\mathbf{v}\,\Vert^{2}
	\; = \;
		\dfrac{(\,\Vert\,m\cdot\mathbf{v}\,\Vert\,)^{2}}{2m}
	\; = \;
		\dfrac{\Vert\,\mathbf{p}\,\Vert^{2}}{2m}
	\end{equation*}
	\vskip 1.0px
\item
	\textbf{First quantization:}\;\; Replacing
	\,$E$, $p_{1}$, $p_{2}$, $p_{3}$\,
	respectively with differential operators:
	\begin{equation}
	\label{FirstQuantization}
	E \;\longmapsto\; \i\,\dfrac{\partial}{\partial t}\,,
	\quad\quad
	p_{j} \;\longmapsto\; -\,\i\,\dfrac{\partial}{\partial x_{j}}\,,
	\quad\quad
	\textnormal{and}
	\end{equation}
	letting the resulting differential operator act on the wave function
	\,$\psi(t,x_{1},x_{2},x_{3})$\,
	yields the Schr\"odinger equation for a free particle:
	\begin{equation}
	\label{equationSchrodinger}
	\left(\;\i\,\dfrac{\partial}{\partial t} + \dfrac{1}{2m}\!
		\left(
			\dfrac{\partial^{2}}{\partial x_{1}^{2}}
			+ \dfrac{\partial^{2}}{\partial x_{2}^{2}}
			+ \dfrac{\partial^{2}}{\partial x_{3}^{2}}
			\right)
		\right)
	\psi
	\,=\,
		0
	\end{equation}
\item
	Schr\"odinger equation is NOT Lorentz-invariant (e.g., asymmetry between time and space) -- incompatible with Special Relativity.
\end{itemize}

          %%%%% ~~~~~~~~~~~~~~~~~~~~ %%%%%

\vskip 0.5cm
\section{The Klein-Gordon equation}

\begin{itemize}
\item
	Special-relativistic Energy-Momentum Relation for a {\color{red}free} particle (see \S3.4, \cite{DeFariaDeMelo2010}):
	\begin{equation}
	\label{RelativisticEnergyMomentumRelation}
	E^{2} \;=\; p^{2} \,+\, m^{2}
	\end{equation}
	\vskip 1.0px
\item
	First quantization on \eqref{RelativisticEnergyMomentumRelation}:
	\begin{equation*}
	E \;\longmapsto\; \i\,\dfrac{\partial}{\partial t}\,,
	\quad\quad
	p_{j} \;\longmapsto\; -\,\i\,\dfrac{\partial}{\partial x_{j}}
	\end{equation*}
	now yields the Klein-Gordon equation for a free particle:
	\begin{equation}
	\label{equationKleinGordon}
	\left(\; \textnormal{\large$\Box$} \,+\, m^{2} \,\right)\psi
	\;=\,
	\left(\; \partial_{\mu}\partial^{\mu} \,+\, m^{2} \,\right)\psi
	\;=\,
	\left(\,\dfrac{\partial^{2}}{\partial t^{2}}
	- \left(
		\dfrac{\partial^{2}}{\partial x_{1}^{2}}
		+ \dfrac{\partial^{2}}{\partial x_{2}^{2}}
		+ \dfrac{\partial^{2}}{\partial x_{3}^{2}}
		\right)
	\,+\,
		m^{2}
		\,\right)
	\psi
	\;=\;
		0
	\end{equation}
\item
	The Klein-Gordon equation is indeed Lorentz-invariant but:
	\vskip 1.5px
	\begin{itemize}\itemindent=-10px
	{\scriptsize
	\item
		it permits negative-energy eigenstates; see Example 6.1, \cite{LancasterBlundell2014}
	\item
		 probabilistic interpretation of (squared modulus of)
		 Schr\"odinger's wave function is lost; see \S6.2, \cite{LancasterBlundell2014}
	}
	\end{itemize}
	\vskip 5.0px
\item
	The Dirac equation addresses the second problem (i.e., restores dynamical/probabilistic interpretation).
\end{itemize}

          %%%%% ~~~~~~~~~~~~~~~~~~~~ %%%%%

\vskip 0.5cm
\section{The Dirac equation}

\begin{itemize}
\item
	Dirac sought a new equation of motion by seeking a first-order linear differential operator
	\begin{equation*}
	D
	\;=\;
		\i\,\gamma^{\mu}\partial_{\mu}
	\;=\;
		\i\left(\,
			\gamma^{0}\dfrac{\partial}{\partial x_{0}} 
			\,+\, \gamma^{1}\dfrac{\partial}{\partial x_{1}} 
			\,+\, \gamma^{2}\dfrac{\partial}{\partial x_{2}} 
			\,+\, \gamma^{3}\dfrac{\partial}{\partial x_{3}} 
			\,\right)
	\end{equation*}
	whose square is {\color{red}minus} the Minkowskian d'Alembertian
	\;$\textnormal{\large$\Box$}
	%\;=\;
	%	\dfrac{\partial^{2}}{\partial t^{2}} \,-\, \Delta
	\;=\;
		\dfrac{\partial^{2}}{\partial t^{2}}
		\,-\, \dfrac{\partial^{2}}{\partial x_{1}^{2}}
		\,-\, \dfrac{\partial^{2}}{\partial x_{2}^{2}}
		\,-\, \dfrac{\partial^{2}}{\partial x_{3}^{2}}
	$.\,
	\vskip 5px
\item
	Dirac arrived at the famous Dirac equation:
	\begin{equation}
	\label{TheDiracEquation}
	\left(\,\i\cdot\gamma^{\mu}\partial_{\mu} - m\,\right)\psi \, = \, 0
	\end{equation}
	\vskip -6.0px
	The requirement
	\,$
	-\,g^{\mu\nu}\partial_{\mu}\partial_{\nu}
	\,=\,
		{\color{red}-\,\textnormal{\large$\Box$}}
	\;{\color{red}=}\;
		{\color{red}\left(\,\i\cdot\gamma^{\mu}\partial_{\mu}\,\right)^{2}}
	%\,=\,
	%	\left(\,\i\cdot\gamma^{\mu}\partial_{\mu}\,\right)
	%	\left(\,\i\cdot\gamma^{\nu}\partial_{\nu}\,\right)
	\,=\,
		-\,\gamma^{\mu}\gamma^{\nu}\partial_{\mu}\partial_{\nu}
	\,=\,
		-\left(\dfrac{\gamma^{\mu}\gamma^{\nu}+\gamma^{\nu}\gamma^{\mu}}{2}\right)\partial_{\mu}\partial_{\nu}
		$\,
	implies that \,$\gamma^{\mu}$\, must satisfy:
	\begin{equation}
	\label{GammaMatricesSatisfyCliffordRelation}
	\left\{\,\gamma^{\mu}\,,\,\gamma^{\nu}\,\right\}
	\;=\;
		2\cdot g^{\mu\nu}\,,
	\end{equation}
	where
	\,$\left\{\,\gamma^{\mu}\,,\,\gamma^{\nu}\,\right\} \,:=\, \gamma^{\mu}\gamma^{\nu} + \gamma^{\nu}\gamma^{\mu}$\,
	and
	\,$g^{\mu\nu} \,=\, \diag(1,-1,-1,-1)$.
	\vskip 12px
\item
	{\color{white}.}
	\vskip -22px
	\begin{equation*}
	\begin{array}{c}\textnormal{Dirac} \\ \textnormal{equation} \end{array}
	\Longleftrightarrow\;\;\,
	\i\,\gamma^{\mu}\partial_{\mu}\psi = m\,\psi
	\;\;\Longrightarrow\;\;
	%\i\,\gamma^{\mu}\partial_{\mu}\left(\i\,\gamma^{\mu}\partial_{\mu}\overset{{\color{white}$.$}{\psi}\right) = m\,\i\,\gamma^{\mu}\partial_{\mu}\psi
	\underset{-\,\Box\psi}{\underbrace{{\color{white}.}{\color{red}\i\,\gamma^{\mu}\partial_{\mu}}\!\left(
		\i\,\gamma^{\nu}\partial_{\nu}\overset{{\color{white}.}}{\psi}
		\right)}}
		= m\cdot\underset{m\,\psi}{\underbrace{{\color{red}\i\,\gamma^{\mu}\partial_{\mu}}\overset{{\color{white}.}}{\psi}}}
	\;\;\,\Longrightarrow
	\begin{array}{c}\textnormal{Klein-Gordon} \\ \textnormal{equation} \end{array}
	\end{equation*}
\end{itemize}

\vskip 0.5cm
\begin{itemize}
\item
	Condition \eqref{GammaMatricesSatisfyCliffordRelation} cannot be satisfied by complex numbers:
	\begin{equation*}
	\left.\begin{array}{c}
		{\color{red}\left\{\,\gamma^{\mu}\,,\,\gamma^{\nu}\,\right\} \;=\;2\cdot g^{\mu\nu}}
		\\
		\overset{{\color{white}1}}{\gamma_{\mu}}\;\in\;\C,\;\;\forall\;\mu = 0,1,2,3
	\end{array}\right\}
	\quad\Longrightarrow\quad
	\textnormal{contradiction}
	\end{equation*}
	But, Dirac observed that Condition \eqref{GammaMatricesSatisfyCliffordRelation}
	could indeed be satisfied by a set of {\color{red}$4 \times 4$} complex matrices:
	\vskip -10px
	
\begin{equation*}
\gamma^{0}
\,=\,
	\textnormal{\tiny$\left(\!\!\begin{array}{rrrr}
		0 & 0 & {\color{white}-}1 & 0 \\
		0 & 0 & 0 & {\color{white}-}1 \\
		{\color{white}-}1 & 0 & 0 & 0 \\
		0 & {\color{white}-}1 & 0 & 0
		\end{array}\!\right)$}
\,=\,
	\textnormal{\tiny$\left(\!\begin{array}{cc}
		0 & I_{2} \\
		I_{2} & \overset{{\color{white}.}}{0} \\
		\end{array}\!\!\right)$},
\quad\quad
\gamma^{1}
\,=\,
	\textnormal{\tiny$\left(\!\begin{array}{rrrr}
		0 & 0 & 0 & {\color{white}-}1 \\
		0 & 0 & {\color{white}-}1 & 0 \\
		0 & -1 & 0 & 0 \\
		-1 & 0 & 0 & 0
		\end{array}\right)$}
\,=\,
	\textnormal{\tiny$\left(\!\begin{array}{cc}
		0 & \sigma_{1} \\
		-\sigma_{1} & \overset{{\color{white}.}}{0} \\
		\end{array}\!\!\right)$}
\end{equation*}
\begin{equation*}
\gamma^{2}
\,=\,
	\textnormal{\tiny$\left(\!\begin{array}{rrrr}
		0 & 0 & 0 & -\i \\
		0 & 0 & {\color{white}-}\i & 0 \\
		0 & {\color{white}-}\i & 0 & 0 \\
		-\i & 0 & 0 & 0
		\end{array}\right)$}
\,=\,
	\textnormal{\tiny$\left(\!\begin{array}{cc}
		0 & \sigma_{2} \\
		-\sigma_{2} & \overset{{\color{white}.}}{0} \\
		\end{array}\!\!\right)$},
\quad\quad
\gamma^{3}
\,=\,
	\textnormal{\tiny$\left(\!\begin{array}{rrrr}
		0 & 0 & {\color{white}-}1 & 0\\
		0 & 0 & 0 & -1 \\
		{\color{white}-}1 & 0 & 0 & 0 \\
		0 & -1 & 0 & 0
		\end{array}\right)$}
\,=\,
	\textnormal{\tiny$\left(\!\begin{array}{cc}
		0 & \sigma_{3} \\
		-\sigma_{3} & \overset{{\color{white}.}}{0} \\
		\end{array}\!\!\right)$}
\end{equation*}

	\vskip 5px
\item
	So, Dirac sought solutions \,$\psi \in C^{\infty}(\mathcal{M},{\color{red}\C^{4}})$\,
	for the Dirac equation: \,$\left(\,\i\,{\color{red}\gamma^{\mu}}\partial_{\mu} \overset{{\color{white}.}}{-} m\right)\psi = 0$.\,
%\item
%	This observation led to the following realization (when expressed in modern geometric language):
%	The ``wave function'' \,$\psi$\, must be smooth sections of a spinor bundle over spacetime.
\end{itemize}

          %%%%% ~~~~~~~~~~~~~~~~~~~~ %%%%%


          %%%%% ~~~~~~~~~~~~~~~~~~~~ %%%%%

