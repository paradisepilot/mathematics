
          %%%%% ~~~~~~~~~~~~~~~~~~~~ %%%%%

\chapter{Classification of elementary particles via mass and spin --
irreducible unitary projective representations of the Poincaré group \,$\SOup(1,3) \ltimes \Re^{1,3}$}
\setcounter{theorem}{0}
\setcounter{equation}{0}

%\cite{vanDerVaart1996}
%\cite{Kosorok2008}

%\renewcommand{\theenumi}{\alph{enumi}}
%\renewcommand{\labelenumi}{\textnormal{(\theenumi)}$\;\;$}
\renewcommand{\theenumi}{\roman{enumi}}
\renewcommand{\labelenumi}{\textnormal{(\theenumi)}$\;\;$}

          %%%%% ~~~~~~~~~~~~~~~~~~~~ %%%%%

%
          %%%%% ~~~~~~~~~~~~~~~~~~~~ %%%%%

\chapter{Irreducible complex representations of a real Lie algebra versus those of its complexification}
\setcounter{theorem}{0}
\setcounter{equation}{0}

%\cite{vanDerVaart1996}
%\cite{Kosorok2008}

%\renewcommand{\theenumi}{\alph{enumi}}
%\renewcommand{\labelenumi}{\textnormal{(\theenumi)}$\;\;$}
\renewcommand{\theenumi}{\roman{enumi}}
\renewcommand{\labelenumi}{\textnormal{(\theenumi)}$\;\;$}

          %%%%% ~~~~~~~~~~~~~~~~~~~~ %%%%%

\section{Complexifications of real Lie algebras}

\begin{definition}
\mbox{}
\vskip 0.05cm
\noindent
Let \,$V$\, be a finite-dimensional vector space over \,$\Re$.\,
Then, the \textbf{complexification} \,$V_{\C}$\, is
the (finite-dimensional) vector space \,$\C$\, obtained as follows:
\begin{itemize}
\item
	The underlying set (of vectors) of \,$V_{\C}$\, is the set of all formal linear combinations
	of the form:
	\begin{equation*}
	v_{1} \, + \, \i\cdot v_{2}\,,
	\end{equation*}
	where \,$v_{1}, v_{2} \in V$.\,
\item
	Vector addition in \,$V_{\C}$\, is defined as follows:
	\begin{equation*}
	\left(\;\overset{{\color{white}.}}{v_{1}} \, + \, \i\cdot v_{2}\,\right)
	\; + \;
	\left(\;\overset{{\color{white}.}}{w_{1}} \, + \, \i\cdot w_{2}\,\right)
	\;\; := \;\;
		\left(\,\overset{{\color{white}.}}{v_{1}} + w_{1} \,\right)
		\; + \;
		\i\cdot\left(\,\overset{{\color{white}.}}{v_{2}} + w_{2} \,\right)
	\end{equation*}	
\item
	Complex scalar multiplication on \,$V_{\C}$\, is defined as follows:
	\begin{equation*}
	(\,a + \i \, b\,) \cdot \left(\;\overset{{\color{white}.}}{v_{1}} \, + \, \i\cdot v_{2}\,\right)
	\;\; := \;\;
		\left(\,a\cdot\overset{{\color{white}.}}{v_{1}} - b \cdot v_{2} \,\right)
		\; + \;
		\i\cdot\left(\,a\cdot\overset{{\color{white}.}}{v_{2}} + b \cdot v_{1} \,\right)
	\end{equation*}	
\end{itemize}
\end{definition}

          %%%%% ~~~~~~~~~~~~~~~~~~~~ %%%%%

\vskip 0.5cm
\begin{proposition}
\mbox{}
\vskip 0.05cm
\noindent
Let \,$\mathfrak{g}$\, be a finite-dimensional real Lie algebra and
\,$\mathfrak{g}_{\C}$\, its vector-space complexification.
Then, the Lie bracket of \,$\mathfrak{g}$\, has a unique extension to \,$\mathfrak{g}_{\C}$\,
that makes  \,$\mathfrak{g}_{\C}$\, into a complex Lie algebra.
The resulting complex Lie algebra -- still denoted as \,$\mathfrak{g}_{\C}$\, -- is called
the \textbf{complexification} of \,$\mathfrak{g}$.\,
\end{proposition}
\proof
The unique extension
\,$[\;\cdot\,,\,\cdot\,]_{\mathfrak{g}_{\C}}$\,
to \,$\mathfrak{g}_{\C}$\, of
\,$[\;\cdot\,,\,\cdot\,]_{\mathfrak{g}}$\,
is given by
\begin{equation*}
\left[\;
	X_{1} \overset{{\color{white}.}}{+} \i\cdot Y_{1}
	\;,\,
	X_{2} \overset{{\color{white}.}}{+} \i\cdot Y_{2}
	\,\right]_{\mathfrak{g}_{\C}}
\;\, := \;\;
	\left(\;
		[\,X_{1},X_{2}\,]_{\mathfrak{g}}
		\; \overset{{\color{white}1}}{-} \,
		[\,Y_{1},Y_{2}\,]_{\mathfrak{g}}
		\,\right)
	\; + \;
	\i \cdot\! \left(\;
		[\,X_{1},Y_{2}\,]_{\mathfrak{g}}
		\; \overset{{\color{white}1}}{+} \,
		[\,Y_{1},X_{2}\,]_{\mathfrak{g}}
		\,\right)
\end{equation*}
for each \,$X_{1}, Y_{1}, X_{2}, Y_{2} \in \mathfrak{g}$.
For the proof, see Proposition 3.37, p.65, \cite{Hall2015}.
\qed

          %%%%% ~~~~~~~~~~~~~~~~~~~~ %%%%%

\vskip 0.5cm
\begin{proposition}[Universal property of the complexification of a real Lie algebra]
\label{UniqueExtensionOfRealLieAlgebraHomomorphisms}
\mbox{}
\vskip 0.05cm
\noindent
Let \,$\mathfrak{g}$\, be a finite-dimensional real Lie algebra and
\,$\mathfrak{g}_{\C}$\, its Lie-algebra complexification.
Let \,$\mathfrak{h}$\, be an arbitrary complex Lie algebra.
Then, every real Lie algebra homomorphism \,$\pi$\,
from \,$\mathfrak{g}$\, into \,$\mathfrak{h}$\,
extends uniquely to a complex Lie algebra homomorphsim \,$\pi_{\C}$\,
from \,$\mathfrak{g}_{\C}$\, into \,$\mathfrak{h}$.\,
\end{proposition}
\proof
The unique extension \,$\pi_{\C}$\, is given by
\begin{equation*}
\pi_{\C}\!\left(\,X \overset{{\color{white}.}}{+} \i\cdot Y \right)
\;\, := \;\;
	\pi(X) \overset{{\color{white}.}}{+} \i\cdot \pi(Y)\,,
\quad
\textnormal{for each \,$X, Y \in \mathfrak{g}$}
\end{equation*}
For the proof, see Proposition 3.39, p.67, \cite{Hall2015}.
\qed

          %%%%% ~~~~~~~~~~~~~~~~~~~~ %%%%%

\section{Irreducible representations of the complexification of a real Lie algebra}

\begin{proposition}
\mbox{}
\vskip 0.05cm
\noindent
Let \,$\mathfrak{g}$\, be a real Lie algebra and
\,$\mathfrak{g}_{\C}$\, its Lie-algebra complexification.
Let \,$V$\, be an arbitrary finite-dimensional complex vector space.
Then, the following statements are true:
\begin{enumerate}
\item
	Every real Lie algebra homomorphism
	\,$\pi : \mathfrak{g} \longrightarrow \End_{\Re}(V)$\,
	has a unique extension to a complex Lie algebra homomorphism
	\,$\pi_{\C} : \mathfrak{g}_{\C} \longrightarrow \End_{\C}(V)$.\,
\item
	$\pi$\, is irreducible if and only if \,$\pi_{\C}$\, is irreducible.
\end{enumerate}
\end{proposition}
\proof
\begin{enumerate}
\item
	Immediate by Proposition \ref{UniqueExtensionOfRealLieAlgebraHomomorphisms}.
\item
	Recall that a representation of (real or complex) Lie algebra is irreducible if it has
	no non-trivial invariant subspaces.
	Thus, the equivalence of the irreducibility of \,$\pi$\, and that of \,$\pi_{\C}$\,
	will follow from the fact that the two representations share the same collection
	of invariant subspaces. In other words, it suffices to establish the following:
	\vskip 0.2cm
	\noindent
	\textbf{Claim 1:}\quad
	Suppose \,$\{\,0\,\} \subsetneq W \subsetneq V$\, is a proper (complex) subspace of \,$V$.\,
	Then, \,$W$\, is \,$\pi$-invariant if and only if it is \,$\pi_{\C}$-invariant. 
	\vskip 0.1cm
	\noindent
	Proof of Claim 1:\quad
	Let \,$w \in W$\, be an arbitrary element of \,$W$.\,
	Suppose first that \,$W$\, is \,$\pi$-invariant.
	Then, for each \,$X, Y \in \mathfrak{g}$,\, we have:
	\begin{eqnarray*}
	\pi_{\C}\!\left(\,X \overset{{\color{white}.}}{+} \i\cdot Y \right) \cdot [\,w\,]
	& = &
		\left(\,\pi(X) \overset{{\color{white}.}}{+} \i\cdot \pi(Y) \right) \cdot [\,w\,]
	\\
	& = &
		\pi(X) \cdot w \;\overset{{\color{white}.}}{+}\; \i\cdot \pi(Y) \cdot w
	\;\; \in \;\;
		W \; + \; \i \cdot W
	\;\; = \;\;
		W,
	\end{eqnarray*}
	which proves that \,$W$\, is also \,$\pi_{\C}$-invariant.
	Conversely, now suppose that \,$W$\, is \,$\pi_{\C}$-invariant.
	Then, for each \,$X \in \mathfrak{g}$,\, we have:
	\begin{equation*}
	\pi(X)\cdot w
	\;\; = \;\;
		\left(\,\pi(X) \overset{{\color{white}.}}{+} \i\cdot \pi(0_{\mathfrak{g}}) \right) \cdot [\,w\,]
	\;\; = \;\;
		\pi_{\C}\!\left(\,X \overset{{\color{white}.}}{+} \i\cdot 0_{\mathfrak{g}} \right) \cdot [\,w\,]
	\;\; \in \;\;
		W,
	\end{equation*}
	which proves that \,$W$\, is also \,$\pi$-invariant.
	This proves Claim 1, as well as completes the proof of the Proposition.
\end{enumerate}
\qed

\vskip 0.5cm
\begin{remark}
\mbox{}
\vskip 0.05cm
\noindent
Intuitively speaking, the preceding Proposition says that
the set of irreducible finite-dimensional complex representations of a real Lie algebra
is a ``subset'' of 
the set of irreducible finite-dimensional complex representations of its (Lie-algebra) complexification.
\end{remark}

          %%%%% ~~~~~~~~~~~~~~~~~~~~ %%%%%

%\vskip 1.0cm
%
          %%%%% ~~~~~~~~~~~~~~~~~~~~ %%%%%

\section{Irreducible representations of \,$\mathfrak{su}(2)$}
\setcounter{theorem}{0}
\setcounter{equation}{0}

%\cite{vanDerVaart1996}
%\cite{Kosorok2008}

%\renewcommand{\theenumi}{\alph{enumi}}
%\renewcommand{\labelenumi}{\textnormal{(\theenumi)}$\;\;$}
\renewcommand{\theenumi}{\roman{enumi}}
\renewcommand{\labelenumi}{\textnormal{(\theenumi)}$\;\;$}

          %%%%% ~~~~~~~~~~~~~~~~~~~~ %%%%%

\begin{definition}[$\textnormal{U}(n)$ and $\mathfrak{u}(n)$]
\mbox{}
\vskip 0.1cm
\noindent
The \textbf{unitary group} is defined as follows:
\begin{equation*}
\textnormal{U}(n)
\; := \;
	\left\{\;\,
		g \overset{{\color{white}.}}{\in} \textnormal{GL}(n,\C)
		\;\left\vert\;\,
			g^{\dagger} \cdot g = I_{n}
			\right.
		\;\right\}
\end{equation*}
where \,$g^{\dagger}$\, is the conjugate transpose of
\,$g \in \textnormal{GL}(n,\C)$\,
and
\,$I_{n} \in \textnormal{GL}(n,\C)$\,
is the identity matrix.
\vskip 0.1cm
\noindent
The \textbf{special unitary group} is defined as follows:
\begin{equation*}
\textnormal{SU}(n)
\; := \;
	\left\{\;\,
		g \overset{{\color{white}.}}{\in} \textnormal{GL}(n,\C)
		\;\left\vert\;\,
			g^{\dagger} \cdot g = I_{n}\,,
			\;
			\textnormal{det}(g) = 1
			\right.
		\;\right\}
\end{equation*}
\end{definition}

\vskip 0.5cm
\begin{proposition}[Parametrization of \,$\textnormal{SU}(2)$]
\mbox{}
\vskip 0.1cm
\noindent
$\textnormal{SU}(2)$ admits the following parametrization:
\begin{equation*}
\textnormal{SU}{(2)}
\; := \;
	\left\{\,
		\left.
		\left(\begin{array}{rr}
		a & -\overline{b}
		\\
		\overset{{\color{white}-}}{b} & \overline{a}
		\end{array}\right)
		\overset{{\color{white}.}}{\in}
		\C^{2 \times 2}
		\;\;\right\vert\;\,
			\vert\, a \,\vert^{2} \,+\, \vert\, b \,\vert^{2} \,=\, 1
		\;\right\}
\end{equation*}
Hence, the (real) Lie group {\color{red}$\textnormal{SU}(2)$ is diffeomorphic to $S^{3}$}, the $3$-dimensional unit sphere
(in $4$-dimensional Euclidean space).
In particular, $\textnormal{SU}(2)$ is a {\color{red}simply connected} $3$-dimensional real manifold.
\end{proposition}
\proof
Suppose:
\begin{equation*}
g
\; = \;
	\left(\begin{array}{cc}
		a & c
		\\
		\overset{{\color{white}-}}{b} & d
		\end{array}\right)
\; \in \;
\textnormal{SU}(2)
\end{equation*}
First note that the component form of the condition \,$g^{\dagger}\cdot g = I_{2}$\, is:
\begin{equation*}
\left(\begin{array}{cc}
	1 & 0
	\\
	\overset{{\color{white}-}}{0} & 1
	\end{array}\right)
\; = \;
	g^{\dagger} \cdot g
\; = \;
	\left(\begin{array}{cc}
		\overline{a} & \overline{b}
		\\
		\overset{{\color{white}-}}{\overline{c}} & \overline{d}
		\end{array}\right)
	\cdot
	\left(\begin{array}{cc}
		a & c
		\\
		\overset{{\color{white}-}}{b} & d
		\end{array}\right)
\; = \;
	\left(\begin{array}{cc}
		a\overline{a} + b\overline{b} & \overline{a}c + \overline{b}d
		\\
		\overset{{\color{white}-}}{a\overline{c} + b\overline{d}} & c\overline{c} + d\overline{d}
		\end{array}\right)
\end{equation*}
Thus, we see that
\begin{equation*}
g
\; = \;
	\left(\begin{array}{cc}
		a & c
		\\
		\overset{{\color{white}-}}{b} & d
		\end{array}\right)
\;\in\;
	\textnormal{SU}(2)
\quad\Longleftrightarrow\quad
\left\{
	\begin{array}{ccc}
		g^{\dagger} \cdot g &=& I_{2}
		\\
		\det(g) &=& \overset{{\color{white}1}}{1}
		\end{array}
		\right.
\quad\Longleftrightarrow\quad
\left\{
	\begin{array}{ccc}
	\vert\,a\,\vert^{2} + \vert\,b\,\vert^{2} &=& 1
	\\
	\vert\,c\,\vert^{2} + \vert\,d\,\vert^{2} &\overset{{\color{white}1}}{=}& 1
	\\
	a\overline{c} \;\, + \,\; b\overline{d} &\overset{{\color{white}1}}{=}& 0
	\\
	ad \;\, - \,\; bc &\overset{{\color{white}1}}{=}& 1
	\end{array}
	\right.
\end{equation*}
Next, note that
\begin{equation*}
a\overline{c} + b\overline{d} = 0
\quad\Longleftrightarrow\quad
	\left\langle
		\left(\begin{array}{c} a \\ b \end{array}\right)
		\,,\,
		\left(\begin{array}{c} c \\ d \end{array}\right)
		\right\rangle_{\C^{2}}
	\;=\;
	0
\end{equation*}
Since
\,$\dim_{\C}\left(\begin{array}{c} a \\ b \end{array}\right)^{\perp} =\, 1$,\,
the above equality (i.e., orthogonality of the two columns of \,$g$) implies:
\begin{equation*}
\left(\begin{array}{c} c \\ d \end{array}\right)
\; \in \;
	\left(\begin{array}{c} a \\ b \end{array}\right)^{\perp}
\; = \;
	\textnormal{span}_{\C}\left\{
		\left(\begin{array}{r} -\overline{b} \\ \overline{a} \end{array}\right)
		\right\}
\quad\Longleftrightarrow\quad
\left(\begin{array}{c} c \\ d \end{array}\right)
\; = \;
	\lambda \left(\begin{array}{r} -\overline{b} \\ \overline{a} \end{array}\right),
	\;\;
	\textnormal{for some $\lambda \in \C$}
\end{equation*}
So, we now know that $g$ has the form:
\begin{equation*}
g
\; = \;
	\left(\begin{array}{rr}
		a & -\lambda\,\overline{b}
		\\
		\overset{{\color{white}-}}{b} & \lambda\,\overline{a}
		\end{array}\right)
\end{equation*}
Next,
\begin{equation*}
1
\,=\, \det(g)
\,=\, a\cdot(\lambda\,\overline{a}) - b \cdot (-\lambda\,\overline{b})
\,=\, \lambda\cdot(\vert\,a\,\vert^{2} + \vert\,b\,\vert^{2})
\quad\Longrightarrow\quad
	\lambda = 1
\end{equation*}
We may now conclude that
\begin{equation*}
g
\; = \;
	\left(\begin{array}{rr}
		a & -\,\overline{b}
		\\
		\overset{{\color{white}-}}{b} & \overline{a}
		\end{array}\right),\,
\quad
\textnormal{where \,$\vert\,a\,\vert^{2} + \vert\,b\,\vert^{2} = 1$}
\end{equation*}
This completes the proof of the Proposition.
\qed

          %%%%% ~~~~~~~~~~~~~~~~~~~~ %%%%%

\vskip 0.5cm
\begin{proposition}[Characterizations of \,$\mathfrak{sl}(n)$, \,$\mathfrak{u}(n)$, and $\mathfrak{su}(n)$]
\mbox{}
\vskip 0.1cm
\begin{enumerate}
\item
	\begin{equation*}
	\mathfrak{sl}(n,\C)
	\; = \;
		\left\{\;
			X \,\in\, \mathfrak{gl}(n,\C) \,=\, \C^{n \times n}
			\;\left\vert\;\,
				\textnormal{trace}(X) = \overset{{\color{white}1}}{0}
				\right.
			\,\right\}
	\end{equation*}
\item
	\begin{equation*}
	\mathfrak{u}(n)
	\; = \;
		\left\{\;
			X \,\in\, \mathfrak{gl}(n,\C) \,=\, \C^{n \times n}
			\;\left\vert\;\,
				X + X^{\dagger} = \overset{{\color{white}1}}{0}
				\right.
			\,\right\}
	\end{equation*}
\item
	\begin{equation*}
	\mathfrak{su}(n)
	\; = \;
		\left\{\;
			X \,\in\, \mathfrak{gl}(n,\C) \,=\, \C^{n \times n}
			\;\left\vert\;\,
				\begin{array}{c}
				X + X^{\dagger} = \overset{{\color{white}1}}{0}
				\\
				\textnormal{trace}(X) = \overset{{\color{white}1}}{0}
				\end{array}
				\right.
			\,\right\}
	\end{equation*}
\end{enumerate}
\end{proposition}
\proof
\begin{enumerate}
\item
	We invoke the fact that \,$\det(e^{\,t\,\cdot\,X}) \,=\, e^{\,t\,\cdot\,\textnormal{trace}(X)}$,
	for each \,$X \in \C^{n \times n}$.
	Thus,
	\begin{eqnarray*}
	&&
		X \,\in\, \mathfrak{sl}(n,\C)
		\quad\Longrightarrow\quad
		e^{\,t\cdot\,X} \,\in\, \textnormal{SL}(n,\C)
		\quad\Longrightarrow\quad
		\det\!\left(\,e^{\,t\cdot\,X}\,\right) \,=\, 1
	\\
	& \Longrightarrow\quad &
		\textnormal{trace}(X)
		\; = \;
			\left.\dfrac{\d}{\d\,t}\right\vert_{t=0}\left(\,\overset{{\color{white}1}}{e^{\,t\,\cdot\,\textnormal{trace}(X)}}\,\right)
		\; = \;
			\left.\dfrac{\d}{\d\,t}\right\vert_{t=0}\left(\,\overset{{\color{white}1}}{\det(e^{\,t\,\cdot\,X})}\,\right)
		\; = \;
			\left.\dfrac{\d}{\d\,t}\right\vert_{t=0}\left(\,\overset{{\color{white}1}}{1}\,\right)
		\; = \;
			0
	\end{eqnarray*}
	Conversely, suppose \,$\textnormal{trace}(X) = 0$.\,
	Then, \,$\det(e^{\,t\,\cdot\,X}) \,=\, e^{\,t\,\cdot\,\textnormal{trace}(X)} \,=\, e^{\,t\,\cdot\,0} \,=\, 1$,\,
	which implies that \,$e^{\,t\,\cdot\,X} \,\in\, \textnormal{SL}(n,\C)$,\, hence \,$X \,\in\, \mathfrak{sl}(n,\C)$.
	This completes the proof of the equality (of sets) in question.
\item
	\begin{eqnarray*}
	&&
		X \,\in\, \mathfrak{u}(n)
		\quad\Longrightarrow\quad
		e^{\,t\,\cdot\,X} \,\in\, \textnormal{U}(n)
	\\
	& \Longrightarrow\quad &
		I_{n}
			\,=\, \left(\,e^{\,t\,\cdot\,X}\,\right)^{\!\dagger} \cdot \left(\,e^{\,t\,\cdot\,X}\,\right)
			\,=\, \left(\,e^{\,t\,\cdot\,X^{\dagger}}\,\right) \cdot \left(\,e^{\,t\,\cdot\,X}\,\right)
			\,=\, e^{\,t\,\cdot\,(X^{\dagger}+X)}
	\\
	& \Longrightarrow\quad &
		X \,+\, X^{\dagger}
		\; = \;
			\left.\dfrac{\d}{\d\,t}\right\vert_{t=0}\left(\,\overset{{\color{white}1}}{e^{\,t\,\cdot\,(X+X^{\dagger})}}\,\right)
		\; = \;
			\left.\dfrac{\d}{\d\,t}\right\vert_{t=0}\left(\,\overset{{\color{white}1}}{I_{n}}\,\right)
		\; = \;
			0
	\end{eqnarray*}
	Conversely, suppose \,$X + X^{\dagger} \,=\, 0$.\,
	Then, \,$I_{n}$
	\,$=$\, $e^{\,0_{n \times n}}$
	\,$=$\, $e^{\,t\,\cdot(X^{\dagger}+X)}$
	\,$=\, \cdots \,=$\, $\left(e^{\,t\,\cdot\,X}\right)^{\!\dagger}\cdot\left(e^{\,t\,\cdot\,X}\right)$,\,
	which implies that \,$e^{\,t\,\cdot\,X} \,\in\, \textnormal{U}(n)$,\, hence \,$X \,\in\, \mathfrak{u}(n)$.
	This completes the proof of the equality (of sets) in question.
\item
	Immediate by the preceding two statements.
	\qed
\end{enumerate}

\vskip 0.5cm
\begin{proposition}[Generators of \,$\mathfrak{su}(2)$]
\mbox{}
\vskip 0.1cm
\noindent
Let \,$\sigma_{1},\, \sigma_{2},\, \sigma_{3} \,\in\, \C^{2 \times 2}$\, be the \textbf{Pauli spin matrices}, i.e.,
\begin{equation*}
\sigma_{1} \,=\, \sigma_{x} \,:=\, \left(\begin{array}{cc} 0 & 1 \\ 1 & 0 \end{array}\right),
\quad
\sigma_{2} \,=\, \sigma_{y} \,:=\, \left(\begin{array}{rr} 0 & -\i \\ \i & 0 \end{array}\right),
\quad
\sigma_{3} \,=\, \sigma_{z} \,:=\, \left(\begin{array}{rr} 1 & 0 \\ 0 & -1 \end{array}\right).
\end{equation*}
Define \,$J_{1},\, J_{2},\, J_{3},\, S_{1},\, S_{2},\, S_{3},\, S_{+},\, S_{-} \,\in\, \C^{2 \times 2}$\, as follows:
\begin{equation*}
Y_{1} \,:=\, \dfrac{\i}{2}\cdot\sigma_{1} \,=\, \dfrac{\i}{2}\cdot\left(\begin{array}{cc} 0 & 1 \\ 1 & 0 \end{array}\right),
\quad
Y_{2} \,:=\, \mathbf{{\color{red}-}}\,\dfrac{\i}{2}\cdot\sigma_{2} \,=\, \dfrac{1}{2}\cdot\left(\begin{array}{rr} 0 & -1 \\ 1 & 0 \end{array}\right),
\quad
Y_{3} \,:=\, \dfrac{\i}{2}\cdot\sigma_{3} \,=\, \dfrac{\i}{2}\cdot\left(\begin{array}{rr} 1 & 0 \\ 0 & -1 \end{array}\right),
\end{equation*}
\begin{equation*}
S_{1} \,:=\, \i \cdot Y_{1} \,=\, \dfrac{-1}{2} \cdot \left(\begin{array}{rr} 0 & 1 \\ 1 & 0 \end{array}\right),
\quad
S_{2} \,:=\, \i \cdot Y_{2} \,=\, \dfrac{\i}{2} \cdot \left(\begin{array}{rr} 0 & -1 \\ 1 & 0 \end{array}\right),
\quad
S_{3} \,:=\, \i \cdot Y_{3} \,=\, \dfrac{1}{2}\cdot\left(\begin{array}{rr} -1 & 0 \\ 0 & 1 \end{array}\right).
\end{equation*}
\begin{equation*}
S_{+} \; := \; S_{1} + \i\,S_{2} \; = \; \left(\begin{array}{rr} 0 & 0 \\ -1 & 0 \end{array}\right),
\quad\quad
S_{-} \; := \; S_{1} - \i\,S_{2} \; = \; \left(\begin{array}{rr} 0 & -1 \\ 0 & 0 \end{array}\right).
\end{equation*}
Then, the following statements are true:
\begin{enumerate}
\item
	$Y_{1},\, Y_{2},\, Y_{3} \,\in\, \mathfrak{su}(2)$.\,
	$Y_{1},\, Y_{2},\, Y_{3}$\,
	form a set of generators for the (real) Lie algebra \,$\mathfrak{su}(2)$\, of the (real) Lie group \,$\textnormal{SU}(2)$.\,
	$Y_{1},\, Y_{2},\, Y_{3}$\, satisfy the following commutation relations:
	\begin{equation*}
	\left[\,Y_{a}\,,\,Y_{b}\,\right] \;\; = \;\; \overset{3}{\underset{c\,=\,1}{\sum}}\;\varepsilon_{abc}\,Y_{c}\,,
	\quad
	\textnormal{for \,$a, b = 1,2,3$}.
	\end{equation*}
\item
	$S_{1},\, S_{2},\, S_{3} \,\in\, \mathfrak{su}(2) \otimes_{\Re} \C$,\,
	where
	\,$\mathfrak{su}(2) \otimes_{\Re} \C$\,
	is the complexification of (the real Lie algebra)
	\,$\mathfrak{su}(2)$.\,
	\,$S_{1},\, S_{2},\, S_{3}$\,
	satisfy the following commutation relations:
	\begin{equation*}
	\left[\,S_{a}\,\overset{{\color{white}1}}{,}\,S_{b}\,\right]
	\;\; = \;\;
		\sqrt{-1}\,\cdot\overset{3}{\underset{c\,=\,1}{\sum}}\;\varepsilon_{abc}\cdot S_{c}\,,
	\quad
	\textnormal{for each \,$a, b \in \{\,1,2,3\,\}$}\,,
	\end{equation*}
	where \,$\varepsilon_{abc}$\, is the fully anti-symmetric tensor.
\item
	$S_{+},\, S_{-} \,\in\, \mathfrak{su}(2) \otimes_{\Re} \C$,\,
	and
	\,$S_{+},\; S_{-},\; S_{3}$\, satisfy the following commutation relations:
	\begin{equation*}
	\left[\,S_{3}\,,\,S_{\pm}\,\right] \, = \, \pm\,S_{\pm}\,,
	\quad
	\left[\,S_{+}\,,\,S_{-}\,\right] \, = \, 2\,S_{3}
	\end{equation*}
\item
	Suppose
	\begin{itemize}
	\item
		$V$\, is a complex vector space,
	\item
		$\rho : \mathfrak{su}(2) \otimes_{\Re} \C \longrightarrow \textnormal{End}(V)$\,
		is a Lie algebra representation, and
	\item	
		$v \in V$\, and \,$\lambda \in \C$\, together satisfy \,$\rho(S_{3})(v) = \lambda \cdot v$.
	\end{itemize}	
	Then, \,$\rho(S_{+})(v) \,\in\, V$\, satisfies:
	\begin{equation*}
	\rho(S_{3})\!\left(\,\rho(\overset{{\color{white}.}}{S}_{+})(v)\,\right)
	\; = \;
		(\lambda+1) \cdot \rho(S_{+})(v)\,
	\end{equation*}
	and
	\,$\rho(S_{-})(v) \,\in\, V$\, satisfies:
	\begin{equation*}
	\rho(S_{3})\!\left(\,\rho(\overset{{\color{white}.}}{S}_{-})(v)\,\right)
	\; = \;
		(\lambda-1) \cdot \rho(S_{-})(v)\,
	\end{equation*}
\end{enumerate}
\end{proposition}
\proof
The Proposition follows straightforwardly by direct computations.
\qed

          %%%%% ~~~~~~~~~~~~~~~~~~~~ %%%%%

\vskip 0.5cm
\begin{proposition}[The irreducible representations \,\textnormal{$\pi_{d} : \su(2) \otimes_{\Re} \C \longrightarrow \gl\!\left(\,\overset{{\color{white}.}}{\C}[X,Y]_{d}\,\right)$}]
\mbox{}
\vskip 0.1cm
\noindent
Let
\,$\C[X,Y]_{d}$,\,
where
\,$d \,\in\, \{\,0, 1, 2, 3, \ldots\,\}$,\,
be the complex vector space of homogeneous polynomials of degree $d$
in the indeterminates $X$ and $Y$ with complex coefficients.
\vskip 0.1cm
\noindent
Define
\,$\pi_{d} : \su(2) \otimes_{\Re} \C \longrightarrow \gl\!\left(\,\overset{{\color{white}.}}{\C}[X,Y]_{d}\,\right)$\,
by complex-linearly extending:
\begin{equation*}
\pi_{d}\!\left(\,S_{+}\,\right)
\; := \;
	-\,Y\cdot\dfrac{\partial}{\partial\,X},
\quad\;\;\;
\pi_{d}\!\left(\,S_{-}\,\right)
\; := \;
	-\,X\cdot\dfrac{\partial}{\partial\,Y},
\quad\;\;\;
\pi_{d}\!\left(\,S_{3}\,\right)
\; := \;
	-\,\dfrac{1}{2}\cdot\left(\,X\cdot\dfrac{\partial}{\partial\,X} - Y\cdot\dfrac{\partial}{\partial\,Y}\right)
\end{equation*}
Then, the following statements are true:
\begin{enumerate}
\item
	$\pi_{d} : \su(2) \otimes_{\Re} \C \longrightarrow \gl\!\left(\,\overset{{\color{white}.}}{\C}[X,Y]_{d}\,\right)$\,
	is a finite-dimensional complex representation of the complex Lie algebra
	\,$\su(2) \otimes_{\Re} \C$.\,
\item
	The representation
	\,$\pi_{d} : \su(2) \otimes_{\Re} \C \longrightarrow \gl\!\left(\,\overset{{\color{white}.}}{\C}[X,Y]_{d}\,\right)$\,
	is irreducible.
\end{enumerate}
\end{proposition}
\proof
\begin{enumerate}
\item
	Linearity of \,$\pi_{d}$\, holds by the definition of \,$\pi_{d}$\,
	(which is obtained by complex-linearly extending its values on the vectors in the basis
	\,$\{\,S_{+}, S_{-}, S_{3}\,\}$\, of \,$\so(u) \otimes_{\Re}\C$).
	Thus, in order to establish that \,$\pi_{d}$\, is a representation (i.e., it is a Lie algebra homomorphism),
	it remains only to show that \,$\pi_{d}$\, preserves commutation relations.
	To this end, observe that:
	\begin{eqnarray*}
	\left[\;
		\overset{{\color{white}1}}{\pi_{d}}\!\left(\,S_{+}\,\right)
		\, , \,
		\pi_{d}\!\left(\,S_{-}\,\right)
		\,\right]
		(X^{m}Y^{n})
	& = &
		\left[\;
			-\,Y\cdot\dfrac{\partial}{\partial\,X}
			\;\; , \,
			-\,X\cdot\dfrac{\partial}{\partial\,Y}
			\,\right]
			(X^{m}Y^{n})
	\\
	& = &
		\left[\;
			Y\cdot\dfrac{\partial}{\partial\,X}
			\;\; , \,
			X\cdot\dfrac{\partial}{\partial\,Y}
			\,\right]
			(X^{m}Y^{n})
	\\
	& = &
		Y\cdot\dfrac{\partial}{\partial\,X}\!\left(\,
			\overset{{\color{white}.}}{X}\,X^{m} \cdot n \cdot Y^{n-1}
			\,\right)
		\, - \,
		X\cdot\dfrac{\partial}{\partial\,Y}\!\left(\,
			\overset{{\color{white}.}}{Y} \cdot m \cdot X^{m-1} \, Y^{n}
			\,\right)
	\\
	& = &
		n \cdot Y \cdot \dfrac{\partial}{\partial\,X}\!\left(\,
			\overset{{\color{white}.}}{X^{m+1}} - Y^{n-1}
			\,\right)
		\, - \,
		m \cdot X \cdot \dfrac{\partial}{\partial\,X}\!\left(\,
			\overset{{\color{white}.}}{X^{m-1}} - Y^{n+1}
			\,\right)
	\\
	& = &
		n(m+1) \cdot \overset{{\color{white}1}}{X^{m}}\,Y^{n} \, - \, m(n+1) \cdot X^{m}\,Y^{n}
	\\
	& = &
		(n-m) \cdot \overset{{\color{white}1}}{X^{m}\,Y^{n}}
	\end{eqnarray*}
	On the other hand,
	\begin{eqnarray*}
	\pi_{d}(S_{3})\!\left(\,\overset{{\color{white}.}}{X^{m}Y^{n}}\,\right)
	& = &
		-\,\dfrac{1}{2}\cdot\left(\,X\cdot\dfrac{\partial}{\partial\,X} - Y\cdot\dfrac{\partial}{\partial\,Y}\right)
		\!\left(\,\overset{{\color{white}.}}{X^{m}Y^{n}}\,\right)
	\\
	& = &
		-\,\dfrac{1}{2}\cdot\left(\,
			\overset{{\color{white}.}}{X} \cdot m \cdot X^{m-1}\,Y^{n}
			\, - \,
			Y \cdot X^{m} \cdot n \cdot Y^{n-1}
			\,\right)
	\\
	& = &
		- \, \dfrac{1}{2} \cdot (m-n) \cdot \overset{{\color{white}1}}{X^{m}\,Y^{n}}
	\;\; = \;\;
		\dfrac{1}{2} \cdot (n-m) \cdot \overset{{\color{white}1}}{X^{m}\,Y^{n}}
	\\
	& = &
		\dfrac{1}{2} \cdot
		\left[\;
			\overset{{\color{white}1}}{\pi_{d}}\!\left(\,S_{+}\,\right)
			\, , \,
			\pi_{d}\!\left(\,S_{-}\,\right)
			\,\right]
			(X^{m}Y^{n})
	\end{eqnarray*}
	This proves that
	\begin{equation*}
	\left[\;
		\overset{{\color{white}1}}{\pi_{d}}\!\left(\,S_{+}\,\right)
		\, , \,
		\pi_{d}\!\left(\,S_{-}\,\right)
		\,\right]
	\;\; = \;\;
		2 \cdot \pi_{d}(S_{3})
	\end{equation*}
	Next, note
	\begin{eqnarray*}
	\left[\;
		\overset{{\color{white}1}}{\pi_{d}}\!\left(\,S_{3}\,\right)
		\, , \,
		\pi_{d}\!\left(\,S_{+}\,\right)
		\,\right]
		(X^{m}Y^{n})
	& = &
		\left[\;\,
			-\,\dfrac{1}{2}\cdot\left(\,X\cdot\dfrac{\partial}{\partial\,X} - Y\cdot\dfrac{\partial}{\partial\,Y}\right)
			\; , \;
			-\,Y\cdot\dfrac{\partial}{\partial\,X}
			\,\right]
			(X^{m}Y^{n})
	\\
	& = &
		\left[\;\,
			\dfrac{1}{2}\cdot\left(\,X\cdot\dfrac{\partial}{\partial\,X} - Y\cdot\dfrac{\partial}{\partial\,Y}\right)
			\; , \;
			Y\cdot\dfrac{\partial}{\partial\,X}
			\,\right]
			(X^{m}Y^{n})
	\\
	& = &
		\dfrac{1}{2}\cdot\left(\,X\cdot\dfrac{\partial}{\partial\,X} - Y\cdot\dfrac{\partial}{\partial\,Y}\right)
		\left(\, Y \cdot m \cdot X^{m-1}\,Y^{n} \,\right)
	\\
	&&
		-\,Y\cdot\dfrac{\partial}{\partial\,X}\left(\,
			\dfrac{1}{2}\left(\,
				X \cdot m \cdot X^{m-1}\,Y^{n} \,-\, Y \cdot X^{m} \cdot n \cdot Y^{n-1}
				\,\right)
			\,\right)
	\\
	& = &
		\dfrac{1}{2}\cdot\left(\,X\cdot\dfrac{\partial}{\partial\,X} - Y\cdot\dfrac{\partial}{\partial\,Y}\right)
		\left(\, m \cdot X^{m-1}\,Y^{n+1} \,\right)
	\\
	&&
		-\,Y\cdot\dfrac{\partial}{\partial\,X}\left(\,
			\dfrac{1}{2}\,(m-n)\,X^{m}\,Y^{n}
			\,\right)
	\\
	& = &
		\dfrac{1}{2}\left(\,
			\overset{{\color{white}.}}{X} \cdot m \cdot (m-1) \cdot X^{m-2}\,Y^{n+1}
			\,-\,
			Y \cdot m \cdot X^{m-1} \cdot (n+1) \cdot Y^{n}
			\,\right)
	\\
	&&
		-\,Y\cdot\dfrac{1}{2}\,(m-n)\cdot m \cdot X^{m-1}\,Y^{n}
	\\
	& = &
		\dfrac{1}{2}\left(\,
			m\,(m-1)\cdot \overset{{\color{white}.}}{X^{m-1}}\,Y^{n+1}
			\,-\,
			m\,(n+1)\cdot X^{m-1}\,Y^{n+1}
			\,\right)
	\\
	&&
		-\,\dfrac{1}{2} \cdot m\,(m-n) \cdot X^{m-1}\,Y^{n+1}
	\\
	& = &
		\dfrac{m}{2}\left(\,
			m - \overset{{\color{white}.}}{1} - n - 1 - m + n
			\,\right)
		X^{m-1}\,Y^{n+1}
	\;\; = \;\;
		\dfrac{m}{2}\left(\,-\,\overset{{\color{white}.}}{2}\,\right)X^{m-1}\,Y^{n+1}
	\\
	& = &
		-\,m\,\overset{{\color{white}1}}{X^{m-1}}\,Y^{n+1}
	\end{eqnarray*}
	and
	\begin{eqnarray*}
	\pi_{d}(S_{+})\!\left(\,\overset{{\color{white}.}}{X^{m}Y^{n}}\,\right)
	& = &
		-\,Y\cdot\dfrac{\partial}{\partial\,X}\!\left(\,\overset{{\color{white}.}}{X^{m}Y^{n}}\,\right)
	\;\; = \;\;
		-\, Y \cdot m \cdot X^{m-1}\,Y^{n}
	\;\; = \;\;
		-\, m \, X^{m-1}\,Y^{n+1}
	\\
	& = &
		\left[\;
			\overset{{\color{white}1}}{\pi_{d}}\!\left(\,S_{3}\,\right)
			\, , \,
			\pi_{d}\!\left(\,S_{+}\,\right)
			\,\right]
			(X^{m}Y^{n})
	\end{eqnarray*}
	This proves:
	\begin{equation*}
	\left[\;
		\overset{{\color{white}1}}{\pi_{d}}\!\left(\,S_{3}\,\right)
		\, , \,
		\pi_{d}\!\left(\,S_{+}\,\right)
		\,\right]
	\;\; = \;\;
		\pi_{d}(S_{+})
	\end{equation*}
	Similar calculations will show:
	\begin{equation*}
	\left[\;
		\overset{{\color{white}1}}{\pi_{d}}\!\left(\,S_{3}\,\right)
		\, , \,
		\pi_{d}\!\left(\,S_{-}\,\right)
		\,\right]
	\;\; = \;\;
		-\,\pi_{d}(S_{-})
	\end{equation*}
	This completes the proof that
	\,$\pi_{d} : \su(2) \otimes_{\Re} \C \longrightarrow \gl\!\left(\,\overset{{\color{white}.}}{\C}[X,Y]_{d}\,\right)$\,
	is indeed a complex representation of the complex Lie algebra
	\,$\su(2) \otimes_{\Re} \C$.\,
	Lastly, \,$\pi_{d}$\, is a finite-dimensional representation since
	\,$\dim_{C}\!\left(\,\overset{{\color{white}.}}{\C}[X,Y]_{d}\,\right) \,=\, d+1$.\,
\item
	\textit{This proof is found in Proposition 4.11, p.84, \cite{Hall2015}.}
	\vskip 0.1cm
	\noindent
	Let \,$W \subset \C[X,Y]_{d}$\, be an nonzero invariant subspace of \,$\C[X,Y]_{d}$.\,
	We need to show that \,$W = \C[X,Y]_{d}$.\,
	\vskip 0.3cm
	\noindent
	\textbf{Claim 1:}\quad
	For \,$k \in \{\,0,1,2,\cdots,d\,\}$,\, we have:
	\begin{eqnarray*}
	\pi_{d}\!\left(\,S_{+}\right)\!\left(\,\overset{{\color{.}}}{X^{d-k}}\,Y^{k}\,\right)
	& = &
		-\,(\,d-k\,) \cdot X^{d-k-1} \, Y^{k+1}
	\\
	\pi_{d}\!\left(\,S_{-}\right)\!\left(\,\overset{{\color{.}}}{X^{d-k}}\,Y^{k}\,\right)
	& = &
		\quad\quad\;\;\;
		 -\,k \cdot X^{d-k+1} \, Y^{k-1}
	\\
	\pi_{d}\!\left(\,S_{3}\right)\!\left(\,\overset{{\color{.}}}{X^{d-k}}\,Y^{k}\,\right)
	& = &
		\left(\,k-\dfrac{d}{2}\,\right) \cdot \overset{{\color{white}1}}{X^{d-k}\,Y^{k}}
	\end{eqnarray*}
	\proofof Claim 1:\quad Straightforward calculations.
	
	\vskip 0.5cm
	\noindent
	\textbf{Claim 2:}\quad $Y^{d} \in W$.
	\vskip -0.05cm
	\noindent
	\proofof Claim 2:\quad 
	Let \,$w \in W$.\, Then, \,$w$\, can be written as:
	\begin{equation*}
	w
	\;\; = \;\;
		a_{0}X^{d} \,+\, a_{1}X^{d-1}Y \,+\, a_{2}X^{d-2}Y^{2} \,+\, \cdots \,+\, a_{d-1}XY^{d-1} \,+\, a_{d}Y^{d},
	\end{equation*}
	where at least one of the coefficients
	\,$a_{0}, a_{1}, a_{2}, \ldots, a_{d} \,\in\, \C$\,
	is nonzero.
	Let \,$k_{0}$\, be the smallest value of \,$k$\, such that \,$a_{k} \neq 0$.\,
	If \,$k_{0} = d$,\ then \,$w = a_{k_{0}}\,Y^{k_{0}} = a_{d}\,Y^{d}$\, is a nonzero multiple of \,$Y^{d}$.\,
	Thus, \,$Y^{d} \,=\, \dfrac{1}{a_{d}}\,w \,\in\, W$,\, i.e., Claim 2 is true.
	If \,$k < d_{0}$,\, we consider
	\begin{equation*}
	\pi_{d}\!\left(\,S_{+}\right)^{d - k_{0}}w
	\end{equation*}
	By Claim 1, \,$\pi_{d}(S_{+})$\, raises the exponent of \,$Y$\, by \,$1$;\, hence,
	\,$\pi_{d}\!\left(\,S_{+}\right)^{d - k_{0}}$\,
	annihilates all terms in \,$w$\, except
	\begin{equation*}
	a_{k_{0}}\,X^{d-k_{0}}\,Y^{k_{0}}
	\end{equation*}
	Hence,
	\begin{eqnarray*}
	\pi_{d}\!\left(\,S_{+}\right)^{d - k_{0}}\!\left(\,\overset{{\color{white}1}}{w}\,\right)
	& = &
		\pi_{d}\!\left(\,S_{+}\right)^{d - k_{0}}\!\left(\,
			\overset{{\color{white}1}}{a_{k_{0}}}\,X^{d-k_{0}}\,Y^{k_{0}}
			\,\right)
	\;\; = \;\;
		\cdots
	\\
	& = &
		(\,-1\,)^{d-k_{0}} \cdot a_{k_{0}} \cdot (d-k_{0}) \cdot (d-k_{0}-1) \cdot 2 \cdot 1 \cdot Y^{d}\,,
	\end{eqnarray*}	
	which shows that
	\,$\pi_{d}\!\left(\,S_{+}\right)^{d - k_{0}}\!\left(\,\overset{{\color{white}1}}{w}\,\right)$\,
	is a nonzero multiple of \,$Y^{d}$\, (since $k_{0} < d$).
	Now, invariance of \,$W$\, and \,$w \in W$\, imply
	\,$\pi_{d}\!\left(\,S_{+}\right)^{d - k_{0}}\!\left(\,\overset{{\color{white}1}}{w}\,\right) \,\in\, W$,\,
	which in turn implies
	\,$Y^{d} \,\in\, W$.\,
	This proves Claim 2.

	\vskip 0.5cm
	\noindent
	\textbf{Claim 3:}\quad $X^{k}\,Y^{d-k} \in W$,\, for each \,$k \in \{\,0,1,2,\cdots,d\,\}$.
	\vskip -0.05cm
	\noindent
	\proofof Claim 3:\quad By Claim 1, Claim 2, and the invariance of \,$W$,\, we have:
	\begin{equation*}
	 -\,d \cdot X \, Y^{d-1}
	\; = \;
		\pi_{d}\!\left(\,S_{-}\right)\!\left(\;\overset{{\color{.}}}{Y^{d}}\,\right)
		\; \in \; W
	\quad \Longrightarrow \quad
		X \, Y^{d-1} \; \in \; W
	\end{equation*}
	By Claim 1 and the invariance of \,$W$,\, we have:
	\begin{equation*}
	 -\,(d-1) \cdot X^{2} \, Y^{d-2}
	\; = \;
		\pi_{d}\!\left(\,S_{-}\right)\!\left(\;\overset{{\color{.}}}{X}\,Y^{d-1}\,\right)
		\; \in \; W
	\quad \Longrightarrow \quad
		X^{2} \, Y^{d-2} \; \in \; W
	\end{equation*}

	\begin{equation*}
	\pi_{d}\!\left(\,S_{-}\right)\!\left(\,\overset{{\color{.}}}{X^{d-k}}\,Y^{k}\,\right)
	\;\; = \;\;
		 -\,k \cdot X^{d-k+1} \, Y^{k-1}
	\end{equation*}
	Claim 3 now follows by repeating the above argument suitably many times.
	
	\vskip 0.5cm
	\noindent
	Now, recall that
	\,$\C[X,Y]_{d}$\,
	is spanned over \,$\C$\, by
	\,$X^{d},\, X^{d-1}Y,\, X^{d-2}Y^{2},\, \cdots,\, XY^{d-1},\, Y^{d}$.\,
	Claim 3 now implies that \,$\C[X,Y]_{d} \,\subset\, W$,\,
	which in turn proves that
	\,$\C[X,Y]_{d} \,=\, W$.\
	This completes the proof of the irreducibility of the representation
	\,$\pi_{d} : \su(2) \otimes_{\Re} \C \longrightarrow \gl\!\left(\,\overset{{\color{white}.}}{\C}[X,Y]_{d}\,\right)$.\,
	\qed
\end{enumerate}


          %%%%% ~~~~~~~~~~~~~~~~~~~~ %%%%%

%\vskip 1.0cm
%
          %%%%% ~~~~~~~~~~~~~~~~~~~~ %%%%%

\section{Irreducible representations of \,$\so(3)$}
\setcounter{theorem}{0}
\setcounter{equation}{0}

%\cite{vanDerVaart1996}
%\cite{Kosorok2008}

%\renewcommand{\theenumi}{\alph{enumi}}
%\renewcommand{\labelenumi}{\textnormal{(\theenumi)}$\;\;$}
\renewcommand{\theenumi}{\roman{enumi}}
\renewcommand{\labelenumi}{\textnormal{(\theenumi)}$\;\;$}

          %%%%% ~~~~~~~~~~~~~~~~~~~~ %%%%%

\vskip 0.5cm
\begin{proposition}[Set-theoretic characterizations of the Lie algebras of $\textnormal{O}(n)$ and $\textnormal{SO}(n)$]
\begin{eqnarray*}
\mathfrak{o}(n)
& = &
	\left\{\;\,
		X \overset{{\color{white}.}}{\in} \gl(n,\Re)
		\;\left\vert\;\,
			X^{T} = -X
			\right.
		\;\right\}
\\
\so(n)
& = &
	\left\{\;\,
		X \overset{{\color{white}.}}{\in} \gl(n,\Re)
		\;\left\vert\;\,
			X^{T} = -X\,,
			\;
			\textnormal{trace}(X) = 0
			\right.
		\;\right\}
\end{eqnarray*}
\end{proposition}

          %%%%% ~~~~~~~~~~~~~~~~~~~~ %%%%%

\vskip 0.5cm
\begin{proposition}[Some {\color{red}convenient generators} of \,$\so(3)$\, and \,$\so(3) \otimes_{\Re} \C$]
\mbox{}
\vskip 0.1cm
\noindent
Let
\,$R_{1}(\phi),\, R_{2}(\psi),\, R_{3}(\theta)$\,
be the one-parameter families of
\textbf{rotation matrices} in $3$-dimensional Euclidean space
around the $x$-, $y$, and $z$-axes, respectively, i.e.,
\begin{eqnarray*}
R_{1}(\phi)
& := &
	\left(\,
		\begin{array}{ccc}
			{\color{white}--}1 & {\color{white}-.}0 & {\color{white}-}0 \\
			{\color{white}--}0 & {\color{white}-.}\cos\phi & -\sin\phi \\
			{\color{white}--}0 & {\color{white}-.}\sin\phi & {\color{white}-}\cos\phi \\
			\end{array}
		\,\right)
\\
R_{2}(\psi)
& := &
	\left(\,
		\begin{array}{ccc}
			{\color{white}-}\cos\psi & {\color{white}-.}0 & {\color{white}--}\sin\psi \\
			{\color{white}-}0 & {\color{white}-.}1 & {\color{white}--}0 \\
			{\color{red}-}\sin\psi & {\color{white}-.}0 & {\color{white}--}\cos\psi \\
			\end{array}
		\right)
\\
R_{3}(\theta)
& := &
	\left(\,
		\begin{array}{ccc}
			{\color{white}-}\cos\theta & -\sin\theta & {\color{white}--}0 \\
			{\color{white}-}\sin\theta & {\color{white}-}\cos\theta & {\color{white}--}0 \\
			{\color{white}-}0 & {\color{white}-}0 & {\color{white}--}1 \\
			\end{array}
		\;\,\right)
\end{eqnarray*}
Let
\,$X_{1},\, X_{2},\, X_{3} \,\in\, \C^{3 \times 3}$\,
be the \textbf{infinitesimal generators} of
\,$R_{1}(\phi),\, R_{2}(\psi),\, R_{3}(\theta)$,\,
respectively, i.e.,
\begin{eqnarray*}
X_{1}
& := &
	\left.\dfrac{\d}{\d\,\phi}\right\vert_{\phi = 0} R_{1}(\phi)
\;\;\; = \;
	\left.\left(\,
		\begin{array}{ccc}
			{\color{white}-}1 & {\color{white}-}0 & {\color{white}-}0 \\
			{\color{white}-}0 & {\color{black}-}\sin\phi & {\color{black}-}\cos\phi \\
			{\color{white}-}0 & {\color{white}-}\cos\phi & {\color{black}-}\sin\phi \\
			\end{array}
		\,\right)\right\vert_{\phi = 0}
\,\, = \;
	\left(
		\begin{array}{ccc}
			{\color{white}-}0 & {\color{white}-}0 & {\color{white}-}0 \\
			{\color{white}-}0 & {\color{white}-}0 & {\color{black}-}1 \\
			{\color{white}-}0 & {\color{white}-}1 & {\color{white}-}0 \\
			\end{array}
		\,\right)
\\
X_{2}
& := &
	\left.\dfrac{\d}{\d\,\psi}\right\vert_{\psi = 0} R_{2}(\psi)
\;\; = \;
	\left.\left(\!\!
		\begin{array}{ccc}
			{\color{black}-}\sin\psi & {\color{white}-}0 & {\color{black}-}\cos\psi \\
			{\color{white}-}0 & {\color{white}-}0 & {\color{white}-}0 \\
			{\color{white}-}\cos\psi & {\color{white}-}0 & {\color{black}-}\sin\psi \\
			\end{array}
		\,\right)\right\vert_{\psi = 0}
\;\; = \;
	\left(
		\begin{array}{ccc}
			{\color{white}-}0 & {\color{white}-}0 & {\color{white}-}1 \\
			{\color{white}-}0 & {\color{white}-}0 & {\color{white}-}0 \\
			{\color{black}-}1 & {\color{white}-}0 & {\color{white}-}0 \\
			\end{array}
		\,\right)
\\
X_{3}
& := &
	\,\left.\dfrac{\d}{\d\,\theta}\right\vert_{\theta = 0} R_{3}(\theta)
\;\;\;\, = \;
	\left.\left(\!
		\begin{array}{ccc}
			{\color{black}-}\sin\theta & {\color{black}-}\cos\theta & {\color{white}-}0 \\
			{\color{white}-}\cos\theta & {\color{black}-}\sin\theta & {\color{white}-}0 \\
			{\color{white}-}0 & {\color{white}-}0 & {\color{white}-}0 \\
			\end{array}
		\,\right)\right\vert_{\psi = 0}
\;\;\, = \;
	\left(
		\begin{array}{ccc}
			{\color{white}-}0 & {\color{black}-}1 & {\color{white}-}0 \\
			{\color{white}-}1 & {\color{white}-}0 & {\color{white}-}0 \\
			{\color{white}-}0 & {\color{white}-}0 & {\color{white}-}0 \\
			\end{array}
		\,\right)
\end{eqnarray*}
Let \,$J_{1}, J_{2}, J_{3} \,\in\, \C^{3 \times 3}$\, be the \textbf{Euler matrices}, i.e.,
\begin{eqnarray*}
J_{1}
& := &
	\i \cdot X_{1}
\;\; = \;\;
	\left(\!\!
		\begin{array}{ccc}
			{\color{white}-}0 & {\color{white}-}0 & {\color{white}-}0 \\
			{\color{white}-}0 & {\color{white}-}0 & {\color{black}-}\i \\
			{\color{white}-}0 & {\color{white}-}\i & {\color{white}-}0 \\
			\end{array}
		\,\right)
\\
J_{2}
& := &
	\i \cdot X_{2}
\;\; = \;\;
	\left(\!\!
		\begin{array}{ccc}
			{\color{white}-}0 & {\color{white}-}0 & {\color{white}-}\i \\
			{\color{white}-}0 & {\color{white}-}0 & {\color{white}-}0 \\
			{\color{black}-}\i & {\color{white}-}0 & {\color{white}-}0 \\
			\end{array}
		\,\right)
\\
J_{3}
& := &
	\i \cdot X_{3}
\;\; = \;\,
	\left(\!
		\begin{array}{ccc}
			{\color{white}-}0 & {\color{black}-}\i & {\color{white}-}0 \\
			{\color{white}-}\i & {\color{white}-}0 & {\color{white}-}0 \\
			{\color{white}-}0 & {\color{white}-}0 & {\color{white}-}0 \\
			\end{array}
		\,\right)
\end{eqnarray*}
Define the \textbf{raising operator}
\,$J_{+} \in \C^{3 \times 3}$\,
and
\textbf{lowering operator}
\,$J_{-} \in \C^{3 \times 3}$\,
as follows:
\begin{equation*}
J_{\pm} \;\; := \;\; J_{1} \, \pm \sqrt{-1}\,J_{2}
\end{equation*}
Thus, explicitly,
\begin{equation*}
J_{+}
\;\; := \;\;
\left(\!
	\begin{array}{ccc}
		{\color{white}-}0 & {\color{white}-}0 & {\color{black}-}1 \\
		{\color{white}-}0 & {\color{white}-}0 & {\color{black}-}\i \\
		{\color{white}-}1 & {\color{white}-}\i & {\color{white}-}0 \\
		\end{array}
	\,\right),
\quad\quad
J_{-}
\;\; := \;\;
\left(\!
	\begin{array}{ccc}
		{\color{white}-}0 & {\color{white}-}0 & {\color{white}-}1 \\
		{\color{white}-}0 & {\color{white}-}0 & {\color{black}-}\i \\
		{\color{black}-}1 & {\color{white}-}\i & {\color{white}-}0 \\
		\end{array}
	\,\right)
\end{equation*}
Then, the following statements are true:
\begin{enumerate}
\item
	$X_{1},\, X_{2},\, X_{3}$ are elements of \,$\so(3)$,\,
	they are furthermore generators of \,$\so(3)$,\, and
	they satisfy the following commutation relations:
	\begin{equation*}
	\left[\,X_{1}\,,\,X_{2}\,\right] \;=\; +\,X_{3}
	\quad
	\left[\,X_{3}\,,\,X_{1}\,\right] \;=\; +\,X_{2}
	\quad
	\left[\,X_{2}\,,\,X_{3}\,\right] \;=\; +\,X_{1}
	\end{equation*}
	More succinctly,
	\begin{equation*}
	\left[\,X_{a}\,,\,X_{b}\,\right] \;\;=\;\; \overset{3}{\underset{c\,=\,1}{\sum}}\;\varepsilon_{abc}\,X_{c}\,,
	\quad
	\textnormal{for each \,$a, b \in \{\,1,2,3\,\}$}\,,
	\end{equation*}
	where \,$\varepsilon_{abc}$\, is the fully anti-symmetric tensor.
\item
	$J_{1},\, J_{2},\, J_{3}$\, are elements of \,$\so(3) \otimes_{\Re} \C$,\,
	they are furthermore generators of \,$\so(3) \otimes_{\Re} \C$,\, and
	they satisfy the following commutation relations:
	\begin{equation*}
	\left[\,J_{a}\,\overset{{\color{white}1}}{,}\,J_{b}\,\right]
	\;\; = \;\;
		\sqrt{-1}\,\cdot\overset{3}{\underset{c\,=\,1}{\sum}}\;\varepsilon_{abc}\cdot J_{c}\,,
	\quad
	\textnormal{for each \,$a, b \in \{\,1,2,3\,\}$}\,,
	\end{equation*}
	where \,$\varepsilon_{abc}$\, is the fully anti-symmetric tensor.
\item
	$J_{+},\, J_{-}$\, are elements of \,$\so(3) \otimes_{\Re} \C$.\,
	$J_{+},\, J_{-}$\, satisfy the following equality:
	\begin{equation*}
	(J_{\pm})^{\dagger} \;\; = \;\; J_{\mp}
	\end{equation*}
	$J_{+},\, J_{-},\, J_{3}$\, are generators of \,$\so(3) \otimes_{\Re} \C$,\, and
	they satisfy the following commutation relations:
	\begin{equation*}
	\left[\,J_{3}\,,\,J_{\pm}\,\right] \;=\; \pm\,J_{\pm}\,,
	\quad
	\left[\,J_{+}\,,\,J_{-}\,\right] \;=\; 2\,J_{3}
	\end{equation*}
\item
	Suppose
	\,$\rho : \so(3) \otimes_{\Re} \C \longrightarrow \gl(V)$\,
	is an irreducible finite-dimensional complex representation, and
	\,$v \in V \backslash\{0\}$\, is an eigenvector of \,$\rho(J_{3})$\,
	corresponding to the eigenvalue \,$\lambda \in \Re$;\, thus, \,$\rho(J_{3})(v) \,=\, \lambda\,v$.
	Then, we have:
	\begin{equation*}
	\rho(J_{3})\!\left(\,\rho(J_{+})(\overset{{\color{white}-}}{v})\,\right) \, = \; (\lambda+1)\cdot\rho(J_{+})(v)\,
	\quad\textnormal{and}\quad\;
	\rho(J_{3})\!\left(\,\rho(J_{-})(\overset{{\color{white}-}}{v})\,\right) \, = \; (\lambda-1)\cdot\rho(J_{-})(v)
	\end{equation*}
\item
	\textbf{Casimir operator:}\;\;
	Define
	\,$J^{2}$
	\,$:=$\,
	$(J_{1})^{2} + (J_{2})^{2} + (J_{3})^{2}$
	\,$\in$\
	 $\mathcal{U}\!\left(\so(3) \overset{{\color{white}.}}{\otimes_{\Re}} \C\right)$.\,
	Then, the following equalities of elements of
	\;$\mathcal{U}\!\left(\so(3) \overset{{\color{white}.}}{\otimes_{\Re}} \C\right)$
	hold:
	\begin{equation*}
	J^{2}
	\;\; =\;\;
		(J_{3})^{2} \,-\, J_{3} \,+\, J_{+}J_{-}
	\;\; =\;\;
		(J_{3})^{2} \,+\, J_{3} \,+\, J_{-}J_{+}\,,
	\end{equation*}
	and
	\begin{equation*}
	\left[\,J^{2}\,\overset{{\color{white}1}}{,}\,J_{a}\,\right]
	\;\; = \;\;
		0\,,
	\quad
	\textnormal{for each \,$a \in \{\,1,2,3\,\}$}\,.
	\end{equation*}
	Consequently (by Schur's Lemma, Corollary 4.30, \cite{Hall2015}), 
	\,$J^{2} \in \mathcal{U}\!\left(\so(3) \overset{{\color{white}.}}{\otimes_{\Re}} \C\right)$\,
	acts as a scalar multiple of the identity in every irreducible
	representation\footnote{Furthermore, this scalar $\lambda \in \C$ uniquely determines
	the irreducible representation.
	Look up the classification theory of irreducible finite-dimensional complex representations
	of complex semisimple Lie algebras.
	Key words: Casimir operator, universal enveloping algebra. See Chapters 9 and 10, \cite{Hall2015}.}
	of \,$\mathcal{U}\!\left(\so(3) \overset{{\color{white}.}}{\otimes_{\Re}} \C\right)$;\,
	more precisely, for each irreducible finite-dimensional complex representation
	\,$\rho : \mathcal{U}\!\left(\so(3) \overset{{\color{white}.}}{\otimes_{\Re}} \C\right) \longrightarrow \gl(V)$,\,
	we have \,$\rho(J^{2}) = \lambda \cdot \textnormal{\textbf{1}}_{V}$,\,
	for some \,$\lambda \in \C$.
\end{enumerate}
\end{proposition}

          %%%%% ~~~~~~~~~~~~~~~~~~~~ %%%%%

\vskip 0.5cm
\begin{theorem}
{\color{white}.}\vskip -0.1cm
\noindent
\begin{enumerate}
\item
	The finite-dimensional irreducible representations of $\so(3) \otimes_{\Re} \C$ is parametrized by the set
	\begin{equation*}
	\left\{\,\overset{{\color{white}.}}{0}\,\right\} \,\bigcup\, \dfrac{1}{2} \cdot \N
	\;\; := \;\;
		\left\{\;0 \,,\, \dfrac{1}{2} \,,\, 1 \,,\, \frac{3}{2} \,,\, 2 \,,\, \frac{5}{2} \,,\, \ldots \;\right\},
	\end{equation*}
	of non-negative integer multiples of \,$\dfrac{1}{2}$, in that, for each
	$s$
	$\in$ $\left\{\,\overset{{\color{white}.}}{0}\,\right\} \,\bigcup\, \dfrac{1}{2} \cdot \N$
	$=$ $\left\{\; 0 \,,\, \frac{1}{2}\,,\, 1\,,\, \frac{3}{2}\,,\, 2\,,\, \frac{5}{2}\,,\, \ldots \;\right\}$,
	there exists a unique (up to equivalence) complex representation
	$\rho_{s} : \mathcal{U}(\so(3)\otimes_{\Re}\C) \longrightarrow \textnormal{End}(V_{s})$
	such that
	\begin{equation*}
	\rho_{s}(J^{2}) \; = \; s(s+1)\cdot\textnormal{\textbf{1}}_{V_{s}}.
	\end{equation*}
\item
	$\dim_{\C}(V_{s}) \, = \, 2s + 1$,\, for each
	$s$
	$\in$ $\left\{\,\overset{{\color{white}.}}{0}\,\right\} \,\bigcup\, \dfrac{1}{2} \cdot \N$
	$=$ $\left\{\; 0 \,,\, \frac{1}{2}\,,\, 1\,,\, \frac{3}{2}\,,\, 2\,,\, \frac{5}{2}\,,\, \ldots \;\right\}$.
\item
	For each
	$s$
	$\in$ $\left\{\,\overset{{\color{white}.}}{0}\,\right\} \,\bigcup\, \dfrac{1}{2} \cdot \N$
	$=$ $\left\{\; 0 \,,\, \frac{1}{2}\,,\, 1\,,\, \frac{3}{2}\,,\, 2\,,\, \frac{5}{2}\,,\, \ldots \;\right\}$,\,
	the spectrum
	$\sigma\!\left(\,\overset{{\color{white}-}}{\rho}_{s}(J_{3})\,\right)$
	of the operator $\rho_{s}(J_{3}) \in \textnormal{End}(V_{s})$
	consists of only eigenvalues and is given by:
	\begin{equation*}
	\sigma\!\left(\,\overset{{\color{white}-}}{\rho}_{s}(J_{3})\,\right)
	\;\; = \;\;
		\left\{\;
			-\overset{{\color{white}-}}{s} \,,\, -(s-1), -(s-2)
			\,,\;\, \ldots \,\;,\,
			(s-2) \,,\, (s-1) \,,\, s
			\;\right\},
	\end{equation*}
	and each eigenvalue in 
	$\sigma\!\left(\,\overset{{\color{white}-}}{\rho}_{s}(J_{3})\,\right)$
	has multiplicity one.
\item
	For each
	$s$
	$\in$ $\left\{\,\overset{{\color{white}.}}{0}\,\right\} \,\bigcup\, \dfrac{1}{2} \cdot \N$
	$=$ $\left\{\; 0 \,,\, \frac{1}{2}\,,\, 1\,,\, \frac{3}{2}\,,\, 2\,,\, \frac{5}{2}\,,\, \ldots \;\right\}$,\,
	let \,$v^{(s)}_{k} \in V_{s}\backslash\{0\}$\, be any normalized eigenvector
	of $\rho_{s}(J_{3})$ corresponding to the eigenvalue
	\,$k$ $\in$ $\sigma\!\left(\,\overset{{\color{white}-}}{\rho}_{s}(J_{3})\,\right)$
	$=$ $\left\{\;-\overset{{\color{white}-}}{s} \,,\, -(s-1) \,,\, \;\ldots\;,\, (s-1) \,,\, s\;\right\}$.\,
	Then, 
	\begin{enumerate}
	\item
		the eigenvectors
		\,$v^{(s)}_{-s} \,,\, v^{(s)}_{-(s-1)} \,,\; \ldots \;,\, v^{(s)}_{s-1} \,,\, v^{(s)}_{s}$\,
		form an orthonormal basis for $V_{s}$, and
	\item
		for each \,$k$ $\in$ $\sigma\!\left(\,\overset{{\color{white}-}}{\rho}_{s}(J_{3})\,\right)$
		$=$ $\left\{\;-\overset{{\color{white}-}}{s} \,,\, -(s-1) \,,\, \;\ldots\;,\, (s-1) \,,\, s\;\right\}$,\,
		we have:
		\begin{equation*}
		J_{\pm}\!\left(\,v^{(s)}_{k}\,\right)
		\; = \;
			\sqrt{{\color{white}.}
			s(s+1) - k(k \pm 1)
			{\color{white}.}}
			\,\cdot\,
			v^{(s)}_{k \pm 1}
		\end{equation*}
		In particular, \,$J_{\pm}\!\left(\,v^{(s)}_{\pm s}\,\right) \; = \; 0$.
	\end{enumerate}
\end{enumerate}
\end{theorem}

          %%%%% ~~~~~~~~~~~~~~~~~~~~ %%%%%


%\vskip 1.0cm
%
          %%%%% ~~~~~~~~~~~~~~~~~~~~ %%%%%

\section{Irreducible representations of the Lorentz algebra $\so(1,3)$}
\setcounter{theorem}{0}
\setcounter{equation}{0}

%\cite{vanDerVaart1996}
%\cite{Kosorok2008}

%\renewcommand{\theenumi}{\alph{enumi}}
%\renewcommand{\labelenumi}{\textnormal{(\theenumi)}$\;\;$}
\renewcommand{\theenumi}{\roman{enumi}}
\renewcommand{\labelenumi}{\textnormal{(\theenumi)}$\;\;$}

          %%%%% ~~~~~~~~~~~~~~~~~~~~ %%%%%

% \subsection{Definition \,$\textnormal{O}(1,n)$}

          %%%%% ~~~~~~~~~~~~~~~~~~~~ %%%%%

\vskip 0.5cm
\begin{proposition}[Parametrization of \,$\so(1,3)$]
\mbox{}
\vskip 0.1cm
\noindent
The Lie algebra \,$\so(1,3)$\, of the real Lie group \,$\SO^{\uparrow}(1,3)$\,
admits the following parametrization:
\begin{equation*}
\so{(1,3)}
\; = \;
	\left\{\;
		A
		\overset{{\color{white}.}}{\in}
		\Re^{4 \times 4}
		\;\;\left\vert\;\;
			A \;=\;
			\left(\begin{array}{rrrr}
			        0 &   a_{01} &  a_{02} & a_{03} \\
			a_{01} &           0 &  a_{12} & a_{13} \\
			a_{02} & -a_{12} &           0 & a_{23} \\
			a_{03} & -a_{13} & -a_{23} &          0 \\
			\end{array}\right)
			\right.
		\;\right\}
\end{equation*}
In particular,
\,$\dim_{\Re}\!\left(\overset{{\color{white}.}}{\SO^{\uparrow}(1,3)}\right)$
\,$=$\,
\,$\dim_{\Re}\!\left(\overset{{\color{white}.}}{\so(1,3)}\right)$
\,$=$\, $6$.\,
\end{proposition}
\proof
First, recall that every element of \,$\so(1,3)$\, is of the form \,$\alpha^{\prime}(0) \in \Re^{4 \times 4}$,\,
where
\,$\alpha : (-\varepsilon,\varepsilon) \longrightarrow \SO^{\uparrow}(1,3)$\,
is a smooth map from an open subinterval \,$(-\varepsilon,\varepsilon) \subset \Re$\,
containing \,$0 \in \Re$\, into
\,$\SO^{\uparrow}(1,3)$\,
such that
\,$\alpha(0) = I_{4 \times 4}$.\,
Thus, \,$\alpha(\,\cdot\,)$\, satisfies:
\begin{equation*}
\alpha(t)^{T} \cdot \Qot \cdot \alpha(t) \;\; = \;\; \Qot,
\quad
\textnormal{for each \,$t \in (-\varepsilon,\varepsilon)$}
\end{equation*}
Differentiation with respect to \,$t$\, yields:
\begin{equation*}
\alpha^{\prime}(t)^{T} \cdot \Qot \cdot \alpha(t) \;+\; \alpha(t)^{T} \cdot \Qot \cdot \alpha^{\prime}(t) \;\; = \;\; 0_{4 \times 4}
\end{equation*}
Evaluating at \,$t = 0$\, and recalling \,$\alpha(0) = I_{4 \times 4}$\, yields:
\begin{equation*}
\alpha^{\prime}(0)^{T} \cdot \Qot \;+\; \Qot \cdot \alpha^{\prime}(0) \;\; = \;\; 0_{4 \times 4}
\end{equation*}
Now, write:
\begin{equation*}
\alpha^{\prime}(0)
\;\; = \;\;
	\left(\begin{array}{cccc}
	a_{00} & a_{01} & a_{02} & a_{03}
	\\
	a_{10} & a_{11} & a_{12} & a_{13}
	\\
	a_{20} & a_{21} & a_{22} & a_{23}
	\\
	a_{30} & a_{31} & a_{32} & a_{33}
	\end{array}\right)
\;\; \in \;\;
	\Re^{4 \times 4}
\end{equation*}
Then,
\begin{equation*}
\alpha^{\prime}(0)^{T} \cdot \Qot
\;\; = \;\;
	\left(\begin{array}{cccc}
	a_{00} & a_{10} & a_{20} & a_{30}
	\\
	a_{01} & a_{11} & a_{21} & a_{31}
	\\
	a_{02} & a_{12} & a_{22} & a_{32}
	\\
	a_{03} & a_{13} & a_{23} & a_{33}
	\end{array}\right)
	\cdot
	\left(\begin{array}{rrrr}
	-1 & 0 & 0 & 0
	\\
	0 & 1 & 0 & 0
	\\
	0 & 0 & 1 & 0
	\\
	0 & 0 & 0 & 0
	\end{array}\right)
\;\; = \;\;
	\left(\begin{array}{cccc}
	-\,a_{00} & a_{10} & a_{20} & a_{30}
	\\
	-\,a_{01} & a_{11} & a_{21} & a_{31}
	\\
	-\,a_{02} & a_{12} & a_{22} & a_{32}
	\\
	-\,a_{03} & a_{13} & a_{23} & a_{33}
	\end{array}\right)
\end{equation*}
and
\begin{equation*}
\Qot \cdot \alpha^{\prime}(0)
\;\; = \;\;
	\left(\begin{array}{rrrr}
	-1 & 0 & 0 & 0
	\\
	0 & 1 & 0 & 0
	\\
	0 & 0 & 1 & 0
	\\
	0 & 0 & 0 & 0
	\end{array}\right)
	\cdot
	\left(\begin{array}{cccc}
	a_{00} & a_{01} & a_{02} & a_{03}
	\\
	a_{10} & a_{11} & a_{12} & a_{13}
	\\
	a_{20} & a_{21} & a_{22} & a_{23}
	\\
	a_{30} & a_{31} & a_{32} & a_{33}
	\end{array}\right)
\;\; = \;\;
	\left(\begin{array}{rrrr}
	-\,a_{00} & -\,a_{01} & -\,a_{02} & -\,a_{03}
	\\
	a_{10} & a_{11} & a_{12} & a_{13}
	\\
	a_{20} & a_{21} & a_{22} & a_{23}
	\\
	a_{30} & a_{31} & a_{32} & a_{33}
	\end{array}\right)
\end{equation*}
Thus,
\begin{equation*}
\alpha^{\prime}(0)^{T} \cdot \Qot \;+\; \Qot \cdot \alpha^{\prime}(0)
\;\; = \;\;
	\left(\begin{array}{cccc}
	-\,2\,a_{00} & a_{10}\,-\,a_{01} & a_{20}\,-\,a_{02} & a_{30}\,-\,a_{03}
	\\
	-\,a_{01} \,+\, a_{10} & 2\,a_{11} & a_{21}\,+\,a_{12} & a_{31}\,+\,a_{13}
	\\
	-\,a_{02} \,+\, a_{20} & a_{12}\,+\,a_{21} & 2\,a_{22} & a_{32}\,+\,a_{23}
	\\
	-\,a_{03} \,+\, a_{30} & a_{13}\,+\,a_{31} & a_{23}\,+\,a_{32} & 2\,a_{33}
	\end{array}\right)
\end{equation*}
Consequently,
\begin{equation*}
\alpha^{\prime}(0)^{T} \cdot \Qot \;+\; \Qot \cdot \alpha^{\prime}(0) \;\; = \;\; 0_{4 \times 4}
\quad\Longleftrightarrow\quad
\left\{\begin{array}{c}
	a_{00} \,=\, a_{11} \,=\, a_{22} \,=\, a_{33} \,=\, 0\,,
	\\
	a_{10} \,=\, a_{01}\,, a_{20} \,=\, a_{02}\,, a_{30} \,=\, a_{03}\,,
	\\
	a_{21} \,=\, -\,a_{12}\,, a_{31} \,=\, -\,a_{13}\,, a_{32} \,=\, -\,a_{23}\,.
	\end{array}\right.
\end{equation*}
This proves:
\begin{equation*}
\so{(1,3)}
\;\; \subset \;\;
	\left\{\;
		A
		\overset{{\color{white}.}}{\in}
		\Re^{4 \times 4}
		\;\;\left\vert\;\;
			A \;=\;
			\left(\begin{array}{rrrr}
			        0 &   a_{01} &  a_{02} & a_{03} \\
			a_{01} &           0 &  a_{12} & a_{13} \\
			a_{02} & -a_{12} &           0 & a_{23} \\
			a_{03} & -a_{13} & -a_{23} &          0 \\
			\end{array}\right)
			\right.
		\;\right\}
\end{equation*}
For the reverse inclusion, consider:
\begin{equation*}
A
\;\; = \;\;
	\left(\begin{array}{rrrr}
	        0 &   a_{01} &  a_{02} & a_{03} \\
		a_{01} &           0 &  a_{12} & a_{13} \\
		a_{02} & -a_{12} &           0 & a_{23} \\
		a_{03} & -a_{13} & -a_{23} &          0 \\
		\end{array}\right)
\;\; \in \;\;
	\Re^{4 \times 4}
\end{equation*}
The required reverse inclusion amounts to the statement that \,$A \in \so(1,3)$.\,
Now, define:
\begin{equation*}
\alpha : \Re \longrightarrow \Re^{4 \times 4} : t \longmapsto \exp\!\left(\,t \cdot \overset{{\color{white}.}}{A}\,\right)
\end{equation*}
Then, \,$\alpha^{\prime}(0) \,=\, A$.\,
Thus, to show that \,$A \in \so(1,3)$,\, it remains only to establish that
\,$\alpha(t) \in \SO^{\uparrow}(1,3)$.\,

\vskip 0.2cm
\noindent
To this end, first note that \,$A \in \Re^{4 \times 4}$\, satisfies:
\,$A^{T} \cdot \Qot + \Qot \cdot A \,=\, 0_{4 \times 4}$,\,
which implies:
\begin{equation*}
A^{T} \cdot \Qot \,=\, -\,\Qot \cdot A\,,
\end{equation*}
which in turn implies:
\begin{equation*}
\left(\,A^{T}\,\right)^{k} \cdot \Qot
\,=\,
	(-1) \cdot \left(\,A^{T}\,\right)^{k-1} \cdot \Qot \cdot A
\,=\,
	(-1)^{2} \cdot \left(\,A^{T}\,\right)^{k-2} \cdot \Qot \cdot A^{2}
\,=\,
	\cdots
\,=\,
	(-1)^{k} \cdot \Qot \cdot A^{k}
\end{equation*}
Therefore,
\begin{equation*}
\exp\!\left(\,t\overset{{\color{white}.}}{A}\,\right)^{T} \cdot \Qot \cdot \exp\!\left(\,t\overset{{\color{white}.}}{A}\,\right)
\,=\,
	\exp\!\left(\,\overset{{\color{white}.}}{t}A^{T}\,\right) \cdot \Qot \cdot \exp\!\left(\,t\overset{{\color{white}.}}{A}\,\right)
\,=\,
	\Qot \cdot \exp\!\left(\,-\,\overset{{\color{white}.}}{t}A\,\right) \cdot \exp\!\left(\,t\overset{{\color{white}.}}{A}\,\right)
\,=\,
	\Qot\,,
\end{equation*}
where the second last equality follows from:
\begin{eqnarray*}
\Qot \cdot \exp\!\left(\,-\,\overset{{\color{white}.}}{t}A\,\right)
& = &
	\Qot \cdot \left(\;\overset{\infty}{\underset{k=0}{\sum}}\,\dfrac{(-1)^{k}t^{k}A^{k}}{k!}\,\right)
\;\; = \;\;
	\left(\;\overset{\infty}{\underset{k=0}{\sum}}\,\dfrac{t^{k}\cdot(-1)^{k}\,\Qot\,A^{k}}{k!}\,\right)
\\
& = &
	\left(\;\overset{\infty}{\underset{k=0}{\sum}}\,\dfrac{t^{k}\cdot(A^{T})^{k}\,\Qot}{k!}\,\right)
\;\; = \;\;
	\left(\;\overset{\infty}{\underset{k=0}{\sum}}\,\dfrac{t^{k}\,(A^{T})^{k}}{k!}\,\right)
	\cdot\Qot
\\
& = &
	\exp\!\left(\,\overset{{\color{white}.}}{t}A^{T}\,\right) \cdot \Qot
\end{eqnarray*}
Thus, we now see that
\,$\alpha(t) \in \textnormal{O}(1,3)$,\, for each \,$t \in \Re$.\,
Next, recall that
\begin{equation*}
\det\!\left(\,\exp(t\overset{{\color{white}.}}{X})\,\right)
\,=\,
	e^{\trace(tX)}\,,
\quad
\textnormal{for each \,$t \in \Re$\, and \,$X \in \Re^{4 \times 4}$}
\end{equation*}
Hence,
\begin{equation*}
\det\!\left(\, \alpha(\overset{{\color{white}.}}{t}) \,\right)
\,=\,
	\det\!\left(\,\exp(t\overset{{\color{white}.}}{A})\,\right)
\,=\,
	e^{\trace(tA)}
\,=\,
	e^{0}
\,=\,
	1
\end{equation*}
Thus, we see furthermore that
\begin{equation*}
\alpha(t) \,\in\, \SO(1,3).
\end{equation*}
However, we also have \,$\alpha(0) \,=\, \exp(0\cdot\!A) \,=\, I_{4 \times 4}$.\,
Continuity of \,$\alpha(\,\cdot\,)$\, now implies that \,$\alpha(\,\cdot\,)$\,
must map all of \,$\Re$\, into the identity component of \,$\SO(1,3)$,\,
i.e., \,$\alpha(t) \in \SO^{\uparrow}(1,3)$,\,
for each \,$t \in \Re$.\,
This proves the reverse inclusion:
\begin{equation*}
\so{(1,3)}
\;\; \supset \;\;
	\left\{\;
		A
		\overset{{\color{white}.}}{\in}
		\Re^{4 \times 4}
		\;\;\left\vert\;\;
			A \;=\;
			\left(\begin{array}{rrrr}
			        0 &   a_{01} &  a_{02} & a_{03} \\
			a_{01} &           0 &  a_{12} & a_{13} \\
			a_{02} & -a_{12} &           0 & a_{23} \\
			a_{03} & -a_{13} & -a_{23} &          0 \\
			\end{array}\right)
			\right.
		\;\right\}
\end{equation*}
and thus completes the proof of the Proposition.
\qed

          %%%%% ~~~~~~~~~~~~~~~~~~~~ %%%%%

\vskip 0.5cm
\begin{corollary}[Generators of \,$\so(1,3)$\, \& their commutation relations]
\mbox{}
\vskip 0.1cm
\noindent
Define the following six matrices with real entries:
\begin{equation*}
R_{23}
\; := \,
	\left(\,\begin{array}{rrrr}
	0 & {\color{white}-}0 & {\color{white}-}0 & {\color{white}-}0 \\
	0 & 0 & 0 & 0 \\
	0 & 0 & 0 & -1 \\
	0 & 0 & 1 & 0 \\
	\end{array}\right),
\;\;
R_{31}
\; := \,
	\left(\,\begin{array}{rrrr}
	0 & {\color{white}-}0 & {\color{white}-}0 & {\color{white}-}0 \\
	0 & 0 & 0 & 1 \\
	0 & 0 & 0 & 0 \\
	0 & -1 & 0 & 0 \\
	\end{array}\right),
\;\;
R_{12}
\; := \,
	\left(\,\begin{array}{rrrr}
	0 & {\color{white}-}0 & {\color{white}-}0 & {\color{white}-}0 \\
	0 & 0 & -1 & 0 \\
	0 & 1 & 0 & 0 \\
	0 & 0 & 0 & 0 \\
	\end{array}\right)
\end{equation*}
\begin{equation*}
B_{01}
\; := \,
	\left(\,\begin{array}{rrrr}
	0 & {\color{white}-}1 & {\color{white}-}0 & {\color{white}-}0 \\
	1 & 0 & 0 & 0 \\
	0 & 0 & 0 & 0 \\
	0 & 0 & 0 & 0 \\
	\end{array}\right),
\;\;
B_{02}
\; := \,
	\left(\,\begin{array}{rrrr}
	0 & {\color{white}-}0 & {\color{white}-}1 & {\color{white}-}0 \\
	0 & 0 & 0 & 0 \\
	1 & 0 & 0 & 0 \\
	0 & 0 & 0 & 0 \\
	\end{array}\right),
\;\;
B_{03}
\; := \,
	\left(\,\begin{array}{rrrr}
	0 & {\color{white}-}0 & {\color{white}-}0 & {\color{white}-}1 \\
	0 & 0 & 0 & 0 \\
	0 & 0 & 0 & 0 \\
	1 & 0 & 0 & 0 \\
	\end{array}\right)
\end{equation*}
\vskip 0.3cm
\noindent
Define also the following six matrices with complex entries:
\vskip -0.9cm
\mbox{}
\begin{multicols}{2}
	\begin{minipage}{6.0cm}
	\begin{eqnarray*}
	J_{1}
	& := &
		\i \cdot R_{23}
	\;\; = \;\;
		\left(\,\begin{array}{rrrr}
		0 & {\color{white}-}0 & {\color{white}-}0 & {\color{white}-}0 \\
		0 & 0 & 0 & 0 \\
		0 & 0 & 0 & -\i \\
		0 & 0 & \i & 0 \\
		\end{array}\right),
	\\
	J_{2}
	& := &
		\i \cdot R_{31}
	\;\; = \;\;
		\left(\,\begin{array}{rrrr}
		0 & {\color{white}-}0 & {\color{white}-}0 & {\color{white}-}0 \\
		0 & 0 & 0 &  \i \\
		0 & 0 & 0 & 0 \\
		0 & -\i & 0 & 0 \\
		\end{array}\right),
	\\
	J_{3}
	& := &
		\i \cdot R_{12}
	\;\; = \;\;
		\left(\,\begin{array}{rrrr}
		0 & {\color{white}-}0 & {\color{white}-}0 & {\color{white}-}0 \\
		0 & 0 & -\i & 0 \\
		0 & \i & 0 & 0 \\
		0 & 0 & 0 & 0 \\
		\end{array}\right),
	\end{eqnarray*}
	\end{minipage}
\columnbreak
	\begin{minipage}{11.5cm}
	\begin{eqnarray*}
	K_{1}
	& := &
		\i \cdot B_{01}
	\;\; = \;\;
		\left(\,\begin{array}{rrrr}
		0 & {\color{white}-}\i & {\color{white}-}0 & {\color{white}-}0 \\
		\i & 0 & 0 & 0 \\
		0 & 0 & 0 & 0 \\
		0 & 0 & 0 & 0 \\
		\end{array}\right),
	\\
	K_{2}
	& := &
		\i \cdot B_{02}
	\;\; = \;\;
		\left(\,\begin{array}{rrrr}
		0 & {\color{white}-}0 & {\color{white}-}\i & {\color{white}-}0 \\
		0 & 0 & 0 & 0 \\
		\i & 0 & 0 & 0 \\
		0 & 0 & 0 & 0 \\
		\end{array}\right),
	\\
	K_{3}
	& := &
		\i \cdot B_{03}
	\;\; = \;\;
		\left(\,\begin{array}{rrrr}
		0 & {\color{white}-}0 & {\color{white}-}0 & {\color{white}-}\i \\
		0 & 0 & 0 & 0 \\
		0 & 0 & 0 & 0 \\
		\i & 0 & 0 & 0 \\
		\end{array}\right).
	\end{eqnarray*}
	\end{minipage}
\end{multicols}
\begin{multicols}{2}
	\begin{minipage}{8cm}
	\begin{eqnarray*}
	N^{+}_{1}
	\; := \;
		\dfrac{1}{2}\left(\,J_{1} + \i \, K_{1}\,\right)
	\; = \;
		\dfrac{1}{2}\,\cdot
		\left(\!\begin{array}{rrrr}
		 0 & {\color{black}-}1 & {\color{white}-}0 & {\color{white}-}0 \\
		-1 & 0 & 0 & 0 \\
		 0 & 0 & 0 & -\i \\
		 0 & 0 & \i & 0 \\
		\end{array}\right),
	\quad\quad{\color{white}.}
	\\
	N^{+}_{2}
	\; := \;
		\dfrac{1}{2}\left(\,J_{2} + \i \, K_{2}\,\right)
	\; = \;
		\dfrac{1}{2}\,\cdot
		\left(\!\begin{array}{rrrr}
		 0 & {\color{white}-}0 & {\color{black}-}1 & {\color{white}-}0 \\
		 0 & 0 & 0 & \i \\
		-1 & 0 & 0 & 0 \\
		 0 & -\i & 0 & 0 \\
		\end{array}\right),
	\quad\quad{\color{white}.}
	\\
	N^{+}_{3}
	\; := \;
		\dfrac{1}{2}\left(\,J_{3} + \i \, K_{3}\,\right)
	\; = \;
		\dfrac{1}{2}\,\cdot
		\left(\!\begin{array}{rrrr}
		 0 & {\color{white}-}0 & {\color{white}-}0 & {\color{black}-}1 \\
		 0 & 0 & -\i & 0 \\
		 0 &  \i & 0 & 0 \\
		-1 & 0 & 0 & 0 \\
		\end{array}\right),
	\quad\quad{\color{white}.}
	\end{eqnarray*}
	\end{minipage}
\columnbreak
	\begin{minipage}{12.0cm}
	\begin{eqnarray*}
	N^{-}_{1}
	\; := \;
		\dfrac{1}{2}\left(\,J_{1} - \i \, K_{1}\,\right)
	\; = \;
		\dfrac{1}{2}\,\cdot
		\left(\!\!\!\begin{array}{rrrr}
		{\color{white}-}0 & {\color{white}-}1 & {\color{white}-}0 & {\color{white}-}0 \\
		1 & 0 & 0 & 0 \\
		0 & 0 & 0 & -\i \\
		0 & 0 & \i & 0 \\
		\end{array}\right),
	\\
	N^{-}_{2}
	\; := \;
		\dfrac{1}{2}\left(\,J_{2} - \i \, K_{2}\,\right)
	\; = \;
		\dfrac{1}{2}\,\cdot
		\left(\!\!\!\begin{array}{rrrr}
		{\color{white}-}0 & {\color{white}-}0 & {\color{white}-}1 & {\color{white}-}0 \\
		0 & 0 & 0 & \i \\
		1 & 0 & 0 & 0 \\
		0 & -\i & 0 & 0 \\
		\end{array}\right),
	\\
	N^{-}_{3}
	\; := \;
		\dfrac{1}{2}\left(\,J_{3} - \i \, K_{3}\,\right)
	\; = \;
		\dfrac{1}{2}\,\cdot
		\left(\!\!\!\begin{array}{rrrr}
		{\color{white}-}0 & {\color{white}-}0 & {\color{white}-}0 & {\color{white}-}1 \\
		0 & 0 & -\i & 0 \\
		0 &  \i & 0 & 0 \\
		1 & 0 & 0 & 0 \\
		\end{array}\right).
	\end{eqnarray*}
	\end{minipage}
\end{multicols}
\noindent
Then, the following statements are true:
\begin{enumerate}
\item
	The matrices
	\,$R_{23},\; R_{31},\; R_{12},\; B_{01},\; B_{02},\; B_{03}$\,
	are elements of
	\,$\so(1,3)$,\, and they form a basis for \,$\so(1,3)$.\,
	The
	\,$15 \,= \left(\begin{array}{c}6 \\ 2\end{array}\right)$\,
	commutation relations satisfied by
	\,$R_{23},\, R_{31},\, R_{12},\, B_{01},\, B_{02},\, B_{03}$\,
	are:
	\begin{equation*}
	\begin{array}{lll}
	\left[\,R_{23}\,,\,R_{31}\,\right] \,=\, +\,R_{12}, &
	\left[\,R_{12}\,,\,R_{23}\,\right] \,=\, +\,R_{31}, &
	\left[\,R_{31}\,,\,R_{12}\,\right] \,=\, +\,R_{23},
	\\ \\
	\left[\,B_{01}\,,\,B_{02}\,\right] \,=\, -\,R_{12}, &
	\left[\,B_{03}\,,\,B_{01}\,\right] \,=\, -\,R_{31}, &
	\left[\,B_{02}\,,\,B_{03}\,\right] \,=\, -\,R_{23},
	\\ \\
	\left[\,R_{23}\,,\,B_{01}\,\right] \,=\, {\color{white}-}\,0,\;\;\;\; &
	\left[\,R_{23}\,,\,B_{02}\,\right] \,=\, +\,B_{03}, &
	\left[\,R_{23}\,,\,B_{03}\,\right] \,=\, -\,B_{02}, 
	\\
	\left[\,R_{31}\,,\,B_{01}\,\right] \,=\, -\,B_{03}, &
	\left[\,R_{31}\,,\,B_{02}\,\right] \,=\, {\color{white}-}\,0,\;\;\;\; &
	\left[\,R_{31}\,,\,B_{03}\,\right] \,=\, +\,B_{01},
	\\
	\left[\,R_{12}\,,\,B_{01}\,\right] \,=\, +\,B_{02}, &
	\left[\,R_{12}\,,\,B_{02}\,\right] \,=\, -\,B_{01}, &
	\left[\,R_{12}\,,\,B_{03}\,\right] \,=\, {\color{white}-}\,0,\;\;\;\;
	\end{array}
	\end{equation*}
\item
	The matrices
	\,$J_{1},\, J_{2},\, J_{3},\, K_{1},\, K_{2},\, K_{3}$\,
	are elements of
	\,$\so(1,3) \otimes_{\Re} \C$,\, and they form a basis for \,$\so(1,3) \otimes_{\Re} \C$.\,
	The
	\,$15 \,= \left(\begin{array}{c}6 \\ 2\end{array}\right)$\,
	commutation relations satisfied by
	\,$J_{1},\, J_{2},\, J_{3},\, K_{1},\, K_{2},\, K_{3}$\,
	are:
	\begin{equation*}
	\begin{array}{lll}
	\left[\,\;J_{1}\,,\,\;J_{2}\,\right] \,=\, + \, \i \, J_{3}, &
	\left[\,\;J_{3}\,,\,\;J_{1}\,\right] \,=\, + \, \i \, J_{2}, &
	\left[\,\;J_{2}\,,\,\;J_{3}\,\right] \,=\, + \, \i \, J_{1},
	\\ \\
	\left[\,K_{1}\,,\,K_{2}\,\right] \,=\, - \, \i \, J_{3}, &
	\left[\,K_{3}\,,\,K_{1}\,\right] \,=\, - \, \i \, J_{2}, &
	\left[\,K_{2}\,,\,K_{3}\,\right] \,=\, - \, \i \, J_{1},
	\\ \\
	\left[\,\;J_{1}\,,\,K_{1}\,\right] \,=\, {\color{white}-}\,0,\;\;\;\; &
	\left[\,\;J_{1}\,,\,K_{2}\,\right] \,=\, + \, \i \, K_{3}, &
	\left[\,\;J_{1}\,,\,K_{3}\,\right] \,=\, - \, \i \, K_{2},
	\\
	\left[\,\;J_{2}\,,\,K_{1}\,\right] \,=\, - \, \i \, K_{3}, &
	\left[\,\;J_{2}\,,\,K_{2}\,\right] \,=\, {\color{white}-}\,0,\;\;\;\; &
	\left[\,\;J_{2}\,,\,K_{3}\,\right] \,=\, + \, \i \, K_{1},
	\\
	\left[\,\;J_{3}\,,\,K_{1}\,\right] \,=\, + \, \i \, K_{2}, &
	\left[\,\;J_{3}\,,\,K_{2}\,\right] \,=\, - \, \i \, K_{1}, &
	\left[\,\;J_{3}\,,\,K_{3}\,\right] \,=\, {\color{white}-}\,0,\;\;\;\;
	\end{array}
	\end{equation*}
\item
	The matrices
	\,$N^{\pm}_{a}$,\, for \,$a \,\in\, \{\,1,2,3\,\}$,
	are elements of
	\,$\so(1,3) \otimes_{\Re} \C$,\, and they form a basis for \,$\so(1,3) \otimes_{\Re} \C$.\,
	The elements
	\,$N^{\pm}_{a}$,\, for \,$a \,\in\, \{\,1,2,3\,\}$,
	satisfy the following commutation relations:
	\begin{equation*}
	\left[\,N^{\pm}_{a}\,,\,N^{\pm}_{b}\,\right]
	\;\;=\;\;
		\i \cdot \overset{3}{\underset{c\,=\,1}{\sum}} \;\varepsilon_{abc} \, N^{\pm}_{c}\,,
	\quad
	\textnormal{for each \,$a, b, c \,\in\, \{\,1, 2, 3\,\}$}
	\end{equation*}
	\begin{equation*}
	\left[\,N^{+}_{a}\,,\,N^{-}_{b}\,\right] \;\;=\;\; 0\,,
	\quad
	\textnormal{for each \,$a, b \,\in\, \{\,1, 2, 3\,\}$}
	\end{equation*}
	Thus,
	\,$\so(1,3) \otimes_{\Re} \C$\,
	contains two commuting copies of
	\,$\su(2) \otimes_{\Re} \C$.\,
\end{enumerate}
\end{corollary}
\proof

\qed

          %%%%% ~~~~~~~~~~~~~~~~~~~~ %%%%%

%\subsection{Generators of \,$\mathfrak{su}(2)$}
%
%          %%%%% ~~~~~~~~~~~~~~~~~~~~ %%%%%
%
%\begin{proposition}[Characterizations of \,$\mathfrak{sl}(n)$, \,$\mathfrak{u}(n)$, and $\mathfrak{su}(n)$]
%\mbox{}
%\vskip 0.1cm
%\begin{enumerate}
%\item
%	\begin{equation*}
%	\mathfrak{sl}(n,\C)
%	\; = \;
%		\left\{\;
%			X \,\in\, \mathfrak{gl}(n,\C) \,=\, \C^{n \times n}
%			\;\left\vert\;\,
%				\textnormal{trace}(X) = \overset{{\color{white}1}}{0}
%				\right.
%			\,\right\}
%	\end{equation*}
%\item
%	\begin{equation*}
%	\mathfrak{u}(n)
%	\; = \;
%		\left\{\;
%			X \,\in\, \mathfrak{gl}(n,\C) \,=\, \C^{n \times n}
%			\;\left\vert\;\,
%				X + X^{\dagger} = \overset{{\color{white}1}}{0}
%				\right.
%			\,\right\}
%	\end{equation*}
%\item
%	\begin{equation*}
%	\mathfrak{su}(n)
%	\; = \;
%		\left\{\;
%			X \,\in\, \mathfrak{gl}(n,\C) \,=\, \C^{n \times n}
%			\;\left\vert\;\,
%				\begin{array}{c}
%				X + X^{\dagger} = \overset{{\color{white}1}}{0}
%				\\
%				\textnormal{trace}(X) = \overset{{\color{white}1}}{0}
%				\end{array}
%				\right.
%			\,\right\}
%	\end{equation*}
%\end{enumerate}
%\end{proposition}
%\proof
%\begin{enumerate}
%\item
%	We invoke the fact that \,$\det(e^{\,t\,\cdot\,X}) \,=\, e^{\,t\,\cdot\,\textnormal{trace}(X)}$,
%	for each \,$X \in \C^{n \times n}$.
%	Thus,
%	\begin{eqnarray*}
%	&&
%		X \,\in\, \mathfrak{sl}(n,\C)
%		\quad\Longrightarrow\quad
%		e^{\,t\cdot\,X} \,\in\, \textnormal{SL}(n,\C)
%		\quad\Longrightarrow\quad
%		\det\!\left(\,e^{\,t\cdot\,X}\,\right) \,=\, 1
%	\\
%	& \Longrightarrow\quad &
%		\textnormal{trace}(X)
%		\; = \;
%			\left.\dfrac{\d}{\d\,t}\right\vert_{t=0}\left(\,\overset{{\color{white}1}}{e^{\,t\,\cdot\,\textnormal{trace}(X)}}\,\right)
%		\; = \;
%			\left.\dfrac{\d}{\d\,t}\right\vert_{t=0}\left(\,\overset{{\color{white}1}}{\det(e^{\,t\,\cdot\,X})}\,\right)
%		\; = \;
%			\left.\dfrac{\d}{\d\,t}\right\vert_{t=0}\left(\,\overset{{\color{white}1}}{1}\,\right)
%		\; = \;
%			0
%	\end{eqnarray*}
%	Conversely, suppose \,$\textnormal{trace}(X) = 0$.\,
%	Then, \,$\det(e^{\,t\,\cdot\,X}) \,=\, e^{\,t\,\cdot\,\textnormal{trace}(X)} \,=\, e^{\,t\,\cdot\,0} \,=\, 1$,\,
%	which implies that \,$e^{\,t\,\cdot\,X} \,\in\, \textnormal{SL}(n,\C)$,\, hence \,$X \,\in\, \mathfrak{sl}(n,\C)$.
%	This completes the proof of the equality (of sets) in question.
%\item
%	\begin{eqnarray*}
%	&&
%		X \,\in\, \mathfrak{u}(n)
%		\quad\Longrightarrow\quad
%		e^{\,t\,\cdot\,X} \,\in\, \textnormal{U}(n)
%	\\
%	& \Longrightarrow\quad &
%		I_{n}
%			\,=\, \left(\,e^{\,t\,\cdot\,X}\,\right)^{\!\dagger} \cdot \left(\,e^{\,t\,\cdot\,X}\,\right)
%			\,=\, \left(\,e^{\,t\,\cdot\,X^{\dagger}}\,\right) \cdot \left(\,e^{\,t\,\cdot\,X}\,\right)
%			\,=\, e^{\,t\,\cdot\,(X^{\dagger}+X)}
%	\\
%	& \Longrightarrow\quad &
%		X \,+\, X^{\dagger}
%		\; = \;
%			\left.\dfrac{\d}{\d\,t}\right\vert_{t=0}\left(\,\overset{{\color{white}1}}{e^{\,t\,\cdot\,(X+X^{\dagger})}}\,\right)
%		\; = \;
%			\left.\dfrac{\d}{\d\,t}\right\vert_{t=0}\left(\,\overset{{\color{white}1}}{I_{n}}\,\right)
%		\; = \;
%			0
%	\end{eqnarray*}
%	Conversely, suppose \,$X + X^{\dagger} \,=\, 0$.\,
%	Then, \,$I_{n}$
%	\,$=$\, $e^{\,0_{n \times n}}$
%	\,$=$\, $e^{\,t\,\cdot(X^{\dagger}+X)}$
%	\,$=\, \cdots \,=$\, $\left(e^{\,t\,\cdot\,X}\right)^{\!\dagger}\cdot\left(e^{\,t\,\cdot\,X}\right)$,\,
%	which implies that \,$e^{\,t\,\cdot\,X} \,\in\, \textnormal{U}(n)$,\, hence \,$X \,\in\, \mathfrak{u}(n)$.
%	This completes the proof of the equality (of sets) in question.
%\item
%	Immediate by the preceding two statements.
%\end{enumerate}
%\qed
%
%\vskip 0.5cm
%\begin{proposition}[Generators of \,$\mathfrak{su}(2)$]
%\mbox{}
%\vskip 0.1cm
%\noindent
%Let \,$\sigma_{1},\, \sigma_{2},\, \sigma_{3} \,\in\, \C^{2 \times 2}$\, be the \textbf{Pauli spin matrices}, i.e.,
%\begin{equation*}
%\sigma_{1} \,=\, \sigma_{x} \,:=\, \left(\begin{array}{cc} 0 & 1 \\ 1 & 0 \end{array}\right),
%\quad
%\sigma_{2} \,=\, \sigma_{y} \,:=\, \left(\begin{array}{rr} 0 & -\i \\ \i & 0 \end{array}\right),
%\quad
%\sigma_{3} \,=\, \sigma_{z} \,:=\, \left(\begin{array}{rr} 1 & 0 \\ 0 & -1 \end{array}\right).
%\end{equation*}
%Define \,$J_{1},\, J_{2},\, J_{3},\, S_{+},\, S_{-},\, S_{3} \,\in\, \C^{2 \times 2}$\, as follows:
%\begin{equation*}
%J_{1} \,:=\, \dfrac{\i}{2}\cdot\sigma_{1} \,=\, \dfrac{\i}{2}\cdot\left(\begin{array}{cc} 0 & 1 \\ 1 & 0 \end{array}\right),
%\quad
%J_{2} \,:=\, \mathbf{{\color{red}-}}\,\dfrac{\i}{2}\cdot\sigma_{2} \,=\, \dfrac{1}{2}\cdot\left(\begin{array}{rr} 0 & -1 \\ 1 & 0 \end{array}\right),
%\quad
%J_{3} \,:=\, \dfrac{\i}{2}\cdot\sigma_{3} \,=\, \dfrac{\i}{2}\cdot\left(\begin{array}{rr} 1 & 0 \\ 0 & -1 \end{array}\right),
%\end{equation*}
%\begin{equation*}
%S_{+} \,:=\, \dfrac{1}{\i}\left(\,J_{1} + \i\,J_{2}\,\right) \,=\, \left(\begin{array}{cc} 0 & 0 \\ 1 & 0 \end{array}\right),
%\quad
%S_{-} \,:=\, \dfrac{1}{\i}\left(\,J_{1} - \i\,J_{2}\,\right) \,=\, \left(\begin{array}{rr} 0 & 1 \\ 0 & 0 \end{array}\right),
%\quad
%S_{3} \,:=\, \i\cdot J_{3} \,=\, \dfrac{1}{2}\cdot\left(\begin{array}{rr} -1 & 0 \\ 0 & 1 \end{array}\right).
%\end{equation*}
%Then, the following statements are true:
%\begin{enumerate}
%\item
%	$J_{1},\, J_{2},\, J_{3} \,\in\, \mathfrak{su}(2)$\,
%\item
%	$J_{1},\, J_{2},\, J_{3}$\,
%	form a set of generators for the (real) Lie algebra \,$\mathfrak{su}(2)$\, of the (real) Lie group \,$\textnormal{SU}(2)$.
%\item
%	$J_{1},\, J_{2},\, J_{3}$\, satisfy the following commutation relations:
%	\begin{equation*}
%	\left[\,J_{a}\,,\,J_{b}\,\right] \;\; = \;\; \overset{3}{\underset{c\,=\,1}{\sum}}\;\epsilon_{abc}\,J_{c}\,,
%	\quad
%	\textnormal{for \,$a, b = 1,2,3$}.
%	\end{equation*}
%\item
%	$S_{+},\, S_{-},\, S_{3} \,\in\, \mathfrak{su}(2) \otimes_{\Re} \C$,\,
%	where
%	\,$\mathfrak{su}(2) \otimes_{\Re} \C$\,
%	is the complexification of (the real Lie algebra)
%	\,$\mathfrak{su}(2)$.
%\item
%	$S_{+},\; S_{-},\; S_{3}$\, satisfy the following commutation relations:
%	\begin{equation*}
%	\left[\,S_{+}\,,\,S_{-}\,\right] \, = \, 2\,S_{3}\,,
%	\quad
%	\left[\,S_{3}\,,\,S_{\pm}\,\right] \, = \, \pm\,S_{\pm}
%	\end{equation*}
%\item
%	Suppose
%	\begin{itemize}
%	\item
%		$V$\, is a complex vector space,
%	\item
%		$\rho : \mathfrak{su}(2) \otimes_{\Re} \C \longrightarrow \textnormal{End}(V)$\,
%		is a Lie algebra representation, and
%	\item	
%		$v \in V$\, and \,$\lambda \in \C$\, together satisfy \,$\rho(S_{3})(v) = \lambda \cdot v$.
%	\end{itemize}	
%	Then, \,$\rho(S_{+})(v) \,\in\, V$\, satisfies:
%	\begin{equation*}
%	\rho(S_{3})\!\left(\,\rho(\overset{{\color{white}.}}{S}_{+})(v)\,\right)
%	\; = \;
%		(\lambda+1) \cdot \rho(S_{+})(v)\,
%	\end{equation*}
%	and
%	\,$\rho(S_{-})(v) \,\in\, V$\, satisfies:
%	\begin{equation*}
%	\rho(S_{3})\!\left(\,\rho(\overset{{\color{white}.}}{S}_{-})(v)\,\right)
%	\; = \;
%		(\lambda-1) \cdot \rho(S_{-})(v)\,
%	\end{equation*}
%\end{enumerate}
%\end{proposition}
%\proof
%The Corollary follows straightforwardly by direct computations.
%\qed
%
%          %%%%% ~~~~~~~~~~~~~~~~~~~~ %%%%%
%
%\vskip 0.5cm
%\begin{definition}[$\textnormal{O}(n)$ and $\textnormal{SO}(n)$]
%\mbox{}
%\vskip 0.1cm
%\noindent
%The \textbf{orthogonal group} is defined as follows:
%\begin{equation*}
%\textnormal{O}(n)
%\; := \;
%	\left\{\;\,
%		g \overset{{\color{white}.}}{\in} \textnormal{GL}(n,\Re)
%		\;\left\vert\;\,
%			g^{T} \cdot g = I_{n}
%			\right.
%		\;\right\}
%\end{equation*}
%The \textbf{special orthogonal group} is defined as follows:
%\begin{equation*}
%\textnormal{SO}(n)
%\; := \;
%	\left\{\;\,
%		g \overset{{\color{white}.}}{\in} \textnormal{GL}(n,\Re)
%		\;\left\vert\;\,
%			g^{T} \cdot g = I_{n}\,,
%			\;
%			\textnormal{det}(g) = 1
%			\right.
%		\;\right\}
%\end{equation*}
%\end{definition}
%
%          %%%%% ~~~~~~~~~~~~~~~~~~~~ %%%%%
%
%\begin{proposition}[Lie algebras of $\textnormal{O}(n)$ and $\textnormal{SO}(n)$]
%\begin{eqnarray*}
%\mathfrak{o}(n)
%& = &
%	\left\{\;\,
%		X \overset{{\color{white}.}}{\in} \mathfrak{gl}(n,\Re)
%		\;\left\vert\;\,
%			X^{T} = -X
%			\right.
%		\;\right\}
%\\
%\mathfrak{so}(n)
%& = &
%	\left\{\;\,
%		X \overset{{\color{white}.}}{\in} \mathfrak{gl}(n,\Re)
%		\;\left\vert\;\,
%			X^{T} = -X\,,
%			\;
%			\textnormal{trace}(X) = 0
%			\right.
%		\;\right\}
%\end{eqnarray*}
%\end{proposition}
%
%          %%%%% ~~~~~~~~~~~~~~~~~~~~ %%%%%
%
%\subsection{Generators of \;$\textnormal{SO}(3)$\, and \,$\mathfrak{so}(3)$}
%
%          %%%%% ~~~~~~~~~~~~~~~~~~~~ %%%%%
%
%\vskip 0.1cm
%\noindent
%\textbf{Euler matrices}
%\begin{equation*}
%R_{1}(\phi)
%\; := \;
%	\left(\,
%		\begin{array}{ccc}
%			{\color{white}-}1 & {\color{white}-}0 & {\color{white}-}0 \\
%			{\color{white}-}0 & {\color{white}-}\cos\phi & -\sin\phi \\
%			{\color{white}-}0 & {\color{white}-}\sin\phi & {\color{white}-}\cos\phi \\
%			\end{array}
%		\,\right)
%\end{equation*}
%\begin{equation*}
%R_{2}(\psi)
%\; := \;
%	\left(\,
%		\begin{array}{ccc}
%			{\color{white}-}\cos\psi & {\color{white}-}0 & {\color{white}-}\sin\psi \\
%			{\color{white}-}0 & {\color{white}-}1 & {\color{white}-}0 \\
%			-\sin\psi & {\color{white}-}0 & {\color{white}-}\cos\psi \\
%			\end{array}
%		\,\right)
%\end{equation*}
%\begin{equation*}
%R_{3}(\theta)
%\; := \;
%	\left(\,
%		\begin{array}{ccc}
%			{\color{white}-}\cos\theta & -\sin\theta & {\color{white}-}0 \\
%			{\color{white}-}\sin\theta & {\color{white}-}\cos\theta & {\color{white}-}0 \\
%			{\color{white}-}0 & {\color{white}-}0 & {\color{white}-}1 \\
%			\end{array}
%		\,\right)
%\end{equation*}
%
%          %%%%% ~~~~~~~~~~~~~~~~~~~~ %%%%%
%
%\vskip 0.5cm
%\noindent
%\textbf{The generators \,$J_{n} \in \C^{3 \times 3}$\, of the Euler matrices}
%\begin{equation*}
%R_{n}(\theta)
%\; = \;
%	\exp\!\left(\;\sqrt{-1}\cdot\theta \overset{{\color{white}1}}{\cdot} J_{n}\,\right)
%\; = \;
%	\exp\!\left(\;\i\cdot\theta \overset{{\color{white}1}}{\cdot} J_{n}\,\right)
%\end{equation*}
%Alternatively, note:
%\begin{equation*}
%\i \cdot J_{1}
%\;\; = \;\;
%	\left.\dfrac{\d}{\d\,\phi}\right\vert_{\phi = 0} R_{1}(\phi)
%\;\; = \;
%	\left.\left(\,
%		\begin{array}{ccc}
%			{\color{white}-}1 & {\color{white}-}0 & {\color{white}-}0 \\
%			{\color{white}-}0 & {\color{white}-}\sin\phi & {\color{white}-}\cos\phi \\
%			{\color{white}-}0 & {\color{black}-}\cos\phi & {\color{white}-}\sin\phi \\
%			\end{array}
%		\,\right)\right\vert_{\phi = 0}
%\;\; = \;
%	\left(\,
%		\begin{array}{ccc}
%			{\color{white}-}0 & {\color{white}-}0 & {\color{white}-}0 \\
%			{\color{white}-}0 & {\color{white}-}0 & {\color{white}-}1 \\
%			{\color{white}-}0 & {\color{black}-}1 & {\color{white}-}0 \\
%			\end{array}
%		\,\right)
%\end{equation*}
%Multiplying both sides by \,$-\,\i = -\,\sqrt{-1}$\, gives:
%\begin{equation*}
%J_{1}
%\;\; = \;
%	\left(\!\!
%		\begin{array}{ccc}
%			{\color{white}-}0 & {\color{white}-}0 & {\color{white}-}0 \\
%			{\color{white}-}0 & {\color{white}-}0 & {\color{black}-}\i \\
%			{\color{white}-}0 & {\color{white}-}\i & {\color{white}-}0 \\
%			\end{array}
%		\,\right)
%\end{equation*}
%Similarly,
%\begin{equation*}
%\i \cdot J_{2}
%\;\; = \;\;
%	\left.\dfrac{\d}{\d\,\psi}\right\vert_{\psi = 0} R_{2}(\psi)
%\;\; = \;
%	\left.\left(\!\!
%		\begin{array}{ccc}
%			{\color{white}-}\sin\psi & {\color{white}-}0 & {\color{black}-}\cos\psi \\
%			{\color{white}-}0 & {\color{white}-}0 & {\color{white}-}0 \\
%			{\color{white}-}\cos\psi & {\color{white}-}0 & {\color{white}-}\sin\psi \\
%			\end{array}
%		\,\right)\right\vert_{\psi = 0}
%\;\; = \;
%	\left(\!\!
%		\begin{array}{ccc}
%			{\color{white}-}0 & {\color{white}-}0 & {\color{black}-}1 \\
%			{\color{white}-}0 & {\color{white}-}0 & {\color{white}-}0 \\
%			{\color{white}-}1 & {\color{white}-}0 & {\color{white}-}0 \\
%			\end{array}
%		\,\right)
%\end{equation*}
%Multiplying both sides by \,$-\,\i = -\,\sqrt{-1}$\, gives:
%\begin{equation*}
%J_{2}
%\;\; = \;
%	\left(\!\!
%		\begin{array}{ccc}
%			{\color{white}-}0 & {\color{white}-}0 & {\color{white}-}\i \\
%			{\color{white}-}0 & {\color{white}-}0 & {\color{white}-}0 \\
%			{\color{black}-}\i & {\color{white}-}0 & {\color{white}-}0 \\
%			\end{array}
%		\,\right)
%\end{equation*}
%Lastly,
%\begin{equation*}
%\i \cdot J_{3}
%\;\; = \;\;
%	\left.\dfrac{\d}{\d\,\theta}\right\vert_{\theta = 0} R_{3}(\theta)
%\;\; = \;
%	\left.\left(\!
%		\begin{array}{ccc}
%			{\color{white}-}\sin\theta & {\color{white}-}\cos\theta & {\color{white}-}0 \\
%			{\color{black}-}\cos\theta & {\color{white}-}\sin\theta & {\color{white}-}0 \\
%			{\color{white}-}0 & {\color{white}-}0 & {\color{white}-}0 \\
%			\end{array}
%		\,\right)\right\vert_{\psi = 0}
%\;\; = \;
%	\left(
%		\begin{array}{ccc}
%			{\color{white}-}0 & {\color{white}-}1 & {\color{white}-}0 \\
%			{\color{black}-}1 & {\color{white}-}0 & {\color{white}-}0 \\
%			{\color{white}-}0 & {\color{white}-}0 & {\color{white}-}0 \\
%			\end{array}
%		\,\right)
%\end{equation*}
%Multiplying both sides by \,$-\,\i = -\,\sqrt{-1}$\, gives:
%\begin{equation*}
%J_{3}
%\;\; = \;
%	\left(\!
%		\begin{array}{ccc}
%			{\color{white}-}0 & {\color{black}-}\i & {\color{white}-}0 \\
%			{\color{white}-}\i & {\color{white}-}0 & {\color{white}-}0 \\
%			{\color{white}-}0 & {\color{white}-}0 & {\color{white}-}0 \\
%			\end{array}
%		\,\right)
%\end{equation*}
%
%          %%%%% ~~~~~~~~~~~~~~~~~~~~ %%%%%
%
%\subsection{Properties of the generators \,$J_{1}, J_{2}, J_{3} \,\in\, \mathfrak{so}(3) \otimes_{\Re} \C$}
%
%          %%%%% ~~~~~~~~~~~~~~~~~~~~ %%%%%
%
%\begin{proposition}
%{\color{white}.}\vskip -0.5cm{\color{white}.}
%\begin{enumerate}
%\item
%	\textbf{Commutation relations:}\;\;
%	\begin{equation*}
%	\left[\,J_{k}\,\overset{{\color{white}1}}{,}\,J_{l}\,\right]
%	\;\; = \;\;
%		\sqrt{-1}\;\overset{3}{\underset{m=1}{\sum}}\,\varepsilon_{klm}\cdot J_{m}\,,
%	\quad
%	\textnormal{for each \,$k, l \in \{\,1,2,3\,\}$}\,,
%	\end{equation*}
%	where \,$\varepsilon_{klm}$\, is the fully anti-symmetric tensor.
%\item
%	\textbf{Raising and lowering operators:}\;\;
%	Define \,$J_{+}\,,\, J_{-} \in \mathfrak{so}(3) \otimes_{\Re} \C \subset \mathcal{U}\!\left(\mathfrak{so}(3) \overset{{\color{white}.}}{\otimes_{\Re}} \C\right)$\, as follows:
%	\begin{equation*}
%	J_{\pm} \;\; := \;\; J_{1} \, \pm \sqrt{-1}\,J_{2}.
%	\end{equation*}
%	Then, the following equalities (of elements of $\mathcal{U}\!\left(\mathfrak{so}(3) \overset{{\color{white}.}}{\otimes_{\Re}} \C\right)$) hold:
%	\begin{enumerate}
%	\item
%		$\left[\,J_{3}\,,\,J_{+}\,\right] \;=\; J_{+}$\,,
%		\quad
%		$\left[\,J_{3}\,,\,J_{-}\,\right] \;=\; -\,J_{-}$\,,
%		\quad
%		$\left[\,J_{+}\,,\,J_{-}\,\right] \;=\; 2\,J_{3}$
%	\item
%		$J^{2}$
%		\; $=$ \; $(J_{3})^{2} \,-\, J_{3} \,+\, J_{+}J_{-}$
%		\; $=$ \; $(J_{3})^{2} \,+\, J_{3} \,+\, J_{-}J_{+}$
%	\item
%		$(J_{\pm})^{\dagger} \; = \; J_{\mp}$
%	\end{enumerate}
%\item
%	Suppose
%	\,$\rho : \mathfrak{so}(3) \otimes_{\Re} \C \longrightarrow \mathfrak{gl}(V)$\,
%	is an irreducible finite-dimensional complex representation, and
%	\,$v \in V \backslash\{0\}$\, is an eigenvector of \,$\rho(J_{3})$\,
%	corresponding to the eigenvalue \,$\lambda \in \Re$;\, thus, \,$\rho(J_{3})(v) \,=\, \lambda\,v$.
%	Then, we have:
%	\begin{equation*}
%	\rho(J_{3})\!\left(\,\rho(J_{+})(\overset{{\color{white}-}}{v})\,\right) \, = \; (\lambda+1)\cdot\rho(J_{+})(v)\,
%	\quad\textnormal{and}\quad\;
%	\rho(J_{3})\!\left(\,\rho(J_{-})(\overset{{\color{white}-}}{v})\,\right) \, = \; (\lambda-1)\cdot\rho(J_{-})(v)
%	\end{equation*}
%\item
%	\textbf{Casimir operator:}\;\;
%	Define
%	\,$J^{2}$
%	\,$:=$\,
%	$(J_{1})^{2} + (J_{2})^{2} + (J_{3})^{2}$
%	\,$\in$\
%	 $\mathcal{U}\!\left(\mathfrak{so}(3) \overset{{\color{white}.}}{\otimes_{\Re}} \C\right)$.
%	Then,
%	\begin{equation*}
%	\left[\,J^{2}\,\overset{{\color{white}1}}{,}\,J_{k}\,\right]
%	\;\; = \;\;
%		0\,,
%	\quad
%	\textnormal{for each \,$k \in \{\,1,2,3\,\}$}\,.
%	\end{equation*}
%	Consequently (by Schur's Lemma, Corollary 4.30, \cite{Hall2015}), 
%	\,$J^{2} \in \mathcal{U}\!\left(\mathfrak{so}(3) \overset{{\color{white}.}}{\otimes_{\Re}} \C\right)$\,
%	acts as a scalar multiple of the identity in every irreducible
%	representation\footnote{Furthermore, this scalar $\lambda \in \C$ uniquely determines
%	the irreducible representation.
%	Look up the classification theory of irreducible finite-dimensional complex representations
%	of complex semisimple Lie algebras.
%	Key words: Casimir operator, universal enveloping algebra. See Chapters 9 and 10, \cite{Hall2015}.}
%	of \,$\mathcal{U}\!\left(\mathfrak{so}(3) \overset{{\color{white}.}}{\otimes_{\Re}} \C\right)$;\,
%	more precisely, for each irreducible finite-dimensional complex representation
%	\,$\rho : \mathcal{U}\!\left(\mathfrak{so}(3) \overset{{\color{white}.}}{\otimes_{\Re}} \C\right) \longrightarrow \mathfrak{gl}(V)$,\,
%	we have \,$\rho(J^{2}) = \lambda \cdot \textnormal{\textbf{1}}_{V}$,\,
%	for some \,$\lambda \in \C$.
%\end{enumerate}
%\end{proposition}
%
%          %%%%% ~~~~~~~~~~~~~~~~~~~~ %%%%%
%
%\begin{theorem}
%{\color{white}.}\vskip -0.1cm
%\noindent
%\begin{enumerate}
%\item
%	The finite-dimensional irreducible representations of $\mathfrak{so}(3) \otimes_{\Re} \C$ is parametrized by the set
%	\begin{equation*}
%	\dfrac{1}{2} \cdot \Z
%	\;\; := \;\;
%		\left\{\;0 \,,\, \dfrac{1}{2} \,,\, 1 \,,\, \frac{3}{2} \,,\, 2 \,,\, \frac{5}{2} \,,\, \ldots \;\right\},
%	\end{equation*}
%	of non-negative integer multiples of \,$\dfrac{1}{2}$, in that, for each
%	$s \in \dfrac{1}{2} \cdot \Z = \left\{\; 0 \,,\, \frac{1}{2}\,,\, 1\,,\, \frac{3}{2}\,,\, 2\,,\, \frac{5}{2}\,,\, \ldots \;\right\}$,
%	there exists a unique (up to equivalence) complex representation
%	$\rho_{s} : \mathcal{U}(\mathfrak{so}(3)\otimes_{\Re}\C) \longrightarrow \textnormal{End}(V_{s})$
%	such that
%	\begin{equation*}
%	\rho_{s}(J^{2}) \; = \; s(s+1)\cdot\textnormal{\textbf{1}}_{V_{s}}.
%	\end{equation*}
%\item
%	$\dim_{\C}(V_{s}) \, = \, 2s + 1$,\, for each
%	\,$s \in \dfrac{1}{2} \cdot \Z = \left\{\; 0 \,,\, \frac{1}{2}\,,\, 1\,,\, \frac{3}{2}\,,\, 2\,,\, \frac{5}{2}\,,\, \ldots \;\right\}$.
%\item
%	For each
%	\,$s \in \dfrac{1}{2} \cdot \Z = \left\{\; 0 \,,\, \frac{1}{2}\,,\, 1\,,\, \frac{3}{2}\,,\, 2\,,\, \frac{5}{2}\,,\, \ldots \;\right\}$,\,
%	the spectrum
%	$\sigma\!\left(\,\overset{{\color{white}-}}{\rho}_{s}(J_{3})\,\right)$
%	of the operator $\rho_{s}(J_{3}) \in \textnormal{End}(V_{s})$
%	consists of only eigenvalues and is given by:
%	\begin{equation*}
%	\sigma\!\left(\,\overset{{\color{white}-}}{\rho}_{s}(J_{3})\,\right)
%	\;\; = \;\;
%		\left\{\;
%			-\overset{{\color{white}-}}{s} \,,\, -(s-1), -(s-2)
%			\,,\;\, \ldots \,\;,\,
%			(s-2) \,,\, (s-1) \,,\, s
%			\;\right\},
%	\end{equation*}
%	and each eigenvalue in 
%	$\sigma\!\left(\,\overset{{\color{white}-}}{\rho}_{s}(J_{3})\,\right)$
%	has multiplicity one.
%\item
%	For each
%	\,$s \in \dfrac{1}{2} \cdot \Z = \left\{\; 0 \,,\, \frac{1}{2}\,,\, 1\,,\, \frac{3}{2}\,,\, 2\,,\, \frac{5}{2}\,,\, \ldots \;\right\}$,\,
%	let \,$v^{(s)}_{k} \in V_{s}\backslash\{0\}$\, be any normalized eigenvector
%	of $\rho_{s}(J_{3})$ corresponding to the eigenvalue
%	\,$k$ $\in$ $\sigma\!\left(\,\overset{{\color{white}-}}{\rho}_{s}(J_{3})\,\right)$
%	$=$ $\left\{\;-\overset{{\color{white}-}}{s} \,,\, -(s-1) \,,\, \;\ldots\;,\, (s-1) \,,\, s\;\right\}$.\,
%	Then, 
%	\begin{enumerate}
%	\item
%		the eigenvectors
%		\,$v^{(s)}_{-s} \,,\, v^{(s)}_{-(s-1)} \,,\; \ldots \;,\, v^{(s)}_{s-1} \,,\, v^{(s)}_{s}$\,
%		form an orthonormal basis for $V_{s}$, and
%	\item
%		for each \,$k$ $\in$ $\sigma\!\left(\,\overset{{\color{white}-}}{\rho}_{s}(J_{3})\,\right)$
%		$=$ $\left\{\;-\overset{{\color{white}-}}{s} \,,\, -(s-1) \,,\, \;\ldots\;,\, (s-1) \,,\, s\;\right\}$,\,
%		we have:
%		\begin{equation*}
%		J_{\pm}\!\left(\,v^{(s)}_{k}\,\right)
%		\; = \;
%			\sqrt{{\color{white}.}
%			s(s+1) - k(k \pm 1)
%			{\color{white}.}}
%			\,\cdot\,
%			v^{(s)}_{k \pm 1}
%		\end{equation*}
%		In particular, \,$J_{\pm}\!\left(\,v^{(s)}_{\pm s}\,\right) \; = \; 0$.
%	\end{enumerate}
%\end{enumerate}
%\end{theorem}
%
%          %%%%% ~~~~~~~~~~~~~~~~~~~~ %%%%%
%

\vskip 1.0cm

          %%%%% ~~~~~~~~~~~~~~~~~~~~ %%%%%

\chapter{Irreducible representations of semisimple complex Lie algebras via Casimir operators in universal enveloping algebras}
\setcounter{theorem}{0}
\setcounter{equation}{0}

%\cite{vanDerVaart1996}
%\cite{Kosorok2008}

%\renewcommand{\theenumi}{\alph{enumi}}
%\renewcommand{\labelenumi}{\textnormal{(\theenumi)}$\;\;$}
\renewcommand{\theenumi}{\roman{enumi}}
\renewcommand{\labelenumi}{\textnormal{(\theenumi)}$\;\;$}

          %%%%% ~~~~~~~~~~~~~~~~~~~~ %%%%%

\begin{enumerate}
\item
	An \textit{algebra} $\mathcal{A}$ over a field $\F$ is a vector space over $\F$
	equipped with a bilinear map $* : \mathcal{A} \times \mathcal{A} \longrightarrow \mathcal{A}$, i.e.,
	\begin{equation*}
	\begin{array}{c}
		(a x + y) * z = a (x * z) + y * z\,,
		\\
		\underset{{\color{white}-}}{\overset{{\color{white}-}}{\textnormal{and}}}
		\\
		x * (a y + z) = a(x * y) + x * z\,,
	\end{array}
	\quad
	\textnormal{for each \,$a, b \in \F$,\, and \,$x, y, z \in \mathcal{A}$}.
	\end{equation*}
	The algebra $\mathcal{A}$ is said to be \textit{associative} if
	\begin{equation*}
	x * (y * z) \; = \; (x * y) * z\,,
	\quad
	\textnormal{for each \,$x, y, z \in \mathcal{A}$}
	\end{equation*}
	The algebra $\mathcal{A}$ is said to be \textit{unital} if
	there exists a multiplicative unit $1 \in \mathcal{A}$, i.e.,
	\begin{equation*}
	1 * x \; = \; x \; = \; x * 1\,,
	\quad
	\textnormal{for each \,$x \in \mathcal{A}$}
	\end{equation*}
\item
	Every associative algebra $\mathcal{A} \cong (\,V,*\,)$
	canonically induces a Lie algebra structure on its underlying vector space $V$ via:
	\begin{equation*}
	\left[\,x\,,\,y\,\right] \; := \; x * y - y * x\,,
	\quad
	\textnormal{for \,$x, y \in V$}
	\end{equation*}
	We denote this canonically induced Lie algebra as $\mathcal{A}^{[,]}$.
	\vskip 0.2cm
	This fact begs the question:
	Does the reverse assoication exist? More precisely, given a Lie algebra $\mathfrak{g}$,
	does there exist an associative algebra $\mathcal{A}$ such that $\mathcal{A}^{[,]} \cong \mathfrak{g}$?
	And, if so, is such an $\mathcal{A}$ unique?
	\vskip 0.2cm
	The answers are Yes and Yes.
	This uniquely determined associative algebra is called the
	\textbf{universal enveloping algebra} of $\mathfrak{g}$,
	and is denoted by $\mathcal{U}(\mathfrak{g})$.
	\vskip 0.1cm
	The universal enveloping algebra can be characterized by the following universal property:
	\begin{center}
	\begin{minipage}{5.25in}
	A \textit{universal enveloping algebra}
	of a Lie algebra $\mathfrak{g}$ over a field $\F$
	is a unital associative algebra $\mathcal{U}$ over $\F$ together with a Lie algebra homomorphism
	$\iota : \mathfrak{g} \longrightarrow \mathcal{U}^{[,]}$ such that,
	for each Lie algebra homomorphism $\phi : \mathfrak{g} \longrightarrow \mathcal{A}^{[,]}$
	(where $\mathcal{A}$ is a unital associative algebra),
	there exists a unique associative algebra homomorphism
	$\phi^{\sharp} : \mathcal{U} \longrightarrow \mathcal{A}$
	whose induced Lie algebra homomorphism
	$(\phi^{\sharp})^{\flat} : \mathcal{U}^{[,]} \longrightarrow \mathcal{A}^{[,]}$
	satisfies:
	$\phi \,=\, (\phi^{\sharp})^{\flat} \,\circ\, \iota$.
	\end{minipage}
	\end{center}
	The universal enveloping algebra $\mathcal{U}(\mathfrak{g})$ can be explicitly constructed as follows:
	\begin{equation*}
	\mathcal{U}(\mathfrak{g}) \;\; \cong \; \left. \overset{{\color{white}.}}{\otimes(\mathfrak{g})} \right\slash \mathcal{I}\,,
	\end{equation*}
	where $\otimes(\mathfrak{g})$ is the tensor algebra of $\mathfrak{g}$, and
	$\mathcal{I} \subset \otimes(\mathfrak{g})$ is the two-sided ideal of $\otimes(\mathfrak{g})$
	generated by elements of the form:
	\begin{equation*}
	x \otimes y \,-\, y \otimes x \,-\, \left[\,x,y\,\right]\,,
	\quad
	\textnormal{for \,$x, y \in \mathfrak{g}$}.
	\end{equation*}
\item
	The representations of a Lie algebra $\mathfrak{g}$ over $\F$ and
	those of its universal enveloping algebra $\mathcal{U}(\mathfrak{g})$
	are in one-to-one correspondence.
	\vskip 0.1cm
	\proof Let $\rho : \mathfrak{g} \longrightarrow \mathfrak{gl}(V)$ be a representation,
	i.e., $\rho$ is a Lie algebra homomorphism from $\mathfrak{g}$ into $\mathfrak{gl}(V) := \textnormal{End}(V)^{[,]}$,
	for some vector space $V$ over $\F$.
	By the universal property of
	$\iota : \mathfrak{g} \longrightarrow \mathcal{U}(\mathfrak{g})^{[,]}$,
	there exists a unique associative algebra homomorphism
	$\rho^{\sharp} : \mathcal{U}(\mathfrak{g}) \longrightarrow \textnormal{End}(V)$
	such that its induced Lie algebra homomorphism
	$(\rho^{\sharp})^{\flat} : \mathcal{U}(\mathfrak{g})^{[,]} \longrightarrow \textnormal{End}(V)^{[,]} =: \mathfrak{gl}(V)$
	satisfies:
	$\rho \,=\, (\rho^{\sharp})^{\flat} \,\circ\, \iota$.
	The association $\rho \longmapsto \rho^{\sharp}$ defines a map
	$\Theta$ from the collection of the representations of $\mathfrak{g}$
	to that of the representations of $\mathcal{U}(\mathfrak{g})$.
	Injectivity of $\Theta$ follows from:
	$\Theta(\rho_{1}) = \Theta(\rho_{2})$
	$\Longleftrightarrow$
	$\rho_{1}^{\sharp} = \rho_{2}^{\sharp}$
	$\Longrightarrow$
	$\rho_{1} \,=\, (\rho_{1}^{\sharp})^{\flat} \circ \iota \,=\, (\rho_{2}^{\sharp})^{\flat} \circ \iota \,=\, \rho_{2}$.
	It remains to establish the surjectivity of $\Theta$.
	To this end, let $\Psi : \mathcal{U}(\mathfrak{g}) \longrightarrow \textnormal{End}(V)$ be an arbitrary representation.
	Define $\psi \, := \, \Psi^{\flat} \,\circ\, \iota : \mathfrak{g} \longrightarrow \textnormal{End}(V)^{[,]} =: \mathfrak{gl}(V)$.
	Then, $\psi$ is a representation of $\mathfrak{g}$.
	By the universal property of
	$\iota : \mathfrak{g} \longrightarrow \mathcal{U}(\mathfrak{g})^{[,]}$,
	there exists a unique associative algebra homomorphism
	$\psi^{\sharp} : \mathcal{U}(\mathfrak{g}) \longrightarrow \textnormal{End}(V)$
	such that its induced Lie algebra homomorphism
	$(\psi^{\sharp})^{\flat} : \mathcal{U}(\mathfrak{g})^{[,]} \longrightarrow \textnormal{End}(V)^{[,]} =: \mathfrak{gl}(V)$
	satisfies:
	$\psi \,=\, (\psi^{\sharp})^{\flat} \,\circ\, \iota$.
	The uniqueness of the associative algebra homomorphism $\psi^{\sharp}$ therefore implies that
	$\Psi = \psi^{\sharp} =: \Theta(\psi)$.
	This proves the surjectivity of $\Theta$.
	\qed
\end{enumerate}

          %%%%% ~~~~~~~~~~~~~~~~~~~~ %%%%%


%\vskip 0.5cm
%
          %%%%% ~~~~~~~~~~~~~~~~~~~~ %%%%%

\section{Induced representations of semidirect products \,$G \ltimes\! H$\, with \,$H$\, abelian}
\setcounter{theorem}{0}
\setcounter{equation}{0}

%\cite{vanDerVaart1996}
%\cite{Kosorok2008}

%\renewcommand{\theenumi}{\alph{enumi}}
%\renewcommand{\labelenumi}{\textnormal{(\theenumi)}$\;\;$}
\renewcommand{\theenumi}{\roman{enumi}}
\renewcommand{\labelenumi}{\textnormal{(\theenumi)}$\;\;$}

          %%%%% ~~~~~~~~~~~~~~~~~~~~ %%%%%

\subsection{Semidirect products}

\begin{definition}
\mbox{}
\vskip 0.1cm
\noindent
Suppose:
\begin{itemize}
\item
	$G$\, and \,$H$\, are two groups, and
\item
	$\rho : G \longrightarrow \textnormal{Aut}(H)$\,
	is a left action of \,$G$\, on \,$H$.\,
\end{itemize}
Then, the \textbf{semidirect product} \,$G \ltimes_{\rho}\! H$\,
of \,$G$\, and \,$H$\, \textbf{\color{red}with respect to \,$\rho$}\,
is defined to be the group
obtained by defining on the Cartesian product
\,$G \times H$\,
the multiplication law:
\begin{equation*}
(\,g_{1},h_{1}\,) \cdot (\,g_{2},h_{2}\,)
\;\; = \;\;
	\left(\,
		\overset{{\color{white}1}}{g_{1}} \cdot g_{2}
		\,\overset{{\color{white}1}}{,}\,
		h_{1} \cdot \rho(g_{1}) \cdot h_{2}
		\,\right)
\end{equation*}
The identity element of \,$G \ltimes_{\rho}\! H$\, is then
\begin{equation*}
1_{G \ltimes H}
\;\; = \;\;
	\left(\,
		\overset{{\color{white}1}}{1_{G}}
		\,\overset{{\color{white}1}}{,}\,
		1_{H}
		\,\right)
\end{equation*}
and the inverse of \,$(\,g,h\,) \in G \ltimes_{\rho}\! H$\, is given by:
\begin{equation*}
(\, g \,,\, h \,)^{-1}
\;\; = \;\;
	\left(\;
		g^{-1}
		\;\overset{{\color{white}1}}{,}\;
		\overset{{\color{white}1}}{\rho}(g^{-1}) \cdot h^{-1}
		\;\right)
\end{equation*}
\end{definition}

          %%%%% ~~~~~~~~~~~~~~~~~~~~ %%%%%

\vskip 0.5cm
\subsection{A representation of a semidirect product is determined by the restrictions to its factors}

\begin{proposition}[Proposition 7.3, p.150, \cite{Berndt2007}]
\mbox{}
\vskip 0.1cm
\noindent
Suppose:
\begin{itemize}
\item
	$G$\, is a group, \,$H$\, an {\color{red}abelian} group,
\item
	$\rho : G \longrightarrow \textnormal{Aut}(H)$\,
	is a left action of \,$G$\, on \,$H$,\,
\item
	$\pi : G \ltimes_{\rho}\! H \longrightarrow \GL(V)$\,
	is an arbitrary representation of the semidirect product
	\,$G \ltimes_{\rho}\! H$\, with respect to \,$\rho$,\, and
\item
	$\pi_{G} : G \longrightarrow \GL(V)$\, and \,$\pi_{H} : G \longrightarrow \GL(V)$\,
	are the restrictions of \,$\pi$\, to \,$G$\, and \,$H$,\, respectively; i.e.,
	\,$\pi_{G}$\, and \,$\pi_{H}$\,
	are given by:
	\begin{equation*}
	\pi_{G}(\,g\,) \; := \; \pi\!\left(\,(\,\overset{{\color{white}1}}{g}\,,0\,)\,\right),
	\quad\quad
	\textnormal{and}
	\quad\quad
	\pi_{H}(\,h\,) \; := \; \pi\!\left(\,(\,1_{G}\,,\overset{{\color{white}1}}{h}\,)\,\right).
	\end{equation*}
\end{itemize}
Then, the following statements are true:
\begin{enumerate}
\item
	$\pi$\, is completely determined by its restrictions \,$\pi_{G}$\, and \,$\pi_{H}$,\, and
\item
	\,$\pi_{G}$\, and \,$\pi_{H}$\, satisfy the following equality:
	\begin{equation*}
	\pi_{H}\!\left(\,\rho(g) \cdot \overset{{\color{white}.}}{h}\,\right)
	\;\; = \;\;
		\pi_{G}\!\left(\; \overset{{\color{white}.}}{g} \,\right)
		\cdot
		\pi_{H}\!\left(\overset{{\color{white}.}}{{\color{white}g}}\!\! h \,\right)
		\cdot
		\pi_{G}\!\left(\; g^{-1} \,\right)
	\end{equation*}
\end{enumerate}
\end{proposition}
\proof
\begin{enumerate}
\item
	Note that
	\,$(\,g\,,h\,) \;=\; \left(\;1_{G} \cdot g\,,\, h + \rho(g)\cdot \overset{{\color{white}.}}{0}\,\right) \;=\; (\,1_{G}\,,h\,)\cdot (\,g\,,0\,)$.\,
	Hence,
	\begin{equation*}
	\pi\!\left(\;(\,\overset{{\color{white}1}}{g}\,,h\,)\;\right)
	\;\; = \;\;
		\pi\!\left(\;
			(\,1_{G}\,,h\,)
			\cdot
			(\,\overset{{\color{white}1}}{g}\,,0\,)
			\;\right)
	\;\; = \;\;
		\pi\!\left(\;
			(\,1_{G}\,,\overset{{\color{white}.}}{h}\,)
			\;\right)
		\cdot
		\pi\!\left(\;
			(\,\overset{{\color{white}.}}{g}\,,\overset{{\color{white}.}}{0}\,)
			\;\right)
	\;\; = \;\;
		\pi_{H}(\,h\,) \cdot \pi_{G}(\,g\,),
	\end{equation*}
	which proves that \,$\pi$\, is indeed completely determined by
	\,$\pi_{G}$\, and \,$\pi_{H}$.\,
\item
	Recall the multiplication law of \,$G \ltimes_{\rho}\! H$:\,
	\begin{equation*}
	(\,g_{1},h_{1}\,) \cdot (\,g_{2},h_{2}\,)
	\;\; = \;\;
		\left(\;
			\overset{{\color{white}1}}{g_{1}} \cdot g_{2}
			\;\overset{{\color{white}1}}{,}\;
			h_{1} + \rho(g_{1}) \cdot h_{2}
			\;\right)
	\end{equation*}
	Hence,
	\begin{eqnarray*}
	\pi_{H}(\,h_{1}\,) \cdot \pi_{G}(\,g_{1}\,)
	\cdot
	\pi_{H}(\,h_{2}\,) \cdot \pi_{G}(\,g_{2}\,)
	& = &
		\pi\!\left(\;
			(\,\overset{{\color{white}1}}{g_{1}},h_{1}\,)
			\;\right)
		\cdot
		\pi\!\left(\;
			(\,\overset{{\color{white}1}}{g_{2}},h_{2}\,)
			\;\right)
	\\
	& = &
		\pi\!\left(\;
			(\,g_{1},h_{1}\,) \overset{{\color{white}1}}{\cdot} (\,g_{2},h_{2}\,)
			\;\right)
	\\
	& = &
		\pi\!\left(\;
			\left(\;
				\overset{{\color{white}1}}{g_{1}} \cdot g_{2}
				\;\overset{{\color{white}1}}{,}\;
				h_{1} + \rho(g_{1}) \cdot h_{2}
				\;\right)
			\;\right)
	\\
	& = &
		\pi_{H}\!\left(\;h_{1} + \rho(g_{1}) \overset{{\color{white}1}}{\cdot} h_{2}\;\right)
		\cdot
		\pi_{G}(\;g_{1}\cdot g_{2}\,)
	\\
	& = &
		\pi_{H}\!\left(\;\overset{{\color{white}.}}{h_{1}}\;\right)
		\cdot
		\pi_{H}\!\left(\;\rho(g_{1}) \overset{{\color{white}1}}{\cdot} h_{2}\;\right)
		\cdot
		\pi_{G}\!\left(\;\overset{{\color{white}.}}{g_{1}}\;\right)
		\cdot
		\pi_{G}\!\left(\;\overset{{\color{white}.}}{g_{2}}\;\right)
	\end{eqnarray*}
	Hence,
	\begin{equation*}
		\pi_{H}\!\left(\;\overset{{\color{white}.}}{h_{1}}\;\right)
		\cdot
		\pi_{H}\!\left(\;\rho(g_{1}) \overset{{\color{white}1}}{\cdot} h_{2}\;\right)
		\cdot
		\pi_{G}\!\left(\;\overset{{\color{white}.}}{g_{1}}\;\right)
		\cdot
		\pi_{G}\!\left(\;\overset{{\color{white}.}}{g_{2}}\;\right)
	\;\; = \;\;
		\pi_{H}(\,h_{1}\,) \cdot \pi_{G}(\,g_{1}\,)
		\cdot
		\pi_{H}(\,h_{2}\,) \cdot \pi_{G}(\,g_{2}\,)
	\end{equation*}
	Setting, in the above equality,
	\,$g_{1} = g$,\,  $h_{1} = 0$,\, $g_{2} = 1_{G}$\, and \,$h_{2} = h$\,
	yields
	\begin{equation*}
		\pi_{H}\!\left(\;\overset{{\color{white}.}}{0}\;\right)
		\cdot
		\pi_{H}\!\left(\;\rho(g) \overset{{\color{white}1}}{\cdot} h\;\right)
		\cdot
		\pi_{G}\!\left(\;\overset{{\color{white}.}}{g}\;\right)
		\cdot
		\pi_{G}\!\left(\;\overset{{\color{white}.}}{1_{G}}\;\right)
	\;\; = \;\;
		\pi_{H}(\,0\,) \cdot \pi_{G}(\,g\,)
		\cdot
		\pi_{H}(\,h\,) \cdot \pi_{G}(\,1_{G}\,)\,,
	\end{equation*}
	which simplifies to
	\begin{equation*}
		\pi_{H}\!\left(\;\rho(g) \overset{{\color{white}1}}{\cdot} h\;\right)
		\cdot
		\pi_{G}\!\left(\;\overset{{\color{white}.}}{g}\;\right)
	\;\; = \;\;
		\pi_{G}(\,g\,)
		\cdot
		\pi_{H}(\,h\,)
	\end{equation*}
	which can further be rewritten as
	\begin{equation*}
		\pi_{H}\!\left(\;\rho(g) \overset{{\color{white}-}}{\cdot} h\;\right)
	\;\; = \;\;
		\pi_{G}(\,g\,)
		\cdot
		\pi_{H}(\,h\,)
		\cdot
		\pi_{G}\!\left(\,\overset{{\color{white}.}}{{\color{white}g}}\!\!g^{-1}\;\right),
	\end{equation*}
	as required. \qed
\end{enumerate}

          %%%%% ~~~~~~~~~~~~~~~~~~~~ %%%%%

\vskip 0.5cm
\subsection{Irreducible unitary representations of regular semidirect products}

\vskip 0.5cm
\begin{definition}[Definition 7.1, p.150, \cite{Berndt2007}]
\mbox{}
\vskip 0.1cm
\noindent
A semidirect
\,$G \ltimes_{\rho}\! H$\, of \,$G$\, amd \,$H$\,
is said to be \textbf{regular} if

\end{definition}

\vskip 0.5cm
\begin{theorem}[Theorem 7.7, p.151, \cite{Berndt2007}]
\mbox{}
\vskip 0.1cm
\noindent
Suppose:
\begin{itemize}
\item
	$G$\, and \,$H$\, are separable and locally compact groups, and
	\,$H$\, is furthermore abelian.
\item
	$\rho : G \longrightarrow \textnormal{Aut}(H)$\,
	is a left action of \,$G$\, on \,$H$,\, and
\item
	the semidirect product
	\,$G \ltimes_{\rho}\! H$\, of \,$G$\, amd \,$H$\,
	with respect to \,$\rho$\,
	is a regular.
\end{itemize}
Then, every irreducible unitary representation of
\,$G \ltimes_{\rho}\! H$\,
is unitarily equivalent to an induced representation.
\end{theorem}

          %%%%% ~~~~~~~~~~~~~~~~~~~~ %%%%%


          %%%%% ~~~~~~~~~~~~~~~~~~~~ %%%%%

\vskip 0.5cm
\section{Irreducible unitary projective representations of the Poincaré group \,$\SOup(1,3) \ltimes \Re^{1,3}$}

\vskip 0.3cm
\begin{theorem}[Theorem 14.3, p.61, \cite{vanDenBan2004}; see also \cite{Bargmann1954}]
\mbox{}
\vskip 0.1cm
\noindent
Let \,$\mathcal{H}$\, be a (not necessarily finite-dimensional) complex Hilbert space.
Every projective representation
\,$\pi : \SL(2,\C) \ltimes \Re^{1,3} \longrightarrow \Aut\!\left(\,\mathbb{P}(\mathcal{H})\right)$\,
lifts to a unique unitary (``ordinary'') representation
\,$\widetilde{\pi}$\, of \,$\SL(2,\C) \ltimes \Re^{1,3}$\,
in \,$\mathcal{H}$.\,
Moreover, \,$\pi$\, is irreducible if and only if \,$\widetilde{\pi}$\, is irreducible.
\end{theorem}

          %%%%% ~~~~~~~~~~~~~~~~~~~~ %%%%%

\vskip 1.0cm
\section{Irreducible unitary representations of \,$\SL(2,\C) \ltimes \Re^{1,3}$}

          %%%%% ~~~~~~~~~~~~~~~~~~~~ %%%%%

We follow Chapter 9, p.209, \cite{Talagrand2022}.

\vskip 0.5cm
\noindent
Let \,$m > 0$\, and
%\,$X_{m}^{\uparrow},\, X_{m}^{\downarrow} \,\subset\, \Re^{1,3}$\,
\,$X_{m} \,\subset\, \Re^{1,3}$\,
be the orbit of
\,$p^{*} = (m,0,0,0)^{T} \in \Re^{1,3}$\,
under the action of
\,$\SL(2,\C) \curvearrowright \Re^{1,3}$,\,
i.e.,
\begin{eqnarray*}
X_{m}
& \!\! := \!\! &
	\left\{\;\,
		\left.
		p = (\,p_{0},p_{1},p_{2},p_{3}\,)
		\overset{{\color{white}1}}{\in}
		\Re^{1,3}
		\;\;\;\right\vert
		\begin{array}{c}
			p_{0} \;{\color{red}>}\; 0\,,
			\\
			-m^{2} \, \overset{{\color{white}1}}{=} \, -\,(p_{0})^{2} + (p_{1})^{2} + (p_{2})^{2} + (p_{3})^{2}
			\end{array}
		\right\}
%\\ \\
%X_{m}^{\downarrow}
%& := &
%	\left\{\;\,
%		\left.
%		p = (\,p_{0},p_{1},p_{2},p_{3}\,)
%		\overset{{\color{white}1}}{\in}
%		\Re^{1,3}
%		\;\;\;\right\vert
%		\begin{array}{c}
%			p_{0} \;{\color{red}<}\; 0\,,
%			\\
%			-m^{2} \, \overset{{\color{white}1}}{=} \, -\,(p_{0})^{2} + (p_{1})^{2} + (p_{2})^{2} + (p_{3})^{2}
%			\end{array}
%		\right\}
\end{eqnarray*}
%where
%\,$m \,\in\, \{\,0\,\} \sqcup (\,0,\infty\,) \sqcup \i\,(\,0,\infty\,)$\, (hence, \,$m^{2}$\, varies over all of \,$\Re$).\,

\vskip 0.5cm
\noindent
Then,
\begin{itemize}
\item
	$\textnormal{Stab}_{\SL(2,\C)}\!\left(\;(m,0,0,0)^{T}\,\right) \;\cong\; \SU(2)$
\item
	Here, we follow \S9.4, p.215, \cite{Talagrand2022}.
	Let
	\,$\pi_{s} : \SU(2) \longrightarrow \GL(V_{s})$\,
	be the irreducible finite-dimensional complex representation of \,$\SU(2)$\,
	determined by
	\,$s \,\in\, \{\,0\,\} \,\bigsqcup\, \dfrac{1}{2}\cdot\N$\,
	(i.e., $\dim_{\C}(V_{s}) = 2s + 1$).
	Let
	\begin{equation*}
	\mathcal{H}_{m,s}
	\; := \;
		L^{2}\!\left(\,X_{m},V_{s},\d\lambda_{m}\,\right)
	\end{equation*}
	denote the (complex) Hilbert space of $\lambda_{m}$-square-integrable
	$V_{s}$-valued functions, where
	\,$\lambda_{m}$\, is the (unique up to scaling)
	\,$\SL(2,\C)$-invariant measure on
	\,$X_{m}$.\,
	For each
	\,$(A,b) \in \SL(2,\C) \ltimes \Re^{1,3}$,\,
	define a map
	\,$\Pi_{m,s}\!\left(A,b\right) : \mathcal{H}_{m,s} \longrightarrow \mathcal{H}_{m,s}$\,
	as follows:
	\begin{equation}\label{PiMSUp}
	\Pi_{m,s}\!\left(A,b\right)[\,\varphi\,](p)
	\;\; := \;\;
		\exp\!\left(\,\overset{{\color{white}.}}{\i}\,\langle\,b,p\,\rangle\right)
		\cdot\,
		\pi_{s}(D^{-1}_{Ap}\,A\,D_{A^{-1}\,p})\!\left[\;
			\overset{{\color{white}.}}{\varphi(A^{-1}p)}
			\;\right],
	\end{equation}
	where
	\,$D_{p} \in \SL(2,\C)$\, is such that \,$D_{p}\!\left(\,(m,0,0,0)^{T}\,\right) \,=\, p$.\,
	\vskip 0.1cm
	\noindent
	Remarks:
	\begin{itemize}
	\item
		For each \,$p \in X_{m}$,\, $D_{p} \in \SL(2,\C)$\, is NOT unique,
		but that \,$\Pi_{m,s}$\, is well-defined
		(i.e., the right-hand side of \eqref{PiMSUp} does not depend on the particular choice of \,$D_{p}$).\,
	\item
		$D^{-1}_{Ap}\,A\,D_{A^{-1}\,p} \in \textnormal{Stab}_{\SL(2,\C)}\!\left(\;(m,0,0,0)^{T}\,\right)$
	\end{itemize}
	\vskip 0.1cm
	\textbf{Claim:}\;
	$\Pi_{m,s} : \SL(2,\C) \ltimes \Re^{1,3} \longrightarrow \Aut\!\left(\mathcal{H}_{m,s}\right)$\,
	is a (complex-linear infinite-dimensional) irreducible unitary representation.
	\vskip 0.1cm
	$\Pi_{m,s}$\, is called the \textbf{induced representation of \,$\SL(2,\C) \times \Re^{1,3}$}\,
	by the (irreducible) representation
	\,$\pi_{s} : \SU(2) \longrightarrow \GL(V_{s})$\,
	of
	\,$\SU(2) \cong \textnormal{Stab}_{\SL(2,\C)}\!\left(\;(m,0,0,0)^{T}\,\right)$.\,
\item
	Every irreducible unitary representation of
	\,$\SL(2,\C) \ltimes \Re^{1,3}$\,
	is unitarily equivalent to an induced representation of
	\,$\SL(2,\C) \ltimes \Re^{1,3}$\,
	by an irreducible finite-dimensional (necessarily unitary, by compactness of $\SU(2)$)
	representation of \,$\SU(2)$.\,
	See \cite{Wigner1939}.
\end{itemize}

          %%%%% ~~~~~~~~~~~~~~~~~~~~ %%%%%

\vskip 0.5cm
\section{The Lie algebra \,$\so(1,3) \ltimes \Re^{1,3}$\, of the Lorentz group \,$\nobreak{\SOup(1,3) \ltimes \Re^{1,3}}$}

\noindent
\textbf{Generators induced by spatial rotations:}
\begin{equation*}
J_{1} = M_{23} = \textnormal{\footnotesize$\left[{\color{gray}
		\begin{array}{rrrrr}
		0 & {\color{white}-}0 & {\color{white}-}0 & {\color{white}-}0 & {\color{white}-}0 \\
		0 & 0 & 0 & 0 & 0\\
		0 & 0 & 0 & {\color{red}-1} & 0 \\
		0 & 0 & {\color{red}1} & 0 & 0 \\
		0 & 0 & 0 & 0 & 0 \\
		\end{array}
		}\right]$},
\;\;
J_{2} = M_{31} = \textnormal{\footnotesize$\left[{\color{gray}
	\begin{array}{rrrrr}
	0 & {\color{white}-}0 & {\color{white}-}0 & {\color{white}-}0 & {\color{white}-}0 \\
	0 & 0 & 0 & {\color{red}1} & 0 \\
	0 & 0 & 0 & 0 & 0 \\
	0 & {\color{red}-1} & 0 & 0 & 0 \\
	0 & 0 & 0 & 0 & 0 \\
	\end{array}
	}\right]$},
\;\;
J_{3} = M_{12} = \textnormal{\footnotesize$\left[{\color{gray}
	\begin{array}{rrrrr}
	0 & {\color{white}-}0 & {\color{white}-}0 & {\color{white}-}0 & {\color{white}-}0 \\
	0 & 0 & {\color{red}-1} & 0 & 0 \\
	0 & {\color{red}1} & 0 & 0 & 0 \\
	0 & 0 & 0 & 0 & 0 \\
	0 & 0 & 0 & 0 & 0 \\
	\end{array}
	}\right]$}
\end{equation*}
\textbf{Generators induced by Lorentz boosts:}
\begin{equation*}
K_{1} = M_{01} = \textnormal{\footnotesize$\left[{\color{gray}
	\begin{array}{rrrrr}
	0 & {\color{red}1} & 0 & 0 & 0\\
	{\color{red}1} & 0 & 0 & 0 & 0\\
	0 & 0 & 0 & 0 & 0 \\
	0 & 0 & 0 & 0 & 0 \\
	0 & 0 & 0 & 0 & 0 \\
	\end{array}
	}\right]$},
\quad\;\;
K_{2} = M_{02} = \textnormal{\footnotesize$\left[{\color{gray}
	\begin{array}{rrrrr}
	0 & 0 & {\color{red}1} & 0 & 0\\
	0 & 0 & 0 & 0 & 0\\
	{\color{red}1} & 0 & 0 & 0 & 0 \\
	0 & 0 & 0 & 0 & 0 \\
	0 & 0 & 0 & 0 & 0 \\
	\end{array}
	}\right]$},
\quad\;\;
K_{3} = M_{03} = \textnormal{\footnotesize$\left[{\color{gray}
	\begin{array}{rrrrr}
	0 & 0 & 0 & {\color{red}1} & 0\\
	0 & 0 & 0 & 0 & 0\\
	0 & 0 & 0 & 0 & 0 \\
	{\color{red}1} & 0 & 0 & 0 & 0 \\
	0 & 0 & 0 & 0 & 0 \\
	\end{array}
	}\right]$}
\end{equation*}
\noindent
\textbf{Generators induced by spatial translations:}
\begin{equation*}
P_{0} = \textnormal{\footnotesize$\left[{\color{gray}
	\begin{array}{rrrrr}
    %0 & {\color{white}-}0 & {\color{white}-}0 & {\color{white}-}0 & {\color{white}-}{\color{red}1} \\
	0 & 0 & 0 & 0 & {\color{red}1} \\
	0 & 0 & 0 & 0 & 0 \\
	0 & 0 & 0 & 0 & 0 \\
	0 & 0 & 0 & 0 & 0 \\
	0 & 0 & 0 & 0 & 0 \\
	\end{array}
	}\right]$},
\;\;
P_{1} = \textnormal{\footnotesize$\left[{\color{gray}
	\begin{array}{rrrrr}
	0 & 0 & 0 & 0 & 0 \\
	0 & 0 & 0 & 0 & {\color{red}1} \\
	0 & 0 & 0 & 0 & 0 \\
	0 & 0 & 0 & 0 & 0 \\
	0 & 0 & 0 & 0 & 0 \\
	\end{array}
	}\right]$},
\;\;
P_{2} = \textnormal{\footnotesize$\left[{\color{gray}
	\begin{array}{rrrrr}
	0 & 0 & 0 & 0 & 0 \\
	0 & 0 & 0 & 0 & 0 \\
	0 & 0 & 0 & 0 & {\color{red}1} \\
	0 & 0 & 0 & 0 & 0 \\
	0 & 0 & 0 & 0 & 0 \\
	\end{array}
	}\right]$},
\;\;
P_{3} = \textnormal{\footnotesize$\left[{\color{gray}
	\begin{array}{rrrrr}
	0 & 0 & 0 & 0 & 0 \\
	0 & 0 & 0 & 0 & 0 \\
	0 & 0 & 0 & 0 & 0 \\
	0 & 0 & 0 & 0 & {\color{red}1} \\
	0 & 0 & 0 & 0 & 0 \\
	\end{array}
	}\right]$}
\end{equation*}

\vskip 0.5cm
\begin{proposition}[Commutation relations of the generators of the Poincaré algebra]
\mbox{}
\vskip 0.1cm
\noindent
The generators
\begin{equation*}
J_{1},\; J_{2},\; J_{3},
\quad\;\;
K_{1},\; K_{2},\; K_{3},
\quad\;\;
P_{0},\; P_{1},\; P_{2},\; P_{3}
\end{equation*}
of the Lie algebra
\,$\so(1,3) \ltimes \Re^{1,3}$\,
of the Poincaré group
\,$\SOup(1,3) \ltimes \Re^{1,3}$\,
satisfy the following commutation relations:
\begin{enumerate}
\item
	$\left[\,J_{m}\,,\,J_{n}\,\right] \; = \; \varepsilon_{mnk}\,J_{k}$,\,
	for each \,$m, n = 1,2,3$,
\item
	\vskip -0.1cm
	$\left[\,J_{m}\,,\,K_{n}\,\right] \; = \; \varepsilon_{mnk}\,K_{k}$,\,
	for each \,$m, n = 1,2,3$,
\item
	\vskip -0.1cm
	$\left[\,K_{m}\,,\,K_{n}\,\right] \; = \; -\,\varepsilon_{mnk}\,J_{k}$,\,
	for each \,$m, n = 1,2,3$,
\item
	\vskip 0.2cm
	$\left[\,P_{\mu}\,,\,P_{\nu}\,\right] \; = \; 0$,\,
	for each \,$\mu,\, \nu = 0,1,2,3$,
\item
	\vskip 0.2cm
	$\left[\,J_{m}\,,\,P_{0}\,\right] \; = \; 0$,\,
	for each \,$m = 1,2,3$,
\item
	\vskip -0.1cm
	$\left[\,K_{m}\,,\,P_{0}\,\right] \; = \; -\,P_{m}$,\,
	for each \,$m = 1,2,3$,
\item
	\vskip -0.1cm
	$\left[\,J_{m}\,,\,P_{n}\,\right] \; = \; \varepsilon_{mnk}\,P_{k}$,\,
	for each \,$m, n = 1,2,3$,
\item
	\vskip -0.1cm
	$\left[\,K_{m}\,,\,P_{n}\,\right] \; = \; \delta_{mn}\,P_{0}$,\,
	for each \,$m, n = 1,2,3$.
\end{enumerate}
\end{proposition}
\proof
Direct computations.
\qed

\vskip 0.5cm
\begin{proposition}[A more compact presentation of the Poincaré algebra]
\mbox{}
\vskip 0.1cm
\noindent
Define \,$M_{\mu\nu} \in \C^{5 \times 5}$,\,
for \,$\mu, \nu \in \{\,0,1,2,3\,\}$,\, as follows:
\begin{itemize}
\item
	$M_{0k} \, := \, K_{k}$,\, and \,$M_{0k} \, := \, -\,K_{k}$,\, for \,$k = 1,2,3$,
\item
	$M_{mn} \, := \, \varepsilon_{mnk} \, J_{k}$,\, for \,$m, n \in \{\,1,2,3\,\}$.
\end{itemize}
Then, the following statements are true:
\begin{enumerate}
\item
	$P_{0}, P_{1}, P_{2}, P_{3}$,\;\;
	$M_{01}, M_{02}, M_{03}$,\;\;
	$M_{12}, M_{23}, M_{31}$\;\;
	generate the Poincaré algebra
	\,$\so(1,3) \ltimes \Re^{1,3}$.\,
\item
	For each $\mu, \nu \in \{\,0,1,2,3\,\}$,\,
	the matrix entries of \,$M_{\mu\nu}$\, are given by:
	\begin{equation*}
	\left(M_{\mu\nu}\right)_{ab}
	\;\; = \;\;
		\left\{\begin{array}{cl}
			\delta_{a\mu}\,\eta_{b\nu}
			\,-\,
			\delta_{a\nu}\,\eta_{b\mu}\,,
			& \textnormal{for \,$a, b = 0,1,2,3$}
			\\
			\overset{{\color{white}1}}{0}\,,
			& \textnormal{if \,$a = 4$\, or \,$b = 4$}
			\end{array}\right.,
	\end{equation*}
	where
	\,$\left(\,\eta_{\mu\nu}\,\right) \,=\, \diag(+1,-1,-1,-1)$.\,
\end{enumerate}
\end{proposition}
\proof
\begin{enumerate}
\item
	Obvious, as the ``new'' generators are in fact just the ``old'' onews with new names.
\item
	First, note that:
	\begin{eqnarray*}
	\left(M_{23}\right)_{{\color{red}12}}
	& = &
		\delta_{{\color{red}1}2}\eta_{{\color{red}2}3}
		\,-\,
		\delta_{{\color{red}1}3}\eta_{{\color{red}2}2}
	\;\; = \;\;
		0 \cdot 0 \,-\, 0 \cdot (+1)
	\;\; = \;\;
		0
	\end{eqnarray*}
	More generally, note:
	\begin{eqnarray*}
	\{\,a,b\,\} \neq \{\,\mu,\nu\,\}
	\quad\Longrightarrow\quad
	\left(M_{\mu\nu}\right)_{ab} \;=\; 0
	\end{eqnarray*}
	Next, observe that:
	\begin{eqnarray*}
	\delta_{{\color{red}2}2}\eta_{{\color{red}3}3}
	\,-\,
	\delta_{{\color{red}2}3}\eta_{{\color{red}3}2}
	& = &
		(\,+1\,)(-1) \,-\, 0 \cdot 0
	\;\; = \;\;
		-1
	\;\; = \;\;
		\left(M_{23}\right)_{{\color{red}23}}
	\\
	\delta_{{\color{red}3}2}\eta_{{\color{red}2}3}
	\,-\,
	\delta_{{\color{red}3}3}\eta_{{\color{red}2}2}
	& = &
		0 \cdot 0 \,-\, (\,+1\,)(-1)
	\;\; = \;\;
		+1
	\;\; = \;\;
		\left(M_{23}\right)_{{\color{red}32}}
	\\
	\delta_{{\color{red}1}3}\eta_{{\color{red}3}1}
	\,-\,
	\delta_{{\color{red}1}1}\eta_{{\color{red}3}3}
	& \overset{{\color{white}\textnormal{\Large$1$}}}{=} &
		0 \cdot 0 \,-\, (\,+1\,)(-1)
	\;\; = \;\;
		+1
	\;\; = \;\;
		\left(M_{31}\right)_{{\color{red}13}}
	\\
	\delta_{{\color{red}3}3}\eta_{{\color{red}1}1}
	\,-\,
	\delta_{{\color{red}3}1}\eta_{{\color{red}1}3}
	& = &
		(\,+1\,)(-1) \,-\, 0 \cdot 0
	\;\; = \;\;
		-1
	\;\; = \;\;
		\left(M_{31}\right)_{{\color{red}31}}
	\\
	\delta_{{\color{red}1}1}\eta_{{\color{red}2}2}
	\,-\,
	\delta_{{\color{red}1}2}\eta_{{\color{red}2}1}
	& \overset{{\color{white}\textnormal{\Large$1$}}}{=} &
		(\,+1\,)(-1) \,-\, 0 \cdot 0
	\;\; = \;\;
		-1
	\;\; = \;\;
	\left(M_{12}\right)_{{\color{red}12}}
	\\
	\delta_{{\color{red}2}1}\eta_{{\color{red}1}2}
	\,-\,
	\delta_{{\color{red}2}2}\eta_{{\color{red}1}1}
	& = &
		0 \cdot 0 \,-\, (\,+1\,)(-1)
	\;\; = \;\;
		+1
	\;\; = \;\;
		\left(M_{12}\right)_{{\color{red}21}}
	\end{eqnarray*}
	Furthermore, for \,$k \in \{\,1,2,3\,\}$,\, we have:
	\begin{eqnarray*}
	\delta_{{\color{red}0}0}\eta_{{\color{red}k}k}
	\,-\,
	\delta_{{\color{red}0}k}\eta_{{\color{red}k}0}
	& = &
		(\,+1\,)(+1) \,-\, 0 \cdot 0
	\;\; = \;\;
		+1
	\;\; = \;\;
		\left(M_{0k}\right)_{{\color{red}0k}}
	\\
	\delta_{{\color{red}k}0}\eta_{{\color{red}0}k}
	\,-\,
	\delta_{{\color{red}k}k}\eta_{{\color{red}0}0}
	& = &
		0 \cdot 0 \,-\, (\,-1\,)(+1)
	\;\; = \;\;
		+1
	\;\; = \;\;
		\left(M_{0k}\right)_{{\color{red}k0}}
	\\
	\delta_{{\color{red}0}k}\eta_{{\color{red}k}0}
	\,-\,
	\delta_{{\color{red}0}0}\eta_{{\color{red}k}k}
	& \overset{{\color{white}\textnormal{\Large$1$}}}{=} &
		0 \cdot 0 \,-\, (\,+1\,)(+1)
	\;\; = \;\;
		-1
	\;\; = \;\;
		\left(M_{k0}\right)_{{\color{red}0k}}
	\\
	\delta_{{\color{red}k}k}\eta_{{\color{red}0}0}
	\,-\,
	\delta_{{\color{red}k}0}\eta_{{\color{red}0}k}
	& = &
		(\,-1\,)(+1) \,-\, 0 \cdot 0
	\;\; = \;\;
		-1
	\;\; = \;\;
		\left(M_{k0}\right)_{{\color{red}k0}}
	\end{eqnarray*}
\end{enumerate}
This completes the proof of the Proposition.
\qed

\vskip 0.5cm
\begin{proposition}[Commutation relations of $P_{\mu}$ and $M_{\mu\nu}$]
\mbox{}
\vskip 0.1cm
\noindent
The generators \;\;
$P_{0}, P_{1}, P_{2}, P_{3}$,\;\;
$M_{01}, M_{02}, M_{03}$,\;\;
$M_{12}, M_{23}, M_{31}$\;\;
satisfy the following commutation relations:
\begin{enumerate}
\item
	$\left[\,P_{\mu}\,\overset{{\color{white}1}}{,}\,P_{\nu}\,\right] \,=\, 0$,\,
	for \,$\mu, \nu \in \{\,0,1,2,3\,\}$,
\item
	$
	\left[\,
		M_{\mu\nu}
		\,\overset{{\color{white}1}}{,}\,
		P_{\rho}
		\,\right]
	\;\; = \;\;
		-\,
		\eta_{\rho\mu}\,P_{\nu}
		\,+\,
		\eta_{\rho\nu}\,P_{\mu}
	$\,,\;
	for \,$\mu, \nu, \rho \in \{\,0,1,2,3\,\}$,\, and
\item
	$
	\left[\,
		M_{\mu\nu}
		\,\overset{{\color{white}1}}{,}\,
		M_{\rho\sigma}
		\,\right]
	\; = \,
		% \,-\,
		% \eta_{\mu\rho}M_{\nu\sigma}
		% \,+\,
		% \eta_{\mu\sigma}M_{\nu\rho}
		% \,+\,
		% \eta_{\nu\rho}M_{\mu\sigma}
		% \,-\,
		% \eta_{\nu\sigma}M_{\mu\rho}
		\,-\,
		\eta_{\rho\mu}M_{\nu\sigma}
		\,+\,
		\eta_{\rho\nu}M_{\mu\sigma}
		\,-\,
		\eta_{\sigma\mu}M_{\rho\nu}
		\,+\,
		\eta_{\sigma\nu}M_{\rho\mu}
	$\,,\;
	for \,$\mu, \nu \in \{\,0,1,2,3\,\}$,
	% where
	% \,$\left(\,\eta_{\mu\nu}\,\right) \,=\, \diag(+1,-1,-1,-1)$.
	% \begin{equation*}
	% \left[\,
	% 	M_{\mu\nu}
	% 	\,\overset{{\color{white}1}}{,}\,
	% 	M_{\rho\sigma}
	% 	\,\right]
	% \; = \,
	% 	\,-\,
	% 	\eta_{\rho\mu}M_{\nu\sigma}
	% 	\,+\,
	% 	\eta_{\rho\nu}M_{\mu\sigma}
	% 	\,-\,
	% 	\eta_{\sigma\mu}M_{\rho\nu}
	% 	\,+\,
	% 	\eta_{\sigma\nu}M_{\rho\mu}
	% \end{equation*}
\end{enumerate}
where
\,$\left(\,\eta_{\mu\nu}\,\right) \,=\, \diag(+1,-1,-1,-1)$.
\end{proposition}
\proof
\begin{enumerate}
\item
	First, we write \,$P_{\rho} \in \Re^{5 \times 5}$\, in block form:
	\begin{equation*}
	P_{\mu}
	\;\; = \;\;
		\left[\begin{array}{cc}
			\mathbf{0}_{4 \times 4} & \mathbf{e}_{\mu}
			\\
			\mathbf{0}_{1 \times 4} & \mathbf{0}_{1 \times 1}
			\end{array}\right]
	\end{equation*}
	Hence,
	\begin{eqnarray*}
	P_{\mu}\,P_{\nu}
	& = &
		\left[\begin{array}{cc}
			\mathbf{0}_{4 \times 4} & \mathbf{e}_{\mu}
			\\
			\mathbf{0}_{1 \times 4} & \mathbf{0}_{1 \times 1}
			\end{array}\right]
		\cdot
		\left[\begin{array}{cc}
			\mathbf{0}_{4 \times 4} & \mathbf{e}_{\nu}
			\\
			\mathbf{0}_{1 \times 4} & \mathbf{0}_{1 \times 1}
			\end{array}\right]
	\;\; = \;\;
		\left[\begin{array}{cc}
			\mathbf{0}_{4 \times 4}\,\mathbf{0}_{4 \times 4} + {\color{white}..}\mathbf{e}_{\mu}{\color{white}1}\mathbf{0}_{1 \times 4}
			&
			\mathbf{0}_{4 \times 4}\,\mathbf{e}_{\nu} + {\color{white}..}\mathbf{e}_{\mu}{\color{white}1}\mathbf{0}_{1 \times 1}
			\\
			\mathbf{0}_{1 \times 4}\,\mathbf{0}_{4 \times 4} + \mathbf{0}_{1 \times 1}\,\mathbf{0}_{1 \times 4}
			&
			\mathbf{0}_{1 \times 4}\,\mathbf{e}_{\nu} + \mathbf{0}_{1 \times 1}\,\mathbf{0}_{1 \times 1}
			\end{array}\right]
	\\
	& = &
		\left[\begin{array}{cc}
			\mathbf{0}_{4 \times 4} & \mathbf{0}_{4 \times 1}
			\\
			\mathbf{0}_{1 \times 4} & \mathbf{0}_{1 \times 1}
			\end{array}\right]
	\;\; = \;\;
		\mathbf{0}_{5 \times 5}
	\end{eqnarray*}
	The above observation immediately implies:
	\begin{equation*}
	\left[\,P_{\mu}\,\overset{{\color{white}1}}{,}\,P_{\nu}\,\right]
	\;\; = \;\;
		P_{\mu}\,P_{\nu} \,-\, P_{\nu}\,P_{\mu}
	\;\; = \;\;
		\mathbf{0}_{5 \times 5} \,-\, \mathbf{0}_{5 \times 5}
	\;\; = \;\;
		\mathbf{0}_{5 \times 5}\,,
	\end{equation*}
	as required.
\item
	First, we write \,$M_{\mu\nu}, P_{\rho} \in \Re^{5 \times 5}$\, in block form:
	\begin{equation*}
	M_{\mu\nu}
	\;\; = \;\;
		\left[\begin{array}{cc}
			J_{\mu\nu} & \mathbf{0}_{4 \times 1}
			\\
			\mathbf{0}_{1 \times 4} & \mathbf{0}_{1 \times 1}
			\end{array}\right]\,,
	\quad\quad
	\textnormal{and}
	\quad\quad
	P_{\rho}
	\;\; = \;\;
		\left[\begin{array}{cc}
			\mathbf{0}_{4 \times 4} & \mathbf{e}_{\rho}
			\\
			\mathbf{0}_{1 \times 4} & \mathbf{0}_{1 \times 1}
			\end{array}\right]\,,
	\end{equation*}

	\vskip 0.3cm
	\noindent
	\textbf{Claim 1:}\quad
	$
	\left[\,
		M_{\mu\nu}
		\,\overset{{\color{white}1}}{,}\,
		P_{\rho}
		\,\right]
	% \; = \;
	% 	M_{\mu\nu}\,P_{\rho} \,-\, P_{\rho}\,M_{\mu\nu}
	\; = \;
		\left[\begin{array}{cc}
			\mathbf{0}_{4 \times 4} & J_{\mu\nu}\,\mathbf{e}_{\rho}
			\\
			\mathbf{0}_{1 \times 4} & \mathbf{0}_{1 \times 1}
			\end{array}\right]
	$
	\vskip 0.1cm
	\noindent
	Proof of Claim 1:\;\;
	Observe
	\begin{eqnarray*}
	M_{\mu\nu}\,P_{\rho}
	& = &
		\left[\begin{array}{cc}
			J_{\mu\nu} & \mathbf{0}_{4 \times 1}
			\\
			\mathbf{0}_{1 \times 4} & \mathbf{0}_{1 \times 1}
			\end{array}\right]
		\cdot
		\left[\begin{array}{cc}
			\mathbf{0}_{4 \times 4} & \mathbf{e}_{\rho}
			\\
			\mathbf{0}_{1 \times 4} & \mathbf{0}_{1 \times 1}
			\end{array}\right]
		\;\; = \;\;
		\left[\begin{array}{cc}
			{\color{white}.}J_{\mu\nu}{\color{white}1}\mathbf{0}_{4 \times 4} + \mathbf{0}_{4 \times 1}\,\mathbf{0}_{1 \times 4}
			&
			{\color{white}.}J_{\mu\nu}{\color{white}1}\mathbf{e}_{\rho} + \mathbf{0}_{4 \times 1}\,\mathbf{0}_{1 \times 1}
			\\
			\mathbf{0}_{1 \times 4}\,\mathbf{0}_{4 \times 4} + \mathbf{0}_{1 \times 1}\,\mathbf{0}_{1 \times 4}
			&
			\mathbf{0}_{1 \times 4}\,\mathbf{e}_{\rho} + \mathbf{0}_{1 \times 1}\,\mathbf{0}_{1 \times 1}
			\end{array}\right]
	\\
	& = &
		\left[\begin{array}{cc}
			\mathbf{0}_{4 \times 4} & J_{\mu\nu}\,\mathbf{e}_{\rho}
			\\
			\mathbf{0}_{1 \times 4} & \mathbf{0}_{1 \times 1}
			\end{array}\right]
	\end{eqnarray*}
	\begin{eqnarray*}
	P_{\rho}\,M_{\mu\nu}
	& = &
		\left[\begin{array}{cc}
			\mathbf{0}_{4 \times 4} & \mathbf{e}_{\rho}
			\\
			\mathbf{0}_{1 \times 4} & \mathbf{0}_{1 \times 1}
			\end{array}\right]
		\cdot
		\left[\begin{array}{cc}
			J_{\mu\nu} & \mathbf{0}_{4 \times 1}
			\\
			\mathbf{0}_{1 \times 4} & \mathbf{0}_{1 \times 1}
			\end{array}\right]
		\;\; = \;\;
		\left[\begin{array}{cc}
			\mathbf{0}_{4 \times 4}{\color{white}.}J_{\mu\nu} + {\color{white}...}\mathbf{e}_{\rho}{\color{white}1}\mathbf{0}_{1 \times 4}
			&
			\mathbf{0}_{4 \times 4}\,\mathbf{0}_{4 \times 1} + {\color{white}..}\mathbf{e}_{\rho}{\color{white}1}\mathbf{0}_{1 \times 1}
			\\
			\mathbf{0}_{1 \times 4}{\color{white}.}J_{\mu\nu} + \mathbf{0}_{1 \times 1}\,\mathbf{0}_{1 \times 4}
			&
			\mathbf{0}_{1 \times 4}\,\mathbf{0}_{4 \times 1} + \mathbf{0}_{1 \times 1}\,\mathbf{0}_{1 \times 1}
			\end{array}\right]
	\\
	& = &
		\left[\begin{array}{cc}
			\mathbf{0}_{4 \times 4} & \mathbf{0}_{4 \times 1}
			\\
			\mathbf{0}_{1 \times 4} & \mathbf{0}_{1 \times 1}
			\end{array}\right]
	\end{eqnarray*}
	Hence,
	\begin{eqnarray*}
	\left[\,
		M_{\mu\nu}
		\,\overset{{\color{white}1}}{,}\,
		P_{\rho}
		\,\right]
	& = &
		M_{\mu\nu}\,P_{\rho} \,-\, P_{\rho}\,M_{\mu\nu}
	\;\; = \;\;
		\left[\begin{array}{cc}
			\mathbf{0}_{4 \times 4} & J_{\mu\nu}\,\mathbf{e}_{\rho}
			\\
			\mathbf{0}_{1 \times 4} & \mathbf{0}_{1 \times 1}
			\end{array}\right]
	\end{eqnarray*}
	This proves Claim 1.

	\vskip 0.3cm
	\noindent
	\textbf{Claim 2}:\quad
	$
	J_{\mu\nu}\,\mathbf{e}_{\rho}
	\; = \;
		-\,\eta_{\rho\mu}\,\mathbf{e}_{\nu}
		\;+\;
		\eta_{\rho\nu}\,\mathbf{e}_{\mu}
	$
	\vskip 0.1cm
	\noindent
	Proof of Claim 2:\;\;
	Note that the matrix entries of \,$J_{\mu\nu}$\, are given by
	\begin{equation*}
	\left(J_{\mu\nu}\right)_{ab}
	\; = \;
		\delta_{a\mu}\,\eta_{b\nu}
		\,-\,
		\delta_{a\nu}\,\eta_{b\mu}\,,
		\quad
		\textnormal{for \,$a, b = 0,1,2,3$}
	\end{equation*}
	Hence,
	\begin{eqnarray*}
	\left(\,J_{\mu\nu}\,\mathbf{e}_{\rho}\,\right)_{a}
	& = &
		\overset{3}{\underset{c=0}{\sum}}
		\left(J_{\mu\nu}\right)_{ac}
		\left(\mathbf{e}_{\rho}\right)_{c}
	\;\; = \;\;
		\overset{3}{\underset{c=0}{\sum}}
		\left(\,
			\delta_{a\mu}\,\eta_{c\nu}
			\,-\,
			\delta_{a\nu}\,\eta_{c\mu}
			\,\right)
		\delta_{c\rho}
	\;\; = \;\;
		\overset{3}{\underset{c=0}{\sum}}\;\,
		\delta_{a\mu}\,\eta_{c\nu}\,\delta_{c\rho}
		\; - \;
		\overset{3}{\underset{c=0}{\sum}}\;\,
		\delta_{a\nu}\,\eta_{c\mu}\,\delta_{c\rho}
	\\
	& = &
		\delta_{a\mu}\,\eta_{\nu\rho}
		\; - \;
		\delta_{a\nu}\,\eta_{\mu\rho}
	\;\; = \;\;
		-\,\eta_{\rho\mu}\left(\,\overset{{\color{white}.}}{\mathbf{e}_{\nu}}\,\right)_{a}
		\;+\;
		\eta_{\rho\nu}\left(\,\overset{{\color{white}.}}{\mathbf{e}_{\mu}}\,\right)_{a}
	\\
	& = &
		\left(\;
			-\,\eta_{\rho\mu}\,\mathbf{e}_{\nu}
			\,\overset{{\color{white}.}}{+}\,
			\eta_{\rho\nu}\,\mathbf{e}_{\mu}
			\;\right)_{a}
	\end{eqnarray*}
	This proves Claim 2.

	\vskip 0.3cm
	\noindent
	By Claim 1 and Claim 2, we see that:
	\begin{eqnarray*}
	\left[\,
		M_{\mu\nu}
		\,\overset{{\color{white}1}}{,}\,
		P_{\rho}
		\,\right]
	& = &
		\left[\begin{array}{cc}
			\mathbf{0}_{4 \times 4} & J_{\mu\nu}\,\mathbf{e}_{\rho}
			\\
			\mathbf{0}_{1 \times 4} & \mathbf{0}_{1 \times 1}
			\end{array}\right]
	\;\; = \;\;
		\left[\begin{array}{cc}
			\mathbf{0}_{4 \times 4}
			&
			- \eta_{\rho\mu}\,\mathbf{e}_{\nu} + \eta_{\rho\nu}\,\mathbf{e}_{\mu}
			\\
			\mathbf{0}_{1 \times 4} & \mathbf{0}_{1 \times 1}
			\end{array}\right]
	\\
	& = &
		-\,\eta_{\rho\mu}
		\left[\begin{array}{cc}
			\mathbf{0}_{4 \times 4} & \mathbf{e}_{\nu}
			\\
			\mathbf{0}_{1 \times 4} & \mathbf{0}_{1 \times 1}
			\end{array}\right]
		\; + \;
		\eta_{\rho\nu}
		\left[\begin{array}{cc}
			\mathbf{0}_{4 \times 4} & \mathbf{e}_{\mu}
			\\
			\mathbf{0}_{1 \times 4} & \mathbf{0}_{1 \times 1}
			\end{array}\right]
	\\
	& \overset{{\color{white}\textnormal{\large$1$}}}{=} &
		-\,\eta_{\rho\mu}\,P_{\nu}
		\, + \,
		\eta_{\rho\nu}\,P_{\mu}\,,
	\end{eqnarray*}
	as required.
\item
	We compute:
	\begin{eqnarray*}
	\left[\,
		M_{\mu\nu}
		\,\overset{{\color{white}1}}{,}\,
		M_{\rho\sigma}
		\,\right]_{ab}
	& = &
		\left(\,
			M_{\mu\nu}\,M_{\rho\sigma}
			\,\overset{{\color{white}1}}{-}\,
			M_{\rho\sigma}\,M_{\mu\nu}
			\,\right)_{ab}
	\;\; = \;\;
		\overset{3}{\underset{c=0}{\sum}}\,
		(M_{\mu\nu})_{ac}(M_{\rho\sigma})_{cb}
		\; - \;
		\overset{3}{\underset{c=0}{\sum}}\,
		(M_{\rho\sigma})_{ac}(M_{\mu\nu})_{cb}
	\\
	& = &
		\overset{3}{\underset{c=0}{\sum}}
		\left(\,%(M_{\mu\nu})_{ac}
			\delta_{a\mu}\,\eta_{c\nu}
			-
			\delta_{a\nu}\,\eta_{c\mu}
			\,\right)
		\left(\,%(M_{\rho\sigma})_{cb}
			\delta_{c\rho}\,\eta_{b\sigma}
			-
			\delta_{c\sigma}\,\eta_{b\rho}
			\,\right)
		\;+\;
		\overset{3}{\underset{c=0}{\sum}}
		\left(\,%(M_{\rho\sigma})_{ac}
			\delta_{a\rho}\,\eta_{c\sigma}
			-
			\delta_{a\sigma}\,\eta_{c\rho}
			\,\right)
		\left(\,%(M_{\mu\nu})_{cb}
			\delta_{c\mu}\,\eta_{b\nu}
			-
			\delta_{c\nu}\,\eta_{b\mu}
			\,\right)
	\\
	& = &
		{\color{white}+}\;
		\overset{3}{\underset{c=0}{\sum}}
		% \left(\,%(M_{\mu\nu})_{ac}
		% 	\eta_{a\mu}\,\delta_{c\nu}
		% 	-
		% 	\eta_{a\nu}\,\delta_{c\mu}
		% 	\,\right)
		% \left(\,%(M_{\rho\sigma})_{cb}
		% 	\eta_{c\rho}\,\delta_{b\sigma}
		% 	-
		% 	\eta_{c\sigma}\,\delta_{b\rho}
		% 	\,\right)
		\left(\,
			\delta_{a\mu}\,\eta_{c\nu}\,\delta_{c\rho}\,\eta_{b\sigma}
			\,\overset{{\color{white}.}}{-}\,
			\delta_{a\mu}\,\eta_{c\nu}\,\delta_{c\sigma}\,\eta_{b\rho}
			\,-\,
			\delta_{a\nu}\,\eta_{c\mu}\,\delta_{c\rho}\,\eta_{b\sigma}
			\,+\,
			\delta_{a\nu}\,\eta_{c\mu}\,\delta_{c\sigma}\,\eta_{b\rho}
			\,\right)
	\\
	&&
		+\;
		\overset{3}{\underset{c=0}{\sum}}
		% \left(\,%(M_{\rho\sigma})_{ac}
		% 	\eta_{a\rho}\,\delta_{c\sigma}
		% 	-
		% 	\eta_{a\sigma}\,\delta_{c\rho}
		% 	\,\right)
		% \left(\,%(M_{\mu\nu})_{cb}
		% 	\eta_{c\mu}\,\delta_{b\nu}
		% 	-
		% 	\eta_{c\nu}\,\delta_{b\mu}
		% 	\,\right)
		\left(\,
			\delta_{a\rho}\,\eta_{c\sigma}\,\delta_{c\mu}\,\eta_{b\nu}
			\,\overset{{\color{white}.}}{-}\,
			\delta_{a\rho}\,\eta_{c\sigma}\,\delta_{c\nu}\,\eta_{b\mu}
			\,-\,
			\delta_{a\sigma}\,\eta_{c\rho}\,\delta_{c\mu}\,\eta_{b\nu}
			\,+\,
			\delta_{a\sigma}\,\eta_{c\rho}\,\delta_{c\nu}\,\eta_{b\mu}
			\,\right)
	\\
	& = &
		{\color{white}+}\;
		\left(\,
			\delta_{a\mu}\,\eta_{\rho\nu}\,\eta_{b\sigma}
			\,\overset{{\color{white}.}}{-}\,
			\delta_{a\mu}\,\eta_{\sigma\nu}\,\eta_{b\rho}
			\,-\,
			\delta_{a\nu}\,\eta_{\rho\mu}\,\eta_{b\sigma}
			\,+\,
			\delta_{a\nu}\,\eta_{\sigma\mu}\,\eta_{b\rho}
			\,\right)
	\\
	& &
		+\;
		\left(\,
			\delta_{a\rho}\,\eta_{\mu\sigma}\,\eta_{b\nu}
			\,\overset{{\color{white}.}}{-}\,
			\delta_{a\rho}\,\eta_{\nu\sigma}\,\eta_{b\mu}
			\,-\,
			\delta_{a\sigma}\,\eta_{\mu\rho}\,\eta_{b\nu}
			\,+\,
			\delta_{a\sigma}\,\eta_{\nu\rho}\,\eta_{b\mu}
			\,\right)
	\\
	& \overset{{\color{white}\textnormal{\large$1$}}}{=} &
		-\,\eta_{\mu\rho}\left(\,
			\delta_{a\nu}\,\eta_{b\sigma}
			+
			\delta_{a\sigma}\,\eta_{b\nu}
			\,\right)
		\,+\,
		\eta_{\mu\sigma}\left(\,
			\delta_{a\nu}\,\eta_{b\rho}
			+
			\delta_{a\rho}\,\eta_{b\nu}
			\,\right)
	\\
	& &
		+\,
		\eta_{\nu\rho}\left(\,
			\delta_{a\mu}\,\eta_{b\sigma}
			+
			\delta_{a\sigma}\,\eta_{b\mu}
			\,\right)
		\,-\,
		\eta_{\nu\sigma}\left(\,
			\delta_{a\mu}\,\eta_{b\rho}
			+
			\delta_{a\rho}\,\eta_{b\mu}
			\,\right)
	\\
	& \overset{{\color{white}\textnormal{\Large$1$}}}{=} &
		-\,\eta_{\mu\rho}\left(\,M_{\nu\sigma}\,\right)_{ab}
		\;+\;
		\eta_{\mu\sigma}\left(\,M_{\nu\rho}\,\right)_{ab}
		\;+\;
		\eta_{\nu\rho}\left(\,M_{\mu\sigma}\,\right)_{ab}
		\;-\;
		\eta_{\nu\sigma}\left(\,M_{\mu\rho}\,\right)_{ab}
	\end{eqnarray*}
	The calculations above show that
	\begin{eqnarray*}
	\left[\,
		M_{\mu\nu}
		\,\overset{{\color{white}1}}{,}\,
		M_{\rho\sigma}
		\,\right]
	& = &
		\,-\,
		\eta_{\mu\rho}M_{\nu\sigma}
		\,+\,
		\eta_{\mu\sigma}M_{\nu\rho}
		\,+\,
		\eta_{\nu\rho}M_{\mu\sigma}
		\,-\,
		\eta_{\nu\sigma}M_{\mu\rho}
	\\
	& \overset{{\color{white}\textnormal{\large$1$}}}{=} &
		\,-\,
		\eta_{\rho\mu}M_{\nu\sigma}
		\,+\,
		\eta_{\rho\nu}M_{\mu\sigma}
		\,-\,
		\eta_{\sigma\mu}M_{\rho\nu}
		\,+\,
		\eta_{\sigma\nu}M_{\rho\mu}\,,
	\end{eqnarray*}
	as required.
\end{enumerate}
\qed

\vskip 0.5cm
\begin{proposition}
\mbox{}
\vskip 0.1cm
\noindent
Suppose:
\begin{itemize}
\item
	$Q_{0},\, Q_{1},\, Q_{2},\, Q_{3}$\, are $\C$-linear operators
	on a complex Hilbert space \,$\mathcal{H}$,\, and
\item
	$U : \SOup(1,3) \longrightarrow U(\mathcal{H})$\,
	is a unitary representation of the proper orthochronous Lorentz group
	\,$\SOup(1,3)$\, on \,$\mathcal{H}$.
\end{itemize}
The quantity
\,$\left(\,Q_{\mu}\,\right) \,:=\, \left(\,Q_{0},Q_{1},Q_{2},Q_{4}\,\right)$\,
transforms as a $4$-vector (with respect to the unitary representation $U$)
if and only if \,$Q_{\mu}$\, satisfies the following commutation relations:
\begin{equation*}
\left[\,M_{\mu\nu}
	\,\overset{{\color{white}1}}{,}\,
	Q_{\rho}
	\,\right]
\;\; = \;\;
	-\,\eta_{\rho\mu}\,Q_{\nu}
	\,+\,
	\,\eta_{\rho\nu}\,Q_{\mu}
\end{equation*}
\end{proposition}
\proof
That \,$Q_{\mu}$\, transforms as a $4$-vector means the following:
\begin{equation*}
U(\Lambda)^{-1} \,\circ\, Q_{\mu} \,\circ\, U(\Lambda)
\;\; = \;\;
	\Lambda_{\mu}^{{\color{white}\mu}\nu}\,Q_{\nu}
\end{equation*}
We will establish that the desired commutation relation
is an infinitesimal version of this transformation law.
\begin{equation*}
\Lambda_{\mu}^{{\color{white}\mu}\nu}
\;\; = \;\;
	\delta_{\mu}^{\nu}
	\,+\,
	\omega_{\mu}^{{\color{white}\mu}\nu}
\end{equation*}
\begin{equation*}
U(\Lambda)
\;\; \approx \;\;
	I
	\,+\,
	\dfrac{1}{2}\,\omega^{\alpha\beta}\,M_{\alpha\beta}
\end{equation*}
\begin{equation*}
\left(\,
	I \,-\, \dfrac{1}{2}\,\omega^{\alpha\beta}\,M_{\alpha\beta}
	\,\right)
\cdot
Q_{\rho}
\cdot
\left(\,
	I \,+\, \dfrac{1}{2}\,\omega^{\mu\nu}\,M_{\mu\nu}
	\,\right)
\;\; \approx \;\;
	U(\Lambda)^{-1} \cdot Q_{\rho} \cdot U(\Lambda)
\;\; = \;\;
	\Lambda_{\rho}^{{\color{white}\mu}\nu}\,Q_{\nu}
\;\; \approx \;\;
	\left(\,
		\delta_{\rho}^{\nu} \,+\, \omega_{\rho}^{{\color{white}\alpha}\nu}
		\,\right)
	Q_{\nu}
\end{equation*}
\begin{eqnarray*}
&&
	\left(\,
		I \,-\, \dfrac{1}{2}\,\omega^{\alpha\beta}\,M_{\alpha\beta}
		\,\right)
	\cdot
	Q_{\rho}
	\cdot
	\left(\,
		I \,+\, \dfrac{1}{2}\,\omega^{\mu\nu}\,M_{\mu\nu}
		\,\right)
\\
& = &
	\left(\,
		Q_{\rho} \,-\, \dfrac{1}{2}\,\omega^{\alpha\beta}\,M_{\alpha\beta}\,Q_{\rho}
		\,\right)
	\cdot
	\left(\,
		I \,+\, \dfrac{1}{2}\,\omega^{\mu\nu}\,M_{\mu\nu}
		\,\right)
\\
& = &
	Q_{\rho}
	\,+\,
	\dfrac{1}{2}\,\omega^{\mu\nu}\,Q_{\rho}\,M_{\mu\nu}
	\,-\,
	\dfrac{1}{2}\,\omega^{\alpha\beta}\,M_{\alpha\beta}\,Q_{\rho}
	\,-\,
	\dfrac{1}{4}\,\omega^{\alpha\beta}\,M_{\alpha\beta}\,Q_{\rho}
	\dfrac{1}{2}\,\omega^{\mu\nu}\,M_{\mu\nu}
	% \left(\,
	% 	Q_{\mu} \,-\, \dfrac{1}{2}\,\omega^{\alpha\beta}\,M_{\alpha\beta}\,Q_{\mu}
	% 	\,\right)
	% \cdot
	% \left(\,
	% 	I \,+\, \dfrac{1}{2}\,\omega^{\alpha\beta}\,M_{\alpha\beta}
	% 	\,\right)
\\
& = &
	Q_{\rho}
	\,+\,
	\dfrac{1}{2}\,\omega^{\mu\nu}\left[\,
		Q_{\rho}\,\overset{{\color{white}1}}{,}\,M_{\mu\nu}
		\,\right]
	\,+\,
	\left\{\begin{array}{c}
		\textnormal{terms of higher}
		\\
		\textnormal{order in \,$\omega_{\mu\nu}$}
		\end{array}\right\}		
\end{eqnarray*}
\begin{equation*}
Q_{\rho}
\,+\,
\dfrac{1}{2}\,\omega^{\mu\nu}\left[\,
	Q_{\rho}\,\overset{{\color{white}1}}{,}\,M_{\mu\nu}
	\,\right]
\;\; = \;\;
	Q_{\rho}
	\,+\,
	\omega_{\rho}^{{\color{white}\mu}\nu}\,Q_{\nu}
\end{equation*}
\begin{equation*}
\dfrac{1}{2}\,\omega^{\mu\nu}\left[\,
	Q_{\rho}\,\overset{{\color{white}1}}{,}\,M_{\mu\nu}
	\,\right]
\;\; = \;\;
	\omega_{\rho}^{{\color{white}\mu}\nu}\,Q_{\nu}
\end{equation*}
Now observe:
% \begin{equation*}
% \omega_{\mu}^{{\color{white}\mu}\nu}\,Q_{\nu}
% \;\; = \;\;
% 	\left(\,\omega^{\alpha\nu}\,\eta_{\alpha\mu}\,\right)\left(\,\eta_{\nu\lambda}\,Q^{\lambda}\,\right)
% \;\; = \;\;
% 	\left(\,\omega^{\alpha\beta}\,\eta_{\alpha\mu}\,\right)\left(\,\eta_{\beta\nu}\,Q^{\nu}\,\right)
% \end{equation*}
\begin{eqnarray*}
\omega_{\rho}^{{\color{white}\mu}\nu}\,Q_{\nu}
& = &
	\left(\,\omega^{\mu\nu}\,\eta_{\mu\rho}\,\right)Q_{\nu}
\;\; = \;\;
	\omega^{\mu\nu}\,\eta_{\mu\rho}\,Q_{\nu}
\\
& = &
	\omega^{\mu\nu}\cdot\dfrac{1}{2}\left(\,
		\eta_{\mu\rho}\,Q_{\nu}
		\,\overset{{\color{white}.}}{-}\,
		\eta_{\nu\rho}\,Q_{\mu}
		\,\right),
	\quad
	\textnormal{due to anti-symmetry of \,$\omega^{\mu\nu}$\, in \,$\mu,\,\nu$}
\end{eqnarray*}
Hence,
\begin{equation*}
\dfrac{1}{2}\,\omega^{\mu\nu}\left[\,
	Q_{\rho}\,\overset{{\color{white}1}}{,}\,M_{\mu\nu}
	\,\right]
\;\; = \;\;
	\omega_{\rho}^{{\color{white}\mu}\nu}\,Q_{\nu}
\;\; = \;\;
	\omega^{\mu\nu}\cdot\dfrac{1}{2}\left(\,
		\eta_{\mu\rho}\,Q_{\nu}
		\,\overset{{\color{white}.}}{-}\,
		\eta_{\nu\rho}\,Q_{\mu}
		\,\right),
\end{equation*}
which implies:
\begin{equation*}
\left[\,
	Q_{\rho}\,\overset{{\color{white}1}}{,}\,M_{\mu\nu}
	\,\right]
\;\; = \;\;
	\eta_{\mu\rho}\,Q_{\nu}
	\,\overset{{\color{white}.}}{-}\,
	\eta_{\nu\rho}\,Q_{\mu}
\end{equation*}
which in turn is equivalent to
\begin{equation*}
\left[\,
	M_{\mu\nu}\,\overset{{\color{white}1}}{,}\,Q_{\rho}
	\,\right]
\;\; = \;\;
	\,\overset{{\color{white}.}}{-}\,
	\eta_{\mu\rho}\,Q_{\nu}
	\,\overset{{\color{white}.}}{+}\,
	\eta_{\nu\rho}\,Q_{\mu}
\;\; = \;\;
	\,\overset{{\color{white}.}}{-}\,
	\eta_{\rho\mu}\,Q_{\nu}
	\,\overset{{\color{white}.}}{+}\,
	\eta_{\rho\nu}\,Q_{\mu}
\end{equation*}
as required.
\qed

\vskip 0.5cm
\begin{definition}[Pauli-Lubanski $4$-vector]
\mbox{}
\vskip 0.1cm
\noindent
The \textbf{Pauli-Lubanski $4$-vector} \,$W^{\mu}$\, is defined as follows:
\begin{equation*}
W_{\mu}
\;\; := \;\;
	\dfrac{1}{2}\,\varepsilon_{\mu\nu\rho\sigma}\,M^{\nu\rho}\,P^{\sigma}
\end{equation*}
\end{definition}

\vskip 0.5cm
\begin{proposition}[Properties of the Pauli-Lubanski $4$-vector]
\mbox{}
\vskip 0.1cm
\noindent
The \textbf{Pauli-Lubanski $4$-vector} \,$W^{\mu}$\,
satisfies the following commutation relations with the generators
\,$P_{\mu}$\, and \,$M_{\mu\nu}$:
\begin{enumerate}
\item
	$
	\left[\,
		P_{\mu}
		\,\overset{{\color{white}1}}{,}\,
		W_{\nu}
		\,\right]
	\; = \;
		0\,,
	$
\item
	$
	\left[\,
		M_{\mu\nu}
		\,\overset{{\color{white}1}}{,}\,
		W_{\rho}
		\,\right]
	\; = \;
		-\,
		\eta_{\rho\mu}\,W_{\nu}
		\,+\,
		\eta_{\rho\nu}\,W_{\mu}
	$
\end{enumerate}
\end{proposition}
\proof
\begin{enumerate}
\item
	We compute:
	\begin{eqnarray*}
	\left[\,
		P_{\mu}
		\,\overset{{\color{white}1}}{,}\,
		W_{\nu}
		\,\right]
	& = &
		P_{\mu}\,W_{\nu} \,-\, W_{\nu}\,P_{\mu}
	\;\; = \;\;
		P_{\mu}
		\cdot\!
		\left(\,\dfrac{1}{2}\,\varepsilon_{\nu\lambda\rho\sigma}\,M^{\lambda\rho}\,P^{\sigma}\,\right)
		\,-\,
		\left(\,\dfrac{1}{2}\,\varepsilon_{\nu\lambda\rho\sigma}\,M^{\lambda\rho}\,P^{\sigma}\,\right)
		\!\cdot
		P_{\mu}
	\\
	& = &
		\dfrac{1}{2}\,\varepsilon_{\nu\lambda\rho\sigma}\,P_{\mu}\left(\,M^{\lambda\rho}\,P^{\sigma}\,\right)
		\;-\;
		\dfrac{1}{2}\,\varepsilon_{\nu\lambda\rho\sigma}\,M^{\lambda\rho}\left(\,P^{\sigma}P_{\mu}\,\right)
	\\
	& = &
		\dfrac{1}{2}\,\varepsilon^{\mu\nu\rho\sigma}\,P_{\lambda}
		\left(\,
			P_{\sigma}\,M_{\nu\rho}
			\;\overset{{\color{white}1}}{-}\;
			\eta_{\sigma\nu}\,P_{\rho}
			\;+\;
			\eta_{\sigma\rho}\,P_{\nu}
			\,\right)
		\;-\;
		\dfrac{1}{2}\,\varepsilon^{\mu\nu\rho\sigma}\,M_{\nu\rho}\left(\;0\;\right)
	\\
	& = &
		\dfrac{1}{2}\,\varepsilon^{\mu\nu\rho\sigma} %\,P_{\lambda}
		\left(\,
			P_{\lambda}\,P_{\sigma}\,M_{\nu\rho}
			\;\overset{{\color{white}1}}{-}\;
			\eta_{\sigma\nu}\,P_{\lambda}\,P_{\rho}
			\;+\;
			\eta_{\sigma\rho}\,P_{\lambda}\,P_{\nu}
			\,\right)
	\\
	& = &
		\dfrac{1}{2}\,\varepsilon^{\mu\nu\rho\sigma}
		\left(\,
			0 \cdot M_{\nu\rho}
			\;\overset{{\color{white}1}}{-}\;
			\eta_{\sigma\nu} \cdot 0
			\;+\;
			\eta_{\sigma\rho} \cdot 0
			\,\right)
	\\
	& \overset{{\color{white}\textnormal{\large$1$}}}{=} &
		0
	\end{eqnarray*}
\item
	\textbf{Claim 1:}\quad
	$W^{\mu}P_{\mu} \,=\, 0$
	\vskip 0.1cm
	\noindent
	Proof of Claim 1:\;\;
	\begin{equation*}
	W^{\mu}\,P_{\mu}
	\;\; = \;\;
		\dfrac{1}{2}\,\varepsilon^{\mu\nu\rho\sigma}\,M_{\nu\rho}\,P_{\sigma}
		\cdot
		P_{\mu}
	\;\; = \;\;
		0\,,
	\end{equation*}
	where the last equality follows from the anti-symmetry of
	\,$\varepsilon^{\mu\nu\rho\sigma}$\, (in \,$\mu, \nu, \rho, \sigma$)
	and the symmetry of
	\,$P_{\sigma}P_{\mu}$\, (in \,$\sigma, \mu$).
	This proves Claim 1.

	\begin{eqnarray*}
	\left[\,M_{\kappa\lambda}\,,W^{\mu}P_{\mu}\,\right]
	& = &
		M_{\kappa\lambda}\,W^{\mu}\,P_{\mu}
		\,-\,
		W^{\mu}\,P_{\mu}\,M_{\kappa\lambda}
	\\
	& = &
		M_{\kappa\lambda}\,W^{\mu}\,P_{\mu}
		{\color{red}
		\;-\;
		W^{\mu}\,M_{\kappa\lambda}\,P_{\mu}
		\,+\,
		W^{\mu}\,M_{\kappa\lambda}\,P_{\mu}
		}
		\,-\,
		W^{\mu}\,P_{\mu}\,M_{\kappa\lambda}
	\\
	& = &
		\left[\,M_{\kappa\lambda}\,,W^{\mu}\,\right] P_{\mu}
		\,+\,
		W^{\mu} \left[\,M_{\kappa\lambda}\,,P_{\mu}\,\right]
	\\
	& = &
		\left[\,M_{\kappa\lambda}\,,W^{\mu}\,\right] P_{\mu}
		\,+\,
		W^{\mu} \left(\,-\,\eta_{\mu\kappa}\,P_{\lambda}\,+\,\eta_{\mu\lambda}\,P_{\kappa}\,\right)
	\end{eqnarray*}

	\begin{eqnarray*}
	\left[\,M_{\alpha\beta}\,,W^{\mu}\,\right] P_{\mu}
	& = &
		W^{\mu} \left(\,
			\eta_{\mu\alpha}\,P_{\beta}
			\,-\,
			\eta_{\mu\beta}\,P_{\alpha}
			\,\right)
	\;\; = \;\;
			W_{\alpha}\,\delta^{\mu}_{\beta}\,P_{\mu}
			\,-\,
			W_{\beta}\,\delta^{\mu}_{\alpha}\,P_{\mu}
	\\
	& = &
		\left(\,
			W_{\alpha}\,\delta^{\mu}_{\beta}
			\,-\,
			W_{\beta}\,\delta^{\mu}_{\alpha}
			\,\right)
			P_{\mu}
	\end{eqnarray*}
	\vskip 0.3cm
	\noindent
	We compute:
	\begin{eqnarray*}
	\left[\,
		M_{\alpha\beta}
		\,\overset{{\color{white}1}}{,}\,
		W^{\mu}
		\,\right]
	& = &
		\left[\,
			M_{\alpha\beta}
			\,\overset{{\color{white}1}}{,}\,
			\dfrac{1}{2}\,\varepsilon^{\mu\nu\rho\sigma}\,M_{\nu\rho}\,P_{\sigma}
			\,\right]
	\;\; = \;\;
		\dfrac{1}{2}\,\varepsilon^{\mu\nu\rho\sigma}
		\left[\,
			M_{\alpha\beta}
			\,\overset{{\color{white}1}}{,}\,
			M_{\nu\rho}\,P_{\sigma}
			\,\right]
	\\
	& = &
		\dfrac{1}{2}\,\varepsilon^{\mu\nu\rho\sigma}
		\left(\,
			M_{\alpha\beta}\,M_{\nu\rho}\,P_{\sigma}
			\,\overset{{\color{white}.}}{-}\,
			M_{\nu\rho}\,P_{\sigma}\,M_{\alpha\beta}
			\,\right)
	\\
	& = &
		\dfrac{1}{2}\,\varepsilon^{\mu\nu\rho\sigma}
		\left(\,
			M_{\alpha\beta}\,M_{\nu\rho}\,P_{\sigma}
			{\color{red}
			\;\,\overset{{\color{white}.}}{-}\;
			M_{\nu\rho}\,M_{\alpha\beta}\,P_{\sigma}
			\,\overset{{\color{white}.}}{+}\,
			M_{\nu\rho}\,M_{\alpha\beta}\,P_{\sigma}
			}
			\,\overset{{\color{white}.}}{-}\,
			M_{\nu\rho}\,P_{\sigma}\,M_{\alpha\beta}
			\,\right)
	\\
	& = &
		\dfrac{1}{2}\,\varepsilon^{\mu\nu\rho\sigma}
		\left(\,
			\left[\,M_{\alpha\beta}\,,M_{\nu\rho}\,\right] P_{\sigma}
			\,\overset{{\color{white}.}}{+}\,
			M_{\nu\rho} \left[\,M_{\alpha\beta}\,,P_{\sigma}\,\right]
			\,\right)
	\\
	& = &
		\dfrac{1}{2}\,\varepsilon^{\mu\nu\rho\sigma}
		\left(\,
			\left(\,
				\,-\,
				\eta_{\alpha\nu}M_{\beta\rho}
				\,+\,
				\eta_{\alpha\rho}M_{\beta\nu}
				\,+\,
				\eta_{\beta\nu}M_{\alpha\rho}
				\,-\,
				\eta_{\beta\rho}M_{\alpha\nu}
				\,\right) P_{\sigma}
			\,\overset{{\color{white}.}}{+}\,
			M_{\nu\rho} \left(\,
				-\,\eta_{\sigma\alpha}\,P_{\beta}
				\, + \,
				\eta_{\sigma\beta}\,P_{\alpha}
				\,\right)
			\,\right)
	\\
	& = &
		\dfrac{1}{2}\,\varepsilon^{\mu\nu\rho\sigma}
		\left(\,
			\left(\,
				\,-\,
				\eta_{\alpha\nu}M_{\beta\rho}
				\;\,{\color{red}-}\;
				\eta_{\alpha{\color{red}\nu}}M_{\beta{\color{red}\rho}}
				\,+\,
				\eta_{\beta\nu}M_{\alpha\rho}
				\;\,{\color{red}+}\;
				\eta_{\beta{\color{red}\nu}}M_{\alpha{\color{red}\rho}}
				\,\right) P_{\sigma}
			\,\overset{{\color{white}.}}{+}\,
			M_{\nu\rho} \left(\,
				-\,\eta_{\sigma\alpha}\,P_{\beta}
				\, + \,
				\eta_{\sigma\beta}\,P_{\alpha}
				\,\right)
			\,\right)
	\\
	& = &
		\varepsilon^{\mu\nu\rho\sigma}
		\left(\,
			\overset{{\color{white}.}}{-}\,
			\eta_{\alpha\nu}M_{\beta\rho}
			\,+\,\eta_{\beta\nu}M_{\alpha\rho}
			\,\right) P_{\sigma}
		\;+\;
		\dfrac{1}{2}\,\varepsilon^{\mu\nu\rho\sigma}\,M_{\nu\rho}
		\left(\,
			\overset{{\color{white}.}}{-}\,
			\eta_{\sigma\alpha}\,P_{\beta}
			\, + \,
			\eta_{\sigma\beta}\,P_{\alpha}
			\,\right)
	\\
	& \overset{{\color{white}\textnormal{\large$1$}}}{=} &
		\cdots
	\\
	& \overset{{\color{white}\textnormal{\large$1$}}}{=} &
		\eta_{\beta}\!{\color{white}.}^{\mu}\left(\,
			\dfrac{1}{2}\,\varepsilon_{\alpha\nu\rho\sigma}\,M^{\nu\rho}\,P^{\sigma}
			\,\right)
		\,-\,
		\eta_{\alpha}\!{\color{white}.}^{\mu}\left(\,
			\dfrac{1}{2}\,\varepsilon_{\beta\nu\rho\sigma}\,M^{\nu\rho}\,P^{\sigma}
			\,\right)
	\\
	& \overset{{\color{white}\textnormal{\large$1$}}}{=} &
		\eta_{\beta}\!{\color{white}.}^{\mu}\,W_{\alpha}
		\,-\,
		\eta_{\alpha}\!{\color{white}.}^{\mu}\,W_{\beta}
	\end{eqnarray*}
\end{enumerate}
\qed

\vskip 0.5cm
\begin{theorem}[The two Casimir elements of \,$\mathcal{U}(\so(1,3)\ltimes\Re^{1,3})$]
\mbox{}
\vskip 0.1cm
\noindent
Define the following two elements in the universal enveloping algebra
\,$\mathcal{U}\!\left(\,\overset{{\color{white}.}}{\so(1,3)\ltimes\Re^{1,3}}\,\right)$\,
of the Lie algebra
\,$\so(1,3)\ltimes\Re^{1,3}$\,
of the Lorentz group
\,$\SOup(1,3)\ltimes\Re^{1,3}$:\,
\begin{equation*}
\begin{array}{ccccccc}
C_{1}
& := &
	P^{\mu}P_{\mu}
& = &
	\eta^{\mu\nu}\,P_{\mu}\,P_{\nu}
& = &
	P_{0}^{2} - P_{1}^{2} - P_{2}^{2} - P_{3}^{2}
\\
C_{2}
& \overset{{\color{white}\textnormal{\normalsize$1$}}}{:=} &
	W^{\mu}W_{\mu}
& = &
	\eta^{\mu\nu}\,W_{\mu}\,W_{\nu}
& = &
	W_{0}^{2} - W_{1}^{2} - W_{2}^{2} - W_{3}^{2}
\end{array},
\end{equation*}
where
\begin{equation*}
W^{\mu}
\;\; := \;\;
	\dfrac{1}{2}\,\varepsilon^{\mu\nu\rho\sigma}\,M_{\nu\rho}\,P_{\sigma}
\end{equation*}
Then,
\,$C_{1}\,, C_{2} \in \mathcal{U}\!\left(\,\overset{{\color{white}.}}{\so(1,3)\ltimes\Re^{1,3}}\,\right)$\,
are in the centre of
\,$\mathcal{U}\!\left(\,\overset{{\color{white}.}}{\so(1,3)\ltimes\Re^{1,3}}\,\right)$.\,
\end{theorem}
\proof
It suffices to show that each of \,$C_{1}$\, and \,$C_{2}$\, commutes with every generator of
\,$\so(1,3)\ltimes\Re^{1,3}$.\,
\qed

          %%%%% ~~~~~~~~~~~~~~~~~~~~ %%%%%

\vskip 0.5cm
\section{The Lie algebra of \,$\SL(2,\C) \ltimes \Re^{1,3}$ -- OBSOLETE}

\begin{equation*}
r_{1} = \left[{\color{gray}
	\begin{array}{rrrrr}
	0 & {\color{white}-}0 & {\color{white}-}0 & {\color{white}-}0 & {\color{white}-}0 \\
	0 & 0 & 0 & 0 & 0\\
	0 & 0 & 0 & {\color{red}-1} & 0 \\
	0 & 0 & {\color{red}1} & 0 & 0 \\
	0 & 0 & 0 & 0 & 0 \\
	\end{array}
	}\right],
\;\;
r_{2} = \left[{\color{gray}
	\begin{array}{rrrrr}
	0 & {\color{white}-}0 & {\color{white}-}0 & {\color{white}-}0 & {\color{white}-}0 \\
	0 & 0 & 0 & {\color{red}1} & 0 \\
	0 & 0 & 0 & 0 & 0 \\
	0 & {\color{red}-1} & 0 & 0 & 0 \\
	0 & 0 & 0 & 0 & 0 \\
	\end{array}
	}\right],
\;\;
r_{3} = \left[{\color{gray}
	\begin{array}{rrrrr}
	0 & {\color{white}-}0 & {\color{white}-}0 & {\color{white}-}0 & {\color{white}-}0 \\
	0 & 0 & {\color{red}-1} & 0 & 0 \\
	0 & {\color{red}1} & 0 & 0 & 0 \\
	0 & 0 & 0 & 0 & 0 \\
	0 & 0 & 0 & 0 & 0 \\
	\end{array}
	}\right]
\end{equation*}
\vskip 0.3cm
\begin{equation*}
b_{1} = \left[{\color{gray}
	\begin{array}{rrrrr}
	0 & {\color{red}1} & 0 & 0 & 0\\
	{\color{red}1} & 0 & 0 & 0 & 0\\
	0 & 0 & 0 & 0 & 0 \\
	0 & 0 & 0 & 0 & 0 \\
	0 & 0 & 0 & 0 & 0 \\
	\end{array}
	}\right],
\;\;
b_{2} = \left[{\color{gray}
	\begin{array}{rrrrr}
	0 & 0 & {\color{red}1} & 0 & 0\\
	0 & 0 & 0 & 0 & 0\\
	{\color{red}1} & 0 & 0 & 0 & 0 \\
	0 & 0 & 0 & 0 & 0 \\
	0 & 0 & 0 & 0 & 0 \\
	\end{array}
	}\right],
\;\;
b_{3} = \left[{\color{gray}
	\begin{array}{rrrrr}
	0 & 0 & 0 & {\color{red}1} & 0\\
	0 & 0 & 0 & 0 & 0\\
	0 & 0 & 0 & 0 & 0 \\
	{\color{red}1} & 0 & 0 & 0 & 0 \\
	0 & 0 & 0 & 0 & 0 \\
	\end{array}
	}\right]
\end{equation*}
\vskip 0.3cm
\begin{equation*}
t_{0} = \left[{\color{gray}
	\begin{array}{rrrrr}
	0 & 0 & 0 & 0 & {\color{red}1} \\
	0 & 0 & 0 & 0 & 0 \\
	0 & 0 & 0 & 0 & 0 \\
	0 & 0 & 0 & 0 & 0 \\
	0 & 0 & 0 & 0 & 0 \\
	\end{array}
	}\right],
\;\;
t_{1} = \left[{\color{gray}
	\begin{array}{rrrrr}
	0 & 0 & 0 & 0 & 0 \\
	0 & 0 & 0 & 0 & {\color{red}1} \\
	0 & 0 & 0 & 0 & 0 \\
	0 & 0 & 0 & 0 & 0 \\
	0 & 0 & 0 & 0 & 0 \\
	\end{array}
	}\right],
\;\;
t_{2} = \left[{\color{gray}
	\begin{array}{rrrrr}
	0 & 0 & 0 & 0 & 0 \\
	0 & 0 & 0 & 0 & 0 \\
	0 & 0 & 0 & 0 & {\color{red}1} \\
	0 & 0 & 0 & 0 & 0 \\
	0 & 0 & 0 & 0 & 0 \\
	\end{array}
	}\right],
\;\;
t_{3} = \left[{\color{gray}
	\begin{array}{rrrrr}
	0 & 0 & 0 & 0 & 0 \\
	0 & 0 & 0 & 0 & 0 \\
	0 & 0 & 0 & 0 & 0 \\
	0 & 0 & 0 & 0 & {\color{red}1} \\
	0 & 0 & 0 & 0 & 0 \\
	\end{array}
	}\right]
\end{equation*}
\vskip 0.3cm
\noindent
\textbf{Commutation relations:}
\begin{enumerate}
\item
	$\left[\,r_{m}\,,r_{n}\,\right] \,=\, \varepsilon_{mnk}\,r_{k}$,\, for each \,$m,n = 1,2,3$
\item
	\vskip 0.3cm
	$\left[\,b_{m}\,,b_{n}\,\right] \,=\, -\,\varepsilon_{mnk}\,r_{k}$,\, for each \,$m,n = 1,2,3$
\item
	\vskip -0.23cm
	$\left[\,r_{m}\,,b_{n}\,\right] \,=\, +\,\varepsilon_{mnk}\,b_{k}$,\, for each \,$m,n = 1,2,3$
\item
	\vskip 0.3cm
	$\left[\,t_{\mu}\,,t_{\nu}\,\right] \,=\, 0$,\, for each \,$\mu, \nu = 0,1,2,3$
\item
	\vskip 0.3cm
	$\left[\,r_{m}\,,t_{0}\,\right] \,=\, 0$,\, for each \,$m = 1,2,3$
\item
	\vskip -0.23cm
	$\left[\,r_{m}\,,t_{n}\,\right] \,=\, \varepsilon_{mnk}\,t_{k}$,\, for each \,$m,n = 1,2,3$
\item
	\vskip 0.3cm
	$\left[\,b_{m}\,,t_{0}\,\right] \,=\, t_{m}$,\, for each \,$m = 1,2,3$
\item
	\vskip -0.23cm
	$\left[\,b_{m}\,,t_{n}\,\right] \,=\, \delta_{mn}\,t_{0}$,\, for each \,$m,n = 1,2,3$
\end{enumerate}
\vskip 0.5cm
\textbf{Remarks}
\begin{itemize}
\item
	(i) says that \,$r_{1}, r_{2}, r_{3}$\, generate a copy of $\so(3)$.
\item
	(i), (ii), (iii) together say that \,$r_{1}, r_{2}, r_{3}, b_{1}, b_{2}, b_{3}$\, generate a copy of $\so(1,3)$.
\end{itemize}

\vskip 0.5cm
\begin{lemma}
\mbox{}
\vskip 0.1cm
\noindent
$t^{2} \,:=\, -t_{0}^{2}+t_{1}^{2}+t_{2}^{2}+t_{3}^{2}$\,
is an element of the centre of the universal enveloping algebra of
\,$\so(1,3) \ltimes \Re^{1,3}$.
\end{lemma}
\proof
We need to show that \,$t^{2}$\, commutes with each generator of
\,$\so(1,3) \ltimes \Re^{1,3}$.\,
First, note that
\begin{equation*}
[\,t_{\mu}\,,t_{\nu}\,] \,=\, 0
\quad
\Longrightarrow
\quad
[\,t_{\mu}\,,t^{2}\,] \,=\, 0
\end{equation*}
So, it remains only to show that
\,$[\,r_{m}\,,t^{2}\,] \,=\, 0$,\,
and
\,$[\,b_{m}\,,t^{2}\,] \,=\, 0$,\,
for each \,$m = 1,2,3$.
To this end, observe:
\begin{eqnarray*}
\left[\,r_{m}\,,t^{2}\,\right]
& = &
	r_{m}\left(\,-t_{0}^{2}+t_{1}^{2}+t_{2}^{2}+t_{3}^{2}\,\right)
	\,-\,
	\left(\,-t_{0}^{2}+t_{1}^{2}+t_{2}^{2}+t_{3}^{2}\,\right)r_{m}
\\
& = &
	-\,r_{m}t_{0}^{2}\,+\,r_{m}t_{1}^{2}\,+\,r_{m}t_{2}^{2}\,+\,r_{m}t_{3}^{2}
	\,+\,t_{0}^{2}r_{m}\,-\,t_{1}^{2}r_{m}\,-\,t_{2}^{2}r_{m}\,-\,t_{3}^{2}r_{m}
\\
& = &
	\left(\,-\,r_{m}t_{0}^{2}\,+\,t_{0}^{2}r_{m}\,\right)
	\;+\;
	\overset{3}{\underset{n=1}{\sum}}\left(\,r_{m}t_{n}^{2}\,-\,t_{n}^{2}r_{m}\,\right)
\;\; = \;\;
	\left(\,\overset{{\color{white}.}}{0}\,\right)
	\;+\;
	\overset{3}{\underset{n=1}{\sum}}\left(\,r_{m}t_{n}^{2}\,-\,t_{n}^{2}r_{m}\,\right),
	\;\;
	\textnormal{since \,$[\,r_{m}\,,t_{0}\,] \,=\, 0$}
\\
& = &
	\overset{3}{\underset{n=1}{\sum}}\left(\,
		(t_{n}r_{m}+\varepsilon_{mnk}\,t_{k})\,t_{n}
		\,\overset{{\color{white}1}}{-}\,
		t_{n}^{2}r_{m}
		\,\right),
	\;\;
	\textnormal{since \,$[\,r_{m}\,,t_{n}\,] \,=\, \varepsilon_{mnk}\,t_{k}$}
\\
& = &
	\overset{3}{\underset{n=1}{\sum}}\left(\,
		t_{n}r_{m}t_{n}+\varepsilon_{mnk}\,t_{k}t_{n}
		\,\overset{{\color{white}1}}{-}\,
		t_{n}^{2}r_{m}
		\,\right)
\;\; = \;\;
	\overset{3}{\underset{n=1}{\sum}}\left(\,
		t_{n}r_{m}t_{n}
		\,\overset{{\color{white}1}}{-}\,
		t_{n}^{2}r_{m}
		\,\right)
\\
& = &
	\overset{3}{\underset{n=1}{\sum}}\left(\,
		t_{n}\,(t_{n}r_{m}\,+\,\varepsilon_{mnk}\,t_{k})
		\,\overset{{\color{white}1}}{-}\,
		t_{n}^{2}r_{m}
		\,\right),
	\;\;
	\textnormal{since \,$[\,r_{m}\,,t_{n}\,] \,=\, \varepsilon_{mnk}\,t_{k}$}
\\
& = &
	\overset{3}{\underset{n=1}{\sum}}\left(\,
		t_{n}^{2}r_{m}\,+\,\varepsilon_{mnk}\,t_{n}t_{k}
		\,\overset{{\color{white}1}}{-}\,
		t_{n}^{2}r_{m}
		\,\right)
\;\; = \;\;
	\overset{3}{\underset{n=1}{\sum}}\left(\,
		t_{n}^{2}r_{m}
		\,\overset{{\color{white}1}}{-}\,
		t_{n}^{2}r_{m}
		\,\right)
\\
& = &
	\overset{{\color{white}1}}{0}
\end{eqnarray*}
\vskip 0.3cm
\begin{eqnarray*}
\left[\,b_{m}\,,t^{2}\,\right]
& = &
	b_{m}\left(\,-t_{0}^{2}+t_{1}^{2}+t_{2}^{2}+t_{3}^{2}\,\right)
	\,-\,
	\left(\,-t_{0}^{2}+t_{1}^{2}+t_{2}^{2}+t_{3}^{2}\,\right)b_{m}
\;\; = \;\;
	\left(\,-\,b_{m}t_{0}^{2}\,+\,t_{0}^{2}\,b_{m}\,\right)
	\;+\;
	\overset{3}{\underset{n=1}{\sum}}\left(\,b_{m}t_{n}^{2}\,-\,t_{n}^{2}b_{m}\,\right)
\\
& = &
	\left(
		\,-\,(t_{0}b_{m}+t_{m})t_{0}
		\,\overset{{\color{white}1}}{+}\,
		t_{0}^{2}\,b_{m}
		\,\right)
	\;+\;
	\overset{3}{\underset{n=1}{\sum}}\left(\,
		(t_{n}b_{m}+\delta_{mn}t_{0})\,t_{n}
		\,\overset{{\color{white}1}}{-}\,
		t_{n}^{2}b_{m}
		\,\right)
\\
& = &
	\left(
		\,-\,t_{0}b_{m}t_{0}\,-\,t_{m}t_{0}
		\,\overset{{\color{white}1}}{+}\,
		t_{0}^{2}\,b_{m}
		\,\right)
	\;+\;
	\overset{3}{\underset{n=1}{\sum}}\left(\,
		t_{n}b_{m}t_{n} \,+\, \delta_{mn}t_{0}t_{n}
		\,\overset{{\color{white}1}}{-}\,
		t_{n}^{2}b_{m}
		\,\right)
\\
& = &
	\left(
		\,-\,t_{0}(t_{0}b_{m}+t_{m})
		\,-\,t_{m}t_{0}
		\,\overset{{\color{white}1}}{+}\,
		t_{0}^{2}\,b_{m}
		\,\right)
	\;+\;
	\overset{3}{\underset{n=1}{\sum}}\left(\,
		t_{n}b_{m}t_{n}
		\,\overset{{\color{white}1}}{-}\,
		t_{n}^{2}b_{m}
		\,\right)
	\;+\;
	t_{0}\cdot\!\left(\;
		\overset{3}{\underset{n=1}{\sum}}\,\delta_{mn}t_{n}
		\,\right)
\\
& = &
	\left(
		\,-\,t_{0}^{2}\,b_{m}\,-\,t_{0}t_{m}
		\,-\,t_{m}t_{0}
		\,\overset{{\color{white}1}}{+}\,
		t_{0}^{2}\,b_{m}
		\,\right)
	\;+\;
	\overset{3}{\underset{n=1}{\sum}}\left(\,
		t_{n}(t_{n}b_{m}+\delta_{mn}t_{0})
		\,\overset{{\color{white}1}}{-}\,
		t_{n}^{2}b_{m}
		\,\right)
	\;+\;
	t_{0}\,t_{m}
\\
& = &
	\left(\,
		\overset{{\color{white}1}}{-}\,2\,t_{0}t_{m}
		\,\right)
	\;+\;
	\overset{3}{\underset{n=1}{\sum}}\left(\,
		t_{n}^{2}b_{m} \,+\, \delta_{mn}t_{n}t_{0}
		\,\overset{{\color{white}1}}{-}\,
		t_{n}^{2}b_{m}
		\,\right)
	\;+\;
	t_{0}\,t_{m}
\\
& = &
	\left(\,
		\overset{{\color{white}1}}{-}\,2\,t_{0}t_{m}
		\,\right)
	\;+\;
	\overset{3}{\underset{n=1}{\sum}}\left(\,
		t_{n}^{2}b_{m}
		\,\overset{{\color{white}1}}{-}\,
		t_{n}^{2}b_{m}
		\,\right)
	\;+\;
	\left(\,\overset{3}{\underset{n=1}{\sum}}\,\delta_{mn}t_{n}\,\right)\!\cdot t_{0}
	\;+\;
	t_{0}\,t_{m}
\\
& = &
	\left(\,
		\overset{{\color{white}1}}{-}\,2\,t_{0}\,t_{m}
		\,\right)
	\;+\;
	t_{m}\,t_{0}
	\;+\;
	t_{0}\,t_{m}
\\
& = &
	\overset{{\color{white}1}}{0}
\end{eqnarray*}
This completes the proof of the Lemma.
\qed

\vskip 0.5cm
\begin{definition}[Pauli–Lubanski operator-vector]
\mbox{}
\vskip 0.1cm
\noindent
The \textbf{Pauli–Lubanski operator-vector}
\,$W \,=\, \left(\,W_{\mu}\,\right) \,=\, \left(\,W_{0},W_{1},W_{2},W_{3}\,\right) $\,
is a ``vector'' of operators defined ``component-wise'' as follows:
\begin{eqnarray*}
W_{0} & := & -\, t_{1}\,r_{1} \,-\, t_{2}\,r_{2} \,-\, t_{3}\,r_{3}
\\
W_{1} & := & -\, t_{0}\,r_{1} \,+\, t_{2}\,b_{3} \,-\, t_{3}\,b_{2}
\\
W_{2} & := & -\, t_{0}\,r_{2} \;\,{\color{red}-}\;\, t_{1}\,b_{3} \;{\color{red}+}\; t_{3}\,b_{1}
\\
W_{3} & := & -\, t_{0}\,r_{3} \,+\, t_{1}\,b_{2} \,-\, t_{2}\,b_{1}
\end{eqnarray*}
Equivalently (and more compactly),
\begin{equation*}
W_{0} \; := \; -\, \overset{3}{\underset{n=1}{\sum}}\;t_{n}\,r_{n}\,,
\quad\quad\textnormal{and}\quad\quad\quad
W_{m}
\; := \;
	-\, t_{0}\,r_{m}
	\;+\,
	\overset{3}{\underset{n=1}{\sum}}\;
	\overset{3}{\underset{k=1}{\sum}}\;
	\varepsilon_{mnk}\,t_{n}\,b_{k}
\end{equation*}
The \textbf{second Casimir operator} of the Poincaré algebra is defined as follows:
\begin{equation*}
W^{2} \;\; := \;\; -\,W_{0}^{2} \,+\, W_{1}^{2} \,+\, W_{2}^{2} \,+\, W_{3}^{2}
\end{equation*}
\end{definition}

\vskip 0.5cm
\begin{lemma}
\mbox{}
\vskip 0.1cm
\noindent
\begin{enumerate}
\item
	$\left[\,t_{\mu}\,,\,W_{\nu}\,\right] \,=\, 0$,\, for each \,$\mu,\nu \in \{\,0,1,2,3\,\}$.
\item
	$\left[\,r_{m}\,,W_{0}\,\right] \,=\, 0$,\, for each \,$m = 1,2,3$
\item
	$\left[\,r_{m}\,,W_{n}\,\right] \,=\, \varepsilon_{mnk}\,W_{k}$,\, for each \,$m,n,k = 1,2,3$
\item
	$\left[\,b_{m}\,,W_{0}\,\right] \,=\, W_{m}$,\, for each \,$m = 1,2,3$
\item
	$\left[\,b_{m}\,,W_{n}\,\right] \,=\, \delta_{mn}\,W_{0}$,\, for each \,$m,n = 1,2,3$
\end{enumerate}
\end{lemma}
\proof
\begin{enumerate}
\item
\item
	\begin{eqnarray*}
	\left[\,r_{m}\,,W_{0}\,\right]
	& = &
		r_{m}\,W_{0} \,-\, W_{0}\,r_{m}
	\;\; = \;\;
		r_{m}\,(\,-\, t_{1}\,r_{1} \,-\, t_{2}\,r_{2} \,-\, t_{3}\,r_{3}\,)
		\,-\,
		(\,-\, t_{1}\,r_{1} \,-\, t_{2}\,r_{2} \,-\, t_{3}\,r_{3}\,)\,r_{m}
	\\
	& = &
		\overset{3}{\underset{n=1}{\sum}}
		\left(\,
			t_{n}r_{n}r_{m} \overset{{\color{white}.}}{-} r_{m}t_{n}r_{n}
			\,\right)
	\;\; = \;\;
		\overset{3}{\underset{n=1}{\sum}}
		\left(\,
			t_{n}(r_{m}r_{n}+\varepsilon_{nmk}r_{k})
			\overset{{\color{white}.}}{-}
			r_{m}t_{n}r_{n}
			\,\right)
	\\
	& = &
		\overset{3}{\underset{n=1}{\sum}}
		\left(\,
			t_{n}r_{m}r_{n}
			\overset{{\color{white}.}}{-}
			r_{m}t_{n}r_{n}
			\,\right)
		\;{\color{red}-}\;\,
		\overset{3}{\underset{n=1}{\sum}}\;
		\overset{3}{\underset{k=1}{\sum}}\;
		\varepsilon_{{\color{red}mn}k}t_{n}r_{k}
	\\
	& = &
		\overset{3}{\underset{n=1}{\sum}}
		\left(\,
			(r_{m}t_{n} - \varepsilon_{mnk}t_{k})\,r_{n}
			\overset{{\color{white}.}}{-}
			r_{m}t_{n}r_{n}
			\,\right)
		\,-\,
		\overset{3}{\underset{n=1}{\sum}}\;
		\overset{3}{\underset{k=1}{\sum}}\;
		\varepsilon_{mnk}t_{n}r_{k}
	\\
	& = &
		\overset{3}{\underset{n=1}{\sum}}
		\left(\,
			r_{m}t_{n}r_{n} - \varepsilon_{mnk}t_{k}r_{n}
			\overset{{\color{white}.}}{-}
			r_{m}t_{n}r_{n}
			\,\right)
		\,-\,
		\overset{3}{\underset{n=1}{\sum}}\;
		\overset{3}{\underset{k=1}{\sum}}\;
		\varepsilon_{mnk}t_{n}r_{k}
	\\
	& = &
		-\,
		\overset{3}{\underset{n=1}{\sum}}\;
		\overset{3}{\underset{k=1}{\sum}}\;
		\varepsilon_{mnk}t_{k}r_{n}
		\,-\,
		\overset{3}{\underset{n=1}{\sum}}\;
		\overset{3}{\underset{k=1}{\sum}}\;
		\varepsilon_{mnk}t_{n}r_{k}
	\;\; = \;\;
		-\,
		\overset{3}{\underset{n=1}{\sum}}\;
		\overset{3}{\underset{k=1}{\sum}}\;
		\varepsilon_{mkn}t_{n}r_{k}
		\,-\,
		\overset{3}{\underset{n=1}{\sum}}\;
		\overset{3}{\underset{k=1}{\sum}}\;
		\varepsilon_{mnk}t_{n}r_{k}
	\\
	& = &
		-\;
		\overset{3}{\underset{n=1}{\sum}}\;
		\overset{3}{\underset{k=1}{\sum}}\;
		\left(\,
			\varepsilon_{mkn}
			\,\overset{{\color{white}.}}{+}\,
			\varepsilon_{mnk}
			\,\right)
		t_{n}r_{k}
	\;\; = \;\;
		-\;
		\overset{3}{\underset{n=1}{\sum}}\;
		\overset{3}{\underset{k=1}{\sum}}\;
		\left(\,
			{\color{red}-}\,\varepsilon_{m{\color{red}nk}}
			\,\overset{{\color{white}.}}{+}\,
			\varepsilon_{mnk}
			\,\right)
		t_{n}r_{k}
	\\
	& = &
		\overset{{\color{white}1}}{0}
	\end{eqnarray*}
\item
	We proceed by establishing a series of Claims.

	\vskip 0.3cm
	\noindent
	\textbf{Claim 1:}\quad
	$\varepsilon_{mnk}\,W_{k}
	\,=\,
		-\,\varepsilon_{mnk}\,t_{0}\,r_{k}
		\;+\;\,
		\overset{3}{\underset{p=1}{\sum}}\;
		\overset{3}{\underset{q=1}{\sum}}
		\left(\;
			\overset{3}{\underset{k=1}{\sum}}\;
			\varepsilon_{mnk}\,\varepsilon_{pqk}
			\,\right)
		t_{p}\,b_{q}
	$
	\vskip 0.1cm
	\noindent
	Proof of Claim 1:
	\begin{eqnarray*}
	\varepsilon_{mnk}\,W_{k}
	& = &
		\varepsilon_{mnk}\left(\,
			-\,t_{0}\,r_{k}
			\;+\,
			\overset{3}{\underset{p=1}{\sum}}\;
			\overset{3}{\underset{q=1}{\sum}}\;
			\varepsilon_{kpq}\,t_{p}\,b_{q}
			\,\right)
	\\
	& = &
		-\,\varepsilon_{mnk}\,t_{0}\,r_{k}
		\;+\;\,
		\overset{3}{\underset{p=1}{\sum}}\;
		\overset{3}{\underset{q=1}{\sum}}
		\left(\;
			\overset{3}{\underset{k=1}{\sum}}\;
			\varepsilon_{mnk}\,\varepsilon_{pqk}
			\,\right)
		t_{p}\,b_{q}
	\end{eqnarray*}
	This completes the proof of Claim 1.

	\vskip 0.3cm
	\textbf{Claim 2:}\quad
	$\left[\,r_{m}\,,W_{n}\,\right]
	\; = \;
		-\,\varepsilon_{mnk}\,t_{0}\,r_{k}
		\,\;+\;\;
		\overset{3}{\underset{p=1}{\sum}}\;
		\overset{3}{\underset{q=1}{\sum}}\,
		\left(\;
			\overset{3}{\underset{k=1}{\sum}}\,(\,
				-\,\varepsilon_{mqk}\,\varepsilon_{npk}
				\,+\,
				\varepsilon_{mpk}\,\varepsilon_{nqk}
				\,)
			\,\right)
		t_{p}\,b_{q}
	$
	\vskip 0.1cm
	\noindent
	Proof of Claim 2:
	\begin{eqnarray*}
	\left[\,r_{m}\,,W_{n}\,\right]
	& = &
		r_{m}\,W_{n} \,-\, W_{n}\,r_{m}
	\\
	& = &
		r_{m}\left(\,
			-\,t_{0}\,r_{n}
			\;+\,
			\overset{3}{\underset{k=1}{\sum}}\;
			\overset{3}{\underset{l=1}{\sum}}\;
			\varepsilon_{nkl}\,t_{k}\,b_{l}
			\,\right)
		\,-\,
		\left(\,
			-\,t_{0}\,r_{n}
			\;+\,
			\overset{3}{\underset{k=1}{\sum}}\;
			\overset{3}{\underset{l=1}{\sum}}\;
			\varepsilon_{nkl}\,t_{k}\,b_{l}
			\,\right)
		r_{m}
	\\
	& = &
		\left(\,
			-\,r_{m}\,t_{0}\,r_{n}
			\,\overset{{\color{white}.}}{+}\,
			t_{0}\,r_{n}\,r_{m}
			\,\right)
		\;+\;
		r_{m}\left(\;
			\overset{3}{\underset{k=1}{\sum}}\;
			\overset{3}{\underset{l=1}{\sum}}\;
			\varepsilon_{nkl}\,t_{k}\,b_{l}
			\,\right)
		\,-\,
		\left(\;
			\overset{3}{\underset{k=1}{\sum}}\;
			\overset{3}{\underset{l=1}{\sum}}\;
			\varepsilon_{nkl}\,t_{k}\,b_{l}
			\,\right)
		r_{m}
	\\
	& = &
		\left(\,
			-\,{\color{red}t_{0}}\,{\color{red}r_{m}}\,r_{n}
			\,\overset{{\color{white}.}}{+}\,
			t_{0}\,r_{n}\,r_{m}
			\,\right)
		\,\;+\;\;
		\overset{3}{\underset{k=1}{\sum}}\;
		\overset{3}{\underset{l=1}{\sum}}\;\,
		\varepsilon_{nkl}\left(\,
			r_{m}\,\,t_{k}\,b_{l}
			\,\overset{{\color{white}.}}{-}\,
			t_{k}\,b_{l}\,r_{m}
			\,\right)
		% \;+\;
		% r_{m}\left(\;
		% 	\overset{3}{\underset{k=1}{\sum}}\;
		% 	\overset{3}{\underset{l=1}{\sum}}\;
		% 	\varepsilon_{nkl}\,t_{k}\,b_{l}
		% 	\,\right)
		% \,-\,
		% \left(\;
		% 	\overset{3}{\underset{k=1}{\sum}}\;
		% 	\overset{3}{\underset{l=1}{\sum}}\;
		% 	\varepsilon_{nkl}\,t_{k}\,b_{l}
		% 	\,\right)
		% r_{m}
	\\
	& = &
		-\,t_{0}\,\left[\,r_{m}\,\overset{{\color{white}1}}{,}\,r_{n}\,\right]
		\,\;+\;\;
		\overset{3}{\underset{k=1}{\sum}}\;
		\overset{3}{\underset{l=1}{\sum}}\;\,
		\varepsilon_{nkl}\left(\,
			(t_{k}\,r_{m} + \varepsilon_{mkp}\,t_{p})\,b_{l}
			\,\overset{{\color{white}.}}{-}\,
			t_{k}\,b_{l}\,r_{m}
			\,\right)
	\\
	& = &
		-\,\varepsilon_{mnk}\,t_{0}\,r_{k}
		\,\;+\;\;
		\overset{3}{\underset{k=1}{\sum}}\;
		\overset{3}{\underset{l=1}{\sum}}\;\,
		\varepsilon_{nkl}\left(\,
			t_{k}\,r_{m}\,b_{l} \,+\, \varepsilon_{mkp}\,t_{p}\,b_{l}
			\,\overset{{\color{white}.}}{-}\,
			t_{k}\,b_{l}\,r_{m}
			\,\right)
	\\
	& = &
		-\,\varepsilon_{mnk}\,t_{0}\,r_{k}
		\,\;+\;\;
		\overset{3}{\underset{k=1}{\sum}}\;
		\overset{3}{\underset{l=1}{\sum}}\;\,
		\varepsilon_{nkl}\,t_{k}\left[\,r_{m}\,\overset{{\color{white}1}}{,}\,b_{l}\,\right]
		\,\;+\;\;
		\overset{3}{\underset{k=1}{\sum}}\;
		\overset{3}{\underset{l=1}{\sum}}\;
		\overset{3}{\underset{p=1}{\sum}}\;\,
		\varepsilon_{nkl}\,\varepsilon_{mkp}\,t_{p}\,b_{l}
	\\
	& = &
		-\,\varepsilon_{mnk}\,t_{0}\,r_{k}
		\,\;+\;\;
		\overset{3}{\underset{k=1}{\sum}}\;
		\overset{3}{\underset{l=1}{\sum}}\;\,
		\varepsilon_{nkl}\,t_{k}\,
		{\color{red}\left(\;\overset{3}{\underset{q=1}{\sum}}\;\varepsilon_{mlq}\,b_{q}\right)}
		\,\;+\;\;
		\overset{3}{\underset{k=1}{\sum}}\;
		\overset{3}{\underset{l=1}{\sum}}\;
		\overset{3}{\underset{p=1}{\sum}}\;\,
		\varepsilon_{nkl}\,\varepsilon_{mkp}\,t_{p}\,b_{l}
	\\
	& = &
		-\,\varepsilon_{mnk}\,t_{0}\,r_{k}
		\,\;+\;\;
		\overset{3}{\underset{k=1}{\sum}}\;
		\overset{3}{\underset{l=1}{\sum}}\;
		\overset{3}{\underset{q=1}{\sum}}\;\,
		\varepsilon_{mlq}\,\varepsilon_{nkl}\,t_{k}\,b_{q}
		\,\;+\;\;
		\overset{3}{\underset{k=1}{\sum}}\;
		\overset{3}{\underset{l=1}{\sum}}\;
		\overset{3}{\underset{p=1}{\sum}}\;\,
		\varepsilon_{nkl}\,\varepsilon_{mkp}\,t_{p}\,b_{l}
	\\
	& = &
		-\,\varepsilon_{mnk}\,t_{0}\,r_{k}
		\,\;+\;\;
		\overset{3}{\underset{{\color{red}p}=1}{\sum}}\;
		\overset{3}{\underset{q=1}{\sum}}\;
		\overset{3}{\underset{l=1}{\sum}}\;\,
		\varepsilon_{mlq}\,\varepsilon_{n{\color{red}p}l}\,t_{{\color{red}p}}\,b_{q}
		\,\;+\;\;
		\overset{3}{\underset{p=1}{\sum}}\;
		\overset{3}{\underset{{\color{red}q}=1}{\sum}}\;
		\overset{3}{\underset{k=1}{\sum}}\;\,
		\varepsilon_{mkp}\,\varepsilon_{nk{\color{red}q}}\,t_{p}\,b_{{\color{red}q}}
	\\
	& = &
		-\,\varepsilon_{mnk}\,t_{0}\,r_{k}
		\,\;+\;\;
		\overset{3}{\underset{p=1}{\sum}}\;
		\overset{3}{\underset{q=1}{\sum}}\;
		\left(\,
			\overset{3}{\underset{{\color{red}k}=1}{\sum}}\;\,
			\varepsilon_{m{\color{red}k}q}\,\varepsilon_{np{\color{red}k}}
			\,+\,
			\overset{3}{\underset{k=1}{\sum}}\;\,
			\varepsilon_{mkp}\,\varepsilon_{nkq}
			\,\right)
		t_{p}\,b_{q}
	\\
	& = &
		-\,\varepsilon_{mnk}\,t_{0}\,r_{k}
		\,\;+\;\;
		\overset{3}{\underset{p=1}{\sum}}\;
		\overset{3}{\underset{q=1}{\sum}}\,
		\left(\;
			\overset{3}{\underset{k=1}{\sum}}\,(\,
				\varepsilon_{mkq}\,\varepsilon_{npk}
				\,+\,
				\varepsilon_{mkp}\,\varepsilon_{nkq}
				\,)
			\,\right)
		t_{p}\,b_{q}
	\\
	& = &
		-\,\varepsilon_{mnk}\,t_{0}\,r_{k}
		\,\;+\;\;
		\overset{3}{\underset{p=1}{\sum}}\;
		\overset{3}{\underset{q=1}{\sum}}\,
		\left(\;
			\overset{3}{\underset{k=1}{\sum}}\,(\,
				-\,\varepsilon_{mqk}\,\varepsilon_{npk}
				\,+\,
				\varepsilon_{mpk}\,\varepsilon_{nqk}
				\,)
			\,\right)
		t_{p}\,b_{q}
	\end{eqnarray*}
	This completes the proof of Claim 2.

	\vskip 0.3cm
	\textbf{Claim 3:}\quad
	\begin{eqnarray*}
	\overset{3}{\underset{k=1}{\sum}}\left(\,
		\varepsilon_{mpk}\,\varepsilon_{nqk}
		\,\overset{{\color{white}.}}{-}\,
		\varepsilon_{mqk}\,\varepsilon_{npk}
		\,\right)
	& = &
		\overset{3}{\underset{k=1}{\sum}}\;
		\varepsilon_{mnk}\,\varepsilon_{pqk}
	\end{eqnarray*}
	Proof of Claim 3:\;\;
	First, note that:
	\begin{equation*}
	\textnormal{R.H.S.}(m,n,p,q)
	\;\; = \;\;\,
		\overset{3}{\underset{k=1}{\sum}}\;
		\varepsilon_{mnk}\,\varepsilon_{pqk}
	\;\;\, = \;\;
		\left\{\begin{array}{rl}
			0\,, & \textnormal{if \,$m = n$\, or \,$p = q$}
			\\
			1\,, & \textnormal{if \,$m \neq n$\, and \,$(m,n) = (p,q)$}
			\\
			-1\,, & \textnormal{if \,$m \neq n$\, and \,$(m,n) = (q,p)$}
			\\
			0\,, & \textnormal{otherwise (i.e., \,$m \neq n$, \,$p \neq q$,\, and \,$\{m,n\} \neq \{p,q\}$)}
			\end{array}\right.
	\end{equation*}
	Next, note the following \textbf{five} implications:
	\begin{equation*}
	m\,=\,n
	\quad\Longrightarrow\quad
	\textnormal{L.H.S.}(m,n,p,q)
	\; = \;
		\overset{3}{\underset{k=1}{\sum}}\left(\,
			\varepsilon_{mpk}\,\varepsilon_{nqk}
			\,\overset{{\color{white}.}}{-}\,
			\varepsilon_{mqk}\,\varepsilon_{npk}
			\,\right)
	\; = \;
		\overset{3}{\underset{k=1}{\sum}}\left(\,
			\varepsilon_{mpk}\,\varepsilon_{{\color{red}m}qk}
			\,\overset{{\color{white}.}}{-}\,
			\varepsilon_{mqk}\,\varepsilon_{{\color{red}m}pk}
			\,\right)
	\; = \;
		0
	\end{equation*}
	\begin{equation*}
	p\,=\,q
	\quad\Longrightarrow\quad
	\textnormal{L.H.S.}(m,n,p,q)
	\; = \;
		\overset{3}{\underset{k=1}{\sum}}\left(\,
			\varepsilon_{mpk}\,\varepsilon_{nqk}
			\,\overset{{\color{white}.}}{-}\,
			\varepsilon_{mqk}\,\varepsilon_{npk}
			\,\right)
	\; = \;
		\overset{3}{\underset{k=1}{\sum}}\left(\,
			\varepsilon_{mpk}\,\varepsilon_{n{\color{red}p}k}
			\,\overset{{\color{white}.}}{-}\,
			\varepsilon_{m{\color{red}p}k}\,\varepsilon_{npk}
			\,\right)
	\; = \;
		0
	\end{equation*}
	\begin{equation*}
	\left.\begin{array}{c}
		m \neq n
		\\
		(m,n) = (p,q)
		\end{array}\right\}
	\quad\Longrightarrow\quad
	\textnormal{L.H.S.}(m,n,p,q)
	% \; = \;
	% 	\overset{3}{\underset{k=1}{\sum}}\left(\,
	% 		\varepsilon_{mpk}\,\varepsilon_{nqk}
	% 		\,\overset{{\color{white}.}}{-}\,
	% 		\varepsilon_{mqk}\,\varepsilon_{npk}
	% 		\,\right)
	\; = \;
		\overset{3}{\underset{k=1}{\sum}}\left(\,
			\varepsilon_{m{\color{red}m}k}\,\varepsilon_{n{\color{red}n}k}
			\,\overset{{\color{white}.}}{-}\,
			\varepsilon_{m{\color{red}n}k}\,\varepsilon_{n{\color{red}m}k}
			\,\right)
	\; = \;
		+\,\overset{3}{\underset{k=1}{\sum}}\;\varepsilon_{mnk}^{2}
	\; = \;
		+1
	\end{equation*}
	\begin{equation*}
	\left.\begin{array}{c}
		m \neq n
		\\
		(m,n) = (q,p)
		\end{array}\right\}
	\quad\Longrightarrow\quad
	\textnormal{L.H.S.}(m,n,p,q)
	% \; = \;
	% 	\overset{3}{\underset{k=1}{\sum}}\left(\,
	% 		\varepsilon_{mpk}\,\varepsilon_{nqk}
	% 		\,\overset{{\color{white}.}}{-}\,
	% 		\varepsilon_{mqk}\,\varepsilon_{npk}
	% 		\,\right)
	\; = \;
		\overset{3}{\underset{k=1}{\sum}}\left(\,
			\varepsilon_{m{\color{red}n}k}\,\varepsilon_{n{\color{red}m}k}
			\,\overset{{\color{white}.}}{-}\,
			\varepsilon_{m{\color{red}m}k}\,\varepsilon_{n{\color{red}n}k}
			\,\right)
	\; = \;
		-\,\overset{3}{\underset{k=1}{\sum}}\;\varepsilon_{mnk}^{2}
	\; = \;
		-1
	\end{equation*}
	\begin{eqnarray*}
	\left.\begin{array}{c}
		m \neq n,\;\, p \neq q
		\\
		\{m,n\} \neq \{p,q\}
		\end{array}\right\}
	& \Longrightarrow &
		\left\{\begin{array}{c}
			m \neq n,\;\, p \neq q,\;\,\{m,n\} \neq \{p,q\}
			\\
			{\color{red}
			\{m,n,p,q\} = \{1,2,3\} \,\overset{{\color{white}1}}{\ni}\, k,
			\;\textnormal{for each \,$k = 1,2,3$}
			}
			\end{array}\right.
	\\
	& \Longrightarrow &
	\textnormal{L.H.S.}(m,n,p,q)
	\; = \;
		\overset{3}{\underset{k=1}{\sum}}\left(\,
			\varepsilon_{mpk}\,\varepsilon_{nqk}
			\,\overset{{\color{white}.}}{-}\,
			\varepsilon_{mqk}\,\varepsilon_{npk}
			\,\right)
	\; = \;
		\overset{3}{\underset{k=1}{\sum}}\left(\,
			0
			\,\overset{{\color{white}.}}{-}\,
			0
			\,\right)
	\; = \;
		0
	\end{eqnarray*}
	The preceding \textbf{five} implications together imply:
	\begin{eqnarray*}
	\textnormal{L.H.S.}(m,n,p,q)
	& = &
		\overset{3}{\underset{k=1}{\sum}}\left(\,
			\varepsilon_{mpk}\,\varepsilon_{nqk}
			\,\overset{{\color{white}.}}{-}\,
			\varepsilon_{mqk}\,\varepsilon_{npk}
			\,\right)
	\\
	& = &
		\left\{\begin{array}{rl}
			0\,, & \textnormal{if \,$m = n$\, or \,$p = q$}
			\\
			1\,, & \textnormal{if \,$m \neq n$\, and \,$(m,n) = (p,q)$}
			\\
			-1\,, & \textnormal{if \,$m \neq n$\, and \,$(m,n) = (q,p)$}
			\\
			0\,, & \textnormal{otherwise (i.e., \,$m \neq n$, \,$p \neq q$,\, and \,$\{m,n\} \neq \{p,q\}$)}
			\end{array}\right.
	\\
	& = &
		\overset{{\color{white}.}}{\textnormal{R.H.S.}(m,n,p,q)}
	\end{eqnarray*}
	This completes the proof of Claim 3.

	\vskip 0.3cm
	\noindent
	Lastly, now note that Claim 1, Claim 2 and Claim 3 together imply:
	\begin{eqnarray*}
	\left[\,r_{m}\,,W_{n}\,\right]
	& = &
		-\,\varepsilon_{mnk}\,t_{0}\,r_{k}
		\,\;+\;\;
		\overset{3}{\underset{p=1}{\sum}}\;
		\overset{3}{\underset{q=1}{\sum}}\,
		\left(\;
			\overset{3}{\underset{k=1}{\sum}}\,(\,
				-\,\varepsilon_{mqk}\,\varepsilon_{npk}
				\,+\,
				\varepsilon_{mpk}\,\varepsilon_{nqk}
				\,)
			\,\right)
		t_{p}\,b_{q}\,,
		\quad
		\textnormal{by Claim 2}
	\\
	& = &
		-\,\varepsilon_{mnk}\,t_{0}\,r_{k}
		\;+\;\,
		\overset{3}{\underset{p=1}{\sum}}\;
		\overset{3}{\underset{q=1}{\sum}}
		\left(\;
			\overset{3}{\underset{k=1}{\sum}}\;
			\varepsilon_{mnk}\,\varepsilon_{pqk}
			\,\right)
		t_{p}\,b_{q}\,,
		\quad
		\textnormal{by Claim 3}
	\\
	& = &
		\overset{{\color{white}1}}{\varepsilon_{mnk}\,W_{k}}\,,
		\quad
		\textnormal{by Claim 1},
	\end{eqnarray*}
	as required.
\item
	\begin{eqnarray*}
	\left[\,b_{m}\,,W_{0}\,\right]
	& = &
		b_{m}\,W_{0} \,-\, W_{0}\,b_{m}
	\;\; = \;\;
		b_{m}\,(\,-\, t_{1}\,r_{1} \,-\, t_{2}\,r_{2} \,-\, t_{3}\,r_{3}\,)
		\,-\,
		(\,-\, t_{1}\,r_{1} \,-\, t_{2}\,r_{2} \,-\, t_{3}\,r_{3}\,)\,b_{m}
	\\
	& = &
		\overset{3}{\underset{n=1}{\sum}}
		\left(\,
			t_{n}r_{n}b_{m} \overset{{\color{white}.}}{-} b_{m}t_{n}r_{n}
			\,\right)
	\;\; = \;\;
		\overset{3}{\underset{n=1}{\sum}}
		\left(\,
			t_{n}(b_{m}r_{n}+\varepsilon_{nmk}\,b_{k})
			\overset{{\color{white}.}}{-}
			b_{m}t_{n}r_{n}
			\,\right)
	\\
	& = &
		\overset{3}{\underset{n=1}{\sum}}
		\left(\,
			t_{n}b_{m}r_{n} + \varepsilon_{nmk}\,t_{n}b_{k}
			\overset{{\color{white}.}}{-}
			b_{m}t_{n}r_{n}
			\,\right)
	\;\; = \;\;
		\overset{3}{\underset{n=1}{\sum}}
		\left(\,
			t_{n}b_{m}r_{n}
			\overset{{\color{white}.}}{-}
			b_{m}t_{n}r_{n}
			\,\right)
		\; {\color{red}-} \;\,
		\overset{3}{\underset{n=1}{\sum}}\;
		\overset{3}{\underset{k=1}{\sum}}\;
		\varepsilon_{{\color{red}mn}k}\,t_{n}b_{k}
	\\
	& = &
		\overset{3}{\underset{n=1}{\sum}}
		\left(\,
			(b_{m}t_{n} - \delta_{mn}t_{0})r_{n}
			\overset{{\color{white}.}}{-}
			b_{m}t_{n}r_{n}
			\,\right)
		\; - \;\,
		\overset{3}{\underset{n=1}{\sum}}\;
		\overset{3}{\underset{k=1}{\sum}}\;
		\varepsilon_{mnk}\,t_{n}b_{k}
	\\
	& = &
		\overset{3}{\underset{n=1}{\sum}}
		\left(\,
			b_{m}t_{n}r_{n} - \delta_{mn}t_{0}r_{n}
			\overset{{\color{white}.}}{-}
			b_{m}t_{n}r_{n}
			\,\right)
		\; - \;\,
		\overset{3}{\underset{n=1}{\sum}}\;
		\overset{3}{\underset{k=1}{\sum}}\;
		\varepsilon_{mnk}\,t_{n}b_{k}
	\\
	& = &
		-\,t_{0}\,r_{m}
		\; - \;\,
		\overset{3}{\underset{n=1}{\sum}}\;
		\overset{3}{\underset{k=1}{\sum}}\;
		\varepsilon_{mnk}\,t_{n}b_{k}
	\\
	& {\color{red}\neq} &
		-\,t_{0}\,r_{m}
		\;\; {\color{red}+} \;\;
		\overset{3}{\underset{n=1}{\sum}}\;
		\overset{3}{\underset{k=1}{\sum}}\;
		\varepsilon_{mnk}\,t_{n}b_{k}
	\;\; =: \;\;
		W_{m}
	\end{eqnarray*}
\item
	\begin{eqnarray*}
	\left[\,b_{m}\,,W_{n}\,\right]
	& = &
		b_{m}\,W_{n} \,-\, W_{n}\,b_{m}
	\\
	& = &
		b_{m}\left(\,
			-\,t_{0}\,r_{n}
			\;+\,
			\overset{3}{\underset{k=1}{\sum}}\;
			\overset{3}{\underset{l=1}{\sum}}\;
			\varepsilon_{nkl}\,t_{k}\,b_{l}
			\,\right)
		\,-\,
		\left(\,
			-\,t_{0}\,r_{n}
			\;+\,
			\overset{3}{\underset{k=1}{\sum}}\;
			\overset{3}{\underset{l=1}{\sum}}\;
			\varepsilon_{nkl}\,t_{k}\,b_{l}
			\,\right)
		b_{m}
	\\
	& = &
		\left(\,
			t_{0}r_{n}b_{m}
			\,\overset{{\color{white}1}}{-}\,
			b_{m}t_{0}r_{n}
			\,\right) 
		\,+\;
			\overset{3}{\underset{k=1}{\sum}}\;
			\overset{3}{\underset{l=1}{\sum}}\;
			\varepsilon_{nkl}\,b_{m}\,t_{k}\,b_{l}
		\;-\;
			\overset{3}{\underset{k=1}{\sum}}\;
			\overset{3}{\underset{l=1}{\sum}}\;
			\varepsilon_{nkl}\,t_{k}\,b_{l}\,b_{m}	
	\\
	& = &
		\left(\,
			t_{0}r_{n}b_{m}
			\,\overset{{\color{white}1}}{-}\,
			(t_{0}b_{m}+t_{m})r_{n}
			\,\right) 
		\;+\;\,
			\overset{3}{\underset{k=1}{\sum}}\;
			\overset{3}{\underset{l=1}{\sum}}\;\,
			\varepsilon_{nkl}
			\left(\,
				b_{m}\,t_{k}\,b_{l}
				\, \overset{{\color{white}1}}{-} \,
				t_{k}\,b_{l}\,b_{m}	
				\,\right)
	\\
	& = &
		\left(\,
			t_{0}\,\left[\,r_{n}\,,\,b_{m}\,\right]
			\,\overset{{\color{white}1}}{-}\,
			t_{m}r_{n}
			\,\right) 
		\;+\;\,
			\overset{3}{\underset{k=1}{\sum}}\;
			\overset{3}{\underset{l=1}{\sum}}\;\,
			\varepsilon_{nkl}
			\left(\,
				(t_{k}\,b_{m}+\delta_{mk}t_{0})\,b_{l}
				\, \overset{{\color{white}1}}{-} \,
				t_{k}\,b_{l}\,b_{m}	
				\,\right)
	\\
	& = &
		\left(\,
			t_{0}\cdot\varepsilon_{nmk}\,b_{k}
			\,\overset{{\color{white}1}}{-}\,
			t_{m}r_{n}
			\,\right) 
		\;+\;\,
			\overset{3}{\underset{k=1}{\sum}}\;
			\overset{3}{\underset{l=1}{\sum}}\;\,
			\varepsilon_{nkl}
			\left(\,
				t_{k}\,b_{m}\,b_{l}
				\,\overset{{\color{white}1}}{-}\,
				t_{k}\,b_{l}\,b_{m}	
				\,\overset{{\color{white}1}}{+}\,
				\delta_{mk}\,t_{0}\,b_{l}
				\,\right)
	\\
	& = &
		\left(\,
			\varepsilon_{nmk}\,t_{0}\,b_{k}
			\,\overset{{\color{white}1}}{-}\,
			t_{m}r_{n}
			\,\right) 
		\;+\;\,
			\overset{3}{\underset{k=1}{\sum}}\;
			\overset{3}{\underset{l=1}{\sum}}\;\,
			\varepsilon_{nkl}
			\left(\,
				t_{k}\left[\,b_{m}\,,\,b_{l}\,\right]
				\,\overset{{\color{white}1}}{+}\,
				\delta_{mk}\,t_{0}\,b_{l}
				\,\right)
	\\
	& = &
		\left(\,
			\varepsilon_{nmk}\,t_{0}\,b_{k}
			\,\overset{{\color{white}1}}{-}\,
			t_{m}r_{n}
			\,\right) 
		\;+\;\,
			\overset{3}{\underset{k=1}{\sum}}\;
			\overset{3}{\underset{l=1}{\sum}}\;\,
			\varepsilon_{nkl}
			\left(\,
				%t_{k}\left[\,b_{m}\,,\,b_{l}\,\right]
				-\,\varepsilon_{mlp}\,t_{k}\,r_{p}
				\,\overset{{\color{white}1}}{+}\,
				\delta_{mk}\,t_{0}\,b_{l}
				\,\right)
	\\
	& = &
		\left(\,
			\varepsilon_{nmk}\,t_{0}\,b_{k}
			\,\overset{{\color{white}1}}{-}\,
			t_{m}r_{n}
			\,\right) 
		\;-\;\,
			\overset{3}{\underset{k=1}{\sum}}\;
			\overset{3}{\underset{l=1}{\sum}}\;
			\overset{3}{\underset{p=1}{\sum}}\;\,
			\varepsilon_{nkl}\,\varepsilon_{mlp}\,t_{k}\,r_{p}
		\;+\;\,
			\overset{3}{\underset{k=1}{\sum}}\;
			\overset{3}{\underset{l=1}{\sum}}\;\,
			\varepsilon_{nkl}\,\delta_{mk}\,t_{0}\,b_{l}
	\\
	& = &
		\left(\,
			\varepsilon_{nmk}\,t_{0}\,b_{k}
			\,\overset{{\color{white}1}}{-}\,
			t_{m}r_{n}
			\,\right) 
		\;-\;\,
			\overset{3}{\underset{k=1}{\sum}}\;
			\overset{3}{\underset{l=1}{\sum}}\;
			\overset{3}{\underset{p=1}{\sum}}\;\,
			\varepsilon_{nkl}\,\varepsilon_{mlp}\,t_{k}\,r_{p}
		\;+\;\,
			\overset{3}{\underset{l=1}{\sum}}\;\,
			\varepsilon_{nml}\,t_{0}\,b_{l}
	\end{eqnarray*}
\end{enumerate}
\qed

          %%%%% ~~~~~~~~~~~~~~~~~~~~ %%%%%

\vskip 0.5cm
\noindent
\textbf{References}
\begin{itemize}
\item
	Main reference:
	\vskip 0.1cm
	\cite{Sternberg1994}: Section 3.9 -- Wigner's classification of the irreducible representation of the Poincaré group
\item
	See also:
	\begin{itemize}
	\item
		\cite{Tung1985}: Section 10.4 -- Unitary Irreducible Representations of the Poincaré Group
	\item
		\cite{Jones2020}: Chapter 10 -- Representations of the Poincaré Group
	\end{itemize}
\end{itemize}

          %%%%% ~~~~~~~~~~~~~~~~~~~~ %%%%%
