
          %%%%% ~~~~~~~~~~~~~~~~~~~~ %%%%%

\chapter{Irreducible representations of the Lorentz algebra $\so(1,3)$}
\setcounter{theorem}{0}
\setcounter{equation}{0}

%\cite{vanDerVaart1996}
%\cite{Kosorok2008}

%\renewcommand{\theenumi}{\alph{enumi}}
%\renewcommand{\labelenumi}{\textnormal{(\theenumi)}$\;\;$}
\renewcommand{\theenumi}{\roman{enumi}}
\renewcommand{\labelenumi}{\textnormal{(\theenumi)}$\;\;$}

          %%%%% ~~~~~~~~~~~~~~~~~~~~ %%%%%

\section{Definition \,$\textnormal{O}(1,n)$}

          %%%%% ~~~~~~~~~~~~~~~~~~~~ %%%%%

\noindent
Let \,$Q_{(1,n)} \,:=\, \diag(-1,1,\cdots,1) \in \Re^{(n+1) \times (n+1)}$.

\begin{definition}[$\textnormal{O}(1,n)$]
\mbox{}
\vskip 0.1cm
\noindent
The \textbf{Lorentz} group is defined to be:
\begin{equation*}
\textnormal{O}(1,n)
\; := \;
	\left\{\;\,
		A \overset{{\color{white}.}}{\in} \textnormal{GL}(n+1,\Re)
		\;\left\vert\;\,
			A^{T} \cdot Q_{(1,n)} \cdot A = Q_{(1,n)}
			\right.
		\;\right\}
\end{equation*}
\end{definition}

\vskip 0.5cm
\begin{proposition}
\mbox{}
\vskip 0.1cm
\noindent
For each \,$A \in \textnormal{O}(1,n)$, we have:
\begin{enumerate}
\item
	$\det(A) = \pm 1$,\, and
\item
	$\vert\,A^{0}_{0}\,\vert^{2} \;\geq\; 1$;\, equivalently, either \,$A^{0}_{0} \,\geq\, 1$\, or \,$A^{0}_{0} \,\leq\, -1$
\end{enumerate}
\end{proposition}
\proof
\begin{enumerate}
\item
\item
\end{enumerate}
\qed

\vskip 0.5cm
\begin{definition}[$\SO(1,n)$\, and \,$\SO^{\uparrow}(1,n)$]
\mbox{}
\vskip 0.1cm
\noindent
The \textbf{proper Lorentz} group is defined to be:
\begin{equation*}
\SO(1,n)
\; := \;
	\left\{\;\,
		A \overset{{\color{white}.}}{\in} \GL(n+1,\Re)
		\;\left\vert\;\,
			\begin{array}{ccc}
			A^{T} \cdot Q_{(1,n)} \cdot A \,=\, Q_{(1,n)}\,,
			\\
			\textnormal{det}(A) \,\overset{{\color{white}1}}{=}\, 1
			\end{array}
			\right.
		\;\right\}
\end{equation*}
The \textbf{proper orthochronous Lorentz} group is defined to be:
\begin{equation*}
\SO^{\uparrow}(1,n)
\; := \;
	\left\{\;\,
		A \overset{{\color{white}.}}{\in} \GL(n+1,\Re)
		\;\left\vert\;\,
			\begin{array}{ccc}
			A^{T} \cdot Q_{(1,n)} \cdot A \,=\, Q_{(1,n)}\,,
			\\
			\textnormal{det}(A) \,\overset{{\color{white}1}}{=}\, 1\,,
			\\
			A^{0}_{0} \,\overset{{\color{white}1}}{\geq}\, 0
			\end{array}
			\right.
		\;\right\}
\end{equation*}
\end{definition}

\vskip 0.5cm
\begin{proposition}[Parametrization of \,$\so(1,3)$]
\mbox{}
\vskip 0.1cm
\noindent
The Lie algebra \,$\so(1,3)$\, of the real Lie group \,$\SO^{\uparrow}(1,3)$\,
admits the following parametrization:
\begin{equation*}
\so{(1,3)}
\; = \;
	\left\{\;
		A
		\overset{{\color{white}.}}{\in}
		\Re^{4 \times 4}
		\;\;\left\vert\;\;
			A \;=\;
			\left(\begin{array}{rrrr}
			        0 &   a_{01} &  a_{02} & a_{03} \\
			a_{01} &           0 &  a_{12} & a_{13} \\
			a_{02} & -a_{12} &           0 & a_{23} \\
			a_{03} & -a_{13} & -a_{23} &          0 \\
			\end{array}\right)
			\right.
		\;\right\}
\end{equation*}
In particular,
\,$\dim_{\Re}\!\left(\overset{{\color{white}.}}{\SO^{\uparrow}(1,3)}\right)$
\,$=$\,
\,$\dim_{\Re}\!\left(\overset{{\color{white}.}}{\so(1,3)}\right)$
\,$=$\, $6$.\,
\end{proposition}
\proof
First, recall that every element of \,$\so(1,3)$\, is of the form \,$\alpha^{\prime}(0) \in \Re^{4 \times 4}$,\,
where
\,$\alpha : (-\varepsilon,\varepsilon) \longrightarrow \SO^{\uparrow}(1,3)$\,
is a smooth map from an open subinterval \,$(-\varepsilon,\varepsilon) \subset \Re$\,
containing \,$0 \in \Re$\, into
\,$\SO^{\uparrow}(1,3)$\,
such that
\,$\alpha(0) = I_{4 \times 4}$.\,
Thus, \,$\alpha(\,\cdot\,)$\, satisfies:
\begin{equation*}
\alpha(t)^{T} \cdot \Qot \cdot \alpha(t) \;\; = \;\; \Qot,
\quad
\textnormal{for each \,$t \in (-\varepsilon,\varepsilon)$}
\end{equation*}
Differentiation with respect to \,$t$\, yields:
\begin{equation*}
\alpha^{\prime}(t)^{T} \cdot \Qot \cdot \alpha(t) \;+\; \alpha(t)^{T} \cdot \Qot \cdot \alpha^{\prime}(t) \;\; = \;\; 0_{4 \times 4}
\end{equation*}
Evaluating at \,$t = 0$\, and recalling \,$\alpha(0) = I_{4 \times 4}$\, yields:
\begin{equation*}
\alpha^{\prime}(0)^{T} \cdot \Qot \;+\; \Qot \cdot \alpha^{\prime}(0) \;\; = \;\; 0_{4 \times 4}
\end{equation*}
Now, write:
\begin{equation*}
\alpha^{\prime}(0)
\;\; = \;\;
	\left(\begin{array}{cccc}
	a_{00} & a_{01} & a_{02} & a_{03}
	\\
	a_{10} & a_{11} & a_{12} & a_{13}
	\\
	a_{20} & a_{21} & a_{22} & a_{23}
	\\
	a_{30} & a_{31} & a_{32} & a_{33}
	\end{array}\right)
\;\; \in \;\;
	\Re^{4 \times 4}
\end{equation*}
Then,
\begin{equation*}
\alpha^{\prime}(0)^{T} \cdot \Qot
\;\; = \;\;
	\left(\begin{array}{cccc}
	a_{00} & a_{10} & a_{20} & a_{30}
	\\
	a_{01} & a_{11} & a_{21} & a_{31}
	\\
	a_{02} & a_{12} & a_{22} & a_{32}
	\\
	a_{03} & a_{13} & a_{23} & a_{33}
	\end{array}\right)
	\cdot
	\left(\begin{array}{rrrr}
	-1 & 0 & 0 & 0
	\\
	0 & 1 & 0 & 0
	\\
	0 & 0 & 1 & 0
	\\
	0 & 0 & 0 & 0
	\end{array}\right)
\;\; = \;\;
	\left(\begin{array}{cccc}
	-\,a_{00} & a_{10} & a_{20} & a_{30}
	\\
	-\,a_{01} & a_{11} & a_{21} & a_{31}
	\\
	-\,a_{02} & a_{12} & a_{22} & a_{32}
	\\
	-\,a_{03} & a_{13} & a_{23} & a_{33}
	\end{array}\right)
\end{equation*}
and
\begin{equation*}
\Qot \cdot \alpha^{\prime}(0)
\;\; = \;\;
	\left(\begin{array}{rrrr}
	-1 & 0 & 0 & 0
	\\
	0 & 1 & 0 & 0
	\\
	0 & 0 & 1 & 0
	\\
	0 & 0 & 0 & 0
	\end{array}\right)
	\cdot
	\left(\begin{array}{cccc}
	a_{00} & a_{01} & a_{02} & a_{03}
	\\
	a_{10} & a_{11} & a_{12} & a_{13}
	\\
	a_{20} & a_{21} & a_{22} & a_{23}
	\\
	a_{30} & a_{31} & a_{32} & a_{33}
	\end{array}\right)
\;\; = \;\;
	\left(\begin{array}{rrrr}
	-\,a_{00} & -\,a_{01} & -\,a_{02} & -\,a_{03}
	\\
	a_{10} & a_{11} & a_{12} & a_{13}
	\\
	a_{20} & a_{21} & a_{22} & a_{23}
	\\
	a_{30} & a_{31} & a_{32} & a_{33}
	\end{array}\right)
\end{equation*}
Thus,
\begin{equation*}
\alpha^{\prime}(0)^{T} \cdot \Qot \;+\; \Qot \cdot \alpha^{\prime}(0)
\;\; = \;\;
	\left(\begin{array}{cccc}
	-\,2\,a_{00} & a_{10}\,-\,a_{01} & a_{20}\,-\,a_{02} & a_{30}\,-\,a_{03}
	\\
	-\,a_{01} \,+\, a_{10} & 2\,a_{11} & a_{21}\,+\,a_{12} & a_{31}\,+\,a_{13}
	\\
	-\,a_{02} \,+\, a_{20} & a_{12}\,+\,a_{21} & 2\,a_{22} & a_{32}\,+\,a_{23}
	\\
	-\,a_{03} \,+\, a_{30} & a_{13}\,+\,a_{31} & a_{23}\,+\,a_{32} & 2\,a_{33}
	\end{array}\right)
\end{equation*}
Consequently,
\begin{equation*}
\alpha^{\prime}(0)^{T} \cdot \Qot \;+\; \Qot \cdot \alpha^{\prime}(0) \;\; = \;\; 0_{4 \times 4}
\quad\Longleftrightarrow\quad
\left\{\begin{array}{c}
	a_{00} \,=\, a_{11} \,=\, a_{22} \,=\, a_{33} \,=\, 0\,,
	\\
	a_{10} \,=\, a_{01}\,, a_{20} \,=\, a_{02}\,, a_{30} \,=\, a_{03}\,,
	\\
	a_{21} \,=\, -\,a_{12}\,, a_{31} \,=\, -\,a_{13}\,, a_{32} \,=\, -\,a_{23}\,.
	\end{array}\right.
\end{equation*}
This proves:
\begin{equation*}
\so{(1,3)}
\;\; \subset \;\;
	\left\{\;
		A
		\overset{{\color{white}.}}{\in}
		\Re^{4 \times 4}
		\;\;\left\vert\;\;
			A \;=\;
			\left(\begin{array}{rrrr}
			        0 &   a_{01} &  a_{02} & a_{03} \\
			a_{01} &           0 &  a_{12} & a_{13} \\
			a_{02} & -a_{12} &           0 & a_{23} \\
			a_{03} & -a_{13} & -a_{23} &          0 \\
			\end{array}\right)
			\right.
		\;\right\}
\end{equation*}
For the reverse inclusion, consider:
\begin{equation*}
A
\;\; = \;\;
	\left(\begin{array}{rrrr}
	        0 &   a_{01} &  a_{02} & a_{03} \\
		a_{01} &           0 &  a_{12} & a_{13} \\
		a_{02} & -a_{12} &           0 & a_{23} \\
		a_{03} & -a_{13} & -a_{23} &          0 \\
		\end{array}\right)
\;\; \in \;\;
	\Re^{4 \times 4}
\end{equation*}
The required reverse inclusion amounts to the statement that \,$A \in \so(1,3)$.\,
Now, define:
\begin{equation*}
\alpha : \Re \longrightarrow \Re^{4 \times 4} : t \longmapsto \exp\!\left(\,t \cdot \overset{{\color{white}.}}{A}\,\right)
\end{equation*}
Then, \,$\alpha^{\prime}(0) \,=\, A$.\,
Thus, to show that \,$A \in \so(1,3)$,\, it remains only to establish that
\,$\alpha(t) \in \SO^{\uparrow}(1,3)$.\,

\vskip 0.2cm
\noindent
To this end, first note that \,$A \in \Re^{4 \times 4}$\, satisfies:
\,$A^{T} \cdot \Qot + \Qot \cdot A \,=\, 0_{4 \times 4}$,\,
which implies:
\begin{equation*}
A^{T} \cdot \Qot \,=\, -\,\Qot \cdot A\,,
\end{equation*}
which in turn implies:
\begin{equation*}
\left(\,A^{T}\,\right)^{k} \cdot \Qot
\,=\,
	(-1) \cdot \left(\,A^{T}\,\right)^{k-1} \cdot \Qot \cdot A
\,=\,
	(-1)^{2} \cdot \left(\,A^{T}\,\right)^{k-2} \cdot \Qot \cdot A^{2}
\,=\,
	\cdots
\,=\,
	(-1)^{k} \cdot \Qot \cdot A^{k}
\end{equation*}
Therefore,
\begin{equation*}
\exp\!\left(\,t\overset{{\color{white}.}}{A}\,\right)^{T} \cdot \Qot \cdot \exp\!\left(\,t\overset{{\color{white}.}}{A}\,\right)
\,=\,
	\exp\!\left(\,\overset{{\color{white}.}}{t}A^{T}\,\right) \cdot \Qot \cdot \exp\!\left(\,t\overset{{\color{white}.}}{A}\,\right)
\,=\,
	\Qot \cdot \exp\!\left(\,-\,\overset{{\color{white}.}}{t}A\,\right) \cdot \exp\!\left(\,t\overset{{\color{white}.}}{A}\,\right)
\,=\,
	\Qot\,,
\end{equation*}
where the second last equality follows from:
\begin{eqnarray*}
\Qot \cdot \exp\!\left(\,-\,\overset{{\color{white}.}}{t}A\,\right)
& = &
	\Qot \cdot \left(\;\overset{\infty}{\underset{k=0}{\sum}}\,\dfrac{(-1)^{k}t^{k}A^{k}}{k!}\,\right)
\;\; = \;\;
	\left(\;\overset{\infty}{\underset{k=0}{\sum}}\,\dfrac{t^{k}\cdot(-1)^{k}\,\Qot\,A^{k}}{k!}\,\right)
\\
& = &
	\left(\;\overset{\infty}{\underset{k=0}{\sum}}\,\dfrac{t^{k}\cdot(A^{T})^{k}\,\Qot}{k!}\,\right)
\;\; = \;\;
	\left(\;\overset{\infty}{\underset{k=0}{\sum}}\,\dfrac{t^{k}\,(A^{T})^{k}}{k!}\,\right)
	\cdot\Qot
\\
& = &
	\exp\!\left(\,\overset{{\color{white}.}}{t}A^{T}\,\right) \cdot \Qot
\end{eqnarray*}
Thus, we now see that
\,$\alpha(t) \in \textnormal{O}(1,3)$,\, for each \,$t \in \Re$.\,
Next, recall that
\begin{equation*}
\det\!\left(\,\exp(t\overset{{\color{white}.}}{X})\,\right)
\,=\,
	e^{\trace(tX)}\,,
\quad
\textnormal{for each \,$t \in \Re$\, and \,$X \in \Re^{4 \times 4}$}
\end{equation*}
Hence,
\begin{equation*}
\det\!\left(\, \alpha(\overset{{\color{white}.}}{t}) \,\right)
\,=\,
	\det\!\left(\,\exp(t\overset{{\color{white}.}}{A})\,\right)
\,=\,
	e^{\trace(tA)}
\,=\,
	e^{0}
\,=\,
	1
\end{equation*}
Thus, we see furthermore that
\begin{equation*}
\alpha(t) \,\in\, \SO(1,3).
\end{equation*}
However, we also have \,$\alpha(0) \,=\, \exp(0\cdot\!A) \,=\, I_{4 \times 4}$.\,
Continuity of \,$\alpha(\,\cdot\,)$\, now implies that \,$\alpha(\,\cdot\,)$\,
must map all of \,$\Re$\, into the identity component of \,$\SO(1,3)$,\,
i.e., \,$\alpha(t) \in \SO^{\uparrow}(1,3)$,\,
for each \,$t \in \Re$.\,
This proves the reverse inclusion:
\begin{equation*}
\so{(1,3)}
\;\; \supset \;\;
	\left\{\;
		A
		\overset{{\color{white}.}}{\in}
		\Re^{4 \times 4}
		\;\;\left\vert\;\;
			A \;=\;
			\left(\begin{array}{rrrr}
			        0 &   a_{01} &  a_{02} & a_{03} \\
			a_{01} &           0 &  a_{12} & a_{13} \\
			a_{02} & -a_{12} &           0 & a_{23} \\
			a_{03} & -a_{13} & -a_{23} &          0 \\
			\end{array}\right)
			\right.
		\;\right\}
\end{equation*}
and thus completes the proof of the Proposition.
\qed

          %%%%% ~~~~~~~~~~~~~~~~~~~~ %%%%%

\vskip 0.5cm
\begin{corollary}[Generators of \,$\so(1,3)$\, \& their commutation relations]
\mbox{}
\vskip 0.1cm
\noindent
Define the following six matrices with real entries:
\begin{equation*}
R_{23}
\; := \,
	\left(\,\begin{array}{rrrr}
	0 & {\color{white}-}0 & {\color{white}-}0 & {\color{white}-}0 \\
	0 & 0 & 0 & 0 \\
	0 & 0 & 0 & -1 \\
	0 & 0 & 1 & 0 \\
	\end{array}\right),
\;\;
R_{31}
\; := \,
	\left(\,\begin{array}{rrrr}
	0 & {\color{white}-}0 & {\color{white}-}0 & {\color{white}-}0 \\
	0 & 0 & 0 & 1 \\
	0 & 0 & 0 & 0 \\
	0 & -1 & 0 & 0 \\
	\end{array}\right),
\;\;
R_{12}
\; := \,
	\left(\,\begin{array}{rrrr}
	0 & {\color{white}-}0 & {\color{white}-}0 & {\color{white}-}0 \\
	0 & 0 & -1 & 0 \\
	0 & 1 & 0 & 0 \\
	0 & 0 & 0 & 0 \\
	\end{array}\right)
\end{equation*}
\begin{equation*}
B_{01}
\; := \,
	\left(\,\begin{array}{rrrr}
	0 & {\color{white}-}1 & {\color{white}-}0 & {\color{white}-}0 \\
	1 & 0 & 0 & 0 \\
	0 & 0 & 0 & 0 \\
	0 & 0 & 0 & 0 \\
	\end{array}\right),
\;\;
B_{02}
\; := \,
	\left(\,\begin{array}{rrrr}
	0 & {\color{white}-}0 & {\color{white}-}1 & {\color{white}-}0 \\
	0 & 0 & 0 & 0 \\
	1 & 0 & 0 & 0 \\
	0 & 0 & 0 & 0 \\
	\end{array}\right),
\;\;
B_{03}
\; := \,
	\left(\,\begin{array}{rrrr}
	0 & {\color{white}-}0 & {\color{white}-}0 & {\color{white}-}1 \\
	0 & 0 & 0 & 0 \\
	0 & 0 & 0 & 0 \\
	1 & 0 & 0 & 0 \\
	\end{array}\right)
\end{equation*}
\vskip 0.3cm
\noindent
Define also the following six matrices with complex entries:
\vskip -0.9cm
\mbox{}
\begin{multicols}{2}
	\begin{minipage}{6.0cm}
	\begin{eqnarray*}
	J_{1}
	& := &
		\i \cdot R_{23}
	\;\; = \;\;
		\left(\,\begin{array}{rrrr}
		0 & {\color{white}-}0 & {\color{white}-}0 & {\color{white}-}0 \\
		0 & 0 & 0 & 0 \\
		0 & 0 & 0 & -\i \\
		0 & 0 & \i & 0 \\
		\end{array}\right),
	\\
	J_{2}
	& := &
		\i \cdot R_{31}
	\;\; = \;\;
		\left(\,\begin{array}{rrrr}
		0 & {\color{white}-}0 & {\color{white}-}0 & {\color{white}-}0 \\
		0 & 0 & 0 &  \i \\
		0 & 0 & 0 & 0 \\
		0 & -\i & 0 & 0 \\
		\end{array}\right),
	\\
	J_{3}
	& := &
		\i \cdot R_{12}
	\;\; = \;\;
		\left(\,\begin{array}{rrrr}
		0 & {\color{white}-}0 & {\color{white}-}0 & {\color{white}-}0 \\
		0 & 0 & -\i & 0 \\
		0 & \i & 0 & 0 \\
		0 & 0 & 0 & 0 \\
		\end{array}\right),
	\end{eqnarray*}
	\end{minipage}
\columnbreak
	\begin{minipage}{11.5cm}
	\begin{eqnarray*}
	K_{1}
	& := &
		\i \cdot B_{01}
	\;\; = \;\;
		\left(\,\begin{array}{rrrr}
		0 & {\color{white}-}\i & {\color{white}-}0 & {\color{white}-}0 \\
		\i & 0 & 0 & 0 \\
		0 & 0 & 0 & 0 \\
		0 & 0 & 0 & 0 \\
		\end{array}\right),
	\\
	K_{2}
	& := &
		\i \cdot B_{02}
	\;\; = \;\;
		\left(\,\begin{array}{rrrr}
		0 & {\color{white}-}0 & {\color{white}-}\i & {\color{white}-}0 \\
		0 & 0 & 0 & 0 \\
		\i & 0 & 0 & 0 \\
		0 & 0 & 0 & 0 \\
		\end{array}\right),
	\\
	K_{3}
	& := &
		\i \cdot B_{03}
	\;\; = \;\;
		\left(\,\begin{array}{rrrr}
		0 & {\color{white}-}0 & {\color{white}-}0 & {\color{white}-}\i \\
		0 & 0 & 0 & 0 \\
		0 & 0 & 0 & 0 \\
		\i & 0 & 0 & 0 \\
		\end{array}\right).
	\end{eqnarray*}
	\end{minipage}
\end{multicols}
\begin{multicols}{2}
	\begin{minipage}{8cm}
	\begin{eqnarray*}
	N^{+}_{1}
	\; := \;
		\dfrac{1}{2}\left(\,J_{1} + \i \, K_{1}\,\right)
	\; = \;
		\dfrac{1}{2}\,\cdot
		\left(\!\begin{array}{rrrr}
		 0 & {\color{black}-}1 & {\color{white}-}0 & {\color{white}-}0 \\
		-1 & 0 & 0 & 0 \\
		 0 & 0 & 0 & -\i \\
		 0 & 0 & \i & 0 \\
		\end{array}\right),
	\quad\quad{\color{white}.}
	\\
	N^{+}_{2}
	\; := \;
		\dfrac{1}{2}\left(\,J_{2} + \i \, K_{2}\,\right)
	\; = \;
		\dfrac{1}{2}\,\cdot
		\left(\!\begin{array}{rrrr}
		 0 & {\color{white}-}0 & {\color{black}-}1 & {\color{white}-}0 \\
		 0 & 0 & 0 & \i \\
		-1 & 0 & 0 & 0 \\
		 0 & -\i & 0 & 0 \\
		\end{array}\right),
	\quad\quad{\color{white}.}
	\\
	N^{+}_{3}
	\; := \;
		\dfrac{1}{2}\left(\,J_{3} + \i \, K_{3}\,\right)
	\; = \;
		\dfrac{1}{2}\,\cdot
		\left(\!\begin{array}{rrrr}
		 0 & {\color{white}-}0 & {\color{white}-}0 & {\color{black}-}1 \\
		 0 & 0 & -\i & 0 \\
		 0 &  \i & 0 & 0 \\
		-1 & 0 & 0 & 0 \\
		\end{array}\right),
	\quad\quad{\color{white}.}
	\end{eqnarray*}
	\end{minipage}
\columnbreak
	\begin{minipage}{12.0cm}
	\begin{eqnarray*}
	N^{-}_{1}
	\; := \;
		\dfrac{1}{2}\left(\,J_{1} - \i \, K_{1}\,\right)
	\; = \;
		\dfrac{1}{2}\,\cdot
		\left(\!\!\!\begin{array}{rrrr}
		{\color{white}-}0 & {\color{white}-}1 & {\color{white}-}0 & {\color{white}-}0 \\
		1 & 0 & 0 & 0 \\
		0 & 0 & 0 & -\i \\
		0 & 0 & \i & 0 \\
		\end{array}\right),
	\\
	N^{-}_{2}
	\; := \;
		\dfrac{1}{2}\left(\,J_{2} - \i \, K_{2}\,\right)
	\; = \;
		\dfrac{1}{2}\,\cdot
		\left(\!\!\!\begin{array}{rrrr}
		{\color{white}-}0 & {\color{white}-}0 & {\color{white}-}1 & {\color{white}-}0 \\
		0 & 0 & 0 & \i \\
		1 & 0 & 0 & 0 \\
		0 & -\i & 0 & 0 \\
		\end{array}\right),
	\\
	N^{-}_{3}
	\; := \;
		\dfrac{1}{2}\left(\,J_{3} - \i \, K_{3}\,\right)
	\; = \;
		\dfrac{1}{2}\,\cdot
		\left(\!\!\!\begin{array}{rrrr}
		{\color{white}-}0 & {\color{white}-}0 & {\color{white}-}0 & {\color{white}-}1 \\
		0 & 0 & -\i & 0 \\
		0 &  \i & 0 & 0 \\
		1 & 0 & 0 & 0 \\
		\end{array}\right).
	\end{eqnarray*}
	\end{minipage}
\end{multicols}
\noindent
Then, the following statements are true:
\begin{enumerate}
\item
	The matrices
	\,$R_{23},\; R_{31},\; R_{12},\; B_{01},\; B_{02},\; B_{03}$\,
	are elements of
	\,$\so(1,3)$,\, and they form a basis for \,$\so(1,3)$.\,
	The
	\,$15 \,= \left(\begin{array}{c}6 \\ 2\end{array}\right)$\,
	commutation relations satisfied by
	\,$R_{23},\, R_{31},\, R_{12},\, B_{01},\, B_{02},\, B_{03}$\,
	are:
	\begin{equation*}
	\begin{array}{lll}
	\left[\,R_{23}\,,\,R_{31}\,\right] \,=\, +\,R_{12}, &
	\left[\,R_{12}\,,\,R_{23}\,\right] \,=\, +\,R_{31}, &
	\left[\,R_{31}\,,\,R_{12}\,\right] \,=\, +\,R_{23},
	\\ \\
	\left[\,B_{01}\,,\,B_{02}\,\right] \,=\, -\,R_{12}, &
	\left[\,B_{03}\,,\,B_{01}\,\right] \,=\, -\,R_{31}, &
	\left[\,B_{02}\,,\,B_{03}\,\right] \,=\, -\,R_{23},
	\\ \\
	\left[\,R_{23}\,,\,B_{01}\,\right] \,=\, {\color{white}-}\,0,\;\;\;\; &
	\left[\,R_{23}\,,\,B_{02}\,\right] \,=\, +\,B_{03}, &
	\left[\,R_{23}\,,\,B_{03}\,\right] \,=\, -\,B_{02}, 
	\\
	\left[\,R_{31}\,,\,B_{01}\,\right] \,=\, -\,B_{03}, &
	\left[\,R_{31}\,,\,B_{02}\,\right] \,=\, {\color{white}-}\,0,\;\;\;\; &
	\left[\,R_{31}\,,\,B_{03}\,\right] \,=\, +\,B_{01},
	\\
	\left[\,R_{12}\,,\,B_{01}\,\right] \,=\, +\,B_{02}, &
	\left[\,R_{12}\,,\,B_{02}\,\right] \,=\, -\,B_{01}, &
	\left[\,R_{12}\,,\,B_{03}\,\right] \,=\, {\color{white}-}\,0,\;\;\;\;
	\end{array}
	\end{equation*}
\item
	The matrices
	\,$J_{1},\, J_{2},\, J_{3},\, K_{1},\, K_{2},\, K_{3}$\,
	are elements of
	\,$\so(1,3) \otimes_{\Re} \C$,\, and they form a basis for \,$\so(1,3) \otimes_{\Re} \C$.\,
	The
	\,$15 \,= \left(\begin{array}{c}6 \\ 2\end{array}\right)$\,
	commutation relations satisfied by
	\,$J_{1},\, J_{2},\, J_{3},\, K_{1},\, K_{2},\, K_{3}$\,
	are:
	\begin{equation*}
	\begin{array}{lll}
	\left[\,\;J_{1}\,,\,\;J_{2}\,\right] \,=\, + \, \i \, J_{3}, &
	\left[\,\;J_{3}\,,\,\;J_{1}\,\right] \,=\, + \, \i \, J_{2}, &
	\left[\,\;J_{2}\,,\,\;J_{3}\,\right] \,=\, + \, \i \, J_{1},
	\\ \\
	\left[\,K_{1}\,,\,K_{2}\,\right] \,=\, - \, \i \, J_{3}, &
	\left[\,K_{3}\,,\,K_{1}\,\right] \,=\, - \, \i \, J_{2}, &
	\left[\,K_{2}\,,\,K_{3}\,\right] \,=\, - \, \i \, J_{1},
	\\ \\
	\left[\,\;J_{1}\,,\,K_{1}\,\right] \,=\, {\color{white}-}\,0,\;\;\;\; &
	\left[\,\;J_{1}\,,\,K_{2}\,\right] \,=\, + \, \i \, K_{3}, &
	\left[\,\;J_{1}\,,\,K_{3}\,\right] \,=\, - \, \i \, K_{2},
	\\
	\left[\,\;J_{2}\,,\,K_{1}\,\right] \,=\, - \, \i \, K_{3}, &
	\left[\,\;J_{2}\,,\,K_{2}\,\right] \,=\, {\color{white}-}\,0,\;\;\;\; &
	\left[\,\;J_{2}\,,\,K_{3}\,\right] \,=\, + \, \i \, K_{1},
	\\
	\left[\,\;J_{3}\,,\,K_{1}\,\right] \,=\, + \, \i \, K_{2}, &
	\left[\,\;J_{3}\,,\,K_{2}\,\right] \,=\, - \, \i \, K_{1}, &
	\left[\,\;J_{3}\,,\,K_{3}\,\right] \,=\, {\color{white}-}\,0,\;\;\;\;
	\end{array}
	\end{equation*}
\item
	The matrices
	\,$N^{\pm}_{a}$,\, for \,$a \,\in\, \{\,1,2,3\,\}$,
	are elements of
	\,$\so(1,3) \otimes_{\Re} \C$,\, and they form a basis for \,$\so(1,3) \otimes_{\Re} \C$.\,
	The elements
	\,$N^{\pm}_{a}$,\, for \,$a \,\in\, \{\,1,2,3\,\}$,
	satisfy the following commutation relations:
	\begin{equation*}
	\left[\,N^{\pm}_{a}\,,\,N^{\pm}_{b}\,\right]
	\;\;=\;\;
		\i \cdot \overset{3}{\underset{c\,=\,1}{\sum}} \;\varepsilon_{abc} \, N^{\pm}_{c}\,,
	\quad
	\textnormal{for each \,$a, b, c \,\in\, \{\,1, 2, 3\,\}$}
	\end{equation*}
	\begin{equation*}
	\left[\,N^{+}_{a}\,,\,N^{-}_{b}\,\right] \;\;=\;\; 0\,,
	\quad
	\textnormal{for each \,$a, b \,\in\, \{\,1, 2, 3\,\}$}
	\end{equation*}
	Thus,
	\,$\so(1,3) \otimes_{\Re} \C$\,
	contains two commuting copies of
	\,$\su(2) \otimes_{\Re} \C$.\,
\end{enumerate}
\end{corollary}
\proof

\qed

          %%%%% ~~~~~~~~~~~~~~~~~~~~ %%%%%

%\section{Generators of \,$\mathfrak{su}(2)$}
%
%          %%%%% ~~~~~~~~~~~~~~~~~~~~ %%%%%
%
%\begin{proposition}[Characterizations of \,$\mathfrak{sl}(n)$, \,$\mathfrak{u}(n)$, and $\mathfrak{su}(n)$]
%\mbox{}
%\vskip 0.1cm
%\begin{enumerate}
%\item
%	\begin{equation*}
%	\mathfrak{sl}(n,\C)
%	\; = \;
%		\left\{\;
%			X \,\in\, \mathfrak{gl}(n,\C) \,=\, \C^{n \times n}
%			\;\left\vert\;\,
%				\textnormal{trace}(X) = \overset{{\color{white}1}}{0}
%				\right.
%			\,\right\}
%	\end{equation*}
%\item
%	\begin{equation*}
%	\mathfrak{u}(n)
%	\; = \;
%		\left\{\;
%			X \,\in\, \mathfrak{gl}(n,\C) \,=\, \C^{n \times n}
%			\;\left\vert\;\,
%				X + X^{\dagger} = \overset{{\color{white}1}}{0}
%				\right.
%			\,\right\}
%	\end{equation*}
%\item
%	\begin{equation*}
%	\mathfrak{su}(n)
%	\; = \;
%		\left\{\;
%			X \,\in\, \mathfrak{gl}(n,\C) \,=\, \C^{n \times n}
%			\;\left\vert\;\,
%				\begin{array}{c}
%				X + X^{\dagger} = \overset{{\color{white}1}}{0}
%				\\
%				\textnormal{trace}(X) = \overset{{\color{white}1}}{0}
%				\end{array}
%				\right.
%			\,\right\}
%	\end{equation*}
%\end{enumerate}
%\end{proposition}
%\proof
%\begin{enumerate}
%\item
%	We invoke the fact that \,$\det(e^{\,t\,\cdot\,X}) \,=\, e^{\,t\,\cdot\,\textnormal{trace}(X)}$,
%	for each \,$X \in \C^{n \times n}$.
%	Thus,
%	\begin{eqnarray*}
%	&&
%		X \,\in\, \mathfrak{sl}(n,\C)
%		\quad\Longrightarrow\quad
%		e^{\,t\cdot\,X} \,\in\, \textnormal{SL}(n,\C)
%		\quad\Longrightarrow\quad
%		\det\!\left(\,e^{\,t\cdot\,X}\,\right) \,=\, 1
%	\\
%	& \Longrightarrow\quad &
%		\textnormal{trace}(X)
%		\; = \;
%			\left.\dfrac{\d}{\d\,t}\right\vert_{t=0}\left(\,\overset{{\color{white}1}}{e^{\,t\,\cdot\,\textnormal{trace}(X)}}\,\right)
%		\; = \;
%			\left.\dfrac{\d}{\d\,t}\right\vert_{t=0}\left(\,\overset{{\color{white}1}}{\det(e^{\,t\,\cdot\,X})}\,\right)
%		\; = \;
%			\left.\dfrac{\d}{\d\,t}\right\vert_{t=0}\left(\,\overset{{\color{white}1}}{1}\,\right)
%		\; = \;
%			0
%	\end{eqnarray*}
%	Conversely, suppose \,$\textnormal{trace}(X) = 0$.\,
%	Then, \,$\det(e^{\,t\,\cdot\,X}) \,=\, e^{\,t\,\cdot\,\textnormal{trace}(X)} \,=\, e^{\,t\,\cdot\,0} \,=\, 1$,\,
%	which implies that \,$e^{\,t\,\cdot\,X} \,\in\, \textnormal{SL}(n,\C)$,\, hence \,$X \,\in\, \mathfrak{sl}(n,\C)$.
%	This completes the proof of the equality (of sets) in question.
%\item
%	\begin{eqnarray*}
%	&&
%		X \,\in\, \mathfrak{u}(n)
%		\quad\Longrightarrow\quad
%		e^{\,t\,\cdot\,X} \,\in\, \textnormal{U}(n)
%	\\
%	& \Longrightarrow\quad &
%		I_{n}
%			\,=\, \left(\,e^{\,t\,\cdot\,X}\,\right)^{\!\dagger} \cdot \left(\,e^{\,t\,\cdot\,X}\,\right)
%			\,=\, \left(\,e^{\,t\,\cdot\,X^{\dagger}}\,\right) \cdot \left(\,e^{\,t\,\cdot\,X}\,\right)
%			\,=\, e^{\,t\,\cdot\,(X^{\dagger}+X)}
%	\\
%	& \Longrightarrow\quad &
%		X \,+\, X^{\dagger}
%		\; = \;
%			\left.\dfrac{\d}{\d\,t}\right\vert_{t=0}\left(\,\overset{{\color{white}1}}{e^{\,t\,\cdot\,(X+X^{\dagger})}}\,\right)
%		\; = \;
%			\left.\dfrac{\d}{\d\,t}\right\vert_{t=0}\left(\,\overset{{\color{white}1}}{I_{n}}\,\right)
%		\; = \;
%			0
%	\end{eqnarray*}
%	Conversely, suppose \,$X + X^{\dagger} \,=\, 0$.\,
%	Then, \,$I_{n}$
%	\,$=$\, $e^{\,0_{n \times n}}$
%	\,$=$\, $e^{\,t\,\cdot(X^{\dagger}+X)}$
%	\,$=\, \cdots \,=$\, $\left(e^{\,t\,\cdot\,X}\right)^{\!\dagger}\cdot\left(e^{\,t\,\cdot\,X}\right)$,\,
%	which implies that \,$e^{\,t\,\cdot\,X} \,\in\, \textnormal{U}(n)$,\, hence \,$X \,\in\, \mathfrak{u}(n)$.
%	This completes the proof of the equality (of sets) in question.
%\item
%	Immediate by the preceding two statements.
%\end{enumerate}
%\qed
%
%\vskip 0.5cm
%\begin{proposition}[Generators of \,$\mathfrak{su}(2)$]
%\mbox{}
%\vskip 0.1cm
%\noindent
%Let \,$\sigma_{1},\, \sigma_{2},\, \sigma_{3} \,\in\, \C^{2 \times 2}$\, be the \textbf{Pauli spin matrices}, i.e.,
%\begin{equation*}
%\sigma_{1} \,=\, \sigma_{x} \,:=\, \left(\begin{array}{cc} 0 & 1 \\ 1 & 0 \end{array}\right),
%\quad
%\sigma_{2} \,=\, \sigma_{y} \,:=\, \left(\begin{array}{rr} 0 & -\i \\ \i & 0 \end{array}\right),
%\quad
%\sigma_{3} \,=\, \sigma_{z} \,:=\, \left(\begin{array}{rr} 1 & 0 \\ 0 & -1 \end{array}\right).
%\end{equation*}
%Define \,$J_{1},\, J_{2},\, J_{3},\, S_{+},\, S_{-},\, S_{3} \,\in\, \C^{2 \times 2}$\, as follows:
%\begin{equation*}
%J_{1} \,:=\, \dfrac{\i}{2}\cdot\sigma_{1} \,=\, \dfrac{\i}{2}\cdot\left(\begin{array}{cc} 0 & 1 \\ 1 & 0 \end{array}\right),
%\quad
%J_{2} \,:=\, \mathbf{{\color{red}-}}\,\dfrac{\i}{2}\cdot\sigma_{2} \,=\, \dfrac{1}{2}\cdot\left(\begin{array}{rr} 0 & -1 \\ 1 & 0 \end{array}\right),
%\quad
%J_{3} \,:=\, \dfrac{\i}{2}\cdot\sigma_{3} \,=\, \dfrac{\i}{2}\cdot\left(\begin{array}{rr} 1 & 0 \\ 0 & -1 \end{array}\right),
%\end{equation*}
%\begin{equation*}
%S_{+} \,:=\, \dfrac{1}{\i}\left(\,J_{1} + \i\,J_{2}\,\right) \,=\, \left(\begin{array}{cc} 0 & 0 \\ 1 & 0 \end{array}\right),
%\quad
%S_{-} \,:=\, \dfrac{1}{\i}\left(\,J_{1} - \i\,J_{2}\,\right) \,=\, \left(\begin{array}{rr} 0 & 1 \\ 0 & 0 \end{array}\right),
%\quad
%S_{3} \,:=\, \i\cdot J_{3} \,=\, \dfrac{1}{2}\cdot\left(\begin{array}{rr} -1 & 0 \\ 0 & 1 \end{array}\right).
%\end{equation*}
%Then, the following statements are true:
%\begin{enumerate}
%\item
%	$J_{1},\, J_{2},\, J_{3} \,\in\, \mathfrak{su}(2)$\,
%\item
%	$J_{1},\, J_{2},\, J_{3}$\,
%	form a set of generators for the (real) Lie algebra \,$\mathfrak{su}(2)$\, of the (real) Lie group \,$\textnormal{SU}(2)$.
%\item
%	$J_{1},\, J_{2},\, J_{3}$\, satisfy the following commutation relations:
%	\begin{equation*}
%	\left[\,J_{a}\,,\,J_{b}\,\right] \;\; = \;\; \overset{3}{\underset{c\,=\,1}{\sum}}\;\epsilon_{abc}\,J_{c}\,,
%	\quad
%	\textnormal{for \,$a, b = 1,2,3$}.
%	\end{equation*}
%\item
%	$S_{+},\, S_{-},\, S_{3} \,\in\, \mathfrak{su}(2) \otimes_{\Re} \C$,\,
%	where
%	\,$\mathfrak{su}(2) \otimes_{\Re} \C$\,
%	is the complexification of (the real Lie algebra)
%	\,$\mathfrak{su}(2)$.
%\item
%	$S_{+},\; S_{-},\; S_{3}$\, satisfy the following commutation relations:
%	\begin{equation*}
%	\left[\,S_{+}\,,\,S_{-}\,\right] \, = \, 2\,S_{3}\,,
%	\quad
%	\left[\,S_{3}\,,\,S_{\pm}\,\right] \, = \, \pm\,S_{\pm}
%	\end{equation*}
%\item
%	Suppose
%	\begin{itemize}
%	\item
%		$V$\, is a complex vector space,
%	\item
%		$\rho : \mathfrak{su}(2) \otimes_{\Re} \C \longrightarrow \textnormal{End}(V)$\,
%		is a Lie algebra representation, and
%	\item	
%		$v \in V$\, and \,$\lambda \in \C$\, together satisfy \,$\rho(S_{3})(v) = \lambda \cdot v$.
%	\end{itemize}	
%	Then, \,$\rho(S_{+})(v) \,\in\, V$\, satisfies:
%	\begin{equation*}
%	\rho(S_{3})\!\left(\,\rho(\overset{{\color{white}.}}{S}_{+})(v)\,\right)
%	\; = \;
%		(\lambda+1) \cdot \rho(S_{+})(v)\,
%	\end{equation*}
%	and
%	\,$\rho(S_{-})(v) \,\in\, V$\, satisfies:
%	\begin{equation*}
%	\rho(S_{3})\!\left(\,\rho(\overset{{\color{white}.}}{S}_{-})(v)\,\right)
%	\; = \;
%		(\lambda-1) \cdot \rho(S_{-})(v)\,
%	\end{equation*}
%\end{enumerate}
%\end{proposition}
%\proof
%The Corollary follows straightforwardly by direct computations.
%\qed
%
%          %%%%% ~~~~~~~~~~~~~~~~~~~~ %%%%%
%
%\vskip 0.5cm
%\begin{definition}[$\textnormal{O}(n)$ and $\textnormal{SO}(n)$]
%\mbox{}
%\vskip 0.1cm
%\noindent
%The \textbf{orthogonal group} is defined as follows:
%\begin{equation*}
%\textnormal{O}(n)
%\; := \;
%	\left\{\;\,
%		g \overset{{\color{white}.}}{\in} \textnormal{GL}(n,\Re)
%		\;\left\vert\;\,
%			g^{T} \cdot g = I_{n}
%			\right.
%		\;\right\}
%\end{equation*}
%The \textbf{special orthogonal group} is defined as follows:
%\begin{equation*}
%\textnormal{SO}(n)
%\; := \;
%	\left\{\;\,
%		g \overset{{\color{white}.}}{\in} \textnormal{GL}(n,\Re)
%		\;\left\vert\;\,
%			g^{T} \cdot g = I_{n}\,,
%			\;
%			\textnormal{det}(g) = 1
%			\right.
%		\;\right\}
%\end{equation*}
%\end{definition}
%
%          %%%%% ~~~~~~~~~~~~~~~~~~~~ %%%%%
%
%\begin{proposition}[Lie algebras of $\textnormal{O}(n)$ and $\textnormal{SO}(n)$]
%\begin{eqnarray*}
%\mathfrak{o}(n)
%& = &
%	\left\{\;\,
%		X \overset{{\color{white}.}}{\in} \mathfrak{gl}(n,\Re)
%		\;\left\vert\;\,
%			X^{T} = -X
%			\right.
%		\;\right\}
%\\
%\mathfrak{so}(n)
%& = &
%	\left\{\;\,
%		X \overset{{\color{white}.}}{\in} \mathfrak{gl}(n,\Re)
%		\;\left\vert\;\,
%			X^{T} = -X\,,
%			\;
%			\textnormal{trace}(X) = 0
%			\right.
%		\;\right\}
%\end{eqnarray*}
%\end{proposition}
%
%          %%%%% ~~~~~~~~~~~~~~~~~~~~ %%%%%
%
%\section{Generators of \;$\textnormal{SO}(3)$\, and \,$\mathfrak{so}(3)$}
%
%          %%%%% ~~~~~~~~~~~~~~~~~~~~ %%%%%
%
%\vskip 0.1cm
%\noindent
%\textbf{Euler matrices}
%\begin{equation*}
%R_{1}(\phi)
%\; := \;
%	\left(\,
%		\begin{array}{ccc}
%			{\color{white}-}1 & {\color{white}-}0 & {\color{white}-}0 \\
%			{\color{white}-}0 & {\color{white}-}\cos\phi & -\sin\phi \\
%			{\color{white}-}0 & {\color{white}-}\sin\phi & {\color{white}-}\cos\phi \\
%			\end{array}
%		\,\right)
%\end{equation*}
%\begin{equation*}
%R_{2}(\psi)
%\; := \;
%	\left(\,
%		\begin{array}{ccc}
%			{\color{white}-}\cos\psi & {\color{white}-}0 & {\color{white}-}\sin\psi \\
%			{\color{white}-}0 & {\color{white}-}1 & {\color{white}-}0 \\
%			-\sin\psi & {\color{white}-}0 & {\color{white}-}\cos\psi \\
%			\end{array}
%		\,\right)
%\end{equation*}
%\begin{equation*}
%R_{3}(\theta)
%\; := \;
%	\left(\,
%		\begin{array}{ccc}
%			{\color{white}-}\cos\theta & -\sin\theta & {\color{white}-}0 \\
%			{\color{white}-}\sin\theta & {\color{white}-}\cos\theta & {\color{white}-}0 \\
%			{\color{white}-}0 & {\color{white}-}0 & {\color{white}-}1 \\
%			\end{array}
%		\,\right)
%\end{equation*}
%
%          %%%%% ~~~~~~~~~~~~~~~~~~~~ %%%%%
%
%\vskip 0.5cm
%\noindent
%\textbf{The generators \,$J_{n} \in \C^{3 \times 3}$\, of the Euler matrices}
%\begin{equation*}
%R_{n}(\theta)
%\; = \;
%	\exp\!\left(\;\sqrt{-1}\cdot\theta \overset{{\color{white}1}}{\cdot} J_{n}\,\right)
%\; = \;
%	\exp\!\left(\;\i\cdot\theta \overset{{\color{white}1}}{\cdot} J_{n}\,\right)
%\end{equation*}
%Alternatively, note:
%\begin{equation*}
%\i \cdot J_{1}
%\;\; = \;\;
%	\left.\dfrac{\d}{\d\,\phi}\right\vert_{\phi = 0} R_{1}(\phi)
%\;\; = \;
%	\left.\left(\,
%		\begin{array}{ccc}
%			{\color{white}-}1 & {\color{white}-}0 & {\color{white}-}0 \\
%			{\color{white}-}0 & {\color{white}-}\sin\phi & {\color{white}-}\cos\phi \\
%			{\color{white}-}0 & {\color{black}-}\cos\phi & {\color{white}-}\sin\phi \\
%			\end{array}
%		\,\right)\right\vert_{\phi = 0}
%\;\; = \;
%	\left(\,
%		\begin{array}{ccc}
%			{\color{white}-}0 & {\color{white}-}0 & {\color{white}-}0 \\
%			{\color{white}-}0 & {\color{white}-}0 & {\color{white}-}1 \\
%			{\color{white}-}0 & {\color{black}-}1 & {\color{white}-}0 \\
%			\end{array}
%		\,\right)
%\end{equation*}
%Multiplying both sides by \,$-\,\i = -\,\sqrt{-1}$\, gives:
%\begin{equation*}
%J_{1}
%\;\; = \;
%	\left(\!\!
%		\begin{array}{ccc}
%			{\color{white}-}0 & {\color{white}-}0 & {\color{white}-}0 \\
%			{\color{white}-}0 & {\color{white}-}0 & {\color{black}-}\i \\
%			{\color{white}-}0 & {\color{white}-}\i & {\color{white}-}0 \\
%			\end{array}
%		\,\right)
%\end{equation*}
%Similarly,
%\begin{equation*}
%\i \cdot J_{2}
%\;\; = \;\;
%	\left.\dfrac{\d}{\d\,\psi}\right\vert_{\psi = 0} R_{2}(\psi)
%\;\; = \;
%	\left.\left(\!\!
%		\begin{array}{ccc}
%			{\color{white}-}\sin\psi & {\color{white}-}0 & {\color{black}-}\cos\psi \\
%			{\color{white}-}0 & {\color{white}-}0 & {\color{white}-}0 \\
%			{\color{white}-}\cos\psi & {\color{white}-}0 & {\color{white}-}\sin\psi \\
%			\end{array}
%		\,\right)\right\vert_{\psi = 0}
%\;\; = \;
%	\left(\!\!
%		\begin{array}{ccc}
%			{\color{white}-}0 & {\color{white}-}0 & {\color{black}-}1 \\
%			{\color{white}-}0 & {\color{white}-}0 & {\color{white}-}0 \\
%			{\color{white}-}1 & {\color{white}-}0 & {\color{white}-}0 \\
%			\end{array}
%		\,\right)
%\end{equation*}
%Multiplying both sides by \,$-\,\i = -\,\sqrt{-1}$\, gives:
%\begin{equation*}
%J_{2}
%\;\; = \;
%	\left(\!\!
%		\begin{array}{ccc}
%			{\color{white}-}0 & {\color{white}-}0 & {\color{white}-}\i \\
%			{\color{white}-}0 & {\color{white}-}0 & {\color{white}-}0 \\
%			{\color{black}-}\i & {\color{white}-}0 & {\color{white}-}0 \\
%			\end{array}
%		\,\right)
%\end{equation*}
%Lastly,
%\begin{equation*}
%\i \cdot J_{3}
%\;\; = \;\;
%	\left.\dfrac{\d}{\d\,\theta}\right\vert_{\theta = 0} R_{3}(\theta)
%\;\; = \;
%	\left.\left(\!
%		\begin{array}{ccc}
%			{\color{white}-}\sin\theta & {\color{white}-}\cos\theta & {\color{white}-}0 \\
%			{\color{black}-}\cos\theta & {\color{white}-}\sin\theta & {\color{white}-}0 \\
%			{\color{white}-}0 & {\color{white}-}0 & {\color{white}-}0 \\
%			\end{array}
%		\,\right)\right\vert_{\psi = 0}
%\;\; = \;
%	\left(
%		\begin{array}{ccc}
%			{\color{white}-}0 & {\color{white}-}1 & {\color{white}-}0 \\
%			{\color{black}-}1 & {\color{white}-}0 & {\color{white}-}0 \\
%			{\color{white}-}0 & {\color{white}-}0 & {\color{white}-}0 \\
%			\end{array}
%		\,\right)
%\end{equation*}
%Multiplying both sides by \,$-\,\i = -\,\sqrt{-1}$\, gives:
%\begin{equation*}
%J_{3}
%\;\; = \;
%	\left(\!
%		\begin{array}{ccc}
%			{\color{white}-}0 & {\color{black}-}\i & {\color{white}-}0 \\
%			{\color{white}-}\i & {\color{white}-}0 & {\color{white}-}0 \\
%			{\color{white}-}0 & {\color{white}-}0 & {\color{white}-}0 \\
%			\end{array}
%		\,\right)
%\end{equation*}
%
%          %%%%% ~~~~~~~~~~~~~~~~~~~~ %%%%%
%
%\section{Properties of the generators \,$J_{1}, J_{2}, J_{3} \,\in\, \mathfrak{so}(3) \otimes_{\Re} \C$}
%
%          %%%%% ~~~~~~~~~~~~~~~~~~~~ %%%%%
%
%\begin{proposition}
%{\color{white}.}\vskip -0.5cm{\color{white}.}
%\begin{enumerate}
%\item
%	\textbf{Commutation relations:}\;\;
%	\begin{equation*}
%	\left[\,J_{k}\,\overset{{\color{white}1}}{,}\,J_{l}\,\right]
%	\;\; = \;\;
%		\sqrt{-1}\;\overset{3}{\underset{m=1}{\sum}}\,\varepsilon_{klm}\cdot J_{m}\,,
%	\quad
%	\textnormal{for each \,$k, l \in \{\,1,2,3\,\}$}\,,
%	\end{equation*}
%	where \,$\varepsilon_{klm}$\, is the fully anti-symmetric tensor.
%\item
%	\textbf{Raising and lowering operators:}\;\;
%	Define \,$J_{+}\,,\, J_{-} \in \mathfrak{so}(3) \otimes_{\Re} \C \subset \mathcal{U}\!\left(\mathfrak{so}(3) \overset{{\color{white}.}}{\otimes_{\Re}} \C\right)$\, as follows:
%	\begin{equation*}
%	J_{\pm} \;\; := \;\; J_{1} \, \pm \sqrt{-1}\,J_{2}.
%	\end{equation*}
%	Then, the following equalities (of elements of $\mathcal{U}\!\left(\mathfrak{so}(3) \overset{{\color{white}.}}{\otimes_{\Re}} \C\right)$) hold:
%	\begin{enumerate}
%	\item
%		$\left[\,J_{3}\,,\,J_{+}\,\right] \;=\; J_{+}$\,,
%		\quad
%		$\left[\,J_{3}\,,\,J_{-}\,\right] \;=\; -\,J_{-}$\,,
%		\quad
%		$\left[\,J_{+}\,,\,J_{-}\,\right] \;=\; 2\,J_{3}$
%	\item
%		$J^{2}$
%		\; $=$ \; $(J_{3})^{2} \,-\, J_{3} \,+\, J_{+}J_{-}$
%		\; $=$ \; $(J_{3})^{2} \,+\, J_{3} \,+\, J_{-}J_{+}$
%	\item
%		$(J_{\pm})^{\dagger} \; = \; J_{\mp}$
%	\end{enumerate}
%\item
%	Suppose
%	\,$\rho : \mathfrak{so}(3) \otimes_{\Re} \C \longrightarrow \mathfrak{gl}(V)$\,
%	is an irreducible finite-dimensional complex representation, and
%	\,$v \in V \backslash\{0\}$\, is an eigenvector of \,$\rho(J_{3})$\,
%	corresponding to the eigenvalue \,$\lambda \in \Re$;\, thus, \,$\rho(J_{3})(v) \,=\, \lambda\,v$.
%	Then, we have:
%	\begin{equation*}
%	\rho(J_{3})\!\left(\,\rho(J_{+})(\overset{{\color{white}-}}{v})\,\right) \, = \; (\lambda+1)\cdot\rho(J_{+})(v)\,
%	\quad\textnormal{and}\quad\;
%	\rho(J_{3})\!\left(\,\rho(J_{-})(\overset{{\color{white}-}}{v})\,\right) \, = \; (\lambda-1)\cdot\rho(J_{-})(v)
%	\end{equation*}
%\item
%	\textbf{Casimir operator:}\;\;
%	Define
%	\,$J^{2}$
%	\,$:=$\,
%	$(J_{1})^{2} + (J_{2})^{2} + (J_{3})^{2}$
%	\,$\in$\
%	 $\mathcal{U}\!\left(\mathfrak{so}(3) \overset{{\color{white}.}}{\otimes_{\Re}} \C\right)$.
%	Then,
%	\begin{equation*}
%	\left[\,J^{2}\,\overset{{\color{white}1}}{,}\,J_{k}\,\right]
%	\;\; = \;\;
%		0\,,
%	\quad
%	\textnormal{for each \,$k \in \{\,1,2,3\,\}$}\,.
%	\end{equation*}
%	Consequently (by Schur's Lemma, Corollary 4.30, \cite{Hall2015}), 
%	\,$J^{2} \in \mathcal{U}\!\left(\mathfrak{so}(3) \overset{{\color{white}.}}{\otimes_{\Re}} \C\right)$\,
%	acts as a scalar multiple of the identity in every irreducible
%	representation\footnote{Furthermore, this scalar $\lambda \in \C$ uniquely determines
%	the irreducible representation.
%	Look up the classification theory of irreducible finite-dimensional complex representations
%	of complex semisimple Lie algebras.
%	Key words: Casimir operator, universal enveloping algebra. See Chapters 9 and 10, \cite{Hall2015}.}
%	of \,$\mathcal{U}\!\left(\mathfrak{so}(3) \overset{{\color{white}.}}{\otimes_{\Re}} \C\right)$;\,
%	more precisely, for each irreducible finite-dimensional complex representation
%	\,$\rho : \mathcal{U}\!\left(\mathfrak{so}(3) \overset{{\color{white}.}}{\otimes_{\Re}} \C\right) \longrightarrow \mathfrak{gl}(V)$,\,
%	we have \,$\rho(J^{2}) = \lambda \cdot \textnormal{\textbf{1}}_{V}$,\,
%	for some \,$\lambda \in \C$.
%\end{enumerate}
%\end{proposition}
%
%          %%%%% ~~~~~~~~~~~~~~~~~~~~ %%%%%
%
%\begin{theorem}
%{\color{white}.}\vskip -0.1cm
%\noindent
%\begin{enumerate}
%\item
%	The finite-dimensional irreducible representations of $\mathfrak{so}(3) \otimes_{\Re} \C$ is parametrized by the set
%	\begin{equation*}
%	\dfrac{1}{2} \cdot \Z
%	\;\; := \;\;
%		\left\{\;0 \,,\, \dfrac{1}{2} \,,\, 1 \,,\, \frac{3}{2} \,,\, 2 \,,\, \frac{5}{2} \,,\, \ldots \;\right\},
%	\end{equation*}
%	of non-negative integer multiples of \,$\dfrac{1}{2}$, in that, for each
%	$s \in \dfrac{1}{2} \cdot \Z = \left\{\; 0 \,,\, \frac{1}{2}\,,\, 1\,,\, \frac{3}{2}\,,\, 2\,,\, \frac{5}{2}\,,\, \ldots \;\right\}$,
%	there exists a unique (up to equivalence) complex representation
%	$\rho_{s} : \mathcal{U}(\mathfrak{so}(3)\otimes_{\Re}\C) \longrightarrow \textnormal{End}(V_{s})$
%	such that
%	\begin{equation*}
%	\rho_{s}(J^{2}) \; = \; s(s+1)\cdot\textnormal{\textbf{1}}_{V_{s}}.
%	\end{equation*}
%\item
%	$\dim_{\C}(V_{s}) \, = \, 2s + 1$,\, for each
%	\,$s \in \dfrac{1}{2} \cdot \Z = \left\{\; 0 \,,\, \frac{1}{2}\,,\, 1\,,\, \frac{3}{2}\,,\, 2\,,\, \frac{5}{2}\,,\, \ldots \;\right\}$.
%\item
%	For each
%	\,$s \in \dfrac{1}{2} \cdot \Z = \left\{\; 0 \,,\, \frac{1}{2}\,,\, 1\,,\, \frac{3}{2}\,,\, 2\,,\, \frac{5}{2}\,,\, \ldots \;\right\}$,\,
%	the spectrum
%	$\sigma\!\left(\,\overset{{\color{white}-}}{\rho}_{s}(J_{3})\,\right)$
%	of the operator $\rho_{s}(J_{3}) \in \textnormal{End}(V_{s})$
%	consists of only eigenvalues and is given by:
%	\begin{equation*}
%	\sigma\!\left(\,\overset{{\color{white}-}}{\rho}_{s}(J_{3})\,\right)
%	\;\; = \;\;
%		\left\{\;
%			-\overset{{\color{white}-}}{s} \,,\, -(s-1), -(s-2)
%			\,,\;\, \ldots \,\;,\,
%			(s-2) \,,\, (s-1) \,,\, s
%			\;\right\},
%	\end{equation*}
%	and each eigenvalue in 
%	$\sigma\!\left(\,\overset{{\color{white}-}}{\rho}_{s}(J_{3})\,\right)$
%	has multiplicity one.
%\item
%	For each
%	\,$s \in \dfrac{1}{2} \cdot \Z = \left\{\; 0 \,,\, \frac{1}{2}\,,\, 1\,,\, \frac{3}{2}\,,\, 2\,,\, \frac{5}{2}\,,\, \ldots \;\right\}$,\,
%	let \,$v^{(s)}_{k} \in V_{s}\backslash\{0\}$\, be any normalized eigenvector
%	of $\rho_{s}(J_{3})$ corresponding to the eigenvalue
%	\,$k$ $\in$ $\sigma\!\left(\,\overset{{\color{white}-}}{\rho}_{s}(J_{3})\,\right)$
%	$=$ $\left\{\;-\overset{{\color{white}-}}{s} \,,\, -(s-1) \,,\, \;\ldots\;,\, (s-1) \,,\, s\;\right\}$.\,
%	Then, 
%	\begin{enumerate}
%	\item
%		the eigenvectors
%		\,$v^{(s)}_{-s} \,,\, v^{(s)}_{-(s-1)} \,,\; \ldots \;,\, v^{(s)}_{s-1} \,,\, v^{(s)}_{s}$\,
%		form an orthonormal basis for $V_{s}$, and
%	\item
%		for each \,$k$ $\in$ $\sigma\!\left(\,\overset{{\color{white}-}}{\rho}_{s}(J_{3})\,\right)$
%		$=$ $\left\{\;-\overset{{\color{white}-}}{s} \,,\, -(s-1) \,,\, \;\ldots\;,\, (s-1) \,,\, s\;\right\}$,\,
%		we have:
%		\begin{equation*}
%		J_{\pm}\!\left(\,v^{(s)}_{k}\,\right)
%		\; = \;
%			\sqrt{{\color{white}.}
%			s(s+1) - k(k \pm 1)
%			{\color{white}.}}
%			\,\cdot\,
%			v^{(s)}_{k \pm 1}
%		\end{equation*}
%		In particular, \,$J_{\pm}\!\left(\,v^{(s)}_{\pm s}\,\right) \; = \; 0$.
%	\end{enumerate}
%\end{enumerate}
%\end{theorem}
%
%          %%%%% ~~~~~~~~~~~~~~~~~~~~ %%%%%
%
