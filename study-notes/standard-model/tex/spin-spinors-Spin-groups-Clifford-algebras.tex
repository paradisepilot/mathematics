
          %%%%% ~~~~~~~~~~~~~~~~~~~~ %%%%%

\chapter{Spin, spinors, Spin groups and Clifford algebras}
\setcounter{theorem}{0}
\setcounter{equation}{0}

%\cite{vanDerVaart1996}
%\cite{Kosorok2008}

%\renewcommand{\theenumi}{\alph{enumi}}
%\renewcommand{\labelenumi}{\textnormal{(\theenumi)}$\;\;$}
\renewcommand{\theenumi}{\roman{enumi}}
\renewcommand{\labelenumi}{\textnormal{(\theenumi)}$\;\;$}

          %%%%% ~~~~~~~~~~~~~~~~~~~~ %%%%%

          %%%%% ~~~~~~~~~~~~~~~~~~~~ %%%%%

\section{Liftings of unitary projective representations of the Poincaré group $\SOup(1,3) \ltimes \Re^{1,3}$
to unitary projections of $\SL(2,\C) \ltimes \Re^{1,3}$}

          %%%%% ~~~~~~~~~~~~~~~~~~~~ %%%%%

\begin{definition}[Strongly continuous unitary representation, \cite{Simms1971}]
\mbox{}
\vskip 0.01cm
\noindent
A \textbf{strongly continuous unitary representation}
of a topological group \,$G$\, on a complex Hilbert space \,$H$\,
is a homomorphism \,$\rho : G \xrightarrow{{\color{white}22}} \mathcal{U}(H)$\,
such that, for each fixed \,$v \in H$,\, the map
\,$G \xrightarrow{{\color{white}222}} H : g \xmapsto{{\color{white}222}} \rho(g) \cdot v$ \,
is a continuous.
Here, \,$\mathcal{U}(H)$\, is the group of unitary automorphisms from \,$H$\, onto itself.
\end{definition}

\vskip 0.5cm
\begin{definition}[Strongly continuous unitary {\color{red}projective} representation, \cite{Simms1971}]
\mbox{}
\vskip 0.01cm
\noindent
A \textbf{strongly continuous unitary projective representation}
of a topological group \,$G$\, on a complex Hilbert space \,$H$\,
is a continuous homomorphism of the form
\begin{equation*}
\rho : G \xrightarrow{{\color{white}222}} \Aut(\mathbb{P}(H)) \,=\, \mathcal{U}(H) / \C^{*}\,,
\end{equation*}
where \,$\mathcal{U}(H)$\, is the group of unitary automorphisms from \,$H$\, onto itself.
A unitary projective representation
\,$\rho : G \xrightarrow{{\color{white}222}} \Aut(\mathbb{P}(H))$\,
is said to be \textbf{induced} by a unitary representation
\,$\widetilde{\rho} : G \xrightarrow{{\color{white}222}} \mathcal{U}(H)$\,
if \,$\rho = \pi \circ \widetilde{\rho}$,\,
where
\,$\pi : \mathcal{U}(H) \xrightarrow{{\color{white}222}} \Aut(\mathbb{P}(H)) = \mathcal{U}(H) / \C^{*}$\,
is the quotient map.
\end{definition}

\vskip 0.5cm
\begin{theorem}[Schur’s lemma for unitary representations, Theorem 3.12, p.55, \cite{Sepanski2010}]
\mbox{}
\vskip 0.01cm
\noindent
Let \,$G$\, be a Lie group and let
\,$\rho_{1} : G \xrightarrow{{\color{white}222}} \mathcal{U}(H_{1})$\,
and
\,$\rho_{2} : G \xrightarrow{{\color{white}222}} \mathcal{U}(H_{2})$\,
be continuous irreducible unitary representations of 
\,$G$\, on a (possibly infinite-dimensional) complex Hilbert spaces \,$H_{1}$\, and $H_{2}$.\,
Then, the following statements are true:
\begin{enumerate}
\item
	If \,$\rho_{1}$\, and \,$\rho_{2}$\, are irreducible, then
	\begin{equation*}
	\dim_{\C}\!\left(\,\overset{{\color{white}.}}{\textnormal{Hom}_{G}(H_{1},H_{2})}\,\right)
	\;\; = \;\;
	\left\{\begin{array}{cl}
	1\,, & \textnormal{if \,$\rho_{1}$\, and \,$\rho_{2}$\, are equivalent}
	\\
	0\,, & \textnormal{otherwise}
	\end{array}\right.
	\end{equation*}
	where
	\begin{equation*}
	\textnormal{Hom}_{G}(H_{1},H_{2})
	\;\; := \;\;
		\left\{\;\,
			T \in \textnormal{Hom}(H_{1},H_{2})
			\;\,\left\vert\;
			\begin{array}{c}
			T \,\circ\, \rho_{1}(g) \,=\, \rho_{2}(g) \,\circ\, T
			\\
			\overset{{\color{white}.}}{\textnormal{for each \,$g \in G$}}
			\end{array}
			\right.
			\right\}
	\end{equation*}
	and \,$\textnormal{Hom}(H_{1},H_{2})$\, is the set of continuous (equivalently, bounded)
	linear maps from \,$H_{1}$\, into \,$H_{2}$.\,
\item
	\,$\rho_{1} : G \xrightarrow{{\color{white}222}} \mathcal{U}(H_{1})$\,
	is irreducible if and only if
	\,$\textnormal{Hom}_{G}(H_{1},H_{1}) \,=\, \C \cdot \mathbf{1}_{H_{1}}$.\,
\end{enumerate}
\end{theorem}

\vskip 0.5cm
\begin{corollary}
\mbox{}
\vskip 0.01cm
\noindent
Let \,$G$\, be a Lie group and let
\,$\rho : G \xrightarrow{{\color{white}222}} \mathcal{U}(H)$\,
be a continuous irreducible unitary representation of 
\,$G$\, on a (possibly infinite-dimensional) complex Hilbert space \,$H$.
Then, every bounded linear operator 
\,$T \in \textnormal{Hom}(H,H)$\,
that commutes with \,$\rho(g)$,\, for each \,$g \in G$,\,
is a scalar multiple of the identity, i.e.,
\begin{equation*}
\left.\begin{array}{c}
T \,\circ\, \rho(g) \;=\; \rho(g) \,\circ\, T
\\
\overset{{\color{white}.}}{\textnormal{for each \,$g \in G$}}
\end{array}\right\}
\quad
\Longrightarrow
\quad
\left\{\begin{array}{c}
T \,=\, \lambda \cdot \mathbf{1}_{H}
\\
\overset{{\color{white}.}}{\textnormal{for some \,$\lambda \in \C$}}
\end{array}\right.
\end{equation*}
\end{corollary}

\vskip 0.5cm
\begin{corollary}
\mbox{}
\vskip 0.01cm
\noindent
Let \,$G$\, be a Lie group and
\,$\rho : G \xrightarrow{{\color{white}222}} \mathcal{U}(H)$\,
a strongly continuous unitary irreducible representation of
\,$G$\, on a (possibly infinite-dimensional) complex Hilbert space \,$H$.\,
Then, every element of the center 
\,$Z(G)$\, acts as a scalar multiple of the identity on \,$H$,\, i.e.,
for each \,$z \in Z(G)$,\, there exists \,$\lambda(z) \in \C$\, such that
\begin{equation*}
\rho(z) \;\;=\;\; \lambda(z) \cdot \mathbf{1}_{H}
\end{equation*}
\end{corollary}
\proof

\qed

\vskip 0.5cm
\begin{proposition}
\textnormal{\bf(Unitary representations of $\SL(2,\C)\ltimes\Re^{4}$ indcues unitary projective representations of the Poincaré group $\SOup(1,3) \ltimes \Re^{1,3}$)}
\mbox{}
\vskip -0.01cm
\noindent
Every unitary representation
\,$\rho : \SL(2,\C) \ltimes \Re^{1,3} \xrightarrow{{\color{white}222}} \mathcal{U}(H)$\,
induces a unitary projective representation of
\,$\rho^{\flat} : \SOup(1,3) \ltimes \Re^{1,3} \xrightarrow{{\color{white}222}} \Aut(\mathbb{P}(H))$.\,
Furthermore, \,$\rho$\, is irreducible if and only if \,$\rho^{\flat}$\, is irreducible.
\end{proposition}
\proof
Write
\,$\widetilde{G} := \SL(2,\C) \ltimes \Re^{1,3}$\,
and 
\,$G := \SOup(1,3) \ltimes \Re^{1,3}$.\,
Let
\,$\pi : \SL(2,\C) \ltimes \Re^{1,3} \xrightarrow{{\color{white}222}} \SOup(1,3) \ltimes \Re^{1,3}$\,
be the quotient map.
Recall that
\,$\ker(\pi)
\,=\,
	\left\{\,\overset{{\color{white}.}}{(\,\pm\,I_{2 \times 2}\,,\,0_{\Re^{1,3}}\,)}\,\right\}
$\,
and that \,$\ker(\pi)$\, equals the centre of \,$\widetilde{G}$.\,

\vskip 0.3cm
\noindent
\textbf{Claim 1:}\quad
Every unitary representation
\,$\rho : \SL(2,\C) \ltimes \Re^{1,3} \xrightarrow{{\color{white}222}} \mathcal{U}(H)$\,
induces a unitary projective representation
\,$\rho^{\flat} : \SOup(1,3) \ltimes \Re^{1,3} \xrightarrow{{\color{white}222}} \Aut(\mathbb{P}(H))$\,
as follows:
For each \,$[g] \in G$\, and \,$[\,v\,] \in \mathbb{P}(H)$,\, we define
\begin{equation*}
\rho^{\flat}(\,[g]\,) \cdot [\,v\,]
\;\; := \;\;
	\left[\,\overset{{\color{white}.}}{\rho(g) \cdot v}\,\right],
\quad
\textnormal{for any \,$g \in [g]$\, and \,$v \in [\,v\,]$}
\end{equation*}
Proof of Claim 1:\;\;
We first prove that the map is well-defined.
Recall that, for any \,$[g] \in G$,\, we have
\,$[g] = g \cdot \ker(\pi) \in G = \widetilde{G} \,/ \ker(\pi)$,\,
where \,$g \in [g]$\, is an arbitrary element in the coset \,$[g] = g \cdot \ker(\pi)$.\,
So, for any \,$g, g_{1} \in [g] \in G$,\, there exists \,$k \in \ker(\pi)$\, such that
\,$g_{1} = g \cdot k$.\,
Similarly, for any $v, v_{1} \in [\,v\,]$,\, there exists \,$\alpha \in \C$\, such that
\,$v_{1} = \alpha \cdot v$.\,
Hence,
\begin{eqnarray*}
\left[\,\overset{{\color{white}.}}{\rho(g_{1}) \cdot v_{1}}\,\right]
& = &
	\left[\,\overset{{\color{white}.}}{\rho(g \cdot k) \cdot (\alpha \cdot v)}\,\right]
\;\; = \;\;
	\left[\,\overset{{\color{white}.}}{
		\rho(g) \cdot \rho(k) \cdot (\alpha \cdot v)
		}\,\right]
\;\; = \;\;
	\left[\,\overset{{\color{white}.}}{
		\lambda_{k} \cdot \alpha \cdot \rho(g) \cdot v
		}\,\right]
\;\; = \;\;
	\left[\,\overset{{\color{white}.}}{
		\rho(g) \cdot v
		}\,\right],
\end{eqnarray*}
where \,$\rho(k) = \lambda_{k}\cdot\mathbf{1}_{H}$.\,
This proves that
\,$\rho^{\flat}(\,[g]\,) \cdot [\,v\,]
\, := \,
\left[\,\overset{{\color{white}.}}{\rho(g) \cdot v}\,\right]
$\,
does not depend on the particular representatives of \,$[g]$\, and \,$[\,v\,]$;\,
hence, the map
\,$\rho^{\flat} : \SOup(1,3) \ltimes \Re^{1,3} \xrightarrow{{\color{white}222}} \Aut(\mathbb{P}(H))$\,
is well-defined.

\vskip 0.3cm
\noindent
\textbf{Claim 2:}\quad
$\rho$\, is irreducible if and only if \,$\rho^{\flat}$\, is irreducible.
\vskip 0.1cm
\noindent
Proof of Claim 2:\;\;
This follows immediately from the observation that a non-trivial closed subspace
\,$V \subset H$\,
is \,$\rho$-invariant if and only if its projectivization
\,$\mathbb{P}(V) \subset \mathbb{P}(H)$\,
is \,$\rho^{\flat}$-invariant.
\qed

\vskip 0.5cm
\begin{lemma}
\textnormal{\bf(Unitary projective representations of the Poincaré group $\SOup(1,3) \ltimes \Re^{1,3}$
induces unitary projective representations of $\SL(2,\C)\ltimes\Re^{4}$)}
\mbox{}
\vskip -0.01cm
\noindent
Every unitary projective representation
\,$\rho : \SOup(1,3) \ltimes \Re^{1,3} \xrightarrow{{\color{white}222}} \Aut(\mathbb{P}(H))$\,
induces a unitary representation
\,$\rho^{\sharp} : \SL(2,\C) \ltimes \Re^{1,3} \xrightarrow{{\color{white}222}} \Aut(\mathbb{P}(H))$.\,
Furthermore, \,$\rho$\, is irreducible if and only if \,$\rho^{\sharp}$\, is irreducible.
\end{lemma}
\proof
Simply define
\begin{equation*}
\rho^{\sharp}
\;\; := \;\;
	\rho \,\circ\, \pi :
	\SL(2,\C) \ltimes \Re^{1,3}
	\xrightarrow{{\color{white}22}\pi{\color{white}22}}
	\SOup(1,3) \ltimes \Re^{1,3}
	\xrightarrow{{\color{white}22}\rho{\color{white}22}}
	\Aut(\mathbb{P}(H))\,,
\end{equation*}
where
\,$\pi : \SL(2,\C) \ltimes \Re^{1,3} \xrightarrow{{\color{white}222}} \SOup(1,3) \ltimes \Re^{1,3}$\,
is the quotient map.
Next, note that a non-trivial closed linear subspace \,$\mathbb{P}(V) \subset \mathbb{P}(H)$\,
is $\rho$-invariant if and only if it is $\rho^{\sharp}$-invariant.
It follows that \,$\rho$\, is invariant if and only if it is \,$\rho^{\sharp}$-invariant.\,
\qed

\vskip 0.5cm
\begin{theorem}
\textnormal{\bf(Unitary projective representations of $\SL(2,\C)\ltimes\Re^{4}$ lift to ordinary representations)}
\mbox{}
\vskip -0.01cm
\noindent
Let \,$H$\, be complex Hilbert space.
Every unitary projective representation
\,$\rho : \SL(2,\C) \ltimes \Re^{4} \xrightarrow{{\color{white}222}} \Aut(\mathbb{P}(H))$\,
lifts to a unique unitary representation
\,$\widetilde{\rho} : \SL(2,\C) \ltimes \Re^{4} \xrightarrow{{\color{white}222}} \mathcal{U}(H)$.\,
\begin{equation*}
\begin{tikzcd}
&&&
\mathcal{U}(H)
	\arrow[dd, thick, two heads, swap]
\\ \\
\SL(2,\C) \ltimes \Re^{4}
	\arrow[uurrr, red, thick, dashed, "\widetilde{\rho}"]
	\arrow[rrr, thick, "\overset{{\color{white}.}}{\rho}", swap]
&&&
\Aut(\mathbb{P}(H))
\end{tikzcd}
\end{equation*}
Moreover, \,$\rho$\, is irreducible if and only if \,$\widetilde{\rho}$\, is irreducible.
\end{theorem}

\vskip 0.5cm
\begin{corollary}
\textnormal{\bf(Unitary projective representations of the Poincaré group $\SOup(1,3)\ltimes\Re^{4}$ lift to ordinary representations of $\SL(2,\C)\ltimes\Re^{4}$)}
\mbox{}
\vskip -0.01cm
\noindent
Let \,$H$\, be a complex Hilbert space and
\,$\rho : \SOup(1,3) \ltimes \Re^{4} \xrightarrow{{\color{white}222}} \Aut(\mathbb{P}(H))$\,
a unitary projective representation of the connected Poincaré group \,$\SOup(1,3) \ltimes \Re^{4}$\,
on \,$H$.\,
Then, there exists an ordinary representation
\,$\widetilde{\rho} : \SL(2,\C) \ltimes \Re^{4} \xrightarrow{{\color{white}222}} \mathcal{U}(H)$\,
such that the following diagram commutes:
\begin{equation*}
\begin{tikzcd}
\SL(2,\C) \ltimes \Re^{4}
	\arrow[dd, thick, two heads, "\pi", swap]
	\arrow[rrr, red, thick, dashed, "\widetilde{\rho}"]
&&&
\mathcal{U}(H)
	\arrow[dd, thick, two heads, swap]
\\ \\
\SOup(1,3) \ltimes \Re^{4}
	\arrow[rrr, thick, "\overset{{\color{white}.}}{\rho}", swap]
&&&
\Aut(\mathbb{P}(H))
\end{tikzcd}
\end{equation*}
Moreover, \,$\rho$\, is irreducible if and only if \,$\widetilde{\rho}$\, is irreducible.
\end{corollary}

\vskip 0.5cm
\begin{definition}[Projective representation]
\mbox{}
\vskip 0.1cm
\noindent
A \textbf{projective representation} of a group \,$G$\, is a group homomorphism
\begin{equation*}
G \xrightarrow{{\color{white}222}} \textnormal{GL}(V)/\mathbb{F}^{*}
\end{equation*}
where \,$V$\, is a vector space over a field \,$\mathbb{F}$\, and
\,$\mathbb{F}^{*} \,=\, \mathbb{F}\,\backslash\{\,0\,\}$.\,
\end{definition}

          %%%%% ~~~~~~~~~~~~~~~~~~~~ %%%%%

          %%%%% ~~~~~~~~~~~~~~~~~~~~ %%%%%

