
          %%%%% ~~~~~~~~~~~~~~~~~~~~ %%%%%

\section{Energy-momentum vector of a material particle in special relativity}
\setcounter{theorem}{0}
\setcounter{equation}{0}

%\cite{vanDerVaart1996}
%\cite{Kosorok2008}

%\renewcommand{\theenumi}{\alph{enumi}}
%\renewcommand{\labelenumi}{\textnormal{(\theenumi)}$\;\;$}
\renewcommand{\theenumi}{\roman{enumi}}
\renewcommand{\labelenumi}{\textnormal{(\theenumi)}$\;\;$}

          %%%%% ~~~~~~~~~~~~~~~~~~~~ %%%%%

\begin{definition}
\mbox{}
\vskip -0.01cm
\noindent
A \,\textbf{spacetime} is a
connected time-oriented four-dimensional
Lorentz manifold.
\end{definition}

\begin{remark}
\mbox{}
\vskip -0.01cm
\noindent
Recall that a Lorentz manifold is a pseudo-Riemannian manifold of signature $(-,+,\cdots,+)$.
\end{remark}

          %%%%% ~~~~~~~~~~~~~~~~~~~~ %%%%%

\vskip 0.5cm
\begin{definition}
\mbox{}
\vskip -0.01cm
\noindent
A \,\textbf{material particle}\, in Minkowski spacetime \,$\Re^{1,3}$\, is a differentiable map
\,$\alpha : I \longrightarrow \Re^{1,3}$\,
where \,$I \subset \Re$\, is an interval such that
\begin{itemize}
\item
	the tangent vector \,$\alpha^{\prime} : I \longrightarrow T(\Re^{1,3})$\,
	is everywhere time-like and future-pointing, and
\item
	$\Vert\,\alpha^{\prime}(\tau)\,\Vert = 1$, for each \,$\tau \in I$. 
\end{itemize}
The parameter \,$\tau \in I$\, is called the Lorentz proper time of the particle \,$\alpha$.
\end{definition}

\begin{remark}
\mbox{}
\vskip -0.01cm
\noindent
An \,\textbf{observer}\, in \,$\Re^{1,3}$\, is just a material particle;
this alternative terminology suggests a different role.
\end{remark}

\begin{definition}
\mbox{}
\vskip -0.01cm
\noindent
A material particle
\,$\alpha : I \longrightarrow \Re^{1,3}$\,
is said to be \,\textbf{freely falling}\,
if \,$\alpha$\, is a geodesic.
\end{definition}

          %%%%% ~~~~~~~~~~~~~~~~~~~~ %%%%%

\vskip 0.5cm
\begin{proposition}
\mbox{}
\vskip -0.01cm
\noindent
Suppose:
\begin{itemize}
\item
	\,$\omega : \Re \longrightarrow \Re^{1,3}$\, is the freely falling observer
	corresponding to the $x^{0}$-axis of \,$\Re^{1,3}$.\,
\item
	\,$\alpha = (\alpha^{0},\alpha^{1},\alpha^{2},\alpha^{3}) : I \longrightarrow \Re^{1,3}$\, is a material particle,
	with proper time \,$\tau \in I$.
\end{itemize}
Then, the following statements are true:
\begin{enumerate}
\item
	$\dfrac{\d\alpha^{0}(\tau)}{\d\tau} \,\geq\, 1$,\, for each \,$\tau \in I$.
	Thus, there exists \,$\varphi : I \longrightarrow [\,0,\infty)$\,
	such that
	\begin{equation*}
	\dfrac{\d\alpha^{0}(\tau)}{\d\tau} \, = \, \cosh\varphi(\tau),
	\quad
	\textnormal{for each \,$\tau \in I$}.
	\end{equation*}
	Reminder: \,$\cosh : [\,0,\infty) \longrightarrow [\,1,\infty)$\, is bijective.
\item
	The $3$-velocity (as a function of $\tau$) of \;$\alpha$\, as observed by \,$\omega$\, is {\color{red}defined} to be:
	\begin{equation*}
	\left(\;
		\dfrac{\d\alpha^{1}}{\d t},
		\dfrac{\d\alpha^{2}}{\d t},
		\dfrac{\d\alpha^{3}}{\d t}
		\,\right)
	\; = \;
	\left(\;
		\dfrac{\d\alpha^{1}/\d\tau}{\d\alpha^{0}/\d\tau},
		\dfrac{\d\alpha^{2}/\d\tau}{\d\alpha^{0}/\d\tau},
		\dfrac{\d\alpha^{3}/\d\tau}{\d\alpha^{0}/\d\tau}
		\,\right),
	\end{equation*}
	where \,$t = t(\tau) = \alpha^{0}(\tau)$.\,
	The speed (as a function of $\tau$) of \,$\alpha$\, as observed by \,$\omega$\, is defined to be
	the Euclidean norm of its 3-velocity as observed by \,$\omega$.\,
	The speed of \;$\alpha$\, as observed by \,$\omega$\, is given by:
	\begin{equation*}
	\left\Vert\;\left(\;
		\dfrac{\d\alpha^{1}}{\d t},
		\dfrac{\d\alpha^{2}}{\d t},
		\dfrac{\d\alpha^{3}}{\d t}
		\,\right)\;\right\Vert
	\;\; = \;\;
	\left\Vert\;\left(\;
		\dfrac{\d\alpha^{1}/\d\tau}{\d\alpha^{0}/\d\tau},
		\dfrac{\d\alpha^{2}/\d\tau}{\d\alpha^{0}/\d\tau},
		\dfrac{\d\alpha^{3}/\d\tau}{\d\alpha^{0}/\d\tau}
		\,\right)\;\right\Vert
	\;\; = \;\;
		\tanh\varphi(\tau)
	\;\; =: \;\;
		v(\tau)
	\;\; \in \;\;
		[\,0,1)
	\end{equation*}
	Reminder: \,$v(\tau)$\, and \,$\varphi(\tau)$\, are related by:
	\,$\cosh\varphi(\tau) \; = \; \dfrac{1}{\sqrt{\,1 - v(\tau)^{2}\,}} \; \geq \; 1$.
\end{enumerate}
\end{proposition}
\proof
\begin{enumerate}
\item
	Observer that:
	\begin{equation*}
	-1
	\; = \;
		\left\langle\;\alpha^{\prime}(\tau)
		\,\overset{{\color{white}\textnormal{\Large1}}}{,}\,
		\alpha^{\prime}(\tau)\;\right\rangle
	\; = \;
		-\left(\dfrac{\d\alpha^{0}}{\d\tau}\right)^{2}
		+\left(\dfrac{\d\alpha^{1}}{\d\tau}\right)^{2}
		+\left(\dfrac{\d\alpha^{2}}{\d\tau}\right)^{2}
		+\left(\dfrac{\d\alpha^{3}}{\d\tau}\right)^{2}
	\end{equation*}
	which implies
	\begin{equation*}
	\left(\dfrac{\d\alpha^{0}}{\d\tau}\right)^{2}
	\; = \;\;
		1\,
		+\left(\dfrac{\d\alpha^{1}}{\d\tau}\right)^{2}
		+\left(\dfrac{\d\alpha^{2}}{\d\tau}\right)^{2}
		+\left(\dfrac{\d\alpha^{3}}{\d\tau}\right)^{2}
	\;\; \geq \;\;
		1
	\end{equation*}
	On the other hand, since \,$\alpha^{\prime}(\tau)$\, is future-pointing, we have:
	\begin{equation*}
	0
	\; > \;
		\left\langle\;
		\alpha^{\prime}(\tau)
		\,\overset{{\color{white}\textnormal{\Large1}}}{,}\,
		\widehat{e}_{0}
		\;\right\rangle
	\; = \;
		-\dfrac{\d\alpha^{0}}{\d\tau}
	\end{equation*}
	which implies
	\begin{equation*}
	\dfrac{\d\alpha^{0}}{\d\tau} \; > \; 0
	\end{equation*}
	Thus, we see that:
	\begin{equation*}
	\dfrac{\d\alpha^{0}}{\d\tau}
	\; = \;\;
		+\;\sqrt{\;
			1\,
			+\left(\dfrac{\d\alpha^{1}}{\d\tau}\right)^{2}
			+\left(\dfrac{\d\alpha^{2}}{\d\tau}\right)^{2}
			+\left(\dfrac{\d\alpha^{3}}{\d\tau}\right)^{2}
			\,}
	\;\; \geq \;\;
		1
	\end{equation*}
	Now, recall that \,$\cosh : [\,0,\infty) \longrightarrow [\,1,\infty)$\, is bijective;
	hence, we may define
	\,$\varphi : I \longrightarrow [\,0,\infty)$\,
	by:
	\begin{equation*}
	\varphi(\tau)
	\; := \;
		\cosh^{-1}\!\left(\,
			\dfrac{\d\alpha^{0}(\tau)}{\d\tau}
			\,\right),
	\quad
	\textnormal{for each \,$\tau \in I$}.
	\end{equation*}
\item
	Observe that:
	\begin{eqnarray*}
	\left\Vert\;\left(\;
		\dfrac{\d\alpha^{1}}{\d t},
		\dfrac{\d\alpha^{2}}{\d t},
		\dfrac{\d\alpha^{3}}{\d t}
		\,\right)\;\right\Vert^{2}
	& = &
	\left\Vert\;\left(\;
		\dfrac{\d\alpha^{1}/\d\tau}{\d\alpha^{0}/\d\tau},
		\dfrac{\d\alpha^{2}/\d\tau}{\d\alpha^{0}/\d\tau},
		\dfrac{\d\alpha^{3}/\d\tau}{\d\alpha^{0}/\d\tau}
		\,\right)\;\right\Vert^{2}
	\\
	& \overset{{\color{white}\textnormal{\Huge1}}}{=} &
		\dfrac{
			(\d\alpha^{1}/\d\tau)^{2}
			+
			(\d\alpha^{2}/\d\tau)^{2}
			+
			(\d\alpha^{3}/\d\tau)^{2}
		}{
			(\d\alpha^{0}/\d\tau)^{2}
		}
	\;\; = \;\;
		\dfrac{
			(\d\alpha^{0}/\d\tau)^{2} - 1
		}{
			(\d\alpha^{0}/\d\tau)^{2}
		}
	\\
	& \overset{{\color{white}\textnormal{\Huge1}}}{=} &
		\dfrac{
			\cosh^{2}\varphi(\tau) - 1
		}{
			\cosh^{2}\varphi(\tau)
		}
	\;\; = \;\;
		\dfrac{
			\sinh^{2}\varphi(\tau)
		}{
			\cosh^{2}\varphi(\tau)
		}
	\;\; = \;\;
		\tanh^{2}\varphi(\tau)
	\end{eqnarray*}
	Hence,
	\begin{eqnarray*}
	\left\Vert\;\left(\;
		\dfrac{\d\alpha^{1}}{\d t},
		\dfrac{\d\alpha^{2}}{\d t},
		\dfrac{\d\alpha^{3}}{\d t}
		\,\right)\;\right\Vert
	& = &
		\left\vert\;\,\overset{{\color{white}.}}{\tanh\varphi(\tau)}\;\right\vert
	\;\; = \;\;
		\tanh\varphi(\tau)
	\;\; \in \;\;
		[\,0,1)\,,
	\end{eqnarray*}
	where the last equality follows from the fact that
	\,$\tanh : [\,0,\infty) \longrightarrow [\,0,1)$\, is bijective,
	and
	\,$\varphi : I \longrightarrow [\,0,\infty)$.\,
	Lastly, since
	\,$v(\tau) \,:=\, \tanh\varphi(\tau) \,\in\, [\,0,1)$,\,
	we have:
	\begin{eqnarray*}
	1
	\;\; \leq \;\;
		\dfrac{1}{1 - v(\tau)^{2}}
	& = &
		\dfrac{1}{1 - \tanh^{2}\varphi(\tau)}
	\; = \;
		\dfrac{1}{1 - \left(\sinh^{2}\varphi(\tau)/\cosh^{2}\varphi(\tau)\right)}
	\; = \;
		\dfrac{\cosh^{2}\varphi(\tau)}{\cosh^{2}\varphi(\tau) - \sinh^{2}\varphi(\tau)}
	\\
	& \overset{{\color{white}\textnormal{\large1}}}{=} &
		\cosh^{2}\varphi(\tau)
	\end{eqnarray*}
	Recalling that
	\,$\varphi(\tau) \in [\,0,\infty)$,\,
	we see that:
	\begin{equation*}
	\cosh\varphi(\tau)
	\;\; = \;\;
		\dfrac{1}{\sqrt{\;1 - v(\tau)^{2}}}
	\;\; \geq \;\;
		1.
	\end{equation*}
\end{enumerate}
This completes the proof of the Proposition.
\qed

          %%%%% ~~~~~~~~~~~~~~~~~~~~ %%%%%

\clearpage
\begin{definition}[Energy-momentum vector of a material particle in special relativity]
\mbox{}
\vskip -0.01cm
\noindent
The \,\textbf{energy-momentum vector field}\,
of a material particle 
\,$\alpha : I \longrightarrow \Re^{1,3}$\,
of rest mass \,$m$\, is the vector field
\,$P : I \longrightarrow T(\Re^{1,3})$\,
defined by:
\,$P(\tau) \,:=\, m\cdot\alpha^{\prime}(\tau)$,\,
for each $\tau \in I$.
\end{definition}

          %%%%% ~~~~~~~~~~~~~~~~~~~~ %%%%%

\vskip 0.5cm
\begin{proposition}
\mbox{}
\vskip -0.01cm
\noindent
Suppose:
\begin{itemize}
\item
	\,$\omega : \Re \longrightarrow \Re^{1,3}$\, is the freely falling observer
	corresponding to the $x^{0}$-axis of \,$\Re^{1,3}$.\,
\item
	\,$\alpha = (\alpha^{0},\alpha^{1},\alpha^{2},\alpha^{3}) : I \longrightarrow \Re^{1,3}$\,
	is a material particle of rest mass \,$m$,\,
	with proper time \,$\tau \in I$.
\item
	\,$\varphi : I \longrightarrow [\,0,\infty)$\, and \,$v : I \longrightarrow [\,0,1)$\,
	are determined by:
	\begin{equation*}
	\cosh\varphi(\tau)
	\;\; = \;\;
		\dfrac{1}{\sqrt{\;1 - v(\tau)^{2}}}
	\;\; = \;\;
		\dfrac{\d\alpha^{0}(\tau)}{\d\tau}
	\;\; \geq \;\;
		1.
	\end{equation*}
\item
	\,$P = (P^{0},P^{1},P^{2},P^{3}) : I \longrightarrow T(\Re^{1,3})$\,
	is the energy-momentum vector field of \,$\alpha$,\,
	i.e.,
	\,$P(\tau) \,:=\, m\cdot\alpha^{\prime}(\tau)$,\,
	for each \,$\tau \in I$.
\end{itemize}
Then, the following statements are true:
\begin{enumerate}
\item
	\begin{equation*}
	P^{0}(\tau)
	\;\; = \;\;
		\dfrac{m}{\sqrt{\;1-v(\tau)^{2}}}
	\;\; = \;\;
		m \,+\, \dfrac{1}{2}\,m\,v(\tau)^{2} \,+\, O\!\left(\overset{{\color{white}.}}{v(\tau)^{4}}\right)
	\end{equation*}
\item
	\begin{equation*}
	\left(\,\overset{{\color{white}.}}{P^{1}},P^{2},P^{3}\,\right)
	\;\; = \;\;
		\dfrac{m}{\sqrt{\;1-v^{2}}}
		\cdot
		\left(\;
			\dfrac{\d\alpha^{1}}{\d t}\,,\,
			\dfrac{\d\alpha^{2}}{\d t}\,,\,
			\dfrac{\d\alpha^{3}}{\d t}
			\;\right),
	\end{equation*}
	where \,$t = t(\tau) = \alpha^{0}(\tau)$.
\end{enumerate}
\end{proposition}
\proof
Simply observe:
\begin{eqnarray*}
\left(\,\overset{{\color{white}.}}{P^{0}},P^{1},P^{2},P^{3}\,\right)
& = &
	m\cdot\left(\;
		\dfrac{\d\alpha^{0}}{\d\tau},
		\dfrac{\d\alpha^{1}}{\d\tau},
		\dfrac{\d\alpha^{2}}{\d\tau},
		\dfrac{\d\alpha^{3}}{\d\tau}
		\;\right)
\;\; = \;\;
	m\cdot\dfrac{\d\alpha^{0}}{\d\tau}\cdot
	\left(\;
		1\,,\,
		\dfrac{\d\alpha^{1}/\d\tau}{\d\alpha^{0}/\d\tau}\,,\,
		\dfrac{\d\alpha^{2}/\d\tau}{\d\alpha^{0}/\d\tau}\,,\,
		\dfrac{\d\alpha^{3}/\d\tau}{\d\alpha^{0}/\d\tau}
		\;\right)
\\
& = &
	m\cdot
	\dfrac{1}{\sqrt{\;1-v^{2}}}
	\cdot
	\left(\;
		1\,,\,
		\dfrac{\d\alpha^{1}}{\d t}\,,\,
		\dfrac{\d\alpha^{2}}{\d t}\,,\,
		\dfrac{\d\alpha^{3}}{\d t}
		\;\right)
\end{eqnarray*}
\qed


