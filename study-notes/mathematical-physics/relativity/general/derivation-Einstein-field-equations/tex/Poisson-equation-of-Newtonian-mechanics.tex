
          %%%%% ~~~~~~~~~~~~~~~~~~~~ %%%%%

\section{Poisson's equation of a conservative gravitational field expressed in terms of second-order derivatives of metric tensor components}
\setcounter{theorem}{0}
\setcounter{equation}{0}

%\cite{vanDerVaart1996}
%\cite{Kosorok2008}

%\renewcommand{\theenumi}{\alph{enumi}}
%\renewcommand{\labelenumi}{\textnormal{(\theenumi)}$\;\;$}
\renewcommand{\theenumi}{\roman{enumi}}
\renewcommand{\labelenumi}{\textnormal{(\theenumi)}$\;\;$}

          %%%%% ~~~~~~~~~~~~~~~~~~~~ %%%%%

The Gauss Law of Gravity states that,
for an arbitrary open set
\,$\Omega \subset \textnormal{domain}(\,\mathbf{g}\,) \cap \textnormal{domain}(\,\rho\,)$\,
with smooth boundary \,$\partial\,\Omega$,\,
we have:
\begin{equation*}
\int_{\partial\,\Omega}\;\mathbf{g} \cdot \d\mathbf{A}
\;\; = \;\;
	4 \pi G \cdot
	\left(\,
		\begin{array}{c}
		\textnormal{mass contained within}
		\\
		\textnormal{the closed surface \,$\partial\Omega$}
		\end{array}
		\,\right)
\;\; = \;\;
	4 \pi G \cdot
	\int_{\,\Omega}\; \rho \,\;\d V
\end{equation*}
where $\mathbf{g}$ is the gravitational field generated by a mass density  $\rho$.
By the Divergence Theorem, we have:
\begin{equation*}
\int_{\partial\,\Omega}\;\mathbf{g} \cdot \d\mathbf{A}
\;\; = \;\;
	\int_{\,\Omega}\; \nabla \cdot \mathbf{g} \,\;\d V
\end{equation*}
Hence,
\begin{equation*}
\int_{\,\Omega}\;
	\left(\,
		\nabla \cdot \mathbf{g}
		\, \overset{{\color{white}\textnormal{\small1}}}{-} \,
		4 \pi G \cdot \rho
		\,\right)
	\,\d V
\;\; = \;\;
	0
\end{equation*}
Since the open set \,$\Omega$\, is arbitrary, the differential form of Gauss' Law of Gravity follows:
\begin{equation*}
\nabla \cdot \mathbf{g} \;\; = \;\; 4 \pi G \, \rho
\end{equation*}

\vskip 0.3cm
\noindent
If the gravitational field \,$\mathbf{g}$\, is conservative, then
\,$\mathbf{g} \,=\, \nabla \phi$,\,
for some gravitational potential \,$\phi$.\,
The differential form of Gauss' Law of Gravity therefore becomes the Poisson equation:
\begin{equation}\label{eqnPoisson}
\nabla^{2}\phi \;\; = \;\; 4 \pi G \, \rho
\end{equation}

          %%%%% ~~~~~~~~~~~~~~~~~~~~ %%%%%

\begin{remark}[Gravitational potential \,$\phi$\, is related to a metric tensor component]
\mbox{}
\vskip -0.01cm
\noindent
Next, we see how the gravitational potential \,$\phi$\, is related to metric tensor components.
Suppose:
\begin{itemize}
\item
	A material particle \,$q$\, freely falling in a weak gravitational field.
\item
	The particle is observed by an inertial observer \,$\omega$,\,
	with respect to whom the speed of \,$q$\, is small.
\end{itemize}
Since the material particle \,$\alpha$\, is freely falling in the given gravitational field,
its worldline \,$x^{\mu}(\tau)$\, -- where \,$\tau$\, is the proper time of \,$q$ --
obeys the geodesic equation:
\begin{equation*}
\dfrac{\d^{2}x^{\mu}}{\d\tau^{2}}
\,+\,
\Gamma^{\mu}_{\alpha\beta}\,\dfrac{\d x^{\alpha}}{\d\tau}\,\dfrac{\d x^{\beta}}{\d\tau}
\; = \;
	0\,,
\end{equation*}
where
\begin{equation*}
\Gamma^{\mu}_{\alpha\beta}
\;\; = \;\;
	\dfrac{1}{2}\,g^{\mu\nu}\left(\;
		\dfrac{\partial g_{\alpha\gamma}}{\partial x^{\beta}}
		\,+\,
		\dfrac{\partial g_{\gamma\beta}}{\partial x^{\alpha}}
		\,-\,
		\dfrac{\partial g_{\beta\alpha}}{\partial x^{\gamma}}
		\,\right)
\end{equation*}
are the Christoffel symbols of the metric tensor \,$g_{\alpha\beta}$\, of the gravitational field.
Let \,$t$\, be the proper time of the observer \,$\omega$.\,
Recall that \,$t$\, and \,$\tau$\, are related by:
\begin{equation*}
\dfrac{\d t}{\d\tau} \; = \; \dfrac{1}{\sqrt{\;1 - v(\tau)^{2}\,}}\,,
\end{equation*}
where \,$v(\tau)$\, is the speed (expressed as a function of the proper time \,$\tau$\, of \,$q$) of $q$
with respect to \,$\omega$.\,
Since we have assumed that \,$v(\tau) \ll 1$,\, it follows that:
\begin{equation*}
\dfrac{\d x^{0}}{\d\tau} \; = \; \dfrac{\d t}{\d\tau} \; = \; \dfrac{1}{\sqrt{\;1 - v(\tau)^{2}}} \; \approx \; 1
\quad\quad
\textnormal{and}
\quad\quad
\dfrac{\d x^{k}}{\d\tau} \; = \; \dfrac{\d x^{k}}{\d t}\,\dfrac{\d t}{\d\tau} \; \approx \; \dfrac{\d x^{k}}{\d t} \; \approx \; 0\,,
\;\;
\textnormal{for \,$k = 1,2,3$}
\end{equation*}
Hence, we see that the geodesic equation can be approximated by:
\begin{equation*}
\dfrac{\d^{2}x^{\mu}}{\d t^{2}}
\;\; \approx \;\;
	-\,\Gamma^{\mu}_{00}\,,
\end{equation*}
since
\begin{eqnarray*}
\dfrac{\d^{2}x^{\mu}}{\d t^{2}}
& \approx &
	\dfrac{\d^{2}x^{\mu}}{\d\tau^{2}}
\;\; = \;\;
	-\,\Gamma^{\mu}_{\alpha\beta}\,\dfrac{\d x^{\alpha}}{\d\tau}\,\dfrac{\d x^{\beta}}{\d\tau}
\;\; = \;\;
	-\,\Gamma^{\mu}_{00}\,\dfrac{\d x^{0}}{\d\tau}\,\dfrac{\d x^{0}}{\d\tau}
	-\,\underset{(\alpha,\beta)\neq(0,0)}{\sum}
	\Gamma^{\mu}_{\alpha\beta}\,\dfrac{\d x^{\alpha}}{\d\tau}\,\dfrac{\d x^{\beta}}{\d\tau}
\\
& \approx &
	-\,\Gamma^{\mu}_{00}
\;\; = \;\;
	\dfrac{1}{2}\,g^{\mu\nu}\left(\;
		\dfrac{\partial g_{0\gamma}}{\partial x^{0}}
		\,+\,
		\dfrac{\partial g_{\gamma0}}{\partial x^{0}}
		\,-\,
		\dfrac{\partial g_{00}}{\partial x^{\gamma}}
		\,\right)
\end{eqnarray*}
\begin{equation*}
\nabla^{2}\!\left(\,\overset{{\color{white}.}}{g^{00}}\,\right) \;\; = \;\; 8 \pi G \, \rho,
\end{equation*}
\end{remark}

          %%%%% ~~~~~~~~~~~~~~~~~~~~ %%%%%

\vskip 0.3cm
\noindent
\textbf{Action Plan}
\begin{itemize}
\item
	The Poisson equation \eqref{eqnPoisson} is not tensorial and is thus inadmissble
	as a physical law within the framework of Einstein's relativistic theory of gravitation.
	We seek therefore a tensorial generalization
	\begin{equation}\label{eqnEinsteinAnsatz}
	G^{\alpha\beta} \;\; = \;\; \kappa\,T^{\alpha\beta}
	\end{equation}
	of the Poisson equation \eqref{eqnPoisson} that admits the the Poisson equation
	as low-speed, week-gravity-field limit.
	This tensorial generalization is what we are seeking, i.e.,
	the Einstein field equations.
\item
	First, note that, in going
	from the Poisson equation \eqref{eqnPoisson}
	to the ansatz \eqref{eqnEinsteinAnsatz} of the Einstein field equations,
	the (scalar) mass density \,$\rho$\, on the R.H.S. of \eqref{eqnPoisson}
	is replaced with what is called the energy-momentum tensor \,$T^{\alpha\beta}$.
\item
	It turns out that \,$T^{\alpha\beta}$\, is symmetric.
\item
	It turns out that \,$T^{\alpha\beta}$\, is also divergence-free,
	which corresponds to the conservation of energy and momentum.
\item
	On the other hand, up to first order, the gravitational potential \,$\phi$\,
	is equal to an affine function of a certain component of the metric tensor.
	Hence, the L.H.S. of \eqref{eqnPoisson} involves second-order derivatives of metric tensor components.
	Consequently, the L.H.S. of the Einstein field equation \eqref{eqnEinsteinAnsatz} is expected to involve
	the Riemann curvature tensor, as that tensor also involves second-order derivatives of metric tensor components.
\item
	Hence, we seek a suitable contraction of the Riemann curvature tensor
	which gives a symmetric divergence-free $(2,0)$-tensor.
	The Einstein tensor is one such tensor, and the Einstein tensor
	indeed turns out to lead to a theory of gravity that makes predictions that agree
	with all experimental observations so far with astounding accuracy.
\end{itemize}

          %%%%% ~~~~~~~~~~~~~~~~~~~~ %%%%%

\vskip 0.3cm
\noindent

          %%%%% ~~~~~~~~~~~~~~~~~~~~ %%%%%



