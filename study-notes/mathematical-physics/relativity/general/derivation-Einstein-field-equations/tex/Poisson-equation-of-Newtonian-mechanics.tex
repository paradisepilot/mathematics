
          %%%%% ~~~~~~~~~~~~~~~~~~~~ %%%%%

\section{Guiding example: a weak gravitational field whose Newtonian limit satisfies a special form of Poisson's equation in which second-order derivatives of metric tensor components appear}
\setcounter{theorem}{0}
\setcounter{equation}{0}

%\cite{vanDerVaart1996}
%\cite{Kosorok2008}

%\renewcommand{\theenumi}{\alph{enumi}}
%\renewcommand{\labelenumi}{\textnormal{(\theenumi)}$\;\;$}
\renewcommand{\theenumi}{\roman{enumi}}
\renewcommand{\labelenumi}{\textnormal{(\theenumi)}$\;\;$}

          %%%%% ~~~~~~~~~~~~~~~~~~~~ %%%%%

In this section, we describe a special scenario in which
second-order derivatives of metric tensor components
of a weak gravitational field appear in the Poisson equation
of the Newtonian limit of the given gravitational field.
This modified form of the Poisson equation is what will be
extended to tensorial form in order to derive the Einstein field equations.

          %%%%% ~~~~~~~~~~~~~~~~~~~~ %%%%%

\vskip 0.3cm
\noindent
We begin by recalling the Gauss Law of Gravity, which states that,
for an arbitrary open set
\,$\Omega \subset \textnormal{domain}(\,\mathbf{g}\,) \cap \textnormal{domain}(\,\rho\,)$\,
with smooth boundary \,$\partial\,\Omega$,\,
we have:
\begin{equation*}
\int_{\partial\,\Omega}\;\mathbf{g} \cdot \d\mathbf{A}
\;\; = \;\;
	4 \pi G \cdot
	\left(\,
		\begin{array}{c}
		\textnormal{mass contained within}
		\\
		\textnormal{the closed surface \,$\partial\Omega$}
		\end{array}
		\,\right)
\;\; = \;\;
	4 \pi G \cdot
	\int_{\,\Omega}\; \rho \,\;\d V
\end{equation*}
where $\mathbf{g}$ is the gravitational field generated by a mass density  $\rho$.
By the Divergence Theorem, we have:
\begin{equation*}
\int_{\partial\,\Omega}\;\mathbf{g} \cdot \d\mathbf{A}
\;\; = \;\;
	\int_{\,\Omega}\; \nabla \cdot \mathbf{g} \,\;\d V
\end{equation*}
Hence,
\begin{equation*}
\int_{\,\Omega}\;
	\left(\,
		\nabla \cdot \mathbf{g}
		\, \overset{{\color{white}\textnormal{\small1}}}{-} \,
		4 \pi G \cdot \rho
		\,\right)
	\,\d V
\;\; = \;\;
	0
\end{equation*}
Since the open set \,$\Omega$\, is arbitrary, the differential form of Gauss' Law of Gravity follows:
\begin{equation*}
\nabla \cdot \mathbf{g} \;\; = \;\; 4 \pi G \, \rho
\end{equation*}

\vskip 0.3cm
\noindent
If the gravitational field \,$\mathbf{g}$\, is conservative, then
\,$\mathbf{g} \,=\, \nabla \phi$,\,
for some gravitational potential \,$\phi$.\,
The differential form of Gauss' Law of Gravity therefore becomes the Poisson equation:
\begin{equation}\label{eqnPoisson}
\nabla^{2}\phi \;\; = \;\; 4 \pi G \, \rho
\end{equation}

          %%%%% ~~~~~~~~~~~~~~~~~~~~ %%%%%

\vskip 0.3cm
\begin{remark}[Special scenario where \,$\phi$\, in Poisson equation becomes a metric tensor component]
\label{remarkPoissonWithMetricComponent}
\mbox{}
\vskip 0.1cm
\noindent
Here, we show that, under a special scenario, the gravitational potential \,$\phi$\,
in the Poisson equation \eqref{eqnPoisson} can be replaced
with (a constant multiple of) a metric tensor component.
The resulting equation inspires us how to seek the Einstein field equations;
more precisely, the Einstein field equations will be sought as a geometric (tensorial) generalization
of that modified form of Poisson equation.
To this end, suppose:
\begin{itemize}
\item
	A material particle \,$q$\, is freely falling in a {\color{red}weak} gravitational field.
\item
	The particle \,$q$\, is observed by an observer \,$\omega$,\,
	with respect to whom the speed of \,$q$\, is small.
\item
	The gravitational field is {\color{red}conservative} and {\color{red}approximately static}
	as observed by \,$\omega$.
\end{itemize}
Since the material particle \,$q$\, is freely falling in the given gravitational field,
its worldline \,$x^{\mu}(\tau)$\, -- where \,$\tau$\, is the proper time of \,$q$ --
obeys the geodesic equation:
\begin{equation*}
\dfrac{\d^{2}x^{\mu}}{\d\tau^{2}}
\,+\,
\Gamma^{\mu}_{\alpha\beta}\,\dfrac{\d x^{\alpha}}{\d\tau}\,\dfrac{\d x^{\beta}}{\d\tau}
\; = \;
	0\,,
\end{equation*}
where
\begin{equation*}
\Gamma^{\mu}_{\alpha\beta}
\;\; = \;\;
	\dfrac{1}{2}\,g^{\mu\gamma}\left(\;
		\dfrac{\partial g_{\alpha\gamma}}{\partial x^{\beta}}
		\,+\,
		\dfrac{\partial g_{\gamma\beta}}{\partial x^{\alpha}}
		\,-\,
		\dfrac{\partial g_{\beta\alpha}}{\partial x^{\gamma}}
		\,\right)
\end{equation*}
are the Christoffel symbols of the metric tensor \,$g_{\alpha\beta}$\, of the gravitational field.
Note that, since the gravitational field is assumed to be weak, we have:
\begin{equation*}
g^{\alpha\beta} \; \approx \; \diag(+1,-1,-1,-1)
\quad\quad
\textnormal{and}
\quad\quad
g_{\alpha\beta} \; \approx \; \diag(+1,-1,-1,-1)
\end{equation*}
Furthermore, since the gravitational field is assumed to be approximately static as observed by \,$\omega$,\,
we have:
\begin{equation*}
\dfrac{\partial\,g^{\alpha\beta}}{\partial\,x^{0}} \; \approx \; 0
\end{equation*}
Now, let \,$t$\, be the proper time of the observer \,$\omega$.\,
Recall that \,$t$\, and \,$\tau$\, are related by:
\begin{equation*}
\dfrac{\d t}{\d\tau} \; = \; \dfrac{1}{\sqrt{\;1 - v(\tau)^{2}\,}}\,,
\end{equation*}
where \,$v(\tau)$\, is the speed (expressed as a function of the proper time \,$\tau$\, of \,$q$) of $q$
with respect to \,$\omega$.\,
Since we have assumed that \,$v(\tau) \ll 1$,\, it follows that:
\begin{equation*}
\dfrac{\d x^{0}}{\d\tau} \; = \; \dfrac{\d t}{\d\tau} \; = \; \dfrac{1}{\sqrt{\;1 - v(\tau)^{2}}} \; \approx \; 1
\quad\quad
\textnormal{and}
\quad\quad
\dfrac{\d x^{k}}{\d\tau} \; = \; \dfrac{\d x^{k}}{\d t}\,\dfrac{\d t}{\d\tau} \; \approx \; \dfrac{\d x^{k}}{\d t} \; \approx \; 0\,,
\;\;
\textnormal{for \,$k = 1,2,3$}
\end{equation*}
Hence, we see that the spatial components of the geodesic equation can be approximated by:
\begin{equation*}
\dfrac{\d^{2}x^{k}}{\d t^{2}}
\;\; \approx \;\;
	-\,\Gamma^{k}_{00}
\;\; \approx \;\;
	-\,\dfrac{1}{2}\,\dfrac{\partial\,g_{00}}{\partial\,x^{k}}\,,
\quad
\textnormal{for \,$k = 1,2,3$}\,,
\end{equation*}
since
\begin{eqnarray*}
\dfrac{\d^{2}x^{k}}{\d t^{2}}
& \approx &
	\dfrac{\d^{2}x^{k}}{\d\tau^{2}}
\;\; = \;\;
	-\,\Gamma^{k}_{\alpha\beta}\,\dfrac{\d x^{\alpha}}{\d\tau}\,\dfrac{\d x^{\beta}}{\d\tau}
\;\; = \;\;
	-\,\Gamma^{k}_{00}\,\dfrac{\d x^{0}}{\d\tau}\,\dfrac{\d x^{0}}{\d\tau}
	-\,\underset{(\alpha,\beta)\neq(0,0)}{\sum}
	\Gamma^{k}_{\alpha\beta}\,\dfrac{\d x^{\alpha}}{\d\tau}\,\dfrac{\d x^{\beta}}{\d\tau}
\\
& \approx &
	-\,\Gamma^{k}_{00}
\;\; = \;\;
	-\,\dfrac{1}{2}\,g^{k\gamma}\left(\;
		\dfrac{\partial\,g_{0\gamma}}{\partial\,x^{0}}
		\,+\,
		\dfrac{\partial\,g_{\gamma0}}{\partial\,x^{0}}
		\,-\,
		\dfrac{\partial\,g_{00}}{\partial\,x^{\gamma}}
		\,\right)
\\
& \approx &
	\dfrac{1}{2}\,g^{k\gamma}\,\dfrac{\partial\,g_{00}}{\partial\,x^{\gamma}}\,,
	\;\;
	\textnormal{
		since \,$g_{\alpha\beta}$\, is approximately static,
		i.e., \,$\dfrac{\partial\,g_{\alpha\beta}}{\partial\,x^{0}} \,\approx\,0$
		}
\\
& \overset{{\color{white}\textnormal{\Large1}}}{\approx} &
	\,-\, \dfrac{1}{2}\,\dfrac{\partial\,g_{00}}{\partial\,x^{k}}\,,
	\;\;
	\textnormal{
		since \,$k \in \{1,2,3\}$\, and the gravitational field is weak,
		i.e., \,$g^{\alpha\beta}\,\approx\,\diag(+1,-1,-1,-1)$
		}
\end{eqnarray*}
Since the gravitational field is assumed to be conservative as observed by \,$\omega$,\,
the gravitational field as observed by \,$\omega$\, can be given as \,$-\,\nabla\phi$,\,
where \,$\phi$\, is the gravitational potential. Hence,
\begin{eqnarray*}
-\,\nabla\phi
& = &
	\left(\!
		\begin{array}{c}
		\textnormal{gravitational}
		\\
		\textnormal{field}
		\end{array}
		\!\right)
\;\; = \;\;
	\left(\!
		\begin{array}{c}
		\textnormal{gravitational}
		\\
		\textnormal{acceleration}
		\end{array}
		\!\right)
\\
& \overset{{\color{white}\textnormal{\huge1}}}{=} &
	\left(\,
		\dfrac{\d^{2}x^{1}}{\d\,t^{2}}\,,\,
		\dfrac{\d^{2}x^{2}}{\d\,t^{2}}\,,\,
		\dfrac{\d^{2}x^{3}}{\d\,t^{2}}
		\,\right)
\;\; {\color{red}\approx} \;\;
	-\,\dfrac{1}{2}
	\left(\,
		\dfrac{\partial\,g_{00}}{\partial\,x^{1}}\,,\,
		\dfrac{\partial\,g_{00}}{\partial\,x^{2}}\,,\,
		\dfrac{\partial\,g_{00}}{\partial\,x^{3}}
		\,\right)
\\
& \overset{{\color{white}\textnormal{\LARGE1}}}{=} &
	-\,\dfrac{1}{2}\,\nabla\,g_{00}\,,
\end{eqnarray*}
which immediately implies
\begin{equation*}
\nabla\!\left(\,\phi \,-\, \dfrac{g_{00}}{2}\,\right) \;\; \approx \;\; 0\,,
\end{equation*}
and hence,
\begin{equation*}
\phi \;\; \approx \;\; \dfrac{g_{00}}{2} \,+\, \textnormal{constant}
\end{equation*}
Substituting this into the Poisson equation \eqref{eqnPoisson} yields:
\begin{equation}\label{eqnPoissonWithMetricComponent}
\nabla^{2}\!\left(\,\overset{{\color{white}.}}{g_{00}}\,\right) \;\; \approx \;\; 8 \pi G \, \rho
\end{equation}
Equation \eqref{eqnPoissonWithMetricComponent} is the promised special form
of the Poisson equation \eqref{eqnPoisson}
in which the gravitational potential \,$\phi$\, is replaced with a scalar multiple of a metric tensor component.
\end{remark}

          %%%%% ~~~~~~~~~~~~~~~~~~~~ %%%%%

\vskip 0.3cm
\noindent
\textbf{Action Plan}
\begin{itemize}
\item
	The Poisson equation \eqref{eqnPoisson} is not tensorial and is thus inadmissble
	as a physical law within the framework of Einstein's relativistic theory of gravitation.
	We seek therefore a tensorial generalization
	\begin{equation}\label{eqnEinsteinAnsatz}
	G^{\alpha\beta} \;\; = \;\; \kappa\,T^{\alpha\beta}
	\end{equation}
	of the Poisson equation \eqref{eqnPoisson} that admits the the Poisson equation
	as low-speed, week-gravity-field limit.
	This tensorial generalization is what we are seeking, i.e.,
	the Einstein field equations.
\item
	First, note that, in going
	from the Poisson equation \eqref{eqnPoisson}
	to the ansatz \eqref{eqnEinsteinAnsatz} of the Einstein field equations,
	the (scalar) mass density \,$\rho$\, on the R.H.S. of \eqref{eqnPoisson}
	is replaced with what is called the energy-momentum tensor \,$T^{\alpha\beta}$.
\item
	It turns out that \,$T^{\alpha\beta}$\, is symmetric.
\item
	It turns out that \,$T^{\alpha\beta}$\, is also divergence-free
	(which is related to the conservation of energy and momentum in special relativity,
	though it cannot be interpreted as conservation of energy-momentum in the setting of general relativity).
\item
	On the other hand, 
	under the special scenario described in Remark \ref{remarkPoissonWithMetricComponent},
	the gravitational potential \,$\phi$\, is well approximated by a scalar multiple of a metric tensor component.
	The Poisson equation \eqref{eqnPoisson} is then well approximated by \eqref{eqnPoissonWithMetricComponent}.
	Now, the L.H.S. of \eqref{eqnPoissonWithMetricComponent} involves
	second-order derivatives of metric tensor components.
	Consequently, the L.H.S. of the Einstein field equation \eqref{eqnEinsteinAnsatz} is expected to involve
	the Riemann curvature tensor, as that tensor also involves second-order derivatives of metric tensor components.
\item
	Hence, we seek a suitable contraction of the Riemann curvature tensor
	which gives a symmetric divergence-free $(2,0)$-tensor.
	The Einstein tensor is one such tensor, and the Einstein tensor
	indeed turns out to lead to a theory of gravity that makes predictions that agree
	with all experimental observations so far with astounding accuracy.
\end{itemize}

          %%%%% ~~~~~~~~~~~~~~~~~~~~ %%%%%

\vskip 0.3cm
\noindent

          %%%%% ~~~~~~~~~~~~~~~~~~~~ %%%%%



