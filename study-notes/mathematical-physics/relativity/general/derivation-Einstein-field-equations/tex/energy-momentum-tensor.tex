
          %%%%% ~~~~~~~~~~~~~~~~~~~~ %%%%%

\section{Energy-momentum tensor}
\setcounter{theorem}{0}
\setcounter{equation}{0}

%\cite{vanDerVaart1996}
%\cite{Kosorok2008}

%\renewcommand{\theenumi}{\alph{enumi}}
%\renewcommand{\labelenumi}{\textnormal{(\theenumi)}$\;\;$}
\renewcommand{\theenumi}{\roman{enumi}}
\renewcommand{\labelenumi}{\textnormal{(\theenumi)}$\;\;$}

          %%%%% ~~~~~~~~~~~~~~~~~~~~ %%%%%

\begin{equation*}
T^{\alpha\beta}
\;\; := \;\;
	\left(\,
		\begin{array}{c}
		\textnormal{``amount'' of the $x_{\alpha}$-component}
		\\
		\textnormal{of energy-momentum}
		\\
		\textnormal{within a unit hyper-cube}
		\\
		\textnormal{orthogonal to \,$\mathbf{e}_{\beta}$}
		\end{array}
		\,\right)
\end{equation*}

\begin{eqnarray*}
T^{00}
& := &
	\left(\!
		\begin{array}{c}
		\textnormal{``amount'' of the $x_{0}$-component}
		\\
		\textnormal{of energy-momentum}
		\\
		\textnormal{within a unit hyper-cube}
		\\
		\textnormal{orthogonal to \,$\mathbf{e}_{0}$}
		\end{array}
		\,\right)
\;\; = \;\;
	\left(\!
		\begin{array}{c}
		\textnormal{energy}
		\\
		\textnormal{within a unit hyper-cube}
		\\
		\textnormal{spanned by $\mathbf{e}_{1}$, $\mathbf{e}_{2}$ and $\mathbf{e}_{3}$}
		\end{array}
		\!\right)
\;\; = \;\;
	\left(\!
		\begin{array}{c}
		\textnormal{energy}
		\\
		\textnormal{density}
		\end{array}
		\!\right)
\\ \\ \\
T^{01}
& := &
	\left(\!
		\begin{array}{c}
		\textnormal{``amount'' of the $x_{0}$-component}
		\\
		\textnormal{of energy-momentum}
		\\
		\textnormal{within a unit hyper-cube}
		\\
		\textnormal{orthogonal to \,$\mathbf{e}_{1}$}
		\end{array}
		\!\right)
\;\; = \;\;
	\left(\,
		\begin{array}{c}
		\textnormal{energy}
		\\
		\textnormal{within a unit hyper-cube}
		\\
		\textnormal{spanned by $\mathbf{e}_{0}$, $\mathbf{e}_{2}$ and $\mathbf{e}_{3}$}
		\end{array}
		\,\right)
\;\; = \;\;
	\left(\!
		\begin{array}{c}
		\textnormal{flux of energy}
		\\
		\textnormal{across a unit area}
		\\
		\textnormal{spanned by $\mathbf{e}_{2}$ and $\mathbf{e}_{3}$}
		\end{array}
		\!\right)
\\ \\ \\
T^{10}
& := &
	\left(\!
		\begin{array}{c}
		\textnormal{``amount'' of the $x_{1}$-component}
		\\
		\textnormal{of energy-momentum}
		\\
		\textnormal{within a unit hyper-cube}
		\\
		\textnormal{orthogonal to \,$\mathbf{e}_{0}$}
		\end{array}
		\!\right)
\;\; = \;\;
	\left(\!
		\begin{array}{c}
		\textnormal{``amount'' of the $x_{1}$-component}
		\\
		\textnormal{of momentum}
		\\
		\textnormal{within a unit hyper-cube}
		\\
		\textnormal{spanned by $\mathbf{e}_{1}$, $\mathbf{e}_{2}$ and $\mathbf{e}_{3}$}
		\end{array}
		\!\right)
\;\; = \;\;
	\left(\!
		\begin{array}{c}
		\textnormal{density}
		\\
		\textnormal{of $x_{1}$-component}
		\\
		\textnormal{of momentum}
		\end{array}
		\!\right)
\\ \\ \\
T^{11}
& := &
	\left(\,
		\begin{array}{c}
		\textnormal{``amount'' of the $x_{1}$-component}
		\\
		\textnormal{of energy-momentum}
		\\
		\textnormal{within a unit hyper-cube}
		\\
		\textnormal{orthogonal to \,$\mathbf{e}_{1}$}
		\end{array}
		\,\right)
\;\; = \;\;
	\left(\,
		\begin{array}{c}
		\textnormal{``amount'' of the $x_{1}$-component}
		\\
		\textnormal{of momentum}
		\\
		\textnormal{within a unit hyper-cube}
		\\
		\textnormal{spanned by $\mathbf{e}_{0}$, $\mathbf{e}_{2}$ and $\mathbf{e}_{3}$}
		\end{array}
		\,\right)
\\
& = &
	\left(\,
		\begin{array}{c}
		\textnormal{``amount'' of the $x_{1}$-component}
		\\
		\textnormal{of momentum}
		\\
		\textnormal{crossing a unit area spanned by $\mathbf{e}_{2}$ and $\mathbf{e}_{3}$}
		\\
		\textnormal{in unit time ($\mathbf{e}_{0}$ direction)}
		\end{array}
		\,\right)
\;\; = \;\;
	\left(\,
		\begin{array}{c}
		\textnormal{$x_{1}$-component of}
		\\
		\textnormal{flux of momentum}
		\\
		\textnormal{across a unit area}
		\\
		\textnormal{spanned by $\mathbf{e}_{2}$ and $\mathbf{e}_{3}$}
		\end{array}
		\,\right)
\\ \\ \\
T^{12}
& := &
	\left(\!
		\begin{array}{c}
		\textnormal{``amount'' of the $x_{1}$-component}
		\\
		\textnormal{of energy-momentum}
		\\
		\textnormal{within a unit hyper-cube}
		\\
		\textnormal{orthogonal to \,$\mathbf{e}_{2}$}
		\end{array}
		\!\right)
\;\; = \;\;
	\left(\!
		\begin{array}{c}
		\textnormal{``amount'' of the $x_{1}$-component}
		\\
		\textnormal{of momentum}
		\\
		\textnormal{within a unit hyper-cube}
		\\
		\textnormal{spanned by $\mathbf{e}_{0}$, $\mathbf{e}_{1}$ and $\mathbf{e}_{3}$}
		\end{array}
		\!\right)
\\
& = &
	\left(\!
		\begin{array}{c}
		\textnormal{``amount'' of the $x_{1}$-component}
		\\
		\textnormal{of momentum}
		\\
		\textnormal{crossing a unit area spanned by $\mathbf{e}_{1}$ and $\mathbf{e}_{3}$}
		\\
		\textnormal{in unit time ($\mathbf{e}_{0}$ direction)}
		\end{array}
		\!\right)
\;\; = \;\;
	\left(\!
		\begin{array}{c}
		\textnormal{$x_{1}$-component of}
		\\
		\textnormal{flux of momentum}
		\\
		\textnormal{across a unit area}
		\\
		\textnormal{spanned by $\mathbf{e}_{1}$ and $\mathbf{e}_{3}$}
		\end{array}
		\!\right)
\end{eqnarray*}

          %%%%% ~~~~~~~~~~~~~~~~~~~~ %%%%%

\vskip 0.5cm
\begin{theorem}
\mbox{}
\vskip 0.05cm
\noindent
The energy-momentum tensor is symmetric.
\end{theorem}
\proof
\vskip 0.3cm
\noindent
\textbf{Observation:}\; $T^{01} = T^{10}$
\begin{eqnarray*}
T^{01}
& := &
	\left(\!
		\begin{array}{c}
		\textnormal{``amount'' of the $x_{0}$-component}
		\\
		\textnormal{of energy-momentum}
		\\
		\textnormal{within a unit hyper-cube}
		\\
		\textnormal{orthogonal to \,$\mathbf{e}_{1}$}
		\end{array}
		\!\right)
\;\; = \;\;
	\left(\,
		\begin{array}{c}
		\textnormal{energy}
		\\
		\textnormal{within a unit hyper-cube}
		\\
		\textnormal{spanned by $\mathbf{e}_{0}$, $\mathbf{e}_{2}$ and $\mathbf{e}_{3}$}
		\end{array}
		\,\right)
\;\; = \;\;
	\left(\!
		\begin{array}{c}
		\textnormal{flux of energy}
		\\
		\textnormal{across a unit area}
		\\
		\textnormal{spanned by $\mathbf{e}_{2}$ and $\mathbf{e}_{3}$}
		\end{array}
		\!\right)
\\
& \overset{{\color{white}\textnormal{\Huge1}}}{=} &
	\underset{*\,\rightarrow\,0}{\lim}\;\;
	\dfrac{{\color{red}\Delta E}}{{\color{red}\Delta x^{0}}\cdot\Delta x^{2}\cdot\Delta x^{3}}
\\
& \overset{{\color{white}\textnormal{\Huge1}}}{=} &
	\underset{*\,\rightarrow\,0}{\lim}\;\;
	\dfrac{{\color{red}\Delta p^{1}}}{{\color{red}\Delta x^{1}}\cdot\Delta x^{2}\cdot\Delta x^{3}}\,,
	\quad
	\textnormal{since \,$
		{\color{red}\Delta p^{1}
		\,=\,
			\left(\,\overset{{\color{white}.}}{\Delta E}\,\right) \cdot v^{1}}
		\,=\,
			\left(\,\overset{{\color{white}.}}{\Delta E}\,\right)\cdot\dfrac{\Delta x^{1}}{\Delta t}
		\,=\,
			\left(\,\overset{{\color{white}.}}{\Delta E}\,\right)\cdot\dfrac{\Delta x^{1}}{\Delta x^{0}}
		$}
\\
&=&
	\left(\!
		\begin{array}{c}
		\textnormal{density}
		\\
		\textnormal{of $x_{1}$-component}
		\\
		\textnormal{of momentum}
		\end{array}
		\!\right)
\;\; = \;\;
	\overset{{\color{white}\textnormal{\LARGE1}}}{
		\left(\!
			\begin{array}{c}
			\textnormal{``amount'' of the $x_{1}$-component}
			\\
			\textnormal{of momentum}
			\\
			\textnormal{within a unit hyper-cube}
			\\
			\textnormal{spanned by $\mathbf{e}_{1}$, $\mathbf{e}_{2}$ and $\mathbf{e}_{3}$}
			\end{array}
			\!\right)
		}
\;\; = \;\;
	\left(\!
		\begin{array}{c}
		\textnormal{``amount'' of the $x_{1}$-component}
		\\
		\textnormal{of energy-momentum}
		\\
		\textnormal{within a unit hyper-cube}
		\\
		\textnormal{orthogonal to \,$\mathbf{e}_{0}$}
		\end{array}
		\!\right)
\\
& =: &
	\overset{{\color{white}\textnormal{\large1}}}{
		T^{10}
		}
\end{eqnarray*}

\vskip 0.3cm
\noindent
\textbf{Observation:}\; $T^{12} = T^{21}$
\begin{eqnarray*}
T^{12}
& := &
	\left(\!
		\begin{array}{c}
		\textnormal{``amount'' of the $x_{1}$-component}
		\\
		\textnormal{of energy-momentum}
		\\
		\textnormal{within a unit hyper-cube}
		\\
		\textnormal{orthogonal to \,$\mathbf{e}_{2}$}
		\end{array}
		\!\right)
\;\; = \;\;
	\left(\!
		\begin{array}{c}
		\textnormal{``amount'' of the $x_{1}$-component}
		\\
		\textnormal{of momentum}
		\\
		\textnormal{within a unit hyper-cube}
		\\
		\textnormal{spanned by $\mathbf{e}_{0}$, $\mathbf{e}_{1}$ and $\mathbf{e}_{3}$}
		\end{array}
		\!\right)
\\
& = &
	\left(\!
		\begin{array}{c}
		\textnormal{``amount'' of the $x_{1}$-component}
		\\
		\textnormal{of momentum}
		\\
		\textnormal{crossing a unit area spanned by $\mathbf{e}_{1}$ and $\mathbf{e}_{3}$}
		\\
		\textnormal{in unit time ($\mathbf{e}_{0}$ direction)}
		\end{array}
		\!\right)
\;\; = \;\;
	\left(\!
		\begin{array}{c}
		\textnormal{$x_{1}$-component of}
		\\
		\textnormal{flux of momentum}
		\\
		\textnormal{across a unit area}
		\\
		\textnormal{spanned by $\mathbf{e}_{1}$ and $\mathbf{e}_{3}$}
		\end{array}
		\!\right)
\\
& \overset{{\color{white}\textnormal{\Huge1}}}{=} &
	\underset{*\,\rightarrow\,0}{\lim}\;\;
	\dfrac{{\color{red}\Delta p^{1}}}{\Delta x^{0}\cdot{\color{red}\Delta x^{1}}\cdot\Delta x^{3}}
\\
& \overset{{\color{white}\textnormal{\Huge1}}}{=} &
	\underset{*\,\rightarrow\,0}{\lim}\;\;
	\dfrac{{\color{red}\Delta p^{2}}}{\Delta x^{0}\cdot{\color{red}\Delta x^{2}}\cdot\Delta x^{3}}\,,
	\quad
	\textnormal{since \,$
		{\color{red}\Delta p^{k}
		\,=\,
			\left(\,\overset{{\color{white}.}}{\Delta E}\,\right) \cdot v^{k}}
%		\,=\,
%			\left(\,\overset{{\color{white}.}}{\Delta E}\,\right)\cdot\dfrac{\Delta x^{k}}{\Delta t}
		\,=\,
			\left(\,\overset{{\color{white}.}}{\Delta E}\,\right)\cdot\dfrac{\Delta x^{k}}{\Delta x^{0}}
		\;\; \Longrightarrow \;\;
		\dfrac{\Delta p^{1}}{\Delta x^{1}}
		\,=\,
			\dfrac{\Delta E}{\Delta x^{0}}
		\,=\,
			\dfrac{\Delta p^{2}}{\Delta x^{2}}
		$}
\\
&=&
	\overset{{\color{white}\textnormal{\LARGE1}}}{
		\left(\!
			\begin{array}{c}
			\textnormal{$x_{2}$-component of}
			\\
			\textnormal{flux of momentum}
			\\
			\textnormal{across a unit area}
			\\
			\textnormal{spanned by $\mathbf{e}_{2}$ and $\mathbf{e}_{3}$}
			\end{array}
			\!\right)
		}
\;\; = \;\;
	\cdots
\;\; = \;\;
	\left(\!
		\begin{array}{c}
		\textnormal{``amount'' of the $x_{2}$-component}
		\\
		\textnormal{of energy-momentum}
		\\
		\textnormal{within a unit hyper-cube}
		\\
		\textnormal{orthogonal to \,$\mathbf{e}_{1}$}
		\end{array}
		\!\right)
\\
& =: &
	\overset{{\color{white}\textnormal{\large1}}}{
		T^{21}
		}
\end{eqnarray*}
\qed

          %%%%% ~~~~~~~~~~~~~~~~~~~~ %%%%%

\vskip 0.5cm
\begin{theorem}
\mbox{}
\vskip 0.05cm
\noindent
The energy-momentum tensor is divergence-free, i.e.,
\begin{equation*}
\nabla_{\alpha}\,T^{\alpha\beta} \;\; = \;\; 0\,,
\quad
\textnormal{for each \,$\beta = 0,1,2,3$}\,,
\end{equation*}
where \,$\nabla_{\alpha}$\, is covariant differentiation.
\end{theorem}
\proof

\noindent
\textbf{Claim 1:}\;\;
$\partial_{\beta}\,T^{\,\alpha\beta} \,=\, T^{\,\alpha\beta}_{\;{\color{white}\alpha\beta},\,\beta} \,=\, 0$\,
holds for each freely falling observer.
\vskip 0.1cm
\noindent
Proof of Claim 1:
\vskip 0.1cm
\noindent
By the {\color{red}principle of equivalence}, all gravitational effects are locally (observationally) eliminated
for freely falling observers.
Hence, we need establish Claim 1 only in the framework of special relativity.
Now, the {\color{red}conservation of energy-momentum} states that there are no sources nor sinks
of energy-momentum.
Hence, the total flux of energy-momentum out of the (assumed smooth) three-dimensional boundary
\,$\partial\,\Omega$\, of an arbitrary finite region \,$\Omega \subset \Re^{1,3}$\, in Minkowski spacetime is zero. 
\begin{equation*}
0
\;\; = \;\;
	\left(\begin{array}{c}
		\textnormal{$\alpha$-component}
		\\
		\textnormal{of total momentum flux}
		\\
		\textnormal{out of $\Omega$}
		\end{array}\right)
\;\; = \;\;
	\int_{\,\partial\,\Omega}\;T^{\,\alpha\beta}\,n_{\beta}\;\d^{3}S
\;\; = \;\;
	\int_{\,\Omega}\;T^{\,\alpha\beta}_{\;{\color{white}\alpha\beta},\,\beta}\;\d^{4}V\,,
\end{equation*}
where \,$n_{\beta}$\, is the outward-pointing normal unit (relative to the Euclidean inner product on $\Re^{4}$)
vector field on \,$\partial\,\Omega$,\,
and the last equality follows from {\color{red}Gauss' Divergence Theorem} (applied in a four-dimensional setting).
Since \,$\Omega$\, is arbitrary, we may conclude that
\,$T^{\,\alpha\beta}_{\;{\color{white}\alpha\beta},\,\beta} \,=\, 0$.\,
This completes the proof of Claim 1.

\vskip 0.3cm
\noindent
The Theorem now follows from Claim 1,
the {\color{red}principle of general covariance} (see \S9.5, p.165, \cite{dInverno2021}),
and the {\color{red}principle of minimal gravitational coupling} (see \S9.6, p.165, \cite{dInverno2021}),
\qed

          %%%%% ~~~~~~~~~~~~~~~~~~~~ %%%%%

          %%%%% ~~~~~~~~~~~~~~~~~~~~ %%%%%

          %%%%% ~~~~~~~~~~~~~~~~~~~~ %%%%%
