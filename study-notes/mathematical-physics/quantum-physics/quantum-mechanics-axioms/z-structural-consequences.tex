
          %%%%% ~~~~~~~~~~~~~~~~~~~~ %%%%%

\section{Structural consequences of the axioms of quantum mechanics}
\setcounter{theorem}{0}
\setcounter{equation}{0}

%\cite{vanDerVaart1996}
%\cite{Kosorok2008}

%\renewcommand{\theenumi}{\alph{enumi}}
%\renewcommand{\labelenumi}{\textnormal{(\theenumi)}$\;\;$}
\renewcommand{\theenumi}{\roman{enumi}}
\renewcommand{\labelenumi}{\textnormal{(\theenumi)}$\;\;$}

          %%%%% ~~~~~~~~~~~~~~~~~~~~ %%%%%

\vskip 0.5cm
\subsection{Commutator with Hamiltonian and time-evolution of expectation values}

\begin{proposition}
\mbox{}
\vskip 0.1cm
\noindent
Suppose:
\begin{itemize}
\item
	$\H$ is a Hilbert space over $\C$,
	and $S_{1}(\H)$ is the unit sphere in $\H$.
\item
	$\widehat{H} : \D(\widehat{H}) \subset \H \longrightarrow \H$
	is a Hamiltonian operator defined on $\H$.
\item
	$\psi : [\,a,b\,) \longrightarrow S_{1}(\H)$ is solution of the Schr\"{o}dinger equation defined by $\widehat{H}$, i.e.
	$\psi$ satisfies
	\begin{equation*}
	\dfrac{\d\psi(t)}{\d t}
	\;\; = \;\;
		-\,\dfrac{\i}{\hbar}\cdot\widehat{H}\!\left[\;\overset{{\color{white}.}}{\psi(t)}\,\right]
	\end{equation*}
\item
	$A : \D(A) \subset \H \longrightarrow \H$
	is a self-adjoint operator defined on $\H$.
\end{itemize}
Then,
\begin{equation*}
\dfrac{\d}{\d t}\, E_{\psi(t)}\!\left[\;A\;\right]
\;\; = \;\;
	\dfrac{\d}{\d t} \left\langle\; \psi(t) \,, A\psi(\overset{{\color{white}.}}{t})\;\right\rangle
\;\; = \;\;
	\left\langle\;
		\psi(t)
		\,,\,
		\dfrac{1}{\i\,\hbar}\cdot\left[A,\widehat{H}\,\right]\psi(t)
		\;\right\rangle,
\quad
\textnormal{for each $t \in (\,a,b\,)$}
\end{equation*}
\end{proposition}

          %%%%% ~~~~~~~~~~~~~~~~~~~~ %%%%%

\vskip 1.0cm
\subsection{Spectral solution of the Schr\"{o}dinger equation \& eigenspaces of the Hamiltonian operator $\widehat{H}$}

Observe that the Schr\"{o}dinger equation is formally of the form:
\begin{equation*}
\dfrac{\d}{\d t}\,\psi(t) \;\; = \;\; A \cdot\psi(t)
\end{equation*}
where $A$ is a linear operator on a Hilbert space.
Consider $\psi(t)$ of the following form:
\begin{equation*}
\psi(t)
\;\; = \;\;
	\exp\!\left(\,tA\,\right)\cdot\psi(0)
\end{equation*}
where $\exp\!\left(\,tA\,\right)$ denotes the ``operator exponential''
of the operator $t A$, in the sense that it satisfies formally:
\begin{equation*}
\dfrac{\d}{\d t}\,\exp\!\left(\,tA\,\right)
\;\; = \;\;
	A \cdot \exp\!\left(\,tA\,\right)
\end{equation*}
Then, $\psi(t) = \exp(tA)\cdot\psi(0)$ gives a formal solution to the original differential equation:
\begin{equation*}
\dfrac{\d}{\d t}\,\psi(t)
\;\; = \;\;
	\dfrac{\d}{\d t}\left(\,\exp\!\left(\,tA\,\right)\overset{{\color{white}1}}{\cdot}\psi(0)\right)
\;\; = \;\;
	\left(\,\dfrac{\d}{\d t}\,\exp\!\left(\,tA\,\right)\right)\cdot\psi(0)
\;\; = \;\;
	A \cdot \exp\!\left(\,tA\,\right)\cdot\psi(0)
\;\; = \;\;
	A \cdot\psi(t)
\end{equation*}
Hence, we attempt to seek a solution to the Schr\"{o}dinger equation of the form:
\begin{equation*}
\psi(t)
\;\; = \;\;
	\exp\!\left(\,-\,\dfrac{{\color{white}.}\i\,t\cdot\widehat{H}}{\hbar}\,\right)\cdot\psi(0)
\end{equation*}
which means that we need the operator exponential
\,$\exp\!\left(\,-\,\dfrac{{\color{white}.}\i\,t\cdot\widehat{H}}{\hbar}\,\right)$\,
to make sense.
To this end, note that, if the Hilbert space $\H$ admits an orthonormal basis
$\left\{\,\overset{{\color{white}.}}{e_{k}}\,\right\}_{k \in K}$
consisting of eigenvectors of $\widehat{H}$,
with
$\widehat{H}\,e_{k} = \lambda_{k}\cdot e_{k}$,
then the operator exponential can be
defined via its action of the $e_{k}$'s (and linear extension):
\begin{equation*}
\exp\!\left(\,-\,\dfrac{{\color{white}.}\i\,t\cdot\widehat{H}}{\hbar}\,\right)\cdot e_{k}
\;\; = \;\;
	\exp\!\left(\,-\,\dfrac{{\color{white}.}\i\,t\cdot\lambda_{k}}{\hbar}\,\right)\cdot e_{k}
\end{equation*}
In practice, the Hamiltonian operator $\widehat{H}$ is often an unbounded self-adjoint operator on $\H$.
The spectral theory for {\color{red}unbounded self-adjoint} operators on Hilbert spaces is invoked to construct solutions
for the Schr\"{o}dinger equation.
More precisely,
\begin{itemize}
\item
	Find an orthonormal basis
	$\left\{\,\overset{{\color{white}.}}{e_{k}}\,\right\}_{k \in K}$
	for $\H$ consisting of eigenvectors of the Hamiltonian operator $\widehat{H}$,
	with $\widehat{H}\,e_{k} = \lambda_{k}\cdot e_{k}$.
	In other words, completely solve the eigenvector/eigenvalue problem for $\widehat{H}$:
	\begin{equation*}
	\widehat{H}\cdot\psi \;\; = \;\; \lambda\cdot\psi\,,
	\quad
	\textnormal{for \,$\lambda \in \Re$\, and \,$\psi \in \H$}.
	\end{equation*}
	The above equation is known as the \textbf{time-independent Schr\"{o}dinger equation}.
\item
	Express the initial state as a linear combination of 
	$\left\{\,\overset{{\color{white}.}}{e_{k}}\,\right\}_{k \in K}$:
	\begin{equation*}
	\psi(0)
	\;\; = \;\;
		\underset{k \in K}{\sum}\;\alpha_{k}\cdot e_{k}
	\;\; = \;\;
		\underset{k \in K}{\sum}\;\langle\,\psi(0),e_{k}\,\rangle\cdot e_{k}
	\end{equation*}
\item
	Obtain a solution via:
	\begin{eqnarray*}
	\psi(t)
	& := &
		\exp\!\left(\,-\,\dfrac{{\color{white}.}\i\,t\cdot\widehat{H}}{\hbar}\,\right)\cdot \psi(0)
	\\
	& = &
		\exp\!\left(\,-\,\dfrac{{\color{white}.}\i\,t\cdot\widehat{H}}{\hbar}\,\right)\cdot
		\left(\,\underset{k \in K}{\sum}\;\alpha_{k}\cdot e_{k}\,\right)
	\;\; = \;\;
		\underset{k \in K}{\sum}\;
		\alpha_{k}\cdot 
		\exp\!\left(\,-\,\dfrac{{\color{white}.}\i\,t\cdot\widehat{H}}{\hbar}\,\right)\cdot e_{k}
	\\
	& = &
		\underset{k \in K}{\sum}\;
		\alpha_{k}\cdot 
		\exp\!\left(\,-\,\dfrac{{\color{white}.}\i\,t\cdot\lambda_{k}}{\hbar}\,\right)\cdot e_{k}
	\end{eqnarray*}
\end{itemize}

          %%%%% ~~~~~~~~~~~~~~~~~~~~ %%%%%

\vskip 0.5cm
\subsection{Symmetries of the Hamiltonian operator $\widehat{H}$ \& entrance of representation theory}

Suppose:
\begin{itemize}
\item
	$G$ is a Lie group with Lie algebra $\mathfrak{g}$.
\item
	$G$ acts on the quantum state $\H$ via a unitary representation $G \longrightarrow \textnormal{U}(\H)$,
	where $\textnormal{U}(\H)$ denotes the group of unitary operators on $\H$.
\item
	The Hamiltonian operator $\widehat{H}$ is $G$-invariant.
\end{itemize}
Then,
\begin{itemize}
\item
	$\widehat{H}$ commutes with (the image in $\mathfrak{u}(H)$ of) every element of $\mathfrak{g}$.
\item
	Consequently, every element of $\mathfrak{g}$ preserves every eigenspace of $\widehat{H}$.
	Hence, every eigenspace of $\widehat{H}$ is a representation space of the Lie algebra $\mathfrak{g}$.
\item
	Hence, classification of the irreducible (finite-dimensional) representations of the Lie algebra $\mathfrak{g}$
	will yield valuable information about the structure of the eigenspaces of $\widehat{H}$, and
	about the solutions to the corresponding Schr\"{o}dinger equation.
\end{itemize}

          %%%%% ~~~~~~~~~~~~~~~~~~~~ %%%%%
