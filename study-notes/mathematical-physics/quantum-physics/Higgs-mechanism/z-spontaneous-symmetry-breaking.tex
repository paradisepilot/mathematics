
          %%%%% ~~~~~~~~~~~~~~~~~~~~ %%%%%

\section{Spontaneous Symmetry Breaking}
\setcounter{theorem}{0}
\setcounter{equation}{0}

%\cite{vanDerVaart1996}
%\cite{Kosorok2008}

%\renewcommand{\theenumi}{\alph{enumi}}
%\renewcommand{\labelenumi}{\textnormal{(\theenumi)}$\;\;$}
\renewcommand{\theenumi}{\roman{enumi}}
\renewcommand{\labelenumi}{\textnormal{(\theenumi)}$\;\;$}

          %%%%% ~~~~~~~~~~~~~~~~~~~~ %%%%%

Suppose:
\begin{itemize}
\item
	$G \longhookrightarrow P \longrightarrow M$\,
	is a principal fibre bundle.
	\vskip 0.01cm
	\,$\mathscr{C}\!\left(G\hookrightarrow P \rightarrow M\right)$\,
	is the collection of all connection $1$-forms on the principal fibre bundle.
\item
	$W$ is a finite-dimensional vector space over $\C$ admitting a representation
	\,$\rho : G \longrightarrow \GL(W)$,\,
	and \,$E$\, is the associated fibre bundle, i.e.
	\,$E \,=\, P \times_{\rho} W$.\,
\item
	$M$\, is equipped with a pseudo-Riemannian metric,
	with the induced volume form defined as
	\,$\d\,\vol_{M}$.\,
	\vskip 0.1cm
	The Lie algebra \,$\mathfrak{g}$\, of the structure (Lie) group \,$G$\,
	is equipped with a positive definite \,$\Ad$-invariant\, inner product
	\,$\langle\,\cdot\,,\cdot\,\rangle_{\mathfrak{g}}$\,
	(which will involve a number of coupling constants).\,
	Note that
	\,$\langle\,\cdot\,,\cdot\,\rangle_{\mathfrak{g}}$\,
	will then induce a bundle metric
	--- which we denote by \,$\langle\,\cdot\,,\cdot\,\rangle_{\Ad(P)}$\, ---
	on the associated real vector bundle
	\,$\Ad(P) \,=\, P \times_{\Ad} \mathfrak{g}$.\,
	\vskip 0.1cm
	$W$\, is equipped with a \,$\rho$-invariant Hermitian inner product
	\,$\langle\,\cdot\,,\cdot\,\rangle_{W}$,\,
	which will induce the associated bundle metric on the associated vector bundle
	\,$E \,=\, P \times_{\rho} W $.\,
	\vskip 0.1cm
\item
	$S : \mathscr{C}\!\left(G\hookrightarrow P \rightarrow M\right) \times \Gamma(E) \longrightarrow \Re$\,
	is the action functional defined as follows:
	\begin{eqnarray*}
	S\!\left(\,\omega\,,\Phi\,\right)
	& := &
		\int_{M}\left(\;
			\dfrac{1}{2}\,\langle\,F^{\omega}_{M} \,, F^{\omega}_{M}\,\rangle_{\Ad(P)}
			\;+\;
			\dfrac{1}{2}\;\langle\,\nabla^{\,\omega}\Phi \,, \nabla^{\,\omega}\Phi\,\rangle_{E}
			\;-\;
			V(\Phi)
			\,\right)
			\,\d\,\vol_{M} 
	\\
	& = &
		\int_{M}\left(\;
			\dfrac{1}{2}\; F^{\omega}_{M} \;\dot{\wedge}\star F^{\omega}_{M}
			\;+\;
			\dfrac{1}{2}\;\nabla^{\,\omega}\Phi\;\dot{\,\wedge} \star\nabla^{\,\omega}\Phi
			\;-\;
			V(\Phi)\,\d\,\vol_{M} 
			\;\right)
	\end{eqnarray*}
	where
	\begin{equation*}
	V(\Phi) \;\; = \;\; V(\vert\,\Phi\,\vert)
	\;\; = \;\;
		\dfrac{1}{2}\,\mu^{2}\,\vert\,\Phi\,\vert^{2}
		\;+\;
		\dfrac{1}{4}\,\lambda\,\vert\,\Phi\,\vert^{4}\,,
	\quad
	{\color{red}\mu^{2} < 0}\,,
	\quad
	\lambda \, > \, 0
	\end{equation*}
\end{itemize}

          %%%%% ~~~~~~~~~~~~~~~~~~~~ %%%%%

\begin{theorem}[Spontaneous Symmetry Breaking]
\mbox{}
\vskip 0.2cm
\noindent
\end{theorem}

          %%%%% ~~~~~~~~~~~~~~~~~~~~ %%%%%


          %%%%% ~~~~~~~~~~~~~~~~~~~~ %%%%%
