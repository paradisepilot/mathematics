
          %%%%% ~~~~~~~~~~~~~~~~~~~~ %%%%%

\section{Spontaneous Symmetry Breaking}
\setcounter{theorem}{0}
\setcounter{equation}{0}

%\cite{vanDerVaart1996}
%\cite{Kosorok2008}

%\renewcommand{\theenumi}{\alph{enumi}}
%\renewcommand{\labelenumi}{\textnormal{(\theenumi)}$\;\;$}
\renewcommand{\theenumi}{\roman{enumi}}
\renewcommand{\labelenumi}{\textnormal{(\theenumi)}$\;\;$}

          %%%%% ~~~~~~~~~~~~~~~~~~~~ %%%%%

Suppose:
\begin{itemize}
\item
	$G \longhookrightarrow P \longrightarrow M$\,
	is a principal fibre bundle.
\item
	$F$ is a smooth manifold on which $G$ acts on the left.
	$E$ is the associated fibre bundle, i.e. $E \,=\, \left.\left(\,P \times_{G} F\,\right)\,\right\slash G$.
\item
	\begin{equation*}
	S\!\left(\,\omega\,,\Phi\,\right)
	\;\; := \;\;
		\int_{M}\left(\;
			\dfrac{1}{2}\;\Omega\;\dot{\wedge}\star\Omega
			\;+\;
			\dfrac{1}{2}\;\nabla^{\omega}\Phi\;\dot{\wedge} \star\nabla^{\omega}\Omega
			\;-\;
			V(\Phi)\,\d\,\vol_{g} 
			\;\right)
	\end{equation*}
	where
	\begin{equation*}
	V(\Phi) \;\; = \;\; V(\vert\,\Phi\,\vert)
	\;\; = \;\;
		\dfrac{1}{2}\,\mu^{2}\,\vert\,\Phi\,\vert^{2}
		\;+\;
		\dfrac{1}{4}\,\lambda\,\vert\,\Phi\,\vert^{4}\,,
	\quad
	{\color{red}\mu^{2} < 0}\,,
	\quad
	\lambda \, > \, 0
	\end{equation*}
\end{itemize}

          %%%%% ~~~~~~~~~~~~~~~~~~~~ %%%%%

\begin{theorem}[Spontaneous Symmetry Breaking]
\mbox{}
\vskip 0.2cm
\noindent
\end{theorem}

          %%%%% ~~~~~~~~~~~~~~~~~~~~ %%%%%


          %%%%% ~~~~~~~~~~~~~~~~~~~~ %%%%%
