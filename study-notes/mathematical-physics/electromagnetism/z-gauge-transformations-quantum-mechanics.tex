
          %%%%% ~~~~~~~~~~~~~~~~~~~~ %%%%%

\section{Gauge transformations and quantum mechanics}
\setcounter{theorem}{0}
\setcounter{equation}{0}

%\cite{vanDerVaart1996}
%\cite{Kosorok2008}

%\renewcommand{\theenumi}{\alph{enumi}}
%\renewcommand{\labelenumi}{\textnormal{(\theenumi)}$\;\;$}
\renewcommand{\theenumi}{\roman{enumi}}
\renewcommand{\labelenumi}{\textnormal{(\theenumi)}$\;\;$}

          %%%%% ~~~~~~~~~~~~~~~~~~~~ %%%%%

Transformations of the Schr\"{o}dinger equation under local $U(1)$ gauge transformations:
\begin{proposition}[Zero-electromagnetic-field Schr\"{o}dinger equation is not $U(1)$-gauge-invariant]
\mbox{}
\vskip 0.2cm
\noindent
Suppose:
\begin{itemize}
\item
	$U \subset \Re^{1,3}$ is an open subset, with coordinates $(t,x_{1},x_{2},x_{3})$.
\item
	$\psi : U \longrightarrow \C$, $\Phi : U \longrightarrow \Re$ are smooth functions.
\item
	$V : U \longrightarrow \Re$ is a smooth function.
\end{itemize}
Define \,$\widetilde{\psi} : U \longrightarrow \C$\, as follows:
\begin{equation*}
\widetilde{\psi}(t,x_{1},x_{2},x_{3})
\;\; := \;\;
	\exp\!\left(\,\overset{{\color{white}.}}{\i}\cdot\Phi(t,x_{1},x_{2},x_{3})\,\right) \cdot \psi(t,x_{1},x_{2},x_{3})
\end{equation*}
Conclusion:\quad If $\psi$ satisfies the following Schr\"{o}dinger equation:
\begin{equation*}
\i\hbar \cdot \dfrac{\partial \psi}{\partial t}
\;\; = \;\;
	-\,\dfrac{\hbar^{2}}{2m}\Delta\psi \,+\, V\cdot\psi
\end{equation*}
then $\widetilde{\psi}$ satisfies the following equation:
\begin{equation*}
\i\hbar \cdot \dfrac{\partial \widetilde{\psi}}{\partial t}
\;\; = \;\;
	-\,\dfrac{\hbar^{2}}{2m}\cdot\left(\,
		\Delta\widetilde{\psi}
		-\,2\,\i\cdot(\nabla\Phi) \cdot (\nabla\widetilde{\psi})
		-\,\i\cdot(\Delta\Phi)\cdot\widetilde{\psi}
		-\,\vert\,\nabla\Phi\,\vert^{2}\cdot\widetilde{\psi}
		\,\right)
	\,+\,
	\left(\,V\,-\,\hbar\cdot\dfrac{\partial\Phi}{\partial t}\,\right)\cdot\psi
\end{equation*}
\end{proposition}

          %%%%% ~~~~~~~~~~~~~~~~~~~~ %%%%%

\vskip 1.0cm
\begin{theorem}[Four-potential Schr\"{o}dinger equation is $U(1)$-gauge-invariant]
\mbox{}
\vskip 0.2cm
\noindent
Suppose:
\begin{itemize}
\item
	$U \subset \Re^{1,3}$ is an open subset, with coordinates $(t,x_{1},x_{2},x_{3})$.
\item
	$\psi : U \longrightarrow \C$, $\Phi : U \longrightarrow \Re$ are smooth functions.
\item
	$\mathbf{A} \, = \, (A_{0}, A_{1}, A_{2}, A_{3}) : U \longrightarrow \Re^{4}$ is a smooth function.
\end{itemize}
Define \,$\widetilde{\psi} : U \longrightarrow \C$\, as follows:
\begin{equation*}
\widetilde{\psi}(t,x_{1},x_{2},x_{3})
\;\; := \;\;
	\exp\!\left(\,-\,\overset{{\color{white}.}}{\i}\,q \cdot\Phi(t,x_{1},x_{2},x_{3})\,\right) \cdot \psi(t,x_{1},x_{2},x_{3})
\end{equation*}
Conclusion:\quad If $\psi$ satisfies the following Schr\"{o}dinger equation:
\begin{equation*}
\i\hbar \cdot \dfrac{\partial \psi}{\partial t}
\;\; = \;\;
	-\,\dfrac{\hbar^{2}}{2m}\cdot\overset{3}{\underset{j=1}{\sum}}\left(\,
		\dfrac{\partial}{\partial x_{j}}
		+
		\i\,q\cdot A_{j}
		\,\right)^{\!2}
		\psi
	\,+\,
	\left(\,q \overset{{\color{white}.}}{\cdot} A_{0}\,\right)\cdot\psi
\end{equation*}
then $\widetilde{\psi}$ satisfies the following equation:
\begin{equation*}
\i\hbar \cdot \dfrac{\partial \widetilde{\psi}}{\partial t}
\;\; = \;\;
	-\,\dfrac{\hbar^{2}}{2m}\cdot\overset{3}{\underset{j=1}{\sum}}\left(\,
		\dfrac{\partial}{\partial x_{j}}
		+
		\i\,q\cdot \widetilde{A}_{j}
		\,\right)^{\!2}
		\widetilde{\psi}
	\,+\,
	\left(\,q \overset{{\color{white}.}}{\cdot} \widetilde{A}_{0}\,\right)\cdot\widetilde{\psi}\,,
\end{equation*}
where
\,$\widetilde{A} \,=\, \left(\,\widetilde{A}_{0},\widetilde{A}_{1},\widetilde{A}_{2},\widetilde{A}_{3}\,\right)$
\,$:=$\,
$A \,+\, q\cdot\d\Phi$.
\end{theorem}

          %%%%% ~~~~~~~~~~~~~~~~~~~~ %%%%%

          %%%%% ~~~~~~~~~~~~~~~~~~~~ %%%%%

          %%%%% ~~~~~~~~~~~~~~~~~~~~ %%%%%

          %%%%% ~~~~~~~~~~~~~~~~~~~~ %%%%%

          %%%%% ~~~~~~~~~~~~~~~~~~~~ %%%%%

