
          %%%%% ~~~~~~~~~~~~~~~~~~~~ %%%%%

\section{Maxwell's equations -- in terms of differential forms}
\setcounter{theorem}{0}
\setcounter{equation}{0}

%\cite{vanDerVaart1996}
%\cite{Kosorok2008}

%\renewcommand{\theenumi}{\alph{enumi}}
%\renewcommand{\labelenumi}{\textnormal{(\theenumi)}$\;\;$}
\renewcommand{\theenumi}{\roman{enumi}}
\renewcommand{\labelenumi}{\textnormal{(\theenumi)}$\;\;$}

          %%%%% ~~~~~~~~~~~~~~~~~~~~ %%%%%

\begin{theorem}[Maxwell's equations in terms of differential forms]
\label{MaxwellsEquationsDifferentialForms}
\mbox{}
\vskip 0.2cm
\noindent
Suppose
\begin{equation*}
E \; = \; E^{1}\,\dfrac{\partial}{\partial x^{1}} + E^{2}\,\dfrac{\partial}{\partial x^{2}} + E^{3}\,\dfrac{\partial}{\partial x^{3}}\,,
\quad
B \; = \; B^{1}\,\dfrac{\partial}{\partial x^{1}} + B^{2}\,\dfrac{\partial}{\partial x^{2}} + B^{3}\,\dfrac{\partial}{\partial x^{3}}\,,
\end{equation*}
\begin{equation*}
\rho\,,
\quad\textnormal{and}\quad
J \; = \; J^{1}\,\dfrac{\partial}{\partial x^{1}} + J^{2}\,\dfrac{\partial}{\partial x^{2}} + J^{3}\,\dfrac{\partial}{\partial x^{3}}
\end{equation*}
are, respectively, the electric (vector) field, magnetic (vector) field,
electric charge density (scalar) field, and electric current (vector) field
defined on $I \times U$,
where $I$ is an open interval in $\Re$, and $U$ is a domain in $\Re^{3}$.
\vskip 0.3cm
\noindent
Define:
\begin{itemize}
\item
	the electric $1$-form field:
	\begin{equation*}
	\mathbf{E} \; := \; E^{1}\,\d x^{1} \,+\, E^{2}\,\d x^{2} \,+\, E^{3}\,\d x^{3}
	\end{equation*}
\item
	the magnetic $2$-form field:
	\begin{equation*}
	\mathbf{B}
	\; := \;
		B^{1}\;\d x^{2} \wedge \d x^{3}
		\;+\;
		B^{2}\;\d x^{3} \wedge \d x^{1}
		\;+\;
		B^{3}\;\d x^{1} \wedge \d x^{2}
	\end{equation*}
\item
	the electromagnetic field strength $2$-form field:
	\begin{eqnarray*}
	\mathbf{F}
	& := &
		- \, \d x^{0} \wedge \mathbf{E} \, + \, \mathbf{B}
	\\
	& = &
		- \, E^{1}\;\d x^{0} \wedge \d x^{1} \, - \, E^{2}\;\d x^{0} \wedge \d x^{2} \, - \, E^{3}\;\d x^{0} \wedge \d x^{3}
		\;+\;
		B^{1}\;\d x^{2} \wedge \d x^{3}
		\;+\;
		B^{2}\;\d x^{3} \wedge \d x^{1}
		\;+\;
		B^{3}\;\d x^{1} \wedge \d x^{2}
	\end{eqnarray*}
	As a matrix of $2$-forms, $\mathbf{F}$ can be expressed as:
	\begin{equation*}
	\left(\,F_{\mu\nu}\overset{{\color{white}1}}{\cdot}\d x^{\mu} \wedge \d x^{\nu}\,\right)
	\;\; = \;\;
	\dfrac{1}{2}\cdot
	\left({\color{gray}\begin{array}{rrrr}
	0\cdot\d x^{0}\wedge\d x^{0} &
	{\color{red}-E^{1}} \cdot\d x^{0}\wedge\d x^{1} &
	{\color{red}-E^{2}} \cdot\d x^{0}\wedge\d x^{2} &
	{\color{red}-E^{3}} \cdot\d x^{0}\wedge\d x^{3}
	\\
	{\color{red}E^{1}} \cdot\d x^{1}\wedge\d x^{0} &
	0 \cdot\d x^{1}\wedge\d x^{1} &
	{\color{red}B^{3}} \cdot\d x^{1}\wedge\d x^{2} &
	{\color{red}-B^{2}} \cdot\d x^{1}\wedge\d x^{3}
	\\
	{\color{red}E^{2}} \cdot\d x^{2}\wedge\d x^{0} &
	{\color{red}-B^{3}} \cdot\d x^{2}\wedge\d x^{1} &
	0 \cdot\d x^{2}\wedge\d x^{2} &
	{\color{red}B^{1}} \cdot\d x^{2}\wedge\d x^{3}
	\\
	{\color{red}E^{3}} \cdot\d x^{3}\wedge\d x^{0} &
	{\color{red}B^{2}} \cdot\d x^{2}\wedge\d x^{1} &
	{\color{red}-B^{1}} \cdot\d x^{2}\wedge\d x^{2} &
	0 \cdot\d x^{3}\wedge\d x^{3}
	\end{array}}\right)
	\end{equation*}
\item
	the current $1$-form field:
	\begin{equation*}
	\mathbf{J} \; := \; \rho \, \d x^{0} \,-\, J^{1}\,\d x^{1} \,-\, J^{2}\,\d x^{2} \,-\, J^{3}\,\d x^{3}
	\end{equation*}
\end{itemize}
Then,
\begin{enumerate}
\item
	Gauss's law for magnetism (non-existence of magnetic monopoles) and Faraday's law are equivalent to the following:
	\begin{eqnarray*}
	\d\,\mathbf{F} & = & 0
	\end{eqnarray*}
\item
	Coulomb's law and Amp\`{e}re-Maxwell's law are equivalent to the following:
	\begin{eqnarray*}
	\star\;\d \left(\;\star\,\mathbf{F}\,\right) & = & 4\,\pi\cdot\mathbf{J}
	\end{eqnarray*}	
\end{enumerate}
\end{theorem}
\proof

\qed

          %%%%% ~~~~~~~~~~~~~~~~~~~~ %%%%%

          %%%%% ~~~~~~~~~~~~~~~~~~~~ %%%%%

          %%%%% ~~~~~~~~~~~~~~~~~~~~ %%%%%

          %%%%% ~~~~~~~~~~~~~~~~~~~~ %%%%%

          %%%%% ~~~~~~~~~~~~~~~~~~~~ %%%%%

