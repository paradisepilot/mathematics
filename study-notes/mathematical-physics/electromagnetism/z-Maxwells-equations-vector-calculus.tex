
          %%%%% ~~~~~~~~~~~~~~~~~~~~ %%%%%

\section{Maxwell's equations -- differential vector calculus form}
\setcounter{theorem}{0}
\setcounter{equation}{0}

%\cite{vanDerVaart1996}
%\cite{Kosorok2008}

%\renewcommand{\theenumi}{\alph{enumi}}
%\renewcommand{\labelenumi}{\textnormal{(\theenumi)}$\;\;$}
\renewcommand{\theenumi}{\roman{enumi}}
\renewcommand{\labelenumi}{\textnormal{(\theenumi)}$\;\;$}

          %%%%% ~~~~~~~~~~~~~~~~~~~~ %%%%%

Suppose
\begin{equation*}
E \; = \; E^{1}\,\dfrac{\partial}{\partial x^{1}} + E^{2}\,\dfrac{\partial}{\partial x^{2}} + E^{3}\,\dfrac{\partial}{\partial x^{3}}\,,
\quad
B \; = \; B^{1}\,\dfrac{\partial}{\partial x^{1}} + B^{2}\,\dfrac{\partial}{\partial x^{2}} + B^{3}\,\dfrac{\partial}{\partial x^{3}}\,,
\end{equation*}
\begin{equation*}
\rho\,,
\quad\textnormal{and}\quad
J \; = \; J^{1}\,\dfrac{\partial}{\partial x^{1}} + J^{2}\,\dfrac{\partial}{\partial x^{2}} + J^{3}\,\dfrac{\partial}{\partial x^{3}}
\end{equation*}
are, respectively, the electric (vector) field, magnetic (vector) field,
electric charge density (scalar) field, and electric current (vector) field
defined on $I \times U$,
where $I$ is an open interval in $\Re$, and $U$ is a domain in $\Re^{3}$.

\vskip 0.5cm
\noindent
Then, Maxwell's equations are the following:
\begin{enumerate}
\item
	Gauss's law for magnetism (non-existence of magnetic monopoles):
	\begin{equation*}
	\nabla \bullet B \;\; = \;\; 0
	\end{equation*}

\item
	Coulomb's law (conservation of electric charge):
	\begin{equation*}
	\nabla \bullet E \;\; = \;\; 4 \pi \cdot \rho
	\end{equation*}

\item
	Faraday's law of induction:
	\begin{equation*}
	\dfrac{\partial B}{\partial t} \;\; = \;\; -\,\nabla \times E
	\end{equation*}

\item
	Amp\`{e}re-Maxwell's law:
	\begin{equation*}
	\dfrac{\partial E}{\partial t} \;\; = \;\; \nabla \times B \,-\, 4\,\pi\cdot J
	\end{equation*}
\end{enumerate}

          %%%%% ~~~~~~~~~~~~~~~~~~~~ %%%%%

\vskip 0.5cm
\noindent
\begin{center}
\textbf{\large Integral form of Maxwell's equations}
\end{center}

          %%%%% ~~~~~~~~~~~~~~~~~~~~ %%%%%

\vskip 0.5cm
\noindent
\textbf{Gauss's law for magnetism (non-existence of magnetic monopoles)}
\begin{eqnarray*}
	\left(\begin{array}{c}
	\textnormal{magnetic charge}
	\\
	\textnormal{contained in}
	\\
	\textnormal{domain $U \subset \Re^{3}$}
	\end{array}\!\right)
&=&
	\left(\!\begin{array}{c}
	\textnormal{outward flux}
	\\
	\textnormal{of magnetic field}
	\\
	\textnormal{through $\partial U$}
	\end{array}\!\right)
\\
0
&=&
	\int\!\!\!\!\int_{\partial U}\, B \bullet \widehat{\mathbf{n}} \;\d S
\\
&=&
	\int\!\!\!\!\int\!\!\!\!\int_{U}\, \nabla \bullet B \;\d V\,,
	\quad
	\textnormal{by Divergence Theorem of Gauss}
\end{eqnarray*}

          %%%%% ~~~~~~~~~~~~~~~~~~~~ %%%%%

\vskip 0.5cm
\noindent
\textbf{Coulomb's law (conservation of electric charge)}
\begin{eqnarray*}
	4\,\pi\cdot\left(\begin{array}{c}
	\textnormal{electric charge}
	\\
	\textnormal{contained in}
	\\
	\textnormal{domain $U \subset \Re^{3}$}
	\end{array}\!\right)
&=&
	\left(\begin{array}{c}
	\textnormal{outward flux}
	\\
	\textnormal{of electric field}
	\\
	\textnormal{through $\partial U$}
	\end{array}\!\right)
\\
4\,\pi\cdot\int\!\!\!\!\int\!\!\!\!\int_{U}\, \rho \;\d V
&=&
	\int\!\!\!\!\int_{\partial U}\, E \bullet \widehat{\mathbf{n}} \;\d S
\\
&=&
	\int\!\!\!\!\int\!\!\!\!\int_{U}\, \nabla \bullet E \;\d V\,,
	\quad
	\textnormal{by Divergence Theorem of Gauss}
\end{eqnarray*}

          %%%%% ~~~~~~~~~~~~~~~~~~~~ %%%%%

\vskip 0.5cm
\noindent
\textbf{Faraday's law of induction}
\begin{eqnarray*}
	\left(\begin{array}{c}
	\textnormal{time rate of change}
	\\
	\textnormal{of magnetic flux}
	\\
	\textnormal{through surface $\Sigma \subset \Re^{3}$}
	\end{array}\!\right)
&=&
	-\,\left(\begin{array}{c}
	\textnormal{electric field}
	\\
	\textnormal{around close path $\partial \Sigma$}
	\end{array}\!\right)
\\
\int\!\!\!\!\int_{\Sigma}\; \dfrac{\partial B}{\partial t} \,\bullet\, \widehat{\mathbf{n}}\;\d S
\;\; = \;\;
\dfrac{\d}{\d t}\left(\,\int\!\!\!\!\int_{\Sigma}\; B \,\bullet\, \widehat{\mathbf{n}}\;\d S\,\right)
&=&
	-\,\int_{\partial\Sigma}\, E \,\bullet\, \widehat{\mathbf{t}} \;\d l
\\
&=&
	-\,\int\!\!\!\!\int_{\Sigma}\; (\nabla \times E) \,\bullet\, \widehat{\mathbf{n}} \;\d S\,,
	\quad
	\textnormal{by Stoke's Theorem}
\end{eqnarray*}

          %%%%% ~~~~~~~~~~~~~~~~~~~~ %%%%%

\vskip 0.5cm
\noindent
\textbf{Amp\`{e}re-Maxwell's law}
\begin{eqnarray*}
\int\!\!\!\!\int_{\Sigma}\; (\nabla \times B) \,\bullet\, \widehat{\mathbf{n}} \;\d S
& = &
\int_{\partial\Sigma}\, B \,\bullet\, \widehat{\mathbf{t}} \;\d l\,,
\quad
\textnormal{by Stoke's Theorem}
\\
&=&
	\left(\begin{array}{c}
	\textnormal{magnetic field}
	\\
	\textnormal{around}
	\\
	\textnormal{close path $\partial\Sigma$}
	\end{array}\!\right)
\\
&=&
	4\,\pi\cdot\left(\begin{array}{c}
	\textnormal{electric current}
	\\
	\textnormal{through}
	\\
	\textnormal{surface $\Sigma \subset \Re^{3}$}
	\end{array}\!\right)
	\; + \;
	\left(\begin{array}{c}
	\textnormal{time rate of change}
	\\
	\textnormal{of electric field}
	\\
	\textnormal{through surface $\Sigma \subset \Re^{3}$}
	\end{array}\!\right)
\\
&=&
	4\,\pi\cdot\int\!\!\!\!\int_{\Sigma}\; J \,\bullet\, \widehat{\mathbf{n}} \,\;\d S
	\; + \;
	\dfrac{\d}{\d t}\left(\,\int\!\!\!\!\int_{\Sigma}\; E \,\bullet\, \widehat{\mathbf{n}} \,\;\d S\,\right)
\\
&=&
	\int\!\!\!\!\int_{\Sigma}\; \left(\,4\,\pi\cdot J \,+\, \dfrac{\partial E}{\partial t} \,\right)\,\bullet\, \widehat{\mathbf{n}} \,\;\d S
\end{eqnarray*}

          %%%%% ~~~~~~~~~~~~~~~~~~~~ %%%%%

          %%%%% ~~~~~~~~~~~~~~~~~~~~ %%%%%

          %%%%% ~~~~~~~~~~~~~~~~~~~~ %%%%%

          %%%%% ~~~~~~~~~~~~~~~~~~~~ %%%%%

