
          %%%%% ~~~~~~~~~~~~~~~~~~~~ %%%%%

\section{Lagrangian mechanics}
\setcounter{theorem}{0}
\setcounter{equation}{0}

%\cite{vanDerVaart1996}
%\cite{Kosorok2008}

%\renewcommand{\theenumi}{\alph{enumi}}
%\renewcommand{\labelenumi}{\textnormal{(\theenumi)}$\;\;$}
\renewcommand{\theenumi}{\roman{enumi}}
\renewcommand{\labelenumi}{\textnormal{(\theenumi)}$\;\;$}

          %%%%% ~~~~~~~~~~~~~~~~~~~~ %%%%%

\begin{definition}[Lagrangian mechanical system, Definition 2.1, \cite{Cortes2017}]
\mbox{}
\vskip 0.1cm
\noindent
A \,\textbf{Lagrangian mechanical system}\, is a pair \,$\left(\,M,\mathscr{L}\,\right)$\,
consisting of a smooth manifold \,$M$ and a smooth $\Re$-valued function
\,$\mathscr{L} : TM \longrightarrow \Re$\,
defined on the tangent bundle \,$TM$ of \,$M$.
The manifold \,$M$ is called the \textbf{configuration space} and
the function \,$\mathscr{L}$\, is called the \textbf{Lagrangian function}
(or simply the \textbf{Lagrangian}) of the system.
\end{definition}

          %%%%% ~~~~~~~~~~~~~~~~~~~~ %%%%%

\vskip 0.5cm
\begin{definition}[Hamilton's principle of least action]
\mbox{}
\vskip 0.1cm
\noindent
Let \,$\left(\,M,\mathscr{L}\,\right)$ be a Lagrangian mechanical system.
The \,\textbf{action}\, of a smooth parametrized curve
\,$\gamma : [\,a, b\,] \longrightarrow M$\,
is defined as
\begin{equation*}
S(\,\gamma\,)
\;\; := \;\;
\int_{a}^{b}
\mathscr{L}(\,\gamma(t)\,)\,\d t.
\end{equation*}
A \,\textbf{motion}\, of the system is a critical point of \,$S$\,
under smooth variations with fixed endpoints.
(This statement is a mathematical formulation of
\textbf{Hamilton's principle of least action},
which should better be called principle of {\color{red}stationary} action.)
\end{definition}

          %%%%% ~~~~~~~~~~~~~~~~~~~~ %%%%%

\vskip 0.5cm
\begin{definition}[Induced coordinate systems]
\mbox{}
\vskip 0.1cm
\noindent
Suppose:
\begin{itemize}
\item
	$\left(\,M,\mathscr{L}\,\right)$ is a Lagrangian mechanical system, with \,$n \,:=\, \dim(M)$, and
\item
	$(x^{1},\ldots,x^{n}) : U \subset M \longrightarrow \Re^{n}$\,
	is a local coordinate system.
\end{itemize}
Then, the \,\textbf{induced coordinate sytem}\, of \,$(x^{1},\ldots,x^{n})$\, is
\,$\left(\,q^{1},\ldots,q^{n},\hat{q}^{\,1},\ldots,\hat{q}^{\,n}\,\right) : \pi^{-1}(U) \longrightarrow \Re^{2n}$\,
given by
\begin{equation*} 
q^{i} \; := \; x^{i} \circ \pi \,:\, \pi^{-1}(U) \,\longrightarrow\, \Re,
\quad\textnormal{and}\quad
\hat{q}^{\,i} \; := \; \d x^{i} \,:\, \pi^{-1}(U) \,\longrightarrow\, \Re\,,
\end{equation*}
where \,$\pi : TM \longrightarrow M$\, is the canoncial projection.
\end{definition}

          %%%%% ~~~~~~~~~~~~~~~~~~~~ %%%%%

\vskip 0.5cm
\begin{remark}[Gradient of the Lagrangian function with respect to an induced coordinate system]
\mbox{}
\vskip 0.1cm
\noindent
Note that
\begin{equation*}
\nabla_{(q^{1},\ldots,q^{n},\hat{q}^{\,1},\ldots,\hat{q}^{\,n})}\,\mathscr{L}
\;\; = \;\;
	\left(\,
		\dfrac{\partial\mathscr{L}}{\partial q^{1}}\,,
		\,\ldots\,,
		\dfrac{\partial\mathscr{L}}{\partial q^{n}}\,,
		\dfrac{\partial\mathscr{L}}{\partial \hat{q}^{\,1}}\,,
		\,\ldots\,
		\dfrac{\partial\mathscr{L}}{\partial \hat{q}^{\,n}}
		\,\right)
\; : \;
	\pi^{-1}(U) \; \longrightarrow \; \Re^{2n}
\end{equation*}
The component functions of
\,$\nabla_{(q^{1},\ldots,q^{n},\hat{q}^{\,1},\ldots,\hat{q}^{\,n})}\,\mathscr{L}$\,
appear in the Euler-Lagrange equations,
the ordinary differential equations that determine the motions
of a Lagrangian mechanical system;
see Proposition \ref{EulerLagrangeEquations}.
\end{remark}

          %%%%% ~~~~~~~~~~~~~~~~~~~~ %%%%%

\vskip 0.5cm
\begin{proposition}[Equations of motion, Euler-Lagrangian equations]
\label{EulerLagrangeEquations}
\mbox{}
\vskip 0.1cm
\noindent
A smooth parametrized curve
\,$\gamma : [\,a,b\,] \longrightarrow M$\,
in a Lagrangian mechanical system
\,$\left(\,M,\mathscr{L}\,\right)$\,
is a motion if and only if
\,$\gamma$\,
satisfies the following ordinary differential equations:
\begin{equation*}
\dfrac{\partial\mathscr{L}}{\partial q^{i}}\!\left(\gamma^{\prime}(t)\right)
\; - \;
\dfrac{\d}{\d t}\!\left(\,\dfrac{\partial\mathscr{L}}{\partial \hat{q}^{\,i}}\!\left(\gamma^{\prime}(t)\right)\right)
\;\; = \;\;
0\,,
\end{equation*}
for each \,$i = 1, 2, \,\ldots\, , n \,:=\, \dim(M)$,
each \,$t \in [\,a,b\,]$, and
each induced coordinate system \,$\left(\,q^{1},\ldots,q^{n},\hat{q}^{\,1},\ldots,\hat{q}^{\,n}\,\right)$\,
containing \,$\gamma(t) \in M$.
\end{proposition}
\proof
\begin{equation}\label{VariationVectorField}
\left.\dfrac{\d}{\d s}\right\vert_{s=0} \mathscr{L}\!\left(\gamma_{s}(t)\right)
\;\; = \;\;
	\overset{n}{\underset{i=1}{\sum}}\,\left\{\,
		\dfrac{\partial\mathscr{L}}{\partial q^{i}}\!\left(\gamma^{\prime}_{0}(t)\right)
		\cdot
		\left.\dfrac{\d}{\d s}\right\vert_{s=0} q^{i}(\gamma^{\prime}_{s}(t))
		\, + \,
		\dfrac{\partial\mathscr{L}}{\partial\hat{q}^{\,i}}\!\left(\gamma^{\prime}_{0}(t)\right)
		\cdot
		\left.\dfrac{\d}{\d s}\right\vert_{s=0} \hat{q}^{\,i}(\gamma^{\prime}_{s}(t))
		\,\right\}
\end{equation}
Now, observe that
\begin{eqnarray*}
\left.\dfrac{\d}{\d s}\right\vert_{s=0} \hat{q}^{\,i}(\gamma^{\prime}_{s}(t))
& = &
	\left.\dfrac{\d}{\d s}\right\vert_{s=0} \d x^{i}(\gamma^{\prime}_{s}(t))
\;\; = \;\;
	\left.\dfrac{\d}{\d s}\right\vert_{s=0} \, \dfrac{\d}{\d t}\,(x^{i}\circ\gamma_{s})(t)
\;\; = \;\;
	\dfrac{\d}{\d t}\; \left.\dfrac{\d}{\d s}\right\vert_{s=0} \, (x^{i}\circ\gamma_{s})(t)
\;\; = \;\;
	\overset{{\color{white}1}}{\dfrac{\d}{\d t}}\, W^{i}(t),
\end{eqnarray*}
where
\,$W(t) \,:=\, \left.\dfrac{\d}{\d s}\right\vert_{s=0} \gamma_{s}(t) \,\in\, T_{\gamma(t)}M$\,
is the variation vector field defined along \,$\gamma$.
Hence,
\begin{eqnarray*}
\dfrac{\partial\mathscr{L}}{\partial\hat{q}^{\,i}}\!\left(\gamma^{\prime}_{0}(t)\right)
\cdot
\left.\dfrac{\d}{\d s}\right\vert_{s=0} \hat{q}^{\,i}(\gamma^{\prime}_{s}(t))
& = &
	\dfrac{\partial\mathscr{L}}{\partial\hat{q}^{\,i}}\!\left(\gamma^{\prime}_{0}(t)\right)
	\cdot
	\dfrac{\d}{\d t}\, W^{i}(t)	
\;\; = \;\;
	\dfrac{\d}{\d t}\!\left(\,
		\dfrac{\partial\mathscr{L}}{\partial\hat{q}^{\,i}}\!\left(\gamma^{\prime}_{0}(t)\right)
		\cdot
		W^{i}(t)
		\right)
	\, - \,
	\dfrac{\d}{\d t}\!\left(\,
		\dfrac{\partial\mathscr{L}}{\partial\hat{q}^{\,i}}\!\left(\gamma^{\prime}_{0}(t)\right)
		\right)	
	\cdot
	W^{i}(t)
\end{eqnarray*}
Substituting the above into \eqref{VariationVectorField} yields:
\begin{eqnarray*}
&&
	\left.\dfrac{\d}{\d s}\right\vert_{s=0} \mathscr{L}\!\left(\gamma_{s}(t)\right)
\\
& = &
	\overset{n}{\underset{i=1}{\sum}}\,\left\{\,
		\dfrac{\partial\mathscr{L}}{\partial q^{i}}\!\left(\gamma^{\prime}_{0}(t)\right)
		\cdot
		\left.\dfrac{\d}{\d s}\right\vert_{s=0} q^{i}(\gamma^{\prime}_{s}(t))
		\, + \,
		\dfrac{\partial\mathscr{L}}{\partial\hat{q}^{\,i}}\!\left(\gamma^{\prime}_{0}(t)\right)
		\cdot
		\left.\dfrac{\d}{\d s}\right\vert_{s=0} \hat{q}^{\,i}(\gamma^{\prime}_{s}(t))
		\,\right\}
\\
& = &
	\overset{n}{\underset{i=1}{\sum}}\,\left\{\,
		\dfrac{\partial\mathscr{L}}{\partial q^{i}}\!\left(\gamma^{\prime}(t)\right)
		\cdot
		\left.\dfrac{\d}{\d s}\right\vert_{s=0} x^{i}\circ\pi(\gamma^{\prime}_{s}(t))
		\, - \,
		\dfrac{\d}{\d t}\!\left(\,
			\dfrac{\partial\mathscr{L}}{\partial\hat{q}^{\,i}}\!\left(\gamma^{\prime}(t)\right)
			\right)	
		\cdot
		W^{i}(t)
		\,\right\}
	\; + \;
	\dfrac{\d}{\d t}\!\left(\,
		\overset{n}{\underset{i=1}{\sum}}\;
		\dfrac{\partial\mathscr{L}}{\partial\hat{q}^{\,i}}\!\left(\gamma^{\prime}(t)\right)
		\cdot
		W^{i}(t)
		\right)
\\
& = &
	\overset{n}{\underset{i=1}{\sum}}\,\left\{\,
		\dfrac{\partial\mathscr{L}}{\partial q^{i}}\!\left(\gamma^{\prime}(t)\right)
		\cdot
		\left.\dfrac{\d}{\d s}\right\vert_{s=0} x^{i}\circ\gamma_{s}(t)
		\, - \,
		\dfrac{\d}{\d t}\!\left(\,
			\dfrac{\partial\mathscr{L}}{\partial\hat{q}^{\,i}}\!\left(\gamma^{\prime}(t)\right)
			\right)	
		\cdot
		W^{i}(t)
		\,\right\}
	\; + \;
	f^{\prime}(t)
\\
& = &
	\overset{n}{\underset{i=1}{\sum}}\,\left\{\;
		\dfrac{\partial\mathscr{L}}{\partial q^{i}}\!\left(\gamma^{\prime}(t)\right)
		\, - \,
		\dfrac{\d}{\d t}\!\left(\,
			\dfrac{\partial\mathscr{L}}{\partial\hat{q}^{\,i}}\!\left(\gamma^{\prime}(t)\right)
			\right)
		\right\}
	\cdot
	W^{i}(t)
	\; + \;
	f^{\prime}(t)
\end{eqnarray*}
where
\begin{equation*}
f(t)
\;\; := \;\;
	\overset{n}{\underset{i=1}{\sum}}\;
	\dfrac{\partial\mathscr{L}}{\partial\hat{q}^{\,i}}\!\left(\gamma^{\prime}(t)\right)
	\cdot
	W^{i}(t)\,,
\quad
\textnormal{for \,$t \in [\,a,b\,]$}
\end{equation*}

\vskip 0.5cm
\noindent
\textbf{Claim 1:}
\vskip 0.1cm
\noindent
$f : [\,a,b\,] \longrightarrow \Re$\, is globally well-defined on \,$[\,a,b\,]$;\,
more precisely, the local definition of \,$f$\, above is in fact
independent of the choice of the local coordinate system.
Furthermore, \,$f(a) = f(b) = 0$.
\vskip 0.2cm
\noindent
Proof of Claim 1: \quad
\begin{equation*}
T^{\textnormal{ver}}_{\gamma(t)}(TM)
\;\; := \;\;
	\textnormal{ker}\!\left(\,\d\pi\vert_{\gamma(t)}\,\right)
\;\; \subset \;\;
	T_{\gamma(t)}(TM).
\end{equation*}
\begin{eqnarray*}
&&
	\left.\dfrac{\d}{\d s}\right\vert_{s=0} S\!\left(\gamma_{s}\right)
\\
& = &
	\left.\dfrac{\d}{\d s}\right\vert_{s=0}\; \int_{a}^{b}\,\mathscr{L}\!\left(\gamma_{s}(t)\right)\d t
\;\; = \;\;
	\int_{a}^{b}\left(\left.\dfrac{\d}{\d s}\right\vert_{s=0}\,\mathscr{L}\!\left(\gamma_{s}(t)\right)\right)\d t
\\
& = &
	\int_{a}^{b}\left(\;
		\overset{n}{\underset{i=1}{\sum}}\,\left\{\;
			\dfrac{\partial\mathscr{L}}{\partial q^{i}}\!\left(\gamma^{\prime}(t)\right)
			\, - \,
			\dfrac{\d}{\d t}\!\left(\,
				\dfrac{\partial\mathscr{L}}{\partial\hat{q}^{\,i}}\!\left(\gamma^{\prime}(t)\right)
				\right)
			\right\}
		\cdot
		W^{i}(t)
		\; + \;
		f^{\prime}(t)
		\right)\d t
\\
& = &
	\int_{a}^{b}\left(\;
		\overset{n}{\underset{i=1}{\sum}}\,\left\{\;
			\dfrac{\partial\mathscr{L}}{\partial q^{i}}\!\left(\gamma^{\prime}(t)\right)
			\, - \,
			\dfrac{\d}{\d t}\!\left(\,
				\dfrac{\partial\mathscr{L}}{\partial\hat{q}^{\,i}}\!\left(\gamma^{\prime}(t)\right)
				\right)
			\right\}
		\cdot
		W^{i}(t)
		\right)\d t
	\; + \;
	\left(\, f(b) \overset{{\color{white}1}}{-} f(a) \,\right)
\\
& = &
	\int_{a}^{b}\left(\;
		\overset{n}{\underset{i=1}{\sum}}\,\left\{\;
			\dfrac{\partial\mathscr{L}}{\partial q^{i}}\!\left(\gamma^{\prime}(t)\right)
			\, - \,
			\dfrac{\d}{\d t}\!\left(\,
				\dfrac{\partial\mathscr{L}}{\partial\hat{q}^{\,i}}\!\left(\gamma^{\prime}(t)\right)
				\right)
			\right\}
		\cdot
		W^{i}(t)
		\right)\d t
\end{eqnarray*}

\qed

          %%%%% ~~~~~~~~~~~~~~~~~~~~ %%%%%

          %%%%% ~~~~~~~~~~~~~~~~~~~~ %%%%%

          %%%%% ~~~~~~~~~~~~~~~~~~~~ %%%%%

