
          %%%%% ~~~~~~~~~~~~~~~~~~~~ %%%%%

\section{Lagrangian mechanics}
\setcounter{theorem}{0}
\setcounter{equation}{0}

%\cite{vanDerVaart1996}
%\cite{Kosorok2008}

%\renewcommand{\theenumi}{\alph{enumi}}
%\renewcommand{\labelenumi}{\textnormal{(\theenumi)}$\;\;$}
\renewcommand{\theenumi}{\roman{enumi}}
\renewcommand{\labelenumi}{\textnormal{(\theenumi)}$\;\;$}

          %%%%% ~~~~~~~~~~~~~~~~~~~~ %%%%%

\begin{definition}[Lagrangian mechanical system, Definition 2.1, \cite{Cortes2017}]
\mbox{}
\vskip 0.1cm
\noindent
A \,\textbf{Lagrangian mechanical system}\, is a pair \,$\left(\,M,\mathscr{L}\,\right)$\,
consisting of a smooth manifold \,$M$ and a smooth $\Re$-valued function
\,$\mathscr{L} : TM \longrightarrow \Re$\,
defined on the tangent bundle \,$TM$ of \,$M$.
The manifold \,$M$ is called the \textbf{configuration space} and
the function \,$\mathscr{L}$\, is called the \textbf{Lagrangian function}
(or simply the \textbf{Lagrangian}) of the system.
\end{definition}

          %%%%% ~~~~~~~~~~~~~~~~~~~~ %%%%%

\vskip 0.5cm
\begin{definition}[Hamilton's principle of least action]
\mbox{}
\vskip 0.1cm
\noindent
Let \,$\left(\,M,\mathscr{L}\,\right)$ be a Lagrangian mechanical system.
The \,\textbf{action}\, of a smooth curve
\,$\gamma : [\,a, b\,] \longrightarrow M$\,
is defined as
\begin{equation*}
S(\,\gamma\,)
\;\; := \;\;
\int_{a}^{b}
\mathscr{L}(\,\gamma(t)\,)\,\d t.
\end{equation*}
A \,\textbf{motion}\, of the system is a critical point of \,$S$\,
under smooth variations with fixed endpoints.
(This statement is a mathematical formulation of
\textbf{Hamilton's principle of least action},
which should better be called principle of {\color{red}stationary} action.)
\end{definition}

          %%%%% ~~~~~~~~~~~~~~~~~~~~ %%%%%

          %%%%% ~~~~~~~~~~~~~~~~~~~~ %%%%%

          %%%%% ~~~~~~~~~~~~~~~~~~~~ %%%%%

          %%%%% ~~~~~~~~~~~~~~~~~~~~ %%%%%

