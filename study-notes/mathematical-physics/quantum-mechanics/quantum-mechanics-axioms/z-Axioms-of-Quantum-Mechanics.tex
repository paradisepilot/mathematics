
          %%%%% ~~~~~~~~~~~~~~~~~~~~ %%%%%

\section{The Axioms of Quantum Mechanics}
\setcounter{theorem}{0}
\setcounter{equation}{0}

%\cite{vanDerVaart1996}
%\cite{Kosorok2008}

%\renewcommand{\theenumi}{\alph{enumi}}
%\renewcommand{\labelenumi}{\textnormal{(\theenumi)}$\;\;$}
\renewcommand{\theenumi}{\roman{enumi}}
\renewcommand{\labelenumi}{\textnormal{(\theenumi)}$\;\;$}

          %%%%% ~~~~~~~~~~~~~~~~~~~~ %%%%%

\begin{enumerate}
\item
	\textbf{Quantum state space is a complex Hilbert space}
	\vskip 0.05cm
	For every given ``physical system'', there exists a Hilbert space $\H$ over $\C$
	such that every possible ``quantum state'' of the given system is a unit vector in $\H$.
	Furthermore, if
	\begin{itemize}
	\item
		$\psi_{1}, \psi_{2} \in \H$ are unit vectors in $\H$, and they both represent quantum states of the given physical system,
	\item
		$\psi_{2} = c \cdot \psi_{1}$, for some $c \in \C$,
	\end{itemize}
	then $\psi_{1}$ and $\psi_{2}$ in fact represent the same quantum state.

\vskip 0.5cm
\item
	\textbf{Classical observable corresponds to self-adjoint operator on quantum state space}
	\vskip 0.05cm
	To each classical observable $f$
	(i.e. $f$ is a Borel-measurable $\Re$-valued function defined on the phase space in the Hamiltonian formalism)
	there corresponds a (possibly unbounded) self-adjoint operator
	$\widehat{f}$ on $\H$.
	$\widehat{f}$ is interpreted to be a quantum observable.

\vskip 0.5cm
\item
	\textbf{Each quantum state induces a probability measure on classical phase space, making classical observables into random variables}
	\vskip 0.05cm
	Every unit vector $\psi \in \H$ that represents a quantum state of a given physical system
	induces a probability measure $\mu_{\psi}$ on the classical phase space of the system,
	making each classical observable into a random variable
	(defined on the phase space, equipped with the $\psi$-induced probability space structure).
	Furthermore, for each classical observable $f$ and for each $m \in \{\,0\,\}\cup\N$,
	the expectation value $E_{\psi}\!\left[\;\overset{{\color{white}.}}{f^{\,m}}\;\right]$ of $f^{\,m}$
	with respect to $\mu_{\psi}$ satisfies the following equality:
	\begin{equation*}
	E_{\psi}\!\left[\;\overset{{\color{white}.}}{f^{\,m}}\;\right]
	\;\; := \;\;
		\int_{\Re^{2n}}
		f^{m}(x,p)
		\;\d\mu_{\psi}(x,p)
	\;\; = \;\;
		\left\langle\;
			\psi
			\,,\,
			\widehat{f}^{\,m}(\psi)
			\;\right\rangle_{\H}
	\end{equation*}

\vskip 0.3cm
\item
	\textbf{Time evolution of quantum states is governed by the Schr\"{o}dinger equation}
	\vskip 0.05cm
	Let $\psi(t) \in \H$ be the quantum state at time $t$ of a given physical system.
	Then, $\psi(t)$ is differentiable with respect to $t$, and its derivative satisfies
	the following equation (the Schr\"{o}dinger equation):
	\begin{equation*}
	\dfrac{\d\psi(t)}{\d t}
	\;\; = \;\;
		-\,\dfrac{\sqrt{-1}}{\hbar}\cdot\widehat{H}\!\left[\;\overset{{\color{white}.}}{\psi(t)}\,\right]
	\end{equation*}
	where $\hbar$ is the reduced Planck constant, and
	$\widehat{H} : \D(\widehat{H}) \subset \H \longrightarrow \H$
	is the (possibly unbounded) self-adjoint operator
	corresponding to the Hamiltonian function $H$ defined on the classical phase space.

\end{enumerate}

          %%%%% ~~~~~~~~~~~~~~~~~~~~ %%%%%


          %%%%% ~~~~~~~~~~~~~~~~~~~~ %%%%%
