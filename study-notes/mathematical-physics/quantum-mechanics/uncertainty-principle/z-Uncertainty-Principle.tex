
          %%%%% ~~~~~~~~~~~~~~~~~~~~ %%%%%

\section{The Uncertainty Principle}
\setcounter{theorem}{0}
\setcounter{equation}{0}

%\cite{vanDerVaart1996}
%\cite{Kosorok2008}

%\renewcommand{\theenumi}{\alph{enumi}}
%\renewcommand{\labelenumi}{\textnormal{(\theenumi)}$\;\;$}
\renewcommand{\theenumi}{\roman{enumi}}
\renewcommand{\labelenumi}{\textnormal{(\theenumi)}$\;\;$}

          %%%%% ~~~~~~~~~~~~~~~~~~~~ %%%%%

\begin{definition}
\mbox{}
\vskip 0.1cm
\noindent
Suppose \,$\H$,\, $\H_{1}$\, and \,$\H_{2}$\, are Hilbert spaces over \,$\C$.
\begin{enumerate}
\item
	A \,\textbf{linear operator}\, from \,$\H_{1}$\, into \,$\H_{2}$\,
	is a linear map
	\,$T : \D(T) \longrightarrow \H_{2}$,\,
	where \,$\D(T) \subset \H_{1}$\, is a linear subspace of \,$\H_{1}$.
\item
	A linear operator
	\,$T : \D(T) \subset \H \longrightarrow \H$\,
	is said to be \,\textbf{Hermitian}\, if
	\begin{equation*}
	\langle\,Tx\,,\,y\,\rangle
	\;\; = \;\;
		\langle\,x\,,\,Ty\,\rangle\,,
	\quad
	\textnormal{for each \,$x, y \in \D(T) \subset \H$}.
	\end{equation*}
\end{enumerate}
\end{definition}

          %%%%% ~~~~~~~~~~~~~~~~~~~~ %%%%%

\begin{theorem}[Uncertainty principle]
\mbox{}
\vskip 0.1cm
\noindent
Suppose:
\begin{itemize}
\item
	$\H$\, is a Hilbert space over \,$\C$.
\item
	$A : \D(A) \subset \H \longrightarrow \H$,\, $B : \D(B) \subset \H\longrightarrow \H$\,
	are Hermitian operators on $\H$.
\item
	The commutator
	\,$[\,A,B\,] : \D(A \circ B)\cap\D(B \circ A) \subset \H \longrightarrow \H$\,
	of \,$A$\, and \,$B$\, is defined as follows:
	\begin{equation*}
	[\,A,B\,](x)
	\;\; := \;\;
		A(Bx) \,-\, B(Ax)\,,
	\quad
	\textnormal{for each \,$x \,\in\, \D(A \circ B) \cap \D(B \circ A) \,\subset\, \H$}
	\end{equation*}
\end{itemize}
Then, the following inequality holds:
\begin{equation*}
\left\vert\; \left\langle\,[A,B]\,x\,,\,\overset{{\color{white}1}}{x}\,\right\rangle \;\right\vert
\;\; \leq \;\;
	2 \cdot
	\left\Vert\; (A \,-\, \alpha\cdot 1_{\H})\,\overset{{\color{white}1}}{x} \;\right\Vert
	\cdot
	\left\Vert\; (B \,-\, \beta\cdot 1_{\H})\,\overset{{\color{white}1}}{x} \;\right\Vert,
\end{equation*}
for each \,$\alpha, \beta \in \Re$,\, and
for each \,$x \,\in\, \D(A \circ B) \cap \D(B \circ A) \,\subset\, \H$.
\end{theorem}
\proof
\vskip 0.3cm
\noindent
\textbf{Claim 1:}\quad
For each \,$x \in \D(A \circ B)\cap\D(B \circ A) \subset \H$,\, we have:
\begin{equation*}
\left\langle\,[\,A,B\,]\,x\,,\,\overset{{\color{white}1}}{x}\,\right\rangle
\;\; = \;\;
	- \, \sqrt{-1} \cdot 2 \cdot
	\textnormal{Im}\!\left(\left\langle\,
		(A -\alpha\cdot 1_{\H})\,\overset{{\color{white}1}}{x}
		\,,\,
		(B - \beta\cdot 1_{\H})\,x
		\,\right\rangle\right)
\end{equation*}
Proof of Claim 1:\quad
First, note that
\begin{eqnarray*}
\left\langle\,[\,A,B\,]\,x\,,\,\overset{{\color{white}1}}{x}\,\right\rangle
&=&
	\left\langle\,AB\,x\,,\,\overset{{\color{white}1}}{x}\,\right\rangle
	\, - \,
	\left\langle\,BA\,x\,,\,\overset{{\color{white}1}}{x}\,\right\rangle
\\
&=&
	\left\langle\,Bx\,,A\overset{{\color{white}1}}{x}\,\right\rangle
	\, - \,
	\left\langle\,Ax\,,B\overset{{\color{white}1}}{x}\,\right\rangle\,,
	\quad
	\textnormal{since $A$, $B$ are Hermitian}
\\
&=&
	\overline{\left\langle\,Ax\,,B\overset{{\color{white}1}}{x}\,\right\rangle}
	\, - \,
	\left\langle\,Ax\,,B\overset{{\color{white}1}}{x}\,\right\rangle
\\
&=&
	-\, \sqrt{-1} \cdot 2 \cdot
	\textnormal{Im}\!\left(\left\langle\,Ax\,,B\overset{{\color{white}1}}{x}\,\right\rangle\right),
\end{eqnarray*}
where the last equality follows from
$\overline{(a + \i\,b)} - (a + \i\,b)$ $=$ $(a - \i\,b) - (a + \i\,b)$ $=$ $-\i\cdot 2\cdot b$,
for $\i\,:=\sqrt{-1}$ and for each $a,b\in\Re$.
On the other hand,
\begin{eqnarray*}
\left\langle\,
	(A -\alpha\cdot 1_{\H})\,\overset{{\color{white}1}}{x}
	\,,\,
	(B - \beta\cdot 1_{\H})\,x
	\,\right\rangle
&=&
	\left\langle\, A\overset{{\color{white}1}}{x} \,, Bx \,\right\rangle
	\,-\,
		\beta\cdot
		\left\langle\, A\overset{{\color{white}1}}{x} \,,\, x \,\right\rangle
	\,-\,
		\alpha\cdot
		\left\langle\, \overset{{\color{white}1}}{x} \,, Bx \,\right\rangle
	\,+\,
		\alpha\,\beta\cdot
		\left\langle\, \overset{{\color{white}1}}{x} \,,\, x \,\right\rangle
\end{eqnarray*}
which implies
\begin{eqnarray*}
\textnormal{Im}\!\left(
	\left\langle\,
		(A -\alpha\cdot 1_{\H})\,\overset{{\color{white}1}}{x}
		\,,\,
		(B - \beta\cdot 1_{\H})\,x
		\,\right\rangle
	\right)
&=&
	\textnormal{Im}\!\left(
		\left\langle\, A\overset{{\color{white}1}}{x} \,, Bx \,\right\rangle
		\right)
\end{eqnarray*}
since
$\beta\cdot\left\langle\, A\overset{{\color{white}1}}{x} \,,\, x \,\right\rangle$,\,
$\alpha\cdot\left\langle\, \overset{{\color{white}1}}{x} \,, Bx \,\right\rangle$,\,
$\alpha\,\beta\cdot\left\langle\, \overset{{\color{white}1}}{x} \,,\, x \,\right\rangle$\,
$\in$ $\Re$.
Thus, we see that
\begin{eqnarray*}
\left\langle\,[\,A,B\,]\,x\,,\,\overset{{\color{white}1}}{x}\,\right\rangle
&=&
	-\, \sqrt{-1} \cdot 2 \cdot
	\textnormal{Im}\!\left(\left\langle\,Ax\,,B\overset{{\color{white}1}}{x}\,\right\rangle\right),
\\
&=&
	-\, \sqrt{-1} \cdot 2 \cdot
	\textnormal{Im}\!\left(
		\left\langle\,
			(A -\alpha\cdot 1_{\H})\,\overset{{\color{white}1}}{x}
			\,,\,
			(B - \beta\cdot 1_{\H})\,x
			\,\right\rangle
		\right)
\end{eqnarray*}
This proves Claim 1.

\vskip 0.5cm
\noindent
Now, observe that
\begin{eqnarray*}
\left\vert\; \left\langle\,[A,B]\,x\,,\,\overset{{\color{white}1}}{x}\,\right\rangle \;\right\vert
& = &
	2 \cdot
	\left\vert\;
		\textnormal{Im}\!\left(
			\left\langle\,
				(A -\alpha\cdot 1_{\H})\,\overset{{\color{white}1}}{x}
				\,,\,
				(B - \beta\cdot 1_{\H})\,x
				\,\right\rangle
			\right)
		\;\right\vert,
	\quad
	\textnormal{by Claim 1}
\\
& \leq &
	2 \cdot
	\left\vert\;
			\left\langle\,
				(A -\alpha\cdot 1_{\H})\,\overset{{\color{white}1}}{x}
				\,,\,
				(B - \beta\cdot 1_{\H})\,x
				\,\right\rangle
		\;\right\vert
\\
& \leq &
	2 \cdot
	\left\Vert\; (A \,-\, \alpha\cdot 1_{\H})\,\overset{{\color{white}1}}{x} \;\right\Vert
	\cdot
	\left\Vert\; (B \,-\, \beta\cdot 1_{\H})\,\overset{{\color{white}1}}{x} \;\right\Vert,
	\quad
	\textnormal{by the Cauchy-Schwarz inequality}
\end{eqnarray*}
This completes the proof of the Theorem.
\qed

          %%%%% ~~~~~~~~~~~~~~~~~~~~ %%%%%
