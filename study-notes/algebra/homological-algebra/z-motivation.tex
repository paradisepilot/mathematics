
          %%%%% ~~~~~~~~~~~~~~~~~~~~ %%%%%

\section{Homological algebra -- motivation}
\setcounter{theorem}{0}
\setcounter{equation}{0}

%\cite{vanDerVaart1996}
%\cite{Kosorok2008}

%\renewcommand{\theenumi}{\alph{enumi}}
%\renewcommand{\labelenumi}{\textnormal{(\theenumi)}$\;\;$}
\renewcommand{\theenumi}{\roman{enumi}}
\renewcommand{\labelenumi}{\textnormal{(\theenumi)}$\;\;$}

          %%%%% ~~~~~~~~~~~~~~~~~~~~ %%%%%

\begin{proposition}
\label{homologicalAlgebraMotivation}
\mbox{}
\vskip 0.1cm
\noindent
Suppose $R$ is a ring with identity, and the following is
a short exact sequence of $R$-modules
\begin{equation}\label{givenShortExactSequence}
0
\;\; \longrightarrow \;\;
	L
\;\; \overset{\psi}{\longrightarrow} \;\;
	M
\;\; \overset{\varphi}{\longrightarrow} \;\;
	N
\;\; \longrightarrow \;\;
	0
\end{equation}
For any $R$-module $D$, let 
\begin{equation*}
\begin{array}{lccccc}
\psi^{\prime} & : & \Hom_{R}(M,D) & \longrightarrow & \Hom_{R}(\,L\,,D) \;\; : \;\; \eta \;\; \longmapsto \;\; \eta \circ \psi
\\
\varphi^{\prime} & : & \Hom_{R}(N,D) & \longrightarrow & \Hom_{R}(M,D) \;\; : \;\; \zeta \;\; \longmapsto \;\; \zeta \circ \varphi
\end{array}
\end{equation*}
Then, the following is an exact sequence of abelian groups:
\begin{equation}\label{inducedExactSequence}
0
\;\; \longrightarrow \;\;
	\Hom_{R}(N,D)
\;\; \overset{\varphi^{\prime}}{\longrightarrow} \;\;
	\Hom_{R}(M,D)
\;\; \overset{\psi^{\prime}}{\longrightarrow} \;\;
	\Hom_{R}(L,D)
\end{equation}
\end{proposition}
\proof

\vskip 0.25cm
\noindent
\underline{{\color{white}.}Exactness of \eqref{inducedExactSequence} at $\Hom_{R}(N,D)$, i.e. {\color{red}injectivity of $\varphi^{\prime}$}{\color{white}.}}
\vskip 0.25cm
\noindent
Let $\zeta_{1}, \zeta_{2} \in \Hom_{R}(N,D)$ be such that
$\varphi^{\prime}(\zeta_{1}) = \varphi^{\prime}(\zeta_{2}) \in \Hom_{R}(M,D)$, i.e.
$\zeta_{1} \circ \varphi = \zeta_{2} \circ \varphi$.
We need to show that $\zeta_{1} = \zeta_{2} \in \Hom_{R}(N,D)$.
But, this follows immediately from the {\color{red}surjectivity of} ${\color{red}\varphi} \in \Hom_{R}(M,N)$.


\vskip 0.50cm
\noindent
\underline{{\color{white}.}Exactness of \eqref{inducedExactSequence} at $\Hom_{R}(M,D)$}
\vskip 0.25cm
\noindent
We need to show that \,$\image(\,\varphi^{\prime}\,) = \ker(\,\psi^{\prime}\,)$,\,
equivalently, $\psi^{\prime} \circ \varphi^{\prime} = 0$.
To this end, let
$\varphi^{\prime}(\zeta) \;=\; \zeta \circ \varphi \;\in\; \image(\,\varphi^{\prime}\,) \;\subset\; \Hom_{R}(M,D)$.
Then, observe that
\begin{equation*}
(\,\psi^{\prime}\circ\varphi^{\prime}\,)(\zeta)
\;\; = \;\;
	\psi^{\prime}(\,\varphi^{\prime}(\zeta)\,)
\;\; = \;\;
	\psi^{\prime}(\,\zeta \circ \varphi\,)
\;\; = \;\;
	(\,\zeta \circ \varphi\,) \circ \psi
\;\; = \;\;
	\zeta \circ (\,\varphi \circ \psi\,),
\quad
	\textnormal{for each \,$\zeta \in \Hom_{R}(N,D)$}
\end{equation*}
Hence,
\begin{eqnarray*}
\textnormal{exactness of \eqref{givenShortExactSequence} at $M$}
& \Longleftrightarrow &
	{\color{red}\image(\,\psi\,) \,=\, \ker(\,\varphi\,)}
\\
& \Longleftrightarrow &
	\varphi \,\circ\, \psi \,= 0
\\
& {\color{red}\Longrightarrow} &
	\psi^{\prime} \circ \varphi^{\prime} = 0
\\
& \Longleftrightarrow &
	{\color{red}\image(\,\varphi^{\prime}\,) = \ker(\,\psi^{\prime}\,)}
\\
& \Longleftrightarrow &
	\textnormal{exactness of \eqref{inducedExactSequence} at $\Hom_{R}(M,D)$},
\end{eqnarray*}
as required.
\qed

          %%%%% ~~~~~~~~~~~~~~~~~~~~ %%%%%

\vskip 0.5cm
\begin{remark}
\mbox{}
\vskip 0.1cm
\noindent
Consider the following diagram:
\begin{center}
\begin{tikzcd}
0 \arrow[r]
& L \arrow[r, "\psi"] \arrow[d, "\xi{\color{white}.}" swap]
& M \arrow[r, "\varphi"] \arrow[dr] \arrow[dl, dashrightarrow, "\exists{\color{white}.}\eta{\color{white}.}?"]
& N \arrow[d, "{\color{white}.}\zeta_{1}\textnormal{,\,}\zeta_{2}"] \arrow[r]
& 0
\\
& D
&
& D
\end{tikzcd}\end{center}
\begin{itemize}
\item
	The exactness of \eqref{inducedExactSequence} at \,$\Hom_{R}(N,D)$
	is equivalent to the injectivity of $\varphi^{\prime}$,
	which is a consequence of the surjectivity of $\varphi$.
\item
	The exactness of \eqref{inducedExactSequence} at \,$\Hom_{R}(M,D)$
	is equivalent to \,$\psi^{\prime}\circ\varphi^{\prime} = 0$,\,
	which follows from the exactness of \eqref{givenShortExactSequence} at \,$M$
	(which in turn is equivalent to \,$\varphi \circ \psi = 0$),
	since
	\begin{equation*}
	(\,\psi^{\prime}\circ\varphi^{\prime}\,)(\zeta)
	\;\; = \;\;
		\psi^{\prime}(\,\varphi^{\prime}(\zeta)\,)
	\;\; = \;\;
		\psi^{\prime}(\,\zeta \circ \varphi\,)
	\;\; = \;\;
		(\,\zeta \circ \varphi\,) \circ \psi
	\;\; = \;\;
		\zeta \circ (\,\varphi \circ \psi\,),
	\quad
		\textnormal{for each \,$\zeta \in \Hom_{R}(N,D)$}
	\end{equation*}
	as mentioned in the proof of Proposition \ref{homologicalAlgebraMotivation}.
\item
	The induced exact sequence \eqref{inducedExactSequence}
	in general cannot be extended on the right to a short exact sequence.
	Indeed, this extendibility is equivalent to the surjectivity of $\psi^{\prime}$,
	which in turn is equivalent to the following statement:
	\begin{center}
	\begin{minipage}{6.0in}
	\textnormal{For each $R$-module homomorphism
	\,$L\,\overset{\xi}{\longrightarrow}\,D$,\,
	there exists an $R$-module homomorphism
	\,$M\,\overset{\eta}{\longrightarrow}\,D$,\,
	such that
	\,$\xi \,=\, \psi^{\prime}(\eta) \,=\, \eta \circ \psi$
	}
	\end{minipage}
	\end{center}
	And, the above statement is clearly not true in general.
\end{itemize}
Homological algebra provides tools to measure the degree to which
\eqref{inducedExactSequence}
{\color{red}fails to extend to a short exact sequence}.
\end{remark}

          %%%%% ~~~~~~~~~~~~~~~~~~~~ %%%%%
