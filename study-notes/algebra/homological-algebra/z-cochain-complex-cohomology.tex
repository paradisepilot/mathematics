
          %%%%% ~~~~~~~~~~~~~~~~~~~~ %%%%%

\section{The long exact sequence in cohomology of a short exact sequence of cochain complexes}
\setcounter{theorem}{0}
\setcounter{equation}{0}

%\cite{vanDerVaart1996}
%\cite{Kosorok2008}

%\renewcommand{\theenumi}{\alph{enumi}}
%\renewcommand{\labelenumi}{\textnormal{(\theenumi)}$\;\;$}
\renewcommand{\theenumi}{\roman{enumi}}
\renewcommand{\labelenumi}{\textnormal{(\theenumi)}$\;\;$}

          %%%%% ~~~~~~~~~~~~~~~~~~~~ %%%%%

\begin{definition}
\mbox{}
\vskip 0.1cm
\noindent
Let \,$\mathcal{C}$\, be a sequence of abelian group homomorphisms:
\begin{equation*}
0
\;\; \longrightarrow \;\;
	C^{0}
\;\; \overset{d_{1}}{\longrightarrow} \;\;
	C^{1}
\;\; \overset{d_{2}}{\longrightarrow} \;\;
	C^{2}
\;\; \longrightarrow \;\;
	\cdots\cdots
\;\; \longrightarrow \;\;
	C^{n-1}
\;\; \overset{d_{n}}{\longrightarrow} \;\;
	C^{n}
\;\; \overset{d_{n+1}}{\longrightarrow} \;\;
\;\; \longrightarrow \;\;
	\cdots\cdots
\end{equation*}
\begin{itemize}
\item
	The sequence \,$\mathcal{C}$\, is called a \textbf{cochain complex} if
	\,$d_{n+1} \circ d_{n} \,=\, 0$,\, for each \,$n = 1, 2, \ldots$.\,
\item
	For a cochain complex \,$\mathcal{C}$,\, its \,\textbf{$n^{\textnormal{th}}$ cohomology group}
	is the quotient group
	\begin{equation*}
	H^{n}(\,\mathcal{C}\,)
	\; := \;
		\left.\overset{{\color{white}.}}{\ker(d_{n+1})}\right/\image(d_{n})
	\end{equation*}
\end{itemize}
\end{definition}

          %%%%% ~~~~~~~~~~~~~~~~~~~~ %%%%%

\begin{definition}
\mbox{}
\vskip 0.1cm
\noindent
Let
\,$\mathcal{A} = \{\,A^{n}\,\}$\,
and
\,$\mathcal{B} = \{\,B^{n}\,\}$\,
be two cochain complexes.
A \textbf{homomorphism of cochain complexes}
\,$\alpha : \mathcal{A} \longrightarrow \mathcal{B}$\,
is a set of homomorphisms
\,$\alpha_{n} : A^{n} \longrightarrow B^{n}$\,
such that for each $n = 0, 1, 2, \ldots$\,,
the following diagram commutes:
\begin{center}
\begin{tikzcd}
\cdots\cdots{\color{white}...} \arrow[r]
& A^{n-1} \arrow[r, "a_{n}"] \arrow[d, "\alpha_{n-1}{\color{white}.}" swap]
& A^{n} \arrow[r, "a_{n+1}"] \arrow[d, "\alpha_{n}{\color{white}.}" swap]
& A^{n+1} \arrow[r] \arrow[d, "{\color{white}.}\alpha_{n+1}"]
& {\color{white}...}\cdots\cdots
\\
\cdots\cdots{\color{white}...} \arrow[r]
& B^{n-1} \arrow[r, "b_{n}" swap]
& B^{n} \arrow[r, "b_{n+1}" swap]
& B^{n+1} \arrow[r]
& {\color{white}...}\cdots\cdots
\end{tikzcd}
\end{center}
\end{definition}

          %%%%% ~~~~~~~~~~~~~~~~~~~~ %%%%%

\begin{proposition}
\mbox{}
\vskip 0.1cm
\noindent
A homomorphism \,$\alpha : \mathcal{A} \longrightarrow \mathcal{B}$\,
of cochain complexes induces group homomorphisms from
\,$H^{n}(\mathcal{A})$\, to \,$H^{n}(\mathcal{B})$,\, for each \,$n = 0, 1, 2, \ldots$\,,
on their respective cohomology groups.
\end{proposition}
\proof
We need to define the map \,$\widehat{\alpha}_{n}$\, induced by \,$\alpha_{n}$:\,
\begin{equation*}
\dfrac{\ker(a_{n+1})}{\image(a_{n})}
\; =: \;
	H^{n}(\,\mathcal{A}\,)
\;\; \overset{\widehat{\alpha}_{n}}{\longrightarrow} \;\;
	H^{n}(\,\mathcal{B}\,)
\; := \;
	\dfrac{\ker(b_{n+1})}{\image(b_{n})}
\end{equation*}
To this end, we define
\begin{equation*}
\widehat{\,\alpha}_{n}\!\left(\,x_{n} \overset{{\color{white}.}}{+} \image(a_{n+1})\,\right)
\;\; := \;\;
	\alpha_{n}(\,x_{n}\,) \,+\, \image(b_{n+1})\,,
\quad
\textnormal{for each \,$x_{n} \in \ker(a_{n+1}) \subset A^{n}$}
\end{equation*}
We need to establish the well-definition of \,$\widehat{\alpha}_{n}$.\,
To this end, suppose
\,$x_{n},\, x_{n}^{\prime} \in \ker(a_{n+1}) \subset A^{n}$\,
are such that
\,$x_{n} + \image(a_{n}) \,=\, x_{n}^{\prime} + \image(a_{n})$,\,
equivalently, \,$x_{n} - x_{n}^{\prime} \,\in\, \image(a_{n}) \subset \ker(a_{n+1})$.\,
We need to show that
\,$\alpha_{n}(x_{n}) + \image(b_{n+1}) \,=\, \alpha_{n}(x_{n}^{\prime}) + \image(b_{n})$,\,
equivalently, \,$\alpha_{n}(x_{n}) - \alpha_{n}(x_{n}^{\prime}) \,\in\, \image(b_{n})$.\,

\vskip 0.25cm
\noindent
Now, observe:
\begin{eqnarray*}
x_{n} - x_{n}^{\prime} \,\in\, \image(a_{n})
& \Longleftrightarrow &
	x_{n} - x_{n}^{\prime} \; = \; a_{n}(z_{n-1}),
	\quad
	\textnormal{for some \,$z_{n-1} \in A^{n-1}$}
\\
& \overset{{\color{white}1}}{\Longrightarrow} &
	\alpha_{n}\!\left(\;x_{n} \overset{{\color{white}.}}{-} x_{n}^{\prime}\,\right)
	\; = \;
		(\,{\color{red}\alpha_{n} \,\circ\, a_{n}}\,)\!\left(\,\overset{{\color{white}-}}{z_{n-1}}\,\right)
	\; = \;
		(\,{\color{red}b_{n} \,\circ\, \alpha_{n-1}}\,)\!\left(\,\overset{{\color{white}-}}{z_{n-1}}\,\right)
	\; \in \;
		\image(\,b_{n}\,),
\end{eqnarray*}
as required.
\qed

          %%%%% ~~~~~~~~~~~~~~~~~~~~ %%%%%

\begin{definition}
\mbox{}
\vskip 0.1cm
\noindent
Let
\,$\mathcal{A} = \{\,A^{n}\,\}$,\,
\,$\mathcal{B} = \{\,B^{n}\,\}$,\,
and
\,$\mathcal{C} = \{\,C^{n}\,\}$\,
be cochain complexes.
A \textbf{short exact sequences of cochain complexes}
\,$0 \,\longrightarrow\, \mathcal{A} \,\overset{\alpha}{\longrightarrow}\, \mathcal{B} \,\overset{\beta}{\longrightarrow}\, \mathcal{C} \,\longrightarrow\, 0$\,
is a sequence of homomorphisms of cochain complexes such that
\begin{equation*}
0
\;\; \longrightarrow \;\;
	A^{n}
\;\; \overset{\alpha_{n}}{\longrightarrow} \;\;
	B^{n}
\;\; \overset{\beta_{n}}{\longrightarrow} \;\;
	C^{n}
\,\longrightarrow\,
	0
\end{equation*}
is a short exact sequence of abelian groups, for each \,$n = 0, 1, 2, \ldots$\,.
\end{definition}

          %%%%% ~~~~~~~~~~~~~~~~~~~~ %%%%%

\begin{theorem}
\mbox{}
\vskip 0.1cm
\noindent
Suppose
\,$0 \,\longrightarrow\, \mathcal{A} \,\overset{\alpha}{\longrightarrow}\, \mathcal{B} \,\overset{\beta}{\longrightarrow}\, \mathcal{C} \,\longrightarrow\, 0$\,
is a short exact sequences of cochain complexes.
Then, there exists a long exact sequence of cohomology groups:
\begin{center}
\begin{tikzcd}
0 \arrow[r]
& {\color{white}.}H^{0}(\mathcal{A}){\color{white}.} \arrow[r,"\widehat{\alpha}_{0}"]
& {\color{white}.}H^{0}(\mathcal{B}){\color{white}.} \arrow[r,"\widehat{\beta}_{0}"] \arrow[d, phantom, ""{coordinate, name=Z}]
& {\color{white}.}H^{0}(\mathcal{C}){\color{white}.} 
%\arrow[dll, "\delta_{0}" swap,
%rounded corners,
%to path={ -- ([xshift=2ex]\tikztostart.east)
%|- (Z) [near end]\tikztonodes
%-| ([xshift=-2ex]\tikztotarget.west)
%-- (\tikztotarget)}]
\arrow[dll, %"\delta_{0}" swap,
rounded corners,
to path={
	-- ([xshift=2ex]\tikztostart.east)
	|- (Z) [near end]\tikztonodes
	-| ([xshift=-2ex]\tikztotarget.west)
	-- (\tikztotarget)
	}]
\\
& {\color{white}.}H^{1}(\mathcal{A}){\color{white}.} \arrow[r,"\widehat{\alpha}_{1}"]
& {\color{white}.}H^{1}(\mathcal{B}){\color{white}.} \arrow[r,"\widehat{\beta}_{1}"] \arrow[d, phantom, ""{coordinate, name=Z}]
& {\color{white}.}H^{1}(\mathcal{C}){\color{white}.}
\arrow[dll,
rounded corners,
to path={
	-- ([xshift=2ex]\tikztostart.east)
	|- (Z) [near end]\tikztonodes
	-| ([xshift=-2ex]\tikztotarget.west)
	-- (\tikztotarget)
	}]
\\
& {\color{white}.}H^{2}(\mathcal{A}){\color{white}.} \arrow[r,"\widehat{\alpha}_{2}"]
& {\color{white}.}H^{2}(\mathcal{B}){\color{white}.} \arrow[r,"\widehat{\beta}_{2}"] \arrow[d, phantom, ""{coordinate, name=Z}]
& {\color{white}.}H^{2}(\mathcal{C}){\color{white}.}
\arrow[dll,
rounded corners,
to path={
	-- ([xshift=2ex]\tikztostart.east)
	|- (Z) [near end]\tikztonodes
	-| ([xshift=-2ex]\tikztotarget.west)
	-- (\tikztotarget)
	}]
\\
& {\color{white}.}H^{3}(\mathcal{A}){\color{white}.} \arrow[r]
& {\color{white}.}\cdots\;\;\cdots\;\;\cdots{\color{white}.} \arrow[r, white]
& {\color{white}.\Ext^{3}_{R}(N,D).}
\end{tikzcd}\end{center}
where the maps
\begin{equation*}
H^{n}(\mathcal{C}) \; \overset{\delta_{n}}{\longrightarrow} \; H^{n+1}(\mathcal{A})\,,
\quad
\textnormal{for \,$n = 0, 1, 2, \ldots\,$},
\end{equation*}
are called the connecting homomorphisms.
\end{theorem}
\proof

\qed

          %%%%% ~~~~~~~~~~~~~~~~~~~~ %%%%%
