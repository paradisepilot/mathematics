
          %%%%% ~~~~~~~~~~~~~~~~~~~~ %%%%%

\section{Homological algebra -- motivation}
\setcounter{theorem}{0}
\setcounter{equation}{0}

%\cite{vanDerVaart1996}
%\cite{Kosorok2008}

%\renewcommand{\theenumi}{\alph{enumi}}
%\renewcommand{\labelenumi}{\textnormal{(\theenumi)}$\;\;$}
\renewcommand{\theenumi}{\roman{enumi}}
\renewcommand{\labelenumi}{\textnormal{(\theenumi)}$\;\;$}

          %%%%% ~~~~~~~~~~~~~~~~~~~~ %%%%%

\begin{proposition}
\label{homologicalAlgebraMotivation}
\mbox{}
\vskip 0.1cm
\noindent
Suppose $R$ is a ring with identity, and the following is
a short exact sequence of $R$-modules
\begin{equation}\label{givenShortExactSequence}
0
\;\; \longrightarrow \;\;
	L
\;\; \overset{\psi}{\longrightarrow} \;\;
	M
\;\; \overset{\varphi}{\longrightarrow} \;\;
	N
\;\; \longrightarrow \;\;
	0
\end{equation}
For any $R$-module $D$, let 
\begin{equation*}
\begin{array}{lccccc}
\psi^{\prime} & : & \Hom_{R}(M,D) & \longrightarrow & \Hom_{R}(\,L\,,D) \;\; : \;\; \eta \;\; \longmapsto \;\; \eta \circ \psi
\\
\varphi^{\prime} & : & \Hom_{R}(N,D) & \longrightarrow & \Hom_{R}(M,D) \;\; : \;\; \zeta \;\; \longmapsto \;\; \zeta \circ \varphi
\end{array}
\end{equation*}
Then, the following is an exact sequence of abelian groups:
\begin{equation}\label{inducedExactSequence}
0
\;\; \longrightarrow \;\;
	\Hom_{R}(N,D)
\;\; \overset{\varphi^{\prime}}{\longrightarrow} \;\;
	\Hom_{R}(M,D)
\;\; \overset{\psi^{\prime}}{\longrightarrow} \;\;
	\Hom_{R}(L,D)
\end{equation}
\end{proposition}
\proof

\vskip 0.25cm
\noindent
\underline{Exactness of \eqref{inducedExactSequence} at $\Hom_{R}(N,D)$, i.e. {\color{red}injectivity of $\varphi^{\prime}$}{\color{white}.}}
\vskip 0.25cm
\noindent
Let $\zeta_{1}, \zeta_{2} \in \Hom_{R}(N,D)$ be such that
$\varphi^{\prime}(\zeta_{1}) = \varphi^{\prime}(\zeta_{2}) \in \Hom_{R}(M,D)$, i.e.
$\zeta_{1} \circ \varphi = \zeta_{2} \circ \varphi$.
We need to show that $\zeta_{1} = \zeta_{2} \in \Hom_{R}(N,D)$.
But, this follows immediately from the {\color{red}surjectivity of} ${\color{red}\varphi} \in \Hom_{R}(M,N)$.


\vskip 0.50cm
\noindent
\underline{Exactness of \eqref{inducedExactSequence} at $\Hom_{R}(M,D)$,\, i.e. $\image(\psi^{\prime}) = \ker(\varphi^{\prime})$}
\vskip 0.25cm
\noindent
We need to show that \,$\image(\,\varphi^{\prime}\,) = \ker(\,\psi^{\prime}\,)$.\,
First, we establish \,$\image(\,\varphi^{\prime}\,) \subset \ker(\,\psi^{\prime}\,)$,\,
equivalently, \,$\psi^{\prime} \circ \varphi^{\prime} = 0$.\,
To this end, let
$\varphi^{\prime}(\zeta) \;=\; \zeta \circ \varphi \;\in\; \image(\,\varphi^{\prime}\,) \;\subset\; \Hom_{R}(M,D)$.
Then, observe that
\begin{equation*}
(\,\psi^{\prime}\circ\varphi^{\prime}\,)(\zeta)
\;\; = \;\;
	\psi^{\prime}(\,\varphi^{\prime}(\zeta)\,)
\;\; = \;\;
	\psi^{\prime}(\,\zeta \circ \varphi\,)
\;\; = \;\;
	(\,\zeta \circ \varphi\,) \circ \psi
\;\; = \;\;
	\zeta \circ (\,\varphi \circ \psi\,),
\quad
	\textnormal{for each \,$\zeta \in \Hom_{R}(N,D)$}
\end{equation*}
Hence,
\begin{equation*}
\textnormal{exactness of \eqref{givenShortExactSequence} at $M$}
\; \Longleftrightarrow \;
	{\color{red}\image(\,\psi\,) \,=\, \ker(\,\varphi\,)}
\; \Longrightarrow \;
	\varphi \,\circ\, \psi \,= 0
\;\; {\color{red}\Longrightarrow} \;\;
	\psi^{\prime} \circ \varphi^{\prime} = 0
\; \Longleftrightarrow \;
	{\color{red}\image(\,\varphi^{\prime}\,) \subset \ker(\,\psi^{\prime}\,)}
\end{equation*}
as required.

\vskip 0.25cm
\noindent
Next, we show that
\,$\image(\,\varphi^{\prime}\,) \supset \ker(\,\psi^{\prime}\,)$.\,
To this end, let \,$\beta \in \ker(\,\psi^{\prime}\,) \subset \Hom_{R}(M,D)$,\,
i.e. \,$\psi^{\prime}(\beta) = \beta \circ \psi = 0$.\,
We need to show \,$\beta \in \image(\,\varphi^{\prime}\,)$,\,
i.e. there exists \,$\alpha \in \Hom_{R}(N,D)$\, such that
\,$\beta = \varphi^{\prime}(\alpha) = \alpha \circ \varphi \in \Hom_{R}(M,D)$.\,
So, we now construct/define such an \,$\alpha \in \Hom_{R}(N,D)$,\, as follows:
\begin{equation*}
\alpha(\,n\,) \; := \; \beta(\,m\,),
\quad
\textnormal{for any \,$m \in \varphi^{-1}(n)$,\; for each \,$n \in N$}
\end{equation*}
For \,$\alpha$\, to be set-theoretically well-defined, we must have:
\begin{itemize}
\item
	$\varphi^{-1}(n) \,\neq\, \varemptyset$,\, for each \,$n \in N$, and
\item
	$\beta(\,m_{1}\,) \,=\, \beta(\,m_{2}\,)$,\, for any \,$m_{1}, m_{2} \in \varphi^{-1}(n)$,\, for each \,$n \in N$.	
\end{itemize}
The first bullet is exactly the {\color{red}surjectivity of \,$\varphi$},\,
equivalently, the exactness of \eqref{givenShortExactSequence} at $N$.
For the second bullet, note that
\begin{eqnarray*}
m_{1}, m_{2} \in \varphi^{-1}(n)
& \Longleftrightarrow &
	\varphi(m_{1}) \,=\, n \,=\, \varphi(m_{2})
\;\; \Longleftrightarrow \;\;
	m_{1} - m_{2} \,\in\, {\color{red}\ker(\varphi) \,=\, \image(\psi)}
\\
& \overset{{\color{white}1}}{\Longrightarrow} &
	\beta(\,m_{1} - m_{2}\,) \,\in\, \beta\!\left(\,\image(\psi)\,\right) \,=\, \{\,0\,\},
	\;\;
	\textnormal{since \,$\psi^{\prime}(\beta) \,=\, \beta \circ \psi \,=\, 0$}
\\
& \overset{{\color{white}1}}{\Longrightarrow} &
	\beta(\,m_{1}\,) \,=\, \beta(\,m_{2}\,) \,\in\, D.
\end{eqnarray*}
This establishes that \,$\alpha : N \longrightarrow D$\, is indeed set-theoretically well-defined.
It is routine to show that \,$\alpha$\, is furthermore an $R$-module homomorphism.
By construction/definition of \,$\alpha$,\, we have
\begin{equation*}
\beta(\,m\,) \; = \; \alpha(\varphi(m)),
\quad
\textnormal{for each \,$m \in M$},
\end{equation*}
i.e. \,$\beta = \alpha \circ \varphi = \varphi^{\prime}(\alpha) \in \image(\varphi^{\prime})$.\,
This proves that
\,$\ker(\,\psi^{\prime}\,) \subset \image(\,\varphi^{\prime}\,)$,\,
and hence
\,$\image(\,\varphi^{\prime}\,) = \ker(\,\psi^{\prime}\,)$,\,
i.e. the exactness of \eqref{inducedExactSequence} at \,$\Hom_{R}(M,D)$.\,
\qed

          %%%%% ~~~~~~~~~~~~~~~~~~~~ %%%%%

\vskip 0.5cm
\begin{remark}
\mbox{}
\vskip 0.1cm
\noindent
Consider the following diagram:
\begin{center}
\begin{tikzcd}
0 \arrow[r]
& L \arrow[r, "\psi"] \arrow[d, "\xi{\color{white}.}" swap]
& M \arrow[r, "\varphi"] \arrow[dr] \arrow[dl, dashrightarrow, "\exists{\color{white}.}\eta{\color{white}.}?", red]
& N \arrow[d, "{\color{white}.}\zeta_{1}\textnormal{,\,}\zeta_{2}"] \arrow[r]
& 0
\\
& D
&
& D
\end{tikzcd}\end{center}
\begin{itemize}
\item
	The exactness of \eqref{inducedExactSequence} at \,$\Hom_{R}(N,D)$
	is equivalent to the injectivity of $\varphi^{\prime}$,
	which is a consequence of the surjectivity of $\varphi$.
\item
	The exactness of \eqref{inducedExactSequence} at \,$\Hom_{R}(M,D)$
	is equivalent to \,$\psi^{\prime}\circ\varphi^{\prime} = 0$,\,
	which follows from the exactness of \eqref{givenShortExactSequence} at \,$M$
	(which in turn is equivalent to \,$\varphi \circ \psi = 0$),
	since
	\begin{equation*}
	(\,\psi^{\prime}\circ\varphi^{\prime}\,)(\zeta)
	\;\; = \;\;
		\psi^{\prime}(\,\varphi^{\prime}(\zeta)\,)
	\;\; = \;\;
		\psi^{\prime}(\,\zeta \circ \varphi\,)
	\;\; = \;\;
		(\,\zeta \circ \varphi\,) \circ \psi
	\;\; = \;\;
		\zeta \circ (\,\varphi \circ \psi\,),
	\quad
		\textnormal{for each \,$\zeta \in \Hom_{R}(N,D)$}
	\end{equation*}
	as mentioned in the proof of Proposition \ref{homologicalAlgebraMotivation}.
\item
	The induced exact sequence \eqref{inducedExactSequence}
	in general cannot be extended on the right to a short exact sequence.
	Indeed, this extendibility is equivalent to the surjectivity of $\psi^{\prime}$,
	which in turn is equivalent to the following statement:
	\begin{center}
	\begin{minipage}{6.0in}
	\textnormal{For each $R$-module homomorphism
	\,$L\,\overset{\xi}{\longrightarrow}\,D$,\,
	there exists an $R$-module homomorphism
	\,$M\,\overset{\eta}{\longrightarrow}\,D$,\,
	such that
	\,$\xi \,=\, \psi^{\prime}(\eta) \,=\, \eta \circ \psi$
	}
	\end{minipage}
	\end{center}
	And, the above statement is clearly not true in general.
\end{itemize}
Homological algebra defines, and provides tools to compute, algebraic objects
that measure the degree to which
\eqref{inducedExactSequence}
{\color{red}fails to extend to a short exact sequence};
more precisely, the aforementioned algebraic objects are abelian groups
that extend \eqref{inducedExactSequence} into a long exact sequence:
\begin{center}
\begin{tikzcd}
0 \arrow[r]
& \Hom_{R}(N,D) \arrow[r, "\varphi^{\prime}"]
& \Hom_{R}(M,D) \arrow[r, "\psi^{\prime}"] \arrow[d, phantom, ""{coordinate, name=Z}]
& \Hom_{R}(L,D) 
\arrow[dll,
rounded corners,
to path={ -- ([xshift=2ex]\tikztostart.east)
|- (Z) [near end]\tikztonodes
-| ([xshift=-2ex]\tikztotarget.west)
-- (\tikztotarget)}]
\\
& {\color{white}.}\Ext^{1}_{R}(N,D){\color{white}.} \arrow[r]
& {\color{white}.}\Ext^{1}_{R}(M,D){\color{white}.} \arrow[r] \arrow[d, phantom, ""{coordinate, name=Z}]
& {\color{white}.}\Ext^{1}_{R}(L,D){\color{white}.} 
\arrow[dll,
rounded corners,
to path={ -- ([xshift=2ex]\tikztostart.east)
|- (Z) [near end]\tikztonodes
-| ([xshift=-2ex]\tikztotarget.west)
-- (\tikztotarget)}]
\\
& {\color{white}.}\Ext^{2}_{R}(N,D){\color{white}.} \arrow[r]
& {\color{white}.}\Ext^{2}_{R}(M,D){\color{white}.} \arrow[r] \arrow[d, phantom, ""{coordinate, name=Z}]
& {\color{white}.}\Ext^{2}_{R}(L,D){\color{white}.}
\arrow[dll,
rounded corners,
to path={ -- ([xshift=2ex]\tikztostart.east)
|- (Z) [near end]\tikztonodes
-| ([xshift=-2ex]\tikztotarget.west)
-- (\tikztotarget)}]
\\
& {\color{white}.}\Ext^{3}_{R}(N,D){\color{white}.} \arrow[r]
& {\color{white}.}\cdots\;\;\cdots\;\;\cdots{\color{white}.} \arrow[r, white]
& {\color{white}.\Ext^{3}_{R}(N,D).}
\end{tikzcd}\end{center}
Consequently, the non-vanishing of the group
\,$\Ext^{1}_{R}(N,D)$\,
can therefore be regarded as {\color{red}obstruction} to
\begin{itemize}
\item
	the extendibility of \eqref{inducedExactSequence} to a short exact sequence, or equivalently
\item
	the surjectivity \,$\psi^{\prime}$\, in \eqref{inducedExactSequence}, or equivalently
\item
	the lifting of an arbitrary \,$\gamma \in \Hom_{R}(L,D)$\, to an element in $\Hom_{R}(M,D)$ via \,$\psi$,\,
	i.e. the existence of $\beta \in \Hom_{R}(M,D)$ such that $\gamma = \beta \circ \psi = \psi^{\prime}(\beta)$.
\end{itemize}
Put yet in another way, an element \,$\gamma \in \Hom_{R}(L,D)$\, can be lifted via \,$\psi$
to an element in $\Hom_{R}(M,D)$ if and only if $\gamma$ maps to zero in
\,$\Ext^{1}_{R}(N,D)$.\,
\end{remark}

          %%%%% ~~~~~~~~~~~~~~~~~~~~ %%%%%
