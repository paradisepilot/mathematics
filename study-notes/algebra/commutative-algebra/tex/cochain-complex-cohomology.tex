
          %%%%% ~~~~~~~~~~~~~~~~~~~~ %%%%%

\section{The long exact sequence in cohomology of a short exact sequence of cochain complexes}
\setcounter{theorem}{0}
\setcounter{equation}{0}

%\cite{vanDerVaart1996}
%\cite{Kosorok2008}

%\renewcommand{\theenumi}{\alph{enumi}}
%\renewcommand{\labelenumi}{\textnormal{(\theenumi)}$\;\;$}
\renewcommand{\theenumi}{\roman{enumi}}
\renewcommand{\labelenumi}{\textnormal{(\theenumi)}$\;\;$}

          %%%%% ~~~~~~~~~~~~~~~~~~~~ %%%%%

\begin{definition}
\mbox{}
\vskip 0.1cm
\noindent
Let \,$\mathcal{C}$\, be a sequence of abelian group homomorphisms:
\begin{equation*}
0
\;\; \longrightarrow \;\;
	C^{0}
\;\; \overset{d_{1}}{\longrightarrow} \;\;
	C^{1}
\;\; \overset{d_{2}}{\longrightarrow} \;\;
	C^{2}
\;\; \longrightarrow \;\;
	\cdots\cdots
\;\; \longrightarrow \;\;
	C^{n-1}
\;\; \overset{d_{n}}{\longrightarrow} \;\;
	C^{n}
\;\; \overset{d_{n+1}}{\longrightarrow} \;\;
\;\; \longrightarrow \;\;
	\cdots\cdots
\end{equation*}
\begin{itemize}
\item
	The sequence \,$\mathcal{C}$\, is called a \textbf{cochain complex} if
	\,$d_{n+1} \circ d_{n} \,=\, 0$,\, for each \,$n = 1, 2, \ldots$.\,
\item
	For a cochain complex \,$\mathcal{C}$,\, its \,\textbf{$n^{\textnormal{th}}$ cohomology group}
	is the quotient group
	\begin{equation*}
	H^{n}(\,\mathcal{C}\,)
	\; := \;
		\left.\overset{{\color{white}.}}{\ker(d_{n+1})}\right/\image(d_{n})
	\end{equation*}
\end{itemize}
\end{definition}

          %%%%% ~~~~~~~~~~~~~~~~~~~~ %%%%%

\begin{definition}
\mbox{}
\vskip 0.1cm
\noindent
Let
\,$\mathcal{A} = \{\,A^{n}\,\}$\,
and
\,$\mathcal{B} = \{\,B^{n}\,\}$\,
be two cochain complexes.
A \textbf{homomorphism of cochain complexes}
\,$\alpha : \mathcal{A} \longrightarrow \mathcal{B}$\,
is a set of homomorphisms
\,$\alpha_{n} : A^{n} \longrightarrow B^{n}$\,
such that for each $n = 0, 1, 2, \ldots$\,,
the following diagram commutes:
\begin{center}
\begin{tikzcd}
\cdots\cdots{\color{white}...} \arrow[r]
& A^{n-1} \arrow[r, "a_{n}"] \arrow[d, "\alpha_{n-1}{\color{white}.}" swap]
& A^{n} \arrow[r, "a_{n+1}"] \arrow[d, "\alpha_{n}{\color{white}.}" swap]
& A^{n+1} \arrow[r] \arrow[d, "{\color{white}.}\alpha_{n+1}"]
& {\color{white}...}\cdots\cdots
\\
\cdots\cdots{\color{white}...} \arrow[r]
& B^{n-1} \arrow[r, "b_{n}" swap]
& B^{n} \arrow[r, "b_{n+1}" swap]
& B^{n+1} \arrow[r]
& {\color{white}...}\cdots\cdots
\end{tikzcd}
\end{center}
\end{definition}

          %%%%% ~~~~~~~~~~~~~~~~~~~~ %%%%%

\begin{proposition}
\mbox{}
\vskip 0.1cm
\noindent
A homomorphism \,$\alpha : \mathcal{A} \longrightarrow \mathcal{B}$\,
of cochain complexes induces group homomorphisms from
\,$H^{n}(\mathcal{A})$\, to \,$H^{n}(\mathcal{B})$,\, for each \,$n = 0, 1, 2, \ldots$\,,
on their respective cohomology groups.
\end{proposition}
\proof
We need to define the map \,$\widehat{\alpha}_{n}$\, induced by \,$\alpha_{n}$:\,
\begin{equation*}
\dfrac{\ker(a_{n+1})}{\image(a_{n})}
\; =: \;
	H^{n}(\,\mathcal{A}\,)
\;\; \overset{\widehat{\alpha}_{n}}{\longrightarrow} \;\;
	H^{n}(\,\mathcal{B}\,)
\; := \;
	\dfrac{\ker(b_{n+1})}{\image(b_{n})}
\end{equation*}
To this end, we define
\begin{equation*}
\widehat{\,\alpha}_{n}\!\left(\,x_{n} \overset{{\color{white}.}}{+} \image(a_{n+1})\,\right)
\;\; := \;\;
	\alpha_{n}(\,x_{n}\,) \,+\, \image(b_{n+1})\,,
\quad
\textnormal{for each \,$x_{n} \in \ker(a_{n+1}) \subset A^{n}$}
\end{equation*}
We need to establish the well-definition of \,$\widehat{\alpha}_{n}$.\,
To this end, suppose
\,$x_{n},\, x_{n}^{\prime} \in \ker(a_{n+1}) \subset A^{n}$\,
are such that
\,$x_{n} + \image(a_{n}) \,=\, x_{n}^{\prime} + \image(a_{n})$,\,
equivalently, \,$x_{n} - x_{n}^{\prime} \,\in\, \image(a_{n}) \subset \ker(a_{n+1})$.\,
We need to show that
\,$\alpha_{n}(x_{n}) + \image(b_{n+1}) \,=\, \alpha_{n}(x_{n}^{\prime}) + \image(b_{n})$,\,
equivalently, \,$\alpha_{n}(x_{n}) - \alpha_{n}(x_{n}^{\prime}) \,\in\, \image(b_{n})$.\,

\vskip 0.25cm
\noindent
Now, observe:
\begin{eqnarray*}
x_{n} - x_{n}^{\prime} \,\in\, \image(a_{n})
& \Longleftrightarrow &
	x_{n} - x_{n}^{\prime} \; = \; a_{n}(z_{n-1}),
	\quad
	\textnormal{for some \,$z_{n-1} \in A^{n-1}$}
\\
& \overset{{\color{white}1}}{\Longrightarrow} &
	\alpha_{n}\!\left(\;x_{n} \overset{{\color{white}.}}{-} x_{n}^{\prime}\,\right)
	\; = \;
		(\,{\color{red}\alpha_{n} \,\circ\, a_{n}}\,)\!\left(\,\overset{{\color{white}-}}{z_{n-1}}\,\right)
	\; = \;
		(\,{\color{red}b_{n} \,\circ\, \alpha_{n-1}}\,)\!\left(\,\overset{{\color{white}-}}{z_{n-1}}\,\right)
	\; \in \;
		\image(\,b_{n}\,),
\end{eqnarray*}
as required.
\qed

          %%%%% ~~~~~~~~~~~~~~~~~~~~ %%%%%

\begin{definition}
\mbox{}
\vskip 0.1cm
\noindent
Let
\,$\mathcal{A} = \{\,A^{n}\,\}$,\,
\,$\mathcal{B} = \{\,B^{n}\,\}$,\,
and
\,$\mathcal{C} = \{\,C^{n}\,\}$\,
be cochain complexes.
A \textbf{short exact sequences of cochain complexes}
\,$0 \,\longrightarrow\, \mathcal{A} \,\overset{\alpha}{\longrightarrow}\, \mathcal{B} \,\overset{\beta}{\longrightarrow}\, \mathcal{C} \,\longrightarrow\, 0$\,
is a sequence of homomorphisms of cochain complexes such that
\begin{equation*}
0
\;\; \longrightarrow \;\;
	A^{n}
\;\; \overset{\alpha_{n}}{\longrightarrow} \;\;
	B^{n}
\;\; \overset{\beta_{n}}{\longrightarrow} \;\;
	C^{n}
\,\longrightarrow\,
	0
\end{equation*}
is a short exact sequence of abelian groups, for each \,$n = 0, 1, 2, \ldots$\,.
\end{definition}

          %%%%% ~~~~~~~~~~~~~~~~~~~~ %%%%%

\begin{theorem}
\mbox{}
\vskip 0.1cm
\noindent
Suppose
\,$0 \,\longrightarrow\, \mathcal{A} \,\overset{\alpha}{\longrightarrow}\, \mathcal{B} \,\overset{\beta}{\longrightarrow}\, \mathcal{C} \,\longrightarrow\, 0$\,
is a short exact sequence of cochain complexes.
Then, there exists a long exact sequence of cohomology groups:
\begin{center}
\begin{tikzcd}
0 \arrow[r]
& {\color{white}.}H^{0}(\mathcal{A}){\color{white}.} \arrow[r,"\widehat{\alpha}_{0}"]
& {\color{white}.}H^{0}(\mathcal{B}){\color{white}.} \arrow[r,"\widehat{\beta}_{0}"] \arrow[d, phantom, ""{coordinate, name=Z}]
& {\color{white}.}H^{0}(\mathcal{C}){\color{white}.} 
%\arrow[dll, "\delta_{0}" swap,
%rounded corners,
%to path={ -- ([xshift=2ex]\tikztostart.east)
%|- (Z) [near end]\tikztonodes
%-| ([xshift=-2ex]\tikztotarget.west)
%-- (\tikztotarget)}]
\arrow[dll, %"\delta_{0}" swap,
rounded corners,
to path={
	-- ([xshift=2ex]\tikztostart.east)
	|- (Z) [near end]\tikztonodes
	-| ([xshift=-2ex]\tikztotarget.west)
	-- (\tikztotarget)
	}]
\\
& {\color{white}.}H^{1}(\mathcal{A}){\color{white}.} \arrow[r,"\widehat{\alpha}_{1}"]
& {\color{white}.}H^{1}(\mathcal{B}){\color{white}.} \arrow[r,"\widehat{\beta}_{1}"] \arrow[d, phantom, ""{coordinate, name=Z}]
& {\color{white}.}H^{1}(\mathcal{C}){\color{white}.}
\arrow[dll,
rounded corners,
to path={
	-- ([xshift=2ex]\tikztostart.east)
	|- (Z) [near end]\tikztonodes
	-| ([xshift=-2ex]\tikztotarget.west)
	-- (\tikztotarget)
	}]
\\
& {\color{white}.}H^{2}(\mathcal{A}){\color{white}.} \arrow[r,"\widehat{\alpha}_{2}"]
& {\color{white}.}H^{2}(\mathcal{B}){\color{white}.} \arrow[r,"\widehat{\beta}_{2}"] \arrow[d, phantom, ""{coordinate, name=Z}]
& {\color{white}.}H^{2}(\mathcal{C}){\color{white}.}
\arrow[dll,
rounded corners,
to path={
	-- ([xshift=2ex]\tikztostart.east)
	|- (Z) [near end]\tikztonodes
	-| ([xshift=-2ex]\tikztotarget.west)
	-- (\tikztotarget)
	}]
\\
& {\color{white}.}H^{3}(\mathcal{A}){\color{white}.} \arrow[r]
& {\color{white}.}\cdots\;\;\cdots\;\;\cdots{\color{white}.} \arrow[r, white]
& {\color{white}.\Ext^{3}_{R}(N,D).}
\end{tikzcd}\end{center}
where the maps
\begin{equation*}
H^{n}(\mathcal{C}) \; \overset{\delta_{n}}{\longrightarrow} \; H^{n+1}(\mathcal{A})\,,
\quad
\textnormal{for \,$n = 0, 1, 2, \ldots\,$},
\end{equation*}
are called the connecting homomorphisms.
\end{theorem}
\proof
The short exact sequence of cochain complexes
\,$0 \,\longrightarrow\, \mathcal{A} \,\overset{\alpha}{\longrightarrow}\, \mathcal{B} \,\overset{\beta}{\longrightarrow}\, \mathcal{C} \,\longrightarrow\, 0$\,
amounts to the following commutative diagram
whose columns are short exact sequences of
abelian groups:
\begin{center}
\begin{tikzcd}
& 0 \arrow[d]
& 0 \arrow[d]
& 0 \arrow[d]
&
\\
\cdots\cdots{\color{white}...} \arrow[r]
& A^{n-1} \arrow[r, "a_{n}"] \arrow[d, "\alpha_{n-1}{\color{white}.}" swap]
& A^{n} \arrow[r, "a_{n+1}"] \arrow[d, "\alpha_{n}{\color{white}.}" swap]
& A^{n+1} \arrow[r] \arrow[d, "{\color{white}.}\alpha_{n+1}"]
& {\color{white}...}\cdots\cdots
\\
\cdots\cdots{\color{white}...} \arrow[r]
& B^{n-1} \arrow[r, "b_{n}"] \arrow[d, "\beta_{n-1}{\color{white}.}" swap]
& B^{n} \arrow[r, "b_{n+1}"] \arrow[d, "\beta_{n}{\color{white}.}" swap]
& B^{n+1} \arrow[r] \arrow[d, "{\color{white}.}\beta_{n+1}"]
& {\color{white}...}\cdots\cdots
\\
\cdots\cdots{\color{white}...} \arrow[r]
& C^{n-1} \arrow[d] \arrow[r, "c_{n}"]
& C^{n} \arrow[d] \arrow[r, "c_{n+1}"]
& C^{n+1} \arrow[d] \arrow[r]
& {\color{white}...}\cdots\cdots
\\
& 0 
& 0 
& 0
&
\end{tikzcd}
\end{center}
\vskip 0.25cm
\noindent
Let \,$\zeta_{n}$\, 
$\in$
\,$H^{n}(\mathcal{C}) := \left.\ker(\overset{{\color{white}.}}{c_{n+1}})\right/\image(c_{n})$.\,
\vskip 0.25cm
\noindent
\textbf{Claim 1:}\;\;
For each representative
\,$z_{n} \in \ker(c_{n+1}) \subset C^{n}$\,
of
\,$\zeta_{n} = \left[\,z_{n}\,\right]$,\,
there exists $y_{n} \in B^{n}$
such that $\beta_{n}(y_{n}) = z_{n}$.
\vskip 0.2cm
\noindent
Proof of Claim 1:
Immediate by the hypothesis that
\,$0 \,\longrightarrow\, \mathcal{A} \,\overset{\alpha}{\longrightarrow}\, \mathcal{B} \,\overset{\beta}{\longrightarrow}\, \mathcal{C} \,\longrightarrow\,0$\,
is a short exact sequence of cochain complexes;
in particular, for each $n \in \{\,0,1,2,\ldots\,\}$,
\,$0 \,\longrightarrow\, A^{n} \,\overset{\alpha_{n}}{\longrightarrow}\, B^{n} \,\overset{\beta_{n}}{\longrightarrow}\, C^{n} \,\longrightarrow\,0$\,
is a short exact sequence of abelian groups;
in particular, each $\beta_{n}$ is surjective.
This proves Claim 1.

\vskip 0.5cm
\noindent
\textbf{Claim 2:}\;\;
$b_{n+1}(y_{n}) \in \ker(\beta_{n+1})$,\, and there exists a unique \,$x_{n+1} \in A^{n+1}$\,
such that \,$\alpha_{n+1}(x_{n+1}) = b_{n+1}(y_{n})$.
\vskip 0.2cm
\noindent
Proof of Claim 2:
Claim 1
\;$\Longrightarrow$\;
\,$z_{n} = \beta_{n}(y_{n})$
\;$\Longrightarrow$\;
$0 = c_{n+1}(z_{n}) = c_{n+1}(\beta_{n}(y_{n})) = \beta_{n+1}(b_{n+1}(y_{n}))$,
which proves that $b_{n+1}(y_{n}) \in \ker(\beta_{n+1}) = \image(\alpha_{n+1})$.
Hence, there exists a unique $x_{n+1} \in A^{n+1}$ such that
$\alpha_{n+1}(x_{n+1}) = b_{n+1}(y_{n})$, by short-exactness.
This proves Claim 2.

\vskip 0.5cm
\noindent
\textbf{Claim 3:}\;\;
$x_{n+1} \in  \ker(a_{n+2}) \subset A^{n+1}$;\,
hence \,$x_{n+1} \in A^{n+1}$\, defines a class
\,$\xi_{n+1} = \left[\,x_{n+1}\,\right]$\,
$\in$
\,$H^{n+1}(\mathcal{A}) := \left.\ker(\overset{{\color{white}.}}{a_{n+2}})\right/\image(a_{n+1})$.\,
\vskip 0.2cm
\noindent
Proof of Claim 3:
Claim 2
\;$\Longrightarrow$\;
$b_{n+1}(y_{n}) = \alpha_{n+1}(x_{n+1})$
\;$\Longrightarrow$\;
$0 = b_{n+2}(b_{n+1}(y_{n})) = b_{n+2}(\alpha_{n+1}(x_{n+1})) = \alpha_{n+2}(a_{n+2}(x_{n+1}))$
\;$\Longrightarrow$\;
$a_{n+2}(x_{n+1}) = 0$,\, by injectivity of \,$\alpha_{n+2}$.
Thus, we have \,$x_{n+1} \in  \ker(a_{n+2}) \subset A^{n+1}$.\,
This proves Claim 3. 

\vskip 0.5cm
\noindent
\textbf{Claim 4:}\;\;
The class
\,$\xi_{n+1} \,=\, \left[\,x_{n+1}\,\right] \,\in\, H^{n+1}(\mathcal{A})$\,
depends only on the class
\,$\zeta_{n} \,=\, \left[\,z_{n}\,\right] \,\in\, H^{n}(\mathcal{C})$;\,
in particular,
\,$\xi_{n+1} \,=\, \left[\,x_{n+1}\,\right]$\,
is independent of the particular choices of
\,$x_{n+1} \in A^{n+1}$,\, $y_{n} \in B^{n}$,\, $z_{n} \in C^{n}$\,
in Claim 1 and Claim 2.
\vskip 0.2cm
\noindent
Proof of Claim 4:\,
Let
\,$x_{n+1}^{\prime} \in A^{n+1}$,
\,$y_{n}^{\prime} \in B^{n}$,
\,$z_{n}^{\prime} \in C^{n}$\,
be alternative choices.
We need to show
\,$\left[\,x_{n+1}\,\right] = \left[\,x_{n+1}^{\prime}\,\right]$\,
$\in$
\,$H^{n+1}(\mathcal{A}) := \left.\ker(\overset{{\color{white}.}}{a_{n+2}})\right/\image(a_{n+1})$,\,
i.e.,
\,$x_{n+1} - x_{n+1}^{\prime} \in \image(a_{n+1})$.\,
To this end, simply observe that
\begin{eqnarray*}
&&
	\left[\,z_{n}\,\right] \,=\, \zeta_{n} \,=\, \left[\,z_{n}^{\prime}\,\right]
	\,\in\,
		H^{n}(\mathcal{C}) \,:=\, \left.\ker(\overset{{\color{white}.}}{c_{n+1}})\right/\image(c_{n})
\\
& \Longrightarrow &
	z_{n} - z_{n}^{\prime} \,\in\, \image(c_{n})
\;\; \Longleftrightarrow \;\;
	z_{n} - z_{n}^{\prime} \,=\, c_{n}(z_{n-1}),
	\;\;
	\textnormal{for some \,$z_{n-1} \in C^{n-1}$}
\\
& \Longrightarrow &
	\beta_{n}(y_{n} - y_{n}^{\prime}) \,=\, c_{n}(z_{n-1}) \,=\, c_{n}(\beta_{n-1}(y_{n-1})) \,=\, \beta_{n}(b_{n}(y_{n-1})),
	\;\;
	\textnormal{for some \,$y_{n-1} \in C^{n-1}$, by surjectivity of $\beta_{n-1}$}
\\
& \Longrightarrow &
	y_{n} - y_{n}^{\prime} - b_{n}(y_{n-1}) \,\in\, \ker(\beta_{n}) \,=\, \image(\alpha_{n})
\\
& \Longrightarrow &
	y_{n} - y_{n}^{\prime} - b_{n}(y_{n-1}) \,=\, \alpha_{n}(x_{n}),
	\;\;
	\textnormal{for some \,$x_{n} \in A^{n}$}
\\
& \Longrightarrow &
	\alpha_{n+1}(x_{n+1} - x_{n+1}^{\prime})
	\,=\,
		b_{n+1}(y_{n} - y_{n}^{\prime}) - b_{n+1}(b_{n}(y_{n-1}))
	\,=\,
		b_{n+1}(\alpha_{n}(x_{n}))
	\,=\,
		\alpha_{n+1}(a_{n+1}(x_{n}))
\\
& \Longrightarrow &
	x_{n+1} - x_{n+1}^{\prime} \,=\, a_{n+1}(x_{n}) \,\in\, \image(a_{a+1}),
	\;\;
	\textnormal{since \,$b_{n+1} \circ b_{n} = 0$\, and \,$\alpha_{n+1}$\, is injective}
\end{eqnarray*}
This proves Claim 4.

\vskip 0.5cm
\noindent
By Claim 4, we may now define, for each $n \in \{0,1,2,\ldots\}$, the connecting homomorphism:
\begin{equation*}
\delta_{n} : H^{n}(\mathcal{C}) \longrightarrow H^{n+1}(\mathcal{A})
\quad\textnormal{via}\quad
\delta_{n}\!\left(\,[\,\overset{{\color{white}-}}{z_{n}}\,]\,\right)
\;\; := \;\;
	[\,x_{n+1}\,]
\;\; \in \;\;
	H^{n+1}(\mathcal{A})\,,
\;\;
\textnormal{for each \,$[\,z_{n}\,] \in H^{n}(\mathcal{C})$}
\end{equation*}

\vskip 0.5cm
\noindent
\textbf{Claim 5:}\;\;
For each \,$n \in \{0, 1, 2, \ldots\}$,\,
\,$\delta_{n} : H^{n}(\mathcal{C}) \longrightarrow H^{n+1}(\mathcal{A})$\,
is an abelian group homomorphism.
\vskip 0.2cm
\noindent
Proof of Claim 5:\,
We need to show:
\begin{equation*}
\delta_{n}\!\left(\,[\,z_{n}\,] \overset{{\color{white}-}}{+} [\,z_{n}^{\prime}\,]\,\right)
\;\; = \;\;
	[\,x_{n+1}\,] \,+\, [\,x_{n+1}^{\prime}\,]
\end{equation*}
where $x_{n+1}$ (respectively, $x_{n+1}^{\prime}$) is determined
by $z_{n}$ (respectively, $z_{n}^{\prime}$) as described in Claim 1 and Claim 2.
To this end, first observe that
\,$[\,z_{n}\,] + [\,z_{n}^{\prime}\,]$\,
$=$
\,$[\,z_{n} \,+\, z_{n}^{\prime}\,]$\,
$=$
\,$[\,z_{n}^{\prime\prime}\,]$,\,
where
\,$z_{n}^{\prime\prime} \,:=\, z_{n} \,+\, z_{n}^{\prime}$.\,
Thus, we may choose
\,$y_{n}^{\prime\prime} \,:=\, y_{n} \,+\, y_{n}^{\prime}$\,
so that
\,$\beta_{n}(y_{n}^{\prime\prime})$
\,$=$\, $\beta_{n}(y_{n} \,+\, y_{n}^{\prime})$
\,$=$\, $\beta_{n}(y_{n}) \,+\, \beta_{n}(y_{n}^{\prime})$
\,$=$\, $z_{n} + z_{n}^{\prime} \,=\, z_{n}^{\prime\prime}$.\,
Next,
\,$b_{n+1}(y_{n}^{\prime\prime})$
\,$=$\, $b_{n+1}(y_{n} + y_{n}^{\prime})$
\,$=$\, $b_{n+1}(y_{n}) + b_{n+1}(y_{n}^{\prime})$
\,$=$\, $\alpha_{n+1}(x_{n+1}) + \alpha_{n+1}(x_{n+1}^{\prime})$
\,$=$\, $\alpha_{n+1}(x_{n+1} + x_{n+1}^{\prime})$\,
implies that
\,$x_{n+1}^{\prime\prime} := x_{n+1} + x_{n+1}^{\prime}$\,
is the unique element in \,$A^{n+1}$\, determined by \,$y_{n}^{\prime\prime} \in B^{n}$\,
according to Claim 2.
Hence,
\begin{eqnarray*}
\delta_{n}\!\left(\,[\,z_{n}\,] \overset{{\color{white}-}}{+} [\,z_{n}^{\prime}\,]\,\right)
& = &
	\delta_{n}\!\left(\,[\,z_{n} \overset{{\color{white}-}}{+} z_{n}^{\prime}\,]\,\right)
\;\; = \;\;
	\delta_{n}\!\left(\,[\,\overset{{\color{white}-}}{z_{n}^{\prime\prime}}\,]\,\right)
\;\; = \;\;
	\left[\,\overset{{\color{white}.}}{x_{n+1}^{\prime\prime}}\,\right]
\;\; = \;\;
	\left[\,x_{n+1} \,+\, \overset{{\color{white}.}}{x_{n+1}^{\prime}}\,\right]
\\
& = &
	\left[\,\overset{{\color{white}.}}{x_{n+1}^{{\color{white}\prime}}}\,\right]
	\, + \,
	\left[\,\overset{{\color{white}.}}{x_{n+1}^{\prime}}\,\right]
\end{eqnarray*}
as required.
This proves Claim 5.

\vskip 0.5cm
\noindent
\textbf{Claim 6:}\;\;
Exactness of
\,$H^{n}(\mathcal{A})$
\,$\overset{\widehat{\alpha}_{n}}{\longrightarrow}$\,
$H^{n}(\mathcal{B})$
\,$\overset{\widehat{\beta}_{n}}{\longrightarrow}$\,
$H^{n+1}(\mathcal{C})$,\;
i.e.
\,$\image\!\left(\,\overset{{\color{white}.}}{\widehat{\alpha}_{n}}\,\right) \,=\, \ker\!\left(\,\widehat{\beta}_{n}\,\right)$.\,
\vskip 0.2cm
\noindent
Proof of Claim 6:\,


\vskip 0.5cm
\noindent
\textbf{Claim 7:}\;\;
Exactness of
\,$H^{n}(\mathcal{B})$
\,$\overset{\widehat{\beta}_{n}}{\longrightarrow}$\,
$H^{n}(\mathcal{C})$
\,$\overset{\delta_{n}}{\longrightarrow}$\,
$H^{n+1}(\mathcal{A})$,\;
i.e.
\,$\image\!\left(\,\widehat{\beta}_{n}\,\right) \,=\, \ker\!\left(\,\overset{{\color{white}.}}{\delta}_{n}\,\right)$.\,
\vskip 0.2cm
\noindent
Proof of Claim 7:\,


\vskip 0.5cm
\noindent
\textbf{Claim 8:}\;\;
Exactness of
\,$H^{n}(\mathcal{C})$
\,$\overset{\delta_{n}}{\longrightarrow}$\,
$H^{n+1}(\mathcal{A})$
\,$\overset{\widehat{\alpha}_{n+1}}{\longrightarrow}$\,
$H^{n+1}(\mathcal{B})$,\;
i.e.
\,$\image\!\left(\,\overset{{\color{white}.}}{\delta}_{n}\,\right) \,=\, \ker\!\left(\,\overset{{\color{white}.}}{\widehat{\alpha}_{n+1}}\,\right)$.\,
\vskip 0.2cm
\noindent
Proof of Claim 8:\,


\qed

          %%%%% ~~~~~~~~~~~~~~~~~~~~ %%%%%
