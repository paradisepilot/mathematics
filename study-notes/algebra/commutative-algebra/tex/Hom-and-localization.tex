
          %%%%% ~~~~~~~~~~~~~~~~~~~~ %%%%%

\section{Hom and localiazation}
\setcounter{theorem}{0}
\setcounter{equation}{0}

%\cite{vanDerVaart1996}
%\cite{Kosorok2008}

%\renewcommand{\theenumi}{\alph{enumi}}
%\renewcommand{\labelenumi}{\textnormal{(\theenumi)}$\;\;$}
\renewcommand{\theenumi}{\roman{enumi}}
\renewcommand{\labelenumi}{\textnormal{(\theenumi)}$\;\;$}

          %%%%% ~~~~~~~~~~~~~~~~~~~~ %%%%%

\begin{proposition}[Proposition 2.10, \S2.2, p.69, \cite{eisenbud1995commutative}]
\label{HomAndLocalization}
\mbox{}
\vskip 0.1cm
\noindent
Suppose that \,$R$\, is a commutative ring with multiplicative identity, \,$S$\, is an $R$-algebra, and \,$M$,\, $N$\, are \,$R$-modules.
Then, the following statements are true:
\begin{enumerate}
\item
	There is a unique $S$-module homomorphism
	\begin{equation*}
	\alpha : S \otimes_{R} \Hom_{R}(M,N) \longrightarrow \Hom_{S}\!\left(\,S\otimes_{R}M\,,\,S\otimes_{R}N\,\right)
	\end{equation*}
	that takes an element
	\,$1_{S}\,\otimes\,\varphi \in S \otimes_{R} \Hom_{R}(M, N)$\,
	to the $S$-module homomorphism
	\,$\id_{S} \,\otimes_{R}\, \varphi : S \otimes_{R} M \longrightarrow S \otimes_{R} N$\,
	in
	\,$\Hom_{S}(S \otimes_{R} M, S \otimes_{R} N)$.
%	\begin{equation*}
%	\left(\,\id_{S} \overset{{\color{white}.}}{\otimes_{R}} \varphi\,\right)\!\left(\,s \otimes_{R} m\,\right)
%	\;\; = \;\;
%		s \otimes_{R} \varphi(m)
%	\end{equation*}
%	\begin{center}
%	\begin{tikzcd}
%	A
%		\arrow[rr, "f"]
%		\arrow[dr, swap, "\nu"]
%		\arrow[ddr, bend right, swap, "\nu^{\prime}"]
%	&&
%	B
%	\\
%	& {\color{red}I}
%		\arrow[ur, hook, swap, "\iota", red]
%		\arrow[d, dashed, "\;\exists ! \, \theta"]
%	\\
%	& I^{\prime}
%		\arrow[uur, bend right, hook, swap, "\iota^{\prime}"]
%	&
%	\end{tikzcd}
%	\end{center}
\item
	If $S$ is flat over $R$ and $M$ is finitely presented, then $\alpha$ is an isomorphism.
\item
	In particular, if $M$ is finitely presented, then $\Hom_{R}(M, N)$ localizes in the
	sense that the map $\alpha$ provides a natural isomorphism
	\begin{equation*}
	D^{-1}\,\Hom_{R}\!\left(\,M\,,\,N\,\right)
	\;\;\cong\;\;
		\Hom_{D^{-1}R}\!\left(\,D^{-1}M\,,\,D^{-1}N\,\right)
	\end{equation*}
	for any subset $D \subset R$.
\end{enumerate}
\end{proposition}
\proof

\begin{enumerate}
\item
	It is clear that the set-theoretic map
	\begin{equation*}
	\alpha^{\prime} \;:\; \Hom_{R}(M,N)
		\;\; \longrightarrow \;\;
		\Hom_{S}\!\left(\, S \otimes_{R} M \,,\, S \otimes_{R} N \,\right)
	\;\;:\;\;
		\varphi \;\; \longmapsto \;\; \id_{S} \,\otimes_{R}\, \varphi
	\end{equation*}
	is an $R$-module homomorphism.
	The map
	\,$\alpha : S \otimes_{R} \Hom_{R}(M,N) \longrightarrow \Hom_{S}\!\left(\,S\otimes_{R}M\,,\,S\otimes_{R}N\,\right)$\,
	is obtained from \,$\alpha^{\prime}$\, by extending its domain from
	\,$\Hom_{R}(M,N)$\, to \,$S \otimes_{R} \Hom_{R}(M,N)$.\,
	The following commutative diagram describes the relation between \,$\alpha$\, and \,$\alpha^{\prime}$:\,
	\begin{center}
	\begin{tikzcd}
	\varphi
		\arrow[rr, maps to]
	&
	&
	\id_{S} \otimes \varphi
	\\
	\Hom_{R}(M,N)
		\arrow[rr, "\alpha^{\prime}"]
		\arrow[dd]
	&
	&
		\Hom_{S}\!\left(\,S\otimes_{R}M\,,\,S\otimes_{R}N\,\right)
		\arrow[dd, equal]
	\\ \\
	S\,\otimes_{R}\Hom_{R}(M,N)
		\arrow[rr, swap, "\alpha"]
	&
	&
		\Hom_{S}\!\left(\,S\otimes_{R}M\,,\,S\otimes_{R}N\,\right)
	\\
	1_{S}\,\otimes\,\varphi
		\arrow[rr, maps to]
	&
	&
	\id_{S} \otimes \varphi
	\end{tikzcd}
	\end{center}
	Since \,$\alpha$\, is an $S$-module homomorphism, its value on a general element
	\,$\xi \,\otimes\, \varphi \,\in\, S \,\otimes_{R}\, \Hom_{R}(M,N)$\,
	is determined by \,$\alpha^{\prime}$\, via:
	\begin{eqnarray*}
	\alpha\!\left(\, \xi \,\overset{{\color{white}.}}{\otimes}\, \varphi \,\right)
	& = &
		\alpha\!\left(\, (\,\xi \cdot 1_{S}) \,\overset{{\color{white}.}}{\otimes}\, \varphi \,\right)
	\;\; = \;\;
		\alpha\!\left(\, \xi \cdot (\,1_{S} \,\overset{{\color{white}.}}{\otimes}\, \varphi\,) \,\right)
	\;\; = \;\;
		\xi \,\cdot\, \alpha\!\left(\, 1_{S} \,\overset{{\color{white}.}}{\otimes}\, \varphi \,\right)
	\;\; = \;\;
		\xi \cdot \left(\, \id_{S} \,\overset{{\color{white}.}}{\otimes}\, \varphi \,\right)
	\\
	& = &
		\xi(\,\cdot\,) \,\overset{{\color{white}.}}{\otimes}\, \varphi\,,
	\end{eqnarray*}
	where \,$\xi(\,\cdot\,) : S \longrightarrow S$\, denotes multiplication by \,$\xi \in S$\, on \,$S$,\,
	and the last equality follows from the following observation:
	\begin{equation*}
	\left[\;\xi \cdot \left(\, \id_{S} \,\overset{{\color{white}.}}{\otimes}\, \varphi \,\right)\,\right](\,s \,\otimes\, m\,)
	\;\; = \;\;
		\xi \cdot \left(\, \id_{S}(s) \,\overset{{\color{white}.}}{\otimes}\, \varphi(m) \,\right)
	\;\; = \;\;
		\xi \cdot \left(\, s \,\overset{{\color{white}.}}{\otimes}\, \varphi(m) \,\right)
	\;\; = \;\;
		\xi(s) \,\overset{{\color{white}.}}{\otimes}\, \varphi(m)
	\end{equation*}
	\vskip 0.3cm
%	$\alpha^{\prime}(\varphi)(s \otimes m)$
%	\,$=$\, $\left(\,\id_{S}\overset{{\color{white}.}}{\otimes}\varphi\,\right)(s \otimes m)$
%	\,$=$\, $\left(\,\id_{S}(\overset{{\color{white}1}}{s})\,\right) \otimes \left(\, \varphi(\overset{{\color{white}1}}{m}) \,\right)$
%	\,$=$\, $s \otimes \varphi(m)$	
%	$\left(\,s \overset{{\color{white}.}}{\otimes_{R}} \varphi\,\right)\!(\,m\,)$
%	\,$=$\, $s \otimes_{R} \varphi(m)$.
\item
	Suppose \,$S$\, is flat over \,$R$\, and \,$M$\, is finitely presented \,$R$-module.\,
	We need to show that, in this case, the map \,$\alpha$\, is an isomorphism.
	
	\vskip 0.3cm
	\textbf{Claim 1}:\;\; If \,$M = R$,\, then \,$\alpha$\, is an isomorphism.
	\vskip 0.01cm
	Proof of Claim 1:\;\;
	In this case, we may identify \,$\Hom_{R}(M,N) \,=\, \Hom_{R}(R,N)$\,
	with \,$N$\, by mapping each \,$\varphi \in \Hom_{R}(R,N)$\, to \,$\varphi(1_{R})$.\,
	On the other hand, we have \,$S\,\otimes_{R}M \,=\, S\,\otimes_{R}R  \,\cong\, S$.\,
	Hence,
	\,$\Hom_{S}(\,S\otimes_{R}M\,,\,S\otimes_{R}N\,)$
	\,$\cong$\, $\Hom_{S}(\,S\otimes_{R}R\,,\,S\otimes_{R}N\,)$
	\,$\cong$\, $\Hom_{S}(\,S\,,\,S\otimes_{R}N\,)$
	\,$\cong$\, $S\otimes_{R}N$.\,
	These identifications fit into the following commutative diagram:
	\begin{center}
	\begin{tikzcd}
	S\,\otimes_{R}\Hom_{R}(M,N)
		\arrow[rr, "\alpha"]
		\arrow[dd, equal]
	&
	&
		\Hom_{S}\!\left(\,S\otimes_{R}M\,,\,S\otimes_{R}N\,\right)
		\arrow[d, equal]
	\\
	&
	&
		\Hom_{S}\!\left(\,S\otimes_{R}R\,,\,S\otimes_{R}N\,\right)
		\arrow[d, "\;\cong"]
	\\
	S\,\otimes_{R}\Hom_{R}(R,N)
		\arrow[d, swap, "\cong\;"]
	&
	&
		\Hom_{S}\!\left(\,S\,,\,S\otimes_{R}N\,\right)
		\arrow[d, "\;\cong"]
%	\\
%	&
%	&
%		\Hom_{S}\!\left(\,S\,,\,S\otimes_{R}N\,\right)
	\\
	S\,\otimes_{R}N
		\arrow[rr, swap, "\id"]
	&
	&
		S\,\otimes_{R}N
	\end{tikzcd}
	\end{center}
	The above diagram shows that \,$\alpha$\, corresponds, in this case, to the identify map
	on \,$S \,\otimes_{R}\, N$;\, in particular, we see that \,$\alpha$\, is an isomorphism.
	This completes the proof of Claim 1.

	\vskip 0.3cm
	\textbf{Claim 2}:\;\; If \,$M = \overset{m}{\underset{i =1}{\bigoplus}}\, R$\, is a free \,$R$-module of finite rank \,$m \geq 1$,\,
	(i.e., the direct sum of \,$m$\, copies of \,$R$),
	then \,$\alpha$\, is an isomorphism.
	\vskip 0.01cm
	Proof of Claim 2:\;\;
	Recall that the functions \,$\Hom$\, and \,$\otimes$\, both commute with finite direct sums.
	Consequently, \,$\alpha_{M}$\, decomposes by summands, i.e.,
	\,$\alpha_{M}$ \,$=$\, $\overset{m}{\underset{i=1}{\bigoplus}}\;\alpha_{R}$.\,
	Since each copy of \,$\alpha_{R}$\, is an isomorphism onto its own image (by Claim 1),
	\,$\alpha_{M}$\, itself is an isomorphism.
	This completes the proof of Claim 2.

	\vskip 0.3cm
	\textbf{Claim 3}:\;\; If \,$M$\, is a finitely presented \,$R$-module,
	then \,$\alpha$\, is an isomorphism.
	\vskip 0.01cm
	Proof of Claim 3:\;\;

	\begin{equation}\label{finitePresentationOfM}
	\begin{tikzcd}
	\overset{b}{\underset{i=1}{\oplus}}R
		\arrow[r]
	&
	\overset{a}{\underset{i=1}{\oplus}}R
		\arrow[r]
	&
	M
		\arrow[r]
	&
	0
	\end{tikzcd}
	\end{equation}
	Since tensoring is right-exact, we get the exact sequence:
	\begin{equation}\label{STensorFinitePresentationOfM}
	\begin{tikzcd}
	S\otimes_{R}\left(\,\overset{b}{\underset{i=1}{\oplus}}R\,\right)
		\arrow[r]
	&
	S\otimes_{R}\left(\,\overset{a}{\underset{i=1}{\oplus}}R\,\right)
		\arrow[r]
	&
	S\otimes_{R}M
		\arrow[r]
	&
	0
	\end{tikzcd}
	\end{equation}
	Now, consider the following commutative diagram:
	\begin{equation}\label{stackedExactSequence}
	\begin{tikzcd}
	0
		\arrow[r]
	&
	S \,\otimes_{R}\, \Hom_{R}\!\left(\,M\,,\,N\,\right)
		\arrow[dd, swap, "\alpha\;"]
		\arrow[r]
	&
	S \,\otimes_{R} \Hom_{S}\!\left(\,\overset{a}{\underset{i=1}{\oplus}}R\,,\,N\,\right)
		\arrow[dd, swap, "\cong\;"]
		\arrow[r]
	&
	S \,\otimes_{R} \Hom_{S}\!\left(\,\overset{b}{\underset{i=1}{\oplus}}R\,,\,N\,\right)
		\arrow[dd, "\;\cong"]
	\\
	\\
	0
		\arrow[r]
	&
	\Hom_{S}\!\left(\overset{{\color{white}.}}{S} \otimes_{R}M , S \otimes_{R}M\right)
		\arrow[r]
	&
	\Hom_{S}\!\left(\overset{{\color{white}.}}{S}\otimes_{R}\!\left(\overset{a}{\underset{i=1}{\oplus}}R\right),S\otimes_{R}N\right)
		\arrow[r]
	&
	\Hom_{S}\!\left(\overset{{\color{white}.}}{S}\otimes_{R}\!\left(\overset{b}{\underset{i=1}{\oplus}}R\right),S\otimes_{R}N\right)
	\end{tikzcd}
	\end{equation}
	The top row of  \eqref{stackedExactSequence} is obtained from \eqref{finitePresentationOfM}
	by applying \,$\Hom(\;\cdot\,,N\,)$\, followed by tensoring \,$S\otimes_{R}(\,\cdot\,)$.
	Left-exactness (and contravariance) of \,$\Hom(\;\cdot\,,N\,)$\,
	and flatness of \,$S$\ imply that the top row of \eqref{stackedExactSequence} is exact.

	\vskip 0.2cm
	The bottom row of  \eqref{stackedExactSequence} is obtained from \eqref{STensorFinitePresentationOfM}
	by applying \,$\Hom(\;\cdot\,,S\otimes_{R}N\,)$.\, 
	Left-exactness (and contravariance) of \,$\Hom(\;\cdot\,,S\otimes_{R}N\,)$\,
	implies that the bottom row of \eqref{stackedExactSequence} is exact.

	\vskip 0.2cm
	By Claim 2, the middle and rightmost vertical arrows in \eqref{stackedExactSequence}.

	\vskip 0.2cm
	That \,$\alpha$\, is an isomorphism can be established via a diagram chase.
\end{enumerate}

\vskip 0.25cm
\noindent
\underline{Exactness of \eqref{inducedExactSequence} at $\Hom_{R}(N,D)$, i.e. {\color{red}injectivity of $\varphi^{\prime}$}{\color{white}.}}
\vskip 0.25cm
\noindent
Let $\zeta_{1}, \zeta_{2} \in \Hom_{R}(N,D)$ be such that
$\varphi^{\prime}(\zeta_{1}) = \varphi^{\prime}(\zeta_{2}) \in \Hom_{R}(M,D)$, i.e.
$\zeta_{1} \circ \varphi = \zeta_{2} \circ \varphi$.
We need to show that $\zeta_{1} = \zeta_{2} \in \Hom_{R}(N,D)$.
But, this follows immediately from the {\color{red}surjectivity of} ${\color{red}\varphi} \in \Hom_{R}(M,N)$.


\vskip 0.50cm
\noindent
\underline{Exactness of \eqref{inducedExactSequence} at $\Hom_{R}(M,D)$,\, i.e. $\image(\psi^{\prime}) = \ker(\varphi^{\prime})$}
\vskip 0.25cm
\noindent
We need to show that \,$\image(\,\varphi^{\prime}\,) = \ker(\,\psi^{\prime}\,)$.\,
First, we establish \,$\image(\,\varphi^{\prime}\,) \subset \ker(\,\psi^{\prime}\,)$,\,
equivalently, \,$\psi^{\prime} \circ \varphi^{\prime} = 0$.\,
To this end, let
$\varphi^{\prime}(\zeta) \;=\; \zeta \circ \varphi \;\in\; \image(\,\varphi^{\prime}\,) \;\subset\; \Hom_{R}(M,D)$.
Then, observe that
\begin{equation*}
(\,\psi^{\prime}\circ\varphi^{\prime}\,)(\zeta)
\;\; = \;\;
	\psi^{\prime}(\,\varphi^{\prime}(\zeta)\,)
\;\; = \;\;
	\psi^{\prime}(\,\zeta \circ \varphi\,)
\;\; = \;\;
	(\,\zeta \circ \varphi\,) \circ \psi
\;\; = \;\;
	\zeta \circ (\,\varphi \circ \psi\,),
\quad
	\textnormal{for each \,$\zeta \in \Hom_{R}(N,D)$}
\end{equation*}
Hence,
\begin{equation*}
\textnormal{exactness of \eqref{givenShortExactSequence} at $M$}
\; \Longleftrightarrow \;
	{\color{red}\image(\,\psi\,) \,=\, \ker(\,\varphi\,)}
\; \Longrightarrow \;
	\varphi \,\circ\, \psi \,= 0
\;\; {\color{red}\Longrightarrow} \;\;
	\psi^{\prime} \circ \varphi^{\prime} = 0
\; \Longleftrightarrow \;
	{\color{red}\image(\,\varphi^{\prime}\,) \subset \ker(\,\psi^{\prime}\,)}
\end{equation*}
as required.

\vskip 0.25cm
\noindent
Next, we show that
\,$\image(\,\varphi^{\prime}\,) \supset \ker(\,\psi^{\prime}\,)$.\,
To this end, let \,$\beta \in \ker(\,\psi^{\prime}\,) \subset \Hom_{R}(M,D)$,\,
i.e. \,$\psi^{\prime}(\beta) = \beta \circ \psi = 0$.\,
We need to show \,$\beta \in \image(\,\varphi^{\prime}\,)$,\,
i.e. there exists \,$\alpha \in \Hom_{R}(N,D)$\, such that
\,$\beta = \varphi^{\prime}(\alpha) = \alpha \circ \varphi \in \Hom_{R}(M,D)$.\,
So, we now construct/define such an \,$\alpha \in \Hom_{R}(N,D)$,\, as follows:
\begin{equation*}
\alpha(\,n\,) \; := \; \beta(\,m\,),
\quad
\textnormal{for any \,$m \in \varphi^{-1}(n)$,\; for each \,$n \in N$}
\end{equation*}
For \,$\alpha$\, to be set-theoretically well-defined, we must have:
\begin{itemize}
\item
	$\varphi^{-1}(n) \,\neq\, \varemptyset$,\, for each \,$n \in N$, and
\item
	$\beta(\,m_{1}\,) \,=\, \beta(\,m_{2}\,)$,\, for any \,$m_{1}, m_{2} \in \varphi^{-1}(n)$,\, for each \,$n \in N$.	
\end{itemize}
The first bullet is exactly the {\color{red}surjectivity of \,$\varphi$},\,
equivalently, the exactness of \eqref{givenShortExactSequence} at $N$.
For the second bullet, note that
\begin{eqnarray*}
m_{1}, m_{2} \in \varphi^{-1}(n)
& \Longleftrightarrow &
	\varphi(m_{1}) \,=\, n \,=\, \varphi(m_{2})
\;\; \Longleftrightarrow \;\;
	m_{1} - m_{2} \,\in\, {\color{red}\ker(\varphi) \,=\, \image(\psi)}
\\
& \overset{{\color{white}1}}{\Longrightarrow} &
	\beta(\,m_{1} - m_{2}\,) \,\in\, \beta\!\left(\,\image(\psi)\,\right) \,=\, \{\,0\,\},
	\;\;
	\textnormal{since \,$\psi^{\prime}(\beta) \,=\, \beta \circ \psi \,=\, 0$}
\\
& \overset{{\color{white}1}}{\Longrightarrow} &
	\beta(\,m_{1}\,) \,=\, \beta(\,m_{2}\,) \,\in\, D.
\end{eqnarray*}
This establishes that \,$\alpha : N \longrightarrow D$\, is indeed set-theoretically well-defined.
It is routine to show that \,$\alpha$\, is furthermore an $R$-module homomorphism.
By construction/definition of \,$\alpha$,\, we have
\begin{equation*}
\beta(\,m\,) \; = \; \alpha(\varphi(m)),
\quad
\textnormal{for each \,$m \in M$},
\end{equation*}
i.e. \,$\beta = \alpha \circ \varphi = \varphi^{\prime}(\alpha) \in \image(\varphi^{\prime})$.\,
This proves that
\,$\ker(\,\psi^{\prime}\,) \subset \image(\,\varphi^{\prime}\,)$,\,
and hence
\,$\image(\,\varphi^{\prime}\,) = \ker(\,\psi^{\prime}\,)$,\,
i.e. the exactness of \eqref{inducedExactSequence} at \,$\Hom_{R}(M,D)$.\,
\qed

          %%%%% ~~~~~~~~~~~~~~~~~~~~ %%%%%

\vskip 0.5cm
\begin{remark}
\mbox{}
\vskip 0.1cm
\noindent
Consider the following diagram:
\begin{center}
\begin{tikzcd}
0 \arrow[r]
& L \arrow[r, "\psi"] \arrow[d, "\xi{\color{white}.}" swap]
& M \arrow[r, "\varphi"] \arrow[dr] \arrow[dl, dashrightarrow, "\exists{\color{white}.}\eta{\color{white}.}?", red]
& N \arrow[d, "{\color{white}.}\zeta_{1}\textnormal{,\,}\zeta_{2}"] \arrow[r]
& 0
\\
& D
&
& D
\end{tikzcd}\end{center}
\begin{itemize}
\item
	The exactness of \eqref{inducedExactSequence} at \,$\Hom_{R}(N,D)$
	is equivalent to the injectivity of $\varphi^{\prime}$,
	which is a consequence of the surjectivity of $\varphi$.
\item
	The exactness of \eqref{inducedExactSequence} at \,$\Hom_{R}(M,D)$
	is equivalent to \,$\psi^{\prime}\circ\varphi^{\prime} = 0$,\,
	which follows from the exactness of \eqref{givenShortExactSequence} at \,$M$
	(which in turn is equivalent to \,$\varphi \circ \psi = 0$),
	since
	\begin{equation*}
	(\,\psi^{\prime}\circ\varphi^{\prime}\,)(\zeta)
	\;\; = \;\;
		\psi^{\prime}(\,\varphi^{\prime}(\zeta)\,)
	\;\; = \;\;
		\psi^{\prime}(\,\zeta \circ \varphi\,)
	\;\; = \;\;
		(\,\zeta \circ \varphi\,) \circ \psi
	\;\; = \;\;
		\zeta \circ (\,\varphi \circ \psi\,),
	\quad
		\textnormal{for each \,$\zeta \in \Hom_{R}(N,D)$}
	\end{equation*}
	as mentioned in the proof of Proposition \ref{homologicalAlgebraMotivation}.
\item
	The induced exact sequence \eqref{inducedExactSequence}
	in general cannot be extended on the right to a short exact sequence.
	Indeed, this extendibility is equivalent to the surjectivity of $\psi^{\prime}$,
	which in turn is equivalent to the following statement:
	\begin{center}
	\begin{minipage}{6.0in}
	\textnormal{For each $R$-module homomorphism
	\,$L\,\overset{\xi}{\longrightarrow}\,D$,\,
	there exists an $R$-module homomorphism
	\,$M\,\overset{\eta}{\longrightarrow}\,D$,\,
	such that
	\,$\xi \,=\, \psi^{\prime}(\eta) \,=\, \eta \circ \psi$
	}
	\end{minipage}
	\end{center}
	And, the above statement is clearly not true in general.
\end{itemize}
Homological algebra defines, and provides tools to compute, algebraic objects
that measure the degree to which
\eqref{inducedExactSequence}
{\color{red}fails to extend to a short exact sequence};
more precisely, the aforementioned algebraic objects are abelian groups
that extend \eqref{inducedExactSequence} into a long exact sequence:
\begin{center}
\begin{tikzcd}
0 \arrow[r]
& \Hom_{R}(N,D) \arrow[r, "\varphi^{\prime}"]
& \Hom_{R}(M,D) \arrow[r, "\psi^{\prime}"] \arrow[d, phantom, ""{coordinate, name=Z}]
& \Hom_{R}(L,D) 
\arrow[dll,
rounded corners,
to path={ -- ([xshift=2ex]\tikztostart.east)
|- (Z) [near end]\tikztonodes
-| ([xshift=-2ex]\tikztotarget.west)
-- (\tikztotarget)}]
\\
& {\color{white}.}\Ext^{1}_{R}(N,D){\color{white}.} \arrow[r]
& {\color{white}.}\Ext^{1}_{R}(M,D){\color{white}.} \arrow[r] \arrow[d, phantom, ""{coordinate, name=Z}]
& {\color{white}.}\Ext^{1}_{R}(L,D){\color{white}.} 
\arrow[dll,
rounded corners,
to path={ -- ([xshift=2ex]\tikztostart.east)
|- (Z) [near end]\tikztonodes
-| ([xshift=-2ex]\tikztotarget.west)
-- (\tikztotarget)}]
\\
& {\color{white}.}\Ext^{2}_{R}(N,D){\color{white}.} \arrow[r]
& {\color{white}.}\Ext^{2}_{R}(M,D){\color{white}.} \arrow[r] \arrow[d, phantom, ""{coordinate, name=Z}]
& {\color{white}.}\Ext^{2}_{R}(L,D){\color{white}.}
\arrow[dll,
rounded corners,
to path={ -- ([xshift=2ex]\tikztostart.east)
|- (Z) [near end]\tikztonodes
-| ([xshift=-2ex]\tikztotarget.west)
-- (\tikztotarget)}]
\\
& {\color{white}.}\Ext^{3}_{R}(N,D){\color{white}.} \arrow[r]
& {\color{white}.}\cdots\;\;\cdots\;\;\cdots{\color{white}.} \arrow[r, white]
& {\color{white}.\Ext^{3}_{R}(N,D).}
\end{tikzcd}\end{center}
Consequently, the non-vanishing of the group
\,$\Ext^{1}_{R}(N,D)$\,
can therefore be regarded as {\color{red}obstruction} to
\begin{itemize}
\item
	the extendibility of \eqref{inducedExactSequence} to a short exact sequence, or equivalently
\item
	the surjectivity \,$\psi^{\prime}$\, in \eqref{inducedExactSequence}, or equivalently
\item
	the lifting of an arbitrary \,$\gamma \in \Hom_{R}(L,D)$\, to an element in $\Hom_{R}(M,D)$ via \,$\psi$,\,
	i.e. the existence of $\beta \in \Hom_{R}(M,D)$ such that $\gamma = \beta \circ \psi = \psi^{\prime}(\beta)$.
\end{itemize}
Put yet in another way, an element \,$\gamma \in \Hom_{R}(L,D)$\, can be lifted via \,$\psi$
to an element in $\Hom_{R}(M,D)$ if and only if $\gamma$ maps to zero in
\,$\Ext^{1}_{R}(N,D)$.\,
\end{remark}

          %%%%% ~~~~~~~~~~~~~~~~~~~~ %%%%%
