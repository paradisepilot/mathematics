
          %%%%% ~~~~~~~~~~~~~~~~~~~~ %%%%%

\section{Hom and localization}
\setcounter{theorem}{0}
\setcounter{equation}{0}

%\cite{vanDerVaart1996}
%\cite{Kosorok2008}

%\renewcommand{\theenumi}{\alph{enumi}}
%\renewcommand{\labelenumi}{\textnormal{(\theenumi)}$\;\;$}
\renewcommand{\theenumi}{\roman{enumi}}
\renewcommand{\labelenumi}{\textnormal{(\theenumi)}$\;\;$}

          %%%%% ~~~~~~~~~~~~~~~~~~~~ %%%%%

\begin{proposition}[Proposition 2.10, \S2.2, p.69, \cite{eisenbud1995commutative}; Lemma 4.87, \S 4.7, p. 200, \cite{rotman2008introduction}]
\label{HomAndLocalization}
\mbox{}
\vskip 0.1cm
\noindent
Suppose that \,$R$\, is a commutative ring with multiplicative identity, \,$S$\, is an $R$-algebra, and \,$M$,\, $N$\, are \,$R$-modules.
Then, the following statements are true:
\begin{enumerate}
\item
	There is a unique $S$-module homomorphism
	\begin{equation*}
	\alpha : S \otimes_{R} \Hom_{R}(M,N) \longrightarrow \Hom_{S}\!\left(\,S\otimes_{R}M\,,\,S\otimes_{R}N\,\right)
	\end{equation*}
	that takes an element
	\,$1_{S}\,\otimes\,\varphi \in S \otimes_{R} \Hom_{R}(M, N)$\,
	to the $S$-module homomorphism
	\,$\id_{S} \,\otimes_{R}\, \varphi : S \otimes_{R} M \longrightarrow S \otimes_{R} N$\,
	in
	\,$\Hom_{S}(S \otimes_{R} M, S \otimes_{R} N)$.
%	\begin{equation*}
%	\left(\,\id_{S} \overset{{\color{white}.}}{\otimes_{R}} \varphi\,\right)\!\left(\,s \otimes_{R} m\,\right)
%	\;\; = \;\;
%		s \otimes_{R} \varphi(m)
%	\end{equation*}
%	\begin{center}
%	\begin{tikzcd}
%	A
%		\arrow[rr, "f"]
%		\arrow[dr, swap, "\nu"]
%		\arrow[ddr, bend right, swap, "\nu^{\prime}"]
%	&&
%	B
%	\\
%	& {\color{red}I}
%		\arrow[ur, hook, swap, "\iota", red]
%		\arrow[d, dashed, "\;\exists ! \, \theta"]
%	\\
%	& I^{\prime}
%		\arrow[uur, bend right, hook, swap, "\iota^{\prime}"]
%	&
%	\end{tikzcd}
%	\end{center}
\item
	If $S$ is flat over $R$ and $M$ is finitely presented, then $\alpha$ is an isomorphism.
\item
	In particular, if $M$ is finitely presented, then $\Hom_{R}(M, N)$ localizes in the
	sense that the map $\alpha$ provides a natural isomorphism
	\begin{equation*}
	D^{-1}\,\Hom_{R}\!\left(\,M\,,\,N\,\right)
	\;\;\cong\;\;
		\Hom_{D^{-1}R}\!\left(\,D^{-1}M\,,\,D^{-1}N\,\right)
	\end{equation*}
	for any subset $D \subset R$.
\end{enumerate}
\end{proposition}
\proof

\begin{enumerate}
\item
	It is clear that the set-theoretic map
	\begin{equation*}
	\alpha^{\prime} \;:\; \Hom_{R}(M,N)
		\;\; \longrightarrow \;\;
		\Hom_{S}\!\left(\, S \otimes_{R} M \,,\, S \otimes_{R} N \,\right)
	\;\;:\;\;
		\varphi \;\; \longmapsto \;\; \id_{S} \,\otimes_{R}\, \varphi
	\end{equation*}
	is an $R$-module homomorphism.
	The map
	\,$\alpha : S \otimes_{R} \Hom_{R}(M,N) \longrightarrow \Hom_{S}\!\left(\,S\otimes_{R}M\,,\,S\otimes_{R}N\,\right)$\,
	is obtained from \,$\alpha^{\prime}$\, by extending its domain from
	\,$\Hom_{R}(M,N)$\, to \,$S \otimes_{R} \Hom_{R}(M,N)$.\,
	The following commutative diagram describes the relation between \,$\alpha$\, and \,$\alpha^{\prime}$:\,
	\begin{center}
	\begin{tikzcd}
	\varphi
		\arrow[rr, maps to]
	&
	&
	\id_{S} \otimes \varphi
	\\
	\Hom_{R}(M,N)
		\arrow[rr, "\alpha^{\prime}"]
		\arrow[dd]
	&
	&
		\Hom_{S}\!\left(\,S\otimes_{R}M\,,\,S\otimes_{R}N\,\right)
		\arrow[dd, equal]
	\\ \\
	S\,\otimes_{R}\Hom_{R}(M,N)
		\arrow[rr, swap, "\alpha"]
	&
	&
		\Hom_{S}\!\left(\,S\otimes_{R}M\,,\,S\otimes_{R}N\,\right)
	\\
	1_{S}\,\otimes\,\varphi
		\arrow[rr, maps to]
	&
	&
	\id_{S} \otimes \varphi
	\end{tikzcd}
	\end{center}
	Since \,$\alpha$\, is an $S$-module homomorphism, its value on a general element
	\,$\xi \,\otimes\, \varphi \,\in\, S \,\otimes_{R}\, \Hom_{R}(M,N)$\,
	is determined by \,$\alpha^{\prime}$\, via:
	\begin{eqnarray*}
	\alpha\!\left(\, \xi \,\overset{{\color{white}.}}{\otimes}\, \varphi \,\right)
	& = &
		\alpha\!\left(\, (\,\xi \cdot 1_{S}) \,\overset{{\color{white}.}}{\otimes}\, \varphi \,\right)
	\;\; = \;\;
		\alpha\!\left(\, \xi \cdot (\,1_{S} \,\overset{{\color{white}.}}{\otimes}\, \varphi\,) \,\right)
	\;\; = \;\;
		\xi \,\cdot\, \alpha\!\left(\, 1_{S} \,\overset{{\color{white}.}}{\otimes}\, \varphi \,\right)
	\;\; = \;\;
		\xi \cdot \left(\, \id_{S} \,\overset{{\color{white}.}}{\otimes}\, \varphi \,\right)
	\\
	& = &
		\xi(\,\cdot\,) \,\overset{{\color{white}.}}{\otimes}\, \varphi\,,
	\end{eqnarray*}
	where \,$\xi(\,\cdot\,) : S \longrightarrow S$\, denotes multiplication by \,$\xi \in S$\, on \,$S$,\,
	and the last equality follows from the following observation:
	\begin{equation*}
	\left[\;\xi \cdot \left(\, \id_{S} \,\overset{{\color{white}.}}{\otimes}\, \varphi \,\right)\,\right](\,s \,\otimes\, m\,)
	\;\; = \;\;
		\xi \cdot \left(\, \id_{S}(s) \,\overset{{\color{white}.}}{\otimes}\, \varphi(m) \,\right)
	\;\; = \;\;
		\xi \cdot \left(\, s \,\overset{{\color{white}.}}{\otimes}\, \varphi(m) \,\right)
	\;\; = \;\;
		\xi(s) \,\overset{{\color{white}.}}{\otimes}\, \varphi(m)
	\end{equation*}
	\vskip 0.3cm
%	$\alpha^{\prime}(\varphi)(s \otimes m)$
%	\,$=$\, $\left(\,\id_{S}\overset{{\color{white}.}}{\otimes}\varphi\,\right)(s \otimes m)$
%	\,$=$\, $\left(\,\id_{S}(\overset{{\color{white}1}}{s})\,\right) \otimes \left(\, \varphi(\overset{{\color{white}1}}{m}) \,\right)$
%	\,$=$\, $s \otimes \varphi(m)$	
%	$\left(\,s \overset{{\color{white}.}}{\otimes_{R}} \varphi\,\right)\!(\,m\,)$
%	\,$=$\, $s \otimes_{R} \varphi(m)$.
\item
	Suppose \,$S$\, is flat over \,$R$\, and \,$M$\, is finitely presented \,$R$-module.\,
	We need to show that, in this case, the map \,$\alpha$\, is an isomorphism.
	
	\vskip 0.3cm
	\textbf{Claim 1}:\;\; If \,$M = R$,\, then \,$\alpha$\, is an isomorphism.
	\vskip 0.01cm
	Proof of Claim 1:\;\;
	In this case, we may identify \,$\Hom_{R}(M,N) \,=\, \Hom_{R}(R,N)$\,
	with \,$N$\, by mapping each \,$\varphi \in \Hom_{R}(R,N)$\, to \,$\varphi(1_{R})$.\,
	On the other hand, we have \,$S\,\otimes_{R}M \,=\, S\,\otimes_{R}R  \,\cong\, S$.\,
	Hence,
	\,$\Hom_{S}(\,S\otimes_{R}M\,,\,S\otimes_{R}N\,)$
	\,$\cong$\, $\Hom_{S}(\,S\otimes_{R}R\,,\,S\otimes_{R}N\,)$
	\,$\cong$\, $\Hom_{S}(\,S\,,\,S\otimes_{R}N\,)$
	\,$\cong$\, $S\otimes_{R}N$.\,
	These identifications fit into the following commutative diagram:
	\begin{center}
	\begin{tikzcd}
	S\,\otimes_{R}\Hom_{R}(M,N)
		\arrow[rr, "\alpha"]
		\arrow[dd, equal]
	&
	&
		\Hom_{S}\!\left(\,S\otimes_{R}M\,,\,S\otimes_{R}N\,\right)
		\arrow[d, equal]
	\\
	&
	&
		\Hom_{S}\!\left(\,S\otimes_{R}R\,,\,S\otimes_{R}N\,\right)
		\arrow[d, "\;\cong"]
	\\
	S\,\otimes_{R}\Hom_{R}(R,N)
		\arrow[d, swap, "\cong\;"]
	&
	&
		\Hom_{S}\!\left(\,S\,,\,S\otimes_{R}N\,\right)
		\arrow[d, "\;\cong"]
%	\\
%	&
%	&
%		\Hom_{S}\!\left(\,S\,,\,S\otimes_{R}N\,\right)
	\\
	S\,\otimes_{R}N
		\arrow[rr, swap, "\id"]
	&
	&
		S\,\otimes_{R}N
	\end{tikzcd}
	\end{center}
	The above diagram shows that \,$\alpha$\, corresponds, in this case, to the identify map
	on \,$S \,\otimes_{R}\, N$;\, in particular, we see that \,$\alpha$\, is an isomorphism.
	This completes the proof of Claim 1.

	\vskip 0.3cm
	\textbf{Claim 2}:\;\; If \,$M = \overset{m}{\underset{i =1}{\bigoplus}}\, R$\, is a free \,$R$-module of finite rank \,$m \geq 1$,\,
	(i.e., the direct sum of \,$m$\, copies of \,$R$),
	then \,$\alpha$\, is an isomorphism.
	\vskip 0.01cm
	Proof of Claim 2:\;\;
	Recall that the functions \,$\Hom$\, and \,$\otimes$\, both commute with finite direct sums.
	Consequently, \,$\alpha_{M}$\, decomposes by summands, i.e.,
	\,$\alpha_{M}$ \,$=$\, $\overset{m}{\underset{i=1}{\bigoplus}}\;\alpha_{R}$.\,
	Since each copy of \,$\alpha_{R}$\, is an isomorphism onto its own image (by Claim 1),
	\,$\alpha_{M}$\, itself is an isomorphism.
	This completes the proof of Claim 2.

	\vskip 0.3cm
	\textbf{Claim 3}:\;\; If \,$M$\, is a finitely presented \,$R$-module,
	then \,$\alpha$\, is an isomorphism.
	\vskip 0.01cm
	Proof of Claim 3:\;\;

	\begin{equation}\label{finitePresentationOfM}
	\begin{tikzcd}
	\overset{b}{\underset{i=1}{\oplus}}R
		\arrow[r]
	&
	\overset{a}{\underset{i=1}{\oplus}}R
		\arrow[r]
	&
	M
		\arrow[r]
	&
	0
	\end{tikzcd}
	\end{equation}
	Since tensoring is right-exact, we get the exact sequence:
	\begin{equation}\label{STensorFinitePresentationOfM}
	\begin{tikzcd}
	S\otimes_{R}\left(\,\overset{b}{\underset{i=1}{\oplus}}R\,\right)
		\arrow[r]
	&
	S\otimes_{R}\left(\,\overset{a}{\underset{i=1}{\oplus}}R\,\right)
		\arrow[r]
	&
	S\otimes_{R}M
		\arrow[r]
	&
	0
	\end{tikzcd}
	\end{equation}
	Now, consider the following commutative diagram:
	\begin{equation}\label{stackedExactSequence}
	\begin{tikzcd}
	0
		\arrow[r]
	&
	S \,\otimes_{R}\, \Hom_{R}\!\left(\,M\,,\,N\,\right)
		\arrow[dd, swap, "\alpha\;"]
		\arrow[r]
	&
	S \,\otimes_{R} \Hom_{S}\!\left(\,\overset{a}{\underset{i=1}{\oplus}}R\,,\,N\,\right)
		\arrow[dd, swap, "\cong\;"]
		\arrow[r]
	&
	S \,\otimes_{R} \Hom_{S}\!\left(\,\overset{b}{\underset{i=1}{\oplus}}R\,,\,N\,\right)
		\arrow[dd, "\;\cong"]
	\\
	\\
	0
		\arrow[r]
	&
	\Hom_{S}\!\left(\overset{{\color{white}.}}{S} \otimes_{R}M , S \otimes_{R}M\right)
		\arrow[r]
	&
	\Hom_{S}\!\left(\overset{{\color{white}.}}{S}\otimes_{R}\!\left(\overset{a}{\underset{i=1}{\oplus}}R\right),S\otimes_{R}N\right)
		\arrow[r]
	&
	\Hom_{S}\!\left(\overset{{\color{white}.}}{S}\otimes_{R}\!\left(\overset{b}{\underset{i=1}{\oplus}}R\right),S\otimes_{R}N\right)
	\end{tikzcd}
	\end{equation}
	The top row of  \eqref{stackedExactSequence} is obtained from \eqref{finitePresentationOfM}
	by applying \,$\Hom(\;\cdot\,,N\,)$\, followed by tensoring \,$S\otimes_{R}(\,\cdot\,)$.
	Left-exactness (and contravariance) of \,$\Hom(\;\cdot\,,N\,)$\,
	and flatness of \,$S$\ imply that the top row of \eqref{stackedExactSequence} is exact.

	\vskip 0.2cm
	The bottom row of  \eqref{stackedExactSequence} is obtained from \eqref{STensorFinitePresentationOfM}
	by applying \,$\Hom(\;\cdot\,,S\otimes_{R}N\,)$.\, 
	Left-exactness (and contravariance) of \,$\Hom(\;\cdot\,,S\otimes_{R}N\,)$\,
	implies that the bottom row of \eqref{stackedExactSequence} is exact.

	\vskip 0.2cm
	By Claim 2, the middle and rightmost vertical arrows in \eqref{stackedExactSequence}.

	\vskip 0.2cm
	That \,$\alpha$\, is an isomorphism can now be established via a diagram chase.
	\vskip 0.3cm
	
\item

	\qed
\end{enumerate}

          %%%%% ~~~~~~~~~~~~~~~~~~~~ %%%%%
