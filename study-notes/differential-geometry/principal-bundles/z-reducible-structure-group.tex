
          %%%%% ~~~~~~~~~~~~~~~~~~~~ %%%%%

\section{Reducibility of the structure group of a principal fibre bundle}
\setcounter{theorem}{0}
\setcounter{equation}{0}

%\cite{vanDerVaart1996}
%\cite{Kosorok2008}

%\renewcommand{\theenumi}{\alph{enumi}}
%\renewcommand{\labelenumi}{\textnormal{(\theenumi)}$\;\;$}
\renewcommand{\theenumi}{\roman{enumi}}
\renewcommand{\labelenumi}{\textnormal{(\theenumi)}$\;\;$}

          %%%%% ~~~~~~~~~~~~~~~~~~~~ %%%%%

\begin{definition}[Morphism of principal fibre bundles]
\mbox{}
\vskip 0.2cm
\noindent
Suppose
\,$G_{1} \longhookrightarrow P_{1} \overset{\pi_{1}}{\longrightarrow} M_{1}$\, and
\,$G_{2} \longhookrightarrow P_{2} \overset{\pi_{2}}{\longrightarrow} M_{2}$\,
are two principal fibre bundles.
%\begin{itemize}
%\item
\vskip 0.1cm
\noindent
	A \,\textbf{morphism of principal fibre bundles}\, from
	\,$G_{1} \longhookrightarrow P_{1} \overset{\pi_{1}}{\longrightarrow} M_{1}$\,
	into
	\,$G_{2} \longhookrightarrow P_{2} \overset{\pi_{2}}{\longrightarrow} M_{2}$\,
	is an ordered pair
	\,$\left(\,\widetilde{f}\,,\,\rho\,\right)$,\,
	where
	\begin{equation*}
	\rho : G_{1} \longrightarrow G_{2}
	\end{equation*}
	is a Lie group homomorphism, and
	\begin{equation*}
	\widetilde{f} : P_{1} \longrightarrow P_{2}
	\end{equation*}
	is a smooth $\rho$-equivariant map, i.e. 
	\begin{equation*}
	\widetilde{f}(p_{1} \cdot g_{1})
	\;\; = \;\;
		\widetilde{f}(p_{1}) \cdot \rho(g_{1})\,,
	\quad
	\textnormal{for each \,$p_{1} \in P_{1}$\, and \,$g_{1} \in G_{1}$}\,;
	\end{equation*}
	in other words, the following diagram commutes for each \,$g_{1} \in G_{1}$:
	\begin{center}
	\begin{tikzcd}
	P_{1} \arrow[r, "\widetilde{f}"] \arrow[d, swap, "(\,\cdot\,)\,\cdot g_{1}\;\;"] & P_{2} \arrow[d, "\;\;(\,\cdot\,)\,\cdot \rho(g_{1})"] \\
	P_{1} \arrow[r, "\widetilde{f}"] & P_{2}
	\end{tikzcd}
	\end{center}
%\end{itemize}
\end{definition}

          %%%%% ~~~~~~~~~~~~~~~~~~~~ %%%%%

\begin{remark}
\mbox{}
\vskip 0.2cm
\noindent
%\begin{itemize}
%\item
	For a morphism \,$(\widetilde{f},\rho)$\, of principal fibre bundles from
	\,$G_{1} \longhookrightarrow P_{1} \overset{\pi_{1}}{\longrightarrow} M_{1}$\,
	into
	\,$G_{2} \longhookrightarrow P_{2} \overset{\pi_{2}}{\longrightarrow} M_{2}$,\,
	the $\rho$-equivariance hypothesis on $\widetilde{f}$ implies that
	\,$\widetilde{f}$\, induces a smooth map \,$f : M_{1} \longrightarrow M_{2}$.
	\vskip 0.2cm
	\proof
	Indeed, let $x_{1} \in M_{1}$. Then, define \,$f : M_{1} \longrightarrow M_{2}$\, as follows:
	\begin{equation*}
	f(x_{1}) \;\; := \;\; \pi_{2}\!\left(\,\widetilde{f}(p_{1})\,\right),
	\quad
	\textnormal{for any \,$p_{1} \in \pi_{1}^{-1}(x_{1}) \subset P_{1}$}.
	\end{equation*}
	We need to establish that the above definition of \,$f(x_{1})$\, does not depend on the
	particular choice of \,$p_{1} \in \pi_{1}^{-1}(x_{1})$.
	To this end, let $p_{1}^{\prime} \in \pi_{1}^{-1}(x_{1})$.
	Since $G_{1}$ acts transitively on the right on each fibre of $P_{1}$, we have:
	$p_{1}^{\prime} = p_{1} \cdot g_{1}$, for some $g_{1} \in G_{1}$.
	Hence, by $\rho$-equivariance of $\widetilde{f}$, we have:
	$\widetilde{f}(p_{1}^{\prime}) = \widetilde{f}(p_{1}\cdot g_{1}) = \widetilde{f}(p_{1})\cdot\rho(g_{1})$,
	which in turn implies
	$\pi_{2}\!\left(\,\widetilde{f}(p_{1}^{\prime})\,\right) = \pi_{2}\!\left(\,\widetilde{f}(p_{1})\,\right)$,
	since the right action of $G_{2}$ on $P_{2}$ preserves fibres of $P_{2}$.	
	\qed
%\end{itemize}
\end{remark}

          %%%%% ~~~~~~~~~~~~~~~~~~~~ %%%%%

\vskip 0.5cm
\begin{definition}[Embeddings and subbundles]
\mbox{}
\vskip 0.2cm
\noindent
Suppose
\,$\left(\,\widetilde{f}:P_{1}\longrightarrow P_{2}\,,\,\rho:G_{1}\longrightarrow G_{2}\,\right)$\,
is a morphism of principal fibre bundles from
\,$G_{1} \longhookrightarrow P_{1} \overset{\pi_{1}}{\longrightarrow} M_{1}$\,
into
\,$G_{2} \longhookrightarrow P_{2} \overset{\pi_{2}}{\longrightarrow} M_{2}$.\,
%\begin{itemize}
%\item
	The morphism
	\,$\left(\,\widetilde{f}\,,\,\rho\,\right)$\,
	is called an \,\textbf{embedding}\,
	if \,$\widetilde{f}$\, is a embedding of smooth manifolds and
	\,$\rho$\, is a monomorphism of Lie groups, in which case we say that
	\,$G_{1} \longhookrightarrow P_{1} \overset{\pi_{1}}{\longrightarrow} M_{1}$\,
	is a \,\textbf{subbundle}\, of
	\,$G_{2} \longhookrightarrow P_{2} \overset{\pi_{2}}{\longrightarrow} M_{2}$.\,
%\item
%	We say that the structure group \,$G_{2}$\, is reducible to $G_{1}$
%\end{itemize}
\end{definition}

          %%%%% ~~~~~~~~~~~~~~~~~~~~ %%%%%

\vskip 0.5cm
\begin{definition}[Reducibility of structure group]
\mbox{}
\vskip 0.2cm
\noindent
Suppose
\,$G \longhookrightarrow P \overset{\pi}{\longrightarrow} M$\,
is a principal fibre bundle,
\,$G_{1} \subset G$\, is a Lie subgroup of \,$G$,\, and
\,$\iota:G_{1}\longhookrightarrow G$\,
is the inclusion Lie group homomorphism.
\begin{itemize}
\item
	\vskip -0.2cm
	The structure group \,$G$\, is said to be \textbf{reducible} to $G_{1}$
	if there exists an embedding
	\,$\left(\,\widetilde{f}:P_{1}\longrightarrow P\,,\,\iota:G_{1}\longhookrightarrow G\,\right)$\,
	of principal fibre bundles mapping
	\,$G_{1} \longhookrightarrow P_{1} \overset{\pi}{\longrightarrow} M_{1}$\,
	into
	\,$G \longhookrightarrow P \overset{\pi}{\longrightarrow} M$\,
	such that
	\,$M_{1} = M$\,
	and the induced map
	\,$f : M_{1} = M \longrightarrow M$\,
	is the identity map on $M$.
\item
	In this case,
	\,$\widetilde{f}$\, is called a \textbf{reduction} of the structure group $G$ of
	\,$G \longhookrightarrow P \overset{\pi}{\longrightarrow} M$\,
	to $G_{1} \subset G$.
	The principal fibre bundle
	\,$G_{1} \longhookrightarrow P_{1} \overset{\pi}{\longrightarrow} M$\,
	is then called a \textbf{reduced bundle} of
	\,$G \longhookrightarrow P \overset{\pi}{\longrightarrow} M$.\,
	
\end{itemize}
\end{definition}

          %%%%% ~~~~~~~~~~~~~~~~~~~~ %%%%%

\vskip 0.5cm
\begin{proposition}[Characterization of structure group reducibility in terms of transition functions]
\mbox{}
\vskip 0.2cm
\noindent
Suppose
\,$G \longhookrightarrow P \overset{\pi}{\longrightarrow} M$\,
is a principal fibre bundle.
The structure group \,$G$\, is reducible to a Lie subgroup 
\,$G_{1} \subset G$\, if and only if
there exists an open covering
\,$\left\{\,U_{\alpha}\,\right\}_{\alpha\in\mathcal{A}}$\,
of \,$M$\, with a set of transition functions
\,$\tau_{\beta\alpha}$\,
which take values in \,$G_{1}$.
\end{proposition}
\proof

\qed

          %%%%% ~~~~~~~~~~~~~~~~~~~~ %%%%%

          %%%%% ~~~~~~~~~~~~~~~~~~~~ %%%%%
