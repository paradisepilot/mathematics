
          %%%%% ~~~~~~~~~~~~~~~~~~~~ %%%%%

\section{Reducibility of the structure group of a principal fibre bundle}
\setcounter{theorem}{0}
\setcounter{equation}{0}

%\cite{vanDerVaart1996}
%\cite{Kosorok2008}

%\renewcommand{\theenumi}{\alph{enumi}}
%\renewcommand{\labelenumi}{\textnormal{(\theenumi)}$\;\;$}
\renewcommand{\theenumi}{\roman{enumi}}
\renewcommand{\labelenumi}{\textnormal{(\theenumi)}$\;\;$}

          %%%%% ~~~~~~~~~~~~~~~~~~~~ %%%%%

\begin{definition}[Morphism of principal fibre bundles]
\mbox{}
\vskip 0.2cm
\noindent
Suppose
\,$G_{1} \longhookrightarrow P_{1} \overset{\pi_{1}}{\longrightarrow} M_{1}$\, and
\,$G_{2} \longhookrightarrow P_{2} \overset{\pi_{2}}{\longrightarrow} M_{2}$\,
are two principal fibre bundles.
%\begin{itemize}
%\item
\vskip 0.1cm
\noindent
	A \,\textbf{morphism of principal fibre bundles}\, from
	\,$G_{1} \longhookrightarrow P_{1} \overset{\pi_{1}}{\longrightarrow} M_{1}$\,
	into
	\,$G_{2} \longhookrightarrow P_{2} \overset{\pi_{2}}{\longrightarrow} M_{2}$\,
	is an ordered pair
	\,$\left(\,\widetilde{f}\,,\,\rho\,\right)$,\,
	where
	\begin{equation*}
	\rho : G_{1} \longrightarrow G_{2}
	\end{equation*}
	is a Lie group homomorphism, and
	\begin{equation*}
	\widetilde{f} : P_{1} \longrightarrow P_{2}
	\end{equation*}
	is a smooth $\rho$-equivariant map, i.e. 
	\begin{equation*}
	\widetilde{f}(p_{1} \cdot g_{1})
	\;\; = \;\;
		\widetilde{f}(p_{1}) \cdot \rho(g_{1})\,,
	\quad
	\textnormal{for each \,$p_{1} \in P_{1}$\, and \,$g_{1} \in G_{1}$}\,;
	\end{equation*}
	in other words, the following diagram commutes for each \,$g_{1} \in G_{1}$:
	\begin{center}
	\begin{tikzcd}
	P_{1} \arrow[r, "\widetilde{f}"] \arrow[d, swap, "(\,\cdot\,)\,\cdot g_{1}\;\;"] & P_{2} \arrow[d, "\;\;(\,\cdot\,)\,\cdot \rho(g_{1})"] \\
	P_{1} \arrow[r, "\widetilde{f}"] & P_{2}
	\end{tikzcd}
	\end{center}
%\end{itemize}
\end{definition}

          %%%%% ~~~~~~~~~~~~~~~~~~~~ %%%%%

\begin{remark}
\mbox{}
\vskip 0.2cm
\noindent
%\begin{itemize}
%\item
	For a morphism \,$(\widetilde{f},\rho)$\, of principal fibre bundles from
	\,$G_{1} \longhookrightarrow P_{1} \overset{\pi_{1}}{\longrightarrow} M_{1}$\,
	into
	\,$G_{2} \longhookrightarrow P_{2} \overset{\pi_{2}}{\longrightarrow} M_{2}$,\,
	the $\rho$-equivariance hypothesis on $\widetilde{f}$ implies that
	\,$\widetilde{f}$\, induces a smooth map \,$f : M_{1} \longrightarrow M_{2}$.
	\vskip 0.2cm
	\proof
	Indeed, let $x_{1} \in M_{1}$. Then, define \,$f : M_{1} \longrightarrow M_{2}$\, as follows:
	\begin{equation*}
	f(x_{1}) \;\; := \;\; \pi_{2}\!\left(\,\widetilde{f}(p_{1})\,\right),
	\quad
	\textnormal{for any \,$p_{1} \in \pi_{1}^{-1}(x_{1}) \subset P_{1}$}.
	\end{equation*}
	We need to establish that the above definition of \,$f(x_{1})$\, does not depend on the
	particular choice of \,$p_{1} \in \pi_{1}^{-1}(x_{1})$.
	To this end, let $p_{1}^{\prime} \in \pi_{1}^{-1}(x_{1})$.
	Since $G_{1}$ acts transitively on the right on each fibre of $P_{1}$, we have:
	$p_{1}^{\prime} = p_{1} \cdot g_{1}$, for some $g_{1} \in G_{1}$.
	Hence, by $\rho$-equivariance of $\widetilde{f}$, we have:
	$\widetilde{f}(p_{1}^{\prime}) = \widetilde{f}(p_{1}\cdot g_{1}) = \widetilde{f}(p_{1})\cdot\rho(g_{1})$,
	which in turn implies
	$\pi_{2}\!\left(\,\widetilde{f}(p_{1}^{\prime})\,\right) = \pi_{2}\!\left(\,\widetilde{f}(p_{1})\,\right)$,
	since the right action of $G_{2}$ on $P_{2}$ preserves fibres of $P_{2}$.	
	\qed
%\end{itemize}
\end{remark}

          %%%%% ~~~~~~~~~~~~~~~~~~~~ %%%%%

\vskip 0.5cm
\begin{definition}[Embeddings and subbundles]
\mbox{}
\vskip 0.2cm
\noindent
Suppose
\,$\left(\,\widetilde{f}:P_{1}\longrightarrow P_{2}\,,\,\rho:G_{1}\longrightarrow G_{2}\,\right)$\,
is a morphism of principal fibre bundles from
\,$G_{1} \longhookrightarrow P_{1} \overset{\pi_{1}}{\longrightarrow} M_{1}$\,
into
\,$G_{2} \longhookrightarrow P_{2} \overset{\pi_{2}}{\longrightarrow} M_{2}$.\,
%\begin{itemize}
%\item
	The morphism
	\,$\left(\,\widetilde{f}\,,\,\rho\,\right)$\,
	is called an \,\textbf{embedding}\,
	if \,$\widetilde{f}$\, is a embedding of smooth manifolds and
	\,$\rho$\, is a monomorphism of Lie groups, in which case we say that
	\,$G_{1} \longhookrightarrow P_{1} \overset{\pi_{1}}{\longrightarrow} M_{1}$\,
	is a \,\textbf{subbundle}\, of
	\,$G_{2} \longhookrightarrow P_{2} \overset{\pi_{2}}{\longrightarrow} M_{2}$.\,
%\item
%	We say that the structure group \,$G_{2}$\, is reducible to $G_{1}$
%\end{itemize}
\end{definition}

          %%%%% ~~~~~~~~~~~~~~~~~~~~ %%%%%

\vskip 0.5cm
\begin{definition}[Reducibility of structure group]
\mbox{}
\vskip 0.2cm
\noindent
Suppose
\,$G \longhookrightarrow P \overset{\pi}{\longrightarrow} M$\,
is a principal fibre bundle,
\,$G_{1} \subset G$\, is a Lie subgroup of \,$G$,\, and
\,$\iota:G_{1}\longhookrightarrow G$\,
is the inclusion Lie group homomorphism.
\begin{itemize}
\item
	\vskip -0.2cm
	The structure group \,$G$\, is said to be \textbf{reducible} to $G_{1}$
	if there exists an embedding
	\,$\left(\,\widetilde{f}:P_{1}\longrightarrow P\,,\,\iota:G_{1}\longhookrightarrow G\,\right)$\,
	of principal fibre bundles mapping
	\,$G_{1} \longhookrightarrow P_{1} \overset{\pi}{\longrightarrow} M_{1}$\,
	into
	\,$G \longhookrightarrow P \overset{\pi}{\longrightarrow} M$\,
	such that
	\,$M_{1} = M$\,
	and the induced map
	\,$f : M_{1} = M \longrightarrow M$\,
	is the identity map on $M$.
\item
	In this case,
	\,$\widetilde{f}$\, is called a \textbf{reduction} of the structure group $G$ of
	\,$G \longhookrightarrow P \overset{\pi}{\longrightarrow} M$\,
	to $G_{1} \subset G$.
	The principal fibre bundle
	\,$G_{1} \longhookrightarrow P_{1} \overset{\pi}{\longrightarrow} M$\,
	is then called a \textbf{reduced bundle} of
	\,$G \longhookrightarrow P \overset{\pi}{\longrightarrow} M$.\,
	
\end{itemize}
\end{definition}

          %%%%% ~~~~~~~~~~~~~~~~~~~~ %%%%%

\vskip 0.5cm
\begin{proposition}[Characterization of structure group reducibility in terms of transition functions]
\mbox{}
\vskip 0.2cm
\noindent
The structure group \,$G$\, of a principal fibre bundle
\,$G \longhookrightarrow P \overset{\pi}{\longrightarrow} M$\,
is reducible to a Lie subgroup 
\,$G_{1} \subset G$\, if and only if
there exists an open covering
\,$\left\{\,U_{\alpha}\,\right\}_{\alpha\in\mathcal{A}}$\,
of \,$M$\, with a set of transition functions
\,$\tau_{\beta\alpha}$\,
which take values in \,$G_{1}$.
\end{proposition}
\proof

\vskip 0.3cm
\noindent
\underline{$(\,\Longrightarrow\,)$}\;\;
Suppose $G$ is reducible to $G_{1}$, i.e. there exists an embedding
\,$\left(\,\widetilde{f} : P_{1} \longrightarrow P \,,\, \iota : G_{1} \longhookrightarrow G\,\right)$\,
of principal fibre bundles
whose induced map \,$f$\, on the base manifold \,$M$\, is the identity map on \,$M$.\,
We consider \,$P_{1}$\, as a submanifold of \,$P$.\,
Let
\,$\left\{\,\Psi^{(1)}_{\alpha} = (\,\pi_{1}\,,\psi^{(1)}_{\alpha}\,) : \pi_{1}^{-1}(U_{\alpha}) \longrightarrow U_{\alpha} \times G_{1}\,\right\}_{\alpha\in\mathcal{A}}$\,
be a bundle atlas of
\,$G_{1} \longhookrightarrow P_{1} \overset{\pi}{\longrightarrow} M$.\,
We define an extension
\,$\Psi_{\alpha} = (\,\pi\,,\psi_{\alpha}\,) : \pi^{-1}(U_{\alpha}) \longrightarrow U_{\alpha} \times G$\,
of each
\,$\Psi^{(1)}_{\alpha} = (\,\pi_{1}\,,\psi^{(1)}_{\alpha}\,) : \pi_{1}^{-1}(U_{\alpha}) \longrightarrow U_{\alpha} \times G_{1}$\,
as follows:
\begin{equation*}
\Psi_{\alpha}(p)
\;\; := \;\;
	\left(\,\pi(p)\,,\,\psi^{(1)}_{\alpha}(p_{1}) \cdot g\,\right)\,,
\quad
\textnormal{for each \,$p \in \pi^{-1}(U_{\alpha})$}\,,
\end{equation*}
where \,$(p_{1},g)$\, is any element of \,$\pi_{1}^{-1}(U_{\alpha}) \times G$\, such that \,$p = p_{1} \cdot g$.\,

\vskip 0.5cm
\noindent
\textbf{Claim 1}:\quad
$\Psi_{\alpha}$\, is well-defined, i.e. \,$\psi^{(1)}_{\alpha}(p_{1}) \cdot g \in G$\, is independent of the particular choice of \,$(p_{1},g)$.\,
\vskip 0.2cm
\proofof Claim 1:\quad
Suppose
\,$(p_{1}^{\prime},g^{\prime}) \in \pi_{1}^{-1}(U_{\alpha}) \times G$\,
also satisfies
\,$p = p_{1}^{\prime} \cdot g^{\prime}$.\,
We need to establish that
\begin{equation*}
\psi^{(1)}_{\alpha}(p_{1}^{\prime}) \cdot g^{\prime}
\;\; = \;\;
	\psi^{(1)}_{\alpha}(p_{1}) \cdot g
\end{equation*}
First, note that there exists \,$g_{1} \in G_{1} \subset G$\, such that \,$p_{1}^{\prime} = p_{1} \cdot g_{1}$.\,
Then,
\begin{equation*}
p_{1}^{\prime} \cdot g^{\prime} \,=\, p \,=\, p_{1} \cdot g
\quad\Longrightarrow\quad
p_{1} \cdot g_{1} \,=\, p_{1}^{\prime} \,=\, p_{1} \cdot \left(g \cdot (g^{\prime})^{-1}\right)
\quad\Longrightarrow\quad
\iota_{p_{1}}(\,g_{1}\,) \,=\, \iota_{p_{1}}\!\!\left(\,\overset{{\color{white}1}}{g} \cdot (g^{\prime}\,)^{-1}\,\right)
\end{equation*}
Hence, the injectivity of the orbit map
\,$\iota_{p_{1}} : G \longrightarrow \pi^{-1}\!\left(\overset{{\color{white}-}}{\pi}(p_{1})\right)$\,
implies that
\,$g \cdot (g^{\prime}\,)^{-1} = g_{1} \in G_{1} \subset G$.\,
We therefore see that
\begin{equation*}
\psi^{(1)}_{\alpha}(p_{1}^{\prime}) \cdot g^{\prime}
\;\; = \;\;
	\psi^{(1)}_{\alpha}(p_{1}^{\prime}) \cdot g^{\prime} \cdot g^{-1} \cdot g
\;\; = \;\;
	\psi^{(1)}_{\alpha}(p_{1}^{\prime}) \cdot g_{1}^{-1} \cdot g
\;\; = \;\;
	\psi^{(1)}_{\alpha}(p_{1}^{\prime} \cdot g_{1}^{-1}) \cdot g
\;\; = \;\;
	\psi^{(1)}_{\alpha}(p_{1}) \cdot g\,,
\end{equation*}
as required. This proves Claim 1.

\vskip 0.5cm
\noindent
\textbf{Claim 2}:\quad
$\Psi_{\alpha} : \pi_{\alpha}^{-1}(U_{\alpha}) \longrightarrow U_{\alpha} \times G$\, is injective.
\vskip 0.2cm
\proofof Claim 2:\quad
Suppose \,$\Psi_{\alpha}(p) \,=\, \Psi_{\alpha}(p^{\prime})$,\, for \,$p, p^{\prime} \in \pi^{-1}(U_{\alpha})$.\,
We need to establish that \,$p = p^{\prime}$.\,
Note first that \,$p = p_{1} \cdot g$\, and \,$p^{\prime} = p_{1}^{\prime} \cdot g^{\prime}$,\,
for some
\,$p_{1}, p_{1}^{\prime} \in \pi_{1}^{-1}(U_{\alpha}) \subset P_{1}$,\, and
\,$g, g^{\prime} \in G$.\,
Then,
\begin{equation*}
\left(\,\pi(p)\,,\,\psi^{(1)}_{\alpha}(p_{1}) \cdot g\,\right)
\;\; =: \;\;
	\Psi_{\alpha}(p)
\;\; = \;\;
	\Psi_{\alpha}(p^{\prime})
\;\; := \;\;
	\left(\,\pi(p^{\prime})\,,\,\psi^{(1)}_{\alpha}(p_{1}^{\prime}) \cdot g^{\prime}\,\right)
\end{equation*}
Now,
\begin{equation*}
\psi^{(1)}_{\alpha}(p_{1}) \cdot g
\;\; = \;\;
	\psi^{(1)}_{\alpha}(p_{1}^{\prime}) \cdot g^{\prime}
\quad \Longrightarrow \quad
g_{1}
\; := \;
	g^{\prime} \cdot g^{-1}
\; = \;
	\left(\psi^{(1)}_{\alpha}(p_{1}^{\prime})\right)^{-1} \cdot \psi^{(1)}_{\alpha}(p_{1})
\; \in \;
	G_{1}
\; \subset \;
	G\,,
\end{equation*}
which in turn implies:
\begin{equation*}
\psi^{(1)}_{\alpha}(p_{1})
\;\; = \;\;
	\psi^{(1)}_{\alpha}(p_{1}^{\prime} \cdot g_{1})
\end{equation*}
Next, note that
\begin{equation*}
\pi_{1}(p_{1}^{\prime} \cdot g_{1})
\; = \;
	\pi_{1}(p_{1}^{\prime})
\; = \;
	\pi(p_{1}^{\prime})
\; = \;
	\pi(p_{1}^{\prime} \cdot g^{\prime})
\; = \;
	\pi(p^{\prime})
\; = \;
	\pi(p)
\; = \;
	\pi(p_{1} \cdot g)
\; = \;
	\pi(p_{1})
\; = \;
	\pi_{1}(p_{1})
\end{equation*}
Therefore,
\begin{equation*}
\Psi^{(1)}_{\alpha}(p_{1})
\; := \;
	\left(\,\pi_{1}(p_{1})\,,\,\psi^{(1)}_{\alpha}(p_{1})\,\right)
\; = \;
	\left(\,\pi_{1}(p_{1}^{\prime} \cdot g_{1})\,,\,\psi^{(1)}_{\alpha}(p_{1}^{\prime} \cdot g_{1})\,\right)
\; =: \;
	\Psi^{(1)}_{\alpha}(p_{1}^{\prime} \cdot g_{1})
\end{equation*}
Bijectivity of \,$\Psi^{(1)}_{\alpha}$\, now implies:
\begin{equation*}
p_{1}
\; = \;
	p_{1}^{\prime} \cdot g_{1}
\; = \;
	p_{1}^{\prime} \cdot g^{\prime} \cdot g^{-1}
\quad \Longrightarrow \quad
p
\; = \;
	p_{1} \cdot g
\; = \;
	p_{1}^{\prime} \cdot g^{\prime}
\; = \;
	p^{\prime}
\end{equation*}
as required. This proves Claim 2.

\vskip 0.5cm
\noindent
\textbf{Claim 3}:\quad
$\Psi_{\alpha} : \pi_{\alpha}^{-1}(U_{\alpha}) \longrightarrow U_{\alpha} \times G$\, is surjective.
\vskip 0.2cm
\proofof Claim 3:\quad
Let \,$(\,x\,,\,g\,) \in U_{\alpha} \times G$.\,
Choose any \,$p^{\prime} \in \pi^{-1}(x)$.
Next, choose \,$p_{1}^{\prime} \in \pi_{1}^{-1}(x) \subset \pi_{1}^{-1}(U_{\alpha})$\, and \,$g^{\prime} \in G$\,
such that \,$p^{\prime} = p_{1}^{\prime} \cdot g^{\prime}$.\,
Now, define \,$g^{\prime\prime} := \left(\,g^{\prime}\,\right)^{-1} \cdot \left(\,\psi_{\alpha}^{-1}(p_{1}^{\prime})\,\right)^{-1} \cdot g$.\,
Lastly, define
\,$p := p^{\prime}\cdot g^{\prime\prime} = p_{1}^{\prime}\cdot g^{\prime} \cdot g^{\prime\prime} \in \pi^{-1}(x) \subset \pi^{-1}(U_{\alpha})$.\,
Then,
\begin{equation*}
\Psi_{\alpha}\!\left(\,p\,\right)
\; := \;
	\left(\,\pi(p)\,,\,\psi_{\alpha}^{(1)}(p_{1}^{\prime}) \cdot g^{\prime} \cdot g^{\prime\prime} \,\right)
\; = \;
	\left(\,x\,,\,g\,\right)
\end{equation*}
This establishes the surjectivity of \,$\Psi_{\alpha}$\, and completes the proof of Claim 3.

\vskip 0.5cm
\noindent
\textbf{Claim 4}:\quad
$\Psi_{\alpha} : \pi_{\alpha}^{-1}(U_{\alpha}) \longrightarrow U_{\alpha} \times G$\, is a bundle chart of
\,$G \longhookrightarrow P \overset{\pi}{\longrightarrow} M$.
\vskip 0.2cm
\proofof Claim 4:\quad
By Claims 1, 2, and 3,
\,$\Psi_{\alpha} : \pi_{\alpha}^{-1}(U_{\alpha}) \longrightarrow U_{\alpha} \times G$\,
is a (well-defined) diffeomorphism.
For each \,$p \in \pi^{-1}(U_{\alpha})$\, and \,$g \in G$,\,
we need to establish that
\,$\Psi_{\alpha}(\,p\,\cdot\,g\,) \, = \, \Psi_{\alpha}(\,p\,)\,\cdot\,g$.\,
To this end, choose \,$p_{1} \in \pi_{1}^{-1}(x) \subset \pi_{1}^{-1}(U_{\alpha})$\,
and \,$g^{\prime} \in G$\, such that \,$p = p_{1} \cdot g^{\prime}$.
Hence,
\begin{eqnarray*}
\Psi_{\alpha}(\,p \cdot g\,)
& = &
	\Psi_{\alpha}(\,p_{1} \cdot g^{\prime} \cdot g\,)
\;\; = \;\;
	\left(\,\pi(p_{1} \cdot g^{\prime} \cdot g)\,,\,\psi_{\alpha}^{(1)}(p_{1}) \cdot g^{\prime} \cdot g\,\right)
\;\; = \;\;
	\left(\,\pi(p_{1} \cdot g^{\prime})\,,\,\psi_{\alpha}^{(1)}(p_{1}) \cdot g^{\prime} \cdot g\,\right)
\\
& = &
	\left(\,\pi(p)\,,\,\psi_{\alpha}^{(1)}(p_{1}) \cdot g^{\prime}\,\right) \cdot g
\\
& = &
	\Psi_{\alpha}(\,p\,) \cdot g
\end{eqnarray*}
This proves Claim 4.

\vskip 0.5cm
\noindent
\textbf{Claim 5}:\quad
This transition functions of
\,$G \longhookrightarrow P \overset{\pi}{\longrightarrow} M$\,
with respect to the bundle atlas
\,$\left\{\,\Psi_{\alpha} : \pi_{\alpha}^{-1}(U_{\alpha}) \longrightarrow U_{\alpha} \times G\,\right\}_{\alpha\in\mathcal{A}}$\,
take values in \,$G_{1} \subset G$.
\vskip 0.2cm
\proofof Claim 5:\quad
Let \,$x \in U_{\alpha} \cap U_{\beta}$.\,
Choose \,$p \in \pi^{-1}(x)$.\,
Then, choose \,$p_{1} \in \pi_{1}^{-1}(x)$\, and \,$g \in G$\, such that \,$p = p_{1} \cdot g$.\,
Hence,
\begin{eqnarray*}
\tau_{\beta\alpha}(x)
& := &
	\psi_{\beta}(p) \cdot \psi_{\alpha}(p)^{-1}
\;\; = \;\;
	\left(\,\psi_{\beta}^{(1)}(p_{1}) \cdot g\,\right) \cdot \left(\,\psi_{\alpha}^{(1)}(p_{1}) \cdot g\,\right)^{-1}	
\;\; = \;\;
	\psi_{\beta}^{(1)}(p_{1}) \cdot g \cdot g^{-1} \cdot \psi_{\alpha}^{(1)}(p_{1})^{-1}	
\\
& = &
	\psi_{\beta}^{(1)}(p_{1}) \cdot \psi_{\alpha}^{(1)}(p_{1})^{-1}	
\\
& \in &
	G_{1}
\end{eqnarray*}
This proves Claim 5, and completes the proof that
the reducibility of the structure group \,$G$\, of
\,$G \longhookrightarrow P \overset{\pi}{\longrightarrow} M$\,
to a subgroup \,$G_{1} \subset G$\, implies that
\,$G \longhookrightarrow P \overset{\pi}{\longrightarrow} M$\,
admits a bundle atlas whose transition functions take values in $G_{1}$.

\vskip 1.0cm
\noindent
\underline{$(\,\Longleftarrow\,)$}\;\;
Conversely, suppose that the principal fibre bundle
\,$G \longhookrightarrow P \overset{\pi}{\longrightarrow} M$\,
admits a bundle atlas whose transition functions take values in the subgroup $G_{1} \subset G$.
We need to prove that the structure group \,$G$\, is reducible to $G_{1}$, i.e. there exists an embedding
\,$\left(\,\widetilde{f} : P_{1} \longrightarrow P \,,\, \iota : G_{1} \longhookrightarrow G\,\right)$\,
of principal fibre bundles
whose induced map \,$f$\, on the base manifold \,$M$\, is the identity map on \,$M$.\,

\qed

          %%%%% ~~~~~~~~~~~~~~~~~~~~ %%%%%

          %%%%% ~~~~~~~~~~~~~~~~~~~~ %%%%%
