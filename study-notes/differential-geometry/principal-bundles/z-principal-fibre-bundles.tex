
          %%%%% ~~~~~~~~~~~~~~~~~~~~ %%%%%

\section{Principal fibre bundles}
\setcounter{theorem}{0}
\setcounter{equation}{0}

%\cite{vanDerVaart1996}
%\cite{Kosorok2008}

%\renewcommand{\theenumi}{\alph{enumi}}
%\renewcommand{\labelenumi}{\textnormal{(\theenumi)}$\;\;$}
\renewcommand{\theenumi}{\roman{enumi}}
\renewcommand{\labelenumi}{\textnormal{(\theenumi)}$\;\;$}

          %%%%% ~~~~~~~~~~~~~~~~~~~~ %%%%%

\begin{definition}[Principal fibre bundle]
\mbox{}
\vskip 0.2cm
\noindent
A \,\textbf{principal fibre bundle}\, (or a \textbf{principal $G$-bundle})
is a fibre bundle \,$\pi : P \longrightarrow M$\,
whose general fibre \,$G$\, is a Lie group such that
\begin{itemize}
\item
	$G$ acts smoothly on $P$ {\color{red}on the right},
\item
	the action of $G$ on $P$ preserves each fibre of $\pi$, i.e. \,$G\cdot\pi^{-1}(x) \subset \pi^{-1}(x)$,\,
	for each $x \in M$,
\item
	the restricted action of $G$ to each fibre is free, i.e.
	the orbit map
	\begin{equation*}
	\iota_{p} : G \longrightarrow \pi^{-1}(x) \; : \; g \longmapsto p \cdot g
	\end{equation*}
	is injective,
	for each $p \in \pi^{-1}(x)$, $x \in M$,
\item
	the restricted action of $G$ to each fibre is transitive, i.e.
	the orbit map $\iota_{p} : G \longrightarrow \pi^{-1}(x)$ is surjective,
	for each $p \in \pi^{-1}(x)$, $x \in M$,
\item
	there exists a \textbf{bundle atlas}
	\,$\left\{\,\Psi_{\alpha} : \pi^{-1}(U_{\alpha}) \longrightarrow U_{\alpha} \times G\,\right\}_{\alpha\in\mathcal{A}}$\,
	of \,{\color{red}$G$-equivariant \textbf{bundle charts}}, i.e.
	\,$\left\{\,U_{\alpha}\,\right\}_{\alpha\in\mathcal{A}}$\,
	is an open covering of \,$M$,\, and
	\begin{equation*}
	\Psi_{\alpha}(p \cdot g) \;\; = \;\; \Psi_{\alpha}(p) \cdot g\,,
	\quad
	\textnormal{for each \,$\alpha \in \mathcal{A}$, \,$p \in \pi^{-1}(U_{\alpha})$, \,$g \in G$},
	\end{equation*}
	where the right action of \,$G$\, on \,$U_{\alpha} \times G$ is given by:
	\begin{equation*}
	(\,x,h\,) \cdot g \;\; = \;\; (\,x,h \cdot g\,)\,,
	\quad
	\textnormal{for each \,$x \in M$, \,$g,h \in G$}.
	\end{equation*}
\end{itemize}
The general fibre \,$G$\, is called the \textbf{structure group} of the principal fibre bundle.
\end{definition}

          %%%%% ~~~~~~~~~~~~~~~~~~~~ %%%%%

\vskip 0.5cm
\begin{definition}[Transition functions of a principal fibre bundle]
\mbox{}
\vskip 0.2cm
\noindent
The set of \textbf{transition functions} of a principal fibre bundle
\,$G \longhookrightarrow P \overset{\pi}{\longrightarrow} M$\,
with respect to the bundle atlas
\,$\left\{\,\Psi_{\alpha} = (\,\pi,\,\psi_{\alpha}\,) : \pi^{-1}(U_{\alpha}) \longrightarrow U_{\alpha} \times G\,\right\}_{\alpha\in\mathcal{A}}$\,
is the collection
\,$\left\{\,\tau_{\beta\alpha} : U_{\beta} \cap U_{\alpha} \longrightarrow G\,\right\}_{\alpha,\beta\in\mathcal{A}}$\,
of smooth $G$-valued functions, where each
\,$\tau_{\beta\alpha}$\,
is defined as follows:
For each $x \in U_{\beta} \cap U_{\alpha}$,
\begin{equation*}
\tau_{\beta\alpha}(x)
\; := \;
	\psi_{\beta}(p) \cdot \psi_{\alpha}(p)^{-1}
\; \in \;
	G\,,
\quad
\textnormal{for any \,$p \in \pi^{-1}(x)$}
\end{equation*}
\end{definition}

          %%%%% ~~~~~~~~~~~~~~~~~~~~ %%%%%

\begin{remark}
\mbox{}
\vskip 0.2cm
\noindent
The above definition is indeed well-defined since
\begin{equation*}
\psi_{\beta}(p \cdot g) \cdot \psi_{\alpha}(p \cdot g)^{-1}
\;\; = \;\;
	\left(\overset{{\color{white}.}}{\psi}_{\beta}(p)\cdot g\right) \cdot \left(\overset{{\color{white}.}}{\psi}_{\alpha}(p)\cdot g\right)^{-1}
\;\; = \;\;
	\psi_{\beta}(p) \cdot g \cdot g^{-1} \cdot \psi_{\alpha}(p)^{-1}
\;\; = \;\;
	\psi_{\beta}(p) \cdot \psi_{\alpha}(p)^{-1}
\end{equation*}
which shows that the expression
\,$\psi_{\beta}(p) \cdot \psi_{\alpha}(p)^{-1}$\,
is independent of the particular choice of \,$p \in \pi^{-1}(x)$.\,
Note that we have used the equivariance assumption on the bundle charts,
and the transitivity of the right $G$-action on each fibre.
\end{remark}

          %%%%% ~~~~~~~~~~~~~~~~~~~~ %%%%%

\vskip 0.5cm
\begin{proposition}[Transition functions satisfy cocycle condition]
\mbox{}
\vskip 0.2cm
\noindent
The set
\,$\left\{\,\tau_{\beta\alpha} : U_{\beta} \cap U_{\alpha} \longrightarrow G\,\right\}_{\alpha,\beta\in\mathcal{A}}$\,
of transition functions of a principal fibre bundle
\,$G \longhookrightarrow P \overset{\pi}{\longrightarrow} M$\,
with respect to a bundle atlas
\,$\left\{\,\Psi_{\alpha} = (\,\pi\,,\psi_{\alpha}\,) : \pi^{-1}(U_{\alpha}) \longrightarrow U_{\alpha} \times G\,\right\}_{\alpha\in\mathcal{A}}$\,
satisfies the following \textbf{cocycle condition}:
\begin{equation*}
\tau_{\gamma\alpha}(x) \; = \; \tau_{\gamma\beta}(x) \cdot \tau_{\beta\alpha}(x) \; \in \; G\,,
\quad
\textnormal{for each \,$x \in U_{\alpha} \cap U_{\beta} \cap U_{\gamma}$}
\end{equation*}
\end{proposition}

          %%%%% ~~~~~~~~~~~~~~~~~~~~ %%%%%

\vskip 0.5cm
\begin{proposition}[Cocycle condition determines a principal fibre bundle]
\label{cocycleDeterminesPFB}
\mbox{}
\vskip 0.2cm
\noindent
Suppose \,$M$\, is a smooth manifold,
\,$\left\{\,U_{\alpha}\,\right\}_{\alpha\in\mathcal{A}}$\,
is an open covering of \,$M$,\, and
\,$G$\, is a Lie group.
If a collection
\,$\left\{\,\tau_{\beta\alpha} : U_{\beta} \cap U_{\alpha} \longrightarrow G\,\right\}_{\alpha,\beta\in\mathcal{A}}$\,
of smooth \,$G$-valued functions
satisfies the cocycle condition:
\begin{equation*}
\tau_{\gamma\alpha}(x) \; = \; \tau_{\gamma\beta}(x) \cdot \tau_{\beta\alpha}(x) \; \in \; G\,,
\quad
\textnormal{for each \,$x \in U_{\alpha} \cap U_{\beta} \cap U_{\gamma}$}\,,
\end{equation*}
then there exists a principal fibre bundle
\,$G \longhookrightarrow P \overset{\pi}{\longrightarrow} M$\,
with a bundle atlas
\,$\left\{\,\Psi_{\alpha} = (\,\pi\,,\psi_{\alpha}\,) : \pi^{-1}(U_{\alpha}) \longrightarrow U_{\alpha} \times G\,\right\}_{\alpha\in\mathcal{A}}$\,
such that
\,$\left\{\,\tau_{\beta\alpha} : U_{\beta} \cap U_{\alpha} \longrightarrow G\,\right\}_{\alpha,\beta\in\mathcal{A}}$\,
is the corresponding collection of transition functions.
\end{proposition}
\proof
See Proposition 5.2, p.52, \cite{Kobayashi1963v1}
\qed

          %%%%% ~~~~~~~~~~~~~~~~~~~~ %%%%%

