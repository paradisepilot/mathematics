
          %%%%% ~~~~~~~~~~~~~~~~~~~~ %%%%%

\section{Principal fibre bundles}
\setcounter{theorem}{0}
\setcounter{equation}{0}

%\cite{vanDerVaart1996}
%\cite{Kosorok2008}

%\renewcommand{\theenumi}{\alph{enumi}}
%\renewcommand{\labelenumi}{\textnormal{(\theenumi)}$\;\;$}
\renewcommand{\theenumi}{\roman{enumi}}
\renewcommand{\labelenumi}{\textnormal{(\theenumi)}$\;\;$}

          %%%%% ~~~~~~~~~~~~~~~~~~~~ %%%%%

\begin{definition}[Principal fibre bundle]
\mbox{}
\vskip 0.1cm
\noindent
A \,\textbf{principal fibre bundle}\, (or a \textbf{principal $G$-bundle})
is a fibre bundle \,$\pi : P \longrightarrow M$\,
whose general fibre \,$G$\, is a Lie group such that
\begin{itemize}
\item
	$G$ acts smoothly on $P$ {\color{red}on the right},
\item
	the action of $G$ on $P$ preserves each fibre of $\pi$, i.e. \,$G\cdot\pi^{-1}(x) \subset \pi^{-1}(x)$,\,
	for each $x \in M$,
\item
	the restricted action of $G$ to each fibre is free, i.e.
	the orbit map
	\begin{equation*}
	\iota_{p} : G \longrightarrow \pi^{-1}(x) \; : \; g \longmapsto p \cdot g
	\end{equation*}
	is injective,
	for each $p \in \pi^{-1}(x)$, $x \in M$,
\item
	the restricted action of $G$ to each fibre is transitive, i.e.
	the orbit map $\iota_{p} : G \longrightarrow \pi^{-1}(x)$ is surjective,
	for each $p \in \pi^{-1}(x)$, $x \in M$,
\item
	there exists a bundle atlas
	\,$\left\{\,\varphi_{\alpha} : \pi^{-1}(U_{\alpha}) \longrightarrow U_{\alpha} \times G\,\right\}_{\alpha\in\mathcal{A}}$\,
	of {\color{red}$G$-equivariant bundle charts}, i.e.
	\begin{equation*}
	\varphi_{\alpha}(p \cdot g) \;\; = \;\; \varphi_{\alpha}(p) \cdot g\,,
	\quad
	\textnormal{for each \,$\alpha \in \mathcal{A}$, \,$p \in \pi^{-1}(U_{\alpha})$, \,$g \in G$},
	\end{equation*}
	where the right action of \,$G$\, on \,$U_{\alpha} \times G$ is given by:
	\begin{equation*}
	(x,h) \cdot g \;\; = \;\; (x,h \cdot g)\,,
	\quad
	\textnormal{for each \,$x \in M$, \,$g,h \in G$}.
	\end{equation*}
\end{itemize}
The general fibre \,$G$\, is called the \textbf{structure group} of the principal fibre bundle.
\end{definition}

          %%%%% ~~~~~~~~~~~~~~~~~~~~ %%%%%

\vskip 0.5cm
\begin{proposition}[Images of orbit maps equal kernels of canonical projection]
\mbox{}
\vskip 0.1cm
\noindent
Let \,$G \longhookrightarrow P \overset{\pi}{\longrightarrow} M$\, be a principal $G$-bundle.
For each \,$p \in P$,\, let \,$\iota_{p} : G \longrightarrow P : g \longmapsto p \cdot g$\, be the orbit map of \,$p \in P$.\,
Then,
\begin{equation*}
\image\!\left(\;T_{e}\iota_{p}\,\right) \;\; = \;\; \ker\!\left(\,T_{p}\pi\,\right)
\end{equation*}
Equivalently, the following is a short exact sequence of linear maps between vector spaces:
\begin{center}
\begin{tikzcd}[column sep=scriptsize]
0 {\color{white}.} \arrow{rr}{} &&
{\color{white}.}\mathfrak{g} := T_{e}G{\color{white}.} \arrow{rr}{T_{e}\iota_{p}} &&
{\color{white}.}T_{p}P{\color{white}.} \arrow{rr}{T_{p}\pi} &&
{\color{white}.}T_{\pi(p)}M{\color{white}.} \arrow{rr}{} &&
0
\end{tikzcd}
\end{center}
\end{proposition}
\proof
\vskip 0.3cm
\noindent
\underline{$\image\!\left(\;T_{e}\iota_{p}\,\right) \;\subset\; \ker\!\left(\,T_{p}\pi\,\right)$}
\vskip 0.2cm
\noindent
Note that, for each \,$A \in \mathfrak{g} := T_{e}G$\, and each \,$p \in P$,\,
\begin{equation*}
T_{p}\pi\!\left(\,T_{e}\iota_{p}(\overset{{\color{white}.}}{A})\right)
\;\; = \;\;
	\left(\,T_{p}\pi \,\overset{{\color{white}1}}{\circ}\, T_{e}\iota_{p}\,\right)\!(A)
\;\; = \;\;
	\left(\,T_{e}(\pi\,\overset{{\color{white}1}}{\circ}\,\iota_{p})\,\right)\!(A)
\;\; = \;\;
	0\,,
\end{equation*}
where the last equality follows from the fact that
\,$\pi\,\circ\,\iota_{p} : G \longrightarrow M$\,
is a constant map:
\,$\pi\,\circ\,\iota_{p}(g) \,=\, \pi(p \cdot g) \,=\, \pi(p)$,\,
as required.

\vskip 0.3cm
\noindent
\underline{$\image\!\left(\;T_{e}\iota_{p}\,\right) \;=\; \ker\!\left(\,T_{p}\pi\,\right)$}
\vskip 0.2cm
\noindent
First, note that the following (trivial) short exact sequence
\begin{center}
\begin{tikzcd}[column sep=scriptsize]
0 {\color{white}.} \arrow{rr}{} &&
{\color{white}.}\ker\!\left(\,T_{p}\pi\,\right){\color{white}.} \arrow{rr}{} &&
{\color{white}.}T_{p}P{\color{white}.} \arrow{rr}{T_{p}\pi} &&
{\color{white}.}T_{\pi(p)}M{\color{white}.} \arrow{rr}{} &&
0
\end{tikzcd}
\end{center}
implies the following series of equalities
\begin{equation*}
\dim M + \dim G
\; = \;
	\dim\!\left(\,U_{\alpha} \times G\;\!\right)
\; = \;
	\dim P
\; = \; 
	\dim T_{p}P
\; = \; 
	\dim\ker\!\left(\,T_{p}\pi\,\right) \, + \, \dim T_{\pi(p)}M
\; = \; 
	\dim\ker\!\left(\,T_{p}\pi\,\right) \, + \, \dim M
\end{equation*}
which in turn implies that
\begin{equation*}
\dim\ker\!\left(\,T_{p}\pi\,\right) \; = \; \dim G
\end{equation*}
On the other hand, recall that, for each $p \in G$,
the fibre \,$\pi^{-1}(\pi(p)) \subset P$\, is a submanifold of \,$P$,\,
and the orbit map
\begin{equation*}
\iota_{p} : G \longrightarrow \pi^{-1}(\pi(p))
\end{equation*}
is a diffeomorphism. Hence, its differential:
\begin{equation*}
T_{e}\iota_{p} \; : \; \mathfrak{g} := T_{e}G \; \longrightarrow \; T_{p}\,\pi^{-1}(\pi(p))
\end{equation*}
is a vector-space isomorphism;
in particular, \,$T_{e}\iota_{p} \,:\, \mathfrak{g} := T_{e}G \,\longrightarrow\, T_{p}P$\, is an injective (linear) map.
Hence,
\begin{equation*}
\dim\,\image\!\left(\,T_{e}\iota_{p}\,\right) \;=\; \dim T_{e}G \;=\; \dim G \;=\; \dim\ker\!\left(\,T_{p}\pi\,\right)
\end{equation*}
But, we have already established that
\,$\image\!\left(\;T_{e}\iota_{p}\,\right) \,\subset\, \ker\!\left(\,T_{p}\pi\,\right)$.\,
Hence, we may now conclude that:
\begin{equation*}
\image\!\left(\;T_{e}\iota_{p}\,\right) \; = \;  \ker\!\left(\,T_{p}\pi\,\right)
\end{equation*}
This completes the proof of the Proposition.
\qed

          %%%%% ~~~~~~~~~~~~~~~~~~~~ %%%%%

\vskip 0.5cm
\begin{definition}[Vertical tangent vectors of a principal fibre bundle]
\mbox{}
\vskip 0.1cm
\noindent
Let \,$G \longhookrightarrow P \overset{\pi}{\longrightarrow} M$\, be a principal $G$-bundle.
For each \,$p \in P$,\, let \,$\iota_{p} : G \longrightarrow P : g \longmapsto p \cdot g$\, be the orbit map of \,$p \in P$.\,
A tangent vector \,$v \in T_{p}P$\, is said to be \textbf{vertical} if
\begin{equation*}
v \;\in\; \image\!\left(\,T_{e}\iota_{p}\right) \;=\; \ker\!\left(\,T_{p}\pi\right) \;\subset\; T_{p}P\,.
\end{equation*}
\end{definition}

          %%%%% ~~~~~~~~~~~~~~~~~~~~ %%%%%


          %%%%% ~~~~~~~~~~~~~~~~~~~~ %%%%%
