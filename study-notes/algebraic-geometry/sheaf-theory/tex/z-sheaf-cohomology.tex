
          %%%%% ~~~~~~~~~~~~~~~~~~~~ %%%%%

\section{Sheaf cohomology}

%\cite{vanDerVaart1996}
%\cite{Kosorok2008}

%\renewcommand{\theenumi}{\alph{enumi}}
%\renewcommand{\labelenumi}{\textnormal{(\theenumi)}$\;\;$}
\renewcommand{\theenumi}{\roman{enumi}}
\renewcommand{\labelenumi}{\textnormal{(\theenumi)}$\;\;$}

          %%%%% ~~~~~~~~~~~~~~~~~~~~ %%%%%

\subsection{Sheaf cohomology theory -- a summary}
\setcounter{theorem}{0}
\setcounter{equation}{0}

\begin{itemize}
\item
	The category of sheaves of abelian groups on any topological space is an abelian category with enough injectives.
	See Example III.1.0.3, p.202, \cite{hartshorne1977algebraic}, and Corollary III.2.3, p.207, \cite{hartshorne1977algebraic}.
	\vskip 0.05cm
	The category of sheaves of \,$\mathcal{O}_{X}$-modules\, on any ringed space \,$(X,\mathcal{O}_{X})$\,
	is an abelian category with enough injectives.
	See Example III.1.0.4, p.202, \cite{hartshorne1977algebraic}, and Proposition III.2.2, p.207, \cite{hartshorne1977algebraic}.
	\vskip 0.05cm
	The category of quasi-coherent sheaves of \,$\mathcal{O}_{X}$-modules\,
	on a scheme \,$(X,\mathcal{O}_{X})$\, is an abelian category.
	See Example III.1.0.4, p.202, \cite{hartshorne1977algebraic}.
	\vskip 0.05cm
	The category of coherent sheaves of \,$\mathcal{O}_{X}$-modules\,
	on a noetherian scheme \,$(X,\mathcal{O}_{X})$\, is an abelian category.
	See Example III.1.0.6, p.202, \cite{hartshorne1977algebraic}.
	\vskip 0.3cm

\item
	Recall:\; A covariant functor \,$F : \mathfrak{A} \longrightarrow \mathfrak{B}$\,
	is said to be \textbf{left-exact} if the exactness of
	\begin{center}
	\begin{tikzcd}
	0
		\arrow[r]
		%\arrow[rr, thick, two heads, "\varepsilon_{A}"]
	&
	 A^{\prime}
		\arrow[r, "\alpha"]
	&
	A
		\arrow[r, "\beta"]
	&
	A^{\prime\prime}
	\end{tikzcd}
	\end{center}
	implies the exactness of
	\begin{center}
	\begin{tikzcd}
	0
		\arrow[r]
	&
	 F(A^{\prime})
		\arrow[r, "F(\alpha)"]
	&
	F(A)
		\arrow[r, "F(\beta)"]
	&
	F(A^{\prime\prime})
	\end{tikzcd}
	\end{center}
	\vskip 0.3cm

\item
	Let \,$\mathfrak{A}$\, be an abelian category with enough injectives.
	Let \,$F : \mathfrak{A} \longrightarrow \mathfrak{B}$\, be a covariant left-exact functor
	from \,$\mathfrak{A}$\, to another abelian category \,$\mathfrak{B}$.\,

	For each integer \,$p \geq 0$,\, the \textbf{$p$-th right derived functor} \,$R^{p}\,F$\, of \,$F$\, is defined by:
	\begin{equation*}
	R^{p}\,F(A)
	\;\; := \;\;
		H^{p}\!\left(\,F(I_{A}^{\bullet})\right)
	\;\; := \;\;
		\dfrac{
			\ker\!\left(\,\overset{{\color{white}.}}{F(\alpha^{p})}\,\right)
		}{
			\image\!\left(\,\overset{{\color{white}.}}{F(\alpha^{p-1})}\,\right)
		}\,,
	\quad
	\textnormal{for each object \,$A \in \Obj(\mathfrak{A})$},
	\end{equation*}
	where \,$I_{A}^{\bullet}$\, is the sequence:\,
	\begin{equation*}
	0
	\;\;\overset{\alpha^{-1}}{\longrightarrow}\;\;
		I_{A}^{0}
	\;\;\overset{\alpha^{0}}{\longrightarrow}\;\;
		I_{A}^{1}
	\;\;\overset{\alpha^{1}}{\longrightarrow}\;\;
		\cdots
	\end{equation*}
	obtained by left-truncating an arbitrary injective resolution of \,$A$:\,
	\begin{equation}\label{injectiveResolutionOfA}
	0
	\;\;\longrightarrow\;\;
		A 
	\;\;\overset{\iota}{\longrightarrow}\;\;
		I_{A}^{0}
	\;\;\overset{\alpha^{0}}{\longrightarrow}\;\;
		I_{A}^{1}
	\;\;\overset{\alpha^{1}}{\longrightarrow}\;\;
		\cdots
	\end{equation}
	Recall that the sequence \eqref{injectiveResolutionOfA} being an injective resolution of \,$A$\,
	means that \eqref{injectiveResolutionOfA} is exact, and \,$I_{A}^{k} \in \Obj(\mathfrak{A})$\,
	is an injective object, for each \,$p = 0, 1, 2, \ldots$\,

	Note that, for \,$p = 0$,\, we have:
	\begin{equation*}
	R^{0}\,F(A)
	\; := \;
		H^{0}\!\left(\,F(I_{A}^{\bullet})\right)
	\; := \;
		\dfrac{
			\ker\!\left(\,\overset{{\color{white}.}}{F(\alpha^{0})}\,\right)
		}{
			\image\!\left(\,\overset{{\color{white}.}}{F(\alpha^{-1})}\,\right)
		}\
	\; = \;
		\ker\!\left(\,\overset{{\color{white}.}}{F(\alpha^{0})}\,\right)
	\; = \;
		\image\!\left(\,\overset{{\color{white}.}}{F(\iota)}\,\right)
	\; = \;
		F(A)\,,
	\end{equation*}
	where the last two equalities follows from the left-exactness of \,$F$.\,
	
	For any short exact sequence,
	\begin{center}
	\begin{tikzcd}
	0
		\arrow[r]
	&
	A^{\prime}
		\arrow[r]
	&
	A
		\arrow[r]
	&
	A^{\prime\prime}
		\arrow[r]
	&
	0
	\end{tikzcd}
	\end{center}
	there exists, for each non-negative integer \,$p \geq 0$,\,
	a natural morphism
	\,$\delta^{i} : R^{p}F(A^{\prime\prime}) \longrightarrow R^{p+1}F(A^{\prime})$\,
	such that we obtain a long exact sequence
	\begin{equation*}
	\cdots
	\;\longrightarrow\; R^{p}F(A^{\prime})
	\;\longrightarrow\; R^{p}F(A)
	\;\longrightarrow\; R^{p}F(A^{\prime\prime})
	\;\overset{\delta^{p}}{\longrightarrow}\; R^{p+1}F(A^{\prime})
	\;\longrightarrow\; R^{p+1}F(A)
	\;\longrightarrow\; R^{p+1}F(A^{\prime\prime})
	\;\longrightarrow\; \cdots
	\end{equation*}
	For a proof of  this fact, see Theorem 11.31, p.479, \cite{gallier2022homology}.

	We give however an outline: The existence of such a long exact sequence follows
	from the Horseshoe Lemma and the Snake Lemma.
	For the Horseshoe Lemma, see Proposition 11.25, p.463, \cite{gallier2022homology}.
	For a proof of the ``projective'' version of the Horseshoe Lemma,
	see Proposition 6.24, p. 349, \cite{rotman2008introduction}.
	For the Snake Lemma, see, for example, \cite{Surowski2000}, or
	Lemma 12.1.1, p.297, \cite{kashiwara2005categories}.

	The Horsehoe Lemma states that, given
	\begin{itemize}
	\item
		a short exact sequence
		\begin{center}
		\begin{tikzcd}
		0
			\arrow[r]
		&
		A^{\prime}
			\arrow[r]
		&
		A
			\arrow[r]
		&
		A^{\prime\prime}
			\arrow[r]
		&
		0\,,
		\quad
		\textnormal{and}
		\end{tikzcd}
		\end{center}
	\item
		injective resolutions of \,$A^{\prime}$\, and \,$A^{\prime\prime}$\,
		\begin{center}
		\begin{tikzcd}
		0
			\arrow[r]
		&
		A^{\prime}
			\arrow[r]
		&
		I^{\,0}_{A^{\prime}}
			\arrow[r]
		&
		I^{\,1}_{A^{\prime}}
			\arrow[r]
		&
		\cdots
		\\
		0
			\arrow[r]
		&
		A^{\prime\prime}
			\arrow[r]
		&
		I^{\,0}_{A^{\prime\prime}}
			\arrow[r]
		&
		I^{\,1}_{A^{\prime\prime}}
			\arrow[r]
		&
		\cdots
		\end{tikzcd}
		\end{center}
	\end{itemize}
	there exists an injective resolution of \,$A$\,
		\begin{center}
		\begin{tikzcd}
		0
			\arrow[r]
		&
		A
			\arrow[r]
		&
		I^{\,0}_{A}
			\arrow[r]
		&
		I^{\,1}_{A}
			\arrow[r]
		&
		\cdots
		\end{tikzcd}
		\end{center}
	and morphisms
	\begin{equation*}
	I^{p}_{A^{\prime}} \longrightarrow I^{p}_{A}
	\quad\quad\quad\textnormal{and}\quad\quad\quad
	I^{p}_{A} \longrightarrow I^{p}_{A^{\prime\prime}}
	\end{equation*}
	\vskip 0.2cm
	such that the following diagram commutes and has exact rows and exact columns:
	\begin{center}
	\begin{tikzcd}
	&\vdots&{\color{red}\vdots}&\vdots&
	\\
	0
		\arrow[r]
	&
	I^{\,1}_{A^{\prime}}
		\arrow[r, red]
		\arrow[u]
	&
	{\color{red}I^{\,1}_{A}}
		\arrow[r, red]
		\arrow[u, red]
	&
	I^{\,1}_{A^{\prime\prime}}
		\arrow[r]
		\arrow[u]
	&
	0
	\\
	0
		\arrow[r]
	&
	I^{\,0}_{A^{\prime}}
		\arrow[r, red]
		\arrow[u]
	&
	{\color{red}I^{\,0}_{A}}
		\arrow[r, red]
		\arrow[u, red]
	&
	I^{\,0}_{A^{\prime\prime}}
		\arrow[r]
		\arrow[u]
	&
	0
	\\
	0
		\arrow[r]
	&
	A^{\prime}
		\arrow[r]
		\arrow[u]
	&
	A
		\arrow[r]
		\arrow[u, red]
	&
	A^{\prime\prime}
		\arrow[r]
		\arrow[u]
	&
	0
	\\
	&
	0
		\arrow[u]
	&
	0
		\arrow[u]
	&
	0
		\arrow[u]
	&
	\end{tikzcd}
	\end{center}
	Removing the bottom row
	($0 \longrightarrow A^{\prime} \longrightarrow A \longrightarrow A^{\prime\prime} \longrightarrow 0$),
	we get a short exact sequence of cochain complexes:
	\begin{center}
	\begin{tikzcd}
	&\vdots&{\color{red}\vdots}&\vdots&
	\\
	0
		\arrow[r]
	&
	I^{\,2}_{A^{\prime}}
		\arrow[r, red]
		\arrow[u]
	&
	{\color{red}I^{\,1}_{A}}
		\arrow[r, red]
		\arrow[u, red]
	&
	I^{\,2}_{A^{\prime\prime}}
		\arrow[r]
		\arrow[u]
	&
	0
	\\
	0
		\arrow[r]
	&
	I^{\,2}_{A^{\prime}}
		\arrow[r, red]
		\arrow[u]
	&
	{\color{red}I^{\,1}_{A}}
		\arrow[r, red]
		\arrow[u, red]
	&
	I^{\,1}_{A^{\prime\prime}}
		\arrow[r]
		\arrow[u]
	&
	0
	\\
	0
		\arrow[r]
	&
	I^{\,0}_{A^{\prime}}
		\arrow[r, red]
		\arrow[u]
	&
	{\color{red}I^{\,0}_{A}}
		\arrow[r, red]
		\arrow[u, red]
	&
	I^{\,0}_{A^{\prime\prime}}
		\arrow[r]
		\arrow[u]
	&
	0
	\\
	&
	0
		\arrow[u]
	&
	0
		\arrow[u]
	&
	0
		\arrow[u]
	&
	\end{tikzcd}
	\end{center}
	Apply the functor \,$F$\,
	\begin{center}
	\begin{tikzcd}
	&\vdots&{\color{red}\vdots}&\vdots&
	\\
	0
		\arrow[r]
	&
	F(\,I^{\,2}_{A^{\prime}}\,)
		\arrow[r, red]
		\arrow[u]
	&
	F(\,{\color{red}I^{\,1}_{A}}\,)
		\arrow[r, red]
		\arrow[u, red]
	&
	F(\,I^{\,2}_{A^{\prime\prime}}\,)
		\arrow[r]
		\arrow[u]
	&
	0
	\\
	0
		\arrow[r]
	&
	F(\,I^{\,2}_{A^{\prime}}\,)
		\arrow[r, red]
		\arrow[u]
	&
	F(\,{\color{red}I^{\,1}_{A}}\,)
		\arrow[r, red]
		\arrow[u, red]
	&
	F(\,I^{\,1}_{A^{\prime\prime}}\,)
		\arrow[r]
		\arrow[u]
	&
	0
	\\
	0
		\arrow[r]
	&
	F(\,I^{\,0}_{A^{\prime}}\,)
		\arrow[r, red]
		\arrow[u]
	&
	F(\,{\color{red}I^{\,0}_{A}}\,)
		\arrow[r, red]
		\arrow[u, red]
	&
	F(\,I^{\,0}_{A^{\prime\prime}}\,)
		\arrow[r]
		\arrow[u]
	&
	0
	\\
	&
	0
		\arrow[u]
	&
	0
		\arrow[u]
	&
	0
		\arrow[u]
	&
	\end{tikzcd}
	\end{center}
	Each row in the above diagram is exact ({\color{red}Why?}
	See proof of Theorem 11.31 p.482, \cite{gallier2022homology}).
	Thus, it remains a short exact sequence of cochain complexes.
	The Snake Lemma now implies that we obtain the desired long exact sequence of cohomology groups.
	\vskip 0.3cm

\item
	Let \,$X$\, be a topological space.
	Let \,$\Gamma_{X} : \mathfrak{Ab}(X) \longrightarrow \mathfrak{Ab}$\,
	be the global section functor,
	where  \,$\mathfrak{Ab}(X)$\, denotes the category of sheaves of abelian groups on \,$X$,\, and
	\,$\mathfrak{Ab}$\ denotes the category of abelian groups.
	\vskip 0.05cm
	Let \,$\mathscr{F} \in \Obj(\mathfrak{Ab}(X))$\, be a sheaf of abelian groups defined on \,$X$.\,
	Since \,$\mathfrak{Ab}(X)$\, has enough injectives, we may choose an injective resolution for \,$\mathscr{F}$:\,
	\begin{center}
	\begin{tikzcd}
	0
		\arrow[r]
	&
	\mathscr{F} 
		\arrow[r]
	&
	I_{\mathscr{F}}^{0}
		\arrow[r]
	&
	I_{\mathscr{F}}^{1}
		\arrow[r]
	&
	\cdots
	\end{tikzcd}
	\end{center}
	For each \,$p = 0, 1, 2, \ldots$\,,\, we define:
	\begin{equation*}
	H^{p}(X,\mathscr{F})
	\;\; := \;\;
		R^{p}\,\Gamma_{X}(\mathscr{F})
	\;\; := \;\;
		H^{p}\!\left(\,\Gamma_{X}(I_{\mathscr{F}}^{\bullet})\right)
	\end{equation*}
	\vskip 0.3cm

\item
	Let \,$X$\, be a topological space, and let
	\begin{center}
	\begin{tikzcd}
	0
		\arrow[r]
		%\arrow[rr, thick, two heads, "\varepsilon_{A}"]
	&
	\mathscr{F}
		\arrow[r]
	&
	\mathscr{G}
		\arrow[r]
	&
	\mathscr{H}
		\arrow[r]
	&
	0
	\end{tikzcd}
	\end{center}
	be a short exact sequence of sheaves of abelian groups defined on \,$X$.\,

\end{itemize}

          %%%%% ~~~~~~~~~~~~~~~~~~~~ %%%%%
