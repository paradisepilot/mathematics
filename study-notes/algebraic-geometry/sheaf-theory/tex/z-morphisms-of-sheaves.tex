
          %%%%% ~~~~~~~~~~~~~~~~~~~~ %%%%%

\vskip 0.5cm
\section{Morphisms of sheaves of abelian groups}

%\cite{vanDerVaart1996}
%\cite{Kosorok2008}

%\renewcommand{\theenumi}{\alph{enumi}}
%\renewcommand{\labelenumi}{\textnormal{(\theenumi)}$\;\;$}
\renewcommand{\theenumi}{\roman{enumi}}
\renewcommand{\labelenumi}{\textnormal{(\theenumi)}$\;\;$}

          %%%%% ~~~~~~~~~~~~~~~~~~~~ %%%%%

\subsection{Summary}
\setcounter{theorem}{0}
\setcounter{equation}{0}

\begin{itemize}
\item
	Let \,$X$\, be topological space, and \,$\mathscr{F}$,\, $\mathscr{G}$\, sheaves of abelian groups over \,$X$.\,
\item
	Definition of a morphism \,$\varphi : \mathscr{F} \longrightarrow \mathscr{G}$\,
	from the sheaf \,$\mathscr{F}$\, to the sheaf \,$\mathscr{G}$.\,
\item
	Definition of the kernel presheaf of \,$\varphi : \mathscr{F} \longrightarrow \mathscr{G}$.\,
	
	The kernel presheaf of
	\,$\varphi : \mathscr{F} \longrightarrow \mathscr{G}$\,
	is in fact always a sheaf and a subsheaf of
	\,$\mathscr{F}$.\,
\item
	Definition of the image presheaf of \,$\varphi : \mathscr{F} \longrightarrow \mathscr{G}$.\,
	
	The image presheaf of
	\,$\varphi : \mathscr{F} \longrightarrow \mathscr{G}$\,
	is a sub-presheaf of
	\,$\mathscr{G}$\,
	that is in general NOT a sheaf.

	The image (sheaf) of
	\,$\varphi : \mathscr{F} \longrightarrow \mathscr{G}$\,
	is the sheafification of the image presheaf of
	\,$\varphi$.\,

	This image sheaf is a subsheaf of
	\,$\mathscr{G}$.\,
\item
	If the sheaf \,$\mathscr{F}$\, is a subsheaf of the sheaf \,$\mathscr{G}$,\,
	then one can define the quotient sub-presheaf via
	\begin{equation*}
	\left(\,\mathscr{G} \overset{{\color{white}.}}{\slash} \mathscr{F}\,\right)(U)
	\;\; := \;\;
		\left. \overset{{\color{white}.}}{\mathscr{G}(U)} \,\right\slash\mathscr{F}(U)
	\end{equation*}
	The quotient sub-presheaf need not be a sheaf.
	
	The quotient sheaf is, by definition, the sheafification of the quotient sub-presheaf.
\item
`	Suppose
	\,$\varphi : \mathscr{F} \longrightarrow \mathscr{G}$\,
	is a morphism of sheaves.
	
	Recall from above that the image sheaf of \,$\varphi$\, is a subsheaf of $\mathscr{G}$.\,
	
	We can thus define the cokernel presheaf of \,$\varphi$\, by
	\begin{equation*}
	\coker(\,\varphi\,)(U)
	\;\; := \;\;
		% \left. \overset{{\color{white}.}}{\mathscr{G}(U)}} \;\right\slash \mathscr{F}(U)
		\left. \overset{{\color{white}.}}{\mathscr{G}(U)} \;\right\slash \varphi_{U}\!\left(\,\overset{{\color{white}.}}{\mathscr{F}(U)}\,\right)
	\end{equation*}
	The cokernel sheaf of the morphism \,$\varphi$\, is the sheafification of its cokernel presheaf.
\item
	Suppose
	\,$\mathscr{F} \overset{\varphi}{\longrightarrow} \mathscr{G} \overset{\psi}{\longrightarrow} \mathscr{H}$\,
	is the composition of two morphisms
	\,$\varphi$\,, \,$\psi$\,
	of sheaves of abelian groups.
	Recall that the image sheaf \,$\image(\varphi)$\, of \,$\varphi$\, and the kernel sheaf \,$\ker(\varphi)$\,
	are both subsheaves of \,$\mathscr{G}$.\,
	
	The sequence
	\,$\mathscr{F} \overset{\varphi}{\longrightarrow} \mathscr{G} \overset{\psi}{\longrightarrow} \mathscr{H}$\,
	is said to be \textbf{exact at \,$\mathscr{G}$\,} if
	$\image(\varphi) \,=\, \ker(\psi)$.

	The sequence
	\,$
	0 \longrightarrow \mathscr{F}
	\overset{\varphi}{\longrightarrow} \mathscr{G}
	\overset{\psi}{\longrightarrow} \mathscr{H}
	\longrightarrow
	0
	$\,
	is called \textbf{a short exact sequence} of morphisms of sheaves over \,$X$,\,
	if it is exact at each of
	\,$\mathscr{F}$,\,
	\,$\mathscr{G}$,\, and
	\,$\mathscr{H}$\,
\end{itemize}

          %%%%% ~~~~~~~~~~~~~~~~~~~~ %%%%%
