
          %%%%% ~~~~~~~~~~~~~~~~~~~~ %%%%%

\section{Sheafification}

%\cite{vanDerVaart1996}
%\cite{Kosorok2008}

%\renewcommand{\theenumi}{\alph{enumi}}
%\renewcommand{\labelenumi}{\textnormal{(\theenumi)}$\;\;$}
\renewcommand{\theenumi}{\roman{enumi}}
\renewcommand{\labelenumi}{\textnormal{(\theenumi)}$\;\;$}

          %%%%% ~~~~~~~~~~~~~~~~~~~~ %%%%%

\subsection{The sheaf of discontinuous sections of a presheaf}
\setcounter{theorem}{0}
\setcounter{equation}{0}

\begin{definition}
\mbox{}\vskip 0.1cm
\noindent
Suppose $\mathscr{P}$ is a presheaf over a topological space $X$.
\vskip 0.1cm
\noindent
The \textbf{sheaf \,$\mathscr{D}(\mathscr{P})$\, of discontinuous sections} of $\mathscr{P}$
is defined as follows:
\begin{itemize}
\item
	For each open subset $U \subset X$,
	\begin{equation*}
	\mathscr{D}(\mathscr{P})(\,U)
	\;\; := \;\;
		\underset{x \in U}{\prod} \, \mathscr{P}_{x}
	\end{equation*}
	In particular, a section $\sigma \in \mathscr{D}(\mathscr{P})(\,U)$ has the form:
	\begin{equation*}
	\sigma
	\;\, = \;
		\left(\,\overset{{\color{white}.}}{\sigma_{x}}\,\right)_{x \in U}
	\quad
	\textnormal{where \;$\sigma_{x} \in \mathscr{P}_{x}$,\; for each $x \in U$}
	\end{equation*}
\item
	For each open subsets $V \subset U \subset X$,
	the restriction map of $\mathscr{D}(\mathscr{P})$ is defined by:
	\begin{equation*}
	\res^{\,U}_{\,V}\!\left(\,\overset{{\color{white}.}}{\sigma}\,\right)
	\;\; = \;\;
		\res^{\,U}_{\,V}\!\left(\,\left(\,\overset{{\color{white}.}}{\sigma_{x}}\,\right)_{x \in U}\,\right)
	\;\; := \;\;
		\left(\,\overset{{\color{white}.}}{\sigma_{x}}\,\right)_{x \in V}
	\end{equation*}
\end{itemize}
\end{definition}

          %%%%% ~~~~~~~~~~~~~~~~~~~~ %%%%%

\vskip 0.5cm
\begin{proposition}
\mbox{}\vskip 0.1cm
\noindent
The sheaf of discontinuous sections of a presheaf over a topological space is indeed a sheaf.
\end{proposition}
\proof
\vskip 0.3cm
\noindent
Suppose $\mathscr{P}$ is a presheaf over a topological space $X$, and
$\mathscr{D}(\mathscr{P})$ is the ``sheaf'' of discontinuous sections of $\mathscr{P}$.

\vskip 0.5cm
\noindent
\textbf{Claim 0:}\quad $\mathscr{D}(\mathscr{P})$ is a presheaf.
\vskip 0.2cm
\noindent
Proof of Claim 0:\;\; Observe that:
\begin{itemize}
\item
	For any open subset \,$U \subset X$,\, and any section
	\,$\sigma = \left(\,\overset{{\color{white}.}}{\sigma}_{x}\,\right)_{x \in U} \in \mathscr{D}(\mathscr{P})(U)$,\,
	we have:
	\begin{equation*}
	\res^{\,U}_{\,{\color{red}U}}\!\left(\,\overset{{\color{white}.}}{\sigma}\,\right)
	\;\; = \;\;
		\res^{\,U}_{\,{\color{red}U}}\!\left(\,\left(\,\overset{{\color{white}.}}{\sigma_{x}}\,\right)_{x \in U}\,\right)
	\;\; = \;\;
		\left(\,\overset{{\color{white}.}}{\sigma_{x}}\,\right)_{x \in {\color{red}U}}
	\end{equation*}
	In other words, \,$\res^{\,U}_{\,U} \,=\, \id_{U}$.
\item
	For any open subsets \,$W \subset V \subset U \subset X$,\, and any section
	\,$\sigma = \left(\,\overset{{\color{white}.}}{\sigma}_{x}\,\right)_{x \in U} \in \mathscr{D}(\mathscr{P})(U)$,\,
	we have:
	\begin{equation*}
	\res^{\,V}_{\,W}\!\left(\,
		\overset{{\color{white}\textnormal{\large$.$}}}{\res^{\,U}_{\,V}}
		\!\left(\,\overset{{\color{white}.}}{\sigma}\,\right)
		\,\right)
	\;\; = \;\;
		\res^{\,V}_{\,W}\!\left(\,\left(\,\overset{{\color{white}.}}{\sigma_{x}}\,\right)_{x \in V}\,\right)
	\;\; = \;\;
		\left(\,\overset{{\color{white}.}}{\sigma_{x}}\,\right)_{x \in W}
	\;\; = \;\;
		\res^{\,U}_{\,W}\!\left(\,\overset{{\color{white}.}}{\sigma}\,\right)
	\end{equation*}
	In other words, \,$\res^{\,V}_{\,W} \,\circ\, \res^{\,U}_{\,V} \; = \; \res^{\,U}_{\,W}$.
\end{itemize} 
Claim 0 follows, by definition of presheaves, immediately from the above two observations.

\vskip 0.5cm
\noindent
Next, let $U = \,\underset{\alpha}{\bigcup}\;U_{\alpha}$ be an open cover of an open subset $U \subset X$.

\vskip 0.5cm
\noindent
\textbf{Claim 1:}\quad
Let \,$\sigma$, $\tau$ $\in$ $\mathscr{D}(\mathscr{P})(U)$.
\,\;Then,
\,$\left.\overset{{\color{white}.}}{\sigma}\,\right\vert_{U_{\alpha}}$
$=$
$\left.\overset{{\color{white}.}}{\tau}\,\right\vert_{U_{\alpha}}$\,,\,
for each $\alpha$
\;\;$\Longrightarrow$\;\;
$\sigma \,=\, \tau$.
\vskip 0.2cm
\noindent
Proof of Claim 1:
\begin{eqnarray*}
	\left.\overset{{\color{white}.}}{\sigma}\,\right\vert_{U_{\alpha}}
	\, = \,
	\left.\overset{{\color{white}.}}{\tau}\,\right\vert_{U_{\alpha}},
	\;\;\textnormal{for each $\alpha$}
& \Longleftrightarrow &
	\left(\,\overset{{\color{white}.}}{\sigma_{x}}\,\right)_{x \in U_{\alpha}}
	\;\; =: \;\;
	\left.\overset{{\color{white}.}}{\sigma}\,\right\vert_{U_{\alpha}}
	\;\; = \;\;
	\left.\overset{{\color{white}.}}{\tau}\,\right\vert_{U_{\alpha}}
	\;\; := \;\;
	\left(\,\overset{{\color{white}.}}{\tau_{x}}\,\right)_{x \in U_{\alpha}},
	\;\;\textnormal{for each $\alpha$}
\\
& \overset{{\color{white}\textnormal{\LARGE$1$}}}{\Longleftrightarrow} &
	\sigma_{x} \; = \; \tau_{x}\,,
	\;\;\textnormal{for each $x \in U_{\alpha}$,\, for each $\alpha$}
\\
& \overset{{\color{white}\textnormal{\LARGE$1$}}}{\Longleftrightarrow} &
	\left(\,\overset{{\color{white}.}}{\sigma_{x}}\,\right)_{x \in U}
	\;\; = \;\;
	\left(\,\overset{{\color{white}.}}{\tau_{x}}\,\right)_{x \in U},
	\;\;\textnormal{for each $x \in U = \,\underset{\alpha}{\bigcup}\;U_{\alpha}$}
\\
& \overset{{\color{white}\textnormal{\tiny$1$}}}{\Longleftrightarrow} &
	\sigma \; = \; \tau
\end{eqnarray*}
This proves Claim 1.

\vskip 0.5cm
\noindent
\textbf{Claim 2:}\quad
Let
\,$\sigma_{\alpha}$ $\in$ $\mathscr{D}(\mathscr{P})(U_{\alpha})$\,
be such that
\,$\left.\overset{{\color{white}.}}{\sigma_{\alpha}}\,\right\vert_{\,U_{\alpha} \cap\,U_{\beta}}$
$=$
\,$\left.\overset{{\color{white}.}}{\sigma_{\beta}}\,\right\vert_{\,U_{\alpha} \cap\,U_{\beta}}$.
\,\;Then,
\begin{equation*}
\textnormal{there exists \;$\sigma \in \mathscr{D}(\mathscr{P})(U)$\; such that}\;\;
\sigma_{\alpha}
\;=\;
\left.\overset{{\color{white}.}}{\sigma}\,\right\vert_{\,U_{\alpha}},
\;\;\textnormal{for each $\alpha$}
\end{equation*}
\vskip 0.2cm
\noindent
Proof of Claim 2:\quad
First, observation that
\begin{eqnarray*}
&&
	\left.\overset{{\color{white}.}}{\sigma_{\alpha}}\,\right\vert_{\,U_{\alpha}\cap\,U_{\beta}}
	\, = \,
	\left.\overset{{\color{white}.}}{\sigma_{\beta}}\,\right\vert_{\,U_{\alpha}\cap\,U_{\beta}},
	\;\;\textnormal{for each $\alpha$, $\beta$}
\\
& \overset{{\color{white}\textnormal{\LARGE$1$}}}{\Longleftrightarrow} &
	\left(\,\overset{{\color{white}.}}{(\sigma_{\alpha})_{x}}\,\right)_{x \,\in\, U_{\alpha}\cap\,U_{\beta}}
	\;\; =: \;\;
	\left.\overset{{\color{white}.}}{\sigma_{\alpha}}\,\right\vert_{\,U_{\alpha}\cap\,U_{\beta}}
	\;\; = \;\;
	\left.\overset{{\color{white}.}}{\sigma_{\beta}}\,\right\vert_{\,U_{\alpha}\cap\,U_{\beta}}
	\;\; := \;\;
	\left(\,\overset{{\color{white}.}}{(\sigma_{\beta})_{x}}\,\right)_{x \,\in\, U_{\alpha}\cap\,U_{\beta}},
	\;\;\textnormal{for each $\alpha$, $\beta$}
\\
& \overset{{\color{white}\textnormal{\LARGE$1$}}}{\Longleftrightarrow} &
	(\,\sigma_{\alpha}\,)_{x} \; = \; (\,\sigma_{\beta}\,)_{x}\,,
	\;\;\textnormal{for each $x \in U_{\alpha}\cap\,U_{\alpha}$,\, for each $\alpha$, $\beta$}
\end{eqnarray*}
The above observation allows us to unambiguously define, for each $x \in U$,
\begin{equation*}
\sigma_{x}
\;\; := \;\;
	(\,\sigma_{\alpha}\,)_{x}\,,
\quad
\textnormal{for any \,$\alpha$\, such that \,$U_{\alpha} \ni x$}
\end{equation*}
Now, define
\,$\sigma \,\in\, \mathscr{D}(\mathscr{P})(U)$\,
by
\,$\sigma \,:=\, \left(\,\overset{{\color{white}.}}{\sigma_{x}}\,\right)_{x \in U}$.\,
Then, for each $\alpha$, we have:
\begin{equation*}
\left.\overset{{\color{white}\textnormal{\Large$.$}}}{\sigma}\,\right\vert_{U_{\alpha}}
\;\; = \;\;
\left(\,\overset{{\color{white}.}}{\sigma_{x}}\,\right)_{x \in U_{\alpha}}
\;\; = \;\;
\left(\,\overset{{\color{white}.}}{(\sigma_{\alpha})_{x}}\,\right)_{x \,\in\, U_{\alpha}}
\;\; = \;\;
\sigma_{\alpha}
\end{equation*}
This proves Claim 2.

\vskip 0.5cm
\noindent
Claim 1 and Claim 2 together imply that
\,$\mathscr{D}(\mathscr{P})$\,
is a sheaf, which completes the proof of the Proposition.
\qed

          %%%%% ~~~~~~~~~~~~~~~~~~~~ %%%%%

\vskip 0.5cm
\begin{lemma}
\mbox{}\vskip 0.1cm
\noindent
Suppose
\,$\mathscr{F}$, $\mathscr{G}$, $\mathscr{H}$\,
are presheaves on a topological space $X$, and
\,$\mathscr{D}(\mathscr{F})$, $\mathscr{D}(\mathscr{G})$, $\mathscr{D}(\mathscr{H})$\,
are the sheaves of discontinuous sections of
\,$\mathscr{F}$, $\mathscr{G}$, $\mathscr{H}$,\,
respectively.
\begin{enumerate}
\item
	For each open subset $U \subset X$ and each $\sigma \in \mathscr{F}(U)$, define
	\begin{equation*}
	\iota_{U}(\,\sigma\,)
	\;\; := \;\;
		\left(\,\overset{{\color{white}.}}{\sigma_{x}}\,\right)_{x \in U}
	\end{equation*}
	Then,
	\,$\mathscr{F}\,\overset{\iota}{\longrightarrow}\,\mathscr{D}(\mathscr{F})$\,
	is a morphism of presheaves.
\item
	For each morphism of presheaves
	\,$\mathscr{F}\,\overset{\varphi}{\longrightarrow}\,\mathscr{G}$,\,
	there exists a unique morphism of presheaves
	\begin{tikzcd}
	\mathscr{D}(\mathscr{F}) \arrow[r, "\mathscr{D}(\varphi)"] & \mathscr{D}(\mathscr{G})
	\end{tikzcd}
	such that the following diagram commutes:
	\begin{equation*}
	\begin{tikzcd}
	                    \mathscr{F}  \arrow[r, "\varphi"] \arrow[d, "\iota"'] & \mathscr{G} \arrow[d, "\iota"] \\
	\mathscr{D}(\mathscr{F}) \arrow[r, "\mathscr{D}(\varphi)"'] & \mathscr{D}(\mathscr{G})
	\end{tikzcd}
	\end{equation*}
\item
	$\mathscr{D}(\,\id_{\mathscr{F}}\,)$ \,$=$\, $\id_{\mathscr{D}(\mathscr{F})}$ 
\item
	For each pair of morphisms of presheaves
	\,$\mathscr{F}\,\overset{\varphi}{\longrightarrow}\,\mathscr{G}$\,
	and
	\,$\mathscr{G}\,\overset{\psi}{\longrightarrow}\,\mathscr{H}$,\,
	we have:
	\begin{equation*}
	\mathscr{D}(\,\psi \circ \varphi\,) \;\; = \;\; \mathscr{D}(\,\psi\,) \circ \mathscr{D}(\,\varphi\,)
	\end{equation*}
\end{enumerate}
\end{lemma}
\proof
\begin{enumerate}
\item
	Recall that 
	\,$\mathscr{F}\,\overset{\iota}{\longrightarrow}\,\mathscr{D}(\mathscr{F})$\,
	being a morphism of presheaves simply means that,
	for each open sets $V \subset U \subset X$, the following diagram commutes:
	\begin{equation*}
	\begin{tikzcd}
	\mathscr{F}(U)  \arrow[r, "\iota_{U}"] \arrow[d, "\res^{U}_{V}"'] & \mathscr{D}(\mathscr{F})(U) \arrow[d, "\res^{U}_{V}"] \\
	\mathscr{F}(V) \arrow[r, "\iota_{V}"'] & \mathscr{D}(\mathscr{F})(V)
	\end{tikzcd}
	\end{equation*}
	Now, for each open subsets \,$V \subset U \subset X$,\, and each \,$\sigma \in \mathscr{F}(U)$,
	we have
	\begin{equation*}
	\res^{\,U}_{\,V}\!\left(\,\overset{{\color{white}.}}{\iota_{U}(\,\sigma\,)}\,\right)
	\;\; = \;\;
		\res^{\,U}_{\,V}\!\left(\,
			\left(\,\overset{{\color{white}.}}{\sigma_{x}}\,\right)_{x \in U}
			\,\right)
	\;\; = \;\;
		\left(\,\overset{{\color{white}.}}{\sigma_{x}}\,\right)_{x \in V}
	\;\; = \;\;
		\left(\,\overset{{\color{white}.}}{\left(\res^{\,U}_{\,V}(\sigma)\right)_{x}}\,\right)_{x \in V}
%	\\
%	& \overset{{\color{white}\textnormal{\large$1$}}}{=} &
	\;\; = \;\;
		\iota_{V}\!\left(\,\overset{{\color{white}.}}{\res^{\,U}_{\,V}(\sigma)}\,\right),
	\end{equation*}
	which proves the commutativity of the above diagram, which in turn proves that
	\,$\mathscr{F}\,\overset{\iota}{\longrightarrow}\,\mathscr{D}(\mathscr{F})$\,
	is indeed a morphism of presheaves.
\item
	For each
	\,$\tau$
	\,$=$\, $\left(\,\overset{{\color{white}.}}{\tau_{x}}\,\right)_{x \in U}$
	\,$\in$\, $\mathscr{D}(\mathscr{F})(U)$,\,
	we define
	\begin{equation*}
	\mathscr{D}(\varphi)\!\left(\,\overset{{\color{white}.}}{\tau}\,\right)
	\;\; = \;\;
		\mathscr{D}(\varphi)\!\left[\;\left(\,\overset{{\color{white}.}}{\tau_{x}}\,\right)_{x \in U}\,\right]
	\;\; := \;\;
		\left(\,\overset{{\color{white}.}}{\varphi_{x}(\tau_{x})}\,\right)_{x \in U}
	\end{equation*}
	Now, observe that, for each \,$\sigma \in \mathscr{F}(U)$,\, we then have
	\begin{equation*}
	\iota\!\left(\,\overset{{\color{white}.}}{\varphi(\,\sigma\,)}\,\right)
	\;\; = \;\;
		\left(\,\overset{{\color{white}.}}{\varphi(\,\sigma\,)_{x}}\,\right)_{x \in U}
	\;\; = \;\;
		\left(\,\overset{{\color{white}.}}{\varphi_{x}(\,\sigma_{x}\,)}\,\right)_{x \in U}
	\;\; = \;\;
		\mathscr{D}(\varphi)\!\left[\;\left(\,\overset{{\color{white}.}}{\sigma_{x}}\,\right)_{x \in U}\,\right]
	\;\; = \;\;
		\mathscr{D}(\varphi)\!\left(\,\overset{{\color{white}.}}{\iota(\,\sigma\,)}\,\right)
	\end{equation*}
	Hence, \;$\iota \circ \varphi \,=\, \mathscr{D}(\varphi) \circ \iota$\,,\; as required.
\item
	For each
	\,$\left(\,\overset{{\color{white}.}}{\sigma_{x}}\,\right)_{x \in U}$
	\,$\in$\, $\mathscr{D}(\mathscr{F})(U)$,\,
	observe that
	\begin{equation*}
	\mathscr{D}(\,\id_{\mathscr{F}}\,)\!\left[\;\left(\,\overset{{\color{white}.}}{\sigma_{x}}\,\right)_{x \in U}\,\right]
	\;\; = \;\;
		\left(\,\overset{{\color{white}.}}{(\id_{\mathscr{F}})_{x}(\,\sigma_{x}\,)}\,\right)_{x \in U}
	\;\; = \;\;
		\left(\;\overset{{\color{white}-}}{\sigma_{x}}\;\right)_{x \in U}
	\;\; = \;\;
		\id_{\mathscr{D}(\mathscr{F})}\!\left[\;\left(\,\overset{{\color{white}.}}{\sigma_{x}}\,\right)_{x \in U}\,\right],
	\end{equation*}
	which proves: \;$\mathscr{D}(\,\id_{\mathscr{F}}\,) \,=\, \id_{\mathscr{D}(\mathscr{F})}$.
\item
	For each
	\,$\left(\,\overset{{\color{white}.}}{\sigma_{x}}\,\right)_{x \in U}$
	\,$\in$\, $\mathscr{D}(\mathscr{F})(U)$,\,
	observe that
	\begin{eqnarray*}
	\mathscr{D}(\,\psi\circ\varphi\,)\!\left[\;\left(\,\overset{{\color{white}.}}{\sigma_{x}}\,\right)_{x \in U}\,\right]
	& = &
		\left(\,\overset{{\color{white}.}}{(\psi\circ\varphi)_{x}(\,\sigma_{x}\,)}\,\right)_{x \in U}
	\;\; = \;\;
		\left(\,\overset{{\color{white}.}}{\psi_{x}(\,\varphi_{x}(\,\sigma_{x}\,))}\,\right)_{x \in U}
	\;\; = \;\;
		\mathscr{D}(\,\psi\,)\!\left[\,\left(\,\overset{{\color{white}.}}{\varphi_{x}(\,\sigma_{x}\,)}\,\right)_{x \in U}\,\right]
	\\
	& \overset{{\color{white}\textnormal{\Large$1$}}}{=} &
		\mathscr{D}(\,\psi\,)\!\left[\;
			\mathscr{D}(\,\varphi\,)\!\left[\;\left(\,\overset{{\color{white}.}}{\sigma_{x}}\,\right)_{x \in U}\,\right]
			\,\right]
	\;\; = \;\;
		\mathscr{D}(\,\psi\,)\circ\mathscr{D}(\,\varphi\,)
			\!\left[\;\left(\,\overset{{\color{white}.}}{\sigma_{x}}\,\right)_{x \in U}\,\right],
	\end{eqnarray*}
	which proves: \;$\mathscr{D}(\,\psi\circ\varphi\,) \,=\, \mathscr{D}(\,\psi\,)\circ\mathscr{D}(\,\varphi\,)$.
\end{enumerate}
\qed

          %%%%% ~~~~~~~~~~~~~~~~~~~~ %%%%%

\subsection{The sheafification of a presheaf}
\setcounter{theorem}{0}
\setcounter{equation}{0}

          %%%%% ~~~~~~~~~~~~~~~~~~~~ %%%%%
