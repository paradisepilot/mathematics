
          %%%%% ~~~~~~~~~~~~~~~~~~~~ %%%%%

\section{Sheafification}

%\cite{vanDerVaart1996}
%\cite{Kosorok2008}

%\renewcommand{\theenumi}{\alph{enumi}}
%\renewcommand{\labelenumi}{\textnormal{(\theenumi)}$\;\;$}
\renewcommand{\theenumi}{\roman{enumi}}
\renewcommand{\labelenumi}{\textnormal{(\theenumi)}$\;\;$}

          %%%%% ~~~~~~~~~~~~~~~~~~~~ %%%%%

\subsection{The sheaf of discontinuous sections of a presheaf}
\setcounter{theorem}{0}
\setcounter{equation}{0}

\begin{definition}
\mbox{}\vskip 0.1cm
\noindent
Suppose $\mathscr{P}$ is a presheaf on a topological space $X$.
\vskip 0.1cm
\noindent
The \textbf{presheaf $\mathscr{D}(\mathscr{P})$ of discontinuous sections} of $\mathscr{P}$
is defined as follows:
\begin{itemize}
\item
	For each open subset $U \subset X$,
	\begin{equation*}
	\mathscr{D}(\mathscr{P})(\,U)
	\;\; := \;\;
		\underset{x \in U}{\prod} \, \mathscr{P}_{x}
	\end{equation*}
	In particular, a section $\sigma \in \mathscr{D}(\mathscr{P})(\,U)$ has the form:
	\begin{equation*}
	\sigma
	\;\, = \;
		\left(\,\overset{{\color{white}.}}{\sigma_{x}}\,\right)_{x \in U}
	\quad
	\textnormal{where \;$\sigma_{x} \in \mathscr{P}_{x}$,\; for each $x \in U$}
	\end{equation*}
\item
	For each open subsets $V \subset U \subset X$,
	the restriction map of $\mathscr{D}(\mathscr{P})$ is defined by:
	\begin{equation*}
	\textnormal{res}^{U}_{V}\!\left(\,\overset{{\color{white}.}}{\sigma}\,\right)
	\;\; = \;\;
		\textnormal{res}^{U}_{V}\!\left(\,\left(\,\overset{{\color{white}.}}{\sigma_{x}}\,\right)_{x \in U}\,\right)
	\;\; := \;\;
		\left(\,\overset{{\color{white}.}}{\sigma_{x}}\,\right)_{x \in V}
	\end{equation*}
\end{itemize}
\end{definition}

          %%%%% ~~~~~~~~~~~~~~~~~~~~ %%%%%

\vskip 0.5cm
\begin{proposition}
\mbox{}\vskip 0.1cm
\noindent
The presheaf of discontinuous sections of a presheaf over a topological space is in fact a sheaf.
\end{proposition}
\proof
\vskip 0.3cm
\noindent
Suppose $\mathscr{P}$ is a presheaf on a topological space $X$, and
$\mathscr{D}(\mathscr{P})$ is the presheaf of discontinuous sections of $\mathscr{P}$.
\vskip 0.3cm
\noindent
Let $U = \,\underset{\alpha}{\bigcup}\;U_{\alpha}$ be an open cover of an open subset $U \subset X$.

\vskip 0.5cm
\noindent
\textbf{Claim 1:}\quad
Let \,$\sigma$, $\tau$ $\in$ $\mathscr{D}(\mathscr{P})(U)$.
\,\;Then,
\,$\left.\overset{{\color{white}.}}{\sigma}\,\right\vert_{U_{\alpha}}$
$=$
$\left.\overset{{\color{white}.}}{\tau}\,\right\vert_{U_{\alpha}}$\,,\,
for each $\alpha$
\;\;$\Longrightarrow$\;\;
$\sigma \,=\, \tau$.
\vskip 0.2cm
\noindent
Proof of Claim 1:
\begin{eqnarray*}
	\left.\overset{{\color{white}.}}{\sigma}\,\right\vert_{U_{\alpha}}
	\, = \,
	\left.\overset{{\color{white}.}}{\tau}\,\right\vert_{U_{\alpha}},
	\;\;\textnormal{for each $\alpha$}
& \Longleftrightarrow &
	\left(\,\overset{{\color{white}.}}{\sigma_{x}}\,\right)_{x \in U_{\alpha}}
	\;\; =: \;\;
	\left.\overset{{\color{white}.}}{\sigma}\,\right\vert_{U_{\alpha}}
	\;\; = \;\;
	\left.\overset{{\color{white}.}}{\tau}\,\right\vert_{U_{\alpha}}
	\;\; := \;\;
	\left(\,\overset{{\color{white}.}}{\tau_{x}}\,\right)_{x \in U_{\alpha}},
	\;\;\textnormal{for each $\alpha$}
\\
& \overset{{\color{white}\textnormal{\LARGE$1$}}}{\Longleftrightarrow} &
	\sigma_{x} \; = \; \tau_{x}\,,
	\;\;\textnormal{for each $x \in U_{\alpha}$,\, for each $\alpha$}
\\
& \overset{{\color{white}\textnormal{\LARGE$1$}}}{\Longleftrightarrow} &
	\left(\,\overset{{\color{white}.}}{\sigma_{x}}\,\right)_{x \in U}
	\;\; = \;\;
	\left(\,\overset{{\color{white}.}}{\tau_{x}}\,\right)_{x \in U},
	\;\;\textnormal{for each $x \in U = \,\underset{\alpha}{\bigcup}\;U_{\alpha}$}
\\
& \overset{{\color{white}\textnormal{\tiny$1$}}}{\Longleftrightarrow} &
	\sigma \; = \; \tau
\end{eqnarray*}
This proves Claim 1.

\vskip 0.5cm
\noindent
\textbf{Claim 2:}\quad
Let
\,$\sigma_{\alpha}$ $\in$ $\mathscr{D}(\mathscr{P})(U_{\alpha})$\,
be such that
\,$\left.\overset{{\color{white}.}}{\sigma_{\alpha}}\,\right\vert_{\,U_{\alpha} \cap\,U_{\beta}}$
$=$
\,$\left.\overset{{\color{white}.}}{\sigma_{\beta}}\,\right\vert_{\,U_{\alpha} \cap\,U_{\beta}}$.
\,\;Then,
\begin{equation*}
\textnormal{there exists \;$\sigma \in \mathscr{D}(\mathscr{P})(U)$\; such that}\;\;
\sigma_{\alpha}
\;=\;
\left.\overset{{\color{white}.}}{\sigma}\,\right\vert_{\,U_{\alpha}},
\;\;\textnormal{for each $\alpha$}
\end{equation*}
\vskip 0.2cm
\noindent
Proof of Claim 2:\quad
First, observation that
\begin{eqnarray*}
&&
	\left.\overset{{\color{white}.}}{\sigma_{\alpha}}\,\right\vert_{\,U_{\alpha}\cap\,U_{\beta}}
	\, = \,
	\left.\overset{{\color{white}.}}{\sigma_{\beta}}\,\right\vert_{\,U_{\alpha}\cap\,U_{\beta}},
	\;\;\textnormal{for each $\alpha$, $\beta$}
\\
& \overset{{\color{white}\textnormal{\LARGE$1$}}}{\Longleftrightarrow} &
	\left(\,\overset{{\color{white}.}}{(\sigma_{\alpha})_{x}}\,\right)_{x \,\in\, U_{\alpha}\cap\,U_{\beta}}
	\;\; =: \;\;
	\left.\overset{{\color{white}.}}{\sigma_{\alpha}}\,\right\vert_{\,U_{\alpha}\cap\,U_{\beta}}
	\;\; = \;\;
	\left.\overset{{\color{white}.}}{\sigma_{\beta}}\,\right\vert_{\,U_{\alpha}\cap\,U_{\beta}}
	\;\; := \;\;
	\left(\,\overset{{\color{white}.}}{(\sigma_{\beta})_{x}}\,\right)_{x \,\in\, U_{\alpha}\cap\,U_{\beta}},
	\;\;\textnormal{for each $\alpha$, $\beta$}
\\
& \overset{{\color{white}\textnormal{\LARGE$1$}}}{\Longleftrightarrow} &
	(\,\sigma_{\alpha}\,)_{x} \; = \; (\,\sigma_{\beta}\,)_{x}\,,
	\;\;\textnormal{for each $x \in U_{\alpha}\cap\,U_{\alpha}$,\, for each $\alpha$, $\beta$}
\end{eqnarray*}
The above observation allows us to unambiguously define, for each $x \in U$,
\begin{equation*}
\sigma_{x}
\;\; := \;\;
	(\,\sigma_{\alpha}\,)_{x}\,,
\quad
\textnormal{for any \,$\alpha$\, such that \,$U_{\alpha} \ni x$}
\end{equation*}
Now, define
\,$\sigma \,\in\, \mathscr{D}(\mathscr{P})(U)$\,
by
\,$\sigma \,:=\, \left(\,\overset{{\color{white}.}}{\sigma_{x}}\,\right)_{x \in U}$.\,
Then, for each $\alpha$, we have:
\begin{equation*}
\left.\overset{{\color{white}\textnormal{\Large$.$}}}{\sigma}\,\right\vert_{U_{\alpha}}
\;\; = \;\;
\left(\,\overset{{\color{white}.}}{\sigma_{x}}\,\right)_{x \in U_{\alpha}}
\;\; = \;\;
\left(\,\overset{{\color{white}.}}{(\sigma_{\alpha})_{x}}\,\right)_{x \,\in\, U_{\alpha}}
\;\; = \;\;
\sigma_{\alpha}
\end{equation*}
This proves Claim 2.

\vskip 0.5cm
\noindent
Claim 1 and Claim 2 together imply that
\,$\mathscr{D}(\mathscr{P})$\,
is a sheaf, which completes the proof of the Proposition.
\qed

          %%%%% ~~~~~~~~~~~~~~~~~~~~ %%%%%

\vskip 0.5cm
\begin{lemma}
\mbox{}\vskip 0.1cm
\noindent
Suppose
\,$\mathscr{F}$, $\mathscr{G}$, $\mathscr{H}$\,
are presheaves on a topological space $X$, and
\,$\mathscr{D}(\mathscr{F})$, $\mathscr{D}(\mathscr{G})$, $\mathscr{D}(\mathscr{H})$\,
are the sheaves of discontinuous sections of
\,$\mathscr{F}$, $\mathscr{G}$, $\mathscr{H}$,\,
respectively.
\begin{enumerate}
\item
	For each open subset $U \subset X$ and each $\sigma \in \mathscr{F}(U)$, define
	\begin{equation*}
	\iota_{U}(\,\sigma\,)
	\;\; := \;\;
		\left(\,\overset{{\color{white}.}}{\sigma_{x}}\,\right)_{x \in U}
	\end{equation*}
	Then,
	\,$\mathscr{F}\,\overset{\iota}{\longrightarrow}\,\mathscr{D}(\mathscr{F})$\,
	is a morphism of presheaves.
\item
	For each morphism of presheaves
	\,$\mathscr{F}\,\overset{\varphi}{\longrightarrow}\,\mathscr{G}$,\,
	there exists a unique morphism of presheaves
	\begin{tikzcd}
	\mathscr{D}(\mathscr{F}) \arrow[r, "\mathscr{D}(\varphi)"] & \mathscr{D}(\mathscr{G})
	\end{tikzcd}
	such that the following diagram commutes:
	\begin{equation*}
	\begin{tikzcd}
	                    \mathscr{F}  \arrow[r, "\varphi"] \arrow[d, "\iota"'] & \mathscr{G} \arrow[d, "\iota"] \\
	\mathscr{D}(\mathscr{F}) \arrow[r, "\mathscr{D}(\varphi)"'] & \mathscr{D}(\mathscr{G})
	\end{tikzcd}
	\end{equation*}
\item
	$\mathscr{D}(\,\id_{\mathscr{F}}\,)$ \,$=$\, $\id_{\mathscr{D}(\mathscr{F})}$ 
\item
	For each pair of morphisms of presheaves
	\,$\mathscr{F}\,\overset{\varphi}{\longrightarrow}\,\mathscr{G}$\,
	and
	\,$\mathscr{G}\,\overset{\psi}{\longrightarrow}\,\mathscr{H}$,\,
	we have:
	\begin{equation*}
	\mathscr{D}(\,\psi \circ \varphi\,) \;\; = \;\; \mathscr{D}(\,\psi\,) \circ \mathscr{D}(\,\varphi\,)
	\end{equation*}
\end{enumerate}
\end{lemma}
\proof
\begin{enumerate}
\item
\item
\item
\item
\end{enumerate}
\qed

          %%%%% ~~~~~~~~~~~~~~~~~~~~ %%%%%

\subsection{The sheafification of a presheaf}
\setcounter{theorem}{0}
\setcounter{equation}{0}

\begin{theorem}
\mbox{}\vskip 0.1cm
\noindent
Suppose $\mathscr{P}$ is a presheaf on a topological space $X$.

\end{theorem}

\noindent

\underline{\textbf{Transition functions}}
\vskip 0.1cm
\noindent
Let $M$ be a complex manifold, and $\pi : E \longrightarrow M$ a holomorphic vector bundle on $M$ of rank $r$.
Then, there exist \textbf{local trivializations}:
\begin{equation*}
\Phi_{\alpha} : \pi^{-1}(U_{\alpha}) \overset{\sim}{\longrightarrow} U_{\alpha} \times \C^{r},
\end{equation*}
where $\{\,U_{\alpha}\,\}_{\alpha}$ is a complex coordinate atlas of $M$,
such that the transition map
$\Phi_{\alpha} \circ \Phi_{\beta}^{-1} : (U_{\alpha} \cap U_{\beta}) \times \C^{r} \longrightarrow (U_{\alpha} \cap U_{\beta}) \times \C^{r}$
has the form:
\begin{equation*}
\Phi_{\alpha} \circ \Phi_{\beta}^{-1}\!\left(\,x,v\right)
\;\; = \;\;
	\left(\, x \overset{{\color{white}-}}{,} \tau_{\alpha\beta}(x) \cdot v\,\right)
\end{equation*}
where
\begin{equation*}
\tau_{\alpha\beta} : U_{\alpha} \cap U_{\beta} \longrightarrow \GL(\C^{r})
\end{equation*}
are the associated \textbf{transition functions}.

\vskip0.75cm
\noindent
\underline{\textbf{Dual bundle}}
\vskip 0.1cm
\noindent
The transition functions
\,$\tau^{*}_{\alpha\beta} : U_{\alpha} \cap U_{\beta} \longrightarrow \GL(\C^{r})$\,
of the \textbf{dual bundle} $E^{*} \longrightarrow M$ is given by:
\begin{equation*}
\tau^{*}_{\alpha\beta}(x)
\;\; = \;\;
	\left(\;\tau^{T}_{\alpha\beta}(x)\,\right)^{-1}
\end{equation*}

\vskip0.5cm
\noindent
\underline{\textbf{Tensor product of two vector bundles}}
\vskip 0.1cm
\noindent
Suppose
\,$E \longrightarrow M$\, and \,$F \longrightarrow M$\,
are two holomorphic vector bundles
determined respectively by the transition functions:
\,$\tau_{\alpha\beta} : U_{\alpha} \cap U_{\beta} \longrightarrow \GL(\C^{r})$\,
and
\,$\rho_{\alpha\beta} : U_{\alpha} \cap U_{\beta} \longrightarrow \GL(\C^{d})$.\,
Then, their \textbf{tensor product}
\,$E \otimes F$\,
is the vector bundle with fibre
\,$\C^{r}\otimes\C^{d} \cong \C^{r \times d}$\,
determined by the transition functions
%\,$\tau_{\alpha\beta} \otimes \rho_{\alpha\beta} : U_{\alpha} \cap U_{\beta} \longrightarrow \GL(\C^{r} \otimes \C^{d})$\,
\begin{equation*}
\tau_{\alpha\beta} \otimes \rho_{\alpha\beta} : U_{\alpha} \cap U_{\beta} \longrightarrow \GL(\C^{r} \otimes \C^{d})\,,
\quad
\left(\,\tau_{\alpha\beta} \otimes \rho_{\alpha\beta}\right)(x)
\;\; = \;\;
	\tau_{\alpha\beta}(x) \otimes_{\C} \rho_{\alpha\beta}(x)
\end{equation*}

          %%%%% ~~~~~~~~~~~~~~~~~~~~ %%%%%

\vskip0.5cm
\subsection{Line bundles}

\noindent
\underline{\textbf{The tensor product of two line bundles is itself a line bundle}}
\vskip 0.1cm
\noindent
Suppose
\,$L_{1} \longrightarrow M$\, and \,$L_{2} \longrightarrow M$\,
are two holomorphic line bundles
determined respectively by the transition functions:
\,$g_{\alpha\beta} : U_{\alpha} \cap U_{\beta} \longrightarrow \GL(\C^{1}) \cong \C^{*}$\,
and
\,$h_{\alpha\beta} : U_{\alpha} \cap U_{\beta} \longrightarrow \GL(\C^{1}) \cong \C^{*}$.\,
Then, their \textbf{tensor product}
\,$L_{1} \otimes L_{2}$\,
is the vector bundle with fibre
\,$\C^{1} \otimes \C^{1} \cong \C$\,
determined by the transition functions
\begin{equation*}
g_{\alpha\beta} \otimes h_{\alpha\beta} : U_{\alpha} \cap U_{\beta} \longrightarrow \GL(\C^{1} \otimes \C^{1}) \cong \C^{*}\,,
\quad
\left(\,g_{\alpha\beta} \otimes h_{\alpha\beta}\right)(x)
\;\; = \;\;
	g_{\alpha\beta}(x) \otimes_{\C} h_{\alpha\beta}(x)
\;\; = \;\;
	g_{\alpha\beta}(x) \cdot h_{\alpha\beta}(x),
\end{equation*}
where the last equality follows from the standard identification
of $\C \otimes \C \overset{\sim}{\longrightarrow} \C : 1 \otimes 1 \longmapsto 1$.
\vskip 0.2cm
\noindent
Hence, the collection of line bundles on $M$ forms an abelian group under tensor product.

\vskip 0.5cm
\noindent
\underline{\textbf{The determinant (line) bundle of a vector bundle}}
\vskip 0.1cm
\noindent
Suppose
\,$E \longrightarrow M$\,
is a holomorphic vector bundle
determined respectively by the transition functions:
\,$\tau_{\alpha\beta} : U_{\alpha} \cap U_{\beta} \longrightarrow \GL(\C^{r})$.\,
Then, its \textbf{determinant bundle} is the line bundle on $M$
determined by the transition functions
\begin{equation*}
g_{\alpha\beta} : U_{\alpha} \cap U_{\beta} \longrightarrow \GL(\C^{1}) \cong \C^{*}\,,
\quad
g_{\alpha\beta}(x)
\;\; = \;\;
	\det\!\left(\,\tau_{\alpha\beta}(x)\,\right)
\;\; \in \;\;
	\C^{*}
\end{equation*}

\vskip 0.5cm
\noindent
\underline{\textbf{Sheaf-cohomological description of the Picard group: $\Pic(M) \,=\, \breve{H}^{1}(M,\mathcal{O}^{*})$}}
\vskip 0.1cm
\noindent

          %%%%% ~~~~~~~~~~~~~~~~~~~~ %%%%%

\subsection{Divisors}

Let $M$ be a complex manifold.
A \textbf{divisor} $D$ on $M$ is a \textit{locally finite} formal linear combination
\begin{equation*}
D
\;\; = \;\;
	\underset{i}{\sum}\;a_{i} \cdot V_{i}
\end{equation*}
of irreducible analytic hypersurfaces $V_{i}$ of $M$ with coefficients $a_{i} \in \Z$.
``Locally finite'' here means:
For each point $x \in M$, there is a open neighbourhood $U \subset M$ of $x \in M$ within which
the divisor $D$ is given by a finite linear combination of irreducible analytic hypersurfaces with coefficients in $\Z$.

\begin{proposition}[Sheaf-cohomological description of the group of divisors]
\mbox{}
\vskip 0.1cm
\noindent
Let \,$M$\, be a complex manifold, and \,$\Div(M)$\, the abelian group of divisors of \,$M$.\,
Then,
\begin{equation*}
\Div(M) \; \cong \; H^{0}\!\left(\,\overset{{\color{white}.}}{M} , \mathcal{M}^{*}/\mathcal{O}^{*}\right),
\;\;
\textnormal{as abelian groups},
\end{equation*}
where $\mathcal{M}^{*}$ is the sheaf of non-identically-zero meromorphic functions on $M$,
and $\mathcal{O}^{*}$ its subsheaf of nowhere vanishing holomorphic functions on $M$.
\end{proposition}
\proof
A global section of the quotient sheaf \,$\mathcal{M}^{*}/\mathcal{O}^{*}$\,
is given by a collection \,$\left\{\,\overset{{\color{white}.}}{f}_{\alpha} \in \mathcal{M}^{*}(U_{\alpha})\,\right\}_{\alpha}$,\,
where \,$\{\,U_{\alpha}\,\}$\, is a complex coordinate atlas of \,$M$,\, and
\begin{equation*}
\dfrac{{\color{white}.}f_{\alpha}{\color{white}.}}{f_{\beta}}
\;\in\;
	\mathcal{O}^{*}\!\left(\,U_{\alpha} \overset{{\color{white}.}}{\cap} U_{\beta}\,\right)
\end{equation*}
This implies that
\begin{equation*}
\ord_{V}(f_{\alpha}) \; = \; \ord_{V}(f_{\beta}),
\quad
\textnormal{for each irreducible hypersurface $V \subset M$}.
\end{equation*}
Hence, the global section
\,$\left\{\,\overset{{\color{white}.}}{f}_{\alpha}\,\right\}$ $\in$
\,$H^{0}\!\left(\,\overset{{\color{white}.}}{M} , \mathcal{M}^{*}/\mathcal{O}^{*}\right)$\,
determines the following divisor in \,$M$:\,
\begin{equation*}
\Div\!\left(\,\{\,\overset{{\color{white}.}}{f}_{\alpha}\,\}\,\right)
\;\; := \;\;
	\underset{V}{\sum} \;\, \ord_{V}(f_{\alpha(V)}) \cdot V
\end{equation*}
where the sum is taken over the collection of all irreducible analytic hypersurfaces
of \,$M$,\, and for each irreducible analytic hypersurface \,$V$,\,
the index $\alpha(V)$ is chosen such that \,$V \cap U_{\alpha} \neq \varemptyset$.
{\color{red}Why is the preceding formal sum locally finite?
(Look up the fact that the local ring at every point of $M$ is a unique factorization domain, and the Weierstrass Preparation Theorem.)}

Conversely, given
\begin{equation*}
D
\;\; = \;\;
	\underset{i}{\sum}\;\, k_{i} \cdot V_{i}
\;\; \in \;\;
	\Div(M),
\end{equation*}
choose an open cover \,$\left\{\,\overset{{\color{white}.}}{U}_{\alpha}\,\right\}$\,
of \,$M$\, such that each irreducible hypersurface $V_{i}$ appearing $D$
has local defining function $h_{\alpha,i} \in \mathcal{O}(U_{\alpha})$.
Then, define
\begin{equation*}
f_{\alpha}
\;\; := \;\;
	\underset{i}{\prod}\; (h_{\alpha,i})^{k_{i}}
\;\; \in \;\;
	\mathcal{M}^{*}(U_{\alpha})
\end{equation*}
It is clear that
\begin{equation*}
\dfrac{{\color{white}.}f_{\alpha}{\color{white}.}}{f_{\beta}}
\;\in\;
	\mathcal{O}^{*}\!\left(\,U_{\alpha} \overset{{\color{white}.}}{\cap} U_{\beta}\,\right)
\end{equation*}
Hence,
\,$\left\{\,\overset{{\color{white}.}}{f}_{\alpha}\,\right\}$\,
defines an element in
\,$H^{0}\!\left(\,\overset{{\color{white}.}}{M} , \mathcal{M}^{*}/\mathcal{O}^{*}\right)$.\,
\qed

          %%%%% ~~~~~~~~~~~~~~~~~~~~ %%%%%

\vskip 1.0cm
\subsection{The correspondence between line bundles and divisors}

          %%%%% ~~~~~~~~~~~~~~~~~~~~ %%%%%

\vskip 1.0cm
\subsection{Line bundles and maps into project space}

Let $M$ be a compact complex manifold, and $L \longrightarrow M$ a holomorphic line bundle on $M$.
Let $H^{0}(M,\mathcal{O}(L))$ be the vector space of global holomorphic sections of $L$.
Let $E \subset H^{0}(M,\mathcal{O}(L))$ be a vector subspace.
For each point \,$p \in M$,
\begin{equation*}
\widetilde{H}_{p} \;\; := \;\; \left\{\,\left. s \overset{{\color{white}.}}{\in} E \,\;\right\vert\, s(p) = 0\,\right\}
\end{equation*}
forms a hyperplane in \,$E$.\,
Clearly, \,$\widetilde{H}_{p}$\, is closed under addition and scalar multiplication;
hence, it is a vector subspace of \,$E$.\,
Secondly, the evaluation map
\,$\ev_{p} : E \longrightarrow \C : s \longrightarrow s(p)$\,
is (clearly) surjective with kernel
\,$\ker(\ev_{p}) = \widetilde{H}_{p}$.\,
Hence,
\begin{equation*}
E \left/ \widetilde{H}_{p}\right.
\;\; = \;\;
	E \left/ \overset{{\color{white}.}}{\ker}(\ev_{p}) \right.
\;\; \cong \;\;
	\image(\ev_{p})
\;\; = \;\;
	\C
\end{equation*}
Thus, \,$\widetilde{H}_{p}$\, has codimension one in \,$E$;\,
hence, \,$\widetilde{H}_{p}$\, is a hyperplane in \,$E$.\,

\begin{equation*}
\iota_{E} \, : \, M \longrightarrow \, \P(E)^{*}\,
\end{equation*}

          %%%%% ~~~~~~~~~~~~~~~~~~~~ %%%%%

\vskip 1.0cm
\subsection{First Chern classes of line bundles}

          %%%%% ~~~~~~~~~~~~~~~~~~~~ %%%%%

\vskip 1.0cm
\subsection{Positivity of line bundles}

          %%%%% ~~~~~~~~~~~~~~~~~~~~ %%%%%
