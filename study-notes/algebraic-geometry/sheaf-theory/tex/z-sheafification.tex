
          %%%%% ~~~~~~~~~~~~~~~~~~~~ %%%%%

\section{Sheafification of presheaves}

%\cite{vanDerVaart1996}
%\cite{Kosorok2008}

%\renewcommand{\theenumi}{\alph{enumi}}
%\renewcommand{\labelenumi}{\textnormal{(\theenumi)}$\;\;$}
\renewcommand{\theenumi}{\roman{enumi}}
\renewcommand{\labelenumi}{\textnormal{(\theenumi)}$\;\;$}

          %%%%% ~~~~~~~~~~~~~~~~~~~~ %%%%%

\begin{example}[A presheaf that is not decent, hence not a sheaf]
\mbox{}\vskip 0.1cm
\noindent
Let \,$X$\, be a two-point topological space \,$\{\,x,y\,\}$\, with the discrete topology.
Define a presheaf \,$\mathscr{F}$\, as follows:
\begin{itemize}
\item
	$
	\mathscr{F}(\varemptyset) \;=\; \{\,0\,\},
	\quad
	\mathscr{F}(\{\,x\,\}) \;=\; \Re,
	\quad
	\mathscr{F}(\{\,y\,\}) \;=\; \Re,
	\quad
	\mathscr{F}(\,X\,) \;=\; \mathscr{F}(\{\,x,y\,\}) \;=\; \Re \times \Re \times \Re
	$
\item
	The restriction map
	\,$\mathscr{F}(\{\,x,y\,\}) \longrightarrow \mathscr{F}(\{\,x\,\})$\,
	is the projection onto the first factor of
	\,$\mathscr{F}(\{\,x,y\,\}) \,=\, \Re \times \Re \times \Re$,\,
	while the restriction map
	\,$\mathscr{F}(\{\,x,y\,\}) \longrightarrow \mathscr{F}(\{\,y\,\})$\,
	is the projection onto the second factor of \,$\Re \times \Re \times \Re$.\,
\end{itemize}
It is straightforward to check that the above indeed defines a presheaf.
Now, consider the two sections
\begin{equation*}
s \;=\; (\,1,2,3\,), \quad s^{\prime} \;=\; (\,1,2,4\,) \;\;\in\;\; \mathscr{F}(\,X\,) \;=\; \mathscr{F}(\{\,x,y\,\})
\end{equation*}
Note that \,$s \neq s^{\prime}$,\, but \,$s$\, and \,$s^{\prime}$\, have the same restrictions:
\begin{equation*}
s\,\vert_{\varemptyset} \;=\; s^{\prime}\,\vert_{\varemptyset} \;=\; 0,
\quad\textnormal{and}\quad
s\,\vert_{\{x\}} \;=\; s^{\prime}\,\vert_{\{x\}} \;=\; 1,
\quad\textnormal{and}\quad
s\,\vert_{\{y\}} \;=\; s^{\prime}\,\vert_{\{y\}} \;=\; 2
\end{equation*}
This proves that \,$\mathscr{F}$\, is not a decent presheaf; in particular, it is not a sheaf.
\end{example}

          %%%%% ~~~~~~~~~~~~~~~~~~~~ %%%%%

\vskip 0.5cm
\subsection{The sheaf of discontinuous sections of a presheaf}
\setcounter{theorem}{0}
\setcounter{equation}{0}

          %%%%% ~~~~~~~~~~~~~~~~~~~~ %%%%%

\vskip 0.5cm
\begin{definition}
\mbox{}\vskip 0.1cm
\noindent
Suppose $\mathscr{P}$ is a presheaf over a topological space $X$.
\vskip 0.1cm
\noindent
The \textbf{sheaf \,$\mathscr{D}(\mathscr{P})$\, of discontinuous sections} of $\mathscr{P}$
is defined as follows:
\begin{itemize}
\item
	For each open subset $U \subset X$,
	\begin{equation*}
	\mathscr{D}(\mathscr{P})(\,U)
	\;\; := \;\;
		\underset{x \in U}{\prod} \, \mathscr{P}_{x}
	\end{equation*}
	In particular, a section $\sigma \in \mathscr{D}(\mathscr{P})(\,U)$ has the form:
	\begin{equation*}
	\sigma
	\;\, = \;
		\left(\,\overset{{\color{white}.}}{\sigma_{x}}\,\right)_{x \in U}
	\quad
	\textnormal{where \;$\sigma_{x} \in \mathscr{P}_{x}$,\; for each $x \in U$}
	\end{equation*}
\item
	For each open subsets $V \subset U \subset X$,
	the restriction map of $\mathscr{D}(\mathscr{P})$ is defined by:
	\begin{equation*}
	\res^{\,U}_{\,V}\!\left(\,\overset{{\color{white}.}}{\sigma}\,\right)
	\;\; = \;\;
		\res^{\,U}_{\,V}\!\left(\,\left(\,\overset{{\color{white}.}}{\sigma_{x}}\,\right)_{x \in U}\,\right)
	\;\; := \;\;
		\left(\,\overset{{\color{white}.}}{\sigma_{x}}\,\right)_{x \in V}
	\end{equation*}
\end{itemize}
\end{definition}

          %%%%% ~~~~~~~~~~~~~~~~~~~~ %%%%%

\vskip 0.5cm
\begin{proposition}
\mbox{}\vskip 0.1cm
\noindent
The sheaf of discontinuous sections of a presheaf over a topological space is indeed a sheaf.
\end{proposition}
\proof
\vskip 0.3cm
\noindent
Suppose $\mathscr{P}$ is a presheaf over a topological space $X$, and
$\mathscr{D}(\mathscr{P})$ is the ``sheaf'' of discontinuous sections of $\mathscr{P}$.

\vskip 0.5cm
\noindent
\textbf{Claim 0:}\quad $\mathscr{D}(\mathscr{P})$ is a presheaf.
\vskip 0.2cm
\noindent
Proof of Claim 0:\;\; Observe that:
\begin{itemize}
\item
	For any open subset \,$U \subset X$,\, and any section
	\,$\sigma = \left(\,\overset{{\color{white}.}}{\sigma}_{x}\,\right)_{x \in U} \in \mathscr{D}(\mathscr{P})(U)$,\,
	we have:
	\begin{equation*}
	\res^{\,U}_{\,{\color{red}U}}\!\left(\,\overset{{\color{white}.}}{\sigma}\,\right)
	\;\; = \;\;
		\res^{\,U}_{\,{\color{red}U}}\!\left(\,\left(\,\overset{{\color{white}.}}{\sigma_{x}}\,\right)_{x \in U}\,\right)
	\;\; = \;\;
		\left(\,\overset{{\color{white}.}}{\sigma_{x}}\,\right)_{x \in {\color{red}U}}
	\end{equation*}
	In other words, \,$\res^{\,U}_{\,U} \,=\, \id_{U}$.
\item
	For any open subsets \,$W \subset V \subset U \subset X$,\, and any section
	\,$\sigma = \left(\,\overset{{\color{white}.}}{\sigma}_{x}\,\right)_{x \in U} \in \mathscr{D}(\mathscr{P})(U)$,\,
	we have:
	\begin{equation*}
	\res^{\,V}_{\,W}\!\left(\,
		\overset{{\color{white}\textnormal{\large$.$}}}{\res^{\,U}_{\,V}}
		\!\left(\,\overset{{\color{white}.}}{\sigma}\,\right)
		\,\right)
	\;\; = \;\;
		\res^{\,V}_{\,W}\!\left(\,\left(\,\overset{{\color{white}.}}{\sigma_{x}}\,\right)_{x \in V}\,\right)
	\;\; = \;\;
		\left(\,\overset{{\color{white}.}}{\sigma_{x}}\,\right)_{x \in W}
	\;\; = \;\;
		\res^{\,U}_{\,W}\!\left(\,\overset{{\color{white}.}}{\sigma}\,\right)
	\end{equation*}
	In other words, \,$\res^{\,V}_{\,W} \,\circ\, \res^{\,U}_{\,V} \; = \; \res^{\,U}_{\,W}$.
\end{itemize} 
Claim 0 follows, by definition of presheaves, immediately from the above two observations.

\vskip 0.5cm
\noindent
Next, let $U = \,\underset{\alpha}{\bigcup}\;U_{\alpha}$ be an open cover of an open subset $U \subset X$.

\vskip 0.5cm
\noindent
\textbf{Claim 1:}\quad
Let \,$\sigma$, $\tau$ $\in$ $\mathscr{D}(\mathscr{P})(U)$.
\,\;Then,
\,$\left.\overset{{\color{white}.}}{\sigma}\,\right\vert_{U_{\alpha}}$
$=$
$\left.\overset{{\color{white}.}}{\tau}\,\right\vert_{U_{\alpha}}$\,,\,
for each $\alpha$
\;\;$\Longrightarrow$\;\;
$\sigma \,=\, \tau$.
\vskip 0.2cm
\noindent
Proof of Claim 1:
\begin{eqnarray*}
	\left.\overset{{\color{white}.}}{\sigma}\,\right\vert_{U_{\alpha}}
	\, = \,
	\left.\overset{{\color{white}.}}{\tau}\,\right\vert_{U_{\alpha}},
	\;\;\textnormal{for each $\alpha$}
& \Longleftrightarrow &
	\left(\,\overset{{\color{white}.}}{\sigma_{x}}\,\right)_{x \in U_{\alpha}}
	\;\; =: \;\;
	\left.\overset{{\color{white}.}}{\sigma}\,\right\vert_{U_{\alpha}}
	\;\; = \;\;
	\left.\overset{{\color{white}.}}{\tau}\,\right\vert_{U_{\alpha}}
	\;\; := \;\;
	\left(\,\overset{{\color{white}.}}{\tau_{x}}\,\right)_{x \in U_{\alpha}},
	\;\;\textnormal{for each $\alpha$}
\\
& \overset{{\color{white}\textnormal{\LARGE$1$}}}{\Longleftrightarrow} &
	\sigma_{x} \; = \; \tau_{x}\,,
	\;\;\textnormal{for each $x \in U_{\alpha}$,\, for each $\alpha$}
\\
& \overset{{\color{white}\textnormal{\LARGE$1$}}}{\Longleftrightarrow} &
	\left(\,\overset{{\color{white}.}}{\sigma_{x}}\,\right)_{x \in U}
	\;\; = \;\;
	\left(\,\overset{{\color{white}.}}{\tau_{x}}\,\right)_{x \in U},
	\;\;\textnormal{for each $x \in U = \,\underset{\alpha}{\bigcup}\;U_{\alpha}$}
\\
& \overset{{\color{white}\textnormal{\tiny$1$}}}{\Longleftrightarrow} &
	\sigma \; = \; \tau
\end{eqnarray*}
This proves Claim 1.

\vskip 0.5cm
\noindent
\textbf{Claim 2:}\quad
Let
\,$\sigma_{\alpha}$ $\in$ $\mathscr{D}(\mathscr{P})(U_{\alpha})$\,
be such that
\,$\left.\overset{{\color{white}.}}{\sigma_{\alpha}}\,\right\vert_{\,U_{\alpha} \cap\,U_{\beta}}$
$=$
\,$\left.\overset{{\color{white}.}}{\sigma_{\beta}}\,\right\vert_{\,U_{\alpha} \cap\,U_{\beta}}$.
\,\;Then,
\begin{equation*}
\textnormal{there exists \;$\sigma \in \mathscr{D}(\mathscr{P})(U)$\; such that}\;\;
\sigma_{\alpha}
\;=\;
\left.\overset{{\color{white}.}}{\sigma}\,\right\vert_{\,U_{\alpha}},
\;\;\textnormal{for each $\alpha$}
\end{equation*}
\vskip 0.2cm
\noindent
Proof of Claim 2:\quad
First, observation that
\begin{eqnarray*}
&&
	\left.\overset{{\color{white}.}}{\sigma_{\alpha}}\,\right\vert_{\,U_{\alpha}\cap\,U_{\beta}}
	\, = \,
	\left.\overset{{\color{white}.}}{\sigma_{\beta}}\,\right\vert_{\,U_{\alpha}\cap\,U_{\beta}},
	\;\;\textnormal{for each $\alpha$, $\beta$}
\\
& \overset{{\color{white}\textnormal{\LARGE$1$}}}{\Longleftrightarrow} &
	\left(\,\overset{{\color{white}.}}{(\sigma_{\alpha})_{x}}\,\right)_{x \,\in\, U_{\alpha}\cap\,U_{\beta}}
	\;\; =: \;\;
	\left.\overset{{\color{white}.}}{\sigma_{\alpha}}\,\right\vert_{\,U_{\alpha}\cap\,U_{\beta}}
	\;\; = \;\;
	\left.\overset{{\color{white}.}}{\sigma_{\beta}}\,\right\vert_{\,U_{\alpha}\cap\,U_{\beta}}
	\;\; := \;\;
	\left(\,\overset{{\color{white}.}}{(\sigma_{\beta})_{x}}\,\right)_{x \,\in\, U_{\alpha}\cap\,U_{\beta}},
	\;\;\textnormal{for each $\alpha$, $\beta$}
\\
& \overset{{\color{white}\textnormal{\LARGE$1$}}}{\Longleftrightarrow} &
	(\,\sigma_{\alpha}\,)_{x} \; = \; (\,\sigma_{\beta}\,)_{x}\,,
	\;\;\textnormal{for each $x \in U_{\alpha}\cap\,U_{\alpha}$,\, for each $\alpha$, $\beta$}
\end{eqnarray*}
The above observation allows us to unambiguously define, for each $x \in U$,
\begin{equation*}
\sigma_{x}
\;\; := \;\;
	(\,\sigma_{\alpha}\,)_{x}\,,
\quad
\textnormal{for any \,$\alpha$\, such that \,$U_{\alpha} \ni x$}
\end{equation*}
Now, define
\,$\sigma \,\in\, \mathscr{D}(\mathscr{P})(U)$\,
by
\,$\sigma \,:=\, \left(\,\overset{{\color{white}.}}{\sigma_{x}}\,\right)_{x \in U}$.\,
Then, for each $\alpha$, we have:
\begin{equation*}
\left.\overset{{\color{white}\textnormal{\Large$.$}}}{\sigma}\,\right\vert_{U_{\alpha}}
\;\; = \;\;
\left(\,\overset{{\color{white}.}}{\sigma_{x}}\,\right)_{x \in U_{\alpha}}
\;\; = \;\;
\left(\,\overset{{\color{white}.}}{(\sigma_{\alpha})_{x}}\,\right)_{x \,\in\, U_{\alpha}}
\;\; = \;\;
\sigma_{\alpha}
\end{equation*}
This proves Claim 2.

\vskip 0.5cm
\noindent
Claim 1 and Claim 2 together imply that
\,$\mathscr{D}(\mathscr{P})$\,
is a sheaf, which completes the proof of the Proposition.
\qed

          %%%%% ~~~~~~~~~~~~~~~~~~~~ %%%%%
          %%%%% ~~~~~~~~~~~~~~~~~~~~ %%%%%

\vskip 1.0cm
\subsection{The discretization morphism of a presheaf}
\setcounter{theorem}{0}
\setcounter{equation}{0}

\begin{lemma}
\mbox{}\vskip 0.1cm
\noindent
Suppose
\,$\mathscr{F}$, $\mathscr{G}$, $\mathscr{H}$\,
are presheaves on a topological space $X$, and
\,$\mathscr{D}(\mathscr{F})$, $\mathscr{D}(\mathscr{G})$, $\mathscr{D}(\mathscr{H})$\,
are the sheaves of discontinuous sections of
\,$\mathscr{F}$, $\mathscr{G}$, $\mathscr{H}$,\,
respectively.
\begin{enumerate}
\item
	For each open subset $U \subset X$ and each $\sigma \in \mathscr{F}(U)$, define
	\begin{equation*}
	\iota_{U}(\,\sigma\,)
	\;\; := \;\;
		\left(\,\overset{{\color{white}.}}{\sigma_{x}}\,\right)_{x \in U}
	\end{equation*}
	Then,
	\,$\mathscr{F}\,\overset{\iota}{\longrightarrow}\,\mathscr{D}(\mathscr{F})$\,
	is a morphism of presheaves.
	\vskip 0.05cm
	We call
	\,$\mathscr{F}\,\overset{\iota}{\longrightarrow}\,\mathscr{D}(\mathscr{F})$\,
	the \textbf{discretization morphism} of the presheaf \,$\mathscr{F}$.
\item
	For each morphism of presheaves
	\,$\mathscr{F}\,\overset{\varphi}{\longrightarrow}\,\mathscr{G}$,\,
	there exists a unique morphism of presheaves
	\begin{tikzcd}
	\mathscr{D}(\mathscr{F}) \arrow[r, "\mathscr{D}(\varphi)"] & \mathscr{D}(\mathscr{G})
	\end{tikzcd}
	such that the following diagram commutes:
	\begin{equation*}
	\begin{tikzcd}
	                    \mathscr{F}  \arrow[r, "\varphi"] \arrow[d, "\iota"'] & \mathscr{G} \arrow[d, "\iota"] \\
	\mathscr{D}(\mathscr{F}) \arrow[r, "\mathscr{D}(\varphi)"'] & \mathscr{D}(\mathscr{G})
	\end{tikzcd}
	\end{equation*}
\item
	$\mathscr{D}(\,\id_{\mathscr{F}}\,)$ \,$=$\, $\id_{\mathscr{D}(\mathscr{F})}$ 
\item
	For each pair of morphisms of presheaves
	\,$\mathscr{F}\,\overset{\varphi}{\longrightarrow}\,\mathscr{G}$\,
	and
	\,$\mathscr{G}\,\overset{\psi}{\longrightarrow}\,\mathscr{H}$,\,
	we have:
	\begin{equation*}
	\mathscr{D}(\,\psi \circ \varphi\,) \;\; = \;\; \mathscr{D}(\,\psi\,) \circ \mathscr{D}(\,\varphi\,)
	\end{equation*}
\end{enumerate}
\end{lemma}
\proof
\begin{enumerate}
\item
	Recall that 
	\,$\mathscr{F}\,\overset{\iota}{\longrightarrow}\,\mathscr{D}(\mathscr{F})$\,
	being a morphism of presheaves simply means that,
	for each open sets $V \subset U \subset X$, the following diagram commutes:
	\begin{equation*}
	\begin{tikzcd}
	\mathscr{F}(U)  \arrow[r, "\iota_{U}"] \arrow[d, "\res^{U}_{V}"'] & \mathscr{D}(\mathscr{F})(U) \arrow[d, "\res^{U}_{V}"] \\
	\mathscr{F}(V) \arrow[r, "\iota_{V}"'] & \mathscr{D}(\mathscr{F})(V)
	\end{tikzcd}
	\end{equation*}
	Now, for each open subsets \,$V \subset U \subset X$,\, and each \,$\sigma \in \mathscr{F}(U)$,
	we have
	\begin{equation*}
	\res^{\,U}_{\,V}\!\left(\,\overset{{\color{white}.}}{\iota_{U}(\,\sigma\,)}\,\right)
	\;\; = \;\;
		\res^{\,U}_{\,V}\!\left(\,
			\left(\,\overset{{\color{white}.}}{\sigma_{x}}\,\right)_{x \in U}
			\,\right)
	\;\; = \;\;
		\left(\,\overset{{\color{white}.}}{\sigma_{x}}\,\right)_{x \in V}
	\;\; = \;\;
		\left(\,\overset{{\color{white}.}}{\left(\res^{\,U}_{\,V}(\sigma)\right)_{x}}\,\right)_{x \in V}
%	\\
%	& \overset{{\color{white}\textnormal{\large$1$}}}{=} &
	\;\; = \;\;
		\iota_{V}\!\left(\,\overset{{\color{white}.}}{\res^{\,U}_{\,V}(\sigma)}\,\right),
	\end{equation*}
	which proves the commutativity of the above diagram, which in turn proves that
	\,$\mathscr{F}\,\overset{\iota}{\longrightarrow}\,\mathscr{D}(\mathscr{F})$\,
	is indeed a morphism of presheaves.
\item
	For each
	\,$\tau$
	\,$=$\, $\left(\,\overset{{\color{white}.}}{\tau_{x}}\,\right)_{x \in U}$
	\,$\in$\, $\mathscr{D}(\mathscr{F})(U)$,\,
	we define
	\begin{equation*}
	\mathscr{D}(\varphi)\!\left(\,\overset{{\color{white}.}}{\tau}\,\right)
	\;\; = \;\;
		\mathscr{D}(\varphi)\!\left[\;\left(\,\overset{{\color{white}.}}{\tau_{x}}\,\right)_{x \in U}\,\right]
	\;\; := \;\;
		\left(\,\overset{{\color{white}.}}{\varphi_{x}(\tau_{x})}\,\right)_{x \in U}
	\end{equation*}
	Now, observe that, for each \,$\sigma \in \mathscr{F}(U)$,\, we then have
	\begin{equation*}
	\iota\!\left(\,\overset{{\color{white}.}}{\varphi(\,\sigma\,)}\,\right)
	\;\; = \;\;
		\left(\,\overset{{\color{white}.}}{\varphi(\,\sigma\,)_{x}}\,\right)_{x \in U}
	\;\; = \;\;
		\left(\,\overset{{\color{white}.}}{\varphi_{x}(\,\sigma_{x}\,)}\,\right)_{x \in U}
	\;\; = \;\;
		\mathscr{D}(\varphi)\!\left[\;\left(\,\overset{{\color{white}.}}{\sigma_{x}}\,\right)_{x \in U}\,\right]
	\;\; = \;\;
		\mathscr{D}(\varphi)\!\left(\,\overset{{\color{white}.}}{\iota(\,\sigma\,)}\,\right)
	\end{equation*}
	Hence, \;$\iota \circ \varphi \,=\, \mathscr{D}(\varphi) \circ \iota$\,,\; as required.
\item
	For each
	\,$\left(\,\overset{{\color{white}.}}{\sigma_{x}}\,\right)_{x \in U}$
	\,$\in$\, $\mathscr{D}(\mathscr{F})(U)$,\,
	observe that
	\begin{equation*}
	\mathscr{D}(\,\id_{\mathscr{F}}\,)\!\left[\;\left(\,\overset{{\color{white}.}}{\sigma_{x}}\,\right)_{x \in U}\,\right]
	\;\; = \;\;
		\left(\,\overset{{\color{white}.}}{(\id_{\mathscr{F}})_{x}(\,\sigma_{x}\,)}\,\right)_{x \in U}
	\;\; = \;\;
		\left(\;\overset{{\color{white}-}}{\sigma_{x}}\;\right)_{x \in U}
	\;\; = \;\;
		\id_{\mathscr{D}(\mathscr{F})}\!\left[\;\left(\,\overset{{\color{white}.}}{\sigma_{x}}\,\right)_{x \in U}\,\right],
	\end{equation*}
	which proves: \;$\mathscr{D}(\,\id_{\mathscr{F}}\,) \,=\, \id_{\mathscr{D}(\mathscr{F})}$.
\item
	For each
	\,$\left(\,\overset{{\color{white}.}}{\sigma_{x}}\,\right)_{x \in U}$
	\,$\in$\, $\mathscr{D}(\mathscr{F})(U)$,\,
	observe that
	\begin{eqnarray*}
	\mathscr{D}(\,\psi\circ\varphi\,)\!\left[\;\left(\,\overset{{\color{white}.}}{\sigma_{x}}\,\right)_{x \in U}\,\right]
	& = &
		\left(\,\overset{{\color{white}.}}{(\psi\circ\varphi)_{x}(\,\sigma_{x}\,)}\,\right)_{x \in U}
	\;\; = \;\;
		\left(\,\overset{{\color{white}.}}{\psi_{x}(\,\varphi_{x}(\,\sigma_{x}\,))}\,\right)_{x \in U}
	\;\; = \;\;
		\mathscr{D}(\,\psi\,)\!\left[\,\left(\,\overset{{\color{white}.}}{\varphi_{x}(\,\sigma_{x}\,)}\,\right)_{x \in U}\,\right]
	\\
	& \overset{{\color{white}\textnormal{\Large$1$}}}{=} &
		\mathscr{D}(\,\psi\,)\!\left[\;
			\mathscr{D}(\,\varphi\,)\!\left[\;\left(\,\overset{{\color{white}.}}{\sigma_{x}}\,\right)_{x \in U}\,\right]
			\,\right]
	\;\; = \;\;
		\mathscr{D}(\,\psi\,)\circ\mathscr{D}(\,\varphi\,)
			\!\left[\;\left(\,\overset{{\color{white}.}}{\sigma_{x}}\,\right)_{x \in U}\,\right],
	\end{eqnarray*}
	which proves: \;$\mathscr{D}(\,\psi\circ\varphi\,) \,=\, \mathscr{D}(\,\psi\,)\circ\mathscr{D}(\,\varphi\,)$.
\end{enumerate}
\qed

          %%%%% ~~~~~~~~~~~~~~~~~~~~ %%%%%
          %%%%% ~~~~~~~~~~~~~~~~~~~~ %%%%%

\vskip 1.0cm
\subsection{Some important properties of sub-presheaves}
\setcounter{theorem}{0}
\setcounter{equation}{0}

\begin{proposition}\label{subPresheafOfDecentPresheafIsDecent}
\mbox{}\vskip 0.1cm
\noindent
A sub-presheaf of a decent presheaf is itself decent.
\end{proposition}
\proof

\qed

\vskip 0.5cm
\begin{proposition}[The smallest sub-sheaf containing a sub-presheaf of a sheaf]
\mbox{}\vskip 0.1cm
\noindent
Suppose \,$\mathscr{S}$\, is sheaf over a topological space \,$X$\,
and \,$\mathscr{P}$\, is a sub-presheaf of \,$\mathscr{S}$.
Then, the smallest sub-sheaf of \,$\mathscr{S}$\, containing \,$\mathscr{P}$ exists.
\end{proposition}
\proof
For each open subset \,$U \subset X$\, of \,$X$,\, define:
\begin{equation*}
\mathscr{F}(U)
\;\; := \;\;
	\left\{\;\,
		\sigma \in \mathscr{S}(U)
		\,\;\left\vert\;
			\begin{array}{c}
			\textnormal{there exists an open cover \,$U \,=\, \underset{\alpha}{\bigcup}\;U_{\alpha}$}
			\\
			\textnormal{such that \,$\left.\overset{{\color{white}.}}{\sigma}\,\right\vert^{(\mathscr{S})}_{\,U_{\alpha}} \in \mathscr{P}(U_{\alpha})$\, for each \,$\alpha$}
			\end{array}
			\right.
		\right\}
\;\; \subset \;\;
	\mathscr{S}(U)
\end{equation*}

\vskip 0.5cm
\noindent
\textbf{Claim 0:}\;\;
For each open subset \,$U \subset X$, we have \,$\mathscr{P}(U) \,\subset \mathscr{F}(U)$.
\vskip 0.2cm
\noindent
Proof of Claim 0:\;\; Immediate.

\vskip 0.5cm
\noindent
\textbf{Claim 1:}\;\;
For each \,$\sigma \,\in\, \mathscr{F}(U) \,\subset\, \mathscr{S}(U)$\, and each open subset \,$V \subset U$,\,
$\left.\overset{{\color{white}.}}{\sigma}\,\right\vert^{(\mathscr{S})}_{\,V}$
\,$\in$\,
${\color{red}\mathscr{F}}(V)$
\vskip 0.2cm
\noindent
Proof of Claim 1:\;\;
Since \,$\sigma \,\in\, \mathscr{F}(U)$,\, there exists an open cover
\,$U \,=\, \underset{\alpha}{\bigcup}\;U_{\alpha}$\, such that
\,$\left.\overset{{\color{white}.}}{\sigma}\,\right\vert^{(\mathscr{S})}_{\,U_{\alpha}} \,\in\, \mathscr{P}(U_{\alpha})$.\,
Next, note that, since \,$V \subset U$,\, we have
\,$V \,=\, \underset{\alpha}{\bigcup}\;(\,V\cap\,U_{\alpha})$.\,
Hence,
\begin{equation*}
\left.\left(\,\left.\overset{{\color{white}.}}{\sigma}\,\right\vert^{({\color{red}\mathscr{S}})}_{\,V}\right)\right\vert^{(\mathscr{S})}_{\,V\cap\,U_{\alpha}}
\;\; = \;\;
	\left.\overset{{\color{white}.}}{\sigma}\,\right\vert^{({\color{red}\mathscr{S}})}_{\,V\cap\,U_{\alpha}}
\;\; = \;\;
	\left.\left(\,\left.\overset{{\color{white}.}}{\sigma}\,\right\vert^{({\color{red}\mathscr{S}})}_{\,{\color{red}U_{\alpha}}}\right)\right\vert^{(\mathscr{S})}_{\,V\cap\,U_{\alpha}}
\;\; \in \;\;
	\mathscr{P}(U_{\alpha}\cap\,V)\,,
\end{equation*}
which shows that
\,$\left.\overset{{\color{white}.}}{\sigma}\,\right\vert^{(\mathscr{S})}_{\,V}$
\,$\in$\,
${\color{red}\mathscr{F}}(V)$.\,
This proves Claim 1.

\vskip 0.5cm
\noindent
Claim 1 allows us to define the restriction maps of \,$\mathscr{F}$:
\begin{equation*}
\left.\overset{{\color{white}.}}{\sigma}\,\right\vert^{(\mathscr{F})}_{\,V}
\; := \;
	\left.\overset{{\color{white}.}}{\sigma}\,\right\vert^{({\color{red}\mathscr{S}})}_{\,V}
\; \in \;
	\mathscr{F}(V)
\; \subset \;
	\mathscr{S}(V)\,,
\quad
\textnormal{for each \,$\sigma \,\in\, \mathscr{F}(U) \,\subset\, \mathscr{S}(U)$\, and each open subset \,$V \subset U$}.
\end{equation*}
Note that \,$\mathscr{F}$\, is then a sub-presheaf of the sheaf \,$\mathscr{S}$.

\vskip 0.5cm
\noindent
\textbf{Claim 2:}\;\; $\mathscr{F}$ is decent.
\vskip 0.2cm
\noindent
Proof of Claim 2:\;\; Immediate by Proposition \ref{subPresheafOfDecentPresheafIsDecent}

\vskip 0.5cm
\noindent
\textbf{Claim 3:}\;\;
Suppose \,$\sigma_{\alpha} \,\in\, \mathscr{F}(U_{\alpha})$ are such that
\,$\left.\overset{{\color{white}.}}{\sigma}_{\alpha}\right\vert^{(\mathscr{F})}_{\,U_{\alpha\,\cap\,U_{\beta}}}$
\,$=$\,
$\left.\overset{{\color{white}.}}{\sigma}_{\beta}\right\vert^{(\mathscr{F})}_{\,U_{\alpha\,\cap\,U_{\beta}}}$,\,
for each \,$\alpha, \beta$.
Then, there exists a unique
\,$\sigma \in \mathscr{F}\!\left(\,\underset{\alpha}{\bigcup}\;U_{\alpha}\right)$\,
such that
\,$\left.\overset{{\color{white}.}}{\sigma}\,\right\vert^{(\mathscr{F})}_{\,U_{\alpha}}$
\,$=$\,
$\sigma_{\alpha} \,\in\, \mathscr{F}(U_{\alpha})$,\,
for each \,$\alpha$.
\vskip 0.3cm
\noindent
Proof of Claim 3:\;\;
Since \,$\mathscr{F}$\, is a sub-presheaf of the the sheaf \,$\mathscr{S}$,\,
we see immediately that there exists a unique
\,$\sigma \,\in\, \mathscr{S}\!\left(\,\underset{\alpha}{\bigcup}\;U_{\alpha}\right)$\,
such that
\,$\left.\overset{{\color{white}.}}{\sigma}\,\right\vert^{({\color{red}\mathscr{S}})}_{\,U_{\alpha}}$
\,$=$\,
$\sigma_{\alpha} \,\in\, \mathscr{F}(U_{\alpha}) \,\subset\, \mathscr{S}(U_{\alpha})$,\, for each \,$\alpha$.
Thus, it remains only to show that we in fact have
\,$\sigma \,\in\, {\color{red}\mathscr{F}}\!\left(\,\underset{\alpha}{\bigcup}\;U_{\alpha}\right)$.\,
Now, each \,$U_{\alpha}$\,
admits an open cover
\,$U_{\alpha} \,= \underset{\beta\,\in\,\Lambda_{\alpha}}{\bigcup}V^{(\alpha)}_{\beta}$\,
such that
\,$\left.\overset{{\color{white}.}}{\sigma}_{\alpha}\right\vert^{(\mathscr{S})}_{\,V^{((\alpha)}_{\beta}}$
\,$\in$\, $\mathscr{P}(V^{((\alpha)}_{\beta})$.\,
Hence,
\begin{eqnarray*}
\underset{\alpha}{\bigcup}\;U_{\alpha}
& = &
	\underset{\alpha}{\bigcup}\left(\,\underset{\beta\,\in\,\Lambda_{\alpha}}{\bigcup}V^{(\alpha)}_{\beta}\,\right),
	\quad\textnormal{and}
\\
\left.\overset{{\color{white}.}}{\sigma}\,\right\vert^{(\mathscr{S})}_{\,V^{((\alpha)}_{\beta}}
& \overset{{\color{white}\textnormal{\Huge$1$}}}{=} &
	\left.
		\left(\;
			\left.\overset{{\color{white}.}}{\sigma}\,\right\vert^{(\mathscr{S})}_{\,U_{\alpha}}
			\;\right)
		\right\vert^{(\mathscr{S})}_{\,V^{((\alpha)}_{\beta}}
\;\; = \;\;
	\left.\overset{{\color{white}.}}{\sigma}_{\alpha}\right\vert^{(\mathscr{S})}_{\,V^{((\alpha)}_{\beta}}
\;\; \in \;\;
	\mathscr{P}(V^{((\alpha)}_{\beta})\,,
\quad\textnormal{for each \,$\beta \in \Lambda_{\alpha}$,\, for each \,$\alpha$}
\end{eqnarray*}
It follows immediately from the two observations above that
\,$\sigma \,\in\, \mathscr{F}\!\left(\,\underset{\alpha}{\bigcup}\;U_{\alpha}\right)$.\,
This proves Claim 3.

\vskip 0.5cm
\noindent
We have established that
\begin{itemize}
\item
	$\mathscr{F}$\, is a sub-presheaf of \,$\mathscr{S}$ (by Claim 1),
\item
	$\mathscr{F}$\, is furthermore a sub-sheaf of \,$\mathscr{S}$ (by Claim 2 and Claim 3), and
\item
	$\mathscr{P}$\, is a sub-presheaf of \,$\mathscr{F}$ (by Claim 0).
\end{itemize}

\vskip 0.5cm
\noindent
\textbf{Claim 4:}\;\;
If \,$\mathscr{G}$\, is a sub-sheaf of \,$\mathscr{S}$\, that contains the sub-presheaf \,$\mathscr{P} \subset \mathscr{S}$,\,
then \,$\mathscr{F}$\, is a sub-sheaf of \,$\mathscr{G}$.
\vskip 0.2cm
\noindent
Proof of Claim 4:\;\;
We need to show that for each
\,$\sigma \,\in\, \mathscr{F}(U) \,\subset\, \mathscr{S}(U)$,\,
we also have 
\,$\sigma \,\in\, \mathscr{G}(U)$.\,
To this end, let 
\,$\sigma \,\in\, \mathscr{F}(U) \,\subset\, \mathscr{S}(U)$.\,
Then, there exists an open cover
\,$U \,=\, \underset{\alpha}{\bigcup}\;U_{\alpha}$\,
such that
\begin{equation*}
\left.\overset{{\color{white}.}}{\sigma}\,\right\vert^{(\mathscr{F})}_{\,U_{\alpha}}
\;\; := \;\;
	\left.\overset{{\color{white}.}}{\sigma}\,\right\vert^{(\mathscr{S})}_{\,U_{\alpha}}
\;\;\in\;\;
	\mathscr{P}(U_{\alpha})
\;\;\subset\;\;
\mathscr{G}(U_{\alpha})
\;\;\subset\;\;
	\mathscr{S}(U_{\alpha})
\end{equation*}
Since \,$\mathscr{G}$\, is by hypothesis a (sub-)sheaf (of $\mathscr{S}$), we therefore have:
\begin{equation*}
\sigma
\;\in\; \mathscr{G}\!\left(\,\underset{\alpha}{\bigcup}\;U_{\alpha}\,\right)
\; = \; \mathscr{G}\!\left(\,\overset{{\color{white}.}}{U}\,\right)
\end{equation*}
This proves Claim 4.

\vskip 0.5cm
\noindent
Claim 4 shows that \,$\mathscr{F}$\, is indeed the smallest sub-sheaf of \,$\mathscr{S}$\,
that contains the sub-presheaf \,$\mathscr{P} \subset \mathscr{S}$.\,
This completes the proof of the Proposition.
\qed

          %%%%% ~~~~~~~~~~~~~~~~~~~~ %%%%%

\vskip 1.0cm
\begin{proposition}[The image of a presheaf morphism is a sub-presheaf of the codomain presheaf]
\mbox{}\vskip 0.1cm
\noindent
Suppose \,$\mathscr{F}$\, and \,$\mathscr{G}$\, are presheaves over a topological space \,$X$,\,
and \,$\varphi : \mathscr{F} \longrightarrow \mathscr{G}$\, is a morphism of presheaves.
For each open subset \,$U \subset X$,\, define:
\begin{equation*}
\varphi(\mathscr{F})(U)
\;\; := \;\;
	\varphi_{U}\!\left(\,\overset{{\color{white}.}}{\mathscr{F}(U)}\,\right)
\;\; \in \;\;
	\mathscr{G}(U)
\end{equation*}
Then,
\begin{enumerate}
\item
	$\varphi(\mathscr{F})$\, is a sub-presheaf of the presheaf \,$\mathscr{G}$,\, and
\item
	$\varphi(\mathscr{F})_{x}$ \,$=$\, $\varphi_{x}\!\left(\,\mathscr{F}_{x}\,\right)$ \,$\subset$\, $\mathscr{G}_{x}$,\,
	for each \,$x \in X$.
\end{enumerate}
We shall call \,$\varphi(\mathscr{F})$\, the \underline{\textbf{image sub-presheaf}} of \,$\mathscr{F}$\,
under the morphism \,$\varphi$.
\end{proposition}
\proof
\begin{enumerate}
\item
	We need to show that for each open subsets $V \subset U \subset X$,
	the restriction map of $\mathscr{G}$ from $U$ to $V$ maps
	$\varphi(\mathscr{F})(U)$ into $\varphi(\mathscr{F})(V)$.
	Hence, let \,$\sigma \,\in\, \mathscr{F}(U)$,\, and
	$\tau$
	\,$:=$\, $\varphi_{U}(\,\sigma\,)$
	\,$\in$\, $\varphi_{U}\!\left(\,\overset{{\color{white}.}}{\mathscr{F}(U)}\,\right)$
	\,$=:$\, $\varphi(\mathscr{F})(U)$.
	\begin{equation*}
	\left.\overset{{\color{white}.}}{\tau}\,\right\vert^{(\mathscr{G})}_{\,V}
	\;\; = \;\;
		\left.\overset{{\color{white}.}}{\varphi_{U}(\,\sigma\,)}\,\right\vert^{(\mathscr{G})}_{\,V}
	\;\; = \;\;
		\varphi_{V}\!\left(\,
			\left.\overset{{\color{white}.}}{\sigma}\,\right\vert^{(\mathscr{F})}_{\,V}
			\,\right)
	\;\; \in \;\;
		\varphi_{V}\!\left(\,\overset{{\color{white}.}}{\mathscr{F}(V)}\,\right)
	\;\; =: \;\;
		\varphi(\mathscr{F})(V)
	\end{equation*}
\item
	\textbf{Claim 1:}\;\; $\varphi(\mathscr{F})_{x} \;\subset\; \varphi_{x}(\,\mathscr{F}_{x}\,)$
	\vskip 0.2cm
	Proof of Claim 1:\quad
	Let \,$\tau_{x} \,\in\, \mathscr{\varphi}(\mathscr{F})_{x}$,\, and
	let \,$\widetilde{\tau} \,\in\, \varphi(\mathscr{F})(U) \,:=\, \varphi_{U}\!\left(\,\overset{{\color{white}.}}{\mathscr{F}(U)}\,\right)$\,
	be a representative of \,$\tau_{x}$,\, where \,$U \,\ni\, x$.\,
	Then, \,$\widetilde{\tau}$
	\,$=$\, $\varphi_{U}(\,\sigma\,)$
	\,$\in$\, $\varphi_{U}\!\left(\,\overset{{\color{white}.}}{\mathscr{F}(U)}\,\right)$
	\,$=:$\, $\varphi(\mathscr{F})(U)$,\,
	for some \,$\sigma \,\in\, \mathscr{F}(U)$.\,
	Hence,
	\,$\tau_{x}$
	\,$=$\, $\left(\,\widetilde{\tau}\,\right)_{x}$
	\,$=$\, $\left(\,\overset{{\color{white}.}}{\varphi_{U}(\sigma)}\,\right)_{x}$
	\,$=$\, $\varphi_{x}(\,\sigma_{x}\,)$
	\,$\in$\, $\varphi_{x}\!\left(\,\mathscr{F}_{x}\,\right)$.\,
	This proves Claim 1.
	
	\vskip 0.5cm
	\textbf{Claim 2:}\;\; $\varphi(\mathscr{F})_{x} \;\supset\; \varphi_{x}(\,\mathscr{F}_{x}\,)$
	\vskip 0.2cm
	Proof of Claim 2:\quad
	Let \,$\varphi_{x}(\,\sigma_{x}\,) \,\in\, \varphi_{x}(\,\mathscr{F}_{x}\,)$.\,
	Let \,$\widetilde{\sigma} \,\in\, \mathscr{F}(U)$\, be representative of \,$\sigma_{x}$,\,
	where \,$U \,\ni\, x$.\,
	Then, since
	\,$\varphi_{U}(\,\widetilde{\sigma}\,)$
	\,$\in$\, $\varphi_{U}\!\left(\,\mathscr{F}(\overset{{\color{white}.}}{U})\,\right)$
	\,$=:$\, $\overset{{\color{white}.}}{\varphi(\mathscr{F})(U)}$,\,
	we see that
	\begin{equation*}
	\varphi_{x}(\,\sigma_{x}\,)
	\;\; = \;\;
		\left(\,\overset{{\color{white}1}}{\varphi}_{U}(\,\widetilde{\sigma}\,)\,\right)_{x}
	\;\; \in \;\;
		\overset{{\color{white}.}}{\varphi(\mathscr{F})}_{x}
	\end{equation*}
	This proves Claim 2.
	
	\vskip 0.3cm
	\noindent
	It follows immediately from Claim 1 and Claim 2 that
	\,$\varphi(\mathscr{F})_{x}$
	\,$=$\, $\varphi_{x}\!\left(\,\mathscr{F}_{x}\,\right)$
	\,$\subset$\, $\mathscr{G}_{x}$,\,
	for each \,$x \in X$.
	\qed
\end{enumerate}

          %%%%% ~~~~~~~~~~~~~~~~~~~~ %%%%%
          %%%%% ~~~~~~~~~~~~~~~~~~~~ %%%%%

\vskip 1.0cm
\subsection{The sheafification of a presheaf}
\setcounter{theorem}{0}
\setcounter{equation}{0}

\begin{definition}
\mbox{}\vskip 0.1cm
\noindent
Suppose:
\begin{itemize}
\item
	$\mathscr{P}$\, is a presheaf over a topological space \,$X$,
\item
	$\mathscr{D}(\mathscr{P})$\, is the sheaf of discontinuous sections of \,$\mathscr{P}$,\,
\item
	$\mathscr{P}\,\overset{\iota}{\longrightarrow}\,\mathscr{D}(\mathscr{P})$\,
	is the discretization morphism of \,$\mathscr{P}$\, and
\item
	$\mathscr{P}^{\flat}$ \,$:=$\, $\iota(\mathscr{P})$\, is the image sub-presheaf of \,$\mathscr{P}$\,
	under its discretization morphism
	\,$\mathscr{P}\,\overset{\iota}{\longrightarrow}\,\mathscr{D}(\mathscr{P})$.\,
\end{itemize}
Then, the \textbf{sheafification} \,$\mathscr{P}^{\sharp}$\, of \,$\mathscr{P}$\,
is by definition the smallest sub-sheaf of $\mathscr{D}(\mathscr{P})$
containing the image sub-presheaf \,$\iota(\mathscr{P}) \,\subset\, \mathscr{D}(\mathscr{P})$.\,
\end{definition}

          %%%%% ~~~~~~~~~~~~~~~~~~~~ %%%%%

\vskip 0.5cm
\begin{theorem}[Properties of the sheafification of a presheaf]\label{PropertiesSheafification}
\mbox{}\vskip 0.1cm
\begin{enumerate}
\item
	Suppose \,$\mathscr{P}$\, is a presheaf over a topological space \,$X$.\,
	Then, the series of morphisms of presheaves
	\begin{equation*}
	\mathscr{P}
	\;\;\overset{\iota}{\longrightarrow}\;\;
		\mathscr{P}^{\flat}
	\;\;\subset\;\;
		\mathscr{P}^{\sharp}
	\;\;\subset\;\;
		\mathscr{D}(\mathscr{P})
	\end{equation*}
	induce respective isomorphisms between stalks
	\begin{equation*}
	\mathscr{P}_{x}
	\;\;\overset{\sim}{\longrightarrow}\;\;
		\mathscr{P}^{\flat}_{x}
	\;\;\overset{\sim}{\longrightarrow}\;\;
		\mathscr{P}^{\sharp}_{x}
	\;\;\overset{\sim}{\longrightarrow}\;\;
		\mathscr{D}(\mathscr{P})_{x}\,,
	\quad
	\textnormal{for each \,$x \,\in\, X$}
	\end{equation*}
\item
	Suppose \,$\mathscr{F}$\, and \,$\mathscr{G}$\, are presheaves over a topological space \,$X$,\,
	and \,$\varphi : \mathscr{F} \longrightarrow \mathscr{G}$\, is a morphism of presheaves.
	Then, there exist unique morphisms of presheaves
	\,$\varphi^{\flat} : \mathscr{F}^{\flat} \longrightarrow \mathscr{G}^{\flat}$\,
	and
	\,$\varphi^{\sharp} : \mathscr{F}^{\sharp} \longrightarrow \mathscr{G}^{\sharp}$\,
	such that the following diagram commutes:
	\begin{equation*}
	\begin{tikzcd}
	\mathscr{F} \arrow[r, "\iota"] \arrow[d, "\varphi"'] &
	\mathscr{F}^{\flat} \arrow[r, hook] \arrow[d, "\varphi^{\flat}"] &
	\mathscr{F}^{\sharp} \arrow[r, hook] \arrow[d, "\varphi^{\sharp}"] &
	\mathscr{D}(\mathscr{F}) \arrow[d, "\mathscr{D}(\varphi)"]
	\\
	\mathscr{G} \arrow[r, "\iota"'] &
	\mathscr{G}^{\flat} \arrow[r, hook] &
	\mathscr{G}^{\sharp} \arrow[r, hook] &
	\mathscr{D}(\mathscr{G}) &
	\end{tikzcd}
	\end{equation*}
\item
	Suppose \,$\mathscr{P}$\, is a presheaf over a topological space \,$X$.\,
	Then,
	\begin{equation*}
	\left(\,\id_{\mathscr{P}}\,\right)^{\flat}
	\;\; = \;\;
		\id_{\mathscr{P}^{\flat}}
	\quad\quad\textnormal{and}\quad\quad
	\left(\,\id_{\mathscr{P}}\,\right)^{\sharp}
	\;\; = \;\;
		\id_{\mathscr{P}^{\sharp}}
	\end{equation*}
\item
	Suppose \,$\mathscr{F}$, \,$\mathscr{G}$, \,$\mathscr{H}$ are presheaves over a topological space \,$X$,\,
	and
	\,$\varphi : \mathscr{F} \longrightarrow \mathscr{G}$\,,
	\,$\psi : \mathscr{G} \longrightarrow \mathscr{H}$\,
	are morphisms of presheaves.
	Then,
	\begin{equation*}
	(\,\psi \,\circ\, \varphi\,)^{\flat}
	\;\; = \;\;
		\psi^{\flat} \,\circ\, \varphi^{\flat}
	\quad\quad\textnormal{and}\quad\quad
	(\,\psi \,\circ\, \varphi\,)^{\sharp}
	\;\; = \;\;
		\psi^{\sharp} \,\circ\, \varphi^{\sharp}
	\end{equation*}
\end{enumerate}
\end{theorem}

          %%%%% ~~~~~~~~~~~~~~~~~~~~ %%%%%

\vskip 0.5cm
\begin{remark}[Local set-theoretic description of $\mathscr{P}^{\sharp}$]
\mbox{}\vskip 0.1cm
\noindent
\begin{eqnarray*}
\mathscr{P}^{\sharp}(U)
& := &
	\left\{\;\,
		\left(\,\overset{{\color{white}.}}{\sigma_{x}}\,\right)_{x \in U} \,\in\, \mathscr{D}(\mathscr{P})(U)
		\,\;\left\vert\;
			\begin{array}{c}
			\textnormal{there exists an open cover \,$U \,=\, \underset{\alpha}{\bigcup}\;U_{\alpha}$\, such that}
			\\
			\textnormal{$\left.\left(\overset{{\color{white}.}}{(\,\sigma_{x}\,)_{x \in U}}\right)\,\right\vert^{(\mathscr{S})}_{\,U_{\alpha}} \,\in\, \mathscr{P}(U_{\alpha})$\,,\, for each \,$\alpha$}
			\end{array}
			\right.
		\right\}
\\ & {\color{white}\textnormal{\large$1$}} &
\\
& = &
	\left\{\;\,
		\left(\,\overset{{\color{white}.}}{\sigma_{x}}\,\right)_{x \in U} \,\in\, \mathscr{D}(\mathscr{P})(U)
		\,\;\left\vert\;
			\begin{array}{c}
			\textnormal{there exists an open cover \,$U \,=\, \underset{\alpha}{\bigcup}\;U_{\alpha}$\, such that}
			\\
			\textnormal{$\left(\,\overset{{\color{white}.}}{\sigma_{x}}\,\right)_{\,U_{\alpha}} \,\in\, \mathscr{P}(U_{\alpha})$\,,\, for each \,$\alpha$}
			\end{array}
			\right.
		\right\}
\\ & {\color{white}\textnormal{\large$1$}} &
\\
& = &
	\left\{\;\;
		\left(\,\overset{{\color{white}.}}{\sigma_{x}}\,\right)_{x \in U} \,\in\, \mathscr{D}(\mathscr{P})(U)
		\;\,\left\vert\;
			\begin{array}{c}
			\textnormal{there exists an open cover \,$U \,=\, \underset{\alpha}{\bigcup}\;U_{\alpha}$,\, and{\color{white}2222.}}
			\\
			\textnormal{there exists \,$\widetilde{\sigma}_{\alpha} \,\in \mathscr{P}(U_{\alpha})$,\, for each \,$\alpha$,\, such that}
			\\
			\textnormal{$\overset{{\color{white}\textnormal{\normalsize$1$}}}{\sigma_{x}} \,=\, \left(\;\overset{{\color{white}.}}{\widetilde{\sigma}_{\alpha}}\,\right)_{x}$\,,\, for each \,$x \in U_{\alpha}$,\, for each $\alpha$}
			\end{array}
			\right.
		\,\right\}
\end{eqnarray*}
\end{remark}

          %%%%% ~~~~~~~~~~~~~~~~~~~~ %%%%%

\vskip 0.5cm
\proofof Proposition \ref{PropertiesSheafification}
\begin{enumerate}
\item
\item
\item
\item
\end{enumerate}
\qed

          %%%%% ~~~~~~~~~~~~~~~~~~~~ %%%%%
