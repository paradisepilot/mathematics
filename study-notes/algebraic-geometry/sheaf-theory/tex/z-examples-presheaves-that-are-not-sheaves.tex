
          %%%%% ~~~~~~~~~~~~~~~~~~~~ %%%%%

\section{Examples: presheaves that are not sheaves}

%\cite{vanDerVaart1996}
%\cite{Kosorok2008}

%\renewcommand{\theenumi}{\alph{enumi}}
%\renewcommand{\labelenumi}{\textnormal{(\theenumi)}$\;\;$}
\renewcommand{\theenumi}{\roman{enumi}}
\renewcommand{\labelenumi}{\textnormal{(\theenumi)}$\;\;$}

          %%%%% ~~~~~~~~~~~~~~~~~~~~ %%%%%

\begin{example}[A presheaf that is not decent, hence not a sheaf]
\mbox{}\vskip 0.1cm
\noindent
Let \,$X$\, be a two-point topological space \,$\{\,x,y\,\}$\, with the discrete topology.
Define a presheaf \,$\mathscr{F}$\, as follows:
\begin{itemize}
\item
	$
	\mathscr{F}(\varemptyset) \;=\; \{\,0\,\},
	\quad
	\mathscr{F}(\{\,x\,\}) \;=\; \Re,
	\quad
	\mathscr{F}(\{\,y\,\}) \;=\; \Re,
	\quad
	\mathscr{F}(\,X\,) \;=\; \mathscr{F}(\{\,x,y\,\}) \;=\; \Re \times \Re \times \Re
	$
\item
	The restriction map
	\,$\mathscr{F}(\{\,x,y\,\}) \longrightarrow \mathscr{F}(\{\,x\,\})$\,
	is the projection onto the first factor of
	\,$\mathscr{F}(\{\,x,y\,\}) \,=\, \Re \times \Re \times \Re$,\,
	while the restriction map
	\,$\mathscr{F}(\{\,x,y\,\}) \longrightarrow \mathscr{F}(\{\,y\,\})$\,
	is the projection onto the second factor of \,$\Re \times \Re \times \Re$.\,
\end{itemize}
It is straightforward to check that the above indeed defines a presheaf.
Now, consider the two sections
\begin{equation*}
s \;=\; (\,1,2,3\,), \quad s^{\prime} \;=\; (\,1,2,4\,) \;\;\in\;\; \mathscr{F}(\,X\,) \;=\; \mathscr{F}(\{\,x,y\,\})
\end{equation*}
Note that \,$s \neq s^{\prime}$,\, but \,$s$\, and \,$s^{\prime}$\, have the same restrictions:
\begin{equation*}
s\,\vert_{\varemptyset} \;=\; s^{\prime}\,\vert_{\varemptyset} \;=\; 0,
\quad\textnormal{and}\quad
s\,\vert_{\{x\}} \;=\; s^{\prime}\,\vert_{\{x\}} \;=\; 1,
\quad\textnormal{and}\quad
s\,\vert_{\{y\}} \;=\; s^{\prime}\,\vert_{\{y\}} \;=\; 2
\end{equation*}
This proves that \,$\mathscr{F}$\, is not a decent presheaf
(sections compatible with respect to restrictions fail to uniquely determine a section defined on the union of their respective domains).
In particular, \,$\mathscr{F}$\, is not a sheaf.
\end{example}

          %%%%% ~~~~~~~~~~~~~~~~~~~~ %%%%%

\vskip 0.5cm
\begin{example}[A presheaf that fails the gluing condition, hence not a sheaf]
\mbox{}\vskip 0.1cm
\noindent
Let \,$X = \Re$\, be the $one$-dimensional Euclidean space.
Define a presheaf \,$\mathscr{F}$\, as follows:
\begin{itemize}
\item
	$
	\mathscr{F}(\varemptyset) \;=\; \{\,0\,\},
	\quad
	\mathscr{F}(U) \;=\; \left\{\begin{array}{c}
		\textnormal{all bounded $\Re$-valued functions}
		\\
		\textnormal{defined on $U$}
		\end{array}\right\},
		\;\;\textnormal{for each open subset \,$U \subset \Re$}
	$
\item
	The restriction maps are the usual set-theoretic restriction maps.
\end{itemize}
It is straightforward to check that the above indeed defines a presheaf.
Now, consider the sequence of sections
\begin{equation*}
s_{n} \;=\; \textnormal{id}_{U_{n}},
\quad\textnormal{where}\quad 
U_{n} \;=\; (\,-n,n\,),
\quad\textnormal{for each \,$n = 1, 2, 3, \ldots$}
\end{equation*}
Clearly the sections \,$s_{n} \in \mathscr{F}(U_{n})$\, are compatible under restrictions.
However, there are no sections in \,$\mathscr{F}(X)$\, that restrict to \,$s_{n}$\, on \,$U_{n}$,\, for each \,$n = 1, 2, 3, \ldots\,$.\,
This proves that \,$\mathscr{F}$\, does not satisfy the gluing condition
(sections compatible with respect to restrictions fail to determine any section at all defined on the union of their respective domains).
In particular, \,$\mathscr{F}$\, is not a sheaf.
\end{example}

          %%%%% ~~~~~~~~~~~~~~~~~~~~ %%%%%
