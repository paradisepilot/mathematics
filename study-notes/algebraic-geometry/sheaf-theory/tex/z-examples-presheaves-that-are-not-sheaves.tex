
          %%%%% ~~~~~~~~~~~~~~~~~~~~ %%%%%

\section{Examples: presheaves that are not sheaves}

%\cite{vanDerVaart1996}
%\cite{Kosorok2008}

%\renewcommand{\theenumi}{\alph{enumi}}
%\renewcommand{\labelenumi}{\textnormal{(\theenumi)}$\;\;$}
\renewcommand{\theenumi}{\roman{enumi}}
\renewcommand{\labelenumi}{\textnormal{(\theenumi)}$\;\;$}

          %%%%% ~~~~~~~~~~~~~~~~~~~~ %%%%%

\begin{example}[A presheaf that is not decent, hence not a sheaf]
\mbox{}\vskip 0.1cm
\noindent
Let \,$X$\, be a two-point topological space \,$\{\,x,y\,\}$\, with the discrete topology.
Define a presheaf \,$\mathscr{F}$\, as follows:
\begin{itemize}
\item
	$
	\mathscr{F}(\varemptyset) \;=\; \{\,0\,\},
	\quad
	\mathscr{F}(\{\,x\,\}) \;=\; \Re,
	\quad
	\mathscr{F}(\{\,y\,\}) \;=\; \Re,
	\quad
	\mathscr{F}(\,X\,) \;=\; \mathscr{F}(\{\,x,y\,\}) \;=\; \Re \times \Re \times \Re
	$
\item
	The restriction map
	\,$\mathscr{F}(\{\,x,y\,\}) \longrightarrow \mathscr{F}(\{\,x\,\})$\,
	is the projection onto the first factor of
	\,$\mathscr{F}(\{\,x,y\,\}) \,=\, \Re \times \Re \times \Re$,\,
	while the restriction map
	\,$\mathscr{F}(\{\,x,y\,\}) \longrightarrow \mathscr{F}(\{\,y\,\})$\,
	is the projection onto the second factor of \,$\Re \times \Re \times \Re$.\,
\end{itemize}
It is straightforward to check that the above indeed defines a presheaf.
Now, consider the two sections
\begin{equation*}
s \;=\; (\,1,2,3\,), \quad s^{\prime} \;=\; (\,1,2,4\,) \;\;\in\;\; \mathscr{F}(\,X\,) \;=\; \mathscr{F}(\{\,x,y\,\})
\end{equation*}
Note that \,$s \neq s^{\prime}$,\, but \,$s$\, and \,$s^{\prime}$\, have the same restrictions:
\begin{equation*}
s\,\vert_{\varemptyset} \;=\; s^{\prime}\,\vert_{\varemptyset} \;=\; 0,
\quad\textnormal{and}\quad
s\,\vert_{\{x\}} \;=\; s^{\prime}\,\vert_{\{x\}} \;=\; 1,
\quad\textnormal{and}\quad
s\,\vert_{\{y\}} \;=\; s^{\prime}\,\vert_{\{y\}} \;=\; 2
\end{equation*}
This proves that \,$\mathscr{F}$\, is not a decent presheaf; in particular, it is not a sheaf.
\end{example}

          %%%%% ~~~~~~~~~~~~~~~~~~~~ %%%%%
