
          %%%%% ~~~~~~~~~~~~~~~~~~~~ %%%%%

\vskip 0.5cm
\section{Short exact sequences of sheaves of Abelian groups}

%\cite{vanDerVaart1996}
%\cite{Kosorok2008}

%\renewcommand{\theenumi}{\alph{enumi}}
%\renewcommand{\labelenumi}{\textnormal{(\theenumi)}$\;\;$}
\renewcommand{\theenumi}{\roman{enumi}}
\renewcommand{\labelenumi}{\textnormal{(\theenumi)}$\;\;$}

          %%%%% ~~~~~~~~~~~~~~~~~~~~ %%%%%

\subsection{Short exact sequences of sheaves of Abelian groups}
\setcounter{theorem}{0}
\setcounter{equation}{0}

\begin{definition}
\mbox{}\vskip 0.1cm
\noindent
Suppose:
\begin{itemize}
\item
	$\mathscr{F}$, $\mathscr{G}$, $\mathscr{H}$\, are sheaves of Abelian groups over a topological space $X$,
\item
	$\varphi : \mathscr{F}\longrightarrow\mathscr{G}$,\,
	$\psi : \mathscr{G}\longrightarrow\mathscr{H}$\,
	are morphisms of sheaves of abelian groups.
\end{itemize}
Then, the sequence of morphisms (of sheaves of Abelian groups)
\begin{equation*}
\begin{tikzcd}
0                 \arrow[r] &
\mathscr{F} \arrow[r, "\varphi"] &
\mathscr{G} \arrow[r, "\psi"] &
\mathscr{H} \arrow[r] &
0
\end{tikzcd}
\end{equation*}
is called a \textbf{short exact sequence of sheaves of Abelian groups}
if, for each $x \in X$, the induced sequence between stalks at $x$
\begin{equation*}
\begin{tikzcd}
0                 \arrow[r] &
\mathscr{F}_{x} \arrow[r, "\varphi_{x}"] &
\mathscr{G}_{x} \arrow[r, "\psi_{x}"] &
\mathscr{H}_{x} \arrow[r] &
0
\end{tikzcd}
\end{equation*}
is a short exact sequence of Abelian groups.
\end{definition}

          %%%%% ~~~~~~~~~~~~~~~~~~~~ %%%%%

\begin{theorem}
\mbox{}\vskip 0.1cm
\noindent
The sequence of morphisms (of sheaves of Abelian groups)
\begin{equation*}
\begin{tikzcd}
0                 \arrow[r] &
\mathscr{F} \arrow[r, "\varphi"] &
\mathscr{G} \arrow[r, "\psi"] &
\mathscr{H} \arrow[r] &
0
\end{tikzcd}
\end{equation*}
is a short exact sequence of sheaves of Abelian groups
if and only if the following two conditions are true:
\begin{itemize}
\item
	For each open subset \,$U \subset X$,\, the following is an exact sequence of Abelian groups:
	\begin{equation*}
	\begin{tikzcd}
	0                 \arrow[r] &
	\mathscr{F}(U) \arrow[r, "\varphi_{U}"] &
	\mathscr{G}(U) \arrow[r, "\psi_{U}"] &
	\mathscr{H}(U)
	\end{tikzcd}
	\end{equation*}
\item
	The sheaf \,$\mathscr{H}$ is isomorphic to the sheafification of the image sub-presheaf
	\,$\psi(\mathscr{G}) \subset \mathscr{H}$\,
	of \,$\mathscr{G}$\, under \,$\psi$;\, more precisely, there exists an isomorphism of sheaves of Abelian groups
	\begin{equation*}
	\begin{tikzcd}
	\mathscr{H} \arrow[r, "\sim"] & \left(\,\overset{{\color{white}.}}{\psi(\mathscr{G})}\,\right)^{\sharp}
	\end{tikzcd}
	\end{equation*}
	such that the following diagram commutes:
	\begin{equation*}
	\begin{tikzcd}
	\mathscr{G} \arrow[r, "\psi"] & \psi(\mathscr{G}) \arrow[r, hook] \arrow[dr] & \mathscr{H} \arrow[d, "\sim"]
	\\
	&& \left(\,\overset{{\color{white}.}}{\psi(\mathscr{G})}\,\right)^{\sharp} \arrow[r, hook] & \mathscr{D}\!\left(\,\overset{{\color{white}.}}{\psi(\mathscr{G})}\,\right)
	\end{tikzcd}
	\end{equation*}
\end{itemize}
\end{theorem}
\proof

\vskip 0.3cm
\noindent
\underline{(\,$\Longrightarrow$\,)}\quad
Suppose
\,$0 \longrightarrow \mathscr{F} \overset{\varphi}{\longrightarrow} \mathscr{G} \overset{\psi}{\longrightarrow} \mathscr{H} \longrightarrow 0$\,
is a short exact sequence of sheaves of Abelian groups.
Then, for each $x \in X$, the following is a short exact sequence of Abelian groups:
\,$0 \longrightarrow \mathscr{F}_{x} \overset{\varphi_{x}}{\longrightarrow} \mathscr{G}_{x} \overset{\psi_{x}}{\longrightarrow} \mathscr{H}_{x} \longrightarrow 0$.\,
Now, exactness at \,$\mathscr{H}$\, means precisely:
For each \,$\tau_{x} \in \mathscr{H}_{x}$,\, the following are true:
\begin{itemize}
\item
	there exists \,$\sigma_{x} \in \mathscr{G}_{x}$\, such that \,$\tau_{x} = \psi_{x}(\sigma_{x})$,
\item
	there exists open
	\,$U \ni x$,\, \,$\widetilde{\tau} \in \mathscr{H}(U)$,\, and \,$\widetilde{\sigma} \in \mathscr{G}(U)$\,
	such that
	\,$\tau_{x} \,=\, \left(\,\widetilde{\tau}\,\right)_{x},$\,
	\,$\widetilde{\tau} \,=\, \psi_{U}(\,\widetilde{\sigma}\,)$,\, and
	\,$\sigma_{x} \,=\, \left(\,\widetilde{\sigma}\,\right)_{x}$,\, hence
	\begin{equation*}
	\tau_{x}
	\;\; = \;\;
		\left(\,\widetilde{\tau}\,\right)_{x}
	\;\; = \;\;
		\left(\,\overset{{\color{white}.}}{\psi}_{U}(\,\widetilde{\sigma}\,)\,\right)_{x}
	\;\; = \;\;
		\psi_{x}\!\left(\,\widetilde{\sigma}_{x}\,\right)
	\end{equation*}
\end{itemize}
These observations imply:
\begin{eqnarray*}
\mathscr{H}(U)
& = &
	\mathscr{H}^{\sharp}(U)
\\ & {\color{white}\textnormal{\large$1$}} &
\\
& := &
	\left\{\;
		\left(\,\overset{{\color{white}.}}{\tau_{x}}\,\right)_{x \in U} \,\in\, {\color{white}1}\mathscr{D}(\mathscr{H})(U){\color{white}1}
		\;\,\left\vert\;
			\begin{array}{c}
			\textnormal{there exists an open cover \,$U \,=\, \underset{\alpha}{\bigcup}\;U_{\alpha}$,\, and{\color{white}2222.}}
			\\
			\textnormal{there exists \,$\widetilde{\tau}_{\alpha} \,\in \mathscr{H}(U_{\alpha})$,\, for each \,$\alpha$,\, such that}
			\\
			\textnormal{$\overset{{\color{white}\textnormal{\normalsize$1$}}}{\tau_{x}} \,=\, \left(\;\overset{{\color{white}.}}{\widetilde{\tau}_{\alpha}}\,\right)_{x}$\,,\, for each \,$x \in U_{\alpha}$,\, for each $\alpha$}
			\end{array}
			\right.
		\,\right\}
\\ & {\color{white}\textnormal{\large$1$}} &
\\
& = &
	\left\{\;
		\left(\,\overset{{\color{white}.}}{\tau_{x}}\,\right)_{x \in U} \,\in\, \mathscr{D}(\psi(\mathscr{G}))(U)
		\;\,\left\vert\;
			\begin{array}{c}
			\textnormal{there exists an open cover \,$U \,=\, \underset{\beta}{\bigcup}\;U_{\beta}$,\, and{\color{white}2222.}}
			\\
			\textnormal{there exists \,$\widetilde{\sigma}_{\beta} \,\in\, \mathscr{G}(U_{\beta})$,\, for each \,$\beta$,\, such that}
			\\
			\textnormal{$\overset{{\color{white}\textnormal{\normalsize$1$}}}{\tau_{x}} \,=\, \left(\;\overset{{\color{white}.}}{\psi_{U_{\beta}}(\,\widetilde{\sigma}_{\beta})}\,\right)_{x}$\,,\, for each \,$x \in U_{\beta}$,\, for each $\beta$}
			\end{array}
			\right.
		\,\right\}
\\ & {\color{white}\textnormal{\large$1$}} &
\\
& =: &
	\mathscr{D}(\psi(\mathscr{G}))^{\sharp}(U)
\end{eqnarray*}
\qed

          %%%%% ~~~~~~~~~~~~~~~~~~~~ %%%%%

\vskip 1.0cm
\begin{example}
\mbox{}\vskip 0.2cm
\noindent
Let
\begin{itemize}
\item
	$X = \Re^{2}$,\,
\item
	$C^{\infty}$ = sheaf over $X$ of infinitely smooth $\Re$-valued functions,\, and
\item
	$\Omega$ = sheaf over $X$ of infinitely smooth $1$-forms.
\end{itemize}
Define the morphism of sheaves
\,$d : C^{\infty} \longrightarrow \Omega$\,
as follows:
For each open subset $U \subset X$,
\begin{equation*}
\begin{array}{cccl}
d_{U} : & C^{\infty}(U) & \longrightarrow & \Omega(U)
\\
& \overset{{\color{white}\textnormal{\Large$1$}}}{f} & \longmapsto & \dfrac{\partial f}{\partial x}\,\d x +  \dfrac{\partial f}{\partial y}\,\d y
\end{array}
\end{equation*}
Then, the following statements hold:
\begin{enumerate}
\item
	The following is a short exact sequence of sheaves of Abelian groups:
	\begin{equation*}
	\begin{tikzcd}
	0                 \arrow[r] &
	\Re^{\sharp} \arrow[r, hook] &
	C^{\infty} \arrow[r, "d"] &
	\left(\,\overset{{\color{white}.}}{d(C^{\infty})}\,\right)^{\sharp} \arrow[r] &
	0
	\end{tikzcd}
	\end{equation*}
	where \,$\Re^{\sharp}$\, is the sheaf over $X$ of locally constant $\Re$-valued functions.
\item
	For each open subset \,$U \subset X$,\,
	\begin{equation*}
	\left(\,\overset{{\color{white}.}}{d(C^{\infty})}\,\right)^{\sharp}\!(U)
	\;\; = \;\;
		\textnormal{the set of all closed forms defined on $U$}
	\end{equation*}
\item
	For \,$U = X \,\backslash \{\,0\,\} = \Re^{2} \,\backslash \{\,0\,\}$,\,
	the map
	\begin{equation*}
	\begin{array}{cccl}
	d_{U} : & C^{\infty}(U) & \longrightarrow & \left(\,\overset{{\color{white}.}}{d(C^{\infty})}\,\right)^{\sharp}\!(U)
	\end{array}
	\end{equation*}
	is not surjective.
\end{enumerate}
\end{example}
\proof
\vskip 0.3cm
\begin{enumerate}
\item
\item
	By the definition of the sheafification of the image sub-presheaf \,$d(C^{\infty})$,\,
	\begin{eqnarray*}
	\left(\,\overset{{\color{white}.}}{d(C^{\infty})}\,\right)^{\sharp}\!(U)
	& := &
		\left\{\;
			\left(\,\overset{{\color{white}.}}{\tau_{x}}\,\right)_{x \in U} \,\in\, \mathscr{D}\!\left(\,\overset{{\color{white}.}}{d(C^{\infty})}\,\right)(U)
			\;\,\left\vert\;
				\begin{array}{c}
				\textnormal{there exist an open cover \,$U \,=\, \underset{\alpha}{\bigcup}\;U_{\alpha}$,\, and{\color{white}	2222.}}
				\\
				\textnormal{$\widetilde{\tau}_{\alpha} \,\in \left(\,\overset{{\color{white}.}}{d(C^{\infty})}\,\right)(U_{\alpha}) \,=\, d_{U_{\alpha}}\!\left(\,\overset{{\color{white}.}}{C^{\infty}(U_{\alpha})}\,\right)$\,, \,for each \,$\alpha$,}
				\\
				\textnormal{such that \,$\overset{{\color{white}\textnormal{\normalsize$1$}}}{\tau_{x}} \,=\, \left(\;\overset{{\color{white}.}}{\widetilde{\tau}_{\alpha}}\,\right)_{x}$\,,\, for each \,$x \in U_{\alpha}$,\, for each $\alpha$}
				\end{array}
				\right.
			\,\right\}
	\\ & {\color{white}\textnormal{\large$1$}} &
	\\
	& = &
		\left\{\;
			\left(\,\overset{{\color{white}.}}{\tau_{x}}\,\right)_{x \in U} \,\in\, \mathscr{D}\!\left(\,\overset{{\color{white}.}}{d(C^{\infty})}\,\right)(U)
			\;\,\left\vert\;
				\begin{array}{c}
				\textnormal{there exist an open cover \,$U \,=\, \underset{\alpha}{\bigcup}\;U_{\alpha}$,\, and{\color{white}	2222.}}
				\\
				\textnormal{$\widetilde{\sigma}_{\alpha} \,\in C^{\infty}(U_{\alpha})$\,, \,for each \,$\alpha$,\, such that}
				\\
				\textnormal{$\overset{{\color{white}\textnormal{\normalsize$1$}}}{\tau_{x}} \,=\, \left(\;\overset{{\color{white}.}}{d(\,\widetilde{\sigma}_{\alpha})}\,\right)_{x}$\,,\, for each \,$x \in U_{\alpha}$,\, for each $\alpha$}
				\end{array}
				\right.
			\,\right\}
	\end{eqnarray*}
	which shows that
	\begin{equation*}
	\left(\,\overset{{\color{white}.}}{d(C^{\infty})}\,\right)^{\sharp}\!(U)
	\;\; = \;\;
		\textnormal{the set of all locally exact (hence also closed) forms defined on $U$}
	\end{equation*}	
	Now, recall the Poincar\'{e} Lemma, which states that
	closed $p$-forms defined on open balls in $\Re^{n}$ are exact;
	in particular, closed $p$-form defined on open subsets of $\Re^{n}$ are therefore locally exact.
	We may now conclude:
	\begin{equation*}
	\left(\,\overset{{\color{white}.}}{d(C^{\infty})}\,\right)^{\sharp}\!(U)
	\;\; = \;\;
		\textnormal{the set of all closed forms defined on $U$}
	\end{equation*}	
\item
	Let \,$\omega \in \Omega(U)$\, be the $1$-form defined on \,$U \ni X \,\backslash \{\,0\,\}$ given by:
	\begin{equation*}
	\omega \;\; := \;\; \left(\dfrac{-\,y}{x^{2}+y^{2}}\right)\d x + \left(\dfrac{x}{x^{2}+y^{2}}\right)\d y
	\end{equation*}
	Then, \,$\omega$\, is in fact a closed $1$-form, i.e.
	 \,$\omega \in \left(\,\overset{{\color{white}.}}{d(C^{\infty})}\,\right)^{\sharp}\!(U)$.\,
	 Indeed,
	\begin{eqnarray*}
	\d\,\omega
	& = &
		\dfrac{\partial}{\partial y}\!\left(\dfrac{-\,y}{x^{2}+y^{2}}\right)\d y \wedge \d x
		\, + \,
		\dfrac{\partial}{\partial x}\!\left(\dfrac{x}{x^{2}+y^{2}}\right)\d x \wedge \d y
	\\
	& = &
		\left[\,\dfrac{-\,1}{x^{2}+y^{2}} - y\cdot(-1)\cdot\dfrac{2y}{(x^{2}+y^{2})^2}\,\right]\left(- \,\d x \wedge \d y\right)
		\; + \;
		\left[\,\dfrac{1}{x^{2}+y^{2}} + x\cdot(-1)\cdot\dfrac{2x}{(x^{2}+y^{2})^2}\,\right]\d x \wedge \d y
	\\
	& = &
		\left[\;
			\dfrac{1}{x^{2}+y^{2}} - \dfrac{2y^{2}}{(x^{2}+y^{2})^2}
			\; + \;
			\dfrac{1}{x^{2}+y^{2}} - \dfrac{2x^{2}}{(x^{2}+y^{2})^2}
			\,\right]\,\d x \wedge \d y
	\\
	& = &
		\left[\;
			\dfrac{2}{x^{2}+y^{2}} - \dfrac{2(x^{2}+y^{2})}{(x^{2}+y^{2})^2}
			\,\right]\,\d x \wedge \d y
	\\
	& \overset{{\color{white}\textnormal{\Large$1$}}}{=} &
		0
	\end{eqnarray*}
	However, \,$\omega$\, is not an exact form.
	Suppose on the contrary that \,$\omega$\, were an exact form, i.e.
	there exists \,$f \in C^{\infty}(U)$\, such that
	\begin{equation*}
	\omega
	\;\; = \;\;
		\d\,f
	\;\; = \;\;
		\dfrac{\partial f}{\partial x}\,\d\,x \; + \; \dfrac{\partial f}{\partial y}\,\d\,y
	\end{equation*}
	Then, integrating \,$\omega$\, along the parametrized closed circle of radius $r > 0$
	centred at $0 \in \Re^{2}$
	\begin{equation*}
	\Gamma
	\; : \;
		\left(\;x(t)\,\overset{{\color{white}1}}{,}\, y(t)\;\right)
	\;\; = \;\;
		\left(\;r \cos t \,\overset{{\color{white}1}}{,}\, r \sin t\;\right),
	\quad
	\textnormal{for \,$t \in [\,0,2\pi\,]$}
	\end{equation*}
	would have yielded zero, since
	\begin{eqnarray*}
	\displaystyle\int_{\,\Gamma}\;\omega
	& = &
		\displaystyle\int_{\,\Gamma}\;\d\,f
	\;\; = \;\;
		f\!\left(\,\overset{{\color{white}}}{\Gamma(2\pi)}\,\right)
		\; - \;
		f\!\left(\,\overset{{\color{white}}}{\Gamma(0)}\,\right)
	\;\; = \;\;
		f\!\left(\,\overset{{\color{white}}}{(1,0)}\,\right)
		\; - \;
		f\!\left(\,\overset{{\color{white}}}{(1,0)}\,\right)
	\;\; = \;\;
		0
	\end{eqnarray*}
	On the other hand, explicit calculations shows:
	\begin{eqnarray*}
	\displaystyle\int_{\,\Gamma}\;\omega
	& = &
		%\ointctrclockwise\;
		\displaystyle\int_{\,\Gamma}\;
			\left(\dfrac{-\,y}{x^{2}+y^{2}}\right)\d\,x
			\;+\;
			\left(\dfrac{x}{x^{2}+y^{2}}\right)\d\,y
	\;\; = \;\;
		\int_{0}^{2\pi}\,
			\left[\;
			\left(\dfrac{-\,y}{x^{2}+y^{2}}\right) \dot{x}(t)
			\; + \;
			\left(\dfrac{x}{x^{2}+y^{2}}\right) \dot{y}(t)
			\,\right]
			\;\d\,t
	\\
	& \overset{{\color{white}\textnormal{\Huge$1$}}}{=} &
		\int_{0}^{2\pi}\,
			\left[\;
			\left(\,\dfrac{-\,r\sin t}{r^{2}}\,\right) \cdot (\,-r \sin t\,)
			\; + \;
			\left(\,\dfrac{r \cos t}{r^{2}}\,\right) \cdot (\,r \cos t\,)
			\,\right]
			\;\d\,t
	\;\; = \;\;
		\int_{0}^{2\pi}\,
			\left[\, \dfrac{r^{2}\cdot(\sin^{2}t+\cos^{2}t)}{r^{2}} \,\right]
			\;\d\,t
	\\
	& \overset{{\color{white}\textnormal{\Huge$1$}}}{=} &
		\int_{0}^{2\pi}\,
			\left[\;\overset{{\color{white}.}}{1}\;\right]
			\;\d\,t
	\\
	& \overset{{\color{white}\textnormal{\Huge$1$}}}{=} &
		2\,\pi
	\;\; \neq \;\;
		0
	\end{eqnarray*}
	We thus see that the closed $1$-form
	\,$\omega \,\in\, \left(\,\overset{{\color{white}.}}{d(C^{\infty})}\,\right)^{\sharp}\!(U)$\,
	cannot be exact.
	This proves that, for \,$U = X \,\backslash \{\,0\,\} = \Re^{2} \,\backslash \{\,0\,\}$,\,
	the map
	\begin{equation*}
	\begin{array}{cccl}
	d_{U} : & C^{\infty}(U) & \longrightarrow & \left(\,\overset{{\color{white}.}}{d(C^{\infty})}\,\right)^{\sharp}\!(U)
	\end{array}
	\end{equation*}
	is not surjective.
	\qed
\end{enumerate}

          %%%%% ~~~~~~~~~~~~~~~~~~~~ %%%%%
