
          %%%%% ~~~~~~~~~~~~~~~~~~~~ %%%%%

\section{Abelian categories}
\setcounter{theorem}{0}
\setcounter{equation}{0}

%\cite{vanDerVaart1996}
%\cite{Kosorok2008}

%\renewcommand{\theenumi}{\alph{enumi}}
%\renewcommand{\labelenumi}{\textnormal{(\theenumi)}$\;\;$}
\renewcommand{\theenumi}{\roman{enumi}}
\renewcommand{\labelenumi}{\textnormal{(\theenumi)}$\;\;$}

          %%%%% ~~~~~~~~~~~~~~~~~~~~ %%%%%

\begin{definition}[Abelian category]
\mbox{}
\vskip 0.15cm
\noindent
An additive category \,$\mathfrak{A}$\, is said to be \textbf{abelian}
if it furthermore satisfies each of the following conditions:
\begin{itemize}
\item
	finite products exist in \,$\mathfrak{A}$\,,
\item
	every morphism in \,$\mathfrak{A}$\, has a kernel as well as a cokernel,
\item
	every monomorphism in \,$\mathfrak{A}$\, is a kernel, and
\item
	every epimorphism in \,$\mathfrak{A}$\, is a cokernel.
\end{itemize}
\end{definition}

          %%%%% ~~~~~~~~~~~~~~~~~~~~ %%%%%

% \vskip 0.5cm
% \begin{lemma}[Lemma 5.4, p.203, \cite{Grillet2007}]
% \label{KernelCokernelDuality}
% \mbox{}
% \vskip 0.1cm
% \noindent

% \end{lemma}

          %%%%% ~~~~~~~~~~~~~~~~~~~~ %%%%%

% [Proposition 1.2.9, p.24, \cite{Lal2021Algebra3}; Theorem 2.12, p.37, \cite{freyd1964abelian}]
\vskip 0.5cm
\begin{proposition}[Proposition 5.5, p.203, \cite{Grillet2007}]
\label{AbelianImpliesIsomorphismIFFMonoEpi}
\mbox{}
\vskip 0.1cm
\noindent
In each abelian category, a morphism is an isomorphism if and only if
it is both a monomorphism and an epimorphism.
\end{proposition}
\proof
\vskip 0.1cm
\noindent
\underline{($\Longrightarrow$)}
\vskip 0.2cm
\noindent
Suppose \,$f \in \Mor_{\mathfrak{A}}(A,B)$\, is an isomorphism; thus, there exists some
\,$f^{-1} \in \Mor_{\mathfrak{A}}(B,A)$\,
such that
\,$f \circ f^{-1} = 1_{B}$\, and \,$f^{-1} \circ f = 1_{A}$.\,
First, we prove that \,$f$\, is a monomorphism.
So, we suppose
\,$g, h \in \Mor_{\mathfrak{C}}(C,A)$\,
are such that
\,$f \circ g = f \circ h \in \Mor_{\mathfrak{A}}(C,B)$,\,
and we need to prove that \,$g = h \in \Mor_{\mathfrak{A}}(C,A)$.\,
To this end, simply observe:
\begin{equation*}
g
\; = \;
	1_{A} \circ g
\; = \;
	(f^{-1} \circ f) \circ g
\; = \;
	f^{-1} \circ (f \circ g)
\; = \;
	f^{-1} \circ (f \circ h)
\; = \;
	(f^{-1} \circ f) \circ h
\; = \;
	1_{A} \circ h
\; = \;
	h
\end{equation*}
This proves that \,$f$\, is indeed a monomorphism.
Next, we establish that \,$f$\, is an epimorphism.
So, we suppose
\,$g, h \in \Mor_{\mathfrak{A}}(B,C)$\,
are such that
\,$g \circ f = h \circ f \in \Mor_{\mathfrak{A}}(A,C)$,\,
and we need to prove
\,$g = h$.\,
To this end, simply observe:
\begin{equation*}
g
\; = \;
	g \circ 1_{B}
\; = \;
	g \circ (f \circ f^{-1})
\; = \;
	(g \circ f) \circ f^{-1}
\; = \;
	(h \circ f) \circ f^{-1}
\; = \;
	h \circ (f \circ f^{-1})
\; = \;
	h \circ 1_{B}
\; = \;
	h
\end{equation*}
This completes the proof that \,$f$\, is indeed both a monomorphism and an epimorphism.

\vskip 0.3cm
\noindent
\underline{($\Longleftarrow$)}
\vskip 0.2cm
\noindent
Conversely, suppose \,$f \in \Mor_{\mathfrak{A}}(A,B)$\, is both a monomorphism and an epimorphism.
Since \,$\mathfrak{A}$\, is an abelian category, the morphism \,$f$\, admits a cokernel
\,$\pi \in \Mor_{\mathfrak{A}}(B,Q)$.\,

\vskip 0.3cm
\noindent
\textbf{Claim 1:}\quad $\pi \,=\, 0_{B,Q}$
\vskip 0.1cm
\noindent
Proof of Claim 1:\;\;
$\pi$\, being a cokernel of \,$f$\, implies
\,$\pi \circ f = 0_{A,Q} = 0_{B,Q} \circ f$.\,
Since \,$f$\, is an epimorphism (i.e., ``right-cancellable''), it follows that
\,$\pi = 0_{B,Q}$.\,
This proves Claim 1.

\vskip 0.3cm
\noindent
\textbf{Claim 2:}\quad
There exists \,$\theta \,\in\, \Mor_{\mathfrak{A}}(B,A)$\, such that \,$f \circ \theta \,=\, 1_{B}$
\vskip 0.1cm
\noindent
Proof of Claim 2:\;\;
Since \,$\mathfrak{A}$\, is an abelian category,
the monomorphism \,$f$\, is a kernel of some morphism.
Hence, by Lemma \ref{AKernelsTheKernelOfItsOwnCokernel} (see also Lemma 5.4, p.203, \cite{Grillet2007}),
the monomorphism \,$f$\, is a kernel of its own cokernel \,$\pi$.\,
Now, by Claim 1, we have:
\,$\pi \,\circ\, 1_{B} \,=\, 0_{B,Q} \,\circ\, 1_{B} \,=\, 0_{B,Q}$,\,
which in turn implies
-- by the universal property of $f$ as kernel of $\pi$ --
that there exists a unique 
\,$\theta \in \Mor_{\mathfrak{A}}(B,A)$\,
such that
\,$1_{B} \,=\, f \circ \theta$.\,
See the following diagram for the universality of \,$f$\, as a kernel of \,$\pi$:
% Then αβα = α = α 1A and βα = 1A.
\begin{center}
\begin{tikzcd}
{\color{blue}B}
	\arrow[dd, thick, dashed, swap, "\exists !\,\theta{\color{white}.}", blue]
	\arrow[dr, thick, swap, "1_{B}\!\!", blue]
	\arrow[drrr, bend left = 15, "0", blue] & & &
\\
& {\color{red}B}
	\arrow[rr, swap, "\pi\,=\,0_{B,Q}{\color{white}...}"] & & Q
\\
{\color{red}A}
	\arrow[ur, swap, "f", red]
	\arrow[urrr, bend right = 15, swap, "0", gray]
\end{tikzcd}
\end{center}
This proves Claim 2.

\vskip 0.3cm
\noindent
\textbf{Claim 3:}\quad
$\theta \circ f \,=\, 1_{A}$
\vskip 0.1cm
\noindent
Proof of Claim 3:\;\;
Note that
\,$f \,\circ\, (\theta \,\circ\, f) \,=\, (f \,\circ\, \theta) \,\circ\, f \,=\, 1_{B} \,\circ\, f \,=\, f \,=\, f \,\circ\, 1_A$,\,
which implies
\,$\theta \circ f \,=\, 1_{A}$,\,
since
\,$f$\, is a monomorphism (i.e., ``left-cancellable'').
This proves Claim 3.

\vskip 0.3cm
\noindent
The Proposition now follows from Claim 2 and Claim 3.
\qed

          %%%%% ~~~~~~~~~~~~~~~~~~~~ %%%%%

\vskip 0.5cm
\begin{proposition}\label{MonomorphicFirstFactorInExactSequenceIsKernelOfSecondFactor}
\mbox{}
\vskip 0.1cm
\noindent
Suppose \,$\mathfrak{A}$\, is an abelian category,
$f \in \Mor_{\mathfrak{A}}(A,B)$, and $g \in \Mor_{\mathfrak{A}}(B,C)$
such that
\begin{itemize}
\item
	$f$\, is a monomorphism,
\item
	$g \circ f = 0$,\, and
\item
	$A \overset{f}{\longrightarrow} B \overset{g}{\longrightarrow} C$\,
	is exact.
\end{itemize}
Then, \,$f$\, is a kernel of \,$g$.
\end{proposition}
\proof
The exactness of
\,$A \overset{f}{\longrightarrow} B \overset{g}{\longrightarrow} C$\,
means that the unique morphism \,$\theta$\, in the following diagram is an isomorphism:
	\begin{center}
	\begin{tikzcd}
	&& && Q_{f}
		\arrow[dd, dashed, "\psi"]
	\\ \\
	A
		\arrow[rr, "f"]
		\arrow[dd, dashed, two heads, swap, "\varepsilon_{f}"]
		\arrow[rrrr, bend left = 30, "0", gray]
	&&
	B
		\arrow[rr, "g"]
		\arrow[uurr, two heads, "\pi_{f}"]
	&&
	C
	\\ \\
	I_{f}
		\arrow[uurr, hook, "\kappa(\pi_{f})"]
		\arrow[rr, thick, dashed, swap, "\exists !\,\theta", red]
	&&
	K_{g}
		\arrow[uu, hook, swap, "\kappa_{g}"]
		%\arrow[rr, thick, two heads, swap, "\pi_{\theta}", red]
	&&
	%Q_{\theta}
	\end{tikzcd}
	\end{center}
Since \,$\mathfrak{A}$\, is an abelian category, it in particular has cokernels and is normal.
By Proposition \ref{FactorizationIntoImageCoimage}, we see that
\,$\kappa(\pi_{f})$\, is an image of \,$f$.\,
Since \,$\mathfrak{A}$\, is an abelian category, it in particular is normal and has cokernels and equalizers.
By Proposition \ref{FactorizationIntoImageCoimage} again, we see that
\,$\varepsilon_{f}$\, is a coimage of \,$f$, which we denote by \,$\widetilde{f}$\, from now on.
By Lemma \ref{fTildeIsMonomorphismWheneverFIs},
since \,$f$\, is a monomorphism, it follows that \,$\varepsilon_{f} \,=\, \widetilde{f}$\,
is itself a monomorphism.
Thus, 
\,$\varepsilon_{f} \,=\, \widetilde{f}$\,
is both a monomorphism and an epimorphism.
By Proposition \ref{AbelianImpliesIsomorphismIFFMonoEpi}, we see that
\,$\varepsilon_{f} \,=\, \widetilde{f}$\,
is an isomorphism.
Thus,
\,$f \,=\, \kappa_{g} \,\circ\, \theta \,\circ\, \varepsilon_{f}$,\,
where \,$\theta$\, and \,$\varepsilon_{f}$\, are isomorphisms.
Since \,$\kappa_{g}$\, is a kernel of \,$g$,\,
we may now conclude that \,$f$\, is also a kernel of \,$g$.\, 
This completes the proof of the Proposition.
\qed

          %%%%% ~~~~~~~~~~~~~~~~~~~~ %%%%%
