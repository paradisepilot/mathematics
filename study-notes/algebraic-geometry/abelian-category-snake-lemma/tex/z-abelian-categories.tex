
          %%%%% ~~~~~~~~~~~~~~~~~~~~ %%%%%

\section{Abelian categories}
\setcounter{theorem}{0}
\setcounter{equation}{0}

%\cite{vanDerVaart1996}
%\cite{Kosorok2008}

%\renewcommand{\theenumi}{\alph{enumi}}
%\renewcommand{\labelenumi}{\textnormal{(\theenumi)}$\;\;$}
\renewcommand{\theenumi}{\roman{enumi}}
\renewcommand{\labelenumi}{\textnormal{(\theenumi)}$\;\;$}

          %%%%% ~~~~~~~~~~~~~~~~~~~~ %%%%%

\begin{definition}[Abelian category]
\mbox{}
\vskip 0.15cm
\noindent
An additive category \,$\mathfrak{A}$\, is said to be \textbf{abelian}
if it furthermore satisfies each of the following conditions:
\begin{itemize}
\item
	finite products exist in \,$\mathfrak{A}$\,,
\item
	every morphism in \,$\mathfrak{A}$\, has a kernel as well as a cokernel,
\item
	every monomorphism in \,$\mathfrak{A}$\, is a kernel, and
\item
	every epimorphism in \,$\mathfrak{A}$\, is a cokernel.
\end{itemize}
\end{definition}

          %%%%% ~~~~~~~~~~~~~~~~~~~~ %%%%%

% \vskip 0.5cm
% \begin{lemma}[Lemma 5.4, p.203, \cite{Grillet2007}]
% \label{KernelCokernelDuality}
% \mbox{}
% \vskip 0.1cm
% \noindent

% \end{lemma}

          %%%%% ~~~~~~~~~~~~~~~~~~~~ %%%%%

% [Proposition 1.2.9, p.24, \cite{Lal2021Algebra3}; Theorem 2.12, p.37, \cite{freyd1964abelian}]
\vskip 0.5cm
\begin{proposition}[Proposition 5.5, p.203, \cite{Grillet2007}]
\label{AbelianImpliesIsomorphismIFFMonoEpi}
\mbox{}
\vskip 0.1cm
\noindent
In each abelian category, a morphism is an isomorphism if and only if
it is both a monomorphism and an epimorphism.
\end{proposition}
\proof
\vskip 0.1cm
\noindent
\underline{($\Longrightarrow$)}
\vskip 0.2cm
\noindent
Suppose \,$f \in \Mor_{\mathfrak{A}}(A,B)$\, is an isomorphism; thus, there exists some
\,$f^{-1} \in \Mor_{\mathfrak{A}}(B,A)$\,
such that
\,$f \circ f^{-1} = 1_{B}$\, and \,$f^{-1} \circ f = 1_{A}$.\,
First, we prove that \,$f$\, is a monomorphism.
So, we suppose
\,$g, h \in \Mor_{\mathfrak{C}}(C,A)$\,
are such that
\,$f \circ g = f \circ h \in \Mor_{\mathfrak{A}}(C,B)$,\,
and we need to prove that \,$g = h \in \Mor_{\mathfrak{A}}(C,A)$.\,
To this end, simply observe:
\begin{equation*}
g
\; = \;
	1_{A} \circ g
\; = \;
	(f^{-1} \circ f) \circ g
\; = \;
	f^{-1} \circ (f \circ g)
\; = \;
	f^{-1} \circ (f \circ h)
\; = \;
	(f^{-1} \circ f) \circ h
\; = \;
	1_{A} \circ h
\; = \;
	h
\end{equation*}
This proves that \,$f$\, is indeed a monomorphism.
Next, we establish that \,$f$\, is an epimorphism.
So, we suppose
\,$g, h \in \Mor_{\mathfrak{A}}(B,C)$\,
are such that
\,$g \circ f = h \circ f \in \Mor_{\mathfrak{A}}(A,C)$,\,
and we need to prove
\,$g = h$.\,
To this end, simply observe:
\begin{equation*}
g
\; = \;
	g \circ 1_{B}
\; = \;
	g \circ (f \circ f^{-1})
\; = \;
	(g \circ f) \circ f^{-1}
\; = \;
	(h \circ f) \circ f^{-1}
\; = \;
	h \circ (f \circ f^{-1})
\; = \;
	h \circ 1_{B}
\; = \;
	h
\end{equation*}
This completes the proof that \,$f$\, is indeed both a monomorphism and an epimorphism.

\vskip 0.3cm
\noindent
\underline{($\Longleftarrow$)}
\vskip 0.2cm
\noindent
Conversely, suppose \,$f \in \Mor_{\mathfrak{A}}(A,B)$\, is both a monomorphism and an epimorphism.
Since \,$\mathfrak{A}$\, is an abelian category, the morphism \,$f$\, admits a cokernel
\,$\pi \in \Mor_{\mathfrak{A}}(B,Q)$.\,

\vskip 0.3cm
\noindent
\textbf{Claim 1:}\quad $\pi \,=\, 0_{B,Q}$
\vskip 0.1cm
\noindent
Proof of Claim 1:\;\;
$\pi$\, being a cokernel of \,$f$\, implies
\,$\pi \circ f = 0_{A,Q} = 0_{B,Q} \circ f$.\,
Since \,$f$\, is an epimorphism (i.e., ``right-cancellable''), it follows that
\,$\pi = 0_{B,Q}$.\,
This proves Claim 1.

\vskip 0.3cm
\noindent
\textbf{Claim 2:}\quad
There exists \,$\theta \,\in\, \Mor_{\mathfrak{A}}(B,A)$\, such that \,$f \circ \theta \,=\, 1_{B}$
\vskip 0.1cm
\noindent
Proof of Claim 2:\;\;
Since \,$\mathfrak{A}$\, is an abelian category,
the monomorphism \,$f$\, is a kernel of some morphism.
Hence, by Lemma \ref{AKernelsTheKernelOfItsOwnCokernel} (see also Lemma 5.4, p.203, \cite{Grillet2007}),
the monomorphism \,$f$\, is a kernel of its own cokernel \,$\pi$.\,
Now, by Claim 1, we have:
\,$\pi \,\circ\, 1_{B} \,=\, 0_{B,Q} \,\circ\, 1_{B} \,=\, 0_{B,Q}$,\,
which in turn implies
-- by the universal property of $f$ as kernel of $\pi$ --
that there exists a unique 
\,$\theta \in \Mor_{\mathfrak{A}}(B,A)$\,
such that
\,$1_{B} \,=\, f \circ \theta$.\,
See the following diagram for the universality of \,$f$\, as a kernel of \,$\pi$:
% Then αβα = α = α 1A and βα = 1A.
\begin{center}
\begin{tikzcd}
{\color{blue}B}
	\arrow[dd, thick, dashed, swap, "\exists !\,\theta{\color{white}.}", blue]
	\arrow[dr, thick, swap, "1_{B}\!\!", blue]
	\arrow[drrr, bend left = 15, "0", blue] & & &
\\
& {\color{red}B}
	\arrow[rr, swap, "\pi\,=\,0_{B,Q}{\color{white}...}"] & & Q
\\
{\color{red}A}
	\arrow[ur, swap, "f", red]
	\arrow[urrr, bend right = 15, swap, "0", gray]
\end{tikzcd}
\end{center}
This proves Claim 2.

\vskip 0.3cm
\noindent
\textbf{Claim 3:}\quad
$\theta \circ f \,=\, 1_{A}$
\vskip 0.1cm
\noindent
Proof of Claim 3:\;\;
Note that
\,$f \,\circ\, (\theta \,\circ\, f) \,=\, (f \,\circ\, \theta) \,\circ\, f \,=\, 1_{B} \,\circ\, f \,=\, f \,=\, f \,\circ\, 1_A$,\,
which implies
\,$\theta \circ f \,=\, 1_{A}$,\,
since
\,$f$\, is a monomorphism (i.e., ``left-cancellable'').
This proves Claim 3.

\vskip 0.3cm
\noindent
The Proposition now follows from Claim 2 and Claim 3.
\qed

          %%%%% ~~~~~~~~~~~~~~~~~~~~ %%%%%

\vskip 0.5cm
\begin{proposition}\label{MonomorphicFirstFactorInExactSequenceIsKernelOfSecondFactor}
\mbox{}
\vskip 0.1cm
\noindent
Suppose \,$\mathfrak{A}$\, is an abelian category,
$f \in \Mor_{\mathfrak{A}}(A,B)$, and $g \in \Mor_{\mathfrak{A}}(B,C)$
such that
\begin{itemize}
\item
	$f$\, is a monomorphism,
\item
	$g \circ f = 0$,\, and
\item
	$A \overset{f}{\longrightarrow} B \overset{g}{\longrightarrow} C$\,
	is exact.
\end{itemize}
Then, \,$f$\, is a kernel of \,$g$.
\end{proposition}
\proof
The exactness of
\,$A \overset{f}{\longrightarrow} B \overset{g}{\longrightarrow} C$\,
means that the unique morphism \,$\theta$\, in the following diagram is an isomorphism:
\begin{center}
\begin{tikzcd}
&& && Q_{f}
	\arrow[dd, dashed, "\psi"]
\\ \\
A
	\arrow[rr, "f"]
	\arrow[dd, dashed, two heads, swap, "\varepsilon_{f}"]
	\arrow[rrrr, bend left = 30, "0", gray]
&&
B
	\arrow[rr, "g"]
	\arrow[uurr, two heads, "\pi_{f}"]
&&
C
\\ \\
I_{f}
	\arrow[uurr, hook, "\kappa(\pi_{f})"]
	\arrow[rr, thick, dashed, swap, "\exists !\,\theta", red]
&&
K_{g}
	\arrow[uu, hook, swap, "\kappa_{g}"]
	%\arrow[rr, thick, two heads, swap, "\pi_{\theta}", red]
&&
%Q_{\theta}
\end{tikzcd}
\end{center}
Since \,$\mathfrak{A}$\, is an abelian category, it in particular has cokernels and is normal.
By Proposition \ref{FactorizationIntoImageCoimage}, we see that
\,$\kappa(\pi_{f})$\, is an image of \,$f$.\,
Since \,$\mathfrak{A}$\, is an abelian category, it in particular is normal and has cokernels and equalizers.
By Proposition \ref{FactorizationIntoImageCoimage} again, we see that
\,$\varepsilon_{f}$\, is a coimage of \,$f$, which we denote by \,$\widetilde{f}$\, from now on.
By Lemma \ref{fTildeIsMonomorphismWheneverFIs},
since \,$f$\, is a monomorphism, it follows that \,$\varepsilon_{f} \,=\, \widetilde{f}$\,
is itself a monomorphism.
Thus, 
\,$\varepsilon_{f} \,=\, \widetilde{f}$\,
is both a monomorphism and an epimorphism.
By Proposition \ref{AbelianImpliesIsomorphismIFFMonoEpi}, we see that
\,$\varepsilon_{f} \,=\, \widetilde{f}$\,
is an isomorphism.
Thus,
\,$f \,=\, \kappa_{g} \,\circ\, \theta \,\circ\, \varepsilon_{f}$,\,
where \,$\theta$\, and \,$\varepsilon_{f}$\, are isomorphisms.
Since \,$\kappa_{g}$\, is a kernel of \,$g$,\,
we may now conclude that \,$f$\, is also a kernel of \,$g$.\, 
This completes the proof of the Proposition.
\qed

          %%%%% ~~~~~~~~~~~~~~~~~~~~ %%%%%

\vskip 0.5cm
\begin{definition}[Fiber products \& fiber coproducts, p.46, \cite{kashiwara2005categories}]
\mbox{}
\vskip 0.15cm
\noindent
Let \,$\mathfrak{C}$\, be a category.
\begin{enumerate}
\item
	Let
	\,$f_{1} \in \Mor_{\mathfrak{C}}(X_{1},Y)$\,
	and
	\,$f_{2} \in \Mor_{\mathfrak{C}}(X_{2},Y)$,\,
	where
	\,$S,\, X_{1},\, X_{2} \in \Obj(\mathfrak{C})$.
	The \textbf{fiber product} of \,$f_{1},\, f_{2}$\, is a triple
	\,$\left(\,X_{1}\overset{{\color{white}.}}{\sqcap_{S}}\!X_{2} \,,\, \pi_{1} \,,\, \pi_{2}\,\right)$,\,
	where
	\,$X_{1}\sqcap_{Y}\!X_{2} \in \Obj(\mathfrak{C})$,\,
	\,$\pi_{1} \in \Mor_{\mathfrak{C}}(\,X_{1}\sqcap_{S}\!X_{2}\,,\,X_{1}\,)$,\,
	\,$\pi_{2} \in \Mor_{\mathfrak{C}}(\,X_{1}\sqcap_{S}\!X_{2}\,,\,X_{2}\,)$,\,
	such that
	\begin{itemize}
	\item
		$f_{1} \circ \pi_{1} \,=\, f_{2} \circ \pi_{2}$,\, and
	\item
		for any
		\,$h_{1} \in \Mor_{\mathfrak{C}}(A,X_{1})$\,
		and
		\,$h_{2} \in \Mor_{\mathfrak{C}}(A,X_{2})$\,
		satisfying
		\,$f_{1} \circ h_{1} \,=\, f_{2} \circ h_{2}$,\,
		there exists a unique
		\,$\theta \in \Mor_{\mathfrak{C}}(\,A\,,\,X_{1}\sqcap_{Y}\!X_{2}\,)$\,
		such that the following diagram commutes:
		\begin{center}
		\begin{tikzcd}
		&&&& {\color{blue}X_{1}}
			\arrow[ddrr, "f_{1}", blue]
		\\ \\
		A
			\arrow[rr, dashed, "\;\;\;\exists\,!\,\theta"]
			\arrow[uurrrr, bend left =  20, "h_{1}"]
			\arrow[ddrrrr, bend left = -20, "h_{2}", swap]
		&&
		{\color{red}X_{1} \sqcap_{Y}\! X_{2}}
			\arrow[uurr, "{\color{red}\pi_{1}}", red]
			\arrow[ddrr, "{\color{red}\pi_{2}}", red, swap]
		&&&&
		{\color{blue}Y}
		\\ \\
		&&&& {\color{blue}X_{2}}
			\arrow[uurr, "f_{2}", swap, blue]
		\end{tikzcd}
		\end{center}
	\end{itemize}
\item
	Dually, let
	\,$f_{1} \in \Mor_{\mathfrak{C}}(Y,X_{1})$\,
	and
	\,$f_{2} \in \Mor_{\mathfrak{C}}(Y,X_{2})$,\,
	where
	\,$S,\, X_{1},\, X_{2} \in \Obj(\mathfrak{C})$.
	The \textbf{fiber coproduct} of \,$f_{1},\, f_{2}$\, is a triple
	\,$\left(\,X_{1}\overset{{\color{white}.}}{\sqcup_{Y}}\!X_{2} \,,\, \iota_{1} \,,\, \iota_{2}\,\right)$,\,
	where
	\,$X_{1}\sqcup_{Y}\!X_{2} \in \Obj(\mathfrak{C})$,\,
	\,$\iota_{1} \in \Mor_{\mathfrak{C}}(\,X_{1}\,,X_{1}\sqcup_{Y}\!X_{2}\,)$,\,
	\,$\iota_{2} \in \Mor_{\mathfrak{C}}(\,X_{2}\,,X_{1}\sqcup_{Y}\!X_{2}\,)$,\,
	such that,
	\begin{itemize}
	\item
		$\iota_{1} \circ f_{1} \,=\, \iota_{2} \circ f_{2}$,\, and
	\item
		for any
		\,$h_{1} \in \Mor_{\mathfrak{C}}(X_{1},A)$\,
		and
		\,$h_{2} \in \Mor_{\mathfrak{C}}(X_{2},A)$\,
		satisfying
		\,$h_{1} \circ f_{1} \,=\, h_{2} \circ f_{2}$,\,
		there exists a unique
		\,$\theta \in \Mor_{\mathfrak{C}}(\,X_{1}\sqcup_{S}\!X_{2}\,,\,A\,)$\,
		such that the following diagram commutes:
		\begin{center}
		\begin{tikzcd}
		&&&& {\color{blue}X_{1}}
			\arrow[ddll, "{\color{red}\iota_{1}}", red, swap]
			\arrow[ddllll, bend left = -20, "h_{1}", swap]
		\\ \\
		A
		&&
		{\color{red}X_{1} \sqcup_{Y}\! X_{2}}
			\arrow[ll, dashed, "\;\;\;\exists\,!\,\theta", swap]
		&&&&
		{\color{blue}Y}
			\arrow[uull, "f_{1}", blue, swap]
			\arrow[ddll, "f_{2}", blue]
		\\ \\
		&&&& {\color{blue}X_{2}}
			\arrow[uull, "{\color{red}\iota_{2}}", red]
			\arrow[uullll, bend left = 20, "h_{2}"]
		\end{tikzcd}
		\end{center}
	\end{itemize}
\end{enumerate}
\end{definition}

          %%%%% ~~~~~~~~~~~~~~~~~~~~ %%%%%

\vskip 0.5cm
\begin{proposition}[p.176, \cite{kashiwara2005categories}]
\label{FiberProductFiberCoproductExistInAbelianCategories}
\mbox{}
\vskip 0.1cm
\noindent
In each abelian category,
fiber products and fiber coproducts always exist.
\end{proposition}
\proof
\begin{enumerate}
\item
	We first prove the existence of fiber products in an arbitrary abelian category.
	Suppose \,$\mathfrak{A}$\, is an abelian category,
	\,$f_{1} \in \Mor_{\mathfrak{A}}(X_{1},Y)$\,
	and
	\,$f_{2} \in \Mor_{\mathfrak{A}}(X_{1},Y)$.\,
	Since \,$\mathfrak{A}$\, is an abelian category, the finite {\color{red}biproduct}
	\begin{equation*}
	\left(\,
		X_{1} \overset{{\color{white}.}}{\oplus} X_{2}
		\,,\,
		p_{1}
		\,,\,
		p_{2}
		\,,\,
		\iota_{1}
		\,,\,
		\iota_{2}
		\,\right)
	\end{equation*}
	of \,$X_{1},\, X_{2}$\, exists, where
	\,$p_{1} : X_{1} \oplus X_{2} \longrightarrow X_{1}$,\,
	\,$p_{2} : X_{1} \oplus X_{2} \longrightarrow X_{2}$,\,
	\,$\iota_{1} : X_{1} \longrightarrow X_{1} \oplus X_{2}$,\,
	\,$\iota_{2} : X_{2} \longrightarrow X_{1} \oplus X_{2}$\,
	satisfy:
	\begin{equation*}
	p_{1} \circ \iota_{1} = 1_{X_{1}}\,,
	\quad
	p_{2} \circ \iota_{2} = 1_{X_{2}}\,,
	\quad
	p_{1} \circ \iota_{2} = 0_{X_{2},X_{1}}\,,
	\quad
	p_{2} \circ \iota_{1} = 0_{X_{1},X_{2}}
	\end{equation*}
	\vskip 0.3cm
	\noindent
	\textbf{Claim 1:}\quad
	Let \,$\kappa \,\in\, \Mor_{\mathfrak{A}}(K, X_{1}\oplus X_{2})$\,
	be any {\color{red}kernel of
	\,$\varphi\,:=\,f_{1}\,\circ\,p_{1}\,-\,f_{2}\,\circ\,p_{2}\,\in\,\Mor_{\mathfrak{A}}(X_{1} \oplus X_{2},Y)$}.\,
	Define
	\,$\pi_{1} := p_{1} \circ \kappa \in \Mor_{\mathfrak{A}}(K,X_{1})$\,
	and
	\,$\pi_{2} := p_{2} \circ \kappa \in \Mor_{\mathfrak{A}}(K,X_{2})$.\,
	Then,
	\,$\left(\,K\,,\,\pi_{1}\,,\,\pi_{2}\,\right)$\,
	is a fiber product of
	\,$f_{1}$\,
	and
	\,$f_{2}$.\,
	\vskip 0.2cm
	\noindent
	Proof of Claim 1:\;\;
	First, observe that:
	\begin{eqnarray*}
	f_{1} \circ \pi_{1} - f_{2} \circ \pi_{2}
	& = &
		f_{1} \circ (\,p_{1} \circ \kappa\,) - f_{2} \circ (\,p_{2} \circ \kappa\,)
	\;\; = \;\;
		(\,f_{1} \circ p_{1}\,) \circ \kappa - (\,f_{2} \circ p_{2}\,) \circ \kappa
	\\
	& = &
		(\,f_{1} \circ p_{1} - f_{2} \circ p_{2}\,) \circ \kappa
	\;\; = \;\;
		\varphi \circ \kappa
	\\
	& = &
		0
	\end{eqnarray*}
	Next, let
	\,$h_{1} \in \Mor_{\mathfrak{A}}(A,X_{1})$\,
	and
	\,$h_{2} \in \Mor_{\mathfrak{A}}(A,X_{2})$\,
	be such that
	\,$f_{1} \circ h_{1} \,=\, f_{2} \circ h_{2}$.\,
	Define
	\,$\psi\,:=\,\iota_{1}\circ h_{1}\,+\,\iota_{2}\circ h_{2}\,\in\,\Mor_{\mathfrak{A}}(A,X_{1}\oplus X_{2})$.\,
	Observe:
	\begin{eqnarray*}
	\varphi \circ \psi
	& = &
		\left(\,f_{1} \circ\, p_{1} \,\overset{{\color{white}.}}{-}\, f_{2} \circ\, p_{2}\,\right)
		\,\circ\,
		\left(\,\iota_{1} \circ h_{1} \,\overset{{\color{white}.}}{+}\, \iota_{2} \circ h_{2}\,\right)
	\\
	& = &
		f_{1} \,\circ\, p_{1} \,\circ\, \iota_{1} \,\circ\, h_{1}
		\; + \;
		f_{1} \,\circ\, p_{1} \,\circ\, \iota_{2} \,\circ\, h_{2}
		\; - \;
		f_{2} \,\circ\, p_{2} \,\circ\, \iota_{1} \,\circ\, h_{1}
		\; - \;
		f_{2} \,\circ\, p_{2} \,\circ\, \iota_{2} \,\circ\, h_{2}
	\\
	& \overset{{\color{white}1}}{=} &
		f_{1} \,\circ\, 1_{X_{1}} \,\circ\, h_{1}
		\; + \;
		f_{1} \circ\, 0_{X_{2},X_{1}} \,\circ\, h_{2}
		\; - \;
		f_{2} \circ\, 0_{X_{1},X_{2}} \,\circ\, h_{1}
		\; - \;
		f_{2} \,\circ\, 1_{X_{2}} \,\circ\, h_{2}
	\\
	& \overset{{\color{white}1}}{=} &
		f_{1} \,\circ\, h_{1}
		\; - \;
		f_{2} \,\circ\, h_{2}
	\;\; = \;\;
		0
	\end{eqnarray*}
	Thus, by the universal property of
	\,$\kappa \,\in\, \Mor_{\mathfrak{A}}(K,X_{1} \oplus X_{2})$\,
	as kernel of
	\,$\varphi\,:=\,f_{1}\,\circ\,p_{1}\,-\,f_{2}\,\circ\,p_{2}\,\in\,\Mor_{\mathfrak{A}}(X_{1} \oplus X_{2},Y)$,\,
	there exists a unique
	\,$\theta \in \Mor_{\mathfrak{A}}(A,K)$\,
	such that
	\,$\psi = \kappa \circ \theta$.\,
	Furthermore, \,$\theta$\, satisfies the following two equalities:
	%\vskip 0.1cm
	\begin{eqnarray*}
	\pi_{1} \circ \theta
	& = &
		(\,p_{1} \circ \kappa\,) \circ \theta
	\;\; = \;\;
		p_{1} \circ (\,\kappa \circ \theta\,)
	\;\; = \;\;
		p_{1} \circ \psi
	\;\; = \;\;
		p_{1} \circ (\,\iota_{1}\circ h_{1}\,+\,\iota_{2}\circ h_{2}\,)
	\\
	& = &
		(\,p_{1} \circ \iota_{1}\,) \circ h_{1}
		\,+\
		(\,p_{1} \circ \iota_{2}\,) \circ h_{2}
	\;\; = \;\;
		1_{X_{1}} \circ h_{1}
		\,+\
		0_{X_{2},X_{1}} \circ h_{2}
	\\
	& = &
		h_{1}
	\\ \\
	\pi_{2} \circ \theta
	& = &
		(\,p_{2} \circ \kappa\,) \circ \theta
	\;\; = \;\;
		p_{2} \circ (\,\kappa \circ \theta\,)
	\;\; = \;\;
		p_{2} \circ \psi
	\;\; = \;\;
		p_{2} \circ (\,\iota_{1}\circ h_{1}\,+\,\iota_{2}\circ h_{2}\,)
	\\
	& = &
		(\,p_{2} \circ \iota_{1}\,) \circ h_{1}
		\,+\
		(\,p_{2} \circ \iota_{2}\,) \circ h_{2}
	\;\; = \;\;
		0_{X_{1},X_{2}} \circ h_{1}
		\,+\,
		1_{X_{2}} \circ h_{2}
	\\
	& = &
		h_{2}
	\end{eqnarray*}
	% It remains to establish the uniqueness of \,$\theta$.\,
	% To this end, suppose
	% \,$\theta^{\prime} \in \Mor_{\mathfrak{A}}(A,K)$\,
	% is any morphism that satisfies
	% \begin{equation*}
	% \psi \,=\, \kappa \circ \theta^{\prime}\,,
	% \quad
	% \pi_{1} \circ \theta^{\prime} \,=\, h_{1}\,,
	% \quad
	% \pi_{2} \circ \theta^{\prime} \,=\, h_{2}\,,
	% \end{equation*}
	The above arguments are summarized in the following commutative diagram:
	\begin{center}
	\begin{tikzcd}
	&&&& {\color{blue}X_{1}}
		\arrow[ddrr, "f_{1}", blue]
		\arrow[ddl, bend left = 10, "\iota_{1}", hook, gray]
	\\ \\
	A
		\arrow[rr, dashed, "\;\;\;\exists\,!\,\theta"]
		\arrow[uurrrr, bend left =  20, "h_{1}"]
		\arrow[ddrrrr, bend left = -20, "h_{2}", swap]
		%\arrow[rrr, bend left = -20, "\psi\,:=\,\iota_{1}\circ h_{1} + \iota_{2}\circ h_{2}", swap]
		\arrow[rrr, bend left = -20, "\psi", swap]
	&&
	%{\color{red}X_{1} \sqcap_{Y}\! X_{2}}
	{\color{red}K}
		\arrow[r, "\kappa", hook, gray]
		\arrow[uurr, "{\color{red}\pi_{1}}", bend left =  10, red]
		\arrow[ddrr, "{\color{red}\pi_{2}}", bend left = -10, red, swap]
	&
	{\color{gray}X_{1} \oplus X_{2}}
		\arrow[rrr, "\varphi\,:=\,f_{1} \circ\, p_{1} \,-\, f_{2} \circ\, p_{2}", gray]
		\arrow[uur, "p_{1}", bend left =  10, gray]
		\arrow[ddr, "p_{2}", bend left = -10, gray, swap]
	&&&
	{\color{blue}Y}
	\\ \\
	&&&& {\color{blue}X_{2}}
		\arrow[uurr, "f_{2}", swap, blue]
		\arrow[uul, bend left = -10, "\iota_{2}", hook, gray, swap]
	\end{tikzcd}
	\end{center}
	This completes the proof of Claim 1,
	as well as that of the existence of fiber products in an arbitrary abelian category.
\item
	We now prove the existence of fiber coproducts in an arbitrary abelian category.
	Suppose \,$\mathfrak{A}$\, is an abelian category,
	\,$f_{1} \in \Mor_{\mathfrak{A}}(Y,X_{1})$\,
	and
	\,$f_{2} \in \Mor_{\mathfrak{A}}(Y,X_{1})$.\,
	Since \,$\mathfrak{A}$\, is an abelian category, the finite {\color{red}biproduct}
	\begin{equation*}
	\left(\,
		X_{1} \overset{{\color{white}.}}{\oplus} X_{2}
		\,,\,
		p_{1}
		\,,\,
		p_{2}
		\,,\,
		\iota_{1}
		\,,\,
		\iota_{2}
		\,\right)
	\end{equation*}
	of \,$X_{1},\, X_{2}$\, exists, where
	\,$p_{1} : X_{1} \oplus X_{2} \longrightarrow X_{1}$,\,
	\,$p_{2} : X_{1} \oplus X_{2} \longrightarrow X_{2}$,\,
	\,$\iota_{1} : X_{1} \longrightarrow X_{1} \oplus X_{2}$,\,
	\,$\iota_{2} : X_{2} \longrightarrow X_{1} \oplus X_{2}$\,
	satisfy:
	\begin{equation*}
	p_{1} \circ \iota_{1} = 1_{X_{1}}\,,
	\quad
	p_{2} \circ \iota_{2} = 1_{X_{2}}\,,
	\quad
	p_{1} \circ \iota_{2} = 0_{X_{2},X_{1}}\,,
	\quad
	p_{2} \circ \iota_{1} = 0_{X_{1},X_{2}}
	\end{equation*}
	\vskip 0.3cm
	\noindent
	\textbf{Claim 2:}\quad
	Let \,$\pi \,\in\, \Mor_{\mathfrak{A}}(\,X_{1}\oplus X_{2}\,,\,Q\,)$\,
	be any {\color{red}cokernel of
	\,$\varphi\,:=\,\iota_{1}\,\circ\,f_{1}\,-\,\iota_{2}\,\circ\,f_{2}\,\in\,\Mor_{\mathfrak{A}}(\,Y\,,\,X_{1} \oplus X_{2}\,)$}.\,
	Define
	\,$\alpha_{1} := \pi \circ \iota_{1} \in \Mor_{\mathfrak{A}}(X_{1},Q)$\,
	and
	\,$\alpha_{2} := \pi \circ \iota_{2} \in \Mor_{\mathfrak{A}}(X_{2},Q)$.\,
	Then,
	\,$\left(\,Q\,,\,\alpha_{1}\,,\,\alpha_{2}\,\right)$\,
	is a fiber coproduct of
	\,$f_{1}$\,
	and
	\,$f_{2}$.\,
	\vskip 0.2cm
	\noindent
	Proof of Claim 2:\;\;
	First, observe that:
	\begin{eqnarray*}
	\alpha_{1} \circ f_{1} - \alpha_{2} \circ f_{2}
	& = &
		(\,\pi \circ \iota_{1}\,) \circ f_{1} - (\,\pi \circ \iota_{2}\,) \circ f_{2}
	\;\; = \;\;
		\pi \circ (\,\iota_{1} \circ f_{1}\,) - \pi \circ (\,\iota_{2} \circ f_{2}\,)
	\\
	& = &
		\pi \circ (\,\iota_{1} \circ f_{1} - \iota_{2} \circ f_{2}\,) 
	\;\; = \;\;
		\pi \circ \varphi
	\\
	& = &
		0
	\end{eqnarray*}
	Next, let
	\,$h_{1} \in \Mor_{\mathfrak{A}}(X_{1},A)$\,
	and
	\,$h_{2} \in \Mor_{\mathfrak{A}}(X_{2},A)$\,
	be such that
	\,$h_{1}\,\circ\,f_{1} \,=\, h_{2}\,\circ\,f_{2}$.\,
	Define
	\,$\psi\,:=\,h_{1}\,\circ\,p_{1}\,+\,h_{2}\,\circ \,p_{2}\,\in\,\Mor_{\mathfrak{A}}(\,X_{1}\oplus X_{2}\,,A\,)$.\,
	Observe:
	\begin{eqnarray*}
	\psi \circ \varphi
	& = &
		\left(\,h_{1} \circ\, p_{1} \,\overset{{\color{white}.}}{+}\, h_{2} \circ\, p_{2}\,\right)
		\,\circ\,
		\left(\,\iota_{1} \circ f_{1} \,\overset{{\color{white}.}}{-}\, \iota_{2} \circ f_{2}\,\right)
	\\
	& = &
		h_{1} \,\circ\, p_{1} \,\circ\, \iota_{1} \,\circ\, f_{1}
		\; + \;
		h_{1} \,\circ\, p_{1} \,\circ\, \iota_{2} \,\circ\, f_{2}
		\; - \;
		h_{2} \,\circ\, p_{2} \,\circ\, \iota_{1} \,\circ\, f_{1}
		\; - \;
		h_{2} \,\circ\, p_{2} \,\circ\, \iota_{2} \,\circ\, f_{2}
	\\
	& \overset{{\color{white}1}}{=} &
		h_{1} \,\circ\, 1_{X_{1}} \,\circ\, f_{1}
		\; + \;
		h_{1} \circ\, 0_{X_{2},X_{1}} \,\circ\, f_{2}
		\; - \;
		h_{2} \circ\, 0_{X_{1},X_{2}} \,\circ\, f_{1}
		\; - \;
		h_{2} \,\circ\, 1_{X_{2}} \,\circ\, f_{2}
	\\
	& \overset{{\color{white}1}}{=} &
		h_{1} \,\circ\, f_{1}
		\; - \;
		h_{2} \,\circ\, f_{2}
	\;\; = \;\;
		0
	\end{eqnarray*}
	Thus, by the universal property of
	\,$\pi \,\in\, \Mor_{\mathfrak{A}}(\,X_{1} \oplus X_{2}\,,\,Q\,)$\,
	as cokernel of
	\,$\varphi\,:=\,\iota_{1}\,\circ\,f_{1}\,-\,\iota_{2}\,\circ\,f_{2}\,\in\,\Mor_{\mathfrak{A}}(\,Y\,,\,X_{1} \oplus X_{2}\,)$,\,
	there exists a unique
	\,$\theta \in \Mor_{\mathfrak{A}}(Q,A)$\,
	such that
	\,$\psi = \theta \circ \pi$.\,
	Furthermore, \,$\theta$\, satisfies the following two equalities:
	%\vskip 0.1cm
	\begin{eqnarray*}
	\theta \circ \alpha_{1}
	& = &
		\theta \circ (\,\pi \circ \iota_{1}\,)
	\;\; = \;\;
		(\,\theta \circ \pi\,) \circ \iota_{1}
	\;\; = \;\;
		\psi \circ \iota_{1} 
	\;\; = \;\;
		(\,h_{1} \circ p_{1} \,+\, h_{2} \circ p_{2} \,) \circ \iota_{1}
	\\
	& = &
		h_{1} \circ (\,p_{1} \circ \iota_{1}\,)
		\,+\
		h_{2} \circ (\,p_{2} \circ \iota_{1}\,)
	\;\; = \;\;
		h_{1} \circ 1_{X_{1}}
		\,+\
		h_{2} \circ 0_{X_{1},X_{2}}
	\\
	& = &
		h_{1}
	\\ \\
	\theta \circ \alpha_{2}
	& = &
		\theta \circ (\,\pi \circ \iota_{2}\,)
	\;\; = \;\;
		(\,\theta \circ \pi\,) \circ \iota_{2}
	\;\; = \;\;
		\psi \circ \iota_{2}
	\;\; = \;\;
		(\,h_{1} \circ p_{1} \,+\, h_{2} \circ p_{2} \,) \circ \iota_{2}
	\\
	& = &
		h_{1} \circ (\,p_{1} \circ \iota_{2}\,)
		\,+\
		h_{2} \circ (\,p_{2} \circ \iota_{2}\,)
	\;\; = \;\;
		h_{1} \circ 0_{X_{2},X_{1}}
		\,+\
		h_{2} \circ 1_{X_{2}}
	\\
	& = &
		h_{2}
	\end{eqnarray*}
	% It remains to establish the uniqueness of \,$\theta$.\,
	% To this end, suppose
	% \,$\theta^{\prime} \in \Mor_{\mathfrak{A}}(A,K)$\,
	% is any morphism that satisfies
	% \begin{equation*}
	% \psi \,=\, \kappa \circ \theta^{\prime}\,,
	% \quad
	% \pi_{1} \circ \theta^{\prime} \,=\, h_{1}\,,
	% \quad
	% \pi_{2} \circ \theta^{\prime} \,=\, h_{2}\,,
	% \end{equation*}
	The above arguments are summarized in the following commutative diagram:
	\begin{center}
	\begin{tikzcd}
	&&&& {\color{blue}X_{1}}
		\arrow[ddl, bend left = 10, "\iota_{1}", hook, gray]
		\arrow[ddllll, bend left = -20, "h_{1}", swap]
		\arrow[ddll, "{\color{red}\alpha_{1}}", bend left = -10, red, swap]
	\\ \\
	A
	&&
	%{\color{red}X_{1} \sqcup_{Y}\! X_{2}}
	{\color{red}Q}
		%\arrow[r, "\kappa", hook, gray]
		\arrow[ll, dashed, "\;\;\;\exists\,!\,\theta", swap]
	&
	{\color{gray}X_{1} \oplus X_{2}}
		\arrow[l, "\quad\pi", gray, swap]
		\arrow[uur, "p_{1}", bend left =  10, gray]
		\arrow[ddr, "p_{2}", bend left = -10, gray, swap]
		\arrow[lll, bend left = 20, "\psi"]
	&&&
	{\color{blue}Y}
		\arrow[uull, "f_{1}", blue, swap]
		\arrow[ddll, "f_{2}", blue]
		\arrow[lll, "\varphi\,:=\, \iota_{1} \circ\, f_{1} \,-\, \iota_{2} \circ\, f_{2}", gray, swap]
	\\ \\
	&&&& {\color{blue}X_{2}}
		\arrow[uull, "{\color{red}\alpha_{2}}", bend left = 10, red]
		\arrow[uullll, bend left = 20, "h_{2}"]
		\arrow[uul, bend left = -10, "\iota_{2}", hook, gray, swap]
	\end{tikzcd}
	\end{center}
	This completes the proof of Claim 2,
	as well as that of the existence of fiber coproducts in an arbitrary abelian category.
	\qed
\end{enumerate}

          %%%%% ~~~~~~~~~~~~~~~~~~~~ %%%%%
