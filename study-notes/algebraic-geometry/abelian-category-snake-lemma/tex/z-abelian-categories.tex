
          %%%%% ~~~~~~~~~~~~~~~~~~~~ %%%%%

\section{Abelian categories}
\setcounter{theorem}{0}
\setcounter{equation}{0}

%\cite{vanDerVaart1996}
%\cite{Kosorok2008}

%\renewcommand{\theenumi}{\alph{enumi}}
%\renewcommand{\labelenumi}{\textnormal{(\theenumi)}$\;\;$}
\renewcommand{\theenumi}{\roman{enumi}}
\renewcommand{\labelenumi}{\textnormal{(\theenumi)}$\;\;$}

          %%%%% ~~~~~~~~~~~~~~~~~~~~ %%%%%

\begin{definition}[Category]
\mbox{}
\vskip 0.15cm
\noindent
A \,\textbf{category}\, $\mathfrak{C}$ consists of the following:
\begin{itemize}
\item
	a class $\Obj(\mathfrak{C})$,
\item
	a set $\Mor_{\mathfrak{C}}(A,B)$, for each $A, B \in \Obj(\mathfrak{C})$,
\item
	a \textbf{composition map}
	\begin{equation*}
	\Mor_{\mathfrak{C}}(A,B) \times \Mor_{\mathfrak{C}}(B,C) \longrightarrow \Mor_{\mathfrak{C}}(A,C) : (f,g) \longmapsto g \circ f\,,
	\end{equation*}
	for each $A, B, C \in \Obj(\mathfrak{C})$,
\end{itemize}
satisfying:
\begin{enumerate}
\item
	the sets $\Mor_{\mathfrak{C}}(\,\cdot\,,\,\cdot\,)$ are pairwise disjoint, i.e.,
	for $A_{1}, B_{1}, A_{2}, A_{2} \in \Obj(\mathfrak{C})$, we have:
	\begin{equation*}
	\Mor_{\mathfrak{C}}(A_{1},B_{2})\;\bigcap\;\Mor_{\mathfrak{C}}(A_{1},B_{2})
	\; = \;
		\varemptyset\,,
	\;\;
	\textnormal{whenever $A_{1} \neq A_{2}$ or $B_{1} \neq B_{2}$}
	\end{equation*}
\item
	for each $A \in \Obj(\mathfrak{C})$, there exists $1_{A} \in \Mor_{\mathfrak{C}}(A,A)$ such that
	\begin{equation*}
	f \circ 1_{A} = f
	\;\;\;\;\textnormal{and}\;\;\;\;
	1_{B} \circ f = f,
	\quad
	\textnormal{for each $f \in \Mor_{\mathfrak{C}}(A,B)$, \,$B \in \Obj(\mathfrak{C})$}
	\end{equation*}
\item
	the composition map is associative, i.e.,
	\begin{equation*}
	(f \circ g) \circ h
	\;\; = \;\;
	f \circ (g \circ h)\,,
	\end{equation*}
	for each
	$h \in \Mor_{\mathfrak{C}}(A,B)$,
	$g \in \Mor_{\mathfrak{C}}(B,C)$,
	$f \in \Mor_{\mathfrak{C}}(C,D)$,\;
	where
	\,$A,B,C,D \in \Obj(\mathfrak{C})$.
\end{enumerate}
The elements of $\Obj(\mathfrak{C})$ are called the \textbf{objects} of $\mathfrak{C}$.
For $A, B \in \Obj(\mathfrak{C})$, the elements of  $\Mor_{\mathfrak{C}}(A,B)$
are called the \textbf{morphisms} in $\mathfrak{C}$ from $A$ to $B$.
For each $A \in \Obj(\mathfrak{C})$, $1_{A} \in \Mor_{\mathfrak{C}}(A,A)$ is called
the \textbf{identity morphism} of $A$.
\end{definition}

          %%%%% ~~~~~~~~~~~~~~~~~~~~ %%%%%

\vskip 0.5cm
\begin{definition}[Initial object, terminal object, zero object]
\mbox{}
\vskip 0.15cm
\noindent
Let \,$\mathfrak{C}$\, be a category.
\begin{itemize}
\item
	An object $A \in \Obj(\mathfrak{C})$ is called an \textbf{initial object} if $\Mor_{\mathfrak{C}}(A,X)$ is a singleton set,
	for each $X \in \Obj(\mathfrak{C})$.
\item
	An object $\Omega \in \Obj(\mathfrak{C})$ is called a \textbf{terminal object} if $\Mor_{\mathfrak{C}}(X,\Omega)$ is a singleton set,
	for each $X \in \Obj(\mathfrak{C})$.
\item
	An object of $\Obj(\mathfrak{C})$ is called a \textbf{zero object} if it is both initial and terminal.
\end{itemize}
\end{definition}

          %%%%% ~~~~~~~~~~~~~~~~~~~~ %%%%%

\vskip 0.5cm
\begin{definition}[Product, coproduct]
\mbox{}
\vskip 0.15cm
\noindent
Let \,$\mathfrak{C}$\, be a category, and
$\{\,A_{i}\,\}_{i \in I}$ a family of objects of $\mathfrak{C}$ indexed by a set $I$.
\begin{itemize}
\item
	A \textbf{product} is an ordered pair
	\,$\left(\,C\,,\,\{p_{i} : C \longrightarrow A_{i}\}_{i \in I}\,\right)$\,
	consisting of an object $C \in \Obj(\mathfrak{C})$ and a family 
	$\{p_{i} : C \longrightarrow A_{i}\}_{i \in I}$ of morphisms in $\mathfrak{C}$
	such that,
	for every object $X \in \Obj(\mathfrak{C})$ and morphisms $f_{i} \in \Mor_{\mathfrak{C}}(X,A_{i})$,
	there exists a unique morphism $\theta \in \Mor_{\mathfrak{C}}(X,C)$ such that,
	for each $i \in I$, the following diagram commutes:
	\begin{center}
	\begin{tikzcd}
	& {\color{red}C} \arrow[d, "p_{i}", red] \\
	X \arrow[ru, dashed,"\exists\,!\;\theta"] \arrow[r, swap, "f_{i}"] & {\color{red}A_{i}}
	\end{tikzcd}
	\end{center}
	If the product exists, it is denoted by: $\underset{i \in I}{\bigsqcap}\,A_{i}$.
	It is unique up to isomorphism.
\item
	A \textbf{coproduct} is an ordered pair
	\,$\left(\,C\,,\,\{\alpha_{i} : A_{i} \longrightarrow C\}_{i \in I}\,\right)$\,
	consisting of an object $C \in \Obj(\mathfrak{C})$ and a family 
	$\{\alpha_{i} : A_{i} \longrightarrow C\}_{i \in I}$ of morphisms in $\mathfrak{C}$
	such that,
	for every object $X \in \Obj(\mathfrak{C})$ and morphisms $g_{i} \in \Mor_{\mathfrak{C}}(A_{i},X)$,
	there exists a unique morphism $\theta \in \Mor_{\mathfrak{C}}(C,X)$ such that,
	for each $i \in I$, the following diagram commutes:
	\begin{center}
	\begin{tikzcd}
	& {\color{red}C} \arrow[ld, dashed, swap, "\exists\,!\;\theta"] \\
	X & {\color{red}A_{i}} \arrow[l, "g_{i}"] \arrow[u, swap, "\alpha_{i}", red]
	\end{tikzcd}
	\end{center}
	If the coproduct exists, it is denoted by: $\underset{i \in I}{\bigsqcup}\,A_{i}$.
	It is unique up to isomorphism.
\end{itemize}
\end{definition}

          %%%%% ~~~~~~~~~~~~~~~~~~~~ %%%%%

\vskip 0.5cm
\begin{definition}[Additive category]
\mbox{}
\vskip 0.15cm
\noindent
A category \,$\mathfrak{C}$\, is said to be \textbf{additive} if
\begin{itemize}
\item
	$\Mor_{\mathfrak{C}}(A,B)$ is an abelian group, for each \,$A, B \in \Obj(\mathfrak{C})$,
\item
	the distributive law holds for the morphism composition map, i.e.
	\begin{equation*}
	h \circ (f + g) \; = \; h \circ f + h \circ g
	\quad\textnormal{and}\quad
	(f + g) \circ k \; = \; f \circ k + g \circ k
	\end{equation*}
	for each
	$f, g \in \Mor_{\mathfrak{C}}(A,B)$,
	$h \in \Mor_{\mathfrak{C}}(B,Y)$,
	$k \in \Mor_{\mathfrak{C}}(X,A)$,
	$A, B, X, Y \in \Obj(\mathfrak{C})$,
\item
	the category $\mathfrak{C}$ has a zero object, and
\item
	finite products and finite coproducts exist in $\mathfrak{C}$.
\end{itemize}
\end{definition}

          %%%%% ~~~~~~~~~~~~~~~~~~~~ %%%%%

\vskip 0.5cm
\begin{proposition}
\mbox{}
\vskip 0.15cm
\noindent
\end{proposition}

          %%%%% ~~~~~~~~~~~~~~~~~~~~ %%%%%

\vskip 0.5cm
\begin{definition}[Kernels and cokernels of morphisms in a additive category]
\mbox{}
\vskip 0.15cm
\noindent
Let \,$\mathfrak{C}$\, be an additive category, and $f : A \longrightarrow B$ a morphism in $\mathfrak{C}$.
\begin{itemize}
\item
	A \textbf{kernel} of $f$ is a morphism $i : K \longrightarrow A$ in $\mathfrak{C}$ such that,
	for each $r \in \Mor_{\mathfrak{C}}(X,A)$ satisfying $f \circ r = 0$,
	there exists a unique $\theta \in \Mor_{\mathfrak{C}}(X,K)$ such that the following diagram commutes:
	\begin{center}
	\begin{tikzcd}
	X \arrow[d, dashed, swap, "\exists !\,\theta"] \arrow[dr, swap, "r"] \arrow[rrd, "0"] & & \\
	{\color{red}K} \arrow[r, swap, "i", red] & {\color{red}A} \arrow[r, swap, "f"] & B
	\end{tikzcd}
	\end{center}
\item
	A \textbf{cokernel} of $f$ is a morphism $\pi : B \longrightarrow Q$ in $\mathfrak{C}$ such that,
	for each $s \in \Mor_{\mathfrak{C}}(B,Y)$ satisfying \,$s \circ f = 0$,
	there exists a unique $\theta \in \Mor_{\mathfrak{C}}(Q,Y)$ such that the following diagram commutes:
	\begin{center}
	\begin{tikzcd}
	A \arrow[r, "f"] \arrow[rrd, swap, "0"] & {\color{red}B} \arrow[rd, "s"] \arrow[r, "\pi", red]& {\color{red}Q} \arrow[d, dashed, "\;\exists !\,\theta"] \\
	& & Y
	\end{tikzcd}
	\end{center}
\end{itemize}
\end{definition}

          %%%%% ~~~~~~~~~~~~~~~~~~~~ %%%%%
