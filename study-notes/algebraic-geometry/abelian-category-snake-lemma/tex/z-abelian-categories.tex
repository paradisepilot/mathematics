
          %%%%% ~~~~~~~~~~~~~~~~~~~~ %%%%%

\section{Abelian categories}
\setcounter{theorem}{0}
\setcounter{equation}{0}

%\cite{vanDerVaart1996}
%\cite{Kosorok2008}

%\renewcommand{\theenumi}{\alph{enumi}}
%\renewcommand{\labelenumi}{\textnormal{(\theenumi)}$\;\;$}
\renewcommand{\theenumi}{\roman{enumi}}
\renewcommand{\labelenumi}{\textnormal{(\theenumi)}$\;\;$}

          %%%%% ~~~~~~~~~~~~~~~~~~~~ %%%%%

\vskip 0.5cm
\begin{definition}[Abelian category]
\mbox{}
\vskip 0.15cm
\noindent
An additive category \,$\mathfrak{A}$\, is said to be \textbf{abelian}
if it furthermore satisfies each of the following conditions:
\begin{itemize}
\item
	finite products exist in \,$\mathfrak{A}$\,,
\item
	every morphism in \,$\mathfrak{A}$\, has a kernel as well as a cokernel,
\item
	every monomorphism in \,$\mathfrak{A}$\, is a kernel, and
\item
	every epimorphism in \,$\mathfrak{A}$\, is a cokernel.
\end{itemize}
\end{definition}

          %%%%% ~~~~~~~~~~~~~~~~~~~~ %%%%%

\vskip 0.5cm
\begin{proposition}[Proposition 1.2.9, p.24, \cite{Lal2021Algebra3}; Theorem 2.12, p.37, \cite{freyd1964abelian}]
\label{AbelianImpliesIsomorphismIFFMonoEpi}
\mbox{}
\vskip 0.1cm
\noindent
In an abelian category, a morphism is an isomorphism if and only if
it is both a monomorphism and an epimorphism.
\end{proposition}
\proof

\qed

          %%%%% ~~~~~~~~~~~~~~~~~~~~ %%%%%

\begin{proposition}\label{MonomorphicFirstFactorInExactSequenceIsKernelOfSecondFactor}
\mbox{}
\vskip 0.1cm
\noindent
Suppose \,$\mathfrak{A}$\, is an abelian category,
$f \in \Mor_{\mathfrak{A}}(A,B)$, and $g \in \Mor_{\mathfrak{A}}(B,C)$
such that
\begin{itemize}
\item
	$f$\, is a monomorphism,
\item
	$g \circ f = 0$,\, and
\item
	$A \overset{f}{\longrightarrow} B \overset{g}{\longrightarrow} C$\,
	is exact.
\end{itemize}
Then, \,$f$\, is a kernel of \,$g$.
\end{proposition}
\proof
The exactness of
\,$A \overset{f}{\longrightarrow} B \overset{g}{\longrightarrow} C$\,
means that the unique morphism \,$\theta$\, in the following diagram is an isomorphism:
	\begin{center}
	\begin{tikzcd}
	&& && Q_{f}
		\arrow[dd, dashed, "\psi"]
	\\ \\
	A
		\arrow[rr, "f"]
		\arrow[dd, dashed, two heads, swap, "\varepsilon_{f}"]
		\arrow[rrrr, bend left = 30, "0", gray]
	&&
	B
		\arrow[rr, "g"]
		\arrow[uurr, two heads, "\pi_{f}"]
	&&
	C
	\\ \\
	I_{f}
		\arrow[uurr, hook, "\kappa(\pi_{f})"]
		\arrow[rr, thick, dashed, swap, "\exists !\,\theta", red]
	&&
	K_{g}
		\arrow[uu, hook, swap, "\kappa_{g}"]
		%\arrow[rr, thick, two heads, swap, "\pi_{\theta}", red]
	&&
	%Q_{\theta}
	\end{tikzcd}
	\end{center}
Since \,$\mathfrak{A}$\, is an abelian category, it in particular has cokernels and is normal.
By Proposition \ref{FactorizationIntoImageCoimage}, we see that
\,$\kappa(\pi_{f})$\, is an image of \,$f$.\,
Since \,$\mathfrak{A}$\, is an abelian category, it in particular is normal and has cokernels and equalizers.
By Proposition \ref{FactorizationIntoImageCoimage} again, we see that
\,$\varepsilon_{f}$\, is a coimage of \,$f$, which we denote by \,$\widetilde{f}$\, from now on.
By Lemma \ref{fTildeIsMonomorphismWheneverFIs},
since \,$f$\, is a monomorphism, it follows that \,$\varepsilon_{f} \,=\, \widetilde{f}$\,
is itself a monomorphism.
Thus, 
\,$\varepsilon_{f} \,=\, \widetilde{f}$\,
is both a monomorphism and an epimorphism.
By Proposition \ref{AbelianImpliesIsomorphismIFFMonoEpi}, we see that
\,$\varepsilon_{f} \,=\, \widetilde{f}$\,
is an isomorphism.
Thus,
\,$f \,=\, \kappa_{g} \,\circ\, \theta \,\circ\, \varepsilon_{f}$,\,
where \,$\theta$\, and \,$\varepsilon_{f}$\, are isomorphisms.
Since \,$\kappa_{g}$\, is a kernel of \,$g$,\,
we may now conclude that \,$f$\, is also a kernel of \,$g$.\, 
This completes the proof of the Proposition.
\qed

          %%%%% ~~~~~~~~~~~~~~~~~~~~ %%%%%
