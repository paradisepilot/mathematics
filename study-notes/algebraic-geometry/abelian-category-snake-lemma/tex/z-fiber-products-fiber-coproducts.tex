
          %%%%% ~~~~~~~~~~~~~~~~~~~~ %%%%%

\section{Fiber products \& fiber coproducts}
\setcounter{theorem}{0}
\setcounter{equation}{0}

%\cite{vanDerVaart1996}
%\cite{Kosorok2008}

%\renewcommand{\theenumi}{\alph{enumi}}
%\renewcommand{\labelenumi}{\textnormal{(\theenumi)}$\;\;$}
\renewcommand{\theenumi}{\roman{enumi}}
\renewcommand{\labelenumi}{\textnormal{(\theenumi)}$\;\;$}

          %%%%% ~~~~~~~~~~~~~~~~~~~~ %%%%%

\begin{definition}[Fiber products \& fiber coproducts, p.46, \cite{kashiwara2005categories}]
\mbox{}
\vskip 0.15cm
\noindent
Let \,$\mathfrak{C}$\, be a category.
\begin{enumerate}
\item
	Let
	\,$f_{1} \in \Mor_{\mathfrak{C}}(X_{1},Y)$\,
	and
	\,$f_{2} \in \Mor_{\mathfrak{C}}(X_{2},Y)$,\,
	where
	\,$X_{1},\, X_{2},\, Y \in \Obj(\mathfrak{C})$.
	The \textbf{fiber product} of \,$f_{1},\, f_{2}$\, is a triple
	\,$\left(\,X_{1}\overset{{\color{white}.}}{\sqcap_{Y}}\!X_{2} \,,\, \pi_{1} \,,\, \pi_{2}\,\right)$,\,
	where
	\,$X_{1}\sqcap_{Y}\!X_{2} \in \Obj(\mathfrak{C})$,\,
	\,$\pi_{1} \in \Mor_{\mathfrak{C}}(\,X_{1}\sqcap_{Y}\!X_{2}\,,\,X_{1}\,)$,\,
	\,$\pi_{2} \in \Mor_{\mathfrak{C}}(\,X_{1}\sqcap_{Y}\!X_{2}\,,\,X_{2}\,)$,\,
	such that
	\begin{itemize}
	\item
		$f_{1} \circ \pi_{1} \,=\, f_{2} \circ \pi_{2}$,\, and
	\item
		for any
		\,$h_{1} \in \Mor_{\mathfrak{C}}(A,X_{1})$\,
		and
		\,$h_{2} \in \Mor_{\mathfrak{C}}(A,X_{2})$\,
		satisfying
		\,$f_{1} \circ h_{1} \,=\, f_{2} \circ h_{2}$,\,
		there exists a unique
		\,$\theta \in \Mor_{\mathfrak{C}}(\,A\,,\,X_{1}\sqcap_{Y}\!X_{2}\,)$\,
		such that the following diagram commutes:
		\begin{center}
		\begin{tikzcd}
		&&&& {\color{blue}X_{1}}
			\arrow[ddrr, "f_{1}", blue]
		\\ \\
		A
			\arrow[rr, dashed, "\;\;\;\exists\,!\,\theta"]
			\arrow[uurrrr, bend left =  20, "h_{1}"]
			\arrow[ddrrrr, bend left = -20, "h_{2}", swap]
		&&
		{\color{red}X_{1} \sqcap_{Y}\! X_{2}}
			\arrow[uurr, "{\color{red}\pi_{1}}", red]
			\arrow[ddrr, "{\color{red}\pi_{2}}", red, swap]
		&&&&
		{\color{blue}Y}
		\\ \\
		&&&& {\color{blue}X_{2}}
			\arrow[uurr, "f_{2}", swap, blue]
		\end{tikzcd}
		\end{center}
	\end{itemize}
\item
	Dually, let
	\,$f_{1} \in \Mor_{\mathfrak{C}}(Y,X_{1})$\,
	and
	\,$f_{2} \in \Mor_{\mathfrak{C}}(Y,X_{2})$,\,
	where
	\,$S,\, X_{1},\, X_{2} \in \Obj(\mathfrak{C})$.
	The \textbf{fiber coproduct} of \,$f_{1},\, f_{2}$\, is a triple
	\,$\left(\,X_{1}\overset{{\color{white}.}}{\sqcup_{Y}}\!X_{2} \,,\, \iota_{1} \,,\, \iota_{2}\,\right)$,\,
	where
	\,$X_{1}\sqcup_{Y}\!X_{2} \in \Obj(\mathfrak{C})$,\,
	\,$\iota_{1} \in \Mor_{\mathfrak{C}}(\,X_{1}\,,X_{1}\sqcup_{Y}\!X_{2}\,)$,\,
	\,$\iota_{2} \in \Mor_{\mathfrak{C}}(\,X_{2}\,,X_{1}\sqcup_{Y}\!X_{2}\,)$,\,
	such that,
	\begin{itemize}
	\item
		$\iota_{1} \circ f_{1} \,=\, \iota_{2} \circ f_{2}$,\, and
	\item
		for any
		\,$h_{1} \in \Mor_{\mathfrak{C}}(X_{1},A)$\,
		and
		\,$h_{2} \in \Mor_{\mathfrak{C}}(X_{2},A)$\,
		satisfying
		\,$h_{1} \circ f_{1} \,=\, h_{2} \circ f_{2}$,\,
		there exists a unique
		\,$\theta \in \Mor_{\mathfrak{C}}(\,X_{1}\sqcup_{S}\!X_{2}\,,\,A\,)$\,
		such that the following diagram commutes:
		\begin{center}
		\begin{tikzcd}
		&&&& {\color{blue}X_{1}}
			\arrow[ddll, "{\color{red}\iota_{1}}", red, swap]
			\arrow[ddllll, bend left = -20, "h_{1}", swap]
		\\ \\
		A
		&&
		{\color{red}X_{1} \sqcup_{Y}\! X_{2}}
			\arrow[ll, dashed, "\;\;\;\exists\,!\,\theta", swap]
		&&&&
		{\color{blue}Y}
			\arrow[uull, "f_{1}", blue, swap]
			\arrow[ddll, "f_{2}", blue]
		\\ \\
		&&&& {\color{blue}X_{2}}
			\arrow[uull, "{\color{red}\iota_{2}}", red]
			\arrow[uullll, bend left = 20, "h_{2}"]
		\end{tikzcd}
		\end{center}
	\end{itemize}
\end{enumerate}
\end{definition}

          %%%%% ~~~~~~~~~~~~~~~~~~~~ %%%%%

\vskip 0.5cm
\begin{proposition}[Fiber products preserve monomormphisms; Proposition I.7.1, p.9, \cite{Mitchell1965}]
\mbox{}
\vskip 0.1cm
\noindent
Let \,$\mathfrak{C}$\, be a category.
\begin{enumerate}
\item
	\vskip -0.1cm
	Suppose that the fiber product
	\,$\left(\,X_{1}\overset{{\color{white}.}}{\sqcap_{Y}}\!X_{2} \,,\, \pi_{1} \,,\, \pi_{2}\,\right)$\,
	exists for two morphisms
	\,$f_{1} \in \Mor_{\mathfrak{C}}(X_{1},Y)$\,
	and
	\,$f_{2} \in \Mor_{\mathfrak{C}}(X_{2},Y)$,\,
	where
	\,$X_{1},\, X_{2},\, Y,\, X_{1}\sqcap_{Y}\!X_{2} \,\in\, \Obj(\mathfrak{C})$,\,
	\,$\pi_{1} \in \Mor_{\mathfrak{C}}(\,X_{1}\sqcap_{Y}\!X_{2}\,,\,X_{1}\,)$,\,
	\,$\pi_{2} \in \Mor_{\mathfrak{C}}(\,X_{1}\sqcap_{Y}\!X_{2}\,,\,X_{2}\,)$.\,
	\begin{center}
	\begin{tikzcd}
	&& {\color{blue}X_{1}}
		\arrow[ddrr, "f_{1}", blue]
	\\ \\
	{\color{red}X_{1} \sqcap_{Y}\! X_{2}}
		\arrow[uurr, "{\color{red}\pi_{1}}", red]
		\arrow[ddrr, "{\color{red}\pi_{2}}", red, swap]
	&&&&
	{\color{blue}Y}
	\\ \\
	&& {\color{blue}X_{2}}
		\arrow[uurr, "f_{2}", swap, blue]
	\end{tikzcd}
	\end{center}
	Then, the following statements are true:
	\begin{itemize}
	\item
		If \,$f_{2}$\, is a monomorphism, then \,$\pi_{1}$\, is a monomorphism.
	\item
	If \,$f_{1}$\, is a monomorphism, then \,$\pi_{2}$\, is a monomorphism.
	\end{itemize}
\item
	Dually, suppose that the fiber coproduct
	\,$\left(\,X_{1}\overset{{\color{white}.}}{\sqcup_{Y}}\!X_{2} \,,\, \iota_{1} \,,\, \iota_{2}\,\right)$\,
	exists for two morphisms
	\,$f_{1} \in \Mor_{\mathfrak{C}}(Y,X_{1})$\,
	and
	\,$f_{2} \in \Mor_{\mathfrak{C}}(Y,X_{2})$,\,
	where
	\,$X_{1},\, X_{2},\, Y,\, X_{1}\sqcup_{Y}\!X_{2} \,\in\, \Obj(\mathfrak{C})$,\,
	\,$\iota_{1} \in \Mor_{\mathfrak{C}}(\,X_{1}\,,\,X_{1}\sqcup_{Y}\!X_{2}\,)$,\,
	\,$\iota_{2} \in \Mor_{\mathfrak{C}}(\,X_{2}\,,\,X_{1}\sqcup_{Y}\!X_{2}\,)$.\,
	\begin{center}
	\begin{tikzcd}
	&& {\color{blue}X_{1}}
		\arrow[ddll, "{\color{red}\iota_{1}}", red, swap]
	\\ \\
	{\color{red}X_{1} \sqcup_{Y}\! X_{2}}
	&&&&
	{\color{blue}Y}
		\arrow[uull, "f_{1}", blue, swap]
		\arrow[ddll, "f_{2}", blue]
	\\ \\
	&& {\color{blue}X_{2}}
		\arrow[uull, "{\color{red}\iota_{2}}", red]
	\end{tikzcd}
	\end{center}
	Then, the following statements are true:
	\begin{itemize}
	\item
		If \,$f_{2}$\, is an epimorphism, then \,$\iota_{1}$\, is an epimorphism.
	\item
		If \,$f_{1}$\, is an epimorphism, then \,$\iota_{2}$\, is an epimorphism.
	\end{itemize}
\end{enumerate}
\end{proposition}
\proof
\begin{enumerate}
\item
	We prove only the first implication, i.e.,
	if \,$f_{2}$\, is a monomorphism, then \,$\pi_{1}$\, is a monomorphism.
	The proof of the other implication follows by symmetry:
	$(\,f_{2}\,,\,\pi_{1}\,) \,\longleftrightarrow\, (\,f_{1}\,,\,\pi_{2}\,)$.

	So, suppose that \,$f_{2}$\, is a monomorphism.
	we need to establish that \,$\pi_{1}$\, is a monomorphism; in other words,
	\,$\pi_{1}$\, can be ``cancelled on the left,'' i.e., for arbitrary
	\,$\varphi,\, \psi \,\in\, \Mor_{\mathfrak{C}}(\,A\,,\,X_{1}\sqcap_{Y}\! X_{2}\,)$,\,
	we have:
	\,$\pi_{1} \,\circ\, \varphi \,=\, \pi_{1} \,\circ\, \psi$\,
	\,$\Longrightarrow$\,
	\,$\varphi \,=\, \psi$.\,

	\vskip 0.1cm
	\noindent
	So, we now furthermore assume
	\,$\pi_{1} \,\circ\, \varphi \,=\, \pi_{1} \,\circ\, \psi$,\,
	which allows us to define:
	\,$h_{1} \, := \, \pi_{1} \,\circ\, \varphi \,=\, \pi_{1} \,\circ\, \psi$.\,

	\vskip 0.3cm
	\noindent
	\textbf{Claim 1:}\quad $\pi_{2} \,\circ\, \varphi \,=\, \pi_{2} \,\circ\, \psi$
	\vskip 0.01cm
	Proof of Claim 1:\;\;Observe:
	\begin{eqnarray*}
	f_{2} \circ (\,\pi_{2} \circ \varphi\,)
	& = &
		(\,f_{2} \circ \pi_{2}\,) \circ \varphi
	\; = \;
		(\,f_{1} \circ \pi_{1}\,) \circ \varphi
	\; = \;
		f_{1} \circ (\,\pi_{1} \circ \varphi\,)
	\\
	& = &
		f_{1} \circ (\,\pi_{1} \circ \psi\,),
		\quad
		\textnormal{since \,$\pi_{1} \circ \varphi = \pi_{1} \circ \psi$}
	\\
	& = &
		(\,f_{1} \circ \pi_{1}\,) \circ \psi
	\; = \;
		(\,f_{2} \circ \pi_{2}\,) \circ \psi
	\\
	& = &
		f_{2} \circ (\,\pi_{2} \circ \psi\,)
	\end{eqnarray*}
	That \,$f_{2}$\, is a monomorphism now implies that
	\,$\pi_{2} \circ \varphi \,=\,\pi_{2} \circ \psi$,\,
	which completes the proof of Claim 1.

	\vskip 0.3cm
	\noindent
	\textbf{Claim 2:}\quad $\varphi \,=\, \psi$
	\vskip 0.01cm
	Proof of Claim 2:\;\; By Claim 1, we may define:
	\,$h_{2} \, := \, \pi_{2} \,\circ\, \varphi \,=\, \pi_{2} \,\circ\, \psi$.\,
	Now, observe:
	\begin{equation*}
	f_{1} \circ h_{1}
	\; = \;
		f_{1} \circ (\,\pi_{1} \circ \varphi\,)
	\; = \;
		(\,f_{1} \circ \pi_{1}\,) \circ \varphi
	\; = \;
		(\,f_{2} \circ \pi_{2}\,) \circ \varphi
	\; = \;
		f_{2} \circ (\,\pi_{2} \circ \varphi\,)
	\; = \;
		f_{2} \circ h_{2}\,,
	\end{equation*}
	which implies -- by the universal property of the fiber product --
	the existence of a unique
	\,$\theta \in \Mor_{\mathfrak{C}}(\,A\,,\,X_{1}\sqcap_{Y}\!X_{2}\,)$\,
	such that
	\,$\pi_{1} \circ \theta \,=\, h_{1}$\,
	and
	\,$\pi_{2} \circ \theta \,=\, h_{2}$.\,
	But now the uniqueness of \,$\theta$\, implies
	\,$\varphi = \theta = \psi$.\,
	This proves Claim 2.

	\vskip 0.3cm
	\noindent
	This completes the proof that \,$\pi_{1}$\, is a monomorphism, as required.

	\vskip 0.3cm
	\noindent
	The situation is depicted in the following diagram:
	\begin{center}
	\begin{tikzcd}
	&&&& {\color{blue}X_{1}}
		\arrow[ddrr, "f_{1}", blue]
	\\ \\
	A
		\arrow[rr, dashed, "\exists ! \theta"]
		\arrow[rr, bend left =  27, "\varphi"]
		\arrow[rr, bend left = -27, "\psi", swap]
		\arrow[uurrrr, bend left =  30, "h_{1}"]
		\arrow[ddrrrr, bend left = -30, "h_{2}", swap]
	&&
	{\color{red}X_{1} \sqcap_{Y}\! X_{2}}
		\arrow[uurr, "{\color{red}\pi_{1}}", red]
		\arrow[ddrr, "{\color{red}\pi_{2}}", red, swap]
	&&&&
	{\color{blue}Y}
	\\ \\
	&&&& {\color{blue}X_{2}}
		\arrow[uurr, "f_{2}", swap, blue]
	\end{tikzcd}
	\end{center}

\item
	We prove only the first implication, i.e.,
	if \,$f_{2}$\, is an epimorphism, then \,$\iota_{1}$\, is an epimorphism.
	The proof of the other implication follows by symmetry:
	$(\,f_{2}\,,\,\pi_{1}\,) \,\longleftrightarrow\, (\,f_{1}\,,\,\pi_{2}\,)$.

	So, suppose that \,$f_{2}$\, is an epimorphism.
	we need to establish that \,$\iota_{1}$\, is an epimorphism; in other words,
	\,$\iota_{1}$\, can be ``cancelled on the right,'' i.e., for arbitrary
	\,$\varphi,\, \psi \,\in\, \Mor_{\mathfrak{C}}(\,X_{1}\sqcup_{Y}\! X_{2}\,,\,A\,)$,\,
	we have:
	\,$\varphi \,\circ\, \iota_{1} \,=\, \psi \,\circ\, \iota_{1}$\,
	\,$\Longrightarrow$\,
	\,$\varphi \,=\, \psi$.\,

	\vskip 0.1cm
	\noindent
	So, we now furthermore assume
	\,$\varphi \,\circ\, \iota_{1} \,=\, \psi \,\circ\, \iota_{1}$,\,
	which allows us to define:
	\,$h_{1} \, := \, \varphi \,\circ\, \iota_{1} \,=\, \psi \,\circ\, \iota_{1}$.\,

	\vskip 0.3cm
	\noindent
	\textbf{Claim 1:}\quad $\varphi \,\circ\, \iota_{2} \,=\, \psi \,\circ\, \iota_{2}$
	\vskip 0.01cm
	Proof of Claim 1:\;\;Observe:
	\begin{eqnarray*}
	(\,\varphi \circ \iota_{2}\,) \circ f_{2}
	& = &
		\varphi \circ (\,\iota_{2} \circ f_{2}\,)
	\; = \;
		\varphi \circ (\,\iota_{1} \circ f_{1}\,)
	\; = \;
		(\,\varphi \circ \iota_{1}\,) \circ f_{1}
	\\
	& = &
		(\,\psi \circ \iota_{1}\,) \circ f_{1},
		\quad
		\textnormal{since \,$\varphi \circ \iota_{1} = \psi \circ \iota_{1}$}
	\\
	& = &
		\psi \circ (\,\iota_{1} \circ f_{1}\,)
	\; = \;
		\psi \circ (\,\iota_{2} \circ f_{2}\,)
	\\
	& = &
		(\,\psi \circ \iota_{2}\,) \circ f_{2}
	\end{eqnarray*}
	That \,$f_{2}$\, is an epimorphism (``cancellable from the right'') now implies that
	\,$\varphi \circ \iota_{2} \,=\, \psi \circ \iota_{2}$,\,
	which completes the proof of Claim 1.

	\vskip 0.3cm
	\noindent
	\textbf{Claim 2:}\quad $\varphi \,=\, \psi$
	\vskip 0.01cm
	Proof of Claim 2:\;\; By Claim 1, we may define:
	\,$h_{2} \, := \, \varphi \,\circ\, \iota_{2} \,=\, \psi \,\circ\, \iota_{2}$.\,
	Now, observe:
	\begin{equation*}
	h_{1} \circ f_{1}
	\; = \;
		(\,\varphi \circ \iota_{1}\,) \circ f_{1}
	\; = \;
		\varphi \circ (\,\iota_{1} \circ f_{1}\,)
	\; = \;
		\varphi \circ (\,\iota_{2} \circ f_{2}\,)
	\; = \;
		(\,\varphi \circ \iota_{2}\,) \circ f_{2}
	\; = \;
		h_{2} \circ f_{2}\,,
	\end{equation*}
	which implies -- by the universal property of the fiber coproduct --
	the existence of a unique
	\begin{equation*}
	\theta \;\in\; \Mor_{\mathfrak{C}}\!\left(\,
		X_{1}\sqcup_{Y}\!X_{2}
		\,\overset{{\color{white}\textnormal{\large1}}}{,}\,
		A
		\,\right)
	\end{equation*}
	such that
	\,$\theta \circ \iota_{1} \,=\, h_{1}$\,
	and
	\,$\theta \circ \iota_{2} \,=\, h_{2}$.\,
	But now the uniqueness of \,$\theta$\, implies
	\,$\varphi = \theta = \psi$.\,
	This proves Claim 2.

	\vskip 0.3cm
	\noindent
	This completes the proof that \,$\pi_{1}$\, is an epimorphism, as required.

	\vskip 0.3cm
	\noindent
	The situation is depicted in the following diagram:
	\begin{center}
	\begin{tikzcd}
	&&&& {\color{blue}X_{1}}
		\arrow[ddll, "{\color{red}\iota_{1}}", red, swap]
		\arrow[ddllll, bend left = -30, "h_{1}", swap]
	\\ \\
	A
	&&
	{\color{red}X_{1} \sqcup_{Y}\! X_{2}}
		\arrow[ll, dashed, "\;\;\;\exists\,!\,\theta", swap]
		\arrow[ll, bend left =  27, "\psi"]
		\arrow[ll, bend left = -27, "\varphi", swap]
	&&&&
	{\color{blue}Y}
		\arrow[uull, "f_{1}", blue, swap]
		\arrow[ddll, "f_{2}", blue]
	\\ \\
	&&&& {\color{blue}X_{2}}
		\arrow[uull, "{\color{red}\iota_{2}}", red]
		\arrow[uullll, bend left = 30, "h_{2}"]
	\end{tikzcd}
	\end{center}
	\qed
\end{enumerate}

          %%%%% ~~~~~~~~~~~~~~~~~~~~ %%%%%

% \vskip 0.5cm
% \begin{proposition}[p.176, \cite{kashiwara2005categories}]
% \label{FiberProductFiberCoproductExistInAbelianCategories}
% \mbox{}
% \vskip 0.1cm
% \noindent
% In each abelian category,
% fiber products and fiber coproducts always exist.
% \end{proposition}
% \proof
% \begin{enumerate}
% \item
% 	We first prove the existence of fiber products in an arbitrary abelian category.
% 	Suppose \,$\mathfrak{A}$\, is an abelian category,
% 	\,$f_{1} \in \Mor_{\mathfrak{A}}(X_{1},Y)$\,
% 	and
% 	\,$f_{2} \in \Mor_{\mathfrak{A}}(X_{1},Y)$.\,
% 	Since \,$\mathfrak{A}$\, is an abelian category, the finite {\color{red}biproduct}
% 	\begin{equation*}
% 	\left(\,
% 		X_{1} \overset{{\color{white}.}}{\oplus} X_{2}
% 		\,,\,
% 		p_{1}
% 		\,,\,
% 		p_{2}
% 		\,,\,
% 		\iota_{1}
% 		\,,\,
% 		\iota_{2}
% 		\,\right)
% 	\end{equation*}
% 	of \,$X_{1},\, X_{2}$\, exists, where
% 	\,$p_{1} : X_{1} \oplus X_{2} \longrightarrow X_{1}$,\,
% 	\,$p_{2} : X_{1} \oplus X_{2} \longrightarrow X_{2}$,\,
% 	\,$\iota_{1} : X_{1} \longrightarrow X_{1} \oplus X_{2}$,\,
% 	\,$\iota_{2} : X_{2} \longrightarrow X_{1} \oplus X_{2}$\,
% 	satisfy:
% 	\begin{equation*}
% 	p_{1} \circ \iota_{1} = 1_{X_{1}}\,,
% 	\quad
% 	p_{2} \circ \iota_{2} = 1_{X_{2}}\,,
% 	\quad
% 	p_{1} \circ \iota_{2} = 0_{X_{2},X_{1}}\,,
% 	\quad
% 	p_{2} \circ \iota_{1} = 0_{X_{1},X_{2}}
% 	\end{equation*}
% 	\vskip 0.3cm
% 	\noindent
% 	\textbf{Claim 1:}\quad
% 	Let \,$\kappa \,\in\, \Mor_{\mathfrak{A}}(K, X_{1}\oplus X_{2})$\,
% 	be any {\color{red}kernel of
% 	\,$\varphi\,:=\,f_{1}\,\circ\,p_{1}\,-\,f_{2}\,\circ\,p_{2}\,\in\,\Mor_{\mathfrak{A}}(\,X_{1} \oplus X_{2}\,,\,Y\,)$}.\,
% 	Define
% 	\,$\pi_{1} := p_{1} \circ \kappa \in \Mor_{\mathfrak{A}}(K,X_{1})$\,
% 	and
% 	\,$\pi_{2} := p_{2} \circ \kappa \in \Mor_{\mathfrak{A}}(K,X_{2})$.\,
% 	Then,
% 	\,$\left(\,K\,,\,\pi_{1}\,,\,\pi_{2}\,\right)$\,
% 	is a fiber product of
% 	\,$f_{1}$\,
% 	and
% 	\,$f_{2}$.\,
% 	\vskip 0.2cm
% 	\noindent
% 	Proof of Claim 1:\;\;
% 	First, observe that:
% 	\begin{eqnarray*}
% 	f_{1} \circ \pi_{1} - f_{2} \circ \pi_{2}
% 	& = &
% 		f_{1} \circ (\,p_{1} \circ \kappa\,) - f_{2} \circ (\,p_{2} \circ \kappa\,)
% 	\;\; = \;\;
% 		(\,f_{1} \circ p_{1}\,) \circ \kappa - (\,f_{2} \circ p_{2}\,) \circ \kappa
% 	\\
% 	& = &
% 		(\,f_{1} \circ p_{1} - f_{2} \circ p_{2}\,) \circ \kappa
% 	\;\; = \;\;
% 		\varphi \circ \kappa
% 	\\
% 	& = &
% 		0
% 	\end{eqnarray*}
% 	Next, let
% 	\,$h_{1} \in \Mor_{\mathfrak{A}}(A,X_{1})$\,
% 	and
% 	\,$h_{2} \in \Mor_{\mathfrak{A}}(A,X_{2})$\,
% 	be such that
% 	\,$f_{1} \circ h_{1} \,=\, f_{2} \circ h_{2}$.\,
% 	Define
% 	{\color{red}
% 	\begin{equation*}
% 	\psi
% 	\;\; := \;\;
% 		\iota_{1}\circ h_{1}\,+\,\iota_{2}\circ h_{2}
% 	\;\; \in \;\;
% 		\Mor_{\mathfrak{A}}(\,A\,,\,X_{1} \oplus X_{2}\,)
% 	\end{equation*}
% 	}
% 	Observe:
% 	\begin{eqnarray*}
% 	\varphi \circ \psi
% 	& = &
% 		\left(\,f_{1} \circ\, p_{1} \,\overset{{\color{white}.}}{-}\, f_{2} \circ\, p_{2}\,\right)
% 		\,\circ\,
% 		\left(\,\iota_{1} \circ h_{1} \,\overset{{\color{white}.}}{+}\, \iota_{2} \circ h_{2}\,\right)
% 	\\
% 	& = &
% 		f_{1} \,\circ\, p_{1} \,\circ\, \iota_{1} \,\circ\, h_{1}
% 		\; + \;
% 		f_{1} \,\circ\, p_{1} \,\circ\, \iota_{2} \,\circ\, h_{2}
% 		\; - \;
% 		f_{2} \,\circ\, p_{2} \,\circ\, \iota_{1} \,\circ\, h_{1}
% 		\; - \;
% 		f_{2} \,\circ\, p_{2} \,\circ\, \iota_{2} \,\circ\, h_{2}
% 	\\
% 	& \overset{{\color{white}1}}{=} &
% 		f_{1} \,\circ\, 1_{X_{1}} \,\circ\, h_{1}
% 		\; + \;
% 		f_{1} \circ\, 0_{X_{2},X_{1}} \,\circ\, h_{2}
% 		\; - \;
% 		f_{2} \circ\, 0_{X_{1},X_{2}} \,\circ\, h_{1}
% 		\; - \;
% 		f_{2} \,\circ\, 1_{X_{2}} \,\circ\, h_{2}
% 	\\
% 	& \overset{{\color{white}1}}{=} &
% 		f_{1} \,\circ\, h_{1}
% 		\; - \;
% 		f_{2} \,\circ\, h_{2}
% 	\;\; = \;\;
% 		0
% 	\end{eqnarray*}
% 	Thus, by the universal property of
% 	\,$\kappa \,\in\, \Mor_{\mathfrak{A}}(K,X_{1} \oplus X_{2})$\,
% 	as kernel of
% 	\,$\varphi\,:=\,f_{1}\,\circ\,p_{1}\,-\,f_{2}\,\circ\,p_{2}\,\in\,\Mor_{\mathfrak{A}}(X_{1} \oplus X_{2},Y)$,\,
% 	there exists a unique
% 	\,$\theta \in \Mor_{\mathfrak{A}}(A,K)$\,
% 	such that
% 	\,$\psi = \kappa \circ \theta$.\,
% 	Furthermore, \,$\theta$\, satisfies the following two equalities:
% 	%\vskip 0.1cm
% 	\begin{eqnarray*}
% 	\pi_{1} \circ \theta
% 	& = &
% 		(\,p_{1} \circ \kappa\,) \circ \theta
% 	\;\; = \;\;
% 		p_{1} \circ (\,\kappa \circ \theta\,)
% 	\;\; = \;\;
% 		p_{1} \circ \psi
% 	\;\; = \;\;
% 		p_{1} \circ (\,\iota_{1}\circ h_{1}\,+\,\iota_{2}\circ h_{2}\,)
% 	\\
% 	& = &
% 		(\,p_{1} \circ \iota_{1}\,) \circ h_{1}
% 		\,+\
% 		(\,p_{1} \circ \iota_{2}\,) \circ h_{2}
% 	\;\; = \;\;
% 		1_{X_{1}} \circ h_{1}
% 		\,+\
% 		0_{X_{2},X_{1}} \circ h_{2}
% 	\\
% 	& = &
% 		h_{1}
% 	\\ \\
% 	\pi_{2} \circ \theta
% 	& = &
% 		(\,p_{2} \circ \kappa\,) \circ \theta
% 	\;\; = \;\;
% 		p_{2} \circ (\,\kappa \circ \theta\,)
% 	\;\; = \;\;
% 		p_{2} \circ \psi
% 	\;\; = \;\;
% 		p_{2} \circ (\,\iota_{1}\circ h_{1}\,+\,\iota_{2}\circ h_{2}\,)
% 	\\
% 	& = &
% 		(\,p_{2} \circ \iota_{1}\,) \circ h_{1}
% 		\,+\
% 		(\,p_{2} \circ \iota_{2}\,) \circ h_{2}
% 	\;\; = \;\;
% 		0_{X_{1},X_{2}} \circ h_{1}
% 		\,+\,
% 		1_{X_{2}} \circ h_{2}
% 	\\
% 	& = &
% 		h_{2}
% 	\end{eqnarray*}
% 	% It remains to establish the uniqueness of \,$\theta$.\,
% 	% To this end, suppose
% 	% \,$\theta^{\prime} \in \Mor_{\mathfrak{A}}(A,K)$\,
% 	% is any morphism that satisfies
% 	% \begin{equation*}
% 	% \psi \,=\, \kappa \circ \theta^{\prime}\,,
% 	% \quad
% 	% \pi_{1} \circ \theta^{\prime} \,=\, h_{1}\,,
% 	% \quad
% 	% \pi_{2} \circ \theta^{\prime} \,=\, h_{2}\,,
% 	% \end{equation*}
% 	The above arguments are summarized in the following commutative diagram:
% 	\begin{center}
% 	\begin{tikzcd}
% 	&&&& {\color{blue}X_{1}}
% 		\arrow[ddrr, "f_{1}", blue]
% 		\arrow[ddl, bend left = 10, "\iota_{1}", hook, gray]
% 	\\ \\
% 	A
% 		\arrow[rr, dashed, "\;\;\;\exists\,!\,\theta"]
% 		\arrow[uurrrr, bend left =  20, "h_{1}"]
% 		\arrow[ddrrrr, bend left = -20, "h_{2}", swap]
% 		%\arrow[rrr, bend left = -20, "\psi\,:=\,\iota_{1}\circ h_{1} + \iota_{2}\circ h_{2}", swap]
% 		\arrow[rrr, bend left = -20, "\psi", swap]
% 	&&
% 	%{\color{red}X_{1} \sqcap_{Y}\! X_{2}}
% 	{\color{red}K}
% 		\arrow[r, "\kappa", hook, gray]
% 		\arrow[uurr, "{\color{red}\pi_{1}}", bend left =  10, red]
% 		\arrow[ddrr, "{\color{red}\pi_{2}}", bend left = -10, red, swap]
% 	&
% 	{\color{gray}X_{1} \oplus X_{2}}
% 		\arrow[rrr, "\varphi\,:=\,f_{1} \circ\, p_{1} \,-\, f_{2} \circ\, p_{2}", gray]
% 		\arrow[uur, "p_{1}", bend left =  10, gray]
% 		\arrow[ddr, "p_{2}", bend left = -10, gray, swap]
% 	&&&
% 	{\color{blue}Y}
% 	\\ \\
% 	&&&& {\color{blue}X_{2}}
% 		\arrow[uurr, "f_{2}", swap, blue]
% 		\arrow[uul, bend left = -10, "\iota_{2}", hook, gray, swap]
% 	\end{tikzcd}
% 	\end{center}
% 	This completes the proof of Claim 1,
% 	as well as that of the existence of fiber products in an arbitrary abelian category.
% \item
% 	We now prove the existence of fiber coproducts in an arbitrary abelian category.
% 	Suppose \,$\mathfrak{A}$\, is an abelian category,
% 	\,$f_{1} \in \Mor_{\mathfrak{A}}(Y,X_{1})$\,
% 	and
% 	\,$f_{2} \in \Mor_{\mathfrak{A}}(Y,X_{1})$.\,
% 	Since \,$\mathfrak{A}$\, is an abelian category, the finite {\color{red}biproduct}
% 	\begin{equation*}
% 	\left(\,
% 		X_{1} \overset{{\color{white}.}}{\oplus} X_{2}
% 		\,,\,
% 		p_{1}
% 		\,,\,
% 		p_{2}
% 		\,,\,
% 		\iota_{1}
% 		\,,\,
% 		\iota_{2}
% 		\,\right)
% 	\end{equation*}
% 	of \,$X_{1},\, X_{2}$\, exists, where
% 	\,$p_{1} : X_{1} \oplus X_{2} \longrightarrow X_{1}$,\,
% 	\,$p_{2} : X_{1} \oplus X_{2} \longrightarrow X_{2}$,\,
% 	\,$\iota_{1} : X_{1} \longrightarrow X_{1} \oplus X_{2}$,\,
% 	\,$\iota_{2} : X_{2} \longrightarrow X_{1} \oplus X_{2}$\,
% 	satisfy:
% 	\begin{equation*}
% 	p_{1} \circ \iota_{1} = 1_{X_{1}}\,,
% 	\quad
% 	p_{2} \circ \iota_{2} = 1_{X_{2}}\,,
% 	\quad
% 	p_{1} \circ \iota_{2} = 0_{X_{2},X_{1}}\,,
% 	\quad
% 	p_{2} \circ \iota_{1} = 0_{X_{1},X_{2}}
% 	\end{equation*}
% 	\vskip 0.3cm
% 	\noindent
% 	\textbf{Claim 2:}\quad
% 	Let \,$\pi \,\in\, \Mor_{\mathfrak{A}}(\,X_{1}\oplus X_{2}\,,\,Q\,)$\,
% 	be any {\color{red}cokernel of
% 	\,$\varphi\,:=\,\iota_{1}\,\circ\,f_{1}\,-\,\iota_{2}\,\circ\,f_{2}\,\in\,\Mor_{\mathfrak{A}}(\,Y\,,\,X_{1} \oplus X_{2}\,)$}.\,
% 	Define
% 	\,$\alpha_{1} := \pi \circ \iota_{1} \in \Mor_{\mathfrak{A}}(X_{1},Q)$\,
% 	and
% 	\,$\alpha_{2} := \pi \circ \iota_{2} \in \Mor_{\mathfrak{A}}(X_{2},Q)$.\,
% 	Then,
% 	\,$\left(\,Q\,,\,\alpha_{1}\,,\,\alpha_{2}\,\right)$\,
% 	is a fiber coproduct of
% 	\,$f_{1}$\,
% 	and
% 	\,$f_{2}$.\,
% 	\vskip 0.2cm
% 	\noindent
% 	Proof of Claim 2:\;\;
% 	First, observe that:
% 	\begin{eqnarray*}
% 	\alpha_{1} \circ f_{1} - \alpha_{2} \circ f_{2}
% 	& = &
% 		(\,\pi \circ \iota_{1}\,) \circ f_{1} - (\,\pi \circ \iota_{2}\,) \circ f_{2}
% 	\;\; = \;\;
% 		\pi \circ (\,\iota_{1} \circ f_{1}\,) - \pi \circ (\,\iota_{2} \circ f_{2}\,)
% 	\\
% 	& = &
% 		\pi \circ (\,\iota_{1} \circ f_{1} - \iota_{2} \circ f_{2}\,) 
% 	\;\; = \;\;
% 		\pi \circ \varphi
% 	\\
% 	& = &
% 		0
% 	\end{eqnarray*}
% 	Next, let
% 	\,$h_{1} \in \Mor_{\mathfrak{A}}(X_{1},A)$\,
% 	and
% 	\,$h_{2} \in \Mor_{\mathfrak{A}}(X_{2},A)$\,
% 	be such that
% 	\,$h_{1}\,\circ\,f_{1} \,=\, h_{2}\,\circ\,f_{2}$.\,
% 	Define
% 	{\color{red}
% 	\begin{equation*}
% 	\psi
% 	\;\; := \;\;
% 		h_{1}\,\circ\,p_{1}\,+\,h_{2}\,\circ \,p_{2}
% 	\;\; \in \;\;
% 		\Mor_{\mathfrak{A}}(\,X_{1}\oplus X_{2}\,,A\,)
% 	\end{equation*}
% 	}
% 	Observe:
% 	\begin{eqnarray*}
% 	\psi \circ \varphi
% 	& = &
% 		\left(\,h_{1} \circ\, p_{1} \,\overset{{\color{white}.}}{+}\, h_{2} \circ\, p_{2}\,\right)
% 		\,\circ\,
% 		\left(\,\iota_{1} \circ f_{1} \,\overset{{\color{white}.}}{-}\, \iota_{2} \circ f_{2}\,\right)
% 	\\
% 	& = &
% 		h_{1} \,\circ\, p_{1} \,\circ\, \iota_{1} \,\circ\, f_{1}
% 		\; + \;
% 		h_{1} \,\circ\, p_{1} \,\circ\, \iota_{2} \,\circ\, f_{2}
% 		\; - \;
% 		h_{2} \,\circ\, p_{2} \,\circ\, \iota_{1} \,\circ\, f_{1}
% 		\; - \;
% 		h_{2} \,\circ\, p_{2} \,\circ\, \iota_{2} \,\circ\, f_{2}
% 	\\
% 	& \overset{{\color{white}1}}{=} &
% 		h_{1} \,\circ\, 1_{X_{1}} \,\circ\, f_{1}
% 		\; + \;
% 		h_{1} \circ\, 0_{X_{2},X_{1}} \,\circ\, f_{2}
% 		\; - \;
% 		h_{2} \circ\, 0_{X_{1},X_{2}} \,\circ\, f_{1}
% 		\; - \;
% 		h_{2} \,\circ\, 1_{X_{2}} \,\circ\, f_{2}
% 	\\
% 	& \overset{{\color{white}1}}{=} &
% 		h_{1} \,\circ\, f_{1}
% 		\; - \;
% 		h_{2} \,\circ\, f_{2}
% 	\;\; = \;\;
% 		0
% 	\end{eqnarray*}
% 	Thus, by the universal property of
% 	\,$\pi \,\in\, \Mor_{\mathfrak{A}}(\,X_{1} \oplus X_{2}\,,\,Q\,)$\,
% 	as cokernel of
% 	\,$\varphi\,:=\,\iota_{1}\,\circ\,f_{1}\,-\,\iota_{2}\,\circ\,f_{2}\,\in\,\Mor_{\mathfrak{A}}(\,Y\,,\,X_{1} \oplus X_{2}\,)$,\,
% 	there exists a unique
% 	\,$\theta \in \Mor_{\mathfrak{A}}(Q,A)$\,
% 	such that
% 	\,$\psi = \theta \circ \pi$.\,
% 	Furthermore, \,$\theta$\, satisfies the following two equalities:
% 	%\vskip 0.1cm
% 	\begin{eqnarray*}
% 	\theta \circ \alpha_{1}
% 	& = &
% 		\theta \circ (\,\pi \circ \iota_{1}\,)
% 	\;\; = \;\;
% 		(\,\theta \circ \pi\,) \circ \iota_{1}
% 	\;\; = \;\;
% 		\psi \circ \iota_{1} 
% 	\;\; = \;\;
% 		(\,h_{1} \circ p_{1} \,+\, h_{2} \circ p_{2} \,) \circ \iota_{1}
% 	\\
% 	& = &
% 		h_{1} \circ (\,p_{1} \circ \iota_{1}\,)
% 		\,+\
% 		h_{2} \circ (\,p_{2} \circ \iota_{1}\,)
% 	\;\; = \;\;
% 		h_{1} \circ 1_{X_{1}}
% 		\,+\
% 		h_{2} \circ 0_{X_{1},X_{2}}
% 	\\
% 	& = &
% 		h_{1}
% 	\\ \\
% 	\theta \circ \alpha_{2}
% 	& = &
% 		\theta \circ (\,\pi \circ \iota_{2}\,)
% 	\;\; = \;\;
% 		(\,\theta \circ \pi\,) \circ \iota_{2}
% 	\;\; = \;\;
% 		\psi \circ \iota_{2}
% 	\;\; = \;\;
% 		(\,h_{1} \circ p_{1} \,+\, h_{2} \circ p_{2} \,) \circ \iota_{2}
% 	\\
% 	& = &
% 		h_{1} \circ (\,p_{1} \circ \iota_{2}\,)
% 		\,+\
% 		h_{2} \circ (\,p_{2} \circ \iota_{2}\,)
% 	\;\; = \;\;
% 		h_{1} \circ 0_{X_{2},X_{1}}
% 		\,+\
% 		h_{2} \circ 1_{X_{2}}
% 	\\
% 	& = &
% 		h_{2}
% 	\end{eqnarray*}
% 	% It remains to establish the uniqueness of \,$\theta$.\,
% 	% To this end, suppose
% 	% \,$\theta^{\prime} \in \Mor_{\mathfrak{A}}(A,K)$\,
% 	% is any morphism that satisfies
% 	% \begin{equation*}
% 	% \psi \,=\, \kappa \circ \theta^{\prime}\,,
% 	% \quad
% 	% \pi_{1} \circ \theta^{\prime} \,=\, h_{1}\,,
% 	% \quad
% 	% \pi_{2} \circ \theta^{\prime} \,=\, h_{2}\,,
% 	% \end{equation*}
% 	The above arguments are summarized in the following commutative diagram:
% 	\begin{center}
% 	\begin{tikzcd}
% 	&&&& {\color{blue}X_{1}}
% 		\arrow[ddl, bend left = 10, "\iota_{1}", hook, gray]
% 		\arrow[ddllll, bend left = -20, "h_{1}", swap]
% 		\arrow[ddll, "{\color{red}\alpha_{1}}", bend left = -10, red, swap]
% 	\\ \\
% 	A
% 	&&
% 	%{\color{red}X_{1} \sqcup_{Y}\! X_{2}}
% 	{\color{red}Q}
% 		%\arrow[r, "\kappa", hook, gray]
% 		\arrow[ll, dashed, "\;\;\;\exists\,!\,\theta", swap]
% 	&
% 	{\color{gray}X_{1} \oplus X_{2}}
% 		\arrow[l, "\quad\pi", gray, swap]
% 		\arrow[uur, "p_{1}", bend left =  10, gray]
% 		\arrow[ddr, "p_{2}", bend left = -10, gray, swap]
% 		\arrow[lll, bend left = 20, "\psi"]
% 	&&&
% 	{\color{blue}Y}
% 		\arrow[uull, "f_{1}", blue, swap]
% 		\arrow[ddll, "f_{2}", blue]
% 		\arrow[lll, "\varphi\,:=\, \iota_{1} \circ\, f_{1} \,-\, \iota_{2} \circ\, f_{2}", gray, swap]
% 	\\ \\
% 	&&&& {\color{blue}X_{2}}
% 		\arrow[uull, "{\color{red}\alpha_{2}}", bend left = 10, red]
% 		\arrow[uullll, bend left = 20, "h_{2}"]
% 		\arrow[uul, bend left = -10, "\iota_{2}", hook, gray, swap]
% 	\end{tikzcd}
% 	\end{center}
% 	This completes the proof of Claim 2,
% 	as well as that of the existence of fiber coproducts in an arbitrary abelian category.
% 	\qed
% \end{enumerate}

          %%%%% ~~~~~~~~~~~~~~~~~~~~ %%%%%

% \vskip 0.5cm
% \begin{proposition}[Fiber products preserve epimorphisms in abelian categories; Propn. I.20.2, p.34, \cite{Mitchell1965}]
% \mbox{}
% \vskip -0.3cm
% \noindent
% Let \,$\mathfrak{A}$\, be an {\color{red}abelian} category.
% \begin{enumerate}
% \item
% 	\vskip -0.1cm
% 	Suppose that the fiber product
% 	\,$\left(\,X_{1}\overset{{\color{white}.}}{\sqcap_{Y}}\!X_{2} \,,\, \pi_{1} \,,\, \pi_{2}\,\right)$\,
% 	exists for two morphisms
% 	\,$f_{1} \in \Mor_{\mathfrak{A}}(X_{1},Y)$\,
% 	and
% 	\,$f_{2} \in \Mor_{\mathfrak{A}}(X_{2},Y)$,\,
% 	where
% 	\,$X_{1},\, X_{2},\, Y,\, X_{1}\sqcap_{Y}\!X_{2} \,\in\, \Obj(\mathfrak{C})$,\,
% 	\,$\pi_{1} \in \Mor_{\mathfrak{A}}(\,X_{1}\sqcap_{Y}\!X_{2}\,,\,X_{1}\,)$,\,
% 	\,$\pi_{2} \in \Mor_{\mathfrak{A}}(\,X_{1}\sqcap_{Y}\!X_{2}\,,\,X_{2}\,)$.\,
% 	\begin{center}
% 	\begin{tikzcd}
% 	&& {\color{blue}X_{1}}
% 		\arrow[ddrr, "f_{1}", blue]
% 	\\ \\
% 	{\color{red}X_{1} \sqcap_{Y}\! X_{2}}
% 		\arrow[uurr, "{\color{red}\pi_{1}}", red]
% 		\arrow[ddrr, "{\color{red}\pi_{2}}", red, swap]
% 	&&&&
% 	{\color{blue}Y}
% 	\\ \\
% 	&& {\color{blue}X_{2}}
% 		\arrow[uurr, "f_{2}", swap, blue]
% 	\end{tikzcd}
% 	\end{center}
% 	Then, the following statements are true:
% 	\begin{itemize}
% 	\item
% 		If \,$f_{2}$\, is an {\color{red}epi}morphism, then \,$\pi_{1}$\, is an {\color{red}epi}morphism.
% 	\item
% 	If \,$f_{1}$\, is an {\color{red}epi}morphism, then \,$\pi_{2}$\, is an {\color{red}epi}morphism.
% 	\end{itemize}
% \item
% 	Dually, suppose that the fiber coproduct
% 	\,$\left(\,X_{1}\overset{{\color{white}.}}{\sqcup_{Y}}\!X_{2} \,,\, \iota_{1} \,,\, \iota_{2}\,\right)$\,
% 	exists for two morphisms
% 	\,$f_{1} \in \Mor_{\mathfrak{A}}(Y,X_{1})$\,
% 	and
% 	\,$f_{2} \in \Mor_{\mathfrak{A}}(Y,X_{2})$,\,
% 	where
% 	\,$X_{1},\, X_{2},\, Y,\, X_{1}\sqcup_{Y}\!X_{2} \,\in\, \Obj(\mathfrak{C})$,\,
% 	\,$\iota_{1} \in \Mor_{\mathfrak{A}}(\,X_{1}\,,\,X_{1}\sqcup_{Y}\!X_{2}\,)$,\,
% 	\,$\iota_{2} \in \Mor_{\mathfrak{A}}(\,X_{2}\,,\,X_{1}\sqcup_{Y}\!X_{2}\,)$.\,
% 	\begin{center}
% 	\begin{tikzcd}
% 	&& {\color{blue}X_{1}}
% 		\arrow[ddll, "{\color{red}\iota_{1}}", red, swap]
% 	\\ \\
% 	{\color{red}X_{1} \sqcup_{Y}\! X_{2}}
% 	&&&&
% 	{\color{blue}Y}
% 		\arrow[uull, "f_{1}", blue, swap]
% 		\arrow[ddll, "f_{2}", blue]
% 	\\ \\
% 	&& {\color{blue}X_{2}}
% 		\arrow[uull, "{\color{red}\iota_{2}}", red]
% 	\end{tikzcd}
% 	\end{center}
% 	Then, the following statements are true:
% 	\begin{itemize}
% 	\item
% 		If \,$f_{2}$\, is a {\color{red}mono}morphism, then \,$\iota_{1}$\, is a {\color{red}mono}morphism.
% 	\item
% 		If \,$f_{1}$\, is a {\color{red}mono}morphism, then \,$\iota_{2}$\, is a {\color{red}mono}morphism.
% 	\end{itemize}
% \end{enumerate}
% \end{proposition}
% \proof
% \begin{enumerate}
% \item
% \item
% \qed
% \end{enumerate}

          %%%%% ~~~~~~~~~~~~~~~~~~~~ %%%%%
