
          %%%%% ~~~~~~~~~~~~~~~~~~~~ %%%%%

\section{Categories with zero objects}
\setcounter{theorem}{0}
\setcounter{equation}{0}

%\cite{vanDerVaart1996}
%\cite{Kosorok2008}

%\renewcommand{\theenumi}{\alph{enumi}}
%\renewcommand{\labelenumi}{\textnormal{(\theenumi)}$\;\;$}
\renewcommand{\theenumi}{\roman{enumi}}
\renewcommand{\labelenumi}{\textnormal{(\theenumi)}$\;\;$}

          %%%%% ~~~~~~~~~~~~~~~~~~~~ %%%%%

In this section, we establish that a category with zero objects possesses sufficient structure
in order to define the notions of kernels and cokernels of morphisms
(although the kernel or cokernel of any particular morphism may or may not exist).

          %%%%% ~~~~~~~~~~~~~~~~~~~~ %%%%%

\vskip 0.5cm
\noindent
\textbf{\large Basic category-theoretic notions and results}

          %%%%% ~~~~~~~~~~~~~~~~~~~~ %%%%%

\vskip 0.5cm
\begin{definition}[Category]
\mbox{}
\vskip 0.15cm
\noindent
A \,\textbf{category}\, $\mathfrak{C}$ consists of the following:
\begin{itemize}
\item
	a class $\Obj(\mathfrak{C})$,
\item
	a set $\Mor_{\mathfrak{C}}(A,B)$, for each $A, B \in \Obj(\mathfrak{C})$,
\item
	a \textbf{composition map}
	\begin{equation*}
	\Mor_{\mathfrak{C}}(A,B) \times \Mor_{\mathfrak{C}}(B,C) \longrightarrow \Mor_{\mathfrak{C}}(A,C) : (f,g) \longmapsto g \circ f\,,
	\end{equation*}
	for each $A, B, C \in \Obj(\mathfrak{C})$,
\end{itemize}
satisfying:
\begin{enumerate}
\item
	the sets $\Mor_{\mathfrak{C}}(\,\cdot\,,\,\cdot\,)$ are pairwise disjoint, i.e.,
	for $A_{1}, B_{1}, A_{2}, A_{2} \in \Obj(\mathfrak{C})$, we have:
	\begin{equation*}
	\Mor_{\mathfrak{C}}(A_{1},B_{1})\;\bigcap\;\Mor_{\mathfrak{C}}(A_{2},B_{2})
	\; = \;
		\varemptyset\,,
	\;\;
	\textnormal{whenever $A_{1} \neq A_{2}$ or $B_{1} \neq B_{2}$}
	\end{equation*}
\item
	for each $A \in \Obj(\mathfrak{C})$, there exists $1_{A} \in \Mor_{\mathfrak{C}}(A,A)$ such that
	\begin{equation*}
	f \circ 1_{A} = f
	\;\;\;\;\textnormal{and}\;\;\;\;
	1_{B} \circ f = f,
	\quad
	\textnormal{for each $f \in \Mor_{\mathfrak{C}}(A,B)$, \,$B \in \Obj(\mathfrak{C})$}
	\end{equation*}
\item
	the composition map is associative, i.e.,
	\begin{equation*}
	(f \circ g) \circ h
	\;\; = \;\;
	f \circ (g \circ h)\,,
	\end{equation*}
	for each
	$h \in \Mor_{\mathfrak{C}}(A,B)$,
	$g \in \Mor_{\mathfrak{C}}(B,C)$,
	$f \in \Mor_{\mathfrak{C}}(C,D)$,\;
	where
	\,$A,B,C,D \in \Obj(\mathfrak{C})$.
\end{enumerate}
The elements of $\Obj(\mathfrak{C})$ are called the \textbf{objects} of $\mathfrak{C}$.
For $A, B \in \Obj(\mathfrak{C})$, the elements of  $\Mor_{\mathfrak{C}}(A,B)$
are called the \textbf{morphisms} in $\mathfrak{C}$ from $A$ to $B$.
For each $A \in \Obj(\mathfrak{C})$, $1_{A} \in \Mor_{\mathfrak{C}}(A,A)$ is called
the \textbf{identity morphism} of $A$.
\end{definition}

          %%%%% ~~~~~~~~~~~~~~~~~~~~ %%%%%

\vskip 0.5cm
\begin{definition}[Monomorphism, epimorphism]
\mbox{}
\vskip 0.1cm
\noindent
Let $\mathfrak{C}$ be a category, and $A, B \in \Obj(\mathfrak{C})$.
A morphism $f \in \Mor_{\mathfrak{C}}(A,B)$ is called
\begin{itemize}
\item
	a \textbf{monomorphism} if it can be {\color{red}left} cancelled in the sense that
	$f \circ g = f \circ h$  implies that $g = h$.
\item
	a \textbf{epimorphism} if it can be {\color{red}right} cancelled in the sense that
	$g \circ f = h \circ f$  implies that $g = h$.
\item
	an \textbf{isomorphism} if there is a morphism $g \in \Mor_{\mathfrak{C}}(B,A)$
	such that $g \circ f = 1_{A}$ and $f \circ g = 1_{B}$.
	Clearly, such a morphism $g$ is unique, and it is called the inverse of $f$.
	The inverse of $f$, if exists, is denoted by $f^{-1}$.
\end{itemize}
\end{definition}

\vskip 0.5cm
\begin{definition}[Images and coimages of morphisms in an arbitrary category]
\mbox{}
\vskip 0.15cm
\noindent
Let \,$\mathfrak{C}$\, be a category, and $f : A \longrightarrow B$ a morphism in $\mathfrak{C}$.
\begin{itemize}
\item
	\vskip -0.1cm
	An \textbf{image} of $f$ (if it exists) is a monomorphism $\iota : I \longrightarrow B$ in $\mathfrak{C}$ such that
	\begin{itemize}
	\item
		$f = \iota \circ \nu$,\, for some \,$\nu \in \Mor_{\mathfrak{C}}(A,I)$, and
	\item
		for each factorization $f = \iota^{\prime} \circ \nu^{\prime}$ of $f$
		where $\iota^{\prime} \in \Mor_{\mathfrak{C}}(I^{\prime},B)$ is a monomorphism,
		there exists a unique $\theta \in \Mor_{\mathfrak{C}}(I,I^{\prime})$ such that the following diagram commutes:
		\begin{center}
		\begin{tikzcd}
		A
			\arrow[rr, "f"]
			\arrow[dr, swap, "\nu"]
			\arrow[ddr, bend right, swap, "\nu^{\prime}"]
		&&
		B
		\\
		& {\color{red}I}
			\arrow[ur, hook, swap, "\iota", red]
			\arrow[d, dashed, "\;\exists ! \, \theta"]
		\\
		& I^{\prime}
			\arrow[uur, bend right, hook, swap, "\iota^{\prime}"]
		&
		\end{tikzcd}
		\end{center}
	\end{itemize}
\item
	A \textbf{coimage} of $f$ (if it exists) is an epimorphism $\varepsilon : A \longrightarrow C$ in $\mathfrak{C}$ such that
	\begin{itemize}
	\item
		$f \,=\, \mu \circ \varepsilon$,\, for some $\mu \in \Mor_{\mathfrak{C}}(C,B)$, and
	\item
		for each factorization $f = \mu^{\prime} \circ \varepsilon^{\prime}$ of $f$
		where $\varepsilon^{\prime} \in \Mor_{\mathfrak{C}}(J,B)$ is an epimorphism,
		there exists a unique $\theta \in \Mor_{\mathfrak{C}}(C^{\prime},C)$ such that the following diagram commutes:
		% \begin{center}
		% \begin{tikzcd}
		% A
		% 	\arrow[rr, "f"]
		% 	\arrow[dr, two heads, swap, "\varepsilon", red]
		% 	\arrow[ddr, two heads, bend right, swap, "\varepsilon^{\prime}"]
		% &&
		% B
		% \\
		% & {\color{red}C}
		% 	\arrow[ur, swap, "\mu"]
		% \\
		% & C^{\prime}
		% 	\arrow[u, dashed, swap, "\;\exists ! \, \theta"]
		% 	\arrow[uur, bend right, swap, "\mu^{\prime}"]
		% &
		% \end{tikzcd}
		% \end{center}
		\begin{center}
		\begin{tikzcd}
		B
		&&
		A
			\arrow[ll, swap, "f"]
			\arrow[dl, two heads, "\varepsilon", red]
			\arrow[ddl, two heads, bend left, "\varepsilon^{\prime}"]
		\\
		& {\color{red}C}
			\arrow[ul, "\mu"] &
		\\
		& C^{\prime}
			\arrow[u, dashed, swap, "\;\exists ! \, \theta"]
			\arrow[uul, bend left, "\mu^{\prime}"]
		&
		\end{tikzcd}
		\end{center}
	\end{itemize}
\end{itemize}
\end{definition}

          %%%%% ~~~~~~~~~~~~~~~~~~~~ %%%%%

\vskip 0.5cm
\begin{lemma}\label{fTildeIsMonomorphismWheneverFIs}
\mbox{}
\vskip 0.1cm
\noindent
Let $\mathfrak{C}$ be a category, $A, B \in \Obj(\mathfrak{C})$, and $f \in \Mor_{\mathfrak{C}}(A,B)$.
\vskip 0.1cm
\noindent
If
\begin{itemize}
\item
	$f \,=\, \iota \circ \widetilde{f}$, where $\iota$ is an image of $f$,
\item
	$f$\, is a monomorphism,
\end{itemize}
then: \,$\widetilde{f}$\, is itself a monomorphism.
\end{lemma}
\proof
Suppose \,$\widetilde{f} \circ g \,=\, \widetilde{f} \circ h$.\,
We need to show that \,$g \,=\, h$.\,
Now,
\begin{eqnarray*}
\widetilde{f} \circ g \,=\, \widetilde{f} \circ h
&\Longrightarrow&
	\iota \circ \widetilde{f} \circ g \,=\, \iota \circ \widetilde{f} \circ h
\\
&\Longrightarrow&
	f \circ g \,=\, f \circ h
\\
&\Longrightarrow&
	g \,=\, h,
	\quad\textnormal{since \,$f$\, is a monomorphism and can be cancelled on the left}
\end{eqnarray*}
This completes the proof of the Lemma.
\qed

          %%%%% ~~~~~~~~~~~~~~~~~~~~ %%%%%

\vskip 0.5cm
\begin{definition}[Equalizer]
\mbox{}
\vskip 0.1cm
\noindent
Let $\mathfrak{C}$ be a category, $A, B \in \Obj(\mathfrak{C})$, and $f, g \in \Mor_{\mathfrak{C}}(A,B)$.
\vskip 0.1cm
\noindent
An \textbf{equalizer} of $f$ and $g$ is a morphism $\gamma \in \Mor_{\mathfrak{C}}(E,A)$ such that
\begin{itemize}
\item
	$f \circ \gamma = g \circ \gamma$, and
\item
	for each morphism $q \in \Mor_{\mathfrak{C}}(X,A)$ satisfying \,$f \circ q = g \circ q$,\,
	there exists a unique morphism $\theta \in \Mor_{\mathfrak{C}}(X,E)$ such that
	$q = \gamma \circ \theta$.
	\begin{center}
	\begin{tikzcd}
	{\color{red}E}
		\arrow[rr, "\gamma", red]
	&&
	A
		\arrow[rr, shift left, "f"]
		\arrow[rr, shift right, swap, "g"]
	&&
	B
	\\
	&
	X
		\arrow[ur, swap, "q"]
		\arrow[ul, dashed, "\exists !\;\theta"]
	\end{tikzcd}
	\end{center}
\end{itemize}
The category $\mathfrak{C}$ is said to be \textbf{with equalizers}
if an equalizer exists for each pair of morphisms with the same domain and codomain.
\end{definition}

          %%%%% ~~~~~~~~~~~~~~~~~~~~ %%%%%

\vskip 0.5cm
\begin{lemma}\label{EqualizersAreMonomorphisms}
\mbox{}
\vskip 0.1cm
\noindent
Every equalizer is a monomorphism.
\end{lemma}
\proof
Let $\mathfrak{C}$ be an arbitrary category, $A, B \in \Obj(\mathfrak{C})$, $f, g \in \Mor_{\mathfrak{C}}(A,B)$, and
$\gamma \in \Mor_{\mathfrak{C}}(E,A)$ be an equalizer of $f$ and $g$.
We need to show that $\gamma$ is a monomorphism, i.e.,
for every pair of morphisms $h_{1}, h_{2} \in \Mor_{\mathfrak{C}}(X,E)$
satisfying \,$\gamma \circ h_{1} = \gamma \circ h_{2}$,\,
we in fact must have: $h_{1} = h_{2}$.

\vskip 0.3cm
\noindent
To this end, define \,$q \,:=\, \gamma \circ h_{1} \,=\, \gamma \circ h_{2}$.\,

\vskip 0.3cm
\noindent
\textbf{Claim 1:}\; $f \circ q \,=\, f \circ q$.
\vskip 0.1cm
\noindent
Proof of Claim 1:\;
Note: $f \circ q \,=\, f \circ (\gamma \circ h_{1}) \,=\, (f \circ \gamma) \circ h_{1} \,=\, (g \circ \gamma) \circ h_{1} \,=\, g \circ (\gamma \circ h_{1}) \,=\, g \circ q$, as required.

\vskip 0.3cm
\noindent
\textbf{Claim 2:}\;\; There exists a unique \,$\theta \in \Mor_{\mathfrak{C}}(X,E)$\, such that \,$q = \gamma \circ \theta$.
\vskip 0.1cm
\noindent
Proof of Claim 2:\;
This follows immediately from two facts:
(a)	\,$f \circ q \,=\, f \circ q$ (Claim 1), and
(b)	the universal property of \,$\gamma$\, as an equalizer of $f$ and $g$.
\begin{center}
\begin{tikzcd}
{\color{red}E}
	\arrow[rr, "\gamma", red]
&&
A
	\arrow[rr, shift left, "f"]
	\arrow[rr, shift right, swap, "g"]
&&
B
\\
&
X
	\arrow[ur, swap, "q"]
	\arrow[ul, dashed, "\exists !\;\theta"]
\end{tikzcd}
\end{center}

\vskip 0.3cm
\noindent
\textbf{Claim 3:}\;\;  $h_{1} \,=\, \theta \,=\, h_{2}$
\vskip 0.1cm
\noindent
Proof of Claim 3:\;
Recall that \,$q \,:=\, \gamma \circ h_{1} \,=\, \gamma \circ h_{2}$.\,
Hence, we have the following augmented commutative diagram:
\begin{center}
\begin{tikzcd}
{\color{red}E}
	\arrow[rr, "\gamma", red]
&&
A
	\arrow[rr, shift left, "f"]
	\arrow[rr, shift right, swap, "g"]
&&
B
\\
&
X
	\arrow[ur, "q"]
	\arrow[ul, dashed, "\exists !\;\theta"]
	\arrow[dr, shift left, "h_{1}", blue]
	\arrow[dr, shift right, swap, "h_{2}", blue]
\\
&&
{\color{blue}E}
	\arrow[uu, swap, "\gamma", blue]
\end{tikzcd}
\end{center}
Now, note that in the above diagram,
$\theta$ can be replaced with either $h_{1}$ or $h_{2}$ while maintaining commutativity of the diagram.
Therefore, the uniqueness of $\theta$ implies: $h_{1} = \theta = h_{2}$.
This proves Claim 3, as well as completes the proof of the Lemma.
\qed

          %%%%% ~~~~~~~~~~~~~~~~~~~~ %%%%%

\vskip 0.5cm
\begin{lemma}\label{fTildeIsEpimorphism}
\mbox{}
\vskip 0.1cm
\noindent
Let $\mathfrak{C}$ be a category, $A, B \in \Obj(\mathfrak{C})$, and $f \in \Mor_{\mathfrak{C}}(A,B)$.
\vskip 0.1cm
\noindent
If
\begin{itemize}
\item
	$\mathfrak{C}$ is a category with equalizers,
\item
	$f \,=\, \iota \circ \widetilde{f}$, where $\iota$ is an image of $f$,
\end{itemize}
then: \,$\widetilde{f}$\, is an epimorphism.
\end{lemma}
\proof
Suppose \,$g_{1} \circ \widetilde{f} \,=\, g_{2} \circ \widetilde{f}$,\,
where \,$g_{1}, g_{2} \in \Mor_{\mathfrak{C}}(I,C)$.\,
We need to prove that \,$g_{1} \,=\, g_{2}$.\,
\vskip 0.3cm
\noindent
Since $\mathfrak{C}$ is a category with equalizers,
we may fix an equalizer \,$\gamma \in \Mor_{\mathfrak{C}}(E,I)$\, of \,$g_{1}$\, and \,$g_{2}$.
\begin{center}
\begin{tikzcd}
&
A
	\arrow[rr, "f"]
	\arrow[dr, "\widetilde{f}"]
	\arrow[dl, dashed, swap, "\exists !\; \theta", red]
&&
B
\\
E
	\arrow[rr, thick, hook, "\gamma"]
&&
I
	\arrow[ur, hook, swap, "\iota"]
	\arrow[dr, shift left, "g_{1}"]
	\arrow[dr, shift right, swap, "g_{2}"]
	\arrow[ll, thick, dashed, bend left = 50, "\exists ! \; \varphi", blue]
\\
&&&
C
\end{tikzcd}
\end{center}

\vskip 0.3cm
\noindent
\textbf{Claim 1:}\;\; There exists a unique \,$\theta \in \Mor_{\mathfrak{C}}(A,E)$\, such that \,$\widetilde{f} \,=\, \gamma \circ \theta$.
\vskip 0.1cm
\noindent
Proof of Claim 1:\; This follows immediately from the hypothesis that
\,$g_{1} \circ \widetilde{f} \,=\, g_{2} \circ \widetilde{f}$,\,
and the universal property of \,$\gamma$\, as an equalizer of \,$g_{1}$\, and \,$g_{2}$.

\vskip 0.3cm
\noindent
\textbf{Claim 2:}\;\; There exists a unique \,$\varphi \in \Mor_{\mathfrak{C}}(I,E)$\, such that
\,$\theta \,=\, \varphi \circ \widetilde{f}$\,
and
\,$\iota \,=\, (\,\iota \circ \gamma\,) \circ \varphi$.\,
\vskip 0.1cm
\noindent
Proof of Claim 2:\; 
By Lemma \ref{EqualizersAreMonomorphisms}, we see that \,$\gamma$\, is a monomorphism,
hence, $\iota \circ \gamma$ is itself a monomorphism, being a composition of two monomorphisms.
Claim 2 now follows immediately from the fact that
\begin{equation*}
f
\;\; = \;\;
	\iota \circ \widetilde{f}
\;\; = \;\;
	\iota \circ (\,\gamma \circ \theta\,)
\;\; = \;\;
	(\,\iota \circ \gamma\,) \circ \theta
\end{equation*}
with \,$\iota \circ \gamma$\, being a monomphism,
and the universal property of \,$\iota$\, as an image of \,$f$.\,
See the following diagram:
\begin{center}
\begin{tikzcd}
A
	\arrow[rr, "f"]
	\arrow[dr, swap, "\widetilde{f}"]
	\arrow[ddr, dashed, bend right, swap, "\theta", red]
&&
B
\\
& I
	\arrow[ur, hook, swap, "\iota"]
	\arrow[d, dashed, "\;\exists ! \, \varphi", blue]
\\
& E
	\arrow[uur, bend right, hook, swap, "\,\iota\,\circ\,\gamma"]
&
\end{tikzcd}
\end{center}

\vskip 0.3cm
\noindent
\textbf{Claim 3:}\;\; $\gamma \circ \varphi \,=\, 1_{I}$\, and \,$\varphi \circ \gamma \,=\, 1_{E}$;
hence, \,$\gamma$\, and \,$\varphi$\, are isomorphisms and are inverses of each other.
\vskip 0.1cm
\noindent
Proof of Claim 3:\;  Note:
\,$\iota \circ (\,\gamma \circ \varphi\,)$
\;$=$\; $(\,\iota \circ \gamma\,) \circ \varphi$
\;$=$\; $\iota$
\;$=$\; $\iota \circ 1_{I}$,\,
which implies
\,$\gamma \circ \varphi \,=\, 1_{I}$,\,
since the image \,$\iota$\, is a monomorphism (being an image of $f$) and can be cancelled on the left.
This proves the first equality.
For the second equality, note:
\,$\gamma \circ 1_{E}$
\;$=$\; $\gamma$
\;$=$\; $1_{I} \circ \gamma$
\;$=$\; $(\,\gamma \circ \varphi\,) \circ \gamma$
\;$=$\; $\gamma \circ (\,\varphi \circ \gamma\,)$,\,
which implies
\,$1_{E} \,=\, \varphi \circ \gamma$,\,
since \,$\gamma$\, is a monomorphism (being an equalizer; see Lemma \ref{EqualizersAreMonomorphisms}).

\vskip 0.4cm
\noindent
\textbf{Claim 4:}\;\; $g_{1} \,=\, g_{2}$.
\vskip 0.1cm
\noindent
Proof of Claim 4:\;  Recall that \,$\gamma$\, is an equalizer of \,$g_{1}$\, and \,$g_{2}$.\,
Hence, we have: \,$g_{1} \circ \gamma \,=\, g_{2} \circ \gamma$.\,
By Claim 3, \,$\gamma$\, is an isomorphism, and hence can be cancelled on the right,
which yields \,$g_{1} \,=\, g_{2}$,\, as required.

\vskip 0.3cm
\noindent
The proof of the present Lemma is complete.
\qed

          %%%%% ~~~~~~~~~~~~~~~~~~~~ %%%%%

\vskip 1.0cm
\noindent
\textbf{\large Zero morphisms in a category with zero objects}

          %%%%% ~~~~~~~~~~~~~~~~~~~~ %%%%%

\vskip 0.5cm
\begin{definition}[Initial object, terminal object, zero object]
\mbox{}
\vskip 0.15cm
\noindent
Let \,$\mathfrak{C}$\, be a category.
\begin{itemize}
\item
	An object $A \in \Obj(\mathfrak{C})$ is called an \textbf{initial object} if $\Mor_{\mathfrak{C}}(A,X)$ is a singleton set,
	for each $X \in \Obj(\mathfrak{C})$.
\item
	An object $\Omega \in \Obj(\mathfrak{C})$ is called a \textbf{terminal object} if $\Mor_{\mathfrak{C}}(X,\Omega)$ is a singleton set,
	for each $X \in \Obj(\mathfrak{C})$.
\item
	An object of $\Obj(\mathfrak{C})$ is called a \textbf{zero object} if it is both initial and terminal.
\end{itemize}
\end{definition}

          %%%%% ~~~~~~~~~~~~~~~~~~~~ %%%%%

\vskip 0.5cm
\begin{lemma}[Existence and uniqueness of zero morphisms in a category with zero objects]\label{ZeroMorphisms}
\mbox{}
\vskip 0.15cm
\noindent
Let \,$\mathfrak{C}$\, be a category with zero objects.
Then, for each $A, B \in \Obj(\mathfrak{C})$, we have:
\begin{equation*}
\iota_{0_{1},B}\circ\iota_{A,0_{1}}
\;\; = \;\;
	\iota_{0_{2},B}\circ\iota_{A,0_{2}}
\;\; \in \;\;
	\Mor_{\mathfrak{C}}(A,B),
\end{equation*}
where
$0_{1}, 0_{2} \in \Obj(\mathfrak{C})$ are two arbitrary zero objects,
$\iota_{A,0_{k}}$ is the unique element in $\Mor_{\mathfrak{C}}(A,0_{k})$,
$\iota_{0_{k},B}$ is the unique element in $\Mor_{\mathfrak{C}}(0_{k},B)$,
$k = 1,2$.
\end{lemma}
\proof
Note that $\iota_{0_{1},0_{2}}\circ\iota_{A,0_{1}}$ and $\iota_{A,0_{2}}$ are both elements of $\Mor_{\mathfrak{C}}(A,0_{2})$,
which is a singleton set since $0_{2}$ is a zero object. Hence, we conclude:
\begin{equation*}
\iota_{0_{1},0_{2}}\circ\iota_{A,0_{1}} \; = \; \iota_{A,0_{2}} \;\; \in \;\; \Mor_{\mathfrak{C}}(A,0_{2})
\end{equation*}
Similarly,
\begin{equation*}
\iota_{0_{2},B}\circ\iota_{0_{1},0_{2}} \; = \; \iota_{0_{1},B} \;\; \in \;\; \Mor_{\mathfrak{C}}(0_{1},B)
\end{equation*}
Hence,
\begin{equation*}
\iota_{0_{1},B} \circ \iota_{A,0_{1}}
\; = \;
	(\,\iota_{0_{2},B}\circ\iota_{0_{1},0_{2}}\,) \circ \iota_{A,0_{1}}
\; = \;
	\iota_{0_{2},B} \circ (\,\iota_{0_{1},0_{2}} \circ \iota_{A,0_{1}}\,)
\; = \;
	\iota_{0_{2},B} \circ \iota_{A,0_{2}}
\end{equation*}
\qed

\vskip 0.5cm
\noindent
Lemma \ref{ZeroMorphisms} implies that the following is well-defined:

          %%%%% ~~~~~~~~~~~~~~~~~~~~ %%%%%

\vskip 0.5cm
\begin{definition}[Zero morphisms in a category with zero objects]
\mbox{}
\vskip 0.15cm
\noindent
Let \,$\mathfrak{C}$\, be a category with zero objects, and $A, B \in \Obj(\mathfrak{C})$.
The \textbf{zero morphism} in $\Mor_{\mathfrak{C}}(A,B)$ is, by definition,
\begin{equation*}
0_{A,B}
\; := \;
	\iota_{0,B} \,\circ\, \iota_{A,0}
\;\; \in \;\;
	\Mor_{\mathfrak{C}}(A,B),
\end{equation*}
where $0$ denotes an arbitrary zero object of $\mathfrak{C}$.
\end{definition}

          %%%%% ~~~~~~~~~~~~~~~~~~~~ %%%%%

\vskip 0.5cm
\begin{lemma}[Compositions with zero morphisms are themselves zero morphisms]
\mbox{}
\vskip 0.15cm
\noindent
Let \,$\mathfrak{C}$\, be a category with zero objects.
Then, for each \,$A, B \in \Obj(\mathfrak{C})$\, and \,$f \in \Mor_{\mathfrak{C}}(A,B)$,\, we have:
\begin{enumerate}
\item
	$f \circ 0_{X,A} \,=\, 0_{X,B}$,\, for each $X \in \Obj(\mathfrak{C})$, and
\item
	$0_{B,Y} \circ f \,=\, 0_{A,Y}$,\, for each $Y \in \Obj(\mathfrak{C})$.
\end{enumerate}
\end{lemma}
\proof
Let $0 \in \Obj(\mathfrak{C})$ represent a fixed zero object in $\mathfrak{C}$.
Then,
\begin{enumerate}
\item
	\begin{eqnarray*}
	f \circ 0_{X,A}
	& = &
		f \circ (\,\iota_{0,A} \circ \iota_{X,0}\,)
	\\
	& = &
		(\,f \circ \iota_{0,A}\,) \circ \iota_{X,0}
	\\
	& = &
		(\,\iota_{0,B}\,) \circ \iota_{X,0}\,,
		\quad
		\textnormal{since $f \circ \iota_{0,A},\, \iota_{0,B} \in \Mor_{\mathfrak{C}}(0,B)$, and $0$ is a zero object}
	\\
	& = &
		0_{X,B}
	\end{eqnarray*}
\item
	\begin{eqnarray*}
	0_{B,Y} \circ f
	& = &
		(\,\iota_{0,Y} \circ \iota_{B,0}\,) \circ f
	\\
	& = &
		\iota_{0,Y} \circ (\,\iota_{B,0} \circ f\,)
	\\
	& = &
		\iota_{0,Y} \circ (\,\iota_{A,0}\,),
		\quad
		\textnormal{since $\iota_{A,0},\, \iota_{B,0} \circ f \in \Mor_{\mathfrak{C}}(A,0)$, and $0$ is a zero object}
	\\
	& = &
		0_{A,Y}
	\end{eqnarray*}
	\vskip -0.5cm
	\qed
\end{enumerate}

          %%%%% ~~~~~~~~~~~~~~~~~~~~ %%%%%

\vskip 1.0cm
\noindent
\textbf{\large Kernels and cokernels of morphisms in a category with zero objects}

          %%%%% ~~~~~~~~~~~~~~~~~~~~ %%%%%

\vskip 0.5cm
\begin{definition}[Kernels and cokernels of morphisms in a category with zero objects]
\mbox{}
\vskip 0.15cm
\noindent
Let \,$\mathfrak{C}$\, be a category with zero objects, and $f : A \longrightarrow B$ a morphism in $\mathfrak{C}$.
\begin{itemize}
\item
	A \textbf{kernel} of $f$ (if it exists) is a morphism $\kappa : K \longrightarrow A$ in $\mathfrak{C}$ such that
	\begin{itemize}
	\item
		\vskip -0.15cm
		$f \circ \kappa \,=\, 0_{K,B}$, and
	\item
		for each $\rho \in \Mor_{\mathfrak{C}}(X,A)$ satisfying $f \circ \rho = 0_{X,B}$,
		there exists a unique $\theta \in \Mor_{\mathfrak{C}}(X,K)$ such that the following diagram commutes:
		\begin{center}
		\begin{tikzcd}
		{\color{blue}X}
			\arrow[dd, thick, dashed, swap, "\exists !\,\theta{\color{white}.}", blue]
			\arrow[dr, thick, swap, "\rho", blue]
			\arrow[drrr, bend left = 15, "0", blue] & & &
		\\
		& {\color{red}A}
			\arrow[rr, swap, "f{\color{white}...}"] & & B
		\\
		{\color{red}K}
			\arrow[ur, swap, "\kappa", red]
			\arrow[urrr, bend right = 15, swap, "0", gray]
		\end{tikzcd}
		\end{center}
	\end{itemize}
\item
	A \textbf{cokernel} of $f$ (if it exists) is a morphism $\pi : B \longrightarrow Q$ in $\mathfrak{C}$ such that
	\begin{itemize}
	\item
		\vskip -0.15cm
		$\pi \circ f \,=\, 0_{A,Q}$, and
	\item
		for each $\sigma \in \Mor_{\mathfrak{C}}(B,Y)$ satisfying \,$\sigma \circ f = 0$,
		there exists a unique $\theta \in \Mor_{\mathfrak{C}}(Q,Y)$ such that the following diagram commutes:
		\begin{center}
		\begin{tikzcd}
		& & & {\color{red}Q}
			\arrow[dd, thick, dashed, "\;\exists !\,\theta", blue]
		\\
		A 
			\arrow[rr, "{\color{white}...}f"]
			\arrow[rrru, bend left = 15, "0", gray]
			\arrow[rrrd, bend right = 15, swap, "0", blue]
			& &
			{\color{red}B}
			\arrow[rd, thick, "\sigma", blue]
			\arrow[ur, "\pi", red]
			&
		\\
		& & & {\color{blue}Y}
		\end{tikzcd}
		\end{center}
	\end{itemize}
%\item
%	An \textbf{image} of $f$ is a kernel of a cokernel of $f$.
\end{itemize}
\end{definition}

          %%%%% ~~~~~~~~~~~~~~~~~~~~ %%%%%

\vskip 0.5cm
\begin{proposition}[Kernels and cokernels are unique up to canonical isomorphism]
\mbox{}
\vskip 0.15cm
\noindent
Let \,$\mathfrak{C}$\, be a category with zero objects.
Let \,$f \in \Mor_{\mathfrak{C}}(A,B)$,\, where \,$A, B \in \Obj(\mathfrak{C})$.
Then, the following statements are true:
\begin{enumerate}
\item
	If \,$\kappa_{1} \in \Mor_{\mathfrak{C}}(K_{1},A)$\, and \,$\kappa_{2} \in \Mor_{\mathfrak{C}}(K_{2},A)$
	are two kernels of $f$, then there exists a unique isomorphism $\theta \in \Mor_{\mathfrak{C}}(K_{2},K_{1})$
	such that the following diagram commutes:
	\begin{center}
	\begin{tikzcd}
	{\color{blue}K_{2}}
		\arrow[dd, thick, dashed, swap, "\exists !\,\theta{\color{white}.}", blue]
		\arrow[dr, thick, swap, "\kappa_{2}", blue]
		\arrow[drrr, bend left = 15, "0", blue] & & &
	\\
	& {\color{red}A}
		\arrow[rr, swap, "f{\color{white}...}"] & & B
	\\
	{\color{red}K_{1}}
		\arrow[ur, swap, "\kappa_{1}", red]
		\arrow[urrr, bend right = 15, swap, "0", gray]
	\end{tikzcd}
	\end{center}
\item
	If \,$\pi_{1} \in \Mor_{\mathfrak{C}}(B,Q_{1})$\, and \,$\pi_{2} \in \Mor_{\mathfrak{C}}(B,Q_{2})$
	are two cokernels of $f$, then there exists a unique isomorphism $\theta \in \Mor_{\mathfrak{C}}(Q_{1},Q_{2})$
	such that the following diagram commutes:
	\begin{center}
	\begin{tikzcd}
	& & & {\color{red}Q_{1}}
		\arrow[dd, thick, dashed, "\;\exists !\,\theta", blue]
	\\
	A 
		\arrow[rr, "{\color{white}...}f"]
		\arrow[rrru, bend left = 15, "0", gray]
		\arrow[rrrd, bend right = 15, swap, "0", blue]
		& &
		{\color{red}B}
		\arrow[ur, "\pi_{1}", red]
		\arrow[rd, thick, "\pi_{2}", blue]
		&
	\\
	& & & {\color{blue}Q_{2}}
	\end{tikzcd}
	\end{center}
\end{enumerate}
\end{proposition}
\proof
\begin{enumerate}
\item
	By the definition of kernels (universal property), there exist unique morphisms
	$\theta_{12} \in \Mor_{\mathfrak{C}}(K_{1},K_{2})$ and $\theta_{21} \in \Mor_{\mathfrak{C}}(K_{2},K_{1})$
	such that the following diagram commutes:
	\begin{center}
	\begin{tikzcd}
	{\color{blue}K_{2}}
		\arrow[dd, thick, dashed, bend right = 90, swap, "\exists !\,\theta_{21}{\color{white}.}", blue]
		\arrow[dr, thick, swap, "\kappa_{2}", blue]
		\arrow[drrr, bend left = 15, "0", blue] & & &
	\\
	& {\color{red}A}
		\arrow[rr, swap, "f{\color{white}...}"] & & B
	\\
	{\color{red}K_{1}}
		\arrow[ur, swap, "\kappa_{1}", red]
		\arrow[uu, thick, dashed, bend left = 60, swap, "\;\,\exists !\,\theta_{12}{\color{white}.}", red]
		\arrow[urrr, bend right = 15, swap, "0", red]
	\end{tikzcd}
	\end{center}
	Again, by the universal property of kernels, we see that we must have
	$\theta_{21} \circ \theta_{12} \,=\, 1_{K_{1}}$.
	By symmetry (i.e., interchanging the indices 1 and 2), we have $\theta_{12} \circ \theta_{21} \,=\, 1_{K_{2}}$.
	This shows that $\theta_{12}$ and $\theta_{21}$ are isomorphisms and are inverses of each other.
	Taking $\theta$ in the proposition statement to be $\theta_{21}$, this completes the proof.
\item
	By the definition of cokernels (universal property), there exist unique morphisms
	$\theta_{12} \in \Mor_{\mathfrak{C}}(Q_{1},Q_{2})$ and $\theta_{21} \in \Mor_{\mathfrak{C}}(Q_{2},Q_{1})$
	such that the following diagram commutes:
	\begin{center}
	\begin{tikzcd}
	& & & {\color{red}Q_{1}}
		\arrow[dd, thick, bend left = 90, dashed, "\;\exists !\,\theta_{12}", blue]
	\\
	A 
		\arrow[rr, "{\color{white}...}f"]
		\arrow[rrru, bend left = 15, "0", gray]
		\arrow[rrrd, bend right = 15, swap, "0", blue]
		& &
		{\color{red}B}
		\arrow[ur, "\pi_{1}", red]
		\arrow[rd, thick, "\pi_{2}", blue]
		&
	\\
	& & & {\color{blue}Q_{2}}
		\arrow[uu, thick, bend right = 60, dashed, "\;\exists !\,\theta_{21}\,", red]
	\end{tikzcd}
	\end{center}
	Again, by the universal property of cokernels, we see that we must have
	$\theta_{21} \circ \theta_{12} \,=\, 1_{Q_{1}}$.
	By symmetry (i.e., interchanging the indices 1 and 2), we have $\theta_{12} \circ \theta_{21} \,=\, 1_{Q_{2}}$.
	This shows that $\theta_{12}$ and $\theta_{21}$ are isomorphisms and are inverses of each other.
	Taking $\theta$ in the proposition statement to be $\theta_{12}$, this completes the proof.
	\qed
\end{enumerate}

          %%%%% ~~~~~~~~~~~~~~~~~~~~ %%%%%

\vskip 0.5cm
\begin{proposition}
\mbox{}
\vskip 0.1cm
\noindent
In a category with zero objects, every kernel of a morphism is a monomorphism, and
every cokernel of a morphism is an epimorphism.
\end{proposition}
\proof
Let $\mathfrak{C}$ be a category with zero objects.
Let $\kappa \in \Mor_{\mathfrak{C}}(K,A)$ be a kernel of $f \in \Mor_{\mathfrak{C}}(A,B)$.
Let $\eta, \mu \in \Mor_{\mathfrak{C}}(X,K)$ be such that
$\kappa \circ \eta \,=\, \kappa \circ \mu \,\in\, \Mor_{\mathfrak{C}}(X,A)$.
We need to show that $\eta = \mu$.
To this end, first, we define $\rho \,:=\, \kappa \circ \eta \,=\, \kappa \circ \mu \,\in\, \Mor_{\mathfrak{C}}(X,A)$.
Then, note
$f \circ \rho \,=\, f \circ (\kappa \circ \eta) \,=\, (f \circ \kappa) \circ \eta \,=\, 0_{K,B} \circ \eta \,=\, 0_{X,B}$.
Since $\kappa$ is a kernel of $f$, we see that there exists a unique $\theta \in \Mor_{\mathfrak{C}}(X,K)$ such that
the following diagram commutes:
\begin{center}
\begin{tikzcd}
X
	\arrow[dd, thick, dashed, bend right = 90, swap, "\exists !\,\theta{\color{white}.}", blue]
	\arrow[dd, shift left, "\mu"]
	\arrow[dd, shift right, swap, "\eta"]
	\arrow[dr, thick, "\rho", blue]
	\arrow[drrr, bend left = 15, "0", blue]
	& &
\\
	& A \arrow[rr, swap, "f"] & & B
\\
{\color{red}K}
	\arrow[ur, swap, "\kappa", red]
	\arrow[urrr, bend right = 15, swap, "0", gray]
\end{tikzcd}
\end{center}
The uniqueness of $\theta$ therefore implies $\eta = \theta = \mu$, as required.
This shows that $\kappa$ can be left cancelled.
Hence, it is a monomorphism.

\vskip 0.1cm
The second part follows similarly.
Indeed, let 
$\pi \in \Mor_{\mathfrak{C}}(B,Q)$ be a cokernel of $f \in \Mor_{\mathfrak{C}}(A,B)$.
Let $\zeta, \nu \in \Mor_{\mathfrak{C}}(Q,Y)$ be such that
$\zeta \circ \pi \,=\, \nu \circ \pi \,\in\, \Mor_{\mathfrak{C}}(B,Y)$.
We need to show that $\zeta = \nu$.
To this end, we define $\sigma \,:=\, \zeta \circ \pi \,=\, \nu \circ \pi \,\in\, \Mor_{\mathfrak{C}}(B,Y)$.
Then, we have: 
$\sigma \circ f \,=\, (\zeta \circ \pi) \circ f \,=\, \zeta \circ (\pi \circ f) \,=\, \zeta \circ 0_{A,Q} \,=\, 0_{A,Y}$.
Since $\pi$ is a cokernel of $f$, we see that there is a unique $\theta \in \Mor_{\mathfrak{C}}(Q,Y)$
such that the following diagram commutes:
\begin{center}
\begin{tikzcd}
& & &
	{\color{red}Q}
	\arrow[dd, thick, dashed, bend left = 90, "\;\exists !\,\theta", blue]
	\arrow[dd, shift left, "\nu"]
	\arrow[dd, shift right, swap, "\zeta"]
\\
A
	\arrow[rr, "f"]
	\arrow[urrr, bend left = 15, swap, "0", gray]
	\arrow[drrr, bend right = 15, swap, "0", blue]
	&&
	{\color{red}B}
	\arrow[ur, "\pi", red]
	\arrow[rd, thick, "\sigma", blue]
	&
\\
& & & Y
\end{tikzcd}
\end{center}
The uniqueness of $\theta$ therefore implies $\zeta = \theta = \nu$, as required.
This shows that $\pi$ can be right cancelled.
Hence, it is an epimorphism.
This completes the proof of the Proposition.
\qed

          %%%%% ~~~~~~~~~~~~~~~~~~~~ %%%%%

\vskip 0.5cm
\begin{lemma}[Composition on the left by a monomorphism preserves kernels]
\label{CompositionOnTheLeftByAMonomorphismPreservesKernels}
\mbox{}
\vskip 0.1cm
\noindent
Let \,$\mathfrak{C}$\, be a category with zero objects,
\,$f \in \Mor_{\mathfrak{C}}(A,B)$,\,
\,$\widetilde{f} \in \Mor_{\mathfrak{C}}(A,I)$,\,
\,$\iota \in \Mor_{\mathfrak{C}}(I,B)$,\,
and \,$f \,=\, \iota \circ \widetilde{f}$.
\vskip 0.1cm
\noindent
If \,$\iota$\, is a monomorphism, then, for any morphism \,$\kappa \in \Mor_{\mathfrak{C}}(K,A)$,
\begin{equation*}
\textnormal{$\kappa$\, is a kernel of \,$f$}
\quad\Longleftrightarrow\quad 
\textnormal{$\kappa$\, is a kernel of \,$\widetilde{f}$}
\end{equation*}
\end{lemma}
\proof
Consider the following diagram:
\begin{center}
\begin{tikzcd}
{\color{blue}X}
	\arrow[dd, thick, dashed, swap, "\exists !\,\theta{\color{white}.}", blue]
	\arrow[dr, thick, swap, "\rho", blue]
	\arrow[drrr, bend left = 25, "0", blue]
	\arrow[drrrrr, bend left = 30, "0", blue]
\\
& {\color{red}A}
	\arrow[rr, thick, "\widetilde{f}{\color{white}..}"]
	\arrow[rrrr, thick, bend right = 32, swap, "f"]
&&
I
	\arrow[rr, thick, hook, "\iota"]
&&
B
\\
{\color{red}K}
	\arrow[ur, swap, "\kappa", red]
	\arrow[urrr, bend right = 25, swap, "0", gray]
	\arrow[urrrrr, bend right = 30, swap, "0", gray]
\end{tikzcd}
\end{center}

\noindent
\underline{(\,$\Longrightarrow$\,)}\;\;
Suppose \,$\kappa \in \Mor_{\mathfrak{C}}(K,A)$\, is a kernel of \,$f$.\,
We need to show \,$\kappa$\, is also a kernel of \,$\widetilde{f}$\,;\,
in other words, for each morphism \,$\rho \in \Mor_{\mathfrak{C}}(X,A)$\, such that \,$\widetilde{f} \circ \rho = 0$,\,
we need to show that there exists a unique morphism \,$\theta \in \Mor_{\mathfrak{C}}(X,K)$\,
such that \,$\rho = \kappa \circ \theta$.\,
To this end, note that
\,$\widetilde{f} \circ \rho \,=\, 0$
\;\;$\Longrightarrow$\;\; $\iota \circ (\,\widetilde{f} \circ \rho\,) \,=\, \iota \circ 0 \,=\, 0$\,
\;\;$\Longrightarrow$\;\; $(\,\iota \circ \widetilde{f} \,) \circ \rho \,=\, 0$\,
\;\;$\Longrightarrow$\;\; $f \circ \rho \,=\, 0$.\,
Thus, by the universal property of \,$\kappa$\, as a kernel of \,$f$,\,
there exists a unique \,$\theta \in \Mor_{\mathfrak{C}}(X,K)$\,
such that \,$\rho \,=\, \kappa \circ \theta$.\,
This proves that \,$\kappa$\, is indeed a kernel of \,$\widetilde{f}$.

\vskip 0.4cm
\noindent
\underline{(\,$\Longleftarrow$\,)}\;\;
Conversely, suppose \,$\kappa \in \Mor_{\mathfrak{C}}(K,A)$\, is a kernel of \,$\widetilde{f}$.\,
We need to show \,$\kappa$\, is also a kernel of \,$f$\,;\,
in other words, for each morphism \,$\rho \in \Mor_{\mathfrak{C}}(X,A)$\, such that \,$f \circ \rho = 0$,\,
we need to show that there exists a unique morphism \,$\theta \in \Mor_{\mathfrak{C}}(X,K)$\,
such that \,$\rho = \kappa \circ \theta$.\,
To this end, note that
\,$f \circ \rho \,=\, 0$
\;\;$\Longleftrightarrow$\;\; $(\,\iota \,\circ\, \widetilde{f} \,) \,\circ\, \rho \,=\, 0$\,
\;\;$\Longrightarrow$\;\; $\iota \,\circ\, (\,\widetilde{f} \,\circ\, \rho\,) \,=\, 0 \,=\, \iota \,\circ\, 0$\,
\;\;$\Longrightarrow$\;\; $\widetilde{f} \,\circ\, \rho \,=\, 0$,\,
since \,$\iota$\, is by hypothesis a monomorphism and can be cancelled on the left.
Thus, by the universal property of \,$\kappa$\, as a kernel of \,$\widetilde{f}$,\,
there exists a unique \,$\theta \in \Mor_{\mathfrak{C}}(X,K)$\,
such that \,$\rho \,=\, \kappa \circ \theta$.\,
This proves that \,$\kappa$\, is indeed a kernel of \,$f$.

\vskip 0.4cm
\noindent
This completes the proof of the Lemma.
\qed

          %%%%% ~~~~~~~~~~~~~~~~~~~~ %%%%%

\vskip 0.5cm
\begin{lemma}[A kernel in a category with zero objects is the kernel of its own cokernel + dual]
\label{AKernelsTheKernelOfItsOwnCokernel}
\mbox{}
\vskip 0.1cm
\noindent
Let \,$\mathfrak{C}$\, be a category with zero objects,
\,$A, B \in \Obj(\mathfrak{C})$,\, and
\,$f \in \Mor_{\mathfrak{C}}(A,B)$.\,
\vskip 0.1cm
\noindent
Then, the following statements are true:
\begin{enumerate}
\item
	If:
	\begin{itemize}
	\item
		$f$\, is a kernel of some morphism, and
	\item
		$\pi_{f} \in \Mor_{\mathfrak{C}}(B,Q)$\, is a cokernel of \,$f$,
	\end{itemize}
	then: \,$f$\, is a kernel of \,$\pi_{f}$.
\item
	Dually, if:
	\begin{itemize}
	\item
		$f$\, is a cokernel of some morphism, and
	\item
		$\kappa_{f} \in \Mor_{\mathfrak{C}}(K,A)$\, is a kernel of \,$f$,
	\end{itemize}
	then: \,$f$\, is a cokernel of \,$\kappa_{f}$.
\end{enumerate}
\end{lemma}
\proof
\begin{enumerate}
\item
	Consider the following diagram:
	\begin{center}
	\begin{tikzcd}
	X
		\arrow[drr, thick, "\rho"]
		\arrow[dd, dashed, thick, swap, "\exists !\; \theta\,", red]
		\arrow[rrrr, swap, bend left = 20, swap, "0", red]
	& & & &
	Q
		\arrow[d, dashed, "\;\exists !\,\varphi", blue]
	\\
	& &
	B
		\arrow[urr, "\pi_{f}", blue]
		\arrow[rr, "\mu"]
	& &
	C
	\\
	{\color{red}A}
		\arrow[urr, hook, swap, "f \,=\, \kappa_{\mu}", red]
		\arrow[urrrr, bend right = 25, swap, "0", blue]
	\end{tikzcd}
	\end{center}
	In order to prove that \,$f$\, is a kernel of \,$\pi_{f}$,\,
	we need to show that, for any \,$\rho \in \Mor_{\mathfrak{C}}(X,B)$\,
	satisfying \,$\pi_{f} \circ \rho = 0$,\,
	there exists a unique \,$\theta \in \Mor_{\mathfrak{C}}(X,A)$\,
	such that \,$\rho = f \circ \theta$.\,

	\vskip 0.3cm
	\noindent
	By hypothesis, \,$f$\, is a kernel of some morphism \,$\mu \in \Mor_{\mathfrak{C}}(B,C)$.\,
	We write \,$\kappa_{\mu}$\, for kernel of \,$\mu$.\, Hence, \,$f = \kappa_{\mu}$.

	\vskip 0.3cm
	\noindent
	\textit{Aside (Plan of Proof):
	Thanks to the universal property of $f = \kappa_{\mu}$ as a kernel of $\mu$,
	a look at the diagram above shows that
	the existence and uniqueness of \,$\theta$ will follow as soon as we establish
	that $\mu \circ \rho = 0$, which in turn follows from the existence of the morphism \,$\varphi$,
	which in turn follows from the universal property of $\pi_{f}$ as a cokernel of $f$.}

	\vskip 0.3cm
	\noindent
	\textbf{Claim 1:}\;\; There exists a unique \,$\varphi \in \Mor_{\mathfrak{C}}(Q,C)$\, such that \,$\mu = \varphi \circ \pi_{f}$.
	\vskip 0.1cm
	\noindent
	Proof of Claim 1:\;
	This follows immediately from two facts:
	(a)	\,$\mu \circ f \,=\, \mu \circ \kappa_{\mu} \,=\, 0$, and
	(b)	the universal property of \,$\pi_{f}$\, as a cokernel of $f$.

	\vskip 0.3cm
	\noindent
	\textbf{Claim 2:}\;\; $\mu \circ \rho \,=\, 0$
	\vskip 0.1cm
	\noindent
	Proof of Claim 2:\;
	$\mu \circ \rho$
	\;$=$\; $\left(\,\varphi \overset{{\color{white}1}}{\circ} \pi_{f}\right)\circ \rho$
	\;$=$\; $\varphi \circ \left(\, \pi_{f} \overset{{\color{white}1}}{\circ} \rho\, \right)$
	\;$=$\; $\varphi \circ 0$
	\;$=$\; $0$,\,
	as required.

	\vskip 0.3cm
	\noindent
	\textbf{Claim 3:}\;\;
	There exists a unique \,$\theta \in \Mor_{\mathfrak{C}}(X,A)$\, such that \,$\rho = f \circ \theta$.
	\vskip 0.1cm
	\noindent
	Proof of Claim 3:\;
	This follows immediately from two facts:
	(a)	\,$\mu \circ \rho \,=\, 0$ (Claim 2), and
	(b)	the universal property of \,$f = \kappa_{\mu}$\, as a kernel of \,$\mu$.

	\vskip 0.3cm
	\noindent
	This completes the proof.

\item
	Consider the dualized version of the preceding diagram:
	\begin{center}
	\begin{tikzcd}
	Y
	& & & &
	K
		\arrow[dll, swap, "\kappa_{f}", blue]
		\arrow[llll, bend right = 20, swap, "0", red]
	\\
	& &
	A
		\arrow[ull, thick, "\rho"]
		\arrow[dll, two heads, "f \,=\, \pi_{\mu}", red]
	& &
	C
		\arrow[u, dashed, swap, "\;\exists !\,\varphi", blue]
		\arrow[ll, "\mu"]
		\arrow[dllll, bend left = 25, "0", blue]
	\\
	{\color{red}B}
		\arrow[uu, dashed, thick, "\exists !\,\theta\;", red]
	\end{tikzcd}
	\end{center}
%	\begin{center}
%	\begin{tikzcd}
%	X
%		\arrow[drr, thick, "\rho"]
%		\arrow[dd, dashed, thick, swap, "\exists !\; \theta\,", red]
%		\arrow[rrrr, swap, bend left = 20, swap, "0", red]
%	& & & &
%	Q
%		\arrow[d, dashed, "\;\exists !\,\varphi", blue]
%	\\
%	& &
%	B
%		\arrow[urr, "\pi_{f}", blue]
%		\arrow[rr, "\mu"]
%	& &
%	C
%	\\
%	{\color{red}A}
%		\arrow[urr, hook, swap, "f \,=\, \kappa_{\mu}", red]
%		\arrow[urrrr, bend right = 25, swap, "0", blue]
%	\end{tikzcd}
%	\end{center}
	In order to prove that \,$f$\, is a cokernel of \,$\kappa_{f}$,\,
	we need to show that, for any \,$\rho \in \Mor_{\mathfrak{C}}(A,Y)$\,
	satisfying \,$\rho \circ \kappa_{f} = 0$,\,
	there exists a unique \,$\theta \in \Mor_{\mathfrak{C}}(B,Y)$\,
	such that \,$\rho = \theta \circ f$.\,

	\vskip 0.3cm
	\noindent
	By hypothesis, \,$f$\, is a cokernel of some morphism \,$\mu \in \Mor_{\mathfrak{C}}(C,A)$.\,
	We write \,$\pi_{\mu}$\, for cokernel of \,$\mu$.\, Hence, \,$f = \pi_{\mu}$.

	\vskip 0.3cm
	\noindent
	\textit{Aside (Plan of Proof):
	Thanks to the universal property of $f = \pi_{\mu}$ as a cokernel of $\mu$,
	a look at the diagram above shows that
	the existence and uniqueness of \,$\theta$ will follow as soon as we establish
	that $\rho \circ \mu = 0$, which in turn follows from the existence of the morphism \,$\varphi$,
	which in turn follows from the universal property of $\kappa_{f}$ as a kernel of $f$.}

	\vskip 0.3cm
	\noindent
	\textbf{Claim 1:}\;\; There exists a unique \,$\varphi \in \Mor_{\mathfrak{C}}(C,K)$\, such that \,$\mu = \kappa_{f} \circ \varphi$.
	\vskip 0.1cm
	\noindent
	Proof of Claim 1:\;
	This follows immediately from two facts:
	(a)	\,$f \circ \mu \,=\, \pi_{\mu} \circ \mu \,=\, 0$, and
	(b)	the universal property of \,$\kappa_{f}$\, as a kernel of $f$.

	\vskip 0.3cm
	\noindent
	\textbf{Claim 2:}\;\; $\rho \circ \mu \,=\, 0$
	\vskip 0.1cm
	\noindent
	Proof of Claim 2:\;
	$\rho \circ \mu$
	\;$=$\; $\rho \circ \left(\, \kappa_{f} \overset{{\color{white}1}}{\circ} \varphi \,\right)$
	\;$=$\; $\left(\, \rho \overset{{\color{white}1}}{\circ} \kappa_{f} \right) \circ \varphi$
	\;$=$\; $0 \circ \varphi$
	\;$=$\; $0$,\,
	as required.

	\vskip 0.3cm
	\noindent
	\textbf{Claim 3:}\;\;
	There exists a unique \,$\theta \in \Mor_{\mathfrak{C}}(B,Y)$\, such that \,$\rho = \theta \circ f$.
	\vskip 0.1cm
	\noindent
	Proof of Claim 3:\;
	This follows immediately from two facts:
	(a)	\,$\rho \circ \mu \,=\, 0$ (Claim 2), and
	(b)	the universal property of \,$f = \pi_{\mu}$\, as a cokernel of \,$\mu$.

	\vskip 0.3cm
	\noindent
	This completes the proof.
	\qed
\end{enumerate}

          %%%%% ~~~~~~~~~~~~~~~~~~~~ %%%%%

\vskip 0.5cm
\begin{corollary}[Proposition 13.3, p.16, \cite{Mitchell1965}]
\mbox{}
\vskip 0.1cm
\noindent
Suppose \,$\mathfrak{C}$\, is a category with zero objects,
\,$A, B \in \Obj(\mathfrak{C})$,\, and
\,$f \in \Mor_{\mathfrak{C}}(A,B)$.
Let \,$\kappa_{f} \in \Mor_{\mathfrak{C}}(K,A)$\, be a kernel of \,$f$,\,
and \,$\pi(\kappa_{f}) \in \Mor_{\mathfrak{C}}(A,Q)$\, be a cokernel of \,$\kappa_{f}$.\,
Then, \,$\kappa_{f}$\, is a kernel of \,$\pi(\kappa_{f})$.
\end{corollary}
%\proof
%Consider the following diagram:
%\begin{center}
%\begin{tikzcd}
%X
%	\arrow[dd, thick, dashed, swap, "\exists !\,\theta\;", red]
%	\arrow[drr, thick, swap, "\rho"]
%	\arrow[rrrr, bend left = 20, "0", red]
%&&&&
%Q
%	\arrow[d, dashed, "\;\exists !\,\varphi\;", blue]
%\\
%&& A
%	\arrow[urr, swap, "\pi(\kappa_{f})", blue]
%	\arrow[rr, swap, "f"]
%&& B
%\\
%K
%	\arrow[urr, swap, "\kappa_{f}", red]
%	\arrow[urrrr, bend right = 30, swap, "0", blue]
%\end{tikzcd}
%\end{center}
%In order to establish that \,$\kappa_{f}$\, is in fact also the kernel of \,$\pi(\kappa_{f})$,\,
%we need to show that, for each \,$\rho \in \Mor_{\mathfrak{C}}(X,A)$\, satisfying \,$\pi(\kappa_{f}) \circ \rho = 0$,\,
%there exists unique \,$\theta \in \Mor_{\mathfrak{C}}(X,K)$\, such that the above diagram commutes, i.e.,
%\,$\rho = \kappa_{f} \circ \theta$.\,
%
%\vskip 0.3cm
%\noindent
%\textbf{Claim 1:}\;\, There exists a unique \,$\varphi \in \Mor_{\mathfrak{C}}(Q,B)$\,
%such that \,$f = \varphi \circ \pi(\kappa_{f})$.
%\vskip 0.1cm
%\noindent
%Proof of Claim 1:\; This follows immediately from two facts:
%(a) \,$f \circ \kappa_{f} = 0$\, ($\kappa_{f}$ being a kernel of $f$), and
%(b) the universal property of \,$\pi(\kappa_{f})$\, as a cokernel of \,$\kappa_{f}$.\,
%This proves Claim 1.
%
%\vskip 0.3cm
%\noindent
%\textbf{Claim 2:}\;\, $f \circ \rho = 0$.
%\vskip 0.05cm
%\noindent
%Proof of Claim 2:\;
%$f \circ \rho$
%\;$=$\; $\left(\,\varphi \overset{{\color{white}1}}{\circ} \pi(\kappa_{f})\,\right) \circ \rho$
%\;$=$\; $\varphi \circ \left(\, \pi(\kappa_{f}) \overset{{\color{white}1}}{\circ} \rho\,\right)$
%\;$=$\; $\varphi \circ 0$
%\;$=$\; $0$,\,
%where the first equality follows from Claim 1, while the last equality follows by hypothesis on \,$\rho$.\,
%This proves Claim 2.
%
%\vskip 0.3cm
%\noindent
%\textbf{Claim 3:}\;\, There exists a unique \,$\theta \in \Mor_{\mathfrak{C}}(X,K)$\,
%such that \,$\rho = \kappa_{f} \circ \theta$.
%\vskip 0.1cm
%\noindent
%Proof of Claim 3:\; This follows immediately from two facts:
%(a) \,$f \circ \rho = 0$\, (Claim 2),
%and (b) the universal property of \,$\kappa_{f}$\, as a kernel of \,$f$.\,
%This proves Claim 3.
%
%\vskip 0.3cm
%\noindent
%The proof of the Proposition is complete.
%\qed

          %%%%% ~~~~~~~~~~~~~~~~~~~~ %%%%%

\vskip 0.5cm
\begin{lemma}\label{CokernelOfAnIsomorphismIsZero}
\mbox{}
\vskip 0.1cm
\noindent
In a category with zero objects, every cokernel of an isomorphism is a zero morphism.
\end{lemma}
\proof
Let
\,$X \,\overset{h}{\longrightarrow}\, Y$\,
be an isomorphism, with inverse
\,$Y \,\overset{h^{-1}}{\longrightarrow}\, X$.\,

\vskip 0.4cm
\noindent
\textbf{Claim 1:}\quad
If a morphism \,$Y \,\overset{\sigma}{\longrightarrow}\, Z$\, satisfies \,$\sigma \circ h = 0$,\,
then \,$\sigma$\, is a zero morphism.
\vskip 0.2cm
\noindent
Proof of Claim 1:\;\;
Simply observe that
\begin{equation*}
\sigma
\;\; = \;\;
	\sigma \circ 1_{Y}
\;\; = \;\;
	\sigma \circ \left(\,h \overset{{\color{white}!}}{\circ} h^{-1}\right)
\;\; = \;\;
	 \left(\, \sigma \overset{{\color{white}!}}{\circ} h \,\right) \circ h^{-1}
\;\; = \;\;
	 0 \circ h^{-1}
\;\; = \;\;
	 0
\end{equation*}
This proves Claim 1.

\vskip 0.4cm
\noindent
Now, in the following diagram, we set
\,$Y \,\overset{\pi_{h}}{\longrightarrow} Q_{h}$\,
to be the zero morphism:
\begin{center}
\begin{tikzcd}
X
	\arrow[rr, swap, "h"]
	\arrow[drrrr, bend right = 17, swap, "0", gray]
&&
Y
	\arrow[ll, bend right = 30, swap, "h^{-1}"]
	\arrow[rr, "\pi_{h}", red]
	\arrow[drr, swap, "\sigma"]
&&
{\color{red}Q_{h}}
	\arrow[d, dashed, "\exists !\,\theta", red]
\\
&&&&
Z
\end{tikzcd}
\end{center}
For any morphism
\,$Y \,\overset{\sigma}{\longrightarrow}\, Z$\, satisfying \,$\sigma \circ h = 0$,\,
we have by Claim 1 that \,$\sigma = 0$.
Consequently, there exists a unique morphism \,$\theta$\, that makes the above diagram
commute for the given \,$\sigma$,\, namely, $\theta = 0$ (since $\sigma = 0$).
This proves that the zero morphism \,$Y \,\overset{0}{\longrightarrow}\, 0$\, is a cokernel of $h$.
Since cokernels are unique up to isomorphism, and compositions with zero morphisms are themselves zero morphisms,
we may now conclude that every cokernel of \,$h$\, is a zero morphism, as required.
\qed

          %%%%% ~~~~~~~~~~~~~~~~~~~~ %%%%%

\vskip 1.0cm
\noindent
\textbf{\large Images of morphisms in a normal category with zero objects and cokernels}

          %%%%% ~~~~~~~~~~~~~~~~~~~~ %%%%%

\vskip 0.5cm
\begin{proposition}[Lemma 14.4, p.17, \cite{Mitchell1965}]
\label{FactorizationIntoImageCoimage}
\mbox{}
\vskip 0.1cm
\noindent
Suppose \,$\mathfrak{C}$\, is a category with zero objects,
\,$A, B \in \Obj(\mathfrak{C})$,\, and
\,$f \in \Mor_{\mathfrak{C}}(A,B)$.
If
\begin{itemize}
\item
	$\pi_{f} \in \Mor_{\mathfrak{C}}(B,Q)$\, is a cokernel of $f$, and
\item
	$\kappa(\pi_{f}) \in \Mor_{\mathfrak{C}}(I,B)$\, is a kernel of \,$\pi_{f}$,
\end{itemize}
then the following statements are true:
\begin{enumerate}
\item
	There exists a unique morphism \,$\widetilde{f} \in \Mor_{\mathfrak{C}}(A,I)$\,
	that makes the following diagram commute: %(due to the universal property of $\kappa(\pi_{f})$):
	\begin{center}
	\begin{tikzcd}
	A
		\arrow[dr, thick, swap, "f", blue]
		\arrow[dd, thick, dashed, swap, "\exists !\,\widetilde{f}{\color{white}.}", blue]
		\arrow[drrr, bend left = 20, "0", blue]
	\\
		&
		B
		\arrow[rr, "\pi_{f}{\color{white}..}"] & & Q
	\\
	I
		\arrow[ur, hook, swap, "\kappa(\pi_{f})", red] 
		\arrow[urrr, bend right = 20, swap, "0", gray]
	\end{tikzcd}
	\end{center}
\item
	If $\mathfrak{C}$ has cokernels and is
	normal\,\footnote{A category is said to be \textbf{normal} if every monomorphism in it is a kernel.},
	then $\kappa(\pi_{f})$ is an image of $f$. [A kernel of a cokernel is an image.]
\item
	If, in addition, $\mathfrak{C}$ has equalizers, then $\widetilde{f}$ is a coimage of $f$.
\item
	If, in addition, $\mathfrak{C}$ has kernels and is
	conormal\,\footnote{A category is said to be \textbf{conormal} if every epimorphism in it is a cokernel.},
	then $\widetilde{f}$ is a cokernel of a kernel of $f$.
\end{enumerate}
\end{proposition}
\proof
\begin{enumerate}
\item
	This follows immediate from two facts:
	(a)	\,$\pi_{f} \circ f = 0$\, ($\pi_{f}$ being a cokernel of $f$), and
	(b)	the universal property of \,$\kappa(\pi_{f})$\, as a kernel of \,$\pi_{f}$.\,
\item
	First of all, we already know \,$f = \kappa(\pi_{f}) \circ \widetilde{f}$,\,
	with \,$\kappa(\pi_{f})$\, being a monomorphism (since $\kappa(\pi_{f})$ is a kernel).
	\begin{center}
	\begin{tikzcd}
	&A
		\arrow[dr, swap, "f"]
		\arrow[dd, dashed, swap, "\exists !\,\widetilde{f}{\color{white}.}"]
		\arrow[drrr, bend left = 155, "0", cyan]
		\arrow[dddl, thick, bend right = 50, swap, "\varepsilon", blue]
	&&& Q_{\mu}
	\\
	&& B
		\arrow[rr, swap, "{\color{white}.....}\pi_{f}", cyan]
		\arrow[urr, "\pi_{\mu}"]
	&& Q
		\arrow[u, thick, dashed, swap, "\;\;\exists !\;q", cyan]
	\\
	& I
		\arrow[ur, hook, swap, "\!\!\kappa(\pi_{f})"] 
		\arrow[uurrr, bend right = 90, swap, "0", red]
		\arrow[dl, thick, dashed, swap, "\exists !\,\theta", red]
	\\
	{\color{blue}I^{\prime}}
		\arrow[uurr, thick, hook, bend right = 68, swap, "\mu \,=\, \kappa(\pi_{\mu})", red]
	\end{tikzcd}
	\end{center}
	In order to show that \,$\kappa(\pi_{f})$\, is indeed an image of \,$f$,\,
	we need to show that, for each factorization \,$f = \mu \circ \varepsilon$\, of \,$f$\,
	with \,$\mu$\, a monomorphism, there exists a unique \,$\theta \in \Mor_{\mathfrak{C}}(I,I^{\prime})$\,
	such that \,$\varepsilon = \theta \circ \widetilde{f}$\, and \,$\kappa(\pi_{f}) = \mu \circ \theta$.
	
	\vskip 0.1cm
	\noindent
	To this end, let \,$\pi_{\mu} \in \Mor_{\mathfrak{C}}(B,Q_{\mu})$\, be a cokernel of \,$\mu$
	(this is valid since \,{\color{red}$\mathfrak{C}$\, has cokernels}).\,

	\vskip 0.3cm
	\noindent
	\textbf{Claim 1:}\;\; $\pi_{\mu} \circ f = 0$
	\vskip -0.1cm
	\noindent
	Proof of Claim 1:\;
	$\pi_{\mu} \circ f$
	\;$=$\; $\pi_{\mu} \circ \left(\,\mu \overset{{\color{white}1}}{\circ}\varepsilon\,\right)$
	\;$=$\; $\left(\,\pi_{\mu} \overset{{\color{white}1}}{\circ} \mu \,\right) \circ \varepsilon$
	\;$=$\; $0 \circ \varepsilon$
	\;$=$\; $0$,\,
	as required.

	\vskip 0.3cm
	\noindent
	\textbf{Claim 2:}\;\; There exists a unique \,$q \in \Mor_{\mathfrak{C}}(Q,Q_{\mu})$\,
	such that \,$\pi_{\mu} = q \circ \pi_{f}$.\,
	\vskip 0.05cm
	\noindent
	Proof of Claim 2:\;
	This follows from two facts:\;
	(a)	\,$\pi_{\mu} \circ f = 0$\, (Claim 1), and
	(b)	\,the universal property of \,$\pi_{f}$\, as a cokernel of \,$f$.\,

	\vskip 0.3cm
	\noindent
	\textbf{Claim 3:}\;\; $\pi_{\mu} \,\circ\, \kappa(\pi_{f}) \,=\, 0$
	\vskip -0.1cm
	\noindent
	Proof of Claim 3:\;
	$\pi_{\mu} \circ \kappa(\pi_{f})$
	\;$=$\; $\left(\, q \overset{{\color{white}1}}{\circ} \pi_{f} \,\right) \circ \kappa(\pi_{f})$
	\;$=$\; $q \circ \left(\, \pi_{f} \overset{{\color{white}1}}{\circ} \kappa(\pi_{f}) \,\right)$
	\;$=$\; $q \circ 0$
	\;$=$\; $0$,\,
	as required.

	\vskip 0.3cm
	\noindent
	\textbf{Claim 4:}\;\; $\mu$\, is a kernel of \,$\pi_{\mu}$.
	\vskip -0.1cm
	\noindent
	Proof of Claim 4:\;
	Since {\color{red}$\mathfrak{C}$ is a normal category}, the monomorphism $\mu$ is a kernel of some morphism.
	Claim 4 now follows immediately by Lemma \ref{AKernelsTheKernelOfItsOwnCokernel}.

	\vskip 0.3cm
	\noindent
	\textbf{Claim 5:}\;\; There exists a unique \,$\theta \in \Mor_{\mathfrak{C}}(I,I^{\prime})$\,
	such that \,$\kappa(\pi_{f}) = \mu \circ \theta$.\,
	\vskip 0.05cm
	\noindent
	Proof of Claim 5:\;
	This follows from two facts:\;
	(a)	\,$\pi_{\mu} \,\circ\, \kappa(\pi_{f}) \,=\, 0$\, (Claim 3), and
	(b)	\,the universal property of \,$\mu$\, as a kernel of \,$\pi_{\mu}$.
	
	\vskip 0.3cm
	\noindent
	\textbf{Claim 6:}\;\; $\varepsilon = \theta \circ \widetilde{f}$.\,
	\vskip 0.05cm
	\noindent
	Proof of Claim 6:\; Note that
	\,$\mu \circ \varepsilon$
	\,$=$\, $f$
	\,$=$\, $\kappa(\pi_{f}) \circ \widetilde{f}$\,
	\,$=$\, $\mu \circ \theta \circ \widetilde{f}$,\,
	which implies
	\,$\varepsilon$ \,$=$\, $\theta \circ \widetilde{f}$,
	since \,$\mu$\, is a monomorphism and can be cancelled on the left.
	This proves Claim 6.

	\vskip 0.3cm
	\noindent
	We may now conclude that \,$\kappa(\pi_{f})$\, is indeed an image of \,$f$.
\item
	Consider the following diagram:
	\begin{center}
	\begin{tikzcd}
	&A
		\arrow[dr, "f"]
		\arrow[dd, dashed, swap, "\exists !\,\widetilde{f}{\color{white}.}"]
		\arrow[drrr, bend left = 25, "0", gray]
		\arrow[dddl, two heads, thick, bend right = 50, swap, "\varepsilon"]
	% &&& Q_{\mu}
	\\
	&& B
		\arrow[rr, "{\color{white}..}\pi_{f}"]
		%\arrow[urr, "\pi_{\mu}"]
	&& Q
		%\arrow[u, thick, dashed, swap, "\;\;\exists !\;q", cyan]
	\\
	& I
		\arrow[ur, hook, swap, "\!\!\kappa(\pi_{f})", red]
	\\
	I^{\prime}
		\arrow[uurr, thick, bend right = 68, swap, "\mu"]
		\arrow[ur, thick, dashed, "\exists !\,\theta", red]
		\arrow[uurrrr, bend right = 51, swap, "0", red]
	\end{tikzcd}
	\end{center}
	By the hypothesis that \,{\color{red}$\mathfrak{C}$\, has equalizers} and Lemma \ref{fTildeIsEpimorphism},
	it follows immediately that \,$\widetilde{f}$\, is an epimorphism.

	\vskip 0.3cm
	In order to show that \,$\widetilde{f}$\, is indeed a coimage of \,$f$,\,
	we need to show furthermore that, for each factorization \,$f = \mu \circ \varepsilon$\, of \,$f$\,
	with \,$\varepsilon$\, an epimorphism, there exists a unique \,$\theta \in \Mor_{\mathfrak{C}}(I^{\prime},I)$\,
	such that \,$\widetilde{f} = \theta \circ \varepsilon$\, and \,$\mu = \kappa(\pi_{f}) \circ \theta$.

	\vskip 0.3cm
	\noindent
	\textbf{Claim 1:}\;\; $\pi_{f} \circ \mu \,=\, 0$
	\vskip -0.05cm
	\noindent
	Proof of Claim 1:\;
	$\mu \circ \varepsilon \,=\, f$
	\;\;$\Longrightarrow$\;\;
	$\pi_{f} \circ \mu \circ \varepsilon$
	\;$=$\; $\pi_{f} \circ f$
	\;$=$\; $0_{A,Q}$
	\;$=$\; $0_{I^{\prime},Q} \circ \varepsilon$
	\;\;$\Longrightarrow$\;\;
	$\pi_{f} \circ \mu = 0_{I^{\prime},Q}$,\,
	since \,$\varepsilon$\, is an epimorphism and can be cancelled on the right.

	\vskip 0.3cm
	\noindent
	\textbf{Claim 2:}\;\; There exists a unique \,$\theta \in \Mor_{\mathfrak{C}}(I^{\prime},I)$\,
	such that \,$\mu \,=\, \kappa(\pi_{f}) \circ \theta$.\,
	\vskip -0.05cm
	\noindent
	Proof of Claim 2:\;
	This follows from two facts:\;
	(a)	\,$\pi_{f} \circ \mu \,=\, 0$\, (Claim 1), and
	(b)	\,the universal property of \,$\kappa(\pi_{f})$\, as a kernel of \,$\pi_{f}$.\,

	\vskip 0.3cm
	\noindent
	\textbf{Claim 3:}\;\; $\widetilde{f} \,=\, \theta \circ \varepsilon$
	\vskip -0.05cm
	\noindent
	Proof of Claim 3:\; Note that
	\,$\kappa(\pi_{f}) \circ \widetilde{f}$
	\;$=$\; $f$
	\;$=$\; $\mu \circ \varepsilon$
	\;$=$\; $\kappa(\pi_{f}) \circ \theta \circ \varepsilon$,\, by Claim 2.
	Hence, we have
	\,$\widetilde{f} \,=\, \theta \circ \varepsilon$,\,
	since \,$\kappa(\pi_{f})$\, is a monomorphism (being a kernel)
	and can be cancelled on the left.
	
	\vskip 0.3cm
	\noindent
	This completes the proof that \,$\widetilde{f}$\, is a coimage of \,$f$.

\item
	We have already established that $\widetilde{f}$ is an epimorphism.
	
	Here, we are assuming the additional hypothesis that
	$\mathfrak{C}$ {\color{red}has kernels} and is a {\color{red}conormal} category.
	Consequently, a kernel \,$\kappa_{\widetilde{f}}$\, of \,$\widetilde{f}$\, exists, and
	\,$\widetilde{f}$\, is a cokernel of some morphism.
	
	By Lemma \ref{AKernelsTheKernelOfItsOwnCokernel},
	\,$\widetilde{f}$\, is a cokernel of \,$\kappa_{\widetilde{f}}$.
	
	So, to complete the proof, it remains only to prove that \,$\kappa_{\widetilde{f}}$\, is in fact also a kernel of $f$.
	
	But this last fact follows immediately by Lemma \ref{CompositionOnTheLeftByAMonomorphismPreservesKernels},
	since \,$f \,=\, \kappa(\pi_{f}) \circ \widetilde{f}$,\, with \,$\kappa(\pi_{f})$\, a monomorphism.
\end{enumerate}
\vskip 0.3cm
This completes the proof of the Proposition.
\qed

          %%%%% ~~~~~~~~~~~~~~~~~~~~ %%%%%

\vskip 0.5cm
\begin{remark}[Image as kernel of cokernel in the category of vector spaces]
\mbox{}
\vskip 0.15cm
\noindent
Suppose $V$ and $W$ are vector spaces and $f : V \longrightarrow W$ is a linear map.
Then, we have:
\begin{center}
\begin{tikzcd}
V \arrow[r, "f"] & W \arrow[r, "\pi"] & \textnormal{coker}(f) & \!\!\!\!\!\!\!\!\!\!\!\!\!\!\!\! :=\, W/\textnormal{image}(f)
\end{tikzcd}
\end{center}
where $\pi$ is the standard projection map, and
\,$\textnormal{image}(f) \,=\, \ker(\pi) \,\subset\, W$.
\end{remark}

          %%%%% ~~~~~~~~~~~~~~~~~~~~ %%%%%

\vskip 1.0cm
\noindent
\textbf{\large Homology of two morphisms that compose to zero in a category with zero objects}

          %%%%% ~~~~~~~~~~~~~~~~~~~~ %%%%%

\vskip 0.5cm
\begin{proposition}\label{propositionHomology}
\mbox{}
\vskip 0.1cm
\noindent
Let \,$\mathfrak{C}$\, be a category with zero objects,
%\,$A, B,C \in \Obj(\mathfrak{C})$,\,
\,$f \in \Mor_{\mathfrak{C}}(A,B)$,\,
\,$g \in \Mor_{\mathfrak{C}}(B,C)$,\, such that
\,$g \circ f = 0 \in \Mor_{\mathfrak{C}}(A,C)$.\,
\vskip 0.1cm
\noindent
Suppose the category \,$\mathfrak{C}$\,
\begin{itemize}
\item
	has kernels, cokernels, equalizers, and
\item
	is normal and conormal.
\end{itemize}
Let
\begin{itemize}
\item
	$\pi_{f}$\, be a cokernel of \,$f$,
\item
	$\kappa(\pi_{f})$\, a kernel of \,$\pi_{f}$, and
\item
	$\kappa_{g}$\, a kernel of \,$g$.
\end{itemize}
Then,
\begin{enumerate}
\item
	there exists a unique morphism \,$\varepsilon_{f} \in \Mor_{\mathfrak{C}}(A,I_{f})$\,
	such that \,$f \,=\, \kappa(\pi_{f}) \circ \varepsilon_{f}$,\, and
\item
	the morphism \,$\varepsilon_{f}$\, is a coimage of \,$f$,\, and
	it is a cokernel of a kernel of \,$f$,\, and
\item
	there exists a unique morphism \,$\theta \in \Mor_{\mathfrak{C}}(I_{f},K_{g})$\,
	such that \,$\kappa(\pi_{f}) \,=\, \kappa_{g} \circ \theta$.
\end{enumerate}
\begin{center}
\begin{tikzcd}
&& && Q_{f}
	\arrow[dd, dashed, "\psi"]
\\ \\
A
	\arrow[rr, "f"]
	\arrow[dd, thick, dashed, two heads, swap, "\varepsilon_{f}", blue]
	\arrow[rrrr, bend left = 30, "0", gray]
&&
B
	\arrow[rr, "g"]
	\arrow[uurr, two heads, "\pi_{f}"]
&&
C
\\ \\
I_{f}
	\arrow[uurr, hook, "\kappa(\pi_{f})"]
	\arrow[rr, thick, dashed, swap, "\exists !\,\theta", red]
&&
K_{g}
	\arrow[uu, hook, swap, "\kappa_{g}"]
\end{tikzcd}
\end{center}
\end{proposition}
\proof
Note that \,(i)\, and \,(ii)\, are immediate consequences of Proposition \ref{FactorizationIntoImageCoimage}.
Thus, we need only prove \,(iii).\,
To this end, note that, by the universal property of \,$\kappa_{g}$\, as a kernel of \,$g$,\,
the desired unique morphism \,$\theta$\, will exist provided that \,$g \circ \kappa(\pi_{f}) = 0$,\,
which in turn easily follows from:
\begin{equation*}
g \circ \kappa(\pi_{f})
\;\; = \;\;
	\left(\,\psi \overset{{\color{white}!}}{\circ} \pi_{f}\,\right) \circ \kappa(\pi_{f})
\;\; = \;\;
	\psi \circ \left(\,\pi_{f} \overset{{\color{white}!}}{\circ} \kappa(\pi_{f})\,\right)
\;\; = \;\;
	\psi \circ 0
\;\; = \;\;
	0
\end{equation*}
This completes the proof of the Proposition.
\qed

          %%%%% ~~~~~~~~~~~~~~~~~~~~ %%%%%

\vskip 0.5cm
\begin{definition}[Exactness and homology of two morphisms composing to zero]
\label{defnExactness}
\mbox{}
\vskip 0.1cm
\noindent
Suppose:
\begin{itemize}
\item
	$\mathfrak{C}$\, is a category which has zero objects, kernels, cokernels, equalizers, and which is normal and conormal.
\item
	$f \in \Mor_{\mathfrak{C}}(A,B)$,\, and \,$g \in \Mor_{\mathfrak{C}}(B,C)$,\,
	are such that
	\,$g \circ f = 0 \in \Mor_{\mathfrak{C}}(A,C)$.\,
\end{itemize}
Let the entities in the following (commutative) diagram be as described in Proposition \ref{propositionHomology}:
\begin{center}
\begin{tikzcd}
&& && Q_{f}
	\arrow[dd, dashed, "\psi"]
\\ \\
A
	\arrow[rr, "f"]
	\arrow[dd, dashed, two heads, swap, "\varepsilon_{f}"]
	\arrow[rrrr, bend left = 30, "0", gray]
&&
B
	\arrow[rr, "g"]
	\arrow[uurr, two heads, "\pi_{f}"]
&&
C
\\ \\
I_{f}
	\arrow[uurr, hook, "\kappa(\pi_{f})"]
	\arrow[rr, thick, dashed, swap, "\exists !\,\theta", red]
&&
K_{g}
	\arrow[uu, hook, swap, "\kappa_{g}"]
	\arrow[rr, thick, two heads, swap, "\pi_{\theta}", red]
&&
Q_{\theta}
\end{tikzcd}
\end{center}
Then, we make the following definitions:
\begin{itemize}
\item
	The composition
	\,$A \,\overset{f}{\longrightarrow}\, B \,\overset{g}{\longrightarrow}\, C$\,
	is said to be \,\textbf{exact}\, (or, \,\textbf{exact at $B$}) if
	\,$\theta$\, is an isomorphism.
\item
	A \,\textbf{homology morphism}\, of the composition
	\,$A \,\overset{f}{\longrightarrow}\, B \,\overset{g}{\longrightarrow}\, C$\,
	is any cokernel \,$\pi_{\theta}$\, of \,$\theta$.
\item
	A \,\textbf{homology object}\, of
	\,$A \,\overset{f}{\longrightarrow}\, B \,\overset{g}{\longrightarrow}\, C$\,
	is the codomain \,$Q_{\theta}$
	of any homology morphism \,$K_{g}\,\overset{\pi_{\theta}}{\longrightarrow}\,Q_{\theta}$\,
	of
	\,$A \,\overset{f}{\longrightarrow}\, B \,\overset{g}{\longrightarrow}\, C$.
\end{itemize}
\end{definition}

          %%%%% ~~~~~~~~~~~~~~~~~~~~ %%%%%

\vskip 0.5cm
\begin{remark}
\mbox{}
\vskip 0.1cm
\noindent
If the composition
\,$A \,\overset{f}{\longrightarrow}\, B \,\overset{g}{\longrightarrow}\, C$\,
is exact, then its homology morphism is a zero morphism.
\end{remark}
\proof
Immediately by Lemma \ref{CokernelOfAnIsomorphismIsZero} (that every cokernel of an isomorphism is a zero morphism).
\qed


          %%%%% ~~~~~~~~~~~~~~~~~~~~ %%%%%
