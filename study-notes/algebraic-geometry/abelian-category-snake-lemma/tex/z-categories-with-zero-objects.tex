
          %%%%% ~~~~~~~~~~~~~~~~~~~~ %%%%%

\section{Categories with zero objects}
\setcounter{theorem}{0}
\setcounter{equation}{0}

%\cite{vanDerVaart1996}
%\cite{Kosorok2008}

%\renewcommand{\theenumi}{\alph{enumi}}
%\renewcommand{\labelenumi}{\textnormal{(\theenumi)}$\;\;$}
\renewcommand{\theenumi}{\roman{enumi}}
\renewcommand{\labelenumi}{\textnormal{(\theenumi)}$\;\;$}

          %%%%% ~~~~~~~~~~~~~~~~~~~~ %%%%%

In this section, we establish that a category with zero objects possesses sufficient structure
in order to define the notions of kernels and cokernels of morphisms
(although the kernel or cokernel of any particular morphism may or may not exist).

          %%%%% ~~~~~~~~~~~~~~~~~~~~ %%%%%

\begin{definition}[Category]
\mbox{}
\vskip 0.15cm
\noindent
A \,\textbf{category}\, $\mathfrak{C}$ consists of the following:
\begin{itemize}
\item
	a class $\Obj(\mathfrak{C})$,
\item
	a set $\Mor_{\mathfrak{C}}(A,B)$, for each $A, B \in \Obj(\mathfrak{C})$,
\item
	a \textbf{composition map}
	\begin{equation*}
	\Mor_{\mathfrak{C}}(A,B) \times \Mor_{\mathfrak{C}}(B,C) \longrightarrow \Mor_{\mathfrak{C}}(A,C) : (f,g) \longmapsto g \circ f\,,
	\end{equation*}
	for each $A, B, C \in \Obj(\mathfrak{C})$,
\end{itemize}
satisfying:
\begin{enumerate}
\item
	the sets $\Mor_{\mathfrak{C}}(\,\cdot\,,\,\cdot\,)$ are pairwise disjoint, i.e.,
	for $A_{1}, B_{1}, A_{2}, A_{2} \in \Obj(\mathfrak{C})$, we have:
	\begin{equation*}
	\Mor_{\mathfrak{C}}(A_{1},B_{2})\;\bigcap\;\Mor_{\mathfrak{C}}(A_{1},B_{2})
	\; = \;
		\varemptyset\,,
	\;\;
	\textnormal{whenever $A_{1} \neq A_{2}$ or $B_{1} \neq B_{2}$}
	\end{equation*}
\item
	for each $A \in \Obj(\mathfrak{C})$, there exists $1_{A} \in \Mor_{\mathfrak{C}}(A,A)$ such that
	\begin{equation*}
	f \circ 1_{A} = f
	\;\;\;\;\textnormal{and}\;\;\;\;
	1_{B} \circ f = f,
	\quad
	\textnormal{for each $f \in \Mor_{\mathfrak{C}}(A,B)$, \,$B \in \Obj(\mathfrak{C})$}
	\end{equation*}
\item
	the composition map is associative, i.e.,
	\begin{equation*}
	(f \circ g) \circ h
	\;\; = \;\;
	f \circ (g \circ h)\,,
	\end{equation*}
	for each
	$h \in \Mor_{\mathfrak{C}}(A,B)$,
	$g \in \Mor_{\mathfrak{C}}(B,C)$,
	$f \in \Mor_{\mathfrak{C}}(C,D)$,\;
	where
	\,$A,B,C,D \in \Obj(\mathfrak{C})$.
\end{enumerate}
The elements of $\Obj(\mathfrak{C})$ are called the \textbf{objects} of $\mathfrak{C}$.
For $A, B \in \Obj(\mathfrak{C})$, the elements of  $\Mor_{\mathfrak{C}}(A,B)$
are called the \textbf{morphisms} in $\mathfrak{C}$ from $A$ to $B$.
For each $A \in \Obj(\mathfrak{C})$, $1_{A} \in \Mor_{\mathfrak{C}}(A,A)$ is called
the \textbf{identity morphism} of $A$.
\end{definition}

          %%%%% ~~~~~~~~~~~~~~~~~~~~ %%%%%

\vskip 0.5cm
\begin{definition}[Initial object, terminal object, zero object]
\mbox{}
\vskip 0.15cm
\noindent
Let \,$\mathfrak{C}$\, be a category.
\begin{itemize}
\item
	An object $A \in \Obj(\mathfrak{C})$ is called an \textbf{initial object} if $\Mor_{\mathfrak{C}}(A,X)$ is a singleton set,
	for each $X \in \Obj(\mathfrak{C})$.
\item
	An object $\Omega \in \Obj(\mathfrak{C})$ is called a \textbf{terminal object} if $\Mor_{\mathfrak{C}}(X,\Omega)$ is a singleton set,
	for each $X \in \Obj(\mathfrak{C})$.
\item
	An object of $\Obj(\mathfrak{C})$ is called a \textbf{zero object} if it is both initial and terminal.
\end{itemize}
\end{definition}

          %%%%% ~~~~~~~~~~~~~~~~~~~~ %%%%%

\vskip 0.5cm
\begin{lemma}[Existence and uniqueness of zero morphisms in a category with zero objects]\label{ZeroMorphisms}
\mbox{}
\vskip 0.15cm
\noindent
Let \,$\mathfrak{C}$\, be a category with zero objects.
Then, for each $A, B \in \Obj(\mathfrak{C})$, we have:
\begin{equation*}
\iota_{0_{1},B}\circ\iota_{A,0_{1}}
\;\; = \;\;
	\iota_{0_{2},B}\circ\iota_{A,0_{2}}
\;\; \in \;\;
	\Mor_{\mathfrak{C}}(A,B),
\end{equation*}
where
$0_{1}, 0_{2} \in \Obj(\mathfrak{C})$ are two arbitrary zero objects,
$\iota_{A,0_{k}}$ is the unique element in $\Mor_{\mathfrak{C}}(A,0_{k})$,
$\iota_{0_{k},B}$ is the unique element in $\Mor_{\mathfrak{C}}(0_{k},B)$,
$i = 1,2$.
\end{lemma}
\proof
Note that $\iota_{0_{1},0_{2}}\circ\iota_{A,0_{1}}$ and $\iota_{A,0_{2}}$ are both elements of $\Mor_{\mathfrak{C}}(A,0_{2})$,
which is a singleton set since $0_{2}$ is a zero object. Hence, we conclude:
\begin{equation*}
\iota_{0_{1},0_{2}}\circ\iota_{A,0_{1}} \; = \; \iota_{A,0_{2}} \;\; \in \;\; \Mor_{\mathfrak{C}}(A,0_{2})
\end{equation*}
Similarly,
\begin{equation*}
\iota_{0_{2},B}\circ\iota_{0_{1},0_{2}} \; = \; \iota_{0_{1},B} \;\; \in \;\; \Mor_{\mathfrak{C}}(0_{1},B)
\end{equation*}
Hence,
\begin{equation*}
\iota_{0_{1},B} \circ \iota_{A,0_{1}}
\; = \;
	(\,\iota_{0_{2},B}\circ\iota_{0_{1},0_{2}}\,) \circ \iota_{A,0_{1}}
\; = \;
	\iota_{0_{2},B} \circ (\,\iota_{0_{1},0_{2}} \circ \iota_{A,0_{1}}\,)
\; = \;
	\iota_{0_{2},B} \circ \iota_{A,0_{2}}
\end{equation*}
\qed

\vskip 0.5cm
\noindent
Lemma \ref{ZeroMorphisms} implies that the following is well-defined:

          %%%%% ~~~~~~~~~~~~~~~~~~~~ %%%%%

\vskip 0.5cm
\begin{definition}[Zero morphisms in a category with zero objects]
\mbox{}
\vskip 0.15cm
\noindent
Let \,$\mathfrak{C}$\, be a category with zero objects, and $A, B \in \Obj(\mathfrak{C})$.
The \textbf{zero morphism} in $\Mor_{\mathfrak{C}}(A,B)$ is, by definition,
\begin{equation*}
0_{A,B}
\; := \;
	\iota_{B,0} \circ \iota_{A,0}
\;\; \in \;\;
	\Mor_{\mathfrak{C}}(A,B),
\end{equation*}
where $0$ denotes an arbitrary zero object of $\mathfrak{C}$.
\end{definition}

          %%%%% ~~~~~~~~~~~~~~~~~~~~ %%%%%

\vskip 0.5cm
\begin{lemma}[Compositions with zero morphisms are themselves zero morphisms]
\mbox{}
\vskip 0.15cm
\noindent
Let \,$\mathfrak{C}$\, be a category with zero objects.
Then, for each \,$A, B \in \Obj(\mathfrak{C})$\, and \,$f \in \Mor_{\mathfrak{C}}(A,B)$,\, we have:
\begin{enumerate}
\item
	$f \circ 0_{X,A} \,=\, 0_{X,B}$,\, for each $X \in \Obj(\mathfrak{C})$, and
\item
	$0_{B,Y} \circ f \,=\, 0_{A,Y}$,\, for each $Y \in \Obj(\mathfrak{C})$.
\end{enumerate}
\end{lemma}
\proof
Let $0 \in \Obj(\mathfrak{C})$ represent a fixed zero object in $\mathfrak{C}$.
Then,
\begin{enumerate}
\item
	\begin{eqnarray*}
	f \circ 0_{X,A}
	& = &
		f \circ (\,\iota_{0,A} \circ \iota_{X,0}\,)
	\\
	& = &
		(\,f \circ \iota_{0,A}\,) \circ \iota_{X,0}
	\\
	& = &
		(\,\iota_{0,B}\,) \circ \iota_{X,0}\,,
		\quad
		\textnormal{since $f \circ \iota_{0,A},\, \iota_{0,B} \in \Mor_{\mathfrak{C}}(0,B)$, and $0$ is a zero object}
	\\
	& = &
		0_{X,B}
	\end{eqnarray*}
\item
	\begin{eqnarray*}
	0_{B,Y} \circ f
	& = &
		(\,\iota_{0,Y} \circ \iota_{B,0}\,) \circ f
	\\
	& = &
		\iota_{0,Y} \circ (\,\iota_{B,0} \circ f\,)
	\\
	& = &
		\iota_{0,Y} \circ (\,\iota_{A,0}\,),
		\quad
		\textnormal{since $\iota_{A,0},\, \iota_{B,0} \circ f \in \Mor_{\mathfrak{C}}(A,0)$, and $0$ is a zero object}
	\\
	& = &
		0_{A,Y}
	\end{eqnarray*}
\end{enumerate}
\qed

          %%%%% ~~~~~~~~~~~~~~~~~~~~ %%%%%

\vskip 0.5cm
\begin{definition}[Kernels and cokernels of morphisms in a category with zero objects]
\mbox{}
\vskip 0.15cm
\noindent
Let \,$\mathfrak{C}$\, be a category with zero objects, and $f : A \longrightarrow B$ a morphism in $\mathfrak{C}$.
\begin{itemize}
\item
	A \textbf{kernel} of $f$ (if it exists) is a morphism $\kappa : K \longrightarrow A$ in $\mathfrak{C}$ such that
	\begin{itemize}
	\item
		\vskip -0.15cm
		$f \circ \kappa \,=\, 0_{K,B}$, and
	\item
		for each $\rho \in \Mor_{\mathfrak{C}}(X,A)$ satisfying $f \circ \rho = 0_{X,B}$,
		there exists a unique $\theta \in \Mor_{\mathfrak{C}}(X,K)$ such that the following diagram commutes:
		\begin{center}
		\begin{tikzcd}
		{\color{blue}X}
			\arrow[dd, thick, dashed, swap, "\exists !\,\theta{\color{white}.}", blue]
			\arrow[dr, thick, swap, "\rho", blue]
			\arrow[drrr, bend left = 15, "0", blue] & & &
		\\
		& {\color{red}A}
			\arrow[rr, swap, "f{\color{white}...}"] & & B
		\\
		{\color{red}K}
			\arrow[ur, swap, "\kappa", red]
			\arrow[urrr, bend right = 15, swap, "0", gray]
		\end{tikzcd}
		\end{center}
	\end{itemize}
\item
	A \textbf{cokernel} of $f$ (if it exists) is a morphism $\pi : B \longrightarrow Q$ in $\mathfrak{C}$ such that
	\begin{itemize}
	\item
		\vskip -0.15cm
		$\pi \circ f \,=\, 0_{A,Q}$, and
	\item
		for each $\sigma \in \Mor_{\mathfrak{C}}(B,Y)$ satisfying \,$\sigma \circ f = 0$,
		there exists a unique $\theta \in \Mor_{\mathfrak{C}}(Q,Y)$ such that the following diagram commutes:
		\begin{center}
		\begin{tikzcd}
		& & & {\color{red}Q}
			\arrow[dd, thick, dashed, "\;\exists !\,\theta", blue]
		\\
		A 
			\arrow[rr, "{\color{white}...}f"]
			\arrow[rrru, bend left = 15, "0", gray]
			\arrow[rrrd, bend right = 15, swap, "0", blue]
			& &
			{\color{red}B}
			\arrow[rd, thick, "\sigma", blue]
			\arrow[ur, "\pi", red]
			&
		\\
		& & & {\color{blue}Y}
		\end{tikzcd}
		\end{center}
	\end{itemize}
%\item
%	An \textbf{image} of $f$ is a kernel of a cokernel of $f$.
\end{itemize}
\end{definition}

          %%%%% ~~~~~~~~~~~~~~~~~~~~ %%%%%

\vskip 0.5cm
\begin{remark}[Motivation behind the definition of the image of a morphism]
\mbox{}
\vskip 0.15cm
\noindent
Suppose $V$ and $W$ are vector spaces and $f : V \longrightarrow W$ is a linear map.
Then, we have:
\begin{center}
\begin{tikzcd}
V \arrow[r, "f"] & W \arrow[r, "\pi"] & \textnormal{coker}(f) & \!\!\!\!\!\!\!\!\!\!\!\!\!\!\!\! :=\, W/\textnormal{image}(f)
\end{tikzcd}
\end{center}
where $\pi$ is the standard projection map, and
\,$\textnormal{image}(f) \,=\, \ker(\pi) \,\subset\, W$.
\end{remark}

          %%%%% ~~~~~~~~~~~~~~~~~~~~ %%%%%

\vskip 0.5cm
\begin{proposition}[Kernels and cokernels are unique up to canonical isomorphism]
\mbox{}
\vskip 0.15cm
\noindent
Let \,$\mathfrak{C}$\, be a category with zero objects.
Let \,$f \in \Mor_{\mathfrak{C}}(A,B)$,\, where \,$A, B \in \Obj(\mathfrak{C})$.
Then, the following statements are true:
\begin{enumerate}
\item
	If \,$\kappa_{1} \in \Mor_{\mathfrak{C}}(K_{1},A)$\, and \,$\kappa_{2} \in \Mor_{\mathfrak{C}}(K_{2},A)$
	are two kernels of $f$, then there exists a unique isomorphism $\theta \in \Mor_{\mathfrak{C}}(K_{2},K_{1})$
	such that the following diagram commutes:
	\begin{center}
	\begin{tikzcd}
	{\color{blue}K_{2}}
		\arrow[dd, thick, dashed, swap, "\exists !\,\theta{\color{white}.}", blue]
		\arrow[dr, thick, swap, "\kappa_{2}", blue]
		\arrow[drrr, bend left = 15, "0", blue] & & &
	\\
	& {\color{red}A}
		\arrow[rr, swap, "f{\color{white}...}"] & & B
	\\
	{\color{red}K_{1}}
		\arrow[ur, swap, "\kappa_{1}", red]
		\arrow[urrr, bend right = 15, swap, "0", gray]
	\end{tikzcd}
	\end{center}
\item
	If \,$\pi_{1} \in \Mor_{\mathfrak{C}}(B,Q_{1})$\, and \,$\pi_{2} \in \Mor_{\mathfrak{C}}(B,Q_{2})$
	are two cokernels of $f$, then there exists a unique isomorphism $\theta \in \Mor_{\mathfrak{C}}(Q_{1},Q_{2})$
	such that the following diagram commutes:
	\begin{center}
	\begin{tikzcd}
	& & & {\color{red}Q_{1}}
		\arrow[dd, thick, dashed, "\;\exists !\,\theta", blue]
	\\
	A 
		\arrow[rr, "{\color{white}...}f"]
		\arrow[rrru, bend left = 15, "0", gray]
		\arrow[rrrd, bend right = 15, swap, "0", blue]
		& &
		{\color{red}B}
		\arrow[ur, "\pi_{1}", red]
		\arrow[rd, thick, "\pi_{2}", blue]
		&
	\\
	& & & {\color{blue}Q_{2}}
	\end{tikzcd}
	\end{center}
\end{enumerate}
\end{proposition}
\proof
\begin{enumerate}
\item
	By the definition of kernels (universal property), there exist unique morphisms
	$\theta_{12} \in \Mor_{\mathfrak{C}}(K_{1},K_{2})$ and $\theta_{21} \in \Mor_{\mathfrak{C}}(K_{2},K_{1})$
	such that the following diagram commutes:
	\begin{center}
	\begin{tikzcd}
	{\color{blue}K_{2}}
		\arrow[dd, thick, dashed, bend right = 90, swap, "\exists !\,\theta_{21}{\color{white}.}", blue]
		\arrow[dr, thick, swap, "\kappa_{2}", blue]
		\arrow[drrr, bend left = 15, "0", blue] & & &
	\\
	& {\color{red}A}
		\arrow[rr, swap, "f{\color{white}...}"] & & B
	\\
	{\color{red}K_{1}}
		\arrow[ur, swap, "\kappa_{1}", red]
		\arrow[uu, thick, dashed, bend left = 60, swap, "\;\,\exists !\,\theta_{12}{\color{white}.}", red]
		\arrow[urrr, bend right = 15, swap, "0", red]
	\end{tikzcd}
	\end{center}
	Again, by the universal property of kernels, we see that we must have
	$\theta_{21} \circ \theta_{12} \,=\, 1_{K_{1}}$.
	By symmetry (i.e., interchanging the indices 1 and 2), we have $\theta_{12} \circ \theta_{21} \,=\, 1_{K_{2}}$.
	This shows that $\theta_{12}$ and $\theta_{21}$ are isomorphisms and are inverses of each other.
	Taking $\theta$ in the proposition statement to be $\theta_{21}$, this completes the proof.
\item
	By the definition of cokernels (universal property), there exist unique morphisms
	$\theta_{12} \in \Mor_{\mathfrak{C}}(Q_{1},Q_{2})$ and $\theta_{21} \in \Mor_{\mathfrak{C}}(Q_{2},Q_{1})$
	such that the following diagram commutes:
	\begin{center}
	\begin{tikzcd}
	& & & {\color{red}Q_{1}}
		\arrow[dd, thick, bend left = 90, dashed, "\;\exists !\,\theta_{12}", blue]
	\\
	A 
		\arrow[rr, "{\color{white}...}f"]
		\arrow[rrru, bend left = 15, "0", gray]
		\arrow[rrrd, bend right = 15, swap, "0", blue]
		& &
		{\color{red}B}
		\arrow[ur, "\pi_{1}", red]
		\arrow[rd, thick, "\pi_{2}", blue]
		&
	\\
	& & & {\color{blue}Q_{2}}
		\arrow[uu, thick, bend right = 60, dashed, "\;\exists !\,\theta_{21}\,", red]
	\end{tikzcd}
	\end{center}
	Again, by the universal property of cokernels, we see that we must have
	$\theta_{21} \circ \theta_{12} \,=\, 1_{Q_{1}}$.
	By symmetry (i.e., interchanging the indices 1 and 2), we have $\theta_{12} \circ \theta_{21} \,=\, 1_{Q_{2}}$.
	This shows that $\theta_{12}$ and $\theta_{21}$ are isomorphisms and are inverses of each other.
	Taking $\theta$ in the proposition statement to be $\theta_{12}$, this completes the proof.
	\qed
\end{enumerate}

          %%%%% ~~~~~~~~~~~~~~~~~~~~ %%%%%

\vskip 0.5cm
\begin{definition}[Monomorphism, epimorphism]
\mbox{}
\vskip 0.1cm
\noindent
Let $\mathfrak{C}$ be a category, and $A, B \in \Obj(\mathfrak{C})$.
A morphism $f \in \Mor_{\mathfrak{C}}(A,B)$ is called
\begin{itemize}
\item
	a \textbf{monomorphism} if it can be {\color{red}left} cancelled in the sense that
	$f \circ g = f \circ h$  implies that $g = h$.
\item
	a \textbf{epimorphism} if it can be {\color{red}right} cancelled in the sense that
	$g \circ f = h \circ f$  implies that $g = h$.
\item
	an \textbf{isomorphism} if there is a morphism $g \in \Mor_{\mathfrak{C}}(B,A)$
	such that $g \circ f = 1_{A}$ and $f \circ g = 1_{B}$.
	Clearly, such a morphism $g$ is unique, and it is called the inverse of $f$.
	The inverse of $f$, if exists, is denoted by $f^{-1}$.
\end{itemize}
\end{definition}

          %%%%% ~~~~~~~~~~~~~~~~~~~~ %%%%%

\vskip 0.5cm
\begin{proposition}
\mbox{}
\vskip 0.1cm
\noindent
In a category with zero objects, every kernel of a morphism is a monomorphism, and
every cokernel of a morphism is an epimorphism.
\end{proposition}
\proof
Let $\mathfrak{C}$ be a category with zero objects.
Let $\kappa \in \Mor_{\mathfrak{C}}(K,A)$ be a kernel of $f \in \Mor_{\mathfrak{C}}(A,B)$.
Let $\eta, \mu \in \Mor_{\mathfrak{C}}(X,K)$ be such that
$\kappa \circ \eta \,=\, \kappa \circ \mu \,\in\, \Mor_{\mathfrak{C}}(X,A)$.
We need to show that $\eta = \mu$.
To this end, first, we define $\rho \,:=\, \kappa \circ \eta \,=\, \kappa \circ \mu \,\in\, \Mor_{\mathfrak{C}}(X,A)$.
Then, note
$f \circ \rho \,=\, f \circ (\kappa \circ \eta) \,=\, (f \circ \kappa) \circ \eta \,=\, 0_{K,B} \circ \eta \,=\, 0_{X,B}$.
Since $\kappa$ is a kernel of $f$, we see that there exists a unique $\theta \in \Mor_{\mathfrak{C}}(X,K)$ such that
the following diagram commutes:
\begin{center}
\begin{tikzcd}
X
	\arrow[dd, thick, dashed, bend right = 90, swap, "\exists !\,\theta{\color{white}.}", blue]
	\arrow[dd, shift left, "\mu"]
	\arrow[dd, shift right, swap, "\eta"]
	\arrow[dr, thick, "\rho", blue]
	\arrow[drrr, bend left = 15, "0", blue]
	& &
\\
	& A \arrow[rr, swap, "f"] & & B
\\
{\color{red}K}
	\arrow[ur, swap, "\kappa", red]
	\arrow[urrr, bend right = 15, swap, "0", gray]
\end{tikzcd}
\end{center}
The uniqueness of $\theta$ therefore implies $\eta = \theta = \mu$, as required.
This shows that $\kappa$ can be left cancelled.
Hence, it is a monomorphism.

\vskip 0.1cm
The second part follows similarly.
Indeed, let 
$\pi \in \Mor_{\mathfrak{C}}(B,Q)$ be a cokernel of $f \in \Mor_{\mathfrak{C}}(A,B)$.
Let $\zeta, \nu \in \Mor_{\mathfrak{C}}(Q,Y)$ be such that
$\zeta \circ \pi \,=\, \nu \circ \pi \,\in\, \Mor_{\mathfrak{C}}(B,Y)$.
We need to show that $\zeta = \nu$.
To this end, we define $\sigma \,:=\, \zeta \circ \pi \,=\, \nu \circ \pi \,\in\, \Mor_{\mathfrak{C}}(B,Y)$.
Then, we have: 
$\sigma \circ f \,=\, (\zeta \circ \pi) \circ f \,=\, \zeta \circ (\pi \circ f) \,=\, \zeta \circ 0_{A,Q} \,=\, 0_{A,Y}$.
Since $\pi$ is a cokernel of $f$, we see that there is a unique $\theta \in \Mor_{\mathfrak{C}}(Q,Y)$
such that the following diagram commutes:
\begin{center}
\begin{tikzcd}
& & &
	{\color{red}Q}
	\arrow[dd, thick, dashed, bend left = 90, "\;\exists !\,\theta", blue]
	\arrow[dd, shift left, "\eta"]
	\arrow[dd, shift right, swap, "\zeta"]
\\
A
	\arrow[rr, "f"]
	\arrow[urrr, bend left = 15, swap, "0", gray]
	\arrow[drrr, bend right = 15, swap, "0", blue]
	&&
	{\color{red}B}
	\arrow[ur, "\pi", red]
	\arrow[rd, thick, "\sigma", blue]
	&
\\
& & & Y
\end{tikzcd}
\end{center}
The uniqueness of $\theta$ therefore implies $\zeta = \theta = \nu$, as required.
This shows that $\pi$ can be right cancelled.
Hence, it is an epimorphism.
This completes the proof of the Proposition.
\qed

          %%%%% ~~~~~~~~~~~~~~~~~~~~ %%%%%


\vskip 0.5cm
\begin{definition}[Images of a morphisms in a category]
\mbox{}
\vskip 0.15cm
\noindent
Let \,$\mathfrak{C}$\, be a category, and $f : A \longrightarrow B$ a morphism in $\mathfrak{C}$.
\begin{itemize}
\item
	\vskip -0.1cm
	An \textbf{image} of $f$ (if it exists) is a monomorphism $\iota : I \longrightarrow B$ in $\mathfrak{C}$ such that
	\begin{itemize}
	\item
		$f = \iota \circ \nu$,\, for some \,$\nu \in \Mor_{\mathfrak{C}}(A,I)$, and
	\item
		for each factorization $f = \iota^{\prime} \circ \nu^{\prime}$ of $f$
		where $\iota^{\prime} \in \Mor_{\mathfrak{C}}(I^{\prime},B)$ is a monomorphism,
		there exists a unique $\theta \in \Mor_{\mathfrak{C}}(I,I^{\prime})$ such that the following diagram commutes:
		\begin{center}
		\begin{tikzcd}
		A \arrow[rr, "f"] \arrow[dr, swap, "\nu"] \arrow[ddr, bend right, swap, "\nu^{\prime}"] && B \\
		& {\color{red}I} \arrow[ur, hook, swap, "\iota", red] \arrow[d, dashed, "\;\exists ! \, \theta"] & \\
		& I^{\prime} \arrow[uur, bend right, hook, swap, "\iota^{\prime}"] & \\
		\end{tikzcd}
		\end{center}
	\end{itemize}
\item
	A \textbf{coimage} of $f$ (if it exists) is an epimorphism $\varepsilon : A \longrightarrow C$ in $\mathfrak{C}$ such that
	\begin{itemize}
	\item
		$f \,=\, \mu \circ \varepsilon$,\, for some $\mu \in \Mor_{\mathfrak{C}}(C,B)$, and
	\item
		for each factorization $f = \mu^{\prime} \circ \varepsilon^{\prime}$ of $f$
		where $\varepsilon^{\prime} \in \Mor_{\mathfrak{C}}(J,B)$ is an epimorphism,
		there exists a unique $\theta \in \Mor_{\mathfrak{C}}(C^{\prime},C)$ such that the following diagram commutes:
		\begin{center}
		\begin{tikzcd}
		A \arrow[rr, "f"] \arrow[dr, two heads, swap, "\varepsilon", red] \arrow[ddr, two heads, bend right, swap, "\varepsilon^{\prime}"] && B \\
		& {\color{red}C} \arrow[ur, swap, "\mu"] & \\
		& C^{\prime} \arrow[u, dashed, swap, "\;\exists ! \, \theta"] \arrow[uur, bend right, swap, "\mu"] & \\
		\end{tikzcd}
		\end{center}
	\end{itemize}
\end{itemize}
\end{definition}

          %%%%% ~~~~~~~~~~~~~~~~~~~~ %%%%%

\vskip 0.5cm
\begin{proposition}[Lemma 14.4, p.17, \cite{Mitchell1965}]
\mbox{}
\vskip 0.1cm
\noindent
Suppose \,$\mathfrak{C}$\, is a category with zero objects,
\,$A, B \in \Obj(\mathfrak{C})$,\, and
\,$f \in \Mor_{\mathfrak{C}}(A,B)$.
If
\begin{itemize}
\item
	$\pi_{f} \in \Mor_{\mathfrak{C}}(B,Q)$\, is a cokernel of $f$, and
\item
	$\kappa(\pi_{f}) \in \Mor_{\mathfrak{C}}(I,B)$\, is a kernel of \,$\pi_{f}$,
\end{itemize}
then the following statements are true:
\begin{enumerate}
\item
	There exists a unique morphism \,$\widetilde{f} \in \Mor_{\mathfrak{C}}(A,I)$\,
	that makes the following diagram commute: %(due to the universal property of $\kappa(\pi_{f})$):
	\begin{center}
	\begin{tikzcd}
	A
		\arrow[dr, thick, swap, "f", blue]
		\arrow[dd, thick, dashed, swap, "\exists !\,\widetilde{f}{\color{white}.}", blue]
		\arrow[drrr, bend left = 20, "0", blue]
	\\
		&
		B
		\arrow[rr, "\pi_{f}{\color{white}..}"] & & Q
	\\
	I
		\arrow[ur, swap, "\kappa(\pi_{f})", red] 
		\arrow[urrr, bend right = 20, swap, "0", gray]
	\end{tikzcd}
	\end{center}
\item
	If $\mathfrak{C}$ has cokernels and
	is normal\,\footnote{A category is normal if every monomorphism in it is a kernel.},
	then $\kappa(\pi_{f})$ is an image of $f$.
\item
	Furthermore, if $\mathfrak{C}$ has equalizers, then $\widetilde{f}$ is a coimage of $f$.
\end{enumerate}
\end{proposition}

          %%%%% ~~~~~~~~~~~~~~~~~~~~ %%%%%
