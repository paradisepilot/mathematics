
          %%%%% ~~~~~~~~~~~~~~~~~~~~ %%%%%

\section{Additive categories}
\setcounter{theorem}{0}
\setcounter{equation}{0}

%\cite{vanDerVaart1996}
%\cite{Kosorok2008}

%\renewcommand{\theenumi}{\alph{enumi}}
%\renewcommand{\labelenumi}{\textnormal{(\theenumi)}$\;\;$}
\renewcommand{\theenumi}{\roman{enumi}}
\renewcommand{\labelenumi}{\textnormal{(\theenumi)}$\;\;$}

          %%%%% ~~~~~~~~~~~~~~~~~~~~ %%%%%

\begin{definition}[Product, coproduct]
\mbox{}
\vskip 0.15cm
\noindent
Let \,$\mathfrak{C}$\, be a category, and
$\{\,A_{i}\,\}_{i \in I}$ a family of objects of $\mathfrak{C}$ indexed by a set $I$.
\begin{itemize}
\item
	A \textbf{product} is an ordered pair
	\,$\left(\,P\,,\,\{\,\pi_{i} : P \overset{{\color{white}1}}{\longrightarrow} A_{i}\,\}_{i \in I}\,\right)$\,
	consisting of an object $P \in \Obj(\mathfrak{C})$ and a family 
	$\{\,\pi_{i} : P \longrightarrow A_{i}\,\}_{i \in I}$ of morphisms in $\mathfrak{C}$
	such that,
	for every object $X \in \Obj(\mathfrak{C})$ and morphisms $f_{i} \in \Mor_{\mathfrak{C}}(X,A_{i})$,
	there exists a unique morphism $\theta \in \Mor_{\mathfrak{C}}(X,P)$ such that,
	for each $i \in I$, the following diagram commutes:
	\begin{center}
	\begin{tikzcd}
	& {\color{red}P} \arrow[d, "\pi_{i}", red] \\
	X \arrow[ru, dashed,"\exists\,!\;\theta"] \arrow[r, swap, "f_{i}"] & {\color{red}A_{i}}
	\end{tikzcd}
	\end{center}
	If the product exists, it is denoted by: $\underset{i \in I}{\bigsqcap}\,A_{i}$.
	It is unique up to isomorphism.
\item
	A \textbf{coproduct} is an ordered pair
	\,$\left(\,Q\,,\,\{\,\alpha_{i} : A_{i} \overset{{\color{white}1}}{\longrightarrow} Q\,\}_{i \in I}\,\right)$\,
	consisting of an object $Q \in \Obj(\mathfrak{C})$ and a family 
	$\{\,\alpha_{i} : A_{i} \longrightarrow Q\,\}_{i \in I}$ of morphisms in $\mathfrak{C}$
	such that,
	for every object $X \in \Obj(\mathfrak{C})$ and morphisms $g_{i} \in \Mor_{\mathfrak{C}}(A_{i},X)$,
	there exists a unique morphism $\theta \in \Mor_{\mathfrak{C}}(Q,X)$ such that,
	for each $i \in I$, the following diagram commutes:
	\begin{center}
	\begin{tikzcd}
	& {\color{red}Q} \arrow[ld, dashed, swap, "\exists\,!\;\theta"] \\
	X & {\color{red}A_{i}} \arrow[l, "g_{i}"] \arrow[u, swap, "\alpha_{i}", red]
	\end{tikzcd}
	\end{center}
	If the coproduct exists, it is denoted by: $\underset{i \in I}{\bigsqcup}\,A_{i}$.
	It is unique up to isomorphism.
\end{itemize}
\end{definition}

          %%%%% ~~~~~~~~~~~~~~~~~~~~ %%%%%

\vskip 0.5cm
\begin{definition}[Additive category]
\mbox{}
\vskip 0.15cm
\noindent
A category \,$\mathfrak{C}$\, is said to be \textbf{additive} if
\begin{itemize}
\item
	the category $\mathfrak{C}$ has a zero object,
\item
	$\Mor_{\mathfrak{C}}(A,B)$ is an abelian group, for each \,$A, B \in \Obj(\mathfrak{C})$,
\item
	the distributive law holds for the morphism composition map, i.e.
	\begin{equation*}
	h \circ (f + g) \; = \; h \circ f + h \circ g
	\quad\textnormal{and}\quad
	(f + g) \circ k \; = \; f \circ k + g \circ k
	\end{equation*}
	for each
	$f, g \in \Mor_{\mathfrak{C}}(A,B)$,
	$h \in \Mor_{\mathfrak{C}}(B,Y)$,
	$k \in \Mor_{\mathfrak{C}}(X,A)$,
	$A, B, X, Y \in \Obj(\mathfrak{C})$, and
\item
	finite products and finite coproducts exist in $\mathfrak{C}$.
\end{itemize}
\end{definition}

          %%%%% ~~~~~~~~~~~~~~~~~~~~ %%%%%

\vskip 0.5cm
\begin{proposition}[Finite products are also finite coproducts and vice versa in an additive category]
\mbox{}
\vskip 0.15cm
\noindent
Let \,$\mathfrak{C}$\, be an additive category, and \,$A_{1}, A_{2} \in \Obj(\mathfrak{C})$.\,
Then, the following statements are true:
\begin{enumerate}
\item
	Suppose the ordered triple \,$(P,\pi_{1},\pi_{2})$\, is a product of \,$A_{1}$\, and \,$A_{2}$,\,
	where \,$P \in \Obj(\mathfrak{C})$, \,$\pi_{1} \in \Mor_{\mathfrak{C}}(P,A_{1})$, \,$\pi_{2} \in \Mor_{\mathfrak{C}}(P,A_{2})$.\,
	Then, the following statements are true:
	\begin{itemize}
	\item
		there exists a unique
		\,$\alpha_{1} \in \Mor_{\mathfrak{C}}(A_{1},P)$\,
		such that
		\;$\pi_{1} \circ \alpha_{1} \,=\, 1_{A_{1}}$\, and \;$\pi_{2} \circ \alpha_{1} \,=\, 0_{A_{1},A_{2}}$,\, and
	\item
		there exists a unique
		\,$\alpha_{2} \in \Mor_{\mathfrak{C}}(A_{2},P)$\,
		such that
		\;$\pi_{2} \circ \alpha_{2} \,=\, 1_{A_{2}}$\, and \;$\pi_{1} \circ \alpha_{2} \,=\, 0_{A_{2},A_{1}}$.\,
	\end{itemize}
	Furthermore, the following statements also hold:
	\begin{itemize}
	\item
		$\pi_{1}$, \,$\pi_{2}$, \,$\alpha_{1}$, \,$\alpha_{2}$\, satisfy:
		\;$\alpha_{1} \circ \pi_{1} \,+\, \alpha_{2} \circ \pi_{2} \; = \; 1_{P}$,\, and
	\item
		$(P,\alpha_{1},\alpha_{2})$\, is a coproduct of \,$A_{1}$\, and \,$A_{2}$.\,
	\end{itemize}
\item
	Suppose the ordered triple \,$(Q,\alpha_{1},\alpha_{2})$\, is a coproduct of \,$A_{1}$\, and \,$A_{2}$,\,
	where
	\,$Q \in \Obj(\mathfrak{C})$,
	\,$\alpha_{1} \in \Mor_{\mathfrak{C}}(A_{1},Q)$,
	\,$\alpha_{2} \in \Mor_{\mathfrak{C}}(A_{2},Q)$.\,
	Then, there {\color{red}exist unique} morphisms
	\,$\pi_{1} \in \Mor_{\mathfrak{C}}(Q,A_{1})$\, and \,$\pi_{2} \in \Mor_{\mathfrak{C}}(Q,A_{2})$\,
	such that the following statements hold:
	\begin{itemize}
	\item
		$\pi_{1} \circ \alpha_{1} \,=\, 1_{A_{1}}$,\; and \,\;$\pi_{2} \circ \alpha_{1} \,=\, 0_{A_{1},A_{2}}$,\,
	\item
		$\pi_{1} \circ \alpha_{1} \,=\, 1_{A_{1}}$,\; and \,\;$\pi_{2} \circ \alpha_{2} \,=\, 1_{A_{2}}$,\,
	\item
		$\alpha_{1} \circ \pi_{1} \,+\, \alpha_{2} \circ \pi_{2} \; = \; 1_{P}$.\,
	\end{itemize}
	Furthermore, \,$(Q,\pi_{1},\pi_{2})$\, is a product of \,$A_{1}$\, and \,$A_{2}$.\,
\end{enumerate}
\end{proposition}
\proof
\begin{enumerate}
\item
	First, consider the following pair of diagrams:
	\begin{center}
	\begin{tikzcd}
	& {\color{red}P} \arrow[d, "\pi_{1}", red] \\
	A_{1} \arrow[ru, dashed,"\exists\,!\;\alpha_{1}"] \arrow[r, swap, "1_{A_{1}}"] & {\color{red}A_{1}}
	\end{tikzcd}
	\quad\quad\quad\quad
	\begin{tikzcd}
	& {\color{red}P} \arrow[d, "\pi_{2}", red] \\
	A_{1} \arrow[ru, dashed,"\exists\,!\;\alpha_{1}"] \arrow[r, swap, "0_{A_{1},A_{2}}"] & {\color{red}A_{2}}
	\end{tikzcd}
	\end{center}
	Since \,$(P,\pi_{1},\pi_{2})$\, is a product of $A_{1}$ and $A_{2}$,
	there exist unique $\alpha_{1} \in \Mor_{\mathfrak{C}}(A_{1},P)$ such that the above two diagrams commute.
	\vskip 0.1cm
	\noindent
	Next, consider the following pair of diagrams:
	\begin{center}
	\begin{tikzcd}
	& {\color{red}P} \arrow[d, "\pi_{1}", red] \\
	A_{2} \arrow[ru, dashed,"\exists\,!\;\alpha_{2}"] \arrow[r, swap, "0_{A_{2},A_{1}}"] & {\color{red}A_{1}}
	\end{tikzcd}
	\quad\quad\quad\quad
	\begin{tikzcd}
	& {\color{red}P} \arrow[d, "\pi_{2}", red] \\
	A_{2} \arrow[ru, dashed,"\exists\,!\;\alpha_{2}"] \arrow[r, swap, "1_{A_{2}}"] & {\color{red}A_{2}}
	\end{tikzcd}
	\end{center}
	Since \,$(P,\pi_{1},\pi_{2})$\, is a product of $A_{1}$ and $A_{2}$,
	there exist unique $\alpha_{2} \in \Mor_{\mathfrak{C}}(A_{2},P)$ such that the above two diagrams commute.
	\vskip 0.1cm
	\noindent
	Next, we prove that \,$\alpha_{1} \circ \pi_{1} \,+\, \alpha_{2} \circ \pi_{2} \; = \; 1_{P}$.\,
	To this end, consider the following diagrams:
	\begin{center}
	\begin{tikzcd}
	& {\color{red}P} \arrow[d, "\pi_{1}", red] \\
	P \arrow[ru, dashed,"\exists\,!\;\theta"] \arrow[r, swap, "\pi_{1}"] & {\color{red}A_{1}}
	\end{tikzcd}
	\quad\quad\quad\quad
	\begin{tikzcd}
	& {\color{red}P} \arrow[d, "\pi_{2}", red] \\
	P \arrow[ru, dashed,"\exists\,!\;\theta"] \arrow[r, swap, "\pi_{2}"] & {\color{red}A_{2}}
	\end{tikzcd}
	\end{center}
	Clearly, the choice \,$\theta = 1_{P}$\, satisfies the commutativity of the above diagrams;
	however, so does \,$\theta = \alpha_{1} \circ \pi_{1} \,+\, \alpha_{2} \circ \pi_{2}$:\,
	\begin{eqnarray*}
	\pi_{1} \,\circ \left(\,\alpha_{1} \circ \pi_{1} \,\overset{{\color{white}.}}{+}\, \alpha_{2} \circ \pi_{2}\,\right)
	& = &
		\pi_{1} \circ \alpha_{1} \circ \pi_{1} \,+\, \pi_{1} \circ \alpha_{2} \circ \pi_{2}
	\;\; = \;\;
		1_{A_{1}} \circ \pi_{1} \,+\, 0_{A_{2},A_{1}} \circ \pi_{2}
	\;\; = \;\;
		\pi_{1}
	\\
	\pi_{2} \,\circ \left(\,\alpha_{1} \circ \pi_{1} \,\overset{{\color{white}.}}{+}\, \alpha_{2} \circ \pi_{2}\,\right)
	& = &
		\pi_{2} \circ \alpha_{1} \circ \pi_{1} \,+\, \pi_{2} \circ \alpha_{2} \circ \pi_{2}
	\;\; = \;\;
		0_{A_{1}A_{2}} \circ \pi_{1} \,+\, \,1_{A_{2}} \circ \pi_{2}
	\;\; = \;\;
		\pi_{2}
	\end{eqnarray*}
	The universal property of the product now implies that
	\,$\alpha_{1} \circ \pi_{1} \,+\, \alpha_{2} \circ \pi_{2} \,=\, 1_{P}$.\,

\item
\end{enumerate}
\qed

          %%%%% ~~~~~~~~~~~~~~~~~~~~ %%%%%
