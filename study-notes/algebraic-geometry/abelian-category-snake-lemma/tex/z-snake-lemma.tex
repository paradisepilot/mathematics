
          %%%%% ~~~~~~~~~~~~~~~~~~~~ %%%%%

\section{The Snake Lemma for abelian categories with enough projectives}
\setcounter{theorem}{0}
\setcounter{equation}{0}

%\cite{vanDerVaart1996}
%\cite{Kosorok2008}

%\renewcommand{\theenumi}{\alph{enumi}}
%\renewcommand{\labelenumi}{\textnormal{(\theenumi)}$\;\;$}
\renewcommand{\theenumi}{\roman{enumi}}
\renewcommand{\labelenumi}{\textnormal{(\theenumi)}$\;\;$}

          %%%%% ~~~~~~~~~~~~~~~~~~~~ %%%%%

\begin{definition}
\mbox{}
\vskip 0.1cm
\noindent
Let \,$\mathfrak{C}$\, be category.
\begin{itemize}
\item
	An object
	\,$P \in \Obj(\mathfrak{C})$\,
	is said to be \textbf{projective} if,
	given any epimorphism
	\,$\varepsilon \in \Mor_{\mathfrak{C}}(A,B)$\,
	and any morphism
	\,$f \in \Mor_{\mathfrak{C}}(P,B)$,\,
	there exists a (not necessarily unique) morphism
	\,$\widetilde{f} \in \Mor_{\mathfrak{C}}(P,A)$\,
	such that
	\,$f \, = \, \varepsilon \, \circ \, \widetilde{f}$,\,
	i.e., the following diagram commutes:
	\begin{center}
	\begin{tikzcd}
	& A \arrow[d, two heads, "\varepsilon"] \\
	P \arrow[ru, dashed,"\exists\;\widetilde{f}"] \arrow[r, swap, "f"] & B
	\end{tikzcd}
	\end{center}
	(Note: Intuitively, \,$\widetilde{f}$\, here could be regarded as a ``{\color{red}codomain extension}'' of \,$f$.
	So, roughly speaking, \,$P \in \Obj(\mathfrak{C})$\, is projective if every morphism in \,$\mathfrak{C}$\,
	with domain \,$P$\, can be codomain-extended.)
\item
	An object
	\,$I \in \Obj(\mathfrak{C})$\,
	is said to be \textbf{injective} if,
	given any monomorphism
	\,$\iota \in \Mor_{\mathfrak{C}}(A,B)$\,
	and any morphism
	\,$f \in \Mor_{\mathfrak{C}}(A,I)$,\,
	there exists a (not necessarily unique) morphism
	\,$\widetilde{f} \in \Mor_{\mathfrak{C}}(B,I)$\,
	such that
	\,$f \, = \, \widetilde{f} \circ \iota$,\,
	i.e., the following diagram commutes:
	\begin{center}
	\begin{tikzcd}
	& B \arrow[ld, dashed, swap,"\exists\;\widetilde{f}"] \\
	I & A  \arrow[l, "\;\;f"] \arrow[u, hook, swap, "\iota"]
	\end{tikzcd}
	\end{center}
	(Note: Intuitively, \,$\widetilde{f}$\, here could be regarded as a ``{\color{red}domain extension}'' of \,$f$.
	So, roughly speaking, \,$I \in \Obj(\mathfrak{C})$\, is injective if every morphism in \,$\mathfrak{C}$\,
	with codomain \,$I$\, can be domain-extended.)
\item
	The category \,$\mathfrak{C}$\, is said to \textbf{have enough projectives},
	if for every object \,$B \in \Obj(\mathfrak{A})$,\,
	there is a projective object \,$P \in \Obj(\mathfrak{C})$,\,
	together with an epimorphism \,$\varepsilon \in \Mor_{\mathfrak{C}}(P,B)$.
\item	
	Dually, the category \,$\mathfrak{C}$\, is said to \textbf{have enough injectives},
	if for every object \,$A \in \Obj(\mathfrak{C})$,\,
	there is an injective object \,$I \in \Obj(\mathfrak{C})$\,
	together with a monomorphism \,$\mu \in \Mor_{\mathfrak{C}}(A,I)$.
\end{itemize}
\end{definition}

          %%%%% ~~~~~~~~~~~~~~~~~~~~ %%%%%

\vskip 0.5cm
\begin{remark}[Snake Lemma for arbitrary abelian categories]
\mbox{}
\vskip 0.05cm
\noindent
We will next state and prove a version of the Snake Lemma
valid for abelian categories with enough projectives (or injectives).
Under the assumption of having enough projectives (or injectives),
the Snake Lemma can be proved via a category-theoretic equivalent
of a diagram chase
(recall that the Snake Lemma for $R$-modules can be be proved via an element-centric diagram chase). 
However, the Snake Lemma in fact holds for arbitrary abelian categories
(not just those with enough projectives or injectives).
For the statement and a proof of this general version, see for example
Lemma 12.1.1, p.297, \cite{kashiwara2005categories}.
\end{remark}

          %%%%% ~~~~~~~~~~~~~~~~~~~~ %%%%%

\vskip 0.5cm
\begin{theorem}[Snake Lemma for abelian categories with enough projectives]
\mbox{}
\vskip 0.15cm
\noindent
Let \,$\mathfrak{A}$\, be an abelian category with enough projectives, and
the following be a commutative diagram with exact rows of morphisms in \,$\mathfrak{A}$:
\begin{center}
\begin{tikzcd}
A^{\prime}
	\arrow[dd, swap, "f^{\prime}"]
	\arrow[rr, "\mu_{A}"]
&&
A
	\arrow[dd, "f"]
	\arrow[rr, thick, two heads, "\varepsilon_{A}"]
&&
A^{\prime\prime}
	\arrow[dd, "f^{\prime\prime}"]
\\ \\
B^{\prime}
	\arrow[rr, thick, hook, swap, "\mu_{B}"]
&&
B
	\arrow[rr, swap, "\varepsilon_{B}"]
&&
B^{\prime\prime}
\end{tikzcd}
\end{center}
Then, the following statements are true:
\begin{enumerate}
\item
	Let
	\,$\kappa^{\prime}$,\, $\kappa$,\, $\kappa^{\prime\prime}$\, 
	denote respectively kernels of
	\,$f^{\prime}$,\, $f$,\, $f^{\prime\prime}$,\,
	and let
	\,$\pi^{\prime}$,\, $\pi$,\, $\pi^{\prime\prime}$\, 
	denote  respectively cokernels of 
	\,$f^{\prime}$,\, $f$,\, $f^{\prime\prime}$.
	Then, there exist morphisms
	\,$\mu_{K} : K^{\prime} \longrightarrow K$,\,
	\,$\varepsilon_{K} : K \longrightarrow K^{\prime\prime}$,\,
	\,$\mu_{Q} : Q^{\prime} \longrightarrow Q$,\,
	and
	\,$\varepsilon_{Q} : Q \longrightarrow Q^{\prime\prime}$\,
	such that the diagram above extends to the following commutative diagram:
	\begin{center}
	\begin{tikzcd}
	K^{\prime}
		\arrow[dd,swap, "\kappa^{\prime}"]
		\arrow[rr, thick, dashed, "\mu_{K}", red]
	&&
	K
		\arrow[dd, "\kappa"]
		\arrow[rr, thick, dashed, "\varepsilon_{K}", red]
	&&
	K^{\prime\prime}
		\arrow[dd, "\kappa^{\prime\prime}"]
	\\ \\
	A^{\prime}
		\arrow[dd, swap, "f^{\prime}"]
		\arrow[rr, "\mu_{A}"]
	&&
	A
		\arrow[dd, "f"]
		\arrow[rr, thick, two heads, "\varepsilon_{A}"]
	&&
	A^{\prime\prime}
		\arrow[dd, "f^{\prime\prime}"]
	\\ \\
	B^{\prime}
		\arrow[rr, thick, hook, swap, "\mu_{B}"]
		\arrow[dd, swap, "\pi^{\prime}"]
	&&
	B
		\arrow[rr, swap, "\varepsilon_{B}"]
		\arrow[dd, "\pi"]
	&&
	B^{\prime\prime}
		\arrow[dd, "\pi^{\prime\prime}"]
	\\ \\
	Q^{\prime}
		\arrow[rr, thick, dashed, swap, "\mu_{Q}", red]
	&&
	Q
		\arrow[rr, thick, dashed, swap, "\varepsilon_{Q}", red]
	&&
	Q^{\prime\prime}
	\end{tikzcd}
	\end{center}
\item
	There exists a morphism \,$\delta : K^{\prime\prime} \longrightarrow Q^{\prime}$\, such that
	\begin{center}
	\begin{tikzcd}
	K^{\prime}
		\arrow[rr, "\mu_{K}"]
	&&
	K
		\arrow[rr, "\varepsilon_{K}"]
	&&
	K^{\prime\prime}
		\arrow[rr, "\delta"]
	&&
	Q^{\prime}
		\arrow[rr, "\mu_{Q}"]
	&&
	Q
		\arrow[rr, "\varepsilon_{Q}"]
	&&
	Q^{\prime\prime}
	\end{tikzcd}
	\end{center}
	is a six-term exact sequence.
	\begin{center}
	\begin{tikzcd}
	K^{\prime}
		\arrow[dd,swap, "\kappa^{\prime}"]
		\arrow[rr, dashed, "\mu_{K}"]
	&&
	K
		\arrow[dd, "\kappa"]
		\arrow[rr, dashed, "\varepsilon_{K}"]
		\arrow[dddddd, phantom, ""{coordinate, name=X}]
	&&
	K^{\prime\prime}
		\arrow[dd, "\kappa^{\prime\prime}"]
		%\arrow[ddddddllll, thick, "\delta", red,
		%	crossing over, rounded corners,
		%	to path = { -- ([xshift=10ex]\tikztostart.east)
		%		|- (X) [near start]\tikztonodes
		%		-| ([xshift=-10ex]\tikztotarget.west)
		%		 -- (\tikztotarget)}
		%	 ]
	\\ \\
	A^{\prime}
		\arrow[dd, swap, near start, "f^{\prime}"]
		\arrow[rr, "\mu_{A}"]
	&&
	A
		\arrow[dd, near start, "f"]
		\arrow[rr, thick, two heads, "\varepsilon_{A}"]
	&&
	A^{\prime\prime}
		\arrow[dd, near start, "f^{\prime\prime}"]
	\\ \\
	B^{\prime}
		\arrow[rr, thick, hook, swap, "\mu_{B}"]
		\arrow[dd, swap, "\pi^{\prime}"]
	&&
	B
		%\arrow[from=uu, near start, "f"]
		\arrow[rr, swap, "\varepsilon_{B}"]
		\arrow[dd, "\pi"]
	&&
	B^{\prime\prime}
		\arrow[dd, "\pi^{\prime\prime}"]
	\\ \\
	Q^{\prime}
		\arrow[rr, dashed, swap, "\mu_{Q}"]
		\arrow[from=uuuuuurrrr, thick, "\textnormal{\LARGE$\delta$}", red,
			crossing over, rounded corners,
			to path = { -- ([xshift=10ex]\tikztostart.east)
				|- (X) [near start]\tikztonodes
				-| ([xshift=-10ex]\tikztotarget.west)
				 -- (\tikztotarget)}
			 ]
	&&
	Q
		\arrow[rr, dashed, swap, "\varepsilon_{Q}"]
	&&
	Q^{\prime\prime}
	\end{tikzcd}
	\end{center}
\end{enumerate}
\end{theorem}
\proof
\begin{enumerate}
\item
	\underline{Existence of $\mu_{K} \in \Mor_{\mathfrak{C}}(K^{\prime},K)$}
	\vskip -0.01cm
	First, note that
	\,$f \circ (\,\mu_{A} \circ \kappa^{\prime}\,)$
	\;$=$\; $(\,f \circ \mu_{A}\,)\circ \kappa^{\prime}$
	\;$=$\; $(\,\mu_{B}\circ f^{\prime}\,) \circ \kappa^{\prime}$
	\;$=$\; $\mu_{B} \circ (\,f^{\prime} \circ \kappa^{\prime}\,)$
	\;$=$\; $\mu_{B} \circ 0$
	\;$=$\; $0$.\,
	The universal property of
	\,$\kappa : K \longrightarrow A$\,
	as a kernel of
	\,$f : A \longrightarrow B$\,
	now implies the existence (and uniqueness) of $\mu_{K} : K^{\prime} \longrightarrow K$.
	\vskip 0.3cm

	\underline{Existence of $\varepsilon_{K} \in \Mor_{\mathfrak{C}}(K,K^{\prime\prime})$}
	\vskip -0.01cm
	This proof is completely analogous as the preceding one.
	More precisely, note that
	\,$f^{\prime\prime} \circ (\,\varepsilon_{A} \circ \kappa\,)$
	\;$=$\; $(\,f^{\prime\prime} \circ \varepsilon_{A}\,)\circ \kappa$
	\;$=$\; $(\,\varepsilon_{B} \circ f\,) \circ \kappa$
	\;$=$\; $\varepsilon_{B} \circ (\,f \circ \kappa\,)$
	\;$=$\; $\varepsilon_{B} \circ 0$
	\;$=$\; $0$.\,
	The universal property of
	\,$\kappa^{\prime\prime} : K^{\prime\prime} \longrightarrow A^{\prime\prime}$\,
	as a kernel of
	\,$f^{\prime\prime} : A^{\prime\prime} \longrightarrow B^{\prime\prime}$\,
	now implies the existence (and uniqueness) of
	\,$\varepsilon_{K} : K \longrightarrow K^{\prime\prime}$.
	\vskip 0.3cm

	\underline{Existence of $\mu_{Q} \in \Mor_{\mathfrak{C}}(Q^{\prime},Q)$}
	\vskip -0.01cm
	Note that
	\,$(\,\pi \circ \mu_{B}\,) \circ f^{\prime}$
	\;$=$\; $\pi \circ (\,\mu_{B} \circ f^{\prime}\,)$
	\;$=$\; $\pi \circ (\,f \circ \mu_{A}\,)$
	\;$=$\; $(\,\pi \circ f\,) \circ \mu_{A}$
	\;$=$\; $0 \,\circ\, \mu_{A}$
	\;$=$\; $0$.\,
	The universal property of
	\,$\pi^{\prime} : B^{\prime} \longrightarrow Q^{\prime}$\,
	as a cokernel of
	\,$f^{\prime} : A^{\prime} \longrightarrow B^{\prime}$\,
	now implies the existence (and uniqueness) of
	\,$\mu_{Q} : Q^{\prime} \longrightarrow Q$.
	\vskip 0.3cm

	\underline{Existence of $\varepsilon_{Q} \in \Mor_{\mathfrak{C}}(Q,Q^{\prime\prime})$}
	\vskip -0.01cm
	Note that
	\,$(\,\pi^{\prime\prime} \circ \varepsilon_{B}\,) \circ f$
	\;$=$\; $\pi^{\prime\prime} \circ (\,\varepsilon_{B} \circ f\,)$
	\;$=$\; $\pi^{\prime\prime} \circ (\,f^{\prime\prime} \circ \varepsilon_{A}\,)$
	\;$=$\; $(\,\pi^{\prime\prime} \circ f^{\prime\prime}\,) \circ \varepsilon_{A}$
	\;$=$\; $0 \,\circ\, \varepsilon_{A}$
	\;$=$\; $0$.\,
	The universal property of
	\,$\pi : B \longrightarrow Q$\,
	as a cokernel of
	\,$f : A \longrightarrow B$\,
	now implies the existence (and uniqueness) of
	\,$\varepsilon_{Q} : Q \longrightarrow Q^{\prime\prime}$.
	\vskip 0.5cm

\item
	First, we construct the morphism \,$\delta : K^{\prime\prime} \longrightarrow Q^{\prime}$,\, as follows:
	\begin{itemize}
	\item
		Since \,$\mathfrak{A}$\, is an (abelian) category with enough projectives,
		there exists an epimorphism
		\,$\varepsilon_{P} : P \longrightarrow K^{\prime\prime}$,\,
		with \,$P \in \Obj(\mathfrak{A})$\, being a projective object.
	\item
		Since \,$P$\, is a projective object and 
		\,$\varepsilon_{A} : A \longrightarrow A^{\prime\prime}$\,
		is an epimorphism, there exists
		\,$f_{1} : P \longrightarrow A$\,
		such that
		\,$\kappa^{\prime\prime} \circ \varepsilon_{P} \,=\, \varepsilon_{A} \circ f_{1}$\,
	\item
		Since \,$\mu_{B}$\, is a monomorphism
		and the composition
		\,$B^{\prime} \overset{\mu_{B}}{\longrightarrow} B \overset{\varepsilon_{B}}{\longrightarrow} B^{\prime\prime}$\,
		is exact, it follows
		by Proposition \ref{MonomorphicFirstFactorInExactSequenceIsKernelOfSecondFactor}
		that
		\,$\mu_{B}$\, is a kernel of \,$\varepsilon_{B}$.\,
		On the other hand, observe that
		\,$\varepsilon_{B} \circ (\,f \circ f_{1}\,)$
		\,$=$\, $(\,\varepsilon_{B} \circ f\,) \circ f_{1}$
		\,$=$\, $(\,f^{\prime\prime} \circ \varepsilon_{A}\,) \circ f_{1}$
		\,$=$\, $f^{\prime\prime} \circ (\,\varepsilon_{A} \circ f_{1}\,)$
		\,$=$\, $f^{\prime\prime} \circ (\,\kappa^{\prime\prime} \circ \varepsilon_{P}\,)$
		\,$=$\, $(\,f^{\prime\prime} \circ \kappa^{\prime\prime}\,) \circ \varepsilon_{P}$
		\,$=$\, $0 \circ \varepsilon_{P}$
		\,$=$\, $0$.\,
		Hence, by the universal property of
		\,$\mu_{B}$\, as a kernel of \,$\varepsilon_{B}$,\,
		we see that there exists a unique
		\,$f_{2} : P \longrightarrow B^{\prime}$\,
		such that
		\,$f \circ f_{1} = \mu_{B} \circ f_{2}$.\,
	\item
		Next, we simply define
		\,$f_{3} \,:=\, \pi^{\prime} \circ f_{2}$.\,
	\end{itemize}
%	\begin{center}
%	\begin{tikzcd}
%	K^{\prime}
%		\arrow[dd,swap, "\kappa^{\prime}"]
%		\arrow[rr, thick, dashed, "\mu_{K}", red]
%	&&
%	K
%		\arrow[dd, "\kappa"]
%		\arrow[rr, thick, dashed, "\varepsilon_{K}", red]
%	&&
%	K^{\prime\prime}
%		\arrow[dd, "\kappa^{\prime\prime}"]
%	\\ \\
%	A^{\prime}
%		\arrow[dd, swap, "f^{\prime}"]
%		\arrow[rr, "\mu_{A}"]
%	&&
%	A
%		\arrow[dd, "f"]
%		\arrow[rr, thick, two heads, "\varepsilon_{A}"]
%	&&
%	A^{\prime\prime}
%		\arrow[dd, "f^{\prime\prime}"]
%	\\ \\
%	B^{\prime}
%		\arrow[rr, thick, hook, swap, "\mu_{B}"]
%		\arrow[dd, swap, "\pi^{\prime}"]
%	&&
%	B
%		\arrow[rr, swap, "\varepsilon_{B}"]
%		\arrow[dd, "\pi"]
%	&&
%	B^{\prime\prime}
%		\arrow[dd, "\pi^{\prime\prime}"]
%	\\ \\
%	Q^{\prime}
%		\arrow[rr, thick, dashed, swap, "\mu_{Q}", red]
%	&&
%	Q
%		\arrow[rr, thick, dashed, swap, "\varepsilon_{Q}", red]
%	&&
%	Q^{\prime\prime}
%	\end{tikzcd}
%	\end{center}
%%%%%%%%%%%%%%%%%%%%%%%%%%%%%%%%%%%%%
%%%%%%%%%%%%%%%%%%%%%%%%%%%%%%%%%%%%%
%%%%%%%%%%%%%%%%%%%%%%%%%%%%%%%%%%%%%
%	\begin{center}
%	\vskip -2.5cm
%	\begin{tikzcd}
%	&&
%	&&&
%	{K_{P}}
%		\arrow[d, "\kappa_{P}"]
%	&
%	\\
%	&&
%	&
%	{\color{white}K_{P}}
%	&
%	&
%	{\color{blue}P}
%		\arrow[dr, two heads, "\varepsilon_{P}", blue]
%		\arrow[dddl, "f_{1}", red]
%		\arrow[dddddlll, swap, "f_{2}", red]
%		\arrow[dddddddlll, bend right = 110, "f_{3}", red]
%	&
%	\\
%	&&
%	K^{\prime}
%		\arrow[dd, swap, "\kappa^{\prime}"]
%		\arrow[rr, thick, dashed, "\mu_{K}"]
%	&&
%	K
%		\arrow[dd, "\kappa"]
%		\arrow[rr, thick, dashed, "\varepsilon_{K}"]
%	&&
%	K^{\prime\prime}
%		\arrow[dd, "\kappa^{\prime\prime}"]
%		\arrow[ddddddllll, thick, dashed, bend left = 29, "\textnormal{\Large$\delta$}", red]
%	\\ \\
%	&&
%	A^{\prime}
%		\arrow[dd, swap, "f^{\prime}"]
%		\arrow[rr, "\mu_{A}"]
%	&&
%	A
%		\arrow[dd, "f"]
%		\arrow[rr, thick, two heads, "\varepsilon_{A}"]
%	&&
%	A^{\prime\prime}
%		\arrow[dd, "f^{\prime\prime}"]
%	\\ \\
%	&&
%	B^{\prime}
%		\arrow[rr, thick, hook, swap, "\mu_{B}"]
%		\arrow[dd, swap, "\pi^{\prime}"]
%	&&
%	B
%		\arrow[rr, swap, "\varepsilon_{B}"]
%		\arrow[dd, "\pi"]
%	&&
%	B^{\prime\prime}
%		\arrow[dd, "\pi^{\prime\prime}"]
%	\\ \\
%	&&
%	Q^{\prime}
%		\arrow[rr, thick, dashed, swap, "\mu_{Q}"]
%	&&
%	Q
%		\arrow[rr, thick, dashed, swap, "\varepsilon_{Q}"]
%	&&
%	Q^{\prime\prime}
%	\end{tikzcd}
%	{\color{white}?????????????????????}
%	\end{center}
	Consider the following diagram:
	\begin{center}
	\vskip -1.5cm
	\begin{tikzcd}
	&&
	&&&
	K_{P}
		\arrow[d, "\kappa_{P}"]
		\arrow[dddddll, dashed, bend right = 30, swap, pos=0.30, "f_{1}^{\prime}", red]		
	&
	\\
	&&
	&
	{\color{white}K_{P}}
	&
	&
	P
		\arrow[dr, two heads, "\varepsilon_{P}"]
		\arrow[dddl, pos=0.60, "f_{1}", blue]
		\arrow[dddddlll, bend right = 35, swap, pos=0.55, "f_{2}", blue]
		\arrow[dddddddlll, bend right = 110, swap, pos=0.75, "f_{3}\,", blue]
	&
	\\
	&&
	K^{\prime}
		\arrow[dd, swap, "\kappa^{\prime}"]
		\arrow[rr, thick, dashed, "\mu_{K}"]
	&&
	K
		\arrow[dd, "\kappa"]
		\arrow[rr, thick, dashed, "\varepsilon_{K}"]
	&&
	K^{\prime\prime}
		\arrow[dd, "\kappa^{\prime\prime}"]
		\arrow[ddddddllll, thick, dashed, bend left = 29, pos=0.35, "\textnormal{\Large$\delta$}", blue]
	\\ \\
	{\color{red}K_{0}}
		\arrow[rr, "\kappa(\,\widetilde{\mu_{A}}\,)", red]
	&&
	A^{\prime}
		\arrow[dd, swap, "f^{\prime}"]
		\arrow[rr, "\mu_{A}"]
		\arrow[dr, two heads, pos=0.6, "\widetilde{\mu_{A}}", red]
	&&
	A
		\arrow[dd, "f"]
		\arrow[rr, thick, two heads, "\varepsilon_{A}"]
	&&
	A^{\prime\prime}
		\arrow[dd, "f^{\prime\prime}"]
	\\
	&&
	&
	{\color{red}I_{\mu_{A}}}
		\arrow[ur, hook, swap, pos=0.35, "\!\!\iota(\mu_{A}){\color{white}}", red]		
		\arrow[dl, dashed, pos=0.35, "f_{2}^{\prime}", red]		
	&&&
	\\
	&&
	B^{\prime}
		\arrow[rr, thick, hook, swap, "\mu_{B}"]
		\arrow[dd, swap, "\pi^{\prime}"]
	&&
	B
		\arrow[rr, swap, "\varepsilon_{B}"]
		\arrow[dd, "\pi"]
	&&
	B^{\prime\prime}
		\arrow[dd, "\pi^{\prime\prime}"]
	\\ \\
	&&
	Q^{\prime}
		\arrow[rr, thick, dashed, swap, "\mu_{Q}"]
	&&
	Q
		\arrow[rr, thick, dashed, swap, "\varepsilon_{Q}"]
	&&
	Q^{\prime\prime}
	\end{tikzcd}
	{\color{white}?????????????????????}
	\end{center}
	Here,
	\,$K_{P} \overset{\kappa_{P}}{\longrightarrow}$ P\,
	is a kernel of
	\,$P \overset{\varepsilon_{P}}{\longrightarrow} K^{\prime\prime}$.\,
	Also, by definition, the monomorphism
	\,$\iota(\mu_{A}) : I_{\mu_{A}} \longrightarrow A$\,
	is an image of
	\,$\mu_{A}$.\,
	By Proposition \ref{FactorizationIntoImageCoimage}(ii),
	we may take \,$\iota(\mu_{A})$\, to be any kernel of any cokernel of \,$\mu_{A}$.\,
	By Lemma \ref{fTildeIsEpimorphism}, the accompanying morphism
	\,$\widetilde{\mu_{A}} : A^{\prime} \longrightarrow I_{\mu_{A}}$\,
	is an epimorphism.


	\vskip 0.3cm
	\textbf{Claim 1:}\;
	\,$\varepsilon_{A} \,\circ\, \iota(\mu_{A}) \,=\, 0$.\,
	\vskip 0.01cm
	Proof of Claim 1:\; Simply note that
	\,$0 \,=\, \varepsilon_{A} \,\circ\, \mu_{A} \,=\,\varepsilon_{A} \,\circ\, \iota(\mu_{A}) \,\circ\, \widetilde{\mu_{A}}$,\,
	which implies
	\,$\varepsilon_{A} \,\circ\, \iota(\mu_{A}) = 0$,\,
	since \,$\widetilde{\mu_{A}}$\, is an epimorphism and can be cancelled on the right.
	This completes the proof of Claim 1.	


	\vskip 0.3cm
	\textbf{Claim 2:}\;
	The composition
	\,$I_{\mu_{A}} \overset{\iota(\mu_{A})}{\longrightarrow} A \overset{\varepsilon_{A}}{\longrightarrow} A^{\prime\prime}$\,
	is exact.
	%\;\,$f_{3} \circ \kappa_{P} \,=\, 0$.\,
	\vskip 0.01cm
	Proof of Claim 2:\;
	This will follow from the exactness of
	\,$A^{\prime} \overset{\mu_{A}}{\longrightarrow} A \overset{\varepsilon_{A}}{\longrightarrow} A^{\prime\prime}$.\,
	Recall from Definition \ref{defnExactness} that, in general, the exactness of the composition
	\,$A \overset{f}{\longrightarrow} B \overset{g}{\longrightarrow} C$\,
	means that the unique morphism \,$\theta$\, in the diagram below is an isomorphism:
	\begin{center}
	\begin{tikzcd}
	&& && Q_{f}
		\arrow[dd, dashed, "\psi"]
	\\ \\
	A
		\arrow[rr, "f"]
		\arrow[dd, dashed, two heads, swap, "\varepsilon_{f}"]
		\arrow[rrrr, bend left = 30, "0", gray]
	&&
	B
		\arrow[rr, "g"]
		\arrow[uurr, two heads, "\pi_{f}"]
	&&
	C
	\\ \\
	I_{f}
		\arrow[uurr, hook, "\kappa(\pi_{f})"]
		\arrow[rr, thick, dashed, swap, "\exists !\,\theta", red]
	&&
	K_{g}
		\arrow[uu, hook, swap, "\kappa_{g}"]
		%\arrow[rr, thick, two heads, swap, "\pi_{\theta}", red]
	&&
	%Q_{\theta}
	\end{tikzcd}
	\end{center}
	Since \,$\mathfrak{A}$\, is an abelian category,
	the morphism \,$\kappa(\pi_{f})$\, -- kernel of cokernel of \,$f$ --
	is an image of \,$f$,\,
	while the morphism \,$\varepsilon_{f}$\, is a coimage of \,$f$.\,
	In other words, since \,$\mathfrak{A}$\, is an abelian category,
	the defining diagram of exactness of
	\,$A \overset{f}{\longrightarrow} B \overset{g}{\longrightarrow} C$\,
	is equivalent to:
	\begin{center}
	\begin{tikzcd}
	&& && Q_{f}
		\arrow[dd, dashed, "\psi"]
	\\ \\
	A
		\arrow[rr, "f"]
		\arrow[dd, dashed, two heads, swap, "\widetilde{f}\;"]
		\arrow[rrrr, bend left = 30, "0", gray]
	&&
	B
		\arrow[rr, "g"]
		\arrow[uurr, two heads, "\pi_{f}"]
	&&
	C
	\\ \\
	I_{f}
		\arrow[uurr, hook, "\iota(f)"]
		\arrow[rr, thick, dashed, swap, "\exists !\,\theta", red]
	&&
	K_{g}
		\arrow[uu, hook, swap, "\kappa_{g}"]
		%\arrow[rr, thick, two heads, swap, "\pi_{\theta}", red]
	&&
	%Q_{\theta}
	\end{tikzcd}
	\end{center}
	Applying this alternative but equivalent exactness-defining diagram to 
	\,$A^{\prime} \overset{\mu_{A}}{\longrightarrow} A \overset{\varepsilon_{A}}{\longrightarrow} A^{\prime\prime}$\,
	gives:	
	\begin{center}
	\begin{tikzcd}
	&& && Q_{\mu_{A}}
		\arrow[dd, dashed, "\psi"]
	\\ \\
	A^{\prime}
		\arrow[rr, "\mu_{A}"]
		\arrow[dd, dashed, two heads, swap, "\widetilde{\mu_{A}}{\color{white}..}", blue]
		\arrow[rrrr, bend left = 30, "0", gray]
	&&
	A
		\arrow[rr, "\varepsilon_{A}"]
		\arrow[uurr, two heads, "\pi(\mu_{A})\!\!\!"]
	&&
	A^{\prime\prime}
	\\ \\
	I_{\mu_{A}}
		\arrow[uurr, hook, "\iota(\mu_{A})\!\!"]
		\arrow[rr, thick, dashed, swap, "\exists !\,\theta_{0}", red]
	&&
	K_{\varepsilon_{A}}
		\arrow[uu, hook, swap, "\;\kappa(\varepsilon_{A})"]
		%\arrow[rr, thick, two heads, swap, "\pi_{\theta}", red]
	&&
	%Q_{\theta}
	\end{tikzcd}
	\end{center}
	The exactness of
	\,$A^{\prime} \overset{\mu_{A}}{\longrightarrow} A \overset{\varepsilon_{A}}{\longrightarrow} A^{\prime\prime}$\,
	means that the unique morphism \,$\theta_{0}$\, in the above diagram is an isomorphism.
	Next, we consider the composition
	\,$I_{\mu_{A}} \overset{\iota(\mu_{A})}{\longrightarrow} A \overset{\varepsilon_{A}}{\longrightarrow} A^{\prime\prime}$.\,
	Noting that \,$\iota(\mu_{A})$\, is a monomorphism, and hence it is an image of itself,
	and the identity morphism \,$\textnormal{id}_{I_{\mu_{A}}}$\, is an (accompanying) coimage of \,$\iota(\mu_{A})$,\,
	we see that the exactness of
	\,$I_{\mu_{A}} \overset{\iota(\mu_{A})}{\longrightarrow} A \overset{\varepsilon_{A}}{\longrightarrow} A^{\prime\prime}$\,
	is precisely the property that the unique morphism \,$\theta_{1}$\, in the following diagram is an isomorphism:
	\begin{center}
	\begin{tikzcd}
	&& && Q_{\mu_{A}}
		\arrow[dd, dashed, "\psi"]
	\\ \\
	I_{\mu_{A}} %A^{\prime}
		\arrow[rr, hook, "\iota(\mu_{A})"]
		%\arrow[dd, dashed, two heads, swap, "\widetilde{\mu_{A}}{\color{white}..}"]
		%\arrow[dd, dashed, two heads, swap, "\varepsilon_{\mu_{A}}{\color{white}..}", blue]
		\arrow[dd, equal, swap, "\textnormal{id}\;"]
		\arrow[rrrr, bend left = 30, "0", gray]
	&&
	A
		\arrow[rr, "\varepsilon_{A}"]
		\arrow[uurr, two heads, "\pi(\mu_{A})\!\!\!"]
	&&
	A^{\prime\prime}
	\\ \\
	I_{\mu_{A}}
		\arrow[uurr, hook, "\iota(\mu_{A})\!\!"]
		\arrow[rr, thick, dashed, swap, "\exists !\,\theta_{1}", red]
	&&
	K_{\mu_{A}}
		\arrow[uu, hook, swap, "\;\kappa(\varepsilon_{A})"]
		%\arrow[rr, thick, two heads, swap, "\pi_{\theta}", red]
	&&
	%Q_{\theta}
	\end{tikzcd}
	\end{center}
	Lastly, the fact that \,$\theta_{1}$\, must be an isomorphism follows from the uniqueness of
	\,$\theta_{0}$\, and \,$\theta_{1}$\, in their respective diagrams.
	Indeed, comparing the two immediately preceding diagrams, and noting the uniqueness of
	\,$\theta_{0}$\, and \,$\theta_{1}$\, in their respective diagrams,
	we see that we in fact must have \,$\theta_{0} = \theta_{1}$;\,
	in particular, we therefore have that \,$\theta_{1} \,=\, \theta_{0}$\, is an isomorphism
	(since \,$\theta_{0}$\, is an isomorphism, by the exactness hypothesis on
	\,$A^{\prime} \overset{\mu_{A}}{\longrightarrow} A \overset{\varepsilon_{A}}{\longrightarrow} A^{\prime\prime}$).
	Thus, the exactness of
	\,$A^{\prime} \overset{\mu_{A}}{\longrightarrow} A \overset{\varepsilon_{A}}{\longrightarrow} A^{\prime\prime}$\,
	indeed implies the exactness of
	\,$I_{\mu_{A}} \overset{\iota(\mu_{A})}{\longrightarrow} A \overset{\varepsilon_{A}}{\longrightarrow} A^{\prime\prime}$.\,
	This completes the proof of Claim 2.	


	\vskip 0.3cm
	\textbf{Claim 3:}\;
	The image monomorphism \,$\iota(\mu_{A})$\, of \,$\mu_{A}$\, is a kernel of \,$\varepsilon_{A}$.\,
	\vskip 0.01cm
	Proof of Claim 3:\;
	Immediate by Claim 1, Claim 2 and
	Proposition \ref{MonomorphicFirstFactorInExactSequenceIsKernelOfSecondFactor}.
	This completes the proof of Claim 3.

	\vskip 0.3cm
	\textbf{Claim 4:}\;
	There exists a morphism
	\,$f_{1}^{\prime} : K_{P} \longrightarrow I_{\mu_{A}}$\,
	such that
	\,$(\,f_{1} \,\circ\, \kappa_{P}\,) \,=\, \iota(\mu_{A}) \,\circ\, f_{1}^{\prime}$.\,
	\vskip 0.01cm
	Proof of Claim 4:\;
	By Claim 3, \,$\iota(\mu_{A})$\, is a kernel of \,$\varepsilon_{A}$.\,
	In order to proof Claim 4, by the universal property of
	\,$\iota(\mu_{A})$\, as kernel of \,$\varepsilon_{A}$,\,
	it suffices to establish that
	\,$\varepsilon_{A} \,\circ\, (\,f_{1} \,\circ\, \kappa_{P}\,) \,=\, 0$.\,
	To this end, simply note that
	\,$\varepsilon_{A} \,\circ\, (\,f_{1} \,\circ\, \kappa_{P}\,)$
	\,$=$\, (\,$\varepsilon_{A} \,\circ\, f_{1}\,) \,\circ\, \kappa_{P}$
	\,$=$\, (\,$\kappa^{\prime\prime} \,\circ\, \varepsilon_{P}\,) \,\circ\, \kappa_{P}$
	\,$=$\, $\kappa^{\prime\prime} \,\circ\, (\,\varepsilon_{P} \,\circ\, \kappa_{P}\,)$
	\,$=$\, $0$.\,
	This completes the proof of Claim 4.


	\vskip 0.3cm
	\textbf{Claim 5:}\;
	There exists a morphism
	\,$f_{2}^{\prime} : I_{\mu_{A}} \longrightarrow B^{\prime}$\,
	such that
	\,$f^{\prime} \,=\, f_{2}^{\prime} \,\circ\, \widetilde{\mu_{A}}$.\,
	\vskip 0.01cm
	Proof of Claim 5:\;
	%Next, we establish the existence of a morphism
	%\,$f_{2}^{\prime} : I_{\mu_{A}} \longrightarrow B^{\prime}$\,
	%which makes the above diagram commute.
	Recall that, since \,$\mathfrak{A}$\, is an abelian category,
	the epimorphism \,$\widetilde{\mu_{A}}$\, is a cokernel (of some morphism),
	and thus, by Lemma \ref{AKernelsTheKernelOfItsOwnCokernel}(ii),
	\,$\widetilde{\mu_{A}}$\, is a cokernel of its own kernel
	\,$\kappa(\widetilde{\mu_{A}})$.

	Next, note that
	\,$\mu_{B} \,\circ\, f^{\prime} \,\circ\, \kappa(\widetilde{\mu_{A}})$
	\,$=$\, $f \,\circ\, \iota(\mu_{A}) \,\circ\, \widetilde{\mu_{A}} \,\circ\, \kappa(\widetilde{\mu_{A}})$
	\,$=$\, $f \,\circ\, \iota(\mu_{A}) \,\circ\, 0$
	\,$=$\, $0$.\,
	Since \,$\mu_{B}$\, is a monomorphism, it can be cancelled on the left,
	which yields
	\,$ f^{\prime} \circ \kappa(\widetilde{\mu_{A}}) \,=\, 0$.\,
	Since \,$\widetilde{\mu_{A}}$\, is a cokernel of \,$\kappa(\widetilde{\mu_{A}})$\, (established in preceding paragraph),
	we see that there exists a (unique) morphism
	\,$f_{2}^{\prime} : I_{\mu_{A}} \longrightarrow B^{\prime}$\,
	such that
	\,$f^{\prime} \,=\, f_{2}^{\prime} \,\circ\, \widetilde{\mu_{A}}$.\,
	This completes the proof of Claim 5.
	

	\vskip 0.3cm
	\textbf{Claim 6:} \;$f_{3} \,\circ\, \kappa_{P} \,=\, 0$.
		\vskip 0.01cm
	Proof of Claim 6:\;
	Note that we now have
	\,$f_{3} \,\circ\, \kappa_{P}$
	\,$=$\, $\pi^{\prime} \,\circ\, f_{2}^{\prime} \,\circ\, f_{1}^{\prime}$.\,
	On the other hand,
	\,$\pi^{\prime} \,\circ\, f_{2}^{\prime} \,\circ\, \widetilde{\mu_{A}} \,=\, \pi^{\prime} \,\circ\, f^{\prime} \,=\, 0$.\,
	Since \,$\widetilde{\mu_{A}}$\, is an epimorphism and can be cancelled on the right,
	we have
	\,$\pi^{\prime} \,\circ\, f_{2}^{\prime} \,=\, 0$.\,
	This therefore implies
	\,$f_{3} \,\circ\, \kappa_{P}$
	\,$=$\, $\pi^{\prime} \,\circ\, f_{2}^{\prime} \,\circ\, f_{1}^{\prime}$
	\,$=$\, $0 \,\circ\, f_{1}^{\prime}$
	\,$=$\, $0$.\,
	This completes the proof of Claim 6.


	\vskip 0.3cm
	\textbf{Claim 7:} \;$f_{3}$\, factors through \,$\varepsilon_{P}$;\,
	more precisely, there exists
	\,$\delta : K^{\prime\prime} \longrightarrow Q^{\prime}$\,
	such that
	\,$f_{3} \,=\, \delta \,\circ\, \varepsilon_{P}$.
	\vskip 0.01cm
	Proof of Claim 7:\;
	Since \,$\mathfrak{A}$\, is an abelian category, it is in particular conormal, i.e.,
	every epimorphism in \,$\mathfrak{A}$\, is a cokernel.
	By Lemma \ref{AKernelsTheKernelOfItsOwnCokernel}(ii),
	the epimorphism \,$\varepsilon_{P}$\, is thus a cokernel of of its own kernel \,$\kappa_{P}$.\,
	By Claim 6, we have \,$f_{3} \,\circ\, \kappa_{P} \,=\, 0$.\,
	Thus, the existence and uniqueness of the morphism
	\,$\delta : K^{\prime\prime} \longrightarrow Q^{\prime}$\,
	now follows by the universal property of
	\,$\varepsilon_{P}$\, as a cokernel of \,$\kappa_{P}$.\,
	This completes the proof of Claim 7.


	\vskip 0.3cm
	This construction of the (snake or connection) morphism \,$\delta$\, is now complete.
\end{enumerate}
\qed

          %%%%% ~~~~~~~~~~~~~~~~~~~~ %%%%%
