
          %%%%% ~~~~~~~~~~~~~~~~~~~~ %%%%%

\section{The Snake Lemma in abelian categories with enough projectives}
\setcounter{theorem}{0}
\setcounter{equation}{0}

%\cite{vanDerVaart1996}
%\cite{Kosorok2008}

%\renewcommand{\theenumi}{\alph{enumi}}
%\renewcommand{\labelenumi}{\textnormal{(\theenumi)}$\;\;$}
\renewcommand{\theenumi}{\roman{enumi}}
\renewcommand{\labelenumi}{\textnormal{(\theenumi)}$\;\;$}

          %%%%% ~~~~~~~~~~~~~~~~~~~~ %%%%%

\begin{theorem}[Snake Lemma]
\mbox{}
\vskip 0.15cm
\noindent
Let \,$\mathfrak{A}$ be an abelian category, and
the following be a commutative diagram with exact rows of morphisms in \,$\mathfrak{A}$:
\begin{center}
\begin{tikzcd}
A^{\prime}
	\arrow[dd, swap, "f^{\prime}"]
	\arrow[rr, "\mu_{A}"]
&&
A
	\arrow[dd, "f"]
	\arrow[rr, thick, two heads, "\varepsilon_{A}"]
&&
A^{\prime\prime}
	\arrow[dd, "f^{\prime\prime}"]
\\ \\
B^{\prime}
	\arrow[rr, thick, hook, swap, "\mu_{B}"]
&&
B
	\arrow[rr, swap, "\varepsilon_{B}"]
&&
B^{\prime\prime}
\end{tikzcd}
\end{center}
Then, the following statements are true:
\begin{enumerate}
\item
	There exist morphisms
	\,$\mu_{K} : K^{\prime} \longrightarrow K$,\,
	\,$\varepsilon_{K} : K \longrightarrow K^{\prime\prime}$,\,
	\,$\mu_{Q} : Q^{\prime} \longrightarrow Q$,\,
	and
	\,$\varepsilon_{Q} : K \longrightarrow Q^{\prime\prime}$\,
	such that the diagram above extends to the following commutative diagram:
	\begin{center}
	\begin{tikzcd}
	K^{\prime}
		\arrow[dd,swap, "\kappa^{\prime}"]
		\arrow[rr, dashed, "\mu_{K}"]
	&&
	K
		\arrow[dd, "\kappa"]
		\arrow[rr, dashed, "\varepsilon_{K}"]
	&&
	K^{\prime\prime}
		\arrow[dd, "\kappa^{\prime\prime}"]
	\\ \\
	A^{\prime}
		\arrow[dd, swap, "f^{\prime}"]
		\arrow[rr, "\mu_{A}"]
	&&
	A
		\arrow[dd, "f"]
		\arrow[rr, thick, two heads, "\varepsilon_{A}"]
	&&
	A^{\prime\prime}
		\arrow[dd, "f^{\prime\prime}"]
	\\ \\
	B^{\prime}
		\arrow[rr, thick, hook, swap, "\mu_{B}"]
		\arrow[dd, swap, "\pi^{\prime}"]
	&&
	B
		\arrow[rr, swap, "\varepsilon_{B}"]
		\arrow[dd, "\pi"]
	&&
	B^{\prime\prime}
		\arrow[dd, "\pi^{\prime\prime}"]
	\\ \\
	Q^{\prime}
		\arrow[rr, dashed, swap, "\mu_{Q}"]
	&&
	Q
		\arrow[rr, dashed, swap, "\varepsilon_{Q}"]
	&&
	Q^{\prime\prime}
	\end{tikzcd}
	\end{center}
	where
	\,$\kappa^{\prime}$,\, $\kappa$,\, $\kappa^{\prime\prime}$\, 
	are respectively kernels of
	\,$f^{\prime}$,\, $f$,\, $f^{\prime\prime}$,\,
	and
	\,$\pi^{\prime}$,\, $\pi$,\, $\pi^{\prime\prime}$\, 
	are respectively cokernels of 
	\,$f^{\prime}$,\, $f$,\, $f^{\prime\prime}$.
\item
	There exists a morphism \,$\delta : K^{\prime\prime} \longrightarrow Q^{\prime}$\,
	such that the following diagram commmutes:
	\begin{center}
	\begin{tikzcd}
	K^{\prime}
		\arrow[dd,swap, "\kappa^{\prime}"]
		\arrow[rr, dashed, "\mu_{K}"]
	&&
	K
		\arrow[dd, "\kappa"]
		\arrow[rr, dashed, "\varepsilon_{K}"]
		\arrow[dddddd, phantom, ""{coordinate, name=X}]
	&&
	K^{\prime\prime}
		\arrow[dd, "\kappa^{\prime\prime}"]
		%\arrow[ddddddllll, thick, "\delta", red,
		%	crossing over, rounded corners,
		%	to path = { -- ([xshift=10ex]\tikztostart.east)
		%		|- (X) [near start]\tikztonodes
		%		-| ([xshift=-10ex]\tikztotarget.west)
		%		 -- (\tikztotarget)}
		%	 ]
	\\ \\
	A^{\prime}
		\arrow[dd, swap, near start, "f^{\prime}"]
		\arrow[rr, "\mu_{A}"]
	&&
	A
		\arrow[dd, near start, "f"]
		\arrow[rr, thick, two heads, "\varepsilon_{A}"]
	&&
	A^{\prime\prime}
		\arrow[dd, near start, "f^{\prime\prime}"]
	\\ \\
	B^{\prime}
		\arrow[rr, thick, hook, swap, "\mu_{B}"]
		\arrow[dd, swap, "\pi^{\prime}"]
	&&
	B
		%\arrow[from=uu, near start, "f"]
		\arrow[rr, swap, "\varepsilon_{B}"]
		\arrow[dd, "\pi"]
	&&
	B^{\prime\prime}
		\arrow[dd, "\pi^{\prime\prime}"]
	\\ \\
	Q^{\prime}
		\arrow[rr, dashed, swap, "\mu_{Q}"]
		\arrow[from=uuuuuurrrr, thick, "\delta", red,
			crossing over, rounded corners,
			to path = { -- ([xshift=10ex]\tikztostart.east)
				|- (X) [near start]\tikztonodes
				-| ([xshift=-10ex]\tikztotarget.west)
				 -- (\tikztotarget)}
			 ]
	&&
	Q
		\arrow[rr, dashed, swap, "\varepsilon_{Q}"]
	&&
	Q^{\prime\prime}
	\end{tikzcd}
	\end{center}
	and such that
	\begin{center}
	\begin{tikzcd}
	K^{\prime}
		\arrow[rr, "\mu_{K}"]
	&&
	K
		\arrow[rr, "\varepsilon_{K}"]
	&&
	K^{\prime\prime}
		\arrow[rr, "\delta"]
	&&
	Q^{\prime}
		\arrow[rr, "\mu_{Q}"]
	&&
	Q
		\arrow[rr, "\varepsilon_{Q}"]
	&&
	Q^{\prime\prime}
	\end{tikzcd}
	\end{center}
	is a six-term exact sequence.
\end{enumerate}

\begin{itemize}
\item
	A \textbf{product} is an ordered pair
	\,$\left(\,C\,,\,\{p_{i} : C \longrightarrow A_{i}\}_{i \in I}\,\right)$\,
	consisting of an object $C \in \Obj(\mathfrak{C})$ and a family 
	$\{p_{i} : C \longrightarrow A_{i}\}_{i \in I}$ of morphisms in $\mathfrak{C}$
	such that,
	for every object $X \in \Obj(\mathfrak{C})$ and morphisms $f_{i} \in \Mor_{\mathfrak{C}}(X,A_{i})$,
	there exists a unique morphism $\theta \in \Mor_{\mathfrak{C}}(X,C)$ such that,
	for each $i \in I$, the following diagram commutes:
	\begin{center}
	\begin{tikzcd}
	& {\color{red}C} \arrow[d, "p_{i}", red] \\
	X \arrow[ru, dashed,"\exists\,!\;\theta"] \arrow[r, swap, "f_{i}"] & {\color{red}A_{i}}
	\end{tikzcd}
	\end{center}
	If the product exists, it is denoted by: $\underset{i \in I}{\bigsqcap}\,A_{i}$.
	It is unique up to isomorphism.
\item
	A \textbf{coproduct} is an ordered pair
	\,$\left(\,C\,,\,\{\alpha_{i} : A_{i} \longrightarrow C\}_{i \in I}\,\right)$\,
	consisting of an object $C \in \Obj(\mathfrak{C})$ and a family 
	$\{\alpha_{i} : A_{i} \longrightarrow C\}_{i \in I}$ of morphisms in $\mathfrak{C}$
	such that,
	for every object $X \in \Obj(\mathfrak{C})$ and morphisms $g_{i} \in \Mor_{\mathfrak{C}}(A_{i},X)$,
	there exists a unique morphism $\theta \in \Mor_{\mathfrak{C}}(C,X)$ such that,
	for each $i \in I$, the following diagram commutes:
	\begin{center}
	\begin{tikzcd}
	& {\color{red}C} \arrow[ld, dashed, swap, "\exists\,!\;\theta"] \\
	X & {\color{red}A_{i}} \arrow[l, "g_{i}"] \arrow[u, swap, "\alpha_{i}", red]
	\end{tikzcd}
	\end{center}
	If the coproduct exists, it is denoted by: $\underset{i \in I}{\bigsqcup}\,A_{i}$.
	It is unique up to isomorphism.
\end{itemize}
\end{theorem}

          %%%%% ~~~~~~~~~~~~~~~~~~~~ %%%%%

\vskip 0.5cm
\begin{definition}[Additive category]
\mbox{}
\vskip 0.15cm
\noindent
A category \,$\mathfrak{C}$\, is said to be \textbf{additive} if
\begin{itemize}
\item
	the category $\mathfrak{C}$ has a zero object,
\item
	$\Mor_{\mathfrak{C}}(A,B)$ is an abelian group, for each \,$A, B \in \Obj(\mathfrak{C})$,
\item
	the distributive law holds for the morphism composition map, i.e.
	\begin{equation*}
	h \circ (f + g) \; = \; h \circ f + h \circ g
	\quad\textnormal{and}\quad
	(f + g) \circ k \; = \; f \circ k + g \circ k
	\end{equation*}
	for each
	$f, g \in \Mor_{\mathfrak{C}}(A,B)$,
	$h \in \Mor_{\mathfrak{C}}(B,Y)$,
	$k \in \Mor_{\mathfrak{C}}(X,A)$,
	$A, B, X, Y \in \Obj(\mathfrak{C})$, and
\item
	finite products and finite coproducts exist in $\mathfrak{C}$.
\end{itemize}
\end{definition}

          %%%%% ~~~~~~~~~~~~~~~~~~~~ %%%%%
