
          %%%%% ~~~~~~~~~~~~~~~~~~~~ %%%%%

\section{The Snake Lemma in abelian categories with enough projectives}
\setcounter{theorem}{0}
\setcounter{equation}{0}

%\cite{vanDerVaart1996}
%\cite{Kosorok2008}

%\renewcommand{\theenumi}{\alph{enumi}}
%\renewcommand{\labelenumi}{\textnormal{(\theenumi)}$\;\;$}
\renewcommand{\theenumi}{\roman{enumi}}
\renewcommand{\labelenumi}{\textnormal{(\theenumi)}$\;\;$}

          %%%%% ~~~~~~~~~~~~~~~~~~~~ %%%%%

\begin{theorem}[Snake Lemma]
\mbox{}
\vskip 0.15cm
\noindent
Let \,$\mathfrak{A}$\, be an abelian category with enough projectives, and
the following be a commutative diagram with exact rows of morphisms in \,$\mathfrak{A}$:
\begin{center}
\begin{tikzcd}
A^{\prime}
	\arrow[dd, swap, "f^{\prime}"]
	\arrow[rr, "\mu_{A}"]
&&
A
	\arrow[dd, "f"]
	\arrow[rr, thick, two heads, "\varepsilon_{A}"]
&&
A^{\prime\prime}
	\arrow[dd, "f^{\prime\prime}"]
\\ \\
B^{\prime}
	\arrow[rr, thick, hook, swap, "\mu_{B}"]
&&
B
	\arrow[rr, swap, "\varepsilon_{B}"]
&&
B^{\prime\prime}
\end{tikzcd}
\end{center}
Then, the following statements are true:
\begin{enumerate}
\item
	Let
	\,$\kappa^{\prime}$,\, $\kappa$,\, $\kappa^{\prime\prime}$\, 
	denote respectively kernels of
	\,$f^{\prime}$,\, $f$,\, $f^{\prime\prime}$,\,
	and let
	\,$\pi^{\prime}$,\, $\pi$,\, $\pi^{\prime\prime}$\, 
	denote  respectively cokernels of 
	\,$f^{\prime}$,\, $f$,\, $f^{\prime\prime}$.
	Then, there exist morphisms
	\,$\mu_{K} : K^{\prime} \longrightarrow K$,\,
	\,$\varepsilon_{K} : K \longrightarrow K^{\prime\prime}$,\,
	\,$\mu_{Q} : Q^{\prime} \longrightarrow Q$,\,
	and
	\,$\varepsilon_{Q} : Q \longrightarrow Q^{\prime\prime}$\,
	such that the diagram above extends to the following commutative diagram:
	\begin{center}
	\begin{tikzcd}
	K^{\prime}
		\arrow[dd,swap, "\kappa^{\prime}"]
		\arrow[rr, thick, dashed, "\mu_{K}", red]
	&&
	K
		\arrow[dd, "\kappa"]
		\arrow[rr, thick, dashed, "\varepsilon_{K}", red]
	&&
	K^{\prime\prime}
		\arrow[dd, "\kappa^{\prime\prime}"]
	\\ \\
	A^{\prime}
		\arrow[dd, swap, "f^{\prime}"]
		\arrow[rr, "\mu_{A}"]
	&&
	A
		\arrow[dd, "f"]
		\arrow[rr, thick, two heads, "\varepsilon_{A}"]
	&&
	A^{\prime\prime}
		\arrow[dd, "f^{\prime\prime}"]
	\\ \\
	B^{\prime}
		\arrow[rr, thick, hook, swap, "\mu_{B}"]
		\arrow[dd, swap, "\pi^{\prime}"]
	&&
	B
		\arrow[rr, swap, "\varepsilon_{B}"]
		\arrow[dd, "\pi"]
	&&
	B^{\prime\prime}
		\arrow[dd, "\pi^{\prime\prime}"]
	\\ \\
	Q^{\prime}
		\arrow[rr, thick, dashed, swap, "\mu_{Q}", red]
	&&
	Q
		\arrow[rr, thick, dashed, swap, "\varepsilon_{Q}", red]
	&&
	Q^{\prime\prime}
	\end{tikzcd}
	\end{center}
\item
	There exists a morphism \,$\delta : K^{\prime\prime} \longrightarrow Q^{\prime}$\, such that
	\begin{center}
	\begin{tikzcd}
	K^{\prime}
		\arrow[rr, "\mu_{K}"]
	&&
	K
		\arrow[rr, "\varepsilon_{K}"]
	&&
	K^{\prime\prime}
		\arrow[rr, "\delta"]
	&&
	Q^{\prime}
		\arrow[rr, "\mu_{Q}"]
	&&
	Q
		\arrow[rr, "\varepsilon_{Q}"]
	&&
	Q^{\prime\prime}
	\end{tikzcd}
	\end{center}
	is a six-term exact sequence.
	\begin{center}
	\begin{tikzcd}
	K^{\prime}
		\arrow[dd,swap, "\kappa^{\prime}"]
		\arrow[rr, dashed, "\mu_{K}"]
	&&
	K
		\arrow[dd, "\kappa"]
		\arrow[rr, dashed, "\varepsilon_{K}"]
		\arrow[dddddd, phantom, ""{coordinate, name=X}]
	&&
	K^{\prime\prime}
		\arrow[dd, "\kappa^{\prime\prime}"]
		%\arrow[ddddddllll, thick, "\delta", red,
		%	crossing over, rounded corners,
		%	to path = { -- ([xshift=10ex]\tikztostart.east)
		%		|- (X) [near start]\tikztonodes
		%		-| ([xshift=-10ex]\tikztotarget.west)
		%		 -- (\tikztotarget)}
		%	 ]
	\\ \\
	A^{\prime}
		\arrow[dd, swap, near start, "f^{\prime}"]
		\arrow[rr, "\mu_{A}"]
	&&
	A
		\arrow[dd, near start, "f"]
		\arrow[rr, thick, two heads, "\varepsilon_{A}"]
	&&
	A^{\prime\prime}
		\arrow[dd, near start, "f^{\prime\prime}"]
	\\ \\
	B^{\prime}
		\arrow[rr, thick, hook, swap, "\mu_{B}"]
		\arrow[dd, swap, "\pi^{\prime}"]
	&&
	B
		%\arrow[from=uu, near start, "f"]
		\arrow[rr, swap, "\varepsilon_{B}"]
		\arrow[dd, "\pi"]
	&&
	B^{\prime\prime}
		\arrow[dd, "\pi^{\prime\prime}"]
	\\ \\
	Q^{\prime}
		\arrow[rr, dashed, swap, "\mu_{Q}"]
		\arrow[from=uuuuuurrrr, thick, "\textnormal{\LARGE$\delta$}", red,
			crossing over, rounded corners,
			to path = { -- ([xshift=10ex]\tikztostart.east)
				|- (X) [near start]\tikztonodes
				-| ([xshift=-10ex]\tikztotarget.west)
				 -- (\tikztotarget)}
			 ]
	&&
	Q
		\arrow[rr, dashed, swap, "\varepsilon_{Q}"]
	&&
	Q^{\prime\prime}
	\end{tikzcd}
	\end{center}
\end{enumerate}
\end{theorem}
\proof
\begin{enumerate}
\item
	\underline{Existence of $\mu_{K} \in \Mor_{\mathfrak{C}}(K^{\prime},K)$}
	\vskip -0.01cm
	First, note that
	\,$f \circ (\,\mu_{A} \circ \kappa^{\prime}\,)$
	\;$=$\; $(\,f \circ \mu_{A}\,)\circ \kappa^{\prime}$
	\;$=$\; $(\,\mu_{B}\circ f^{\prime}\,) \circ \kappa^{\prime}$
	\;$=$\; $\mu_{B} \circ (\,f^{\prime} \circ \kappa^{\prime}\,)$
	\;$=$\; $\mu_{B} \circ 0$
	\;$=$\; $0$.\,
	The universal property of
	\,$\kappa : K \longrightarrow A$\,
	as a kernel of
	\,$f : A \longrightarrow B$\,
	now implies the existence (and uniqueness) of $\mu_{K} : K^{\prime} \longrightarrow K$.
	\vskip 0.3cm

	\underline{Existence of $\varepsilon_{K} \in \Mor_{\mathfrak{C}}(K,K^{\prime\prime})$}
	\vskip -0.01cm
	This proof is completely analogous as the preceding one.
	More precisely, note that
	\,$f^{\prime\prime} \circ (\,\varepsilon_{A} \circ \kappa\,)$
	\;$=$\; $(\,f^{\prime\prime} \circ \varepsilon_{A}\,)\circ \kappa$
	\;$=$\; $(\,\varepsilon_{B} \circ f\,) \circ \kappa$
	\;$=$\; $\varepsilon_{B} \circ (\,f \circ \kappa\,)$
	\;$=$\; $\varepsilon_{B} \circ 0$
	\;$=$\; $0$.\,
	The universal property of
	\,$\kappa^{\prime\prime} : K^{\prime\prime} \longrightarrow A^{\prime\prime}$\,
	as a kernel of
	\,$f^{\prime\prime} : A^{\prime\prime} \longrightarrow B^{\prime\prime}$\,
	now implies the existence (and uniqueness) of
	\,$\varepsilon_{K} : K \longrightarrow K^{\prime\prime}$.
	\vskip 0.3cm

	\underline{Existence of $\mu_{Q} \in \Mor_{\mathfrak{C}}(Q^{\prime},Q)$}
	\vskip -0.01cm
	Note that
	\,$(\,\pi \circ \mu_{B}\,) \circ f^{\prime}$
	\;$=$\; $\pi \circ (\,\mu_{B} \circ f^{\prime}\,)$
	\;$=$\; $\pi \circ (\,f \circ \mu_{A}\,)$
	\;$=$\; $(\,\pi \circ f\,) \circ \mu_{A}$
	\;$=$\; $0 \,\circ\, \mu_{A}$
	\;$=$\; $0$.\,
	The universal property of
	\,$\pi^{\prime} : B^{\prime} \longrightarrow Q^{\prime}$\,
	as a cokernel of
	\,$f^{\prime} : A^{\prime} \longrightarrow B^{\prime}$\,
	now implies the existence (and uniqueness) of
	\,$\mu_{Q} : Q^{\prime} \longrightarrow Q$.
	\vskip 0.3cm

	\underline{Existence of $\varepsilon_{Q} \in \Mor_{\mathfrak{C}}(Q,Q^{\prime\prime})$}
	\vskip -0.01cm
	Note that
	\,$(\,\pi^{\prime\prime} \circ \varepsilon_{B}\,) \circ f$
	\;$=$\; $\pi^{\prime\prime} \circ (\,\varepsilon_{B} \circ f\,)$
	\;$=$\; $\pi^{\prime\prime} \circ (\,f^{\prime\prime} \circ \varepsilon_{A}\,)$
	\;$=$\; $(\,\pi^{\prime\prime} \circ f^{\prime\prime}\,) \circ \varepsilon_{A}$
	\;$=$\; $0 \,\circ\, \varepsilon_{A}$
	\;$=$\; $0$.\,
	The universal property of
	\,$\pi : B \longrightarrow Q$\,
	as a cokernel of
	\,$f : A \longrightarrow B$\,
	now implies the existence (and uniqueness) of
	\,$\varepsilon_{Q} : Q \longrightarrow Q^{\prime\prime}$.
	\vskip 0.5cm

\item
	First, we construct the morphism \,$\delta : K^{\prime\prime} \longrightarrow Q^{\prime}$,\, as follows:
	\begin{itemize}
	\item
		Since \,$\mathfrak{A}$\, is an (abelian) category with enough projectives,
		there exists an epimorphism
		\,$\varepsilon_{P} : P \longrightarrow K^{\prime\prime}$,\,
		with \,$P \in \Obj(\mathfrak{A})$\, being a projective object.
	\item
		Since \,$P$\, is a projective object and 
		\,$\varepsilon_{A} : A \longrightarrow A^{\prime\prime}$\,
		is an epimorphism, there exists
		\,$f_{1} : P \longrightarrow A$\,
		such that
		\,$\kappa^{\prime\prime} \circ \varepsilon_{P} \,=\, \varepsilon_{A} \circ f_{1}$\,
	\item
		Since \,$\mu_{B}$\, is a monomorphism
		and the composition
		\,$B^{\prime} \overset{\mu_{B}}{\longrightarrow} B \overset{\varepsilon_{B}}{\longrightarrow} B^{\prime\prime}$\,
		is exact, it follows
		by Proposition \ref{MonomorphicFirstFactorInExactSequenceIsKernelOfSecondFactor}
		that
		\,$\mu_{B}$\, is a kernel of \,$\varepsilon_{B}$.\,
		On the other hand, observe that
		\,$\varepsilon_{B} \circ (\,f \circ f_{1}\,)$
		\,$=$\, $(\,\varepsilon_{B} \circ f\,) \circ f_{1}$
		\,$=$\, $(\,f^{\prime\prime} \circ \varepsilon_{A}\,) \circ f_{1}$
		\,$=$\, $f^{\prime\prime} \circ (\,\varepsilon_{A} \circ f_{1}\,)$
		\,$=$\, $f^{\prime\prime} \circ (\,\kappa^{\prime\prime} \circ \varepsilon_{P}\,)$
		\,$=$\, $(\,f^{\prime\prime} \circ \kappa^{\prime\prime}\,) \circ \varepsilon_{P}$
		\,$=$\, $0 \circ \varepsilon_{P}$
		\,$=$\, $0$.\,
		Hence, there exists a unique
		\,$f_{2} : P \longrightarrow B^{\prime}$\,
		such that
		\,$f \circ f_{1} = \mu_{B} \circ f_{2}$.\,
	\item
		Next, we simply define
		\,$f_{3} \,:=\, \pi^{\prime} \circ f_{2}$.\,
	\end{itemize}
%	\begin{center}
%	\begin{tikzcd}
%	K^{\prime}
%		\arrow[dd,swap, "\kappa^{\prime}"]
%		\arrow[rr, thick, dashed, "\mu_{K}", red]
%	&&
%	K
%		\arrow[dd, "\kappa"]
%		\arrow[rr, thick, dashed, "\varepsilon_{K}", red]
%	&&
%	K^{\prime\prime}
%		\arrow[dd, "\kappa^{\prime\prime}"]
%	\\ \\
%	A^{\prime}
%		\arrow[dd, swap, "f^{\prime}"]
%		\arrow[rr, "\mu_{A}"]
%	&&
%	A
%		\arrow[dd, "f"]
%		\arrow[rr, thick, two heads, "\varepsilon_{A}"]
%	&&
%	A^{\prime\prime}
%		\arrow[dd, "f^{\prime\prime}"]
%	\\ \\
%	B^{\prime}
%		\arrow[rr, thick, hook, swap, "\mu_{B}"]
%		\arrow[dd, swap, "\pi^{\prime}"]
%	&&
%	B
%		\arrow[rr, swap, "\varepsilon_{B}"]
%		\arrow[dd, "\pi"]
%	&&
%	B^{\prime\prime}
%		\arrow[dd, "\pi^{\prime\prime}"]
%	\\ \\
%	Q^{\prime}
%		\arrow[rr, thick, dashed, swap, "\mu_{Q}", red]
%	&&
%	Q
%		\arrow[rr, thick, dashed, swap, "\varepsilon_{Q}", red]
%	&&
%	Q^{\prime\prime}
%	\end{tikzcd}
%	\end{center}
	\begin{center}
	\vskip -1.5cm
	\begin{tikzcd}
	&&&
	{\color{blue}K_{P}}
		\arrow[d, "\kappa_{P}", blue]
	&
	\\
	&
	{\color{white}K_{P}}
	&
	&
	{\color{blue}P}
		\arrow[dr, two heads, "\varepsilon_{P}", blue]
		\arrow[dddl, "f_{1}", red]
		\arrow[dddddlll, swap, "f_{2}", red]
		\arrow[dddddddlll, bend right = 110, "f_{3}", red]
	&
	\\
	K^{\prime}
		\arrow[dd, swap, "\kappa^{\prime}"]
		\arrow[rr, thick, dashed, "\mu_{K}"]
	&&
	K
		\arrow[dd, "\kappa"]
		\arrow[rr, thick, dashed, "\varepsilon_{K}"]
	&&
	K^{\prime\prime}
		\arrow[dd, "\kappa^{\prime\prime}"]
		\arrow[ddddddllll, thick, dashed, bend left = 29, "\textnormal{\Large$\delta$}", red]
	\\ \\
	A^{\prime}
		\arrow[dd, swap, "f^{\prime}"]
		\arrow[rr, "\mu_{A}"]
	&&
	A
		\arrow[dd, "f"]
		\arrow[rr, thick, two heads, "\varepsilon_{A}"]
	&&
	A^{\prime\prime}
		\arrow[dd, "f^{\prime\prime}"]
	\\ \\
	B^{\prime}
		\arrow[rr, thick, hook, swap, "\mu_{B}"]
		\arrow[dd, swap, "\pi^{\prime}"]
	&&
	B
		\arrow[rr, swap, "\varepsilon_{B}"]
		\arrow[dd, "\pi"]
	&&
	B^{\prime\prime}
		\arrow[dd, "\pi^{\prime\prime}"]
	\\ \\
	Q^{\prime}
		\arrow[rr, thick, dashed, swap, "\mu_{Q}"]
	&&
	Q
		\arrow[rr, thick, dashed, swap, "\varepsilon_{Q}"]
	&&
	Q^{\prime\prime}
	\end{tikzcd}
	\end{center}
	\vskip 0.3cm
	\textbf{Claim:} \;$f_{3}$\, factors through \,$\varepsilon_{P}$;\,
	more precisely, there exists
	\,$\delta : K^{\prime\prime} \longrightarrow Q^{\prime}$\,
	such that
	\,$f_{3} \,=\, \delta \circ \varepsilon_{P}$.
	\vskip 0.01cm
	Proof of Claim 1:\;
	Since \,$\mathfrak{A}$\, is an abelian category, every morphism in \,$\mathfrak{A}$\,
	admits a kernel.
	Let \,$\kappa_{P} : K_{P} \longrightarrow P$\, be a kernel of the epimorphism
	\,$\varepsilon_{P} : P \longrightarrow K^{\prime\prime}$.\,
	Again, since \,$\mathfrak{A}$\, is an abelian category, it is in particular conormal, i.e.,
	every epimorphism in \,$\mathfrak{A}$\, is a cokernel.
	By Lemma \ref{AKernelsTheKernelOfItsOwnCokernel} (ii),
	the epimorphism \,$\varepsilon_{P}$\, is thus a cokernel of of its kernel \,$\kappa_{P}$.\,
	Thus, the existence and uniqueness of the morphism
	\,$\delta : K^{\prime\prime} \longrightarrow Q^{\prime}$\,
	will follow immediately by the universal property of
	\,$\varepsilon_{P}$\, as a cokernel of \,$\kappa_{P}$,\,
	as soon as we establish that
	\,$f_{3} \circ \kappa_{P} \,=\, 0$.\,
	
\end{enumerate}
\qed

          %%%%% ~~~~~~~~~~~~~~~~~~~~ %%%%%
