
          %%%%% ~~~~~~~~~~~~~~~~~~~~ %%%%%

\section{Summary}
\setcounter{theorem}{0}
\setcounter{equation}{0}

%\cite{vanDerVaart1996}
%\cite{Kosorok2008}

%\renewcommand{\theenumi}{\alph{enumi}}
%\renewcommand{\labelenumi}{\textnormal{(\theenumi)}$\;\;$}
\renewcommand{\theenumi}{\roman{enumi}}
\renewcommand{\labelenumi}{\textnormal{(\theenumi)}$\;\;$}

          %%%%% ~~~~~~~~~~~~~~~~~~~~ %%%%%

Homological algebra ``makes sense'' in the category of $R$-modules,
in the sense that every short exact sequence of (co)chain complexes of $R$-modules
gives rise to a long exact sequence of (co)homology groups.

We would like to extend homological algebra to the setting of categories.
This extended theory, when applied to, for example,
the abelian category of sheaves of $\mathcal{O}_{X}$-modules
on a scheme or complex analytic space $(X,\mathcal{O}_{X})$,
allows us to construct sheaf cohomology groups as invariants for $(X,\mathcal{O}_{X})$.
The foundation of modern algebraic geometry constitutes in large part the vast body of sheaf-cohomological results established
for various types of schemes and complex analytic spaces.

Clearly, it is not possible to develop homological algebra
(short exact sequence of (co)chain complex giving rise to long exact sequence of (co)homologies)
in an arbitrary category.
More precisely, at least we would need to work in a category with sufficient structure
for the notion of short exact sequences of complexes to be well-defined.
In particular, one can begin the development of homological algebra only in a category
in which the notions of zero objects/morphisms, kernels of morphisms, and images of morphisms are defined.

Abelian categories are precisely the categories in which homological algebra makes sense.
In an abelian category, the following statements are true:
\begin{itemize}
\item
	A zero object exists.
	Kernels and cokernels of morphisms exist.
	Hence, the notion of a short exact sequence makes sense.
\item
	The Snake Lemma holds, which in turn implies that
	short exact sequences of cochain complexes generate long exact sequences in cohomology.
\item
	When the abelian category is the category of abelian groups or $R$-modules,
	the construction of the connecting homomorphisms involves a simple diagram chase.
	For an arbitrary abelian category, that construction involves either the deep result
	of Mitchell's Embedding Theorem, or clever but hard-to-motivate tricks.
\item
	However, when the abelian category has enough injectives,
	the construction of the connecting homomorphisms is formally identical to
	that for $R$-modules. 
\item
	Let $(X,\mathcal{O}_{X})$ be a ringed space.
	Then, the category of sheaves of $\mathcal{O}_{X}$-modules
	is an abelian category with enough injectives, and
	the global section functor $\Gamma_{X}$ is a left-exact functor.
	The $p^{\textnormal{th}}$ right derived functor
	\begin{equation*}
	R^{p}\Gamma_{X}
	\;\; : \;\;
	\left(\begin{array}{c} \textnormal{abelian category of} \\ \textnormal{sheaves of $\mathcal{O}_{X}$-modules} \end{array}\right)
	\;\;\longrightarrow\;\;
	\left(\begin{array}{c} \textnormal{abelian category of} \\ \textnormal{abelian groups} \end{array}\right)
	\end{equation*}
	is thus well-defined.
	The $p^{\textnormal{th}}$ sheaf cohomology functor $H^{p}(X,\,\cdot\,)$ is defined to be:
	\begin{equation*}
	H^{p}(X,\,\cdot\,)
	\;\; := \;\;
		R^{p}\Gamma_{X}(\,\cdot\,)
	\end{equation*}
\item
	Let $(X,\mathcal{O}_{X})$ be a scheme (respectively, complex analytic space).
	The category of \textit{coherent algebraic sheaves} (respectively, \textit{coherent analytic sheaves})
	is an abelian category, and it is a full subcategory of the category of sheaves of $\mathcal{O}_{X}$-modules.
\item
	Let $(X,\mathcal{O}_{X})$ be a scheme.
	Let $\mathcal{F}$ be an arbitrary coherent or quasi-coherent sheaf of $\mathcal{O}_{X}$-modules,
	and $H^{p}(X,\mathcal{F})$ its $p^{\textnormal{th}}$ sheaf cohomology.
	The large body of results about $H^{p}(X,\mathcal{F})$ forms the core foundational results of modern algebraic geometry.
\end{itemize}

          %%%%% ~~~~~~~~~~~~~~~~~~~~ %%%%%
