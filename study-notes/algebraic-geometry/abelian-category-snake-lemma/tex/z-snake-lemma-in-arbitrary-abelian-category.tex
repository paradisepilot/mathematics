
          %%%%% ~~~~~~~~~~~~~~~~~~~~ %%%%%

\section{The Snake Lemma for an arbitrary abelian category}
\setcounter{theorem}{0}
\setcounter{equation}{0}

%\cite{vanDerVaart1996}
%\cite{Kosorok2008}

%\renewcommand{\theenumi}{\alph{enumi}}
%\renewcommand{\labelenumi}{\textnormal{(\theenumi)}$\;\;$}
\renewcommand{\theenumi}{\roman{enumi}}
\renewcommand{\labelenumi}{\textnormal{(\theenumi)}$\;\;$}
%\renewcommand{\labelenumii}{$\quad\quad\quad\quad$\textnormal{(\theenumii)}}
\renewcommand{\labelenumii}{\quad\quad\textnormal{(\theenumii)}}

          %%%%% ~~~~~~~~~~~~~~~~~~~~ %%%%%

\begin{theorem}[Snake Lemma for an arbitrary abelian category, Lemma 12.1.1, p.297, \cite{kashiwara2005categories}]
\mbox{}
\vskip 0.15cm
\noindent
Let \,$\mathfrak{A}$\, be an abelian category, and
the following be a commutative diagram with exact rows of morphisms in \,$\mathfrak{A}$:
\begin{center}
\begin{tikzcd}
A^{\prime}
	\arrow[dd, swap, "f^{\prime}"]
	\arrow[rr, "\mu_{A}"]
&&
A
	\arrow[dd, "f"]
	\arrow[rr, thick, two heads, "\varepsilon_{A}"]
&&
A^{\prime\prime}
	\arrow[dd, "f^{\prime\prime}"]
\\ \\
B^{\prime}
	\arrow[rr, thick, hook, swap, "\mu_{B}"]
&&
B
	\arrow[rr, swap, "\varepsilon_{B}"]
&&
B^{\prime\prime}
\end{tikzcd}
\end{center}
Then, the following statements are true:
\begin{enumerate}
\item
	Let
	\,$\kappa^{\prime}$,\, $\kappa$,\, $\kappa^{\prime\prime}$\, 
	denote respectively kernels of
	\,$f^{\prime}$,\, $f$,\, $f^{\prime\prime}$,\,
	and let
	\,$\pi^{\prime}$,\, $\pi$,\, $\pi^{\prime\prime}$\, 
	denote  respectively cokernels of 
	\,$f^{\prime}$,\, $f$,\, $f^{\prime\prime}$.
	Then, there exist morphisms
	\,$\mu_{K} : K^{\prime} \longrightarrow K$,\,
	\,$\varepsilon_{K} : K \longrightarrow K^{\prime\prime}$,\,
	\,$\mu_{Q} : Q^{\prime} \longrightarrow Q$,\,
	and
	\,$\varepsilon_{Q} : Q \longrightarrow Q^{\prime\prime}$\,
	such that the diagram above extends to the following commutative diagram:
	\begin{center}
	\begin{tikzcd}
	K^{\prime}
		\arrow[dd,swap, "\kappa^{\prime}"]
		\arrow[rr, thick, dashed, "\mu_{K}", red]
	&&
	K
		\arrow[dd, "\kappa"]
		\arrow[rr, thick, dashed, "\varepsilon_{K}", red]
	&&
	K^{\prime\prime}
		\arrow[dd, "\kappa^{\prime\prime}"]
	\\ \\
	A^{\prime}
		\arrow[dd, swap, "f^{\prime}"]
		\arrow[rr, "\mu_{A}"]
	&&
	A
		\arrow[dd, "f"]
		\arrow[rr, thick, two heads, "\varepsilon_{A}"]
	&&
	A^{\prime\prime}
		\arrow[dd, "f^{\prime\prime}"]
	\\ \\
	B^{\prime}
		\arrow[rr, thick, hook, swap, "\mu_{B}"]
		\arrow[dd, swap, "\pi^{\prime}"]
	&&
	B
		\arrow[rr, swap, "\varepsilon_{B}"]
		\arrow[dd, "\pi"]
	&&
	B^{\prime\prime}
		\arrow[dd, "\pi^{\prime\prime}"]
	\\ \\
	Q^{\prime}
		\arrow[rr, thick, dashed, swap, "\mu_{Q}", red]
	&&
	Q
		\arrow[rr, thick, dashed, swap, "\varepsilon_{Q}", red]
	&&
	Q^{\prime\prime}
	\end{tikzcd}
	\end{center}
	Furthermore, the two compositions
	\begin{tikzcd}
	K^{\prime}
		\arrow[r, "\mu_{K}"]
	&
	K
		\arrow[r, "\varepsilon_{K}"]
	&
	K^{\prime\prime}
	\end{tikzcd}
	and
	\begin{tikzcd}
	Q^{\prime}
		\arrow[r, "\mu_{Q}"]
	&
	Q
		\arrow[r, "\varepsilon_{Q}"]
	&
	Q^{\prime\prime}
	\end{tikzcd}
	are exact.
\item
	There exists a morphism \,$\delta : K^{\prime\prime} \longrightarrow Q^{\prime}$\, such that
	\begin{center}
	\begin{tikzcd}
	K^{\prime}
		\arrow[rr, "\mu_{K}"]
	&&
	K
		\arrow[rr, "\varepsilon_{K}"]
	&&
	K^{\prime\prime}
		\arrow[rr, "\delta"]
	&&
	Q^{\prime}
		\arrow[rr, "\mu_{Q}"]
	&&
	Q
		\arrow[rr, "\varepsilon_{Q}"]
	&&
	Q^{\prime\prime}
	\end{tikzcd}
	\end{center}
	is a six-term exact sequence.
	\begin{center}
	\begin{tikzcd}
	K^{\prime}
		\arrow[dd,swap, "\kappa^{\prime}"]
		\arrow[rr, dashed, "\mu_{K}"]
	&&
	K
		\arrow[dd, "\kappa"]
		\arrow[rr, dashed, "\varepsilon_{K}"]
		\arrow[dddddd, phantom, ""{coordinate, name=X}]
	&&
	K^{\prime\prime}
		\arrow[dd, "\kappa^{\prime\prime}"]
		%\arrow[ddddddllll, thick, "\delta", red,
		%	crossing over, rounded corners,
		%	to path = { -- ([xshift=10ex]\tikztostart.east)
		%		|- (X) [near start]\tikztonodes
		%		-| ([xshift=-10ex]\tikztotarget.west)
		%		 -- (\tikztotarget)}
		%	 ]
	\\ \\
	A^{\prime}
		\arrow[dd, swap, near start, "f^{\prime}"]
		\arrow[rr, "\mu_{A}"]
	&&
	A
		\arrow[dd, near start, "f"]
		\arrow[rr, thick, two heads, "\varepsilon_{A}"]
	&&
	A^{\prime\prime}
		\arrow[dd, near start, "f^{\prime\prime}"]
	\\ \\
	B^{\prime}
		\arrow[rr, thick, hook, swap, "\mu_{B}"]
		\arrow[dd, swap, "\pi^{\prime}"]
	&&
	B
		%\arrow[from=uu, near start, "f"]
		\arrow[rr, swap, "\varepsilon_{B}"]
		\arrow[dd, "\pi"]
	&&
	B^{\prime\prime}
		\arrow[dd, "\pi^{\prime\prime}"]
	\\ \\
	Q^{\prime}
		\arrow[rr, dashed, swap, "\mu_{Q}"]
		\arrow[from=uuuuuurrrr, thick, "\textnormal{\LARGE$\delta$}", red,
			crossing over, rounded corners,
			to path = { -- ([xshift=10ex]\tikztostart.east)
				|- (X) [near start]\tikztonodes
				-| ([xshift=-10ex]\tikztotarget.west)
				 -- (\tikztotarget)}
			 ]
	&&
	Q
		\arrow[rr, dashed, swap, "\varepsilon_{Q}"]
	&&
	Q^{\prime\prime}
	\end{tikzcd}
	\end{center}
\end{enumerate}
\end{theorem}
\proof
\begin{enumerate}
\item
	\underline{Existence of \,$\mu_{K} \in \Mor_{\mathfrak{A}}(K^{\prime},K)$}
	\vskip -0.01cm
	First, note that
	\,$f \circ (\,\mu_{A} \circ \kappa^{\prime}\,)$
	\;$=$\; $(\,f \circ \mu_{A}\,)\circ \kappa^{\prime}$
	\;$=$\; $(\,\mu_{B}\circ f^{\prime}\,) \circ \kappa^{\prime}$
	\;$=$\; $\mu_{B} \circ (\,f^{\prime} \circ \kappa^{\prime}\,)$
	\;$=$\; $\mu_{B} \circ 0$
	\;$=$\; $0$.\,
	The universal property of
	\,$\kappa : K \longrightarrow A$\,
	as a kernel of
	\,$f : A \longrightarrow B$\,
	now implies the existence (and uniqueness) of $\mu_{K} : K^{\prime} \longrightarrow K$.

	\vskip 0.3cm
	\underline{Existence of \,$\varepsilon_{K} \in \Mor_{\mathfrak{A}}(K,K^{\prime\prime})$}
	\vskip -0.01cm
	This proof is completely analogous as the preceding one.
	More precisely, note that
	\,$f^{\prime\prime} \circ (\,\varepsilon_{A} \circ \kappa\,)$
	\;$=$\; $(\,f^{\prime\prime} \circ \varepsilon_{A}\,)\circ \kappa$
	\;$=$\; $(\,\varepsilon_{B} \circ f\,) \circ \kappa$
	\;$=$\; $\varepsilon_{B} \circ (\,f \circ \kappa\,)$
	\;$=$\; $\varepsilon_{B} \circ 0$
	\;$=$\; $0$.\,
	The universal property of
	\,$\kappa^{\prime\prime} : K^{\prime\prime} \longrightarrow A^{\prime\prime}$\,
	as a kernel of
	\,$f^{\prime\prime} : A^{\prime\prime} \longrightarrow B^{\prime\prime}$\,
	now implies the existence (and uniqueness) of
	\,$\varepsilon_{K} : K \longrightarrow K^{\prime\prime}$.

	\vskip 0.3cm
	\underline{Existence of \,$\mu_{Q} \in \Mor_{\mathfrak{A}}(Q^{\prime},Q)$}
	\vskip -0.01cm
	Note that
	\,$(\,\pi \circ \mu_{B}\,) \circ f^{\prime}$
	\;$=$\; $\pi \circ (\,\mu_{B} \circ f^{\prime}\,)$
	\;$=$\; $\pi \circ (\,f \circ \mu_{A}\,)$
	\;$=$\; $(\,\pi \circ f\,) \circ \mu_{A}$
	\;$=$\; $0 \,\circ\, \mu_{A}$
	\;$=$\; $0$.\,
	The universal property of
	\,$\pi^{\prime} : B^{\prime} \longrightarrow Q^{\prime}$\,
	as a cokernel of
	\,$f^{\prime} : A^{\prime} \longrightarrow B^{\prime}$\,
	now implies the existence (and uniqueness) of
	\,$\mu_{Q} : Q^{\prime} \longrightarrow Q$.

	\vskip 0.3cm
	\underline{Existence of \,$\varepsilon_{Q} \in \Mor_{\mathfrak{A}}(Q,Q^{\prime\prime})$}
	\vskip -0.01cm
	Note that
	\,$(\,\pi^{\prime\prime} \circ \varepsilon_{B}\,) \circ f$
	\;$=$\; $\pi^{\prime\prime} \circ (\,\varepsilon_{B} \circ f\,)$
	\;$=$\; $\pi^{\prime\prime} \circ (\,f^{\prime\prime} \circ \varepsilon_{A}\,)$
	\;$=$\; $(\,\pi^{\prime\prime} \circ f^{\prime\prime}\,) \circ \varepsilon_{A}$
	\;$=$\; $0 \,\circ\, \varepsilon_{A}$
	\;$=$\; $0$.\,
	The universal property of
	\,$\pi : B \longrightarrow Q$\,
	as a cokernel of
	\,$f : A \longrightarrow B$\,
	now implies the existence (and uniqueness) of
	\,$\varepsilon_{Q} : Q \longrightarrow Q^{\prime\prime}$.

	\vskip 0.5cm
	\underline{Exactness of{\color{white}.q}$
		K^{\prime}
		\,\overset{\mu_{K}}{\longrightarrow}\,
		K
		\,\overset{\varepsilon_{K}}{\longrightarrow}\,
		K^{\prime\prime}
		$}
	\vskip -0.15cm
	By Proposition \ref{NecessarySufficientConditionForExactnessInAbelianCategories},
	in order to establish the exactness of the composition
	\begin{tikzcd}
	K^{\prime}
		\arrow[r, "\mu_{K}"]
	&
	K
		\arrow[r, "\varepsilon_{K}"]
	&
	K^{\prime\prime}
	\end{tikzcd}\!\!,
	it suffices to show that, given an arbitrary morphism
	\,$h \,\in\, \Mor_{\mathfrak{A}}(X,K)$\,
	such that
	\,$\varepsilon_{K} \,\circ\, h \,=\, 0$,\,
	there exist
	a morphism \,$h^{\prime} \,\in\, \Mor_{\mathfrak{A}}(X^{\prime},K^{\prime})$\, and
	an epimorphism \,$\mu^{\prime} \,\in\, \Mor_{\mathfrak{A}}(X^{\prime},X)$\,
	such that
	\,$\mu_{K} \,\circ\, h^{\prime} \,=\, h \,\circ\, \mu^{\prime}$,\,
	i.e., the following diagram commutes:
	\begin{center}
	\begin{tikzcd}
	{\color{red}X^{\prime}}
		\arrow[rr, red, thick, dashed, two heads, "\mu^{\prime}"]
		\arrow[dd, red, thick, dashed, "h^{\prime}", swap]
	&&
	{\color{blue}X}
		\arrow[dd, blue, thick, "h"]
		\arrow[ddrr, gray, bend left = 20, "0"]
	\\ \\
	K^{\prime}
		\arrow[rr, black, thick, "\mu_{K}"]
	&&
	K
		\arrow[rr, black, thick, "\varepsilon_{K}"]
	&&
	K^{\prime\prime}
	\end{tikzcd}
	\end{center}
	To this end, consider the following extended diagram:
	\begin{center}
	\begin{tikzcd}
	{\color{red}X^{\prime}}
		\arrow[rr, red, thick, dashed, two heads, "\mu^{\prime}"]
		\arrow[dd, red, thick, dashed, "h^{\prime}", swap]
		\arrow[dddd, red, thick, dashed, bend right = 80, "\rho", swap]
	&&
	{\color{blue}X}
		\arrow[dd, blue, thick, "h"]
		\arrow[ddrr, gray, bend left = 20, "0"]
	\\ \\
	K^{\prime}
		\arrow[rr, black, thick, "\mu_{K}"]
		\arrow[dd, black, thick, "\kappa^{\prime}", swap]
	&&
	K
		\arrow[rr, black, thick, "\varepsilon_{K}"]
		\arrow[dd, black, thick, "\kappa"]
	&&
	K^{\prime\prime}
		\arrow[dd, black, thick, "\kappa^{\prime\prime}"]
	\\ \\
	A^{\prime}
		\arrow[rr, black, thick, "\mu_{A}"]
		\arrow[dd, black, thick, "f^{\prime}", swap]
	&&
	A
		\arrow[rr, black, thick, two heads, "\varepsilon_{A}"]
		\arrow[dd, black, thick, "f"]
		\arrow[rr, black, thick, "\varepsilon_{A}"]
	&&
	A^{\prime\prime}
		\arrow[dd, black, thick, "f^{\prime\prime}"]
	\\ \\
	B^{\prime}
		\arrow[rr, black, thick, hook, "\mu_{B}"]
	&&
	B
		\arrow[rr, black, thick, "\varepsilon_{B}"]
	&&
	B^{\prime\prime}
	\end{tikzcd}
	\end{center}
	% \begin{center}
	% \begin{tikzcd}
	% {\color{red}X^{\prime}}
	% 	\arrow[ddddrr, red, thick, dashed, bend right = 30, "\rho", swap]
	% 	\arrow[rrrr, red, thick, dashed, two heads, "\mu^{\prime}"]
	% 	\arrow[ddrr, red, thick, dashed, "h^{\prime}", swap]
	% &&
	% % {\color{red}X^{\prime}}
	% % 	\arrow[dd, red, thick, dashed, "h^{\prime}", swap]
	% &&
	% {\color{blue}X}
	% 	\arrow[dd, blue, thick, "h"]
	% 	\arrow[ddrr, gray, bend left = 20, "0"]
	% \\ \\
	% &&
	% K^{\prime}
	% 	\arrow[rr, black, thick, "\mu_{K}"]
	% 	\arrow[dd, black, thick, "\kappa^{\prime}", swap]
	% &&
	% K
	% 	\arrow[rr, black, thick, "\varepsilon_{K}"]
	% 	\arrow[dd, black, thick, "\kappa"]
	% &&
	% K^{\prime\prime}
	% 	\arrow[dd, black, thick, "\kappa^{\prime\prime}"]
	% \\ \\
	% &&
	% A^{\prime}
	% 	\arrow[rr, black, thick, "\mu_{A}"]
	% 	\arrow[dd, black, thick, "f^{\prime}", swap]
	% &&
	% A
	% 	\arrow[rr, black, thick, two heads, "\varepsilon_{A}"]
	% 	\arrow[dd, black, thick, "f"]
	% 	\arrow[rr, black, thick, "\varepsilon_{A}"]
	% &&
	% A^{\prime\prime}
	% 	\arrow[dd, black, thick, "f^{\prime\prime}"]
	% \\ \\
	% &&
	% B^{\prime}
	% 	\arrow[rr, black, thick, hook, "\mu_{B}"]
	% &&
	% B
	% 	\arrow[rr, black, thick, "\varepsilon_{B}"]
	% &&
	% B^{\prime\prime}
	% \end{tikzcd}
	% \end{center}
	Note that:
	\begin{eqnarray*}
	\varepsilon_{A} \,\circ \left(\,
		\kappa \overset{{\color{white}1}}{\circ} h
		\,\right)
	& = &
		\left(\,\varepsilon_{A} \overset{{\color{white}1}}{\circ} \kappa \,\right)
		\circ \, h
	\;\; = \;\;
		\left(\,\kappa^{\prime\prime} \overset{{\color{white}1}}{\circ} \varepsilon_{K} \,\right)
		\circ \, h
	\;\; = \;\;
		\kappa^{\prime\prime} \,\circ \left(\,
		\varepsilon_{K}
		\overset{{\color{white}1}}{\circ}
		h
		 \,\right)
	\;\; = \;\;
		\kappa^{\prime\prime} \,\circ\, 0
	\;\; = \;\;
		0
	\end{eqnarray*}
	By Proposition \ref{NecessarySufficientConditionForExactnessInAbelianCategories},
	the exactness of
	\begin{tikzcd}
	A^{\prime}
		\arrow[r, "\mu_{A}"]
	&
	A
		\arrow[r, "\varepsilon_{A}"]
	&
	A^{\prime\prime}
	\end{tikzcd}
	implies the existence of
	a morphism \,$\rho \,\in\, \Mor_{\mathfrak{A}}(X^{\prime},A^{\prime})$\,
	and
	an epimorphism \,$\mu^{\prime} \,\in\, \Mor_{\mathfrak{A}}(X^{\prime},X)$\,
	such that
	\,$\mu_{A} \,\circ\, \rho
	\, = \,
		\left(\,
			\kappa \overset{{\color{white}1}}{\circ} h
			\,\right)
		\circ\, \mu^{\prime}
	$.\,
	\vskip 0.3cm
	\noindent
	\textbf{Claim 1:}\quad $f^{\prime} \,\circ\, \rho \,=\, 0$
	\vskip 0.1cm
	\noindent
	Proof of Claim 1:\;\;
	Observe that
	\begin{eqnarray*}
	\mu_{B} \,\circ\, f^{\prime} \,\circ\, \rho
	& = &
		f \,\circ\, \mu_{A} \,\circ\, \rho
	\;\; = \;\;
		f \,\circ\, (\,\kappa \circ h\,) \,\circ\, \mu^{\prime}
	\;\; = \;\;
		(\,f \circ \kappa\,) \,\circ\, h \,\circ\, \mu^{\prime}
	\;\; = \;\;
		(\,0\,) \,\circ\, h \,\circ\, \mu^{\prime}
	\;\; = \;\;
		0\,,
	\end{eqnarray*}
	which implies
	\,$f^{\prime} \,\circ\, \rho \,=\, 0$,\,
	since \,$\mu_{B}$\, is by hypothesis a monomorphism.
	This proves Claim 1.
	\vskip 0.3cm
	\noindent
	The universal property of \,$\kappa^{\prime}$\, as a kernel of \,$f^{\prime}$\,
	now implies that there exists a unique morphism
	\,$h^{\prime} \,\in\, \Mor_{\mathfrak{A}}(X^{\prime},K^{\prime})$\,
	such that
	\,$\rho \,=\, \kappa^{\prime} \,\circ\, h^{\prime}$.\,
	\vskip 0.3cm
	\noindent
	\textbf{Claim 2:}\quad $\mu_{K} \,\circ\, h^{\prime} \,=\, h \,\circ\, \mu^{\prime}$
	\vskip 0.1cm
	\noindent
	Proof of Claim 2:\;\;
	Observe that
	\begin{eqnarray*}
	\kappa \,\circ\, \mu_{K} \,\circ\, h^{\prime}
	& = &
		\mu_{A} \,\circ\, \kappa^{\prime} \,\circ\, h^{\prime}
	\;\; = \;\;
		\mu_{A} \,\circ\, \rho
	\;\; = \;\;
		\kappa \,\circ\, h \,\circ\, \mu^{\prime}\,,
	\end{eqnarray*}
	which implies
	\,$\mu_{K} \,\circ\, h^{\prime} \,=\, h \,\circ\, \mu^{\prime}$,\,
	since \,$\kappa$\, is a monomorphism (being a kernel of a morphism in an abelian category).
	This proves Claim 2, as well as completes the proof of the exactness of
	\begin{tikzcd}
	K^{\prime}
		\arrow[r, "\mu_{K}"]
	&
	K
		\arrow[r, "\varepsilon_{K}"]
	&
	K^{\prime\prime}
	\end{tikzcd}\!\!.

	\vskip 0.5cm
	\underline{Exactness of{\color{white}.q}$
		Q^{\prime}
		\,\overset{\mu_{Q}}{\longrightarrow}\,
		Q
		\,\overset{\varepsilon_{Q}}{\longrightarrow}\,
		Q^{\prime\prime}
		$}
	\vskip 0.05cm
	\noindent
	This proof can be established by reversing the arrows in the preceding argument.
	We omit this proof.
	\vskip 0.5cm

\item
	We first give the definition of
	\,$\delta \,\in\, \Mor_{\mathfrak{A}}(K^{\prime\prime},Q^{\prime})$.\,
	Consider the following extended diagram:
	\begin{center}
	\begin{tikzcd}
	&&{\color{blue}K(\pi_{K^{\prime\prime}})}
		\arrow[rr, blue, thick, hook, "\kappa(\pi_{K^{\prime\prime}})"]
		\arrow[dd, blue, thick, hook, "\alpha", swap]
	&&
	{\color{blue}A \,\sqcap_{A^{\prime\prime}}\! K^{\prime\prime}}
		\arrow[dd, blue, thick, hook, "\,\pi_{A}"]
		\arrow[rr, blue, thick, two heads, "\pi_{K^{\prime\prime}}"]
		\arrow[ddddll, red, thick, dashed, bend right = 5, "\theta_{1}", swap]
	&&
	K^{\prime\prime}
		\arrow[dd, thick, hook, "\kappa^{\prime\prime}"]
		\arrow[ddddddllll, red, thick, dashed, bend left = 30, "\textnormal{\LARGE$\delta$}"]
	\\ \\
	A^{\prime}
		\arrow[ddrr, thick, "f^{\prime}", swap]
		\arrow[rr, blue, thick, two heads, "{\color{white}...}\overset{{\color{white}.}}{\widetilde{\mu_{A}}}", swap]
		\arrow[rrrr, thick, bend left = 30, "\quad{\color{white}....}\mu_{A}"]
	&&
	{\color{blue}I(\mu_{A})\,\cong\,K(\varepsilon_{A})}
		\arrow[rr, blue, thick, hook,   "\kappa(\varepsilon_{A})", swap]
		\arrow[dd, red,  thick, dashed, "\theta_{2}", swap]
	&&
	A
		\arrow[dd, thick, "f"]
		\arrow[rr, thick, two heads, "\varepsilon_{A}"]
	&&
	A^{\prime\prime}
		\arrow[dd, thick, "f^{\prime\prime}"]
	\\ \\
	&&B^{\prime}
		\arrow[rr, thick, hook, swap, "\mu_{B}"]
		\arrow[dd, thick, two heads, "\pi^{\prime}", swap]
	&&
	B
		\arrow[rr, thick, "\varepsilon_{B}", swap]
		\arrow[dd, blue, thick, two heads, "\iota_{B}"]
	&&
	B^{\prime\prime}
		% \arrow[dd, "\pi^{\prime\prime}"]
	\\ \\
	&&Q^{\prime}
		\arrow[rr, blue, thick, hook, "\iota_{Q^{\prime}}", swap]
	&&
	{\color{blue}Q^{\prime} \sqcup_{B^{\prime}}\! B}
		\arrow[rr, blue, thick, two heads, "\pi(\iota_{Q^{\prime}}){\color{white}..}", swap]
	&&
	{\color{blue}Q(\iota_{Q^{\prime}})}
	\end{tikzcd}
	\end{center}
	First, note:
	\begin{itemize}
	\item
		By hypothesis, \,$\varepsilon_{A}$\, is an epimorphism; hence,
		\,$\pi_{K^{\prime\prime}}$\, is an epimorphism
		by Proposition \ref{FiberProductsPreserveEpimorphismsInAbelianCategories}(i).
	\item
		By hypothesis,
		\,$\iota_{Q^{\prime}}$\, is a monomorphism; hence
		\,$\mu_{B}$\, is a monomorphism
		by Proposition \ref{FiberProductsPreserveEpimorphismsInAbelianCategories}(ii).
	\item
		Being a kernel,
		\,$\kappa^{\prime\prime}$\, is a monomorphism; hence
		\,$\pi_{A}$\, is a monomorphism
		by Proposition \ref{FiberProductsPreserveMonomorphisms}(i).
	\item
		Being a cokernel,
		\,$\pi^{\prime}$\, is an epimorphism; hence
		\,$\iota_{B}$\, is an epimorphism
		by Proposition \ref{FiberProductsPreserveMonomorphisms}(ii).
	\item
		By Lemma \ref{FiberProductKernalMonomorphism},
		there exists a unique
		\,$\alpha \,\in\, \Mor_{\mathfrak{A}}\!\left(\,
			K(\pi_{K^{\prime\prime}})
			\, \overset{{\color{white}\textnormal{\large1}}}{,} \,
			K(\varepsilon_{A})
			\,\right)$\,
		such that
		\,$\kappa(\varepsilon_{A}) \,\circ\, \alpha \;=\; \pi_{A} \,\circ\, \kappa(\pi_{K^{\prime\prime}})$.
		Since \,$\kappa^{\prime\prime}$\, is a monomorphism,
		it follows by Lemma \ref{FiberProductKernalMonomorphism} again that
		\,$\alpha$\, is furthermore a monomorphism.
	\item
		By Proposition \ref{propositionHomology}, the fact that
		\,$\varepsilon_{A} \,\circ\, \mu_{A} \,=\, 0$\,
		(this is part of the exactness hypothesis on\\
		\begin{tikzcd}
		A^{\prime}
			\arrow[r, "\mu_{A}"]
		&
		A
			\arrow[r, "\varepsilon_{A}"]
		&
		A^{\prime\prime}
		\end{tikzcd}\!)
		implies that there exist unique morphisms
		\,$\widetilde{\mu_{A}}$\,
		and
		\,$\theta$\,
		such that the following diagram commutes:
		\begin{center}
		\begin{tikzcd}
		&& && Q(\mu_{A})
			\arrow[dd, thick, dashed]
		\\ \\
		A^{\prime}
			\arrow[rr, thick, "\mu_{A}"]
			\arrow[dd, red, thick, dashed, two heads, swap, "\widetilde{\mu_{A}}"]
			\arrow[rrrr, thick, bend left = 20, "0", gray]
		&&
		A
			\arrow[rr, thick, "\varepsilon_{A}"]
			\arrow[uurr, two heads, "\pi(\mu_{A})"]
		&&
		A^{\prime\prime}
		\\ \\
		I(\mu_{A})
			\arrow[uurr, hook, "\kappa(\pi(\mu_{A}))"]
			\arrow[rr, red, thick, dashed, "\exists !\,\theta", swap]
		&&
		K(\varepsilon_{A})
			\arrow[uu, thick, hook, "\kappa(\varepsilon_{A})", swap]
			\arrow[rr, thick, two heads, "\pi_{\theta}", swap]
		&&
		Q_{\theta}
		\end{tikzcd}
		\end{center}
		Lastly, the exactness hypothesis on
		\begin{tikzcd}
		A^{\prime}
			\arrow[r, "\mu_{A}"]
		&
		A
			\arrow[r, "\varepsilon_{A}"]
		&
		A^{\prime\prime}
		\end{tikzcd}
		is equivalent to the assumption that \,$\theta$\, is in addition an isomorphism.
	\end{itemize}

	\vskip 0.3cm
	\noindent
	\textbf{Claim 1:}\quad
	There exists a unique
	\,$\theta_{1} \in \Mor_{\mathfrak{A}}(\,A\,\sqcap_{A^{\prime\prime}} K^{\prime\prime}\,,\,B^{\prime}\,)$\,
	such that
	\,$f \,\circ\, \pi_{A} \,=\, \mu_{B} \,\circ\, \theta_{1}$.\,
	% \,$\delta \in \Mor_{\mathfrak{A}}(\,K^{\prime\prime}\,,\,Q^{\prime}\,)$\,
	% such that
	% \,$\iota_{B} \,\circ\, f \,\circ\, \pi_{A} \,=\, \iota_{Q^{\prime}} \,\circ\, \delta \,\circ\, \pi_{K^{\prime\prime}}$\,
	\vskip 0.01cm
	\noindent
	Proof of Claim 1:\;\;
	The monomorphicity hypothesis on \,$\mu_{B}$\, and the exactness hypothesis at \,$B$\,
	together imply that
	\,$\mu_{B}$\, is a kernel of \,$\varepsilon_{B}$,\,
	by Proposition \ref{MonomorphicFirstFactorInExactSequenceIsKernelOfSecondFactor}.
	On the other hand, observe that
	\begin{eqnarray*}
	\varepsilon_{B} \,\circ\, (\,f \,\circ\, \pi_{A}\,)
	& = &
		(\,\varepsilon_{B} \,\circ\, f\,) \,\circ\, \pi_{A}
	\;\; = \;\;
		f^{\prime\prime} \,\circ\, \varepsilon_{A} \,\circ\, \pi_{A}
	\;\; = \;\;
		f^{\prime\prime} \,\circ\, \kappa^{\prime\prime} \,\circ\, \pi_{\kappa^{\prime\prime}}
	\;\; = \;\;
		0 \,\circ\, \pi_{\kappa^{\prime\prime}}
	\;\; = \;\;
		0
	\end{eqnarray*}
	Hence, the universal property of \,$\mu_{B}$\, as a kernel of \,$\varepsilon_{B}$\,
	implies the existence of a unique
	\,$\theta_{1} \in \Mor_{\mathfrak{A}}(\,A\,\sqcap_{A^{\prime\prime}} K^{\prime\prime}\,,\,B^{\prime}\,)$\,
	such that
	\,$f \,\circ\, \pi_{A} \,=\, \mu_{B} \,\circ\, \theta_{1}$.\,
	This proves Claim 1.

	\vskip 0.3cm
	\noindent
	\textbf{Claim 2:}\quad
	$(\,\pi^{\prime} \,\circ\, \theta_{1}\,) \,\circ\, \kappa(\pi_{K^{\prime\prime}}) \; = \; 0$
	\vskip 0.01cm
	\noindent
	Proof of Claim 2:\;\;
	First, recall from the proof of Claim 1 that
	\,$\mu_{B}$\, is a kernel of \,$\varepsilon_{B}$.\,
	Now, observe that:
	\begin{eqnarray*}
	\varepsilon_{B} \circ \left(\,f \overset{{\color{white}1}}{\circ} \kappa(\varepsilon_{A})\,\right)
	& = &
		\left(\,\varepsilon_{B} \overset{{\color{white}1}}{\circ} f \,\right) \circ\, \kappa(\varepsilon_{A})
	\;\; = \;\;
		\left(\,f^{\prime\prime}\overset{{\color{white}1}}{\circ}\varepsilon_{A}\,\right) \circ\, \kappa(\varepsilon_{A})
	\;\; = \;\;
		f^{\prime\prime}\,\circ\left(\,\varepsilon_{A}\overset{{\color{white}1}}{\circ}\kappa(\varepsilon_{A})\,\right)
	\;\; = \;\;
		f^{\prime\prime}\,\circ\,0 %\left(\,\overset{{\color{white}.}}{0}\,\right)
	\\
	& = &
		0
	\end{eqnarray*}
	Hence, the universal property of \,$\mu_{B}$\, as a kernel of \,$\varepsilon_{B}$\,
	implies the existence of a unique
	\,$\theta_{2} \in \Mor_{\mathfrak{A}}(\,K(\varepsilon_{A})\,,\,B^{\prime}\,)$\,
	such that
	\,$f \,\circ\, \kappa(\varepsilon_{A}) \,=\, \mu_{B} \,\circ\, \theta_{2}$.\,
	Now, observe that \,$\theta_{2}$\, furthermore satisfies the following three equalities:
	\begin{enumerate}
	\item
		$f^{\prime} \,=\, \theta_{2} \,\circ\, \widetilde{\mu_{A}}$
		\vskip 0.1cm
		Proof:\; Note that
		\begin{eqnarray*}
		\mu_{B} \,\circ\, f^{\prime}
		& = &
			f \,\circ\, \mu_{A}
		\;\; = \;\;
			f \,\circ\, \kappa(\varepsilon_{A}) \,\circ\, \widetilde{\mu_{A}}
		\;\; = \;\;
			\mu_{B} \,\circ\, \theta_{2} \,\circ\, \widetilde{\mu_{A}},
		\end{eqnarray*}
		which then implies
		\,$f^{\prime} \,=\, \theta_{2} \,\circ\, \widetilde{\mu_{A}}$,\,
		by the monomorphicity of \,$\mu_{B}$.
	\item
		$\theta_{1} \,\circ\, \kappa(\pi_{K^{\prime\prime}}) \,=\, \theta_{2} \,\circ\, \alpha$\,
		\vskip 0.1cm
		Proof:\; Note that
		\begin{eqnarray*}
		\mu_{B} \,\circ\, \theta_{1} \,\circ\, \kappa(\pi_{K^{\prime\prime}})
		& = &
			f \,\circ\, \pi_{A} \,\circ\, \kappa(\pi_{K^{\prime\prime}})\,,
			\quad\textnormal{by Claim 1}
		\\
		& = &
			f \,\circ\, \kappa(\varepsilon_{A}) \,\circ\, \alpha\,,
			\quad\textnormal{by
				Lemma \ref{FiberProductKernalMonomorphism};
				see also Lemma 8.3.11(a)(i), p.180, \cite{kashiwara2005categories}
				}
		\\
		& = &
			\mu_{B} \,\circ\, \theta_{2} \,\circ\, \alpha\,,
			\quad\textnormal{by definition of \,$\theta_{2}$}
		\end{eqnarray*}
		which implies
		\,$\theta_{1} \,\circ\, \kappa(\pi_{K^{\prime\prime}}) \,=\, \theta_{2} \,\circ\, \alpha$,\,
		by the monomorphicity of \,$\mu_{B}$.\,
	\item
		$\pi^{\prime} \,\circ\, \theta_{2} \,=\, 0$
		\vskip 0.1cm
		Proof:\; Note that
		\begin{eqnarray*}
		\pi^{\prime} \,\circ\, \theta_{2} \,\circ\, \widetilde{\mu_{A}}
		& = &
			\pi^{\prime} \,\circ\, f^{\prime}\,,
			\quad\textnormal{by (a)}
		\\
		& = &
			0\,,
			\quad\textnormal{since \,$\pi^{\prime}$\, is a cokernel of \,$f^{\prime}$}
		\end{eqnarray*}
		which then implies
		\,$\pi^{\prime} \,\circ\, \theta_{2} \,=\, 0$,\,
		by epimorphicity of \,$\widetilde{\mu_{A}}$.\,
	\end{enumerate}
	Lastly, observe now:
	\begin{eqnarray*}
	\left(\,
		\pi^{\prime} \overset{{\color{white}1}}{\circ} \theta_{1}
		\,\right)
		\circ\, \kappa(\pi_{K^{\prime\prime}})
	& = &
		\pi^{\prime} \,\circ \left(\,
			\theta_{1} \overset{{\color{white}1}}{\circ} \kappa(\pi_{K^{\prime\prime}})
			\,\right)
	\;\; = \;\;
		\pi^{\prime} \,\circ \left(\,
			\theta_{2} \overset{{\color{white}1}}{\circ} \alpha
			\,\right)
	\;\; = \;\;
		\left(\,
			\pi^{\prime} \overset{{\color{white}1}}{\circ} \theta_{2}
			\,\right)
			\circ\, \alpha
	\;\; = \;\;
		0 \,\circ\, \alpha
	\;\; = \;\;
		0
	\end{eqnarray*}
	This completes the proof of Claim 2.

	\vskip 0.3cm
	\noindent
	\textbf{Claim 3:}\quad
	There exists a unique
	\,$\delta \in \Mor_{\mathfrak{A}}(\,K^{\prime\prime}\,,\,Q^{\prime}\,)$\,
	such that
	\,$\pi^{\prime} \,\circ\, \theta_{1} \;=\; \delta \,\circ\, \pi_{K^{\prime\prime}}$\,
	\vskip 0.01cm
	\noindent
	Proof of Claim 3:\;\;
	Since \,$\mathfrak{A}$\, is an abelian category, every epimorphism in \,$\mathfrak{A}$\,
	is a cokernel.
	Since \,$\pi_{K^{\prime\prime}}$\, is an epimorphism (established above), it is a cokernel.
	Thus, by Lemma \ref{AKernelsTheKernelOfItsOwnCokernel}(ii),
	\,$\pi_{K^{\prime\prime}}$\, is a cokernel of its kernel
	\,$\kappa(\pi_{K^{\prime\prime}})$.\,
	Therefore, Claim 2 and the universal property of \,$\pi_{K^{\prime\prime}}$\, as a cokernel of
	\,$\kappa(\pi_{K^{\prime\prime}})$\,
	together imply that there exists a unique
	\,$\delta \in \Mor_{\mathfrak{A}}(\,K^{\prime\prime}\,,\,Q^{\prime}\,)$\,
	such that
	\,$\pi^{\prime} \,\circ\, \theta_{1} \;=\; \delta \,\circ\, \pi_{K^{\prime\prime}}$.\,
	\begin{center}
	\begin{tikzcd}
	& & & {\color{red}K^{\prime\prime}}
		\arrow[dd, thick, dashed, "\;\exists !\,\delta", blue]
	\\
	K(\pi_{K^{\prime\prime}}) 
		\arrow[rr, hook, "{\color{white}.}\kappa(\pi_{K^{\prime\prime}})"]
		\arrow[rrru, gray, bend  left = 15, "0"]
		\arrow[rrrd, blue, bend right = 15, "0", swap]
		& &
		{\color{red}A\,\sqcap_{A^{\prime\prime}}\! K^{\prime\prime}}
		\arrow[rd, blue, thick, "\pi^{\prime}\,\circ\,\theta_{1}", swap]
		\arrow[ur, red, "\pi_{K^{\prime\prime}}"]
		&
	\\
	& & & {\color{blue}Q^{\prime}}
	\end{tikzcd}
	\end{center}
	This completes the proof of Claim 3.

	\vskip 0.5cm
	\noindent
	This construction of the (snake or connection) morphism \,$\delta$\, is now complete.
	As for the exactness statement, in the proof of (i), we have already established the exactness of
	\begin{tikzcd}
	K^{\prime}
		\arrow[r, "\mu_{K}"]
	&
	K
		\arrow[r, "\varepsilon_{K}"]
	&
	K^{\prime\prime}
	\end{tikzcd}
	and
	\begin{tikzcd}
	Q^{\prime}
		\arrow[r, "\mu_{Q}"]
	&
	Q
		\arrow[r, "\varepsilon_{Q}"]
	&
	Q^{\prime\prime}
	\end{tikzcd}\!\!.\,
	Hence, it remains only to establish the exactness of
	\begin{tikzcd}
	K
		\arrow[r, "\varepsilon_{K}"]
	&
	K^{\prime\prime}
		\arrow[r, "\delta"]
	&
	Q^{\prime}
	\end{tikzcd}
	and
	\begin{tikzcd}
	K^{\prime\prime}
		\arrow[r, "\delta"]
	&
	Q^{\prime}
		\arrow[r, "\mu_{Q}"]
	&
	Q
	\end{tikzcd}\!\!.\,

	\vskip 0.5cm
	\noindent
	\underline{Exactness of{\color{white}.q}$
		K
		\,\overset{\varepsilon_{K}}{\longrightarrow}\,
		K^{\prime\prime}
		\,\overset{\delta}{\longrightarrow}\,
		Q^{\prime}
		$}
	\vskip -0.15cm
	By Proposition \ref{NecessarySufficientConditionForExactnessInAbelianCategories},
	in order to establish the exactness of the composition
	\begin{tikzcd}
	K
		\arrow[r, "\varepsilon_{K}"]
	&
	K^{\prime\prime}
		\arrow[r, "\delta"]
	&
	Q^{\prime}
	\end{tikzcd}\!\!,
	it suffices to show that, given an arbitrary morphism
	\,$h \,\in\, \Mor_{\mathfrak{A}}(X,K^{\prime\prime})$\,
	such that
	\,$\delta \,\circ\, h \,=\, 0$,\,
	there exist
	a morphism \,$h^{\prime} \,\in\, \Mor_{\mathfrak{A}}(X^{\prime},K)$\, and
	an epimorphism \,$\varepsilon^{\prime} \,\in\, \Mor_{\mathfrak{A}}(X^{\prime},X)$\,
	such that
	\,$\varepsilon_{K} \,\circ\, h^{\prime} \,=\, h \,\circ\, \varepsilon^{\prime}$,\,
	i.e., the following diagram commutes:
	\begin{center}
	\begin{tikzcd}
	{\color{red}X^{\prime}}
		\arrow[rr, red, thick, dashed, two heads, "\varepsilon^{\prime}"]
		\arrow[dd, red, thick, dashed, "h^{\prime}", swap]
	&&
	{\color{blue}X}
		\arrow[dd, blue, thick, "h"]
		\arrow[ddrr, gray, bend left = 20, "0"]
	\\ \\
	K
		\arrow[rr, black, thick, "\varepsilon_{K}"]
	&&
	K^{\prime\prime}
		\arrow[rr, black, thick, "\delta"]
	&&
	Q^{\prime}
	\end{tikzcd}
	\end{center}
	Let
	\,$Y \,:=\, X \,\sqcap_{K^{\prime\prime}}\!(\,A\,\sqcap_{A^{\prime\prime}}\!K^{\prime\prime}\,)$\,
	be the fiber product of \,$X$\, and
	\,$A\,\sqcap_{A^{\prime\prime}}\!K^{\prime\prime}$\,
	over \,$K^{\prime\prime}$.\,
	Consider the following extended diagram:
	\begin{center}
	\begin{tikzcd}
	&&
	{\color{blue}Y^{\prime}}
		\arrow[rr, blue, thick, two heads, "\beta"]
		\arrow[ddddll, blue, thick, bend right = 21, "\gamma", swap]
		\arrow[ddddrr, blue, thick, bend left  = 21, "\lambda"]
	&&
	{\color{blue}Y}
		\arrow[rr, blue, thick, two heads, "p_{1}"]
		\arrow[dd, blue, thick, "p_{2}"]
	&&
	{\color{blue}X}
		\arrow[dd, blue, thick, "h"]
	\\ \\
	&&{\color{gray}K(\pi_{K^{\prime\prime}})}
		\arrow[rr, gray, thick, hook, "\kappa(\pi_{K^{\prime\prime}})"]
		\arrow[dd, gray, thick, hook, "\alpha", swap]
	&&
	{\color{gray}A \,\sqcap_{A^{\prime\prime}}\! K^{\prime\prime}}
		\arrow[dd, gray, thick, hook, "\,\pi_{A}"]
		\arrow[rr, gray, thick, two heads, "\pi_{K^{\prime\prime}}"]
		\arrow[ddddll, gray, thick, dashed, bend right = 5, "\theta_{1}", swap]
	&&
	K^{\prime\prime}
		\arrow[dd, thick, hook, "\kappa^{\prime\prime}"]
		\arrow[ddddddllll, red, thick, dashed, bend left = 30, "\textnormal{\LARGE$\delta$}"]
	\\ \\
	A^{\prime}
		\arrow[ddrr, thick, "f^{\prime}", swap]
		\arrow[rr, gray, thick, two heads, "{\color{white}...}\overset{{\color{white}.}}{\widetilde{\mu_{A}}}", swap]
		\arrow[rrrr, thick, bend left = 30, "\quad{\color{white}....}\mu_{A}"]
	&&
	{\color{gray}I(\mu_{A})\,\cong\,K(\varepsilon_{A})}
		\arrow[rr, gray, thick, hook,   "\kappa(\varepsilon_{A})", swap]
		\arrow[dd, gray,  thick, dashed, "\theta_{2}", swap]
	&&
	A
		\arrow[dd, thick, "f"]
		\arrow[rr, thick, two heads, "\varepsilon_{A}"]
	&&
	A^{\prime\prime}
		\arrow[dd, thick, "f^{\prime\prime}"]
	\\ \\
	&&B^{\prime}
		\arrow[rr, thick, hook, swap, "\mu_{B}"]
		\arrow[dd, thick, two heads, "\pi^{\prime}", swap]
	&&
	B
		\arrow[rr, thick, "\varepsilon_{B}", swap]
		\arrow[dd, gray, thick, two heads, "\iota_{B}"]
	&&
	B^{\prime\prime}
		% \arrow[dd, "\pi^{\prime\prime}"]
	\\ \\
	&&Q^{\prime}
		\arrow[rr, gray, thick, hook, "\iota_{Q^{\prime}}", swap]
	&&
	{\color{gray}Q^{\prime} \sqcup_{B^{\prime}}\! B}
		\arrow[rr, gray, thick, two heads, "\pi(\iota_{Q^{\prime}}){\color{white}..}", swap]
	&&
	{\color{gray}Q(\iota_{Q^{\prime}})}
	\end{tikzcd}
	\end{center}
	\vskip 0.3cm
	\noindent
	\textbf{Claim 4:}\quad
	\,$\pi^{\prime} \,\circ\, (\,\theta_{1} \,\circ\, p_{2}\,) \,=\, 0$\,
	\vskip 0.01cm
	\noindent
	Proof of Claim 4:\;\;
	Simply observe
	\begin{eqnarray*}
	\pi^{\prime} \,\circ\, \theta_{1} \,\circ\, p_{2}
	& = &
		\delta \,\circ\, \pi_{K^{\prime\prime}} \,\circ\, p_{2}
	\;\; = \;\;
		\delta \,\circ\, h \,\circ\, p_{1}
	\;\; = \;\;
		0 \,\circ\, p_{1}
	\;\; = \;\;
		0\,,
	\end{eqnarray*}
	where the first equality holds by Claim 3,
	the second equality holds by the defining property of the fiber product, and
	the third equality holds by hypothesis ($\delta \circ h = 0$).
	This proves Claim 4.
	\vskip 0.3cm
	\noindent
	\textbf{Claim 5:}\quad
	There exist
	an epimorphism \,$\beta \in \Mor_{\mathfrak{A}}(Y^{\prime},Y)$\,
	and
	a morphism \,$\gamma \in \Mor_{\mathfrak{A}}(Y^{\prime},A^{\prime})$\,
	such that
	\begin{equation*}
	f^{\prime} \,\circ\, \gamma \,=\, (\,\theta_{1} \circ p_{2}\,) \,\circ\, \beta
	\end{equation*}
	\vskip 0.01cm
	\noindent
	Proof of Claim 5:\;\;
	Since \,$\pi^{\prime}$\, is a cokernel of \,$f^{\prime}$,\,
	the composition
	\begin{tikzcd}
	A^{\prime}
		\arrow[r, "f^{\prime}"]
	&
	B^{\prime}
		\arrow[r, "\pi^{\prime}"]
	&
	Q^{\prime}
	\end{tikzcd}
	is exact, by Corollary \ref{MorphismCokernelExactness}.
	By Claim 4 and Proposition \ref{NecessarySufficientConditionForExactnessInAbelianCategories},
	there exist
	an epimorphism \,$\beta \in \Mor_{\mathfrak{A}}(Y^{\prime},Y)$\,
	and
	a morphism \,$\gamma \in \Mor_{\mathfrak{A}}(Y^{\prime},A^{\prime})$\,
	such that
	\,$f^{\prime} \,\circ\, \gamma \,=\, (\,\theta_{1} \circ p_{2}\,) \,\circ\, \beta$.\,
	See the following diagram:
	\begin{center}
	\begin{tikzcd}
	{\color{red}Y^{\prime}}
		\arrow[rr, red, thick, dashed, two heads, "\beta"]
		\arrow[dd, red, thick, dashed, "\gamma", swap]
	&&
	{\color{blue}Y}
		\arrow[dd, blue, thick, "\,\theta_{1} \circ\,p_{2}"]
		\arrow[ddrr, gray, bend left = 30, "0"]
	&&
	\\ \\
	A^{\prime}
		\arrow[rr, thick, "f^{\prime}", swap]
	&&
	B^{\prime}
		\arrow[rr, thick, two heads, "\pi^{\prime}", swap]
	&&
	Q^{\prime}
	\end{tikzcd}
	\end{center}
	This proves Claim 5.
	\vskip 0.3cm
	\noindent
	\textbf{Claim 6:}\quad
	Define
	\,$\lambda \,:=\, \pi_{A} \,\circ\, p_{2} \,\circ\, \beta \,\in\, \Mor_{\mathfrak{A}}(Y^{\prime},A)$.\,
	Then, \,$\lambda$\, satisfies
	\,$f \,\circ\, (\,\lambda \,-\, \mu_{A}\circ\gamma\,)\,=\, 0$.\,
	Consequently, there exists a unique morphism
	\,$\theta_{3} \in \Mor_{\mathfrak{A}}(Y^{\prime},K)$\,
	such that
	\,$\lambda - \mu_{A}\circ\gamma \,=\, \kappa \,\circ\, \theta_{3}$.\,
	\vskip 0.01cm
	\noindent
	Proof of Claim 6:\;\;
	\begin{eqnarray*}
	f \,\circ\, \lambda
	& = &
		f \,\circ\, \pi_{A} \,\circ\, p_{2} \,\circ\, \beta\,,
		\quad\textnormal{by definition of \,$\lambda$}
	\\
	& = &
		\mu_{B} \,\circ\, \theta_{1} \,\circ\, p_{2} \,\circ\, \beta\,,
		\quad\textnormal{by Claim 1}
	\\
	& = &
		\mu_{B} \,\circ\, f^{\prime} \,\circ\, \gamma\,,
		\quad\textnormal{by Claim 5}
	\\
	& = &
		f \,\circ\, \mu_{A} \,\circ\, \gamma\,,
		\quad\textnormal{by the (commutativity) hypothesis that \,$\mu_{B}\circ f^{\prime}\,=\,f\circ\mu_{A}$}
	\end{eqnarray*}
	The existence and uniqueness of \,$\theta_{3}$\,
	now follows from the universial property of \,$\kappa$\,
	as a kernel of \,$f$.\, See the following diagram:
	\begin{center}
	\begin{tikzcd}
	{\color{blue}Y^{\prime}}
		\arrow[dd,   blue, thick, dashed, "\exists !\,\theta_{3}{\color{white}.}", swap]
		\arrow[dr,   blue, thick, "\lambda \,-\, \mu_{A} \circ \gamma"]
		\arrow[drrr, blue, bend left = 25, "0"] & & &
	\\
	& {\color{red}A}
		\arrow[rr, swap, "f{\color{white}...}"] & & B
	\\
	{\color{red}K}
		\arrow[ur, red, "\kappa", swap]
		\arrow[urrr, gray, bend right = 25, swap, "0"]
	\end{tikzcd}
	\end{center}
	This completes the proof of Claim 6.
	\vskip 0.3cm
	\noindent
	\textbf{Claim 7:}\quad
	\,$\varepsilon_{K} \,\circ\, \theta_{3} \,=\, h \,\circ\, (\,p_{1} \,\circ\, \beta\,)$\,
	\vskip 0.01cm
	\noindent
	Proof of Claim 7:\;\;
	Observe that
	\begin{eqnarray*}
	\kappa^{\prime\prime} \,\circ\, \varepsilon_{K} \,\circ\, \theta_{3}
	& = &
		\varepsilon_{A} \,\circ\, \kappa \,\circ\, \theta_{3}\,,
		\;\;
		\textnormal{since \,$
			\kappa^{\prime\prime}\circ\varepsilon_{K}
			\, = \, \varepsilon_{A}\circ\kappa
			$,\, by construction of \,$\varepsilon_{K}$\, in proof of (i)}
	\\
	& = &
		\varepsilon_{A} \,\circ\, (\,\lambda \,-\, \mu_{A}\circ\gamma\,)\,,
		\;\;
		\textnormal{by Claim 6}
	\\
	& = &
		\varepsilon_{A} \,\circ\, \lambda
		\,-\,
		\varepsilon_{A} \,\circ\, \mu_{A} \,\circ\, \gamma
	\\
	& = &
		\varepsilon_{A} \,\circ\, \lambda
		\,-\,
		0 \,\circ\, \gamma\,,
		\;\;
		\textnormal{since
			\,$\varepsilon_{A}\circ\mu_{A} = 0$,\,
			by exactness of
			\,$A^{\prime} \overset{\mu_{A}}{\longrightarrow} A \overset{\varepsilon_{A}}{\longrightarrow} A^{\prime\prime}$
			}
	\\
	& = &
		\varepsilon_{A} \,\circ\, \pi_{A} \,\circ\, p_{2} \,\circ\, \beta\,,
		\quad\textnormal{by definition of \,$\lambda \,:=\, \pi_{A} \,\circ\, p_{2} \,\circ\, \beta$}
	\\
	& = &
		\kappa^{\prime\prime} \,\circ\, \pi_{K^{\prime\prime}} \,\circ\, p_{2} \,\circ\, \beta\,,
		\;\;
		\textnormal{
			since \,$\varepsilon_{A}\circ\pi_{A} \,=\, \kappa^{\prime\prime}\circ\pi_{K^{\prime\prime}}$,\,
			by construction of \,$A\,\sqcap_{A^{\prime\prime}}K^{\prime\prime}$
			}
	\\
	& = &
		\kappa^{\prime\prime} \,\circ\, h \,\circ\, p_{1} \,\circ\, \beta\,,
		\;\;
		\textnormal{
			since \,$\pi_{K^{\prime\prime}} \,\circ\, p_{2} \,=\, h \,\circ\, p_{1}$,\,
			by construction of
			\,$Y \,:=\, X\sqcap_{K^{\prime\prime}}\!(A\sqcap_{A^{\prime\prime}}\!K^{\prime\prime})$
			}
	\end{eqnarray*}
	The monomorphicity of \,$\kappa^{\prime\prime}$\, now implies that
	\,$\varepsilon_{K} \,\circ\, \theta_{3} \,=\, h \,\circ\, p_{1} \,\circ\, \beta$,\,
	which completes the proof of Claim 7.
	\vskip 0.3cm
	\noindent
	We remark here that the epimorphicity of
	\,$p_{1} \circ \beta$\,
	trivially follows from the epimorphicity of
	\,$p_{1}$\, and \,$\beta$.\,
	\vskip 0.3cm
	\noindent
	In summary, we have established that, given an arbitrary morphism
	\,$h \in \Mor_{\mathfrak{A}}(X,K^{\prime\prime})$\,
	satisfying
	\,$\delta \,\circ\, h \,=\, 0$,\,
	we have the following commutativity diagram:
	\begin{center}
	\begin{tikzcd}
	{\color{red}Y^{\prime}}
		\arrow[rr,   red,  thick, dashed, two heads, "p_{1}\,\circ\,\beta{\color{white}...}"]
		\arrow[dd,   red,  thick, dashed, "\theta_{3}\,", swap]
		\arrow[dddd, gray, thick, bend right = 80, "\lambda\,-\,\mu_{A}\,\circ\,\gamma\;", swap]
	&&
	{\color{blue}X}
		\arrow[dd, blue, thick, "h"]
		\arrow[ddrr, gray, bend left = 20, "0"]
	\\ \\
	K
		\arrow[rr, black, thick, "\varepsilon_{K}"]
		\arrow[dd, gray,  thick, hook, "\kappa"]
	&&
	K^{\prime\prime}
		\arrow[rr, black, thick, "\delta"]
		\arrow[dd, gray,  thick, hook, "\kappa^{\prime\prime}"]
	&&
	Q^{\prime}
	\\ \\
	{\color{gray}A}
		\arrow[rr, gray, thick, two heads, "\varepsilon_{A}"]
	&&
	{\color{gray}A^{\prime\prime}}
	\end{tikzcd}
	\end{center}
	The exactness of
	\begin{tikzcd}
	K
		\arrow[r, "\varepsilon_{K}"]
	&
	K^{\prime\prime}
		\arrow[r, "\delta"]
	&
	Q^{\prime}
	\end{tikzcd}
	now follows by Proposition \ref{NecessarySufficientConditionForExactnessInAbelianCategories}.
	This completes the proof of the exactness of
	\begin{tikzcd}
	K
		\arrow[r, "\varepsilon_{K}"]
	&
	K^{\prime\prime}
		\arrow[r, "\delta"]
	&
	Q^{\prime}
	\end{tikzcd}\!\!.

	\vskip 0.5cm
	\noindent
	\underline{Exactness of{\color{white}.q}$
		K^{\prime\prime}
		\,\overset{\delta}{\longrightarrow}\,
		Q^{\prime}
		\,\overset{\mu_{Q}}{\longrightarrow}\,
		Q
		$}
	\vskip 0.01cm
	\noindent
	This proof can be established by reversing the arrows in the preceding argument.
	We omit this proof.

	\vskip 0.5cm
	\noindent
	The proof of the Proposition is now complete.
	\qed

% \vskip 0.5cm
% \item
% 	First, we construct the morphism \,$\delta : K^{\prime\prime} \longrightarrow Q^{\prime}$,\, as follows:
% 	\begin{itemize}
% 	\item
% 		Since \,$\mathfrak{A}$\, is an (abelian) category with enough projectives,
% 		there exists an epimorphism
% 		\,$\varepsilon_{P} : P \longrightarrow K^{\prime\prime}$,\,
% 		with \,$P \in \Obj(\mathfrak{A})$\, being a projective object.
% 	\item
% 		Since \,$P$\, is a projective object and 
% 		\,$\varepsilon_{A} : A \longrightarrow A^{\prime\prime}$\,
% 		is an epimorphism, there exists
% 		\,$f_{1} : P \longrightarrow A$\,
% 		such that
% 		\,$\kappa^{\prime\prime} \circ \varepsilon_{P} \,=\, \varepsilon_{A} \circ f_{1}$\,
% 	\item
% 		Since \,$\mu_{B}$\, is a monomorphism
% 		and the composition
% 		\,$B^{\prime} \overset{\mu_{B}}{\longrightarrow} B \overset{\varepsilon_{B}}{\longrightarrow} B^{\prime\prime}$\,
% 		is exact, it follows
% 		by Proposition \ref{MonomorphicFirstFactorInExactSequenceIsKernelOfSecondFactor}
% 		that
% 		\,$\mu_{B}$\, is a kernel of \,$\varepsilon_{B}$.\,
% 		On the other hand, observe that
% 		\,$\varepsilon_{B} \circ (\,f \circ f_{1}\,)$
% 		\,$=$\, $(\,\varepsilon_{B} \circ f\,) \circ f_{1}$
% 		\,$=$\, $(\,f^{\prime\prime} \circ \varepsilon_{A}\,) \circ f_{1}$
% 		\,$=$\, $f^{\prime\prime} \circ (\,\varepsilon_{A} \circ f_{1}\,)$
% 		\,$=$\, $f^{\prime\prime} \circ (\,\kappa^{\prime\prime} \circ \varepsilon_{P}\,)$
% 		\,$=$\, $(\,f^{\prime\prime} \circ \kappa^{\prime\prime}\,) \circ \varepsilon_{P}$
% 		\,$=$\, $0 \circ \varepsilon_{P}$
% 		\,$=$\, $0$.\,
% 		Hence, by the universal property of
% 		\,$\mu_{B}$\, as a kernel of \,$\varepsilon_{B}$,\,
% 		we see that there exists a unique
% 		\,$f_{2} : P \longrightarrow B^{\prime}$\,
% 		such that
% 		\,$f \circ f_{1} = \mu_{B} \circ f_{2}$.\,
% 	\item
% 		Next, we simply define
% 		\,$f_{3} \,:=\, \pi^{\prime} \circ f_{2}$.\,
% 	\end{itemize}
% %	\begin{center}
% %	\begin{tikzcd}
% %	K^{\prime}
% %		\arrow[dd,swap, "\kappa^{\prime}"]
% %		\arrow[rr, thick, dashed, "\mu_{K}", red]
% %	&&
% %	K
% %		\arrow[dd, "\kappa"]
% %		\arrow[rr, thick, dashed, "\varepsilon_{K}", red]
% %	&&
% %	K^{\prime\prime}
% %		\arrow[dd, "\kappa^{\prime\prime}"]
% %	\\ \\
% %	A^{\prime}
% %		\arrow[dd, swap, "f^{\prime}"]
% %		\arrow[rr, "\mu_{A}"]
% %	&&
% %	A
% %		\arrow[dd, "f"]
% %		\arrow[rr, thick, two heads, "\varepsilon_{A}"]
% %	&&
% %	A^{\prime\prime}
% %		\arrow[dd, "f^{\prime\prime}"]
% %	\\ \\
% %	B^{\prime}
% %		\arrow[rr, thick, hook, swap, "\mu_{B}"]
% %		\arrow[dd, swap, "\pi^{\prime}"]
% %	&&
% %	B
% %		\arrow[rr, swap, "\varepsilon_{B}"]
% %		\arrow[dd, "\pi"]
% %	&&
% %	B^{\prime\prime}
% %		\arrow[dd, "\pi^{\prime\prime}"]
% %	\\ \\
% %	Q^{\prime}
% %		\arrow[rr, thick, dashed, swap, "\mu_{Q}", red]
% %	&&
% %	Q
% %		\arrow[rr, thick, dashed, swap, "\varepsilon_{Q}", red]
% %	&&
% %	Q^{\prime\prime}
% %	\end{tikzcd}
% %	\end{center}
% %%%%%%%%%%%%%%%%%%%%%%%%%%%%%%%%%%%%%
% %%%%%%%%%%%%%%%%%%%%%%%%%%%%%%%%%%%%%
% %%%%%%%%%%%%%%%%%%%%%%%%%%%%%%%%%%%%%
% %	\begin{center}
% %	\vskip -2.5cm
% %	\begin{tikzcd}
% %	&&
% %	&&&
% %	{K_{P}}
% %		\arrow[d, "\kappa_{P}"]
% %	&
% %	\\
% %	&&
% %	&
% %	{\color{white}K_{P}}
% %	&
% %	&
% %	{\color{blue}P}
% %		\arrow[dr, two heads, "\varepsilon_{P}", blue]
% %		\arrow[dddl, "f_{1}", red]
% %		\arrow[dddddlll, swap, "f_{2}", red]
% %		\arrow[dddddddlll, bend right = 110, "f_{3}", red]
% %	&
% %	\\
% %	&&
% %	K^{\prime}
% %		\arrow[dd, swap, "\kappa^{\prime}"]
% %		\arrow[rr, thick, dashed, "\mu_{K}"]
% %	&&
% %	K
% %		\arrow[dd, "\kappa"]
% %		\arrow[rr, thick, dashed, "\varepsilon_{K}"]
% %	&&
% %	K^{\prime\prime}
% %		\arrow[dd, "\kappa^{\prime\prime}"]
% %		\arrow[ddddddllll, thick, dashed, bend left = 29, "\textnormal{\Large$\delta$}", red]
% %	\\ \\
% %	&&
% %	A^{\prime}
% %		\arrow[dd, swap, "f^{\prime}"]
% %		\arrow[rr, "\mu_{A}"]
% %	&&
% %	A
% %		\arrow[dd, "f"]
% %		\arrow[rr, thick, two heads, "\varepsilon_{A}"]
% %	&&
% %	A^{\prime\prime}
% %		\arrow[dd, "f^{\prime\prime}"]
% %	\\ \\
% %	&&
% %	B^{\prime}
% %		\arrow[rr, thick, hook, swap, "\mu_{B}"]
% %		\arrow[dd, swap, "\pi^{\prime}"]
% %	&&
% %	B
% %		\arrow[rr, swap, "\varepsilon_{B}"]
% %		\arrow[dd, "\pi"]
% %	&&
% %	B^{\prime\prime}
% %		\arrow[dd, "\pi^{\prime\prime}"]
% %	\\ \\
% %	&&
% %	Q^{\prime}
% %		\arrow[rr, thick, dashed, swap, "\mu_{Q}"]
% %	&&
% %	Q
% %		\arrow[rr, thick, dashed, swap, "\varepsilon_{Q}"]
% %	&&
% %	Q^{\prime\prime}
% %	\end{tikzcd}
% %	{\color{white}?????????????????????}
% %	\end{center}
% 	Consider the following diagram:
% 	\begin{center}
% 	\vskip -1.5cm
% 	\begin{tikzcd}
% 	&&
% 	&&&
% 	K_{P}
% 		\arrow[d, "\kappa_{P}"]
% 		\arrow[dddddll, dashed, bend right = 30, swap, pos=0.30, "f_{1}^{\prime}", red]		
% 	&
% 	\\
% 	&&
% 	&
% 	{\color{white}K_{P}}
% 	&
% 	&
% 	P
% 		\arrow[dr, two heads, "\varepsilon_{P}"]
% 		\arrow[dddl, pos=0.60, "f_{1}", blue]
% 		\arrow[dddddlll, bend right = 35, swap, pos=0.55, "f_{2}", blue]
% 		\arrow[dddddddlll, bend right = 110, swap, pos=0.75, "f_{3}\,", blue]
% 	&
% 	\\
% 	&&
% 	K^{\prime}
% 		\arrow[dd, swap, "\kappa^{\prime}"]
% 		\arrow[rr, thick, dashed, "\mu_{K}"]
% 	&&
% 	K
% 		\arrow[dd, "\kappa"]
% 		\arrow[rr, thick, dashed, "\varepsilon_{K}"]
% 	&&
% 	K^{\prime\prime}
% 		\arrow[dd, "\kappa^{\prime\prime}"]
% 		\arrow[ddddddllll, thick, dashed, bend left = 29, pos=0.35, "\textnormal{\Large$\delta$}", blue]
% 	\\ \\
% 	{\color{red}K_{0}}
% 		\arrow[rr, "\kappa(\,\widetilde{\mu_{A}}\,)", red]
% 	&&
% 	A^{\prime}
% 		\arrow[dd, swap, "f^{\prime}"]
% 		\arrow[rr, "\mu_{A}"]
% 		\arrow[dr, two heads, pos=0.6, "\widetilde{\mu_{A}}", red]
% 	&&
% 	A
% 		\arrow[dd, "f"]
% 		\arrow[rr, thick, two heads, "\varepsilon_{A}"]
% 	&&
% 	A^{\prime\prime}
% 		\arrow[dd, "f^{\prime\prime}"]
% 	\\
% 	&&
% 	&
% 	{\color{red}I_{\mu_{A}}}
% 		\arrow[ur, hook, swap, pos=0.35, "\!\!\iota(\mu_{A}){\color{white}}", red]		
% 		\arrow[dl, dashed, pos=0.35, "f_{2}^{\prime}", red]		
% 	&&&
% 	\\
% 	&&
% 	B^{\prime}
% 		\arrow[rr, thick, hook, swap, "\mu_{B}"]
% 		\arrow[dd, swap, "\pi^{\prime}"]
% 	&&
% 	B
% 		\arrow[rr, swap, "\varepsilon_{B}"]
% 		\arrow[dd, "\pi"]
% 	&&
% 	B^{\prime\prime}
% 		\arrow[dd, "\pi^{\prime\prime}"]
% 	\\ \\
% 	&&
% 	Q^{\prime}
% 		\arrow[rr, thick, dashed, swap, "\mu_{Q}"]
% 	&&
% 	Q
% 		\arrow[rr, thick, dashed, swap, "\varepsilon_{Q}"]
% 	&&
% 	Q^{\prime\prime}
% 	\end{tikzcd}
% 	{\color{white}?????????????????????}
% 	\end{center}
% 	Here,
% 	\,$K_{P} \overset{\kappa_{P}}{\longrightarrow}$ P\,
% 	is a kernel of
% 	\,$P \overset{\varepsilon_{P}}{\longrightarrow} K^{\prime\prime}$.\,
% 	Also, by definition, the monomorphism
% 	\,$\iota(\mu_{A}) : I_{\mu_{A}} \longrightarrow A$\,
% 	is an image of
% 	\,$\mu_{A}$.\,
% 	By Proposition \ref{FactorizationIntoImageCoimage}(ii),
% 	we may take \,$\iota(\mu_{A})$\, to be any kernel of any cokernel of \,$\mu_{A}$.\,
% 	By Lemma \ref{fTildeIsEpimorphism}, the accompanying morphism
% 	\,$\widetilde{\mu_{A}} : A^{\prime} \longrightarrow I_{\mu_{A}}$\,
% 	is an epimorphism.


% 	\vskip 0.3cm
% 	\textbf{Claim 1:}\;
% 	\,$\varepsilon_{A} \,\circ\, \iota(\mu_{A}) \,=\, 0$.\,
% 	\vskip 0.01cm
% 	Proof of Claim 1:\; Simply note that
% 	\,$0 \,=\, \varepsilon_{A} \,\circ\, \mu_{A} \,=\,\varepsilon_{A} \,\circ\, \iota(\mu_{A}) \,\circ\, \widetilde{\mu_{A}}$,\,
% 	which implies
% 	\,$\varepsilon_{A} \,\circ\, \iota(\mu_{A}) = 0$,\,
% 	since \,$\widetilde{\mu_{A}}$\, is an epimorphism and can be cancelled on the right.
% 	This completes the proof of Claim 1.	


% 	\vskip 0.3cm
% 	\textbf{Claim 2:}\;
% 	The composition
% 	\,$I_{\mu_{A}} \overset{\iota(\mu_{A})}{\longrightarrow} A \overset{\varepsilon_{A}}{\longrightarrow} A^{\prime\prime}$\,
% 	is exact.
% 	%\;\,$f_{3} \circ \kappa_{P} \,=\, 0$.\,
% 	\vskip 0.01cm
% 	Proof of Claim 2:\;
% 	This will follow from the exactness of
% 	\,$A^{\prime} \overset{\mu_{A}}{\longrightarrow} A \overset{\varepsilon_{A}}{\longrightarrow} A^{\prime\prime}$.\,
% 	Recall from Definition \ref{defnExactness} that, in general, the exactness of the composition
% 	\,$A \overset{f}{\longrightarrow} B \overset{g}{\longrightarrow} C$\,
% 	means that the unique morphism \,$\theta$\, in the diagram below is an isomorphism:
% 	\begin{center}
% 	\begin{tikzcd}
% 	&& && Q_{f}
% 		\arrow[dd, dashed, "\psi"]
% 	\\ \\
% 	A
% 		\arrow[rr, "f"]
% 		\arrow[dd, dashed, two heads, swap, "\varepsilon_{f}"]
% 		\arrow[rrrr, bend left = 30, "0", gray]
% 	&&
% 	B
% 		\arrow[rr, "g"]
% 		\arrow[uurr, two heads, "\pi_{f}"]
% 	&&
% 	C
% 	\\ \\
% 	I_{f}
% 		\arrow[uurr, hook, "\kappa(\pi_{f})"]
% 		\arrow[rr, thick, dashed, swap, "\exists !\,\theta", red]
% 	&&
% 	K_{g}
% 		\arrow[uu, hook, swap, "\kappa_{g}"]
% 		%\arrow[rr, thick, two heads, swap, "\pi_{\theta}", red]
% 	&&
% 	%Q_{\theta}
% 	\end{tikzcd}
% 	\end{center}
% 	Since \,$\mathfrak{A}$\, is an abelian category,
% 	the morphism \,$\kappa(\pi_{f})$\, -- kernel of cokernel of \,$f$ --
% 	is an image of \,$f$,\,
% 	while the morphism \,$\varepsilon_{f}$\, is a coimage of \,$f$.\,
% 	In other words, since \,$\mathfrak{A}$\, is an abelian category,
% 	the defining diagram of exactness of
% 	\,$A \overset{f}{\longrightarrow} B \overset{g}{\longrightarrow} C$\,
% 	is equivalent to:
% 	\begin{center}
% 	\begin{tikzcd}
% 	&& && Q_{f}
% 		\arrow[dd, dashed, "\psi"]
% 	\\ \\
% 	A
% 		\arrow[rr, "f"]
% 		\arrow[dd, dashed, two heads, swap, "\widetilde{f}\;"]
% 		\arrow[rrrr, bend left = 30, "0", gray]
% 	&&
% 	B
% 		\arrow[rr, "g"]
% 		\arrow[uurr, two heads, "\pi_{f}"]
% 	&&
% 	C
% 	\\ \\
% 	I_{f}
% 		\arrow[uurr, hook, "\iota(f)"]
% 		\arrow[rr, thick, dashed, swap, "\exists !\,\theta", red]
% 	&&
% 	K_{g}
% 		\arrow[uu, hook, swap, "\kappa_{g}"]
% 		%\arrow[rr, thick, two heads, swap, "\pi_{\theta}", red]
% 	&&
% 	%Q_{\theta}
% 	\end{tikzcd}
% 	\end{center}
% 	Applying this alternative but equivalent exactness-defining diagram to 
% 	\,$A^{\prime} \overset{\mu_{A}}{\longrightarrow} A \overset{\varepsilon_{A}}{\longrightarrow} A^{\prime\prime}$\,
% 	gives:	
% 	\begin{center}
% 	\begin{tikzcd}
% 	&& && Q_{\mu_{A}}
% 		\arrow[dd, dashed, "\psi"]
% 	\\ \\
% 	A^{\prime}
% 		\arrow[rr, "\mu_{A}"]
% 		\arrow[dd, dashed, two heads, swap, "\widetilde{\mu_{A}}{\color{white}..}", blue]
% 		\arrow[rrrr, bend left = 30, "0", gray]
% 	&&
% 	A
% 		\arrow[rr, "\varepsilon_{A}"]
% 		\arrow[uurr, two heads, "\pi(\mu_{A})\!\!\!"]
% 	&&
% 	A^{\prime\prime}
% 	\\ \\
% 	I_{\mu_{A}}
% 		\arrow[uurr, hook, "\iota(\mu_{A})\!\!"]
% 		\arrow[rr, thick, dashed, swap, "\exists !\,\theta_{0}", red]
% 	&&
% 	K_{\varepsilon_{A}}
% 		\arrow[uu, hook, swap, "\;\kappa(\varepsilon_{A})"]
% 		%\arrow[rr, thick, two heads, swap, "\pi_{\theta}", red]
% 	&&
% 	%Q_{\theta}
% 	\end{tikzcd}
% 	\end{center}
% 	The exactness of
% 	\,$A^{\prime} \overset{\mu_{A}}{\longrightarrow} A \overset{\varepsilon_{A}}{\longrightarrow} A^{\prime\prime}$\,
% 	means that the unique morphism \,$\theta_{0}$\, in the above diagram is an isomorphism.
% 	Next, we consider the composition
% 	\,$I_{\mu_{A}} \overset{\iota(\mu_{A})}{\longrightarrow} A \overset{\varepsilon_{A}}{\longrightarrow} A^{\prime\prime}$.\,
% 	Noting that \,$\iota(\mu_{A})$\, is a monomorphism, and hence it is an image of itself,
% 	and the identity morphism \,$\textnormal{id}_{I_{\mu_{A}}}$\, is an (accompanying) coimage of \,$\iota(\mu_{A})$,\,
% 	we see that the exactness of
% 	\,$I_{\mu_{A}} \overset{\iota(\mu_{A})}{\longrightarrow} A \overset{\varepsilon_{A}}{\longrightarrow} A^{\prime\prime}$\,
% 	is precisely the property that the unique morphism \,$\theta_{1}$\, in the following diagram is an isomorphism:
% 	\begin{center}
% 	\begin{tikzcd}
% 	&& && Q_{\mu_{A}}
% 		\arrow[dd, dashed, "\psi"]
% 	\\ \\
% 	I_{\mu_{A}} %A^{\prime}
% 		\arrow[rr, hook, "\iota(\mu_{A})"]
% 		%\arrow[dd, dashed, two heads, swap, "\widetilde{\mu_{A}}{\color{white}..}"]
% 		%\arrow[dd, dashed, two heads, swap, "\varepsilon_{\mu_{A}}{\color{white}..}", blue]
% 		\arrow[dd, equal, swap, "\textnormal{id}\;"]
% 		\arrow[rrrr, bend left = 30, "0", gray]
% 	&&
% 	A
% 		\arrow[rr, "\varepsilon_{A}"]
% 		\arrow[uurr, two heads, "\pi(\mu_{A})\!\!\!"]
% 	&&
% 	A^{\prime\prime}
% 	\\ \\
% 	I_{\mu_{A}}
% 		\arrow[uurr, hook, "\iota(\mu_{A})\!\!"]
% 		\arrow[rr, thick, dashed, swap, "\exists !\,\theta_{1}", red]
% 	&&
% 	K_{\mu_{A}}
% 		\arrow[uu, hook, swap, "\;\kappa(\varepsilon_{A})"]
% 		%\arrow[rr, thick, two heads, swap, "\pi_{\theta}", red]
% 	&&
% 	%Q_{\theta}
% 	\end{tikzcd}
% 	\end{center}
% 	Lastly, the fact that \,$\theta_{1}$\, must be an isomorphism follows from the uniqueness of
% 	\,$\theta_{0}$\, and \,$\theta_{1}$\, in their respective diagrams.
% 	Indeed, comparing the two immediately preceding diagrams, and noting the uniqueness of
% 	\,$\theta_{0}$\, and \,$\theta_{1}$\, in their respective diagrams,
% 	we see that we in fact must have \,$\theta_{0} = \theta_{1}$;\,
% 	in particular, we therefore have that \,$\theta_{1} \,=\, \theta_{0}$\, is an isomorphism
% 	(since \,$\theta_{0}$\, is an isomorphism, by the exactness hypothesis on
% 	\,$A^{\prime} \overset{\mu_{A}}{\longrightarrow} A \overset{\varepsilon_{A}}{\longrightarrow} A^{\prime\prime}$).
% 	Thus, the exactness of
% 	\,$A^{\prime} \overset{\mu_{A}}{\longrightarrow} A \overset{\varepsilon_{A}}{\longrightarrow} A^{\prime\prime}$\,
% 	indeed implies the exactness of
% 	\,$I_{\mu_{A}} \overset{\iota(\mu_{A})}{\longrightarrow} A \overset{\varepsilon_{A}}{\longrightarrow} A^{\prime\prime}$.\,
% 	This completes the proof of Claim 2.	


% 	\vskip 0.3cm
% 	\textbf{Claim 3:}\;
% 	The image monomorphism \,$\iota(\mu_{A})$\, of \,$\mu_{A}$\, is a kernel of \,$\varepsilon_{A}$.\,
% 	\vskip 0.01cm
% 	Proof of Claim 3:\;
% 	Immediate by Claim 1, Claim 2 and
% 	Proposition \ref{MonomorphicFirstFactorInExactSequenceIsKernelOfSecondFactor}.
% 	This completes the proof of Claim 3.

% 	\vskip 0.3cm
% 	\textbf{Claim 4:}\;
% 	There exists a morphism
% 	\,$f_{1}^{\prime} : K_{P} \longrightarrow I_{\mu_{A}}$\,
% 	such that
% 	\,$(\,f_{1} \,\circ\, \kappa_{P}\,) \,=\, \iota(\mu_{A}) \,\circ\, f_{1}^{\prime}$.\,
% 	\vskip 0.01cm
% 	Proof of Claim 4:\;
% 	By Claim 3, \,$\iota(\mu_{A})$\, is a kernel of \,$\varepsilon_{A}$.\,
% 	In order to proof Claim 4, by the universal property of
% 	\,$\iota(\mu_{A})$\, as kernel of \,$\varepsilon_{A}$,\,
% 	it suffices to establish that
% 	\,$\varepsilon_{A} \,\circ\, (\,f_{1} \,\circ\, \kappa_{P}\,) \,=\, 0$.\,
% 	To this end, simply note that
% 	\,$\varepsilon_{A} \,\circ\, (\,f_{1} \,\circ\, \kappa_{P}\,)$
% 	\,$=$\, (\,$\varepsilon_{A} \,\circ\, f_{1}\,) \,\circ\, \kappa_{P}$
% 	\,$=$\, (\,$\kappa^{\prime\prime} \,\circ\, \varepsilon_{P}\,) \,\circ\, \kappa_{P}$
% 	\,$=$\, $\kappa^{\prime\prime} \,\circ\, (\,\varepsilon_{P} \,\circ\, \kappa_{P}\,)$
% 	\,$=$\, $0$.\,
% 	This completes the proof of Claim 4.


% 	\vskip 0.3cm
% 	\textbf{Claim 5:}\;
% 	There exists a morphism
% 	\,$f_{2}^{\prime} : I_{\mu_{A}} \longrightarrow B^{\prime}$\,
% 	such that
% 	\,$f^{\prime} \,=\, f_{2}^{\prime} \,\circ\, \widetilde{\mu_{A}}$.\,
% 	\vskip 0.01cm
% 	Proof of Claim 5:\;
% 	%Next, we establish the existence of a morphism
% 	%\,$f_{2}^{\prime} : I_{\mu_{A}} \longrightarrow B^{\prime}$\,
% 	%which makes the above diagram commute.
% 	Recall that, since \,$\mathfrak{A}$\, is an abelian category,
% 	the epimorphism \,$\widetilde{\mu_{A}}$\, is a cokernel (of some morphism),
% 	and thus, by Lemma \ref{AKernelsTheKernelOfItsOwnCokernel}(ii),
% 	\,$\widetilde{\mu_{A}}$\, is a cokernel of its own kernel
% 	\,$\kappa(\widetilde{\mu_{A}})$.

% 	Next, note that
% 	\,$\mu_{B} \,\circ\, f^{\prime} \,\circ\, \kappa(\widetilde{\mu_{A}})$
% 	\,$=$\, $f \,\circ\, \iota(\mu_{A}) \,\circ\, \widetilde{\mu_{A}} \,\circ\, \kappa(\widetilde{\mu_{A}})$
% 	\,$=$\, $f \,\circ\, \iota(\mu_{A}) \,\circ\, 0$
% 	\,$=$\, $0$.\,
% 	Since \,$\mu_{B}$\, is a monomorphism, it can be cancelled on the left,
% 	which yields
% 	\,$ f^{\prime} \circ \kappa(\widetilde{\mu_{A}}) \,=\, 0$.\,
% 	Since \,$\widetilde{\mu_{A}}$\, is a cokernel of \,$\kappa(\widetilde{\mu_{A}})$\, (established in preceding paragraph),
% 	we see that there exists a (unique) morphism
% 	\,$f_{2}^{\prime} : I_{\mu_{A}} \longrightarrow B^{\prime}$\,
% 	such that
% 	\,$f^{\prime} \,=\, f_{2}^{\prime} \,\circ\, \widetilde{\mu_{A}}$.\,
% 	This completes the proof of Claim 5.
	

% 	\vskip 0.3cm
% 	\textbf{Claim 6:} \;$f_{3} \,\circ\, \kappa_{P} \,=\, 0$.
% 		\vskip 0.01cm
% 	Proof of Claim 6:\;
% 	Note that we now have
% 	\,$f_{3} \,\circ\, \kappa_{P}$
% 	\,$=$\, $\pi^{\prime} \,\circ\, f_{2}^{\prime} \,\circ\, f_{1}^{\prime}$.\,
% 	On the other hand,
% 	\,$\pi^{\prime} \,\circ\, f_{2}^{\prime} \,\circ\, \widetilde{\mu_{A}} \,=\, \pi^{\prime} \,\circ\, f^{\prime} \,=\, 0$.\,
% 	Since \,$\widetilde{\mu_{A}}$\, is an epimorphism and can be cancelled on the right,
% 	we have
% 	\,$\pi^{\prime} \,\circ\, f_{2}^{\prime} \,=\, 0$.\,
% 	This therefore implies
% 	\,$f_{3} \,\circ\, \kappa_{P}$
% 	\,$=$\, $\pi^{\prime} \,\circ\, f_{2}^{\prime} \,\circ\, f_{1}^{\prime}$
% 	\,$=$\, $0 \,\circ\, f_{1}^{\prime}$
% 	\,$=$\, $0$.\,
% 	This completes the proof of Claim 6.


% 	\vskip 0.3cm
% 	\textbf{Claim 7:} \;$f_{3}$\, factors through \,$\varepsilon_{P}$;\,
% 	more precisely, there exists
% 	\,$\delta : K^{\prime\prime} \longrightarrow Q^{\prime}$\,
% 	such that
% 	\,$f_{3} \,=\, \delta \,\circ\, \varepsilon_{P}$.
% 	\vskip 0.01cm
% 	Proof of Claim 7:\;
% 	Since \,$\mathfrak{A}$\, is an abelian category, it is in particular conormal, i.e.,
% 	every epimorphism in \,$\mathfrak{A}$\, is a cokernel.
% 	By Lemma \ref{AKernelsTheKernelOfItsOwnCokernel}(ii),
% 	the epimorphism \,$\varepsilon_{P}$\, is thus a cokernel of of its own kernel \,$\kappa_{P}$.\,
% 	By Claim 6, we have \,$f_{3} \,\circ\, \kappa_{P} \,=\, 0$.\,
% 	Thus, the existence and uniqueness of the morphism
% 	\,$\delta : K^{\prime\prime} \longrightarrow Q^{\prime}$\,
% 	now follows by the universal property of
% 	\,$\varepsilon_{P}$\, as a cokernel of \,$\kappa_{P}$.\,
% 	This completes the proof of Claim 7.


% 	\vskip 0.3cm
% 	This construction of the (snake or connection) morphism \,$\delta$\, is now complete.
\end{enumerate}

          %%%%% ~~~~~~~~~~~~~~~~~~~~ %%%%%
